\documentclass[11pt,twoside]{article}\makeatletter

\IfFileExists{xcolor.sty}%
  {\RequirePackage{xcolor}}%
  {\RequirePackage{color}}
\usepackage{colortbl}
\usepackage{wrapfig}
\usepackage{ifxetex}
\ifxetex
  \usepackage{fontspec}
  \usepackage{xunicode}
  \catcode`⃥=\active \def⃥{\textbackslash}
  \catcode`❴=\active \def❴{\{}
  \catcode`❵=\active \def❵{\}}
  \def\textJapanese{\fontspec{IPAMincho}}
  \def\textChinese{\fontspec{HAN NOM A}\XeTeXlinebreaklocale "zh"\XeTeXlinebreakskip = 0pt plus 1pt }
  \def\textKorean{\fontspec{Baekmuk Gulim} }
  \setmonofont{DejaVu Sans Mono}
  
\else
  \IfFileExists{utf8x.def}%
   {\usepackage[utf8x]{inputenc}
      \PrerenderUnicode{–}
    }%
   {\usepackage[utf8]{inputenc}}
  \usepackage[english]{babel}
  \usepackage[T1]{fontenc}
  \usepackage{float}
  \usepackage[]{ucs}
  \uc@dclc{8421}{default}{\textbackslash }
  \uc@dclc{10100}{default}{\{}
  \uc@dclc{10101}{default}{\}}
  \uc@dclc{8491}{default}{\AA{}}
  \uc@dclc{8239}{default}{\,}
  \uc@dclc{20154}{default}{ }
  \uc@dclc{10148}{default}{>}
  \def\textschwa{\rotatebox{-90}{e}}
  \def\textJapanese{}
  \def\textChinese{}
  \IfFileExists{tipa.sty}{\usepackage{tipa}}{}
  \usepackage{times}
\fi
\def\exampleFont{\ttfamily\small}
\DeclareTextSymbol{\textpi}{OML}{25}
\usepackage{relsize}
\RequirePackage{array}
\def\@testpach{\@chclass
 \ifnum \@lastchclass=6 \@ne \@chnum \@ne \else
  \ifnum \@lastchclass=7 5 \else
   \ifnum \@lastchclass=8 \tw@ \else
    \ifnum \@lastchclass=9 \thr@@
   \else \z@
   \ifnum \@lastchclass = 10 \else
   \edef\@nextchar{\expandafter\string\@nextchar}%
   \@chnum
   \if \@nextchar c\z@ \else
    \if \@nextchar l\@ne \else
     \if \@nextchar r\tw@ \else
   \z@ \@chclass
   \if\@nextchar |\@ne \else
    \if \@nextchar !6 \else
     \if \@nextchar @7 \else
      \if \@nextchar (8 \else
       \if \@nextchar )9 \else
  10
  \@chnum
  \if \@nextchar m\thr@@\else
   \if \@nextchar p4 \else
    \if \@nextchar b5 \else
   \z@ \@chclass \z@ \@preamerr \z@ \fi \fi \fi \fi
   \fi \fi  \fi  \fi  \fi  \fi  \fi \fi \fi \fi \fi \fi}
\gdef\arraybackslash{\let\\=\@arraycr}
\def\@textsubscript#1{{\m@th\ensuremath{_{\mbox{\fontsize\sf@size\z@#1}}}}}
\def\Panel#1#2#3#4{\multicolumn{#3}{){\columncolor{#2}}#4}{#1}}
\def\abbr{}
\def\corr{}
\def\expan{}
\def\gap{}
\def\orig{}
\def\reg{}
\def\ref{}
\def\sic{}
\def\persName{}\def\name{}
\def\placeName{}
\def\orgName{}
\def\textcal#1{{\fontspec{Lucida Calligraphy}#1}}
\def\textgothic#1{{\fontspec{Lucida Blackletter}#1}}
\def\textlarge#1{{\large #1}}
\def\textoverbar#1{\ensuremath{\overline{#1}}}
\def\textquoted#1{‘#1’}
\def\textsmall#1{{\small #1}}
\def\textsubscript#1{\@textsubscript{\selectfont#1}}
\def\textxi{\ensuremath{\xi}}
\def\titlem{\itshape}
\newenvironment{biblfree}{}{\ifvmode\par\fi }
\newenvironment{bibl}{}{}
\newenvironment{byline}{\vskip6pt\itshape\fontsize{16pt}{18pt}\selectfont}{\par }
\newenvironment{citbibl}{}{\ifvmode\par\fi }
\newenvironment{docAuthor}{\ifvmode\vskip4pt\fontsize{16pt}{18pt}\selectfont\fi\itshape}{\ifvmode\par\fi }
\newenvironment{docDate}{}{\ifvmode\par\fi }
\newenvironment{docImprint}{\vskip 6pt}{\ifvmode\par\fi }
\newenvironment{docTitle}{\vskip6pt\bfseries\fontsize{18pt}{22pt}\selectfont}{\par }
\newenvironment{msHead}{\vskip 6pt}{\par}
\newenvironment{msItem}{\vskip 6pt}{\par}
\newenvironment{rubric}{}{}
\newenvironment{titlePart}{}{\par }

\newcolumntype{L}[1]{){\raggedright\arraybackslash}p{#1}}
\newcolumntype{C}[1]{){\centering\arraybackslash}p{#1}}
\newcolumntype{R}[1]{){\raggedleft\arraybackslash}p{#1}}
\newcolumntype{P}[1]{){\arraybackslash}p{#1}}
\newcolumntype{B}[1]{){\arraybackslash}b{#1}}
\newcolumntype{M}[1]{){\arraybackslash}m{#1}}
\definecolor{label}{gray}{0.75}
\def\unusedattribute#1{\sout{\textcolor{label}{#1}}}
\DeclareRobustCommand*{\xref}{\hyper@normalise\xref@}
\def\xref@#1#2{\hyper@linkurl{#2}{#1}}
\begingroup
\catcode`\_=\active
\gdef_#1{\ensuremath{\sb{\mathrm{#1}}}}
\endgroup
\mathcode`\_=\string"8000
\catcode`\_=12\relax

\usepackage[a4paper,twoside,lmargin=1in,rmargin=1in,tmargin=1in,bmargin=1in,marginparwidth=0.75in]{geometry}
\usepackage{framed}

\definecolor{shadecolor}{gray}{0.95}
\usepackage{longtable}
\usepackage[normalem]{ulem}
\usepackage{fancyvrb}
\usepackage{fancyhdr}
\usepackage{graphicx}
\usepackage{marginnote}


\renewcommand*{\marginfont}{\itshape\footnotesize}

\def\Gin@extensions{.pdf,.png,.jpg,.mps,.tif}

  \pagestyle{fancy}

\usepackage[pdftitle={TEI for Leges Langobardorum},
 pdfauthor={RRDT}]{hyperref}
\hyperbaseurl{}

	 \paperwidth210mm
	 \paperheight297mm
              
\def\@pnumwidth{1.55em}
\def\@tocrmarg {2.55em}
\def\@dotsep{4.5}
\setcounter{tocdepth}{3}
\clubpenalty=8000
\emergencystretch 3em
\hbadness=4000
\hyphenpenalty=400
\pretolerance=750
\tolerance=2000
\vbadness=4000
\widowpenalty=10000

\renewcommand\section{\@startsection {section}{1}{\z@}%
     {-1.75ex \@plus -0.5ex \@minus -.2ex}%
     {0.5ex \@plus .2ex}%
     {\reset@font\Large\bfseries\sffamily}}
\renewcommand\subsection{\@startsection{subsection}{2}{\z@}%
     {-1.75ex\@plus -0.5ex \@minus- .2ex}%
     {0.5ex \@plus .2ex}%
     {\reset@font\Large\sffamily}}
\renewcommand\subsubsection{\@startsection{subsubsection}{3}{\z@}%
     {-1.5ex\@plus -0.35ex \@minus -.2ex}%
     {0.5ex \@plus .2ex}%
     {\reset@font\large\sffamily}}
\renewcommand\paragraph{\@startsection{paragraph}{4}{\z@}%
     {-1ex \@plus-0.35ex \@minus -0.2ex}%
     {0.5ex \@plus .2ex}%
     {\reset@font\normalsize\sffamily}}
\renewcommand\subparagraph{\@startsection{subparagraph}{5}{\parindent}%
     {1.5ex \@plus1ex \@minus .2ex}%
     {-1em}%
     {\reset@font\normalsize\bfseries}}


\def\l@section#1#2{\addpenalty{\@secpenalty} \addvspace{1.0em plus 1pt}
 \@tempdima 1.5em \begingroup
 \parindent \z@ \rightskip \@pnumwidth 
 \parfillskip -\@pnumwidth 
 \bfseries \leavevmode #1\hfil \hbox to\@pnumwidth{\hss #2}\par
 \endgroup}
\def\l@subsection{\@dottedtocline{2}{1.5em}{2.3em}}
\def\l@subsubsection{\@dottedtocline{3}{3.8em}{3.2em}}
\def\l@paragraph{\@dottedtocline{4}{7.0em}{4.1em}}
\def\l@subparagraph{\@dottedtocline{5}{10em}{5em}}
\@ifundefined{c@section}{\newcounter{section}}{}
\@ifundefined{c@chapter}{\newcounter{chapter}}{}
\newif\if@mainmatter 
\@mainmattertrue
\def\chaptername{Chapter}
\def\frontmatter{%
  \pagenumbering{roman}
  \def\thechapter{\@roman\c@chapter}
  \def\theHchapter{\roman{chapter}}
  \def\thesection{\@roman\c@section}
  \def\theHsection{\roman{section}}
  \def\@chapapp{}%
}
\def\mainmatter{%
  \cleardoublepage
  \def\thechapter{\@arabic\c@chapter}
  \setcounter{chapter}{0}
  \setcounter{section}{0}
  \pagenumbering{arabic}
  \setcounter{secnumdepth}{6}
  \def\@chapapp{\chaptername}%
  \def\theHchapter{\arabic{chapter}}
  \def\thesection{\@arabic\c@section}
  \def\theHsection{\arabic{section}}
}
\def\backmatter{%
  \cleardoublepage
  \setcounter{chapter}{0}
  \setcounter{section}{0}
  \setcounter{secnumdepth}{2}
  \def\@chapapp{\appendixname}%
  \def\thechapter{\@Alph\c@chapter}
  \def\theHchapter{\Alph{chapter}}
  \appendix
}
\newenvironment{bibitemlist}[1]{%
   \list{\@biblabel{\@arabic\c@enumiv}}%
       {\settowidth\labelwidth{\@biblabel{#1}}%
        \leftmargin\labelwidth
        \advance\leftmargin\labelsep
        \@openbib@code
        \usecounter{enumiv}%
        \let\p@enumiv\@empty
        \renewcommand\theenumiv{\@arabic\c@enumiv}%
	}%
  \sloppy
  \clubpenalty4000
  \@clubpenalty \clubpenalty
  \widowpenalty4000%
  \sfcode`\.\@m}%
  {\def\@noitemerr
    {\@latex@warning{Empty `bibitemlist' environment}}%
    \endlist}

\def\tableofcontents{\section*{\contentsname}\@starttoc{toc}}
\parskip0pt
\parindent1em
\def\Panel#1#2#3#4{\multicolumn{#3}{){\columncolor{#2}}#4}{#1}}
\newenvironment{reflist}{%
  \begin{raggedright}\begin{list}{}
  {%
   \setlength{\topsep}{0pt}%
   \setlength{\rightmargin}{0.25in}%
   \setlength{\itemsep}{0pt}%
   \setlength{\itemindent}{0pt}%
   \setlength{\parskip}{0pt}%
   \setlength{\parsep}{2pt}%
   \def\makelabel##1{\itshape ##1}}%
  }
  {\end{list}\end{raggedright}}
\newenvironment{sansreflist}{%
  \begin{raggedright}\begin{list}{}
  {%
   \setlength{\topsep}{0pt}%
   \setlength{\rightmargin}{0.25in}%
   \setlength{\itemindent}{0pt}%
   \setlength{\parskip}{0pt}%
   \setlength{\itemsep}{0pt}%
   \setlength{\parsep}{2pt}%
   \def\makelabel##1{\upshape\sffamily ##1}}%
  }
  {\end{list}\end{raggedright}}
\newenvironment{specHead}[2]%
 {\vspace{20pt}\hrule\vspace{10pt}%
  \label{#1}\markright{#2}%

  \pdfbookmark[2]{#2}{#1}%
  \hspace{-0.75in}{\bfseries\fontsize{16pt}{18pt}\selectfont#2}%
  }{}
      \def\TheFullDate{2017-05-15 (revised: 2017-05-15T11:36:31Z)}
\def\TheID{\makeatother }
\def\TheDate{2017-05-15}
\title{TEI for Leges Langobardorum}
\author{RRDT}\makeatletter 
\makeatletter
\newcommand*{\cleartoleftpage}{%
  \clearpage
    \if@twoside
    \ifodd\c@page
      \hbox{}\newpage
      \if@twocolumn
        \hbox{}\newpage
      \fi
    \fi
  \fi
}
\makeatother
\makeatletter
\thispagestyle{empty}
\markright{\@title}\markboth{\@title}{\@author}
\renewcommand\small{\@setfontsize\small{9pt}{11pt}\abovedisplayskip 8.5\p@ plus3\p@ minus4\p@
\belowdisplayskip \abovedisplayskip
\abovedisplayshortskip \z@ plus2\p@
\belowdisplayshortskip 4\p@ plus2\p@ minus2\p@
\def\@listi{\leftmargin\leftmargini
               \topsep 2\p@ plus1\p@ minus1\p@
               \parsep 2\p@ plus\p@ minus\p@
               \itemsep 1pt}
}
\makeatother
\fvset{frame=single,numberblanklines=false,xleftmargin=5mm,xrightmargin=5mm}
\fancyhf{} 
\setlength{\headheight}{14pt}
\fancyhead[LE]{\bfseries\leftmark} 
\fancyhead[RO]{\bfseries\rightmark} 
\fancyfoot[RO]{}
\fancyfoot[CO]{\thepage}
\fancyfoot[LO]{\TheID}
\fancyfoot[LE]{}
\fancyfoot[CE]{\thepage}
\fancyfoot[RE]{\TheID}
\hypersetup{linkbordercolor=0.75 0.75 0.75,urlbordercolor=0.75 0.75 0.75,bookmarksnumbered=true}
\fancypagestyle{plain}{\fancyhead{}\renewcommand{\headrulewidth}{0pt}}\makeatother 
\begin{document}

\makeatletter
\noindent\parbox[b]{.75\textwidth}{\fontsize{14pt}{16pt}\bfseries\raggedright\sffamily\selectfont \@title}
\vskip20pt
\par\noindent{\fontsize{11pt}{13pt}\sffamily\itshape\raggedright\selectfont\@author\hfill\TheDate}
\vspace{18pt}
\makeatother
\let\tabcellsep&\par
TEI schema for the Leges Langobardorum project.
\section[{Elements}]{Elements}\index{TEI=<TEI>|oddindex}\index{version=@version!<TEI>|oddindex}
\begin{reflist}
\item[]\begin{specHead}{TEI.TEI}{<TEI> }(TEI document) contains a single TEI-conformant document, combining a single TEI header with one or more members of the \textsf{model.resourceLike} class. Multiple TEI elements may be combined to form a <teiCorpus> element. [\xref{http://www.tei-c.org/release/doc/tei-p5-doc/en/html/DS.html\#DS}{4. Default Text Structure} \xref{http://www.tei-c.org/release/doc/tei-p5-doc/en/html/CC.html\#CCDEF}{15.1. Varieties of Composite Text}]\end{specHead} 
    \item[{Module}]
  textstructure
    \item[{Attributes}]
  Attributes att.global (\textit{@xml:id}, \textit{@n}, \textit{@xml:lang}, \textit{@xml:base}, \textit{@xml:space})  (att.global.rendition (\textit{@rend}, \textit{@style}, \textit{@rendition})) (att.global.linking (\textit{@corresp}, \textit{@synch}, \textit{@sameAs}, \textit{@copyOf}, \textit{@next}, \textit{@prev}, \textit{@exclude}, \textit{@select})) (att.global.analytic (\textit{@ana})) (att.global.facs (\textit{@facs})) (att.global.change (\textit{@change})) (att.global.responsibility (\textit{@cert}, \textit{@resp})) (att.global.source (\textit{@source})) att.typed (\textit{@type}, \textit{@subtype}) \hfil\\[-10pt]\begin{sansreflist}
    \item[@version]
  specifies the major version number of the TEI Guidelines against which this document is valid.
\begin{reflist}
    \item[{Status}]
  Optional
    \item[{Datatype}]
  teidata.version
    \item[{Note}]
  \par
The major version number is historically prefixed by a P (for Proposal), and is distinct from the version number used for individual releases of the Guidelines, as used by (for example) the {\itshape source} of the \texttt{<schemaSpec>} element. The current version is P5.
\end{reflist}  
\end{sansreflist}  
    \item[{Contained by}]
  
    \item[core: ]
   teiCorpus
    \item[{May contain}]
  
    \item[header: ]
   teiHeader\par 
    \item[textstructure: ]
   text\par 
    \item[transcr: ]
   facsimile sourceDoc
    \item[{Note}]
  \par
This element is required. It is customary to specify the TEI namespace \texttt{http://www.tei-c.org/ns/1.0} on it, using the {\itshape xmlns} attribute.
    \item[{Example}]
  \leavevmode\bgroup\exampleFont \begin{shaded}\noindent\mbox{}{<\textbf{TEI}\hspace*{6pt}{version}="{5.0}" xmlns="http://www.tei-c.org/ns/1.0">}\mbox{}\newline 
\hspace*{6pt}{<\textbf{teiHeader}>}\mbox{}\newline 
\hspace*{6pt}\hspace*{6pt}{<\textbf{fileDesc}>}\mbox{}\newline 
\hspace*{6pt}\hspace*{6pt}\hspace*{6pt}{<\textbf{titleStmt}>}\mbox{}\newline 
\hspace*{6pt}\hspace*{6pt}\hspace*{6pt}\hspace*{6pt}{<\textbf{title}>}The shortest TEI Document Imaginable{</\textbf{title}>}\mbox{}\newline 
\hspace*{6pt}\hspace*{6pt}\hspace*{6pt}{</\textbf{titleStmt}>}\mbox{}\newline 
\hspace*{6pt}\hspace*{6pt}\hspace*{6pt}{<\textbf{publicationStmt}>}\mbox{}\newline 
\hspace*{6pt}\hspace*{6pt}\hspace*{6pt}\hspace*{6pt}{<\textbf{p}>}First published as part of TEI P2, this is the P5\mbox{}\newline 
\hspace*{6pt}\hspace*{6pt}\hspace*{6pt}\hspace*{6pt}\hspace*{6pt}\hspace*{6pt}\hspace*{6pt}\hspace*{6pt} version using a name space.{</\textbf{p}>}\mbox{}\newline 
\hspace*{6pt}\hspace*{6pt}\hspace*{6pt}{</\textbf{publicationStmt}>}\mbox{}\newline 
\hspace*{6pt}\hspace*{6pt}\hspace*{6pt}{<\textbf{sourceDesc}>}\mbox{}\newline 
\hspace*{6pt}\hspace*{6pt}\hspace*{6pt}\hspace*{6pt}{<\textbf{p}>}No source: this is an original work.{</\textbf{p}>}\mbox{}\newline 
\hspace*{6pt}\hspace*{6pt}\hspace*{6pt}{</\textbf{sourceDesc}>}\mbox{}\newline 
\hspace*{6pt}\hspace*{6pt}{</\textbf{fileDesc}>}\mbox{}\newline 
\hspace*{6pt}{</\textbf{teiHeader}>}\mbox{}\newline 
\hspace*{6pt}{<\textbf{text}>}\mbox{}\newline 
\hspace*{6pt}\hspace*{6pt}{<\textbf{body}>}\mbox{}\newline 
\hspace*{6pt}\hspace*{6pt}\hspace*{6pt}{<\textbf{p}>}This is about the shortest TEI document imaginable.{</\textbf{p}>}\mbox{}\newline 
\hspace*{6pt}\hspace*{6pt}{</\textbf{body}>}\mbox{}\newline 
\hspace*{6pt}{</\textbf{text}>}\mbox{}\newline 
{</\textbf{TEI}>}\end{shaded}\egroup 


    \item[{Example}]
  \leavevmode\bgroup\exampleFont \begin{shaded}\noindent\mbox{}{<\textbf{TEI}\hspace*{6pt}{version}="{5.0}" xmlns="http://www.tei-c.org/ns/1.0">}\mbox{}\newline 
\hspace*{6pt}{<\textbf{teiHeader}>}\mbox{}\newline 
\hspace*{6pt}\hspace*{6pt}{<\textbf{fileDesc}>}\mbox{}\newline 
\hspace*{6pt}\hspace*{6pt}\hspace*{6pt}{<\textbf{titleStmt}>}\mbox{}\newline 
\hspace*{6pt}\hspace*{6pt}\hspace*{6pt}\hspace*{6pt}{<\textbf{title}>}A TEI Document containing four page images {</\textbf{title}>}\mbox{}\newline 
\hspace*{6pt}\hspace*{6pt}\hspace*{6pt}{</\textbf{titleStmt}>}\mbox{}\newline 
\hspace*{6pt}\hspace*{6pt}\hspace*{6pt}{<\textbf{publicationStmt}>}\mbox{}\newline 
\hspace*{6pt}\hspace*{6pt}\hspace*{6pt}\hspace*{6pt}{<\textbf{p}>}Unpublished demonstration file.{</\textbf{p}>}\mbox{}\newline 
\hspace*{6pt}\hspace*{6pt}\hspace*{6pt}{</\textbf{publicationStmt}>}\mbox{}\newline 
\hspace*{6pt}\hspace*{6pt}\hspace*{6pt}{<\textbf{sourceDesc}>}\mbox{}\newline 
\hspace*{6pt}\hspace*{6pt}\hspace*{6pt}\hspace*{6pt}{<\textbf{p}>}No source: this is an original work.{</\textbf{p}>}\mbox{}\newline 
\hspace*{6pt}\hspace*{6pt}\hspace*{6pt}{</\textbf{sourceDesc}>}\mbox{}\newline 
\hspace*{6pt}\hspace*{6pt}{</\textbf{fileDesc}>}\mbox{}\newline 
\hspace*{6pt}{</\textbf{teiHeader}>}\mbox{}\newline 
\hspace*{6pt}{<\textbf{facsimile}>}\mbox{}\newline 
\hspace*{6pt}\hspace*{6pt}{<\textbf{graphic}\hspace*{6pt}{url}="{page1.png}"/>}\mbox{}\newline 
\hspace*{6pt}\hspace*{6pt}{<\textbf{graphic}\hspace*{6pt}{url}="{page2.png}"/>}\mbox{}\newline 
\hspace*{6pt}\hspace*{6pt}{<\textbf{graphic}\hspace*{6pt}{url}="{page3.png}"/>}\mbox{}\newline 
\hspace*{6pt}\hspace*{6pt}{<\textbf{graphic}\hspace*{6pt}{url}="{page4.png}"/>}\mbox{}\newline 
\hspace*{6pt}{</\textbf{facsimile}>}\mbox{}\newline 
{</\textbf{TEI}>}\end{shaded}\egroup 


    \item[{Schematron}]
   <s:ns prefix="tei"  uri="http://www.tei-c.org/ns/1.0"/> <s:ns prefix="xs"  uri="http://www.w3.org/2001/XMLSchema"/>
    \item[{Schematron}]
   <s:ns prefix="rng"  uri="http://relaxng.org/ns/structure/1.0"/>
    \item[{Content model}]
  \mbox{}\hfill\\[-10pt]\begin{Verbatim}[fontsize=\small]
<content>
 <sequence>
  <elementRef key="teiHeader"/>
  <classRef key="model.resourceLike"
   maxOccurs="unbounded" minOccurs="1"/>
 </sequence>
</content>
    
\end{Verbatim}

    \item[{Schema Declaration}]
  \mbox{}\hfill\\[-10pt]\begin{Verbatim}[fontsize=\small]
element TEI
{
   att.global.attributes,
   att.typed.attributes,
   attribute version { text }?,
   ( teiHeader, model.resourceLike+ )
}
\end{Verbatim}

\end{reflist}  \index{ab=<ab>|oddindex}
\begin{reflist}
\item[]\begin{specHead}{TEI.ab}{<ab> }(anonymous block) contains any arbitrary component-level unit of text, acting as an anonymous container for phrase or inter level elements analogous to, but without the semantic baggage of, a paragraph. [\xref{http://www.tei-c.org/release/doc/tei-p5-doc/en/html/SA.html\#SASE}{16.3. Blocks, Segments, and Anchors}]\end{specHead} 
    \item[{Module}]
  linking
    \item[{Attributes}]
  Attributes att.global (\textit{@xml:id}, \textit{@n}, \textit{@xml:lang}, \textit{@xml:base}, \textit{@xml:space})  (att.global.rendition (\textit{@rend}, \textit{@style}, \textit{@rendition})) (att.global.linking (\textit{@corresp}, \textit{@synch}, \textit{@sameAs}, \textit{@copyOf}, \textit{@next}, \textit{@prev}, \textit{@exclude}, \textit{@select})) (att.global.analytic (\textit{@ana})) (att.global.facs (\textit{@facs})) (att.global.change (\textit{@change})) (att.global.responsibility (\textit{@cert}, \textit{@resp})) (att.global.source (\textit{@source})) att.typed (\textit{@type}, \textit{@subtype}) att.declaring (\textit{@decls}) att.fragmentable (\textit{@part}) att.written (\textit{@hand}) 
    \item[{Member of}]
  model.pLike
    \item[{Contained by}]
  
    \item[core: ]
   item note q quote said sp stage\par 
    \item[figures: ]
   cell figure\par 
    \item[header: ]
   abstract application availability cRefPattern calendar change correction correspAction correspContext correspDesc editionStmt editorialDecl encodingDesc handNote hyphenation interpretation langUsage licence normalization prefixDef projectDesc publicationStmt punctuation quotation refsDecl samplingDecl scriptNote segmentation seriesStmt sourceDesc stdVals styleDefDecl typeNote\par 
    \item[msdescription: ]
   accMat acquisition additions binding bindingDesc collation condition custEvent custodialHist decoDesc decoNote filiation foliation handDesc history layout layoutDesc msContents msDesc msFrag msItem msItemStruct msPart musicNotation objectDesc origin physDesc provenance recordHist scriptDesc seal sealDesc signatures source summary support supportDesc surrogates typeDesc\par 
    \item[namesdates: ]
   climate event langKnowledge listRelation nym occupation org person personGrp place population state terrain trait\par 
    \item[textcrit: ]
   lem rdg\par 
    \item[textstructure: ]
   argument back body div epigraph front postscript\par 
    \item[transcr: ]
   metamark
    \item[{May contain}]
  
    \item[analysis: ]
   c cl interp interpGrp m pc phr s span spanGrp w\par 
    \item[core: ]
   abbr add address bibl biblStruct cb choice cit corr date del desc distinct email emph expan foreign gap gb gloss graphic hi index l label lb lg list listBibl measure measureGrp media mentioned milestone name note num orig pb ptr q quote ref reg rs said sic soCalled stage term time title unclear\par 
    \item[figures: ]
   figure formula notatedMusic table\par 
    \item[gaiji: ]
   g\par 
    \item[header: ]
   biblFull idno\par 
    \item[linking: ]
   alt altGrp anchor join joinGrp link linkGrp seg timeline\par 
    \item[msdescription: ]
   catchwords depth dim dimensions height heraldry locus locusGrp material msDesc objectType origDate origPlace secFol signatures stamp watermark width\par 
    \item[namesdates: ]
   addName affiliation bloc climate country district forename genName geo geogFeat geogName listEvent listNym listOrg listPerson listPlace location nameLink offset orgName persName placeName population region roleName settlement state surname terrain trait\par 
    \item[textcrit: ]
   app listApp listWit witDetail\par 
    \item[textstructure: ]
   floatingText\par 
    \item[transcr: ]
   addSpan am damage damageSpan delSpan ex fw handShift listTranspose metamark mod redo restore retrace secl space subst substJoin supplied surplus undo\par character data
    \item[{Note}]
  \par
The <ab> element may be used at the encoder's discretion to mark any component-level elements in a text for which no other more specific appropriate markup is defined.
    \item[{Example}]
  \leavevmode\bgroup\exampleFont \begin{shaded}\noindent\mbox{}{<\textbf{div}\hspace*{6pt}{n}="{Genesis}"\hspace*{6pt}{type}="{book}">}\mbox{}\newline 
\hspace*{6pt}{<\textbf{div}\hspace*{6pt}{n}="{1}"\hspace*{6pt}{type}="{chapter}">}\mbox{}\newline 
\hspace*{6pt}\hspace*{6pt}{<\textbf{ab}>}In the beginning God created the heaven and the earth.{</\textbf{ab}>}\mbox{}\newline 
\hspace*{6pt}\hspace*{6pt}{<\textbf{ab}>}And the earth was without form, and void; and\mbox{}\newline 
\hspace*{6pt}\hspace*{6pt}\hspace*{6pt}\hspace*{6pt} darkness was upon the face of the deep. And the\mbox{}\newline 
\hspace*{6pt}\hspace*{6pt}\hspace*{6pt}\hspace*{6pt} spirit of God moved upon the face of the waters.{</\textbf{ab}>}\mbox{}\newline 
\hspace*{6pt}\hspace*{6pt}{<\textbf{ab}>}And God said, Let there be light: and there was light.{</\textbf{ab}>}\mbox{}\newline 
\textit{<!-- ...-->}\mbox{}\newline 
\hspace*{6pt}{</\textbf{div}>}\mbox{}\newline 
{</\textbf{div}>}\end{shaded}\egroup 


    \item[{Schematron}]
   <s:report test="(ancestor::tei:p or ancestor::tei:ab) and not(parent::tei:exemplum   |parent::tei:item |parent::tei:note |parent::tei:q |parent::tei:quote   |parent::tei:remarks |parent::tei:said |parent::tei:sp |parent::tei:stage   |parent::tei:cell |parent::tei:figure)"> Abstract model violation: ab may not contain paragraphs or other ab elements. </s:report>
    \item[{Schematron}]
   <s:report test="ancestor::tei:l or ancestor::tei:lg"> Abstract model violation: Lines may not contain higher-level divisions such as p or ab. </s:report>
    \item[{Content model}]
  \mbox{}\hfill\\[-10pt]\begin{Verbatim}[fontsize=\small]
<content>
 <macroRef key="macro.paraContent"/>
</content>
    
\end{Verbatim}

    \item[{Schema Declaration}]
  \mbox{}\hfill\\[-10pt]\begin{Verbatim}[fontsize=\small]
element ab
{
   att.global.attributes,
   att.typed.attributes,
   att.declaring.attributes,
   att.fragmentable.attributes,
   att.written.attributes,
   macro.paraContent}
\end{Verbatim}

\end{reflist}  \index{abbr=<abbr>|oddindex}\index{type=@type!<abbr>|oddindex}
\begin{reflist}
\item[]\begin{specHead}{TEI.abbr}{<abbr> }(abbreviation) contains an abbreviation of any sort. [\xref{http://www.tei-c.org/release/doc/tei-p5-doc/en/html/CO.html\#CONAAB}{3.5.5. Abbreviations and Their Expansions}]\end{specHead} 
    \item[{Module}]
  core
    \item[{Attributes}]
  Attributes att.global (\textit{@xml:id}, \textit{@n}, \textit{@xml:lang}, \textit{@xml:base}, \textit{@xml:space})  (att.global.rendition (\textit{@rend}, \textit{@style}, \textit{@rendition})) (att.global.linking (\textit{@corresp}, \textit{@synch}, \textit{@sameAs}, \textit{@copyOf}, \textit{@next}, \textit{@prev}, \textit{@exclude}, \textit{@select})) (att.global.analytic (\textit{@ana})) (att.global.facs (\textit{@facs})) (att.global.change (\textit{@change})) (att.global.responsibility (\textit{@cert}, \textit{@resp})) (att.global.source (\textit{@source})) att.typed (\unusedattribute{type}, @subtype) \hfil\\[-10pt]\begin{sansreflist}
    \item[@type]
  allows the encoder to classify the abbreviation according to some convenient typology.
\begin{reflist}
    \item[{Derived from}]
  att.typed
    \item[{Status}]
  Optional
    \item[{Datatype}]
  teidata.enumerated
    \item[{Sample values include:}]
  \begin{description}

\item[{suspension}]the abbreviation provides the first letter(s) of the word or phrase, omitting the remainder.
\item[{contraction}]the abbreviation omits some letter(s) in the middle.
\item[{brevigraph}]the abbreviation comprises a special symbol or mark.
\item[{superscription}]the abbreviation includes writing above the line.
\item[{acronym}]the abbreviation comprises the initial letters of the words of a phrase.
\item[{title}]the abbreviation is for a title of address (Dr, Ms, Mr, …)
\item[{organization}]the abbreviation is for the name of an organization.
\item[{geographic}]the abbreviation is for a geographic name.
\end{description} 
    \item[{Note}]
  \par
The {\itshape type} attribute is provided for the sake of those who wish to classify abbreviations at their point of occurrence; this may be useful in some circumstances, though usually the same abbreviation will have the same type in all occurrences. As the sample values make clear, abbreviations may be classified by the method used to construct them, the method of writing them, or the referent of the term abbreviated; the typology used is up to the encoder and should be carefully planned to meet the needs of the expected use. For a typology of Middle English abbreviations, see PETTY
\end{reflist}  
\end{sansreflist}  
    \item[{Member of}]
  model.choicePart model.pPart.editorial
    \item[{Contained by}]
  
    \item[analysis: ]
   cl pc phr s span w\par 
    \item[core: ]
   abbr add addrLine author bibl biblScope choice citedRange corr date del desc distinct editor email emph expan foreign gloss head headItem headLabel hi item l label measure meeting mentioned name note num orig p pubPlace publisher q quote ref reg resp rs said sic soCalled speaker stage street term textLang time title unclear\par 
    \item[figures: ]
   cell figDesc\par 
    \item[header: ]
   authority catDesc change classCode creation distributor edition extent funder geoDecl handNote language licence principal rendition scriptNote sponsor tagUsage typeNote\par 
    \item[linking: ]
   ab seg\par 
    \item[msdescription: ]
   accMat acquisition additions catchwords collation colophon condition custEvent decoNote explicit filiation finalRubric foliation heraldry incipit layout material musicNotation objectType origDate origPlace origin provenance rubric secFol signatures source stamp summary support surrogates watermark\par 
    \item[namesdates: ]
   addName affiliation age birth bloc country death district education faith floruit forename genName geogFeat geogName langKnown nameLink nationality occupation offset orgName persName placeName region residence roleName settlement sex socecStatus surname\par 
    \item[textcrit: ]
   lem rdg wit witDetail witness\par 
    \item[textstructure: ]
   byline closer dateline docAuthor docDate docEdition docImprint imprimatur opener salute signed titlePart trailer\par 
    \item[transcr: ]
   damage fw metamark mod restore retrace secl supplied surplus
    \item[{May contain}]
  
    \item[analysis: ]
   c cl interp interpGrp m pc phr s span spanGrp w\par 
    \item[core: ]
   abbr add address cb choice corr date del distinct email emph expan foreign gap gb gloss graphic hi index lb measure measureGrp media mentioned milestone name note num orig pb ptr ref reg rs sic soCalled term time title unclear\par 
    \item[figures: ]
   figure formula notatedMusic\par 
    \item[gaiji: ]
   g\par 
    \item[header: ]
   idno\par 
    \item[linking: ]
   alt altGrp anchor join joinGrp link linkGrp seg timeline\par 
    \item[msdescription: ]
   catchwords depth dim dimensions height heraldry locus locusGrp material objectType origDate origPlace secFol signatures stamp watermark width\par 
    \item[namesdates: ]
   addName affiliation bloc climate country district forename genName geo geogFeat geogName location nameLink offset orgName persName placeName population region roleName settlement state surname terrain trait\par 
    \item[textcrit: ]
   app witDetail\par 
    \item[transcr: ]
   addSpan am damage damageSpan delSpan ex fw handShift listTranspose metamark mod redo restore retrace secl space subst substJoin supplied surplus undo\par character data
    \item[{Note}]
  \par
The <abbr> tag is not required; if appropriate, the encoder may transcribe abbreviations in the source text silently, without tagging them. If abbreviations are not transcribed directly but \textit{expanded} silently, then the TEI header should so indicate.
    \item[{Example}]
  \leavevmode\bgroup\exampleFont \begin{shaded}\noindent\mbox{}{<\textbf{choice}>}\mbox{}\newline 
\hspace*{6pt}{<\textbf{expan}>}North Atlantic Treaty Organization{</\textbf{expan}>}\mbox{}\newline 
\hspace*{6pt}{<\textbf{abbr}\hspace*{6pt}{cert}="{low}">}NorATO{</\textbf{abbr}>}\mbox{}\newline 
\hspace*{6pt}{<\textbf{abbr}\hspace*{6pt}{cert}="{high}">}NATO{</\textbf{abbr}>}\mbox{}\newline 
\hspace*{6pt}{<\textbf{abbr}\hspace*{6pt}{cert}="{high}"\hspace*{6pt}{xml:lang}="{fr}">}OTAN{</\textbf{abbr}>}\mbox{}\newline 
{</\textbf{choice}>}\end{shaded}\egroup 


    \item[{Example}]
  \leavevmode\bgroup\exampleFont \begin{shaded}\noindent\mbox{}{<\textbf{choice}>}\mbox{}\newline 
\hspace*{6pt}{<\textbf{abbr}>}SPQR{</\textbf{abbr}>}\mbox{}\newline 
\hspace*{6pt}{<\textbf{expan}>}senatus populusque romanorum{</\textbf{expan}>}\mbox{}\newline 
{</\textbf{choice}>}\end{shaded}\egroup 


    \item[{Content model}]
  \mbox{}\hfill\\[-10pt]\begin{Verbatim}[fontsize=\small]
<content>
 <macroRef key="macro.phraseSeq"/>
</content>
    
\end{Verbatim}

    \item[{Schema Declaration}]
  \mbox{}\hfill\\[-10pt]\begin{Verbatim}[fontsize=\small]
element abbr
{
   att.global.attributes,
   att.typed.attribute.subtype,
   attribute type { text }?,
   macro.phraseSeq}
\end{Verbatim}

\end{reflist}  \index{abstract=<abstract>|oddindex}
\begin{reflist}
\item[]\begin{specHead}{TEI.abstract}{<abstract> }contains a summary or formal abstract prefixed to an existing source document by the encoder. [\xref{http://www.tei-c.org/release/doc/tei-p5-doc/en/html/HD.html\#HD4ABS}{2.4.4. Abstracts}]\end{specHead} 
    \item[{Module}]
  header
    \item[{Attributes}]
  Attributes att.global (\textit{@xml:id}, \textit{@n}, \textit{@xml:lang}, \textit{@xml:base}, \textit{@xml:space})  (att.global.rendition (\textit{@rend}, \textit{@style}, \textit{@rendition})) (att.global.linking (\textit{@corresp}, \textit{@synch}, \textit{@sameAs}, \textit{@copyOf}, \textit{@next}, \textit{@prev}, \textit{@exclude}, \textit{@select})) (att.global.analytic (\textit{@ana})) (att.global.facs (\textit{@facs})) (att.global.change (\textit{@change})) (att.global.responsibility (\textit{@cert}, \textit{@resp})) (att.global.source (\textit{@source}))
    \item[{Member of}]
  model.profileDescPart
    \item[{Contained by}]
  
    \item[header: ]
   profileDesc
    \item[{May contain}]
  
    \item[core: ]
   list p\par 
    \item[figures: ]
   table\par 
    \item[linking: ]
   ab\par 
    \item[namesdates: ]
   listEvent listNym listOrg listPerson listPlace\par 
    \item[textcrit: ]
   listApp listWit
    \item[{Note}]
  \par
This element is intended only for cases where no abstract is available in the original source. Any abstract already present in the source document should be encoded as a <div> within the <front>, as it should for a born-digital document.
    \item[{Example}]
  \leavevmode\bgroup\exampleFont \begin{shaded}\noindent\mbox{}{<\textbf{profileDesc}>}\mbox{}\newline 
\hspace*{6pt}{<\textbf{abstract}\hspace*{6pt}{resp}="{\#LB}">}\mbox{}\newline 
\hspace*{6pt}\hspace*{6pt}{<\textbf{p}>}Good database design involves the acquisition and deployment of\mbox{}\newline 
\hspace*{6pt}\hspace*{6pt}\hspace*{6pt}\hspace*{6pt} skills which have a wider relevance to the educational process. From\mbox{}\newline 
\hspace*{6pt}\hspace*{6pt}\hspace*{6pt}\hspace*{6pt} a set of more or less instinctive rules of thumb a formal discipline\mbox{}\newline 
\hspace*{6pt}\hspace*{6pt}\hspace*{6pt}\hspace*{6pt} or "methodology" of database design has evolved. Applying that\mbox{}\newline 
\hspace*{6pt}\hspace*{6pt}\hspace*{6pt}\hspace*{6pt} methodology can be of great benefit to a very wide range of academic\mbox{}\newline 
\hspace*{6pt}\hspace*{6pt}\hspace*{6pt}\hspace*{6pt} subjects: it requires fundamental skills of abstraction and\mbox{}\newline 
\hspace*{6pt}\hspace*{6pt}\hspace*{6pt}\hspace*{6pt} generalisation and it provides a simple mechanism whereby complex\mbox{}\newline 
\hspace*{6pt}\hspace*{6pt}\hspace*{6pt}\hspace*{6pt} ideas and information structures can be represented and manipulated,\mbox{}\newline 
\hspace*{6pt}\hspace*{6pt}\hspace*{6pt}\hspace*{6pt} even without the use of a computer. {</\textbf{p}>}\mbox{}\newline 
\hspace*{6pt}{</\textbf{abstract}>}\mbox{}\newline 
{</\textbf{profileDesc}>}\end{shaded}\egroup 


    \item[{Content model}]
  \mbox{}\hfill\\[-10pt]\begin{Verbatim}[fontsize=\small]
<content>
 <alternate maxOccurs="unbounded"
  minOccurs="1">
  <classRef key="model.pLike"/>
  <classRef key="model.listLike"/>
 </alternate>
</content>
    
\end{Verbatim}

    \item[{Schema Declaration}]
  \mbox{}\hfill\\[-10pt]\begin{Verbatim}[fontsize=\small]
element abstract { att.global.attributes, ( model.pLike | model.listLike )+ }
\end{Verbatim}

\end{reflist}  \index{accMat=<accMat>|oddindex}
\begin{reflist}
\item[]\begin{specHead}{TEI.accMat}{<accMat> }(accompanying material) contains details of any significant additional material which may be closely associated with the manuscript being described, such as non-contemporaneous documents or fragments bound in with the manuscript at some earlier historical period. [\xref{http://www.tei-c.org/release/doc/tei-p5-doc/en/html/MS.html\#msadac}{10.7.3.3. Accompanying Material}]\end{specHead} 
    \item[{Module}]
  msdescription
    \item[{Attributes}]
  Attributes att.global (\textit{@xml:id}, \textit{@n}, \textit{@xml:lang}, \textit{@xml:base}, \textit{@xml:space})  (att.global.rendition (\textit{@rend}, \textit{@style}, \textit{@rendition})) (att.global.linking (\textit{@corresp}, \textit{@synch}, \textit{@sameAs}, \textit{@copyOf}, \textit{@next}, \textit{@prev}, \textit{@exclude}, \textit{@select})) (att.global.analytic (\textit{@ana})) (att.global.facs (\textit{@facs})) (att.global.change (\textit{@change})) (att.global.responsibility (\textit{@cert}, \textit{@resp})) (att.global.source (\textit{@source})) att.typed (\textit{@type}, \textit{@subtype}) 
    \item[{Member of}]
  model.physDescPart
    \item[{Contained by}]
  
    \item[msdescription: ]
   physDesc
    \item[{May contain}]
  
    \item[analysis: ]
   c cl interp interpGrp m pc phr s span spanGrp w\par 
    \item[core: ]
   abbr add address bibl biblStruct cb choice cit corr date del desc distinct email emph expan foreign gap gb gloss graphic hi index l label lb lg list listBibl measure measureGrp media mentioned milestone name note num orig p pb ptr q quote ref reg rs said sic soCalled sp stage term time title unclear\par 
    \item[figures: ]
   figure formula notatedMusic table\par 
    \item[gaiji: ]
   g\par 
    \item[header: ]
   biblFull idno\par 
    \item[linking: ]
   ab alt altGrp anchor join joinGrp link linkGrp seg timeline\par 
    \item[msdescription: ]
   catchwords depth dim dimensions height heraldry locus locusGrp material msDesc objectType origDate origPlace secFol signatures stamp watermark width\par 
    \item[namesdates: ]
   addName affiliation bloc climate country district forename genName geo geogFeat geogName listEvent listNym listOrg listPerson listPlace location nameLink offset orgName persName placeName population region roleName settlement state surname terrain trait\par 
    \item[textcrit: ]
   app listApp listWit witDetail\par 
    \item[textstructure: ]
   floatingText\par 
    \item[transcr: ]
   addSpan am damage damageSpan delSpan ex fw handShift listTranspose metamark mod redo restore retrace secl space subst substJoin supplied surplus undo\par character data
    \item[{Example}]
  \leavevmode\bgroup\exampleFont \begin{shaded}\noindent\mbox{}{<\textbf{accMat}>}A copy of a tax form from 1947 is included in the envelope\mbox{}\newline 
 with the letter. It is not catalogued separately.{</\textbf{accMat}>}\end{shaded}\egroup 


    \item[{Content model}]
  \mbox{}\hfill\\[-10pt]\begin{Verbatim}[fontsize=\small]
<content>
 <macroRef key="macro.specialPara"/>
</content>
    
\end{Verbatim}

    \item[{Schema Declaration}]
  \mbox{}\hfill\\[-10pt]\begin{Verbatim}[fontsize=\small]
element accMat
{
   att.global.attributes,
   att.typed.attributes,
   macro.specialPara}
\end{Verbatim}

\end{reflist}  \index{acquisition=<acquisition>|oddindex}
\begin{reflist}
\item[]\begin{specHead}{TEI.acquisition}{<acquisition> }contains any descriptive or other information concerning the process by which a manuscript or manuscript part entered the holding institution. [\xref{http://www.tei-c.org/release/doc/tei-p5-doc/en/html/MS.html\#mshy}{10.8. History}]\end{specHead} 
    \item[{Module}]
  msdescription
    \item[{Attributes}]
  Attributes att.global (\textit{@xml:id}, \textit{@n}, \textit{@xml:lang}, \textit{@xml:base}, \textit{@xml:space})  (att.global.rendition (\textit{@rend}, \textit{@style}, \textit{@rendition})) (att.global.linking (\textit{@corresp}, \textit{@synch}, \textit{@sameAs}, \textit{@copyOf}, \textit{@next}, \textit{@prev}, \textit{@exclude}, \textit{@select})) (att.global.analytic (\textit{@ana})) (att.global.facs (\textit{@facs})) (att.global.change (\textit{@change})) (att.global.responsibility (\textit{@cert}, \textit{@resp})) (att.global.source (\textit{@source})) att.datable (\textit{@calendar}, \textit{@period})  (att.datable.w3c (\textit{@when}, \textit{@notBefore}, \textit{@notAfter}, \textit{@from}, \textit{@to})) (att.datable.iso (\textit{@when-iso}, \textit{@notBefore-iso}, \textit{@notAfter-iso}, \textit{@from-iso}, \textit{@to-iso})) (att.datable.custom (\textit{@when-custom}, \textit{@notBefore-custom}, \textit{@notAfter-custom}, \textit{@from-custom}, \textit{@to-custom}, \textit{@datingPoint}, \textit{@datingMethod}))
    \item[{Contained by}]
  
    \item[msdescription: ]
   history
    \item[{May contain}]
  
    \item[analysis: ]
   c cl interp interpGrp m pc phr s span spanGrp w\par 
    \item[core: ]
   abbr add address bibl biblStruct cb choice cit corr date del desc distinct email emph expan foreign gap gb gloss graphic hi index l label lb lg list listBibl measure measureGrp media mentioned milestone name note num orig p pb ptr q quote ref reg rs said sic soCalled sp stage term time title unclear\par 
    \item[figures: ]
   figure formula notatedMusic table\par 
    \item[gaiji: ]
   g\par 
    \item[header: ]
   biblFull idno\par 
    \item[linking: ]
   ab alt altGrp anchor join joinGrp link linkGrp seg timeline\par 
    \item[msdescription: ]
   catchwords depth dim dimensions height heraldry locus locusGrp material msDesc objectType origDate origPlace secFol signatures stamp watermark width\par 
    \item[namesdates: ]
   addName affiliation bloc climate country district forename genName geo geogFeat geogName listEvent listNym listOrg listPerson listPlace location nameLink offset orgName persName placeName population region roleName settlement state surname terrain trait\par 
    \item[textcrit: ]
   app listApp listWit witDetail\par 
    \item[textstructure: ]
   floatingText\par 
    \item[transcr: ]
   addSpan am damage damageSpan delSpan ex fw handShift listTranspose metamark mod redo restore retrace secl space subst substJoin supplied surplus undo\par character data
    \item[{Example}]
  \leavevmode\bgroup\exampleFont \begin{shaded}\noindent\mbox{}{<\textbf{acquisition}>}Left to the {<\textbf{name}\hspace*{6pt}{type}="{place}">}Bodleian{</\textbf{name}>} by\mbox{}\newline 
{<\textbf{name}\hspace*{6pt}{type}="{person}">}Richard Rawlinson{</\textbf{name}>} in 1755.\mbox{}\newline 
{</\textbf{acquisition}>}\end{shaded}\egroup 


    \item[{Content model}]
  \mbox{}\hfill\\[-10pt]\begin{Verbatim}[fontsize=\small]
<content>
 <macroRef key="macro.specialPara"/>
</content>
    
\end{Verbatim}

    \item[{Schema Declaration}]
  \mbox{}\hfill\\[-10pt]\begin{Verbatim}[fontsize=\small]
element acquisition
{
   att.global.attributes,
   att.datable.attributes,
   macro.specialPara}
\end{Verbatim}

\end{reflist}  \index{add=<add>|oddindex}
\begin{reflist}
\item[]\begin{specHead}{TEI.add}{<add> }(addition) contains letters, words, or phrases inserted in the source text by an author, scribe, or a previous annotator or corrector. [\xref{http://www.tei-c.org/release/doc/tei-p5-doc/en/html/CO.html\#COEDADD}{3.4.3. Additions, Deletions, and Omissions}]\end{specHead} 
    \item[{Module}]
  core
    \item[{Attributes}]
  Attributes att.global (\textit{@xml:id}, \textit{@n}, \textit{@xml:lang}, \textit{@xml:base}, \textit{@xml:space})  (att.global.rendition (\textit{@rend}, \textit{@style}, \textit{@rendition})) (att.global.linking (\textit{@corresp}, \textit{@synch}, \textit{@sameAs}, \textit{@copyOf}, \textit{@next}, \textit{@prev}, \textit{@exclude}, \textit{@select})) (att.global.analytic (\textit{@ana})) (att.global.facs (\textit{@facs})) (att.global.change (\textit{@change})) (att.global.responsibility (\textit{@cert}, \textit{@resp})) (att.global.source (\textit{@source})) att.transcriptional (\textit{@status}, \textit{@cause}, \textit{@seq})  (att.editLike (\textit{@evidence}, \textit{@instant}) (att.dimensions (\textit{@unit}, \textit{@quantity}, \textit{@extent}, \textit{@precision}, \textit{@scope}) (att.ranging (\textit{@atLeast}, \textit{@atMost}, \textit{@min}, \textit{@max}, \textit{@confidence})) ) ) (att.written (\textit{@hand})) att.placement (\textit{@place}) att.typed (\textit{@type}, \textit{@subtype}) 
    \item[{Member of}]
  model.linePart model.pPart.transcriptional 
    \item[{Contained by}]
  
    \item[analysis: ]
   cl pc phr s w\par 
    \item[core: ]
   abbr add addrLine author bibl biblScope citedRange corr date del distinct editor email emph expan foreign gloss head headItem headLabel hi item l label measure mentioned name note num orig p pubPlace publisher q quote ref reg rs said sic soCalled speaker stage street term textLang time title unclear\par 
    \item[figures: ]
   cell\par 
    \item[header: ]
   change distributor edition extent geoDecl handNote licence scriptNote typeNote\par 
    \item[linking: ]
   ab seg\par 
    \item[msdescription: ]
   accMat acquisition additions catchwords collation colophon condition custEvent decoNote explicit filiation finalRubric foliation heraldry incipit layout material musicNotation objectType origDate origPlace origin provenance rubric secFol signatures source stamp summary support surrogates watermark\par 
    \item[namesdates: ]
   addName affiliation birth bloc country death district education faith floruit forename genName geogFeat geogName nameLink nationality occupation offset orgName persName placeName region residence roleName settlement sex socecStatus surname\par 
    \item[textcrit: ]
   lem rdg wit witDetail\par 
    \item[textstructure: ]
   byline closer dateline docAuthor docDate docEdition docImprint imprimatur opener salute signed titlePart trailer\par 
    \item[transcr: ]
   am damage fw line metamark mod restore retrace secl subst supplied surplus zone
    \item[{May contain}]
  
    \item[analysis: ]
   c cl interp interpGrp m pc phr s span spanGrp w\par 
    \item[core: ]
   abbr add address bibl biblStruct cb choice cit corr date del desc distinct email emph expan foreign gap gb gloss graphic hi index l label lb lg list listBibl measure measureGrp media mentioned milestone name note num orig pb ptr q quote ref reg rs said sic soCalled stage term time title unclear\par 
    \item[figures: ]
   figure formula notatedMusic table\par 
    \item[gaiji: ]
   g\par 
    \item[header: ]
   biblFull idno\par 
    \item[linking: ]
   alt altGrp anchor join joinGrp link linkGrp seg timeline\par 
    \item[msdescription: ]
   catchwords depth dim dimensions height heraldry locus locusGrp material msDesc objectType origDate origPlace secFol signatures stamp watermark width\par 
    \item[namesdates: ]
   addName affiliation bloc climate country district forename genName geo geogFeat geogName listEvent listNym listOrg listPerson listPlace location nameLink offset orgName persName placeName population region roleName settlement state surname terrain trait\par 
    \item[textcrit: ]
   app listApp listWit witDetail\par 
    \item[textstructure: ]
   floatingText\par 
    \item[transcr: ]
   addSpan am damage damageSpan delSpan ex fw handShift listTranspose metamark mod redo restore retrace secl space subst substJoin supplied surplus undo\par character data
    \item[{Note}]
  \par
In a diplomatic edition attempting to represent an original source, the <add> element should not be used for additions to the current TEI electronic edition made by editors or encoders. In these cases, either the <corr> or <supplied> element are recommended.\par
In a TEI edition of a historical text with previous editorial emendations in which such additions or reconstructions are considered part of the source text, the use of <add> may be appropriate, dependent on the editorial philosophy of the project.
    \item[{Example}]
  \leavevmode\bgroup\exampleFont \begin{shaded}\noindent\mbox{}The story I am\mbox{}\newline 
 going to relate is true as to its main facts, and as to the\mbox{}\newline 
 consequences {<\textbf{add}\hspace*{6pt}{place}="{above}">}of these facts{</\textbf{add}>} from which\mbox{}\newline 
 this tale takes its title.\end{shaded}\egroup 


    \item[{Content model}]
  \mbox{}\hfill\\[-10pt]\begin{Verbatim}[fontsize=\small]
<content>
 <macroRef key="macro.paraContent"/>
</content>
    
\end{Verbatim}

    \item[{Schema Declaration}]
  \mbox{}\hfill\\[-10pt]\begin{Verbatim}[fontsize=\small]
element add
{
   att.global.attributes,
   att.transcriptional.attributes,
   att.placement.attributes,
   att.typed.attributes,
   macro.paraContent}
\end{Verbatim}

\end{reflist}  \index{addName=<addName>|oddindex}
\begin{reflist}
\item[]\begin{specHead}{TEI.addName}{<addName> }(additional name) contains an additional name component, such as a nickname, epithet, or alias, or any other descriptive phrase used within a personal name. [\xref{http://www.tei-c.org/release/doc/tei-p5-doc/en/html/ND.html\#NDPER}{13.2.1. Personal Names}]\end{specHead} 
    \item[{Module}]
  namesdates
    \item[{Attributes}]
  Attributes att.global (\textit{@xml:id}, \textit{@n}, \textit{@xml:lang}, \textit{@xml:base}, \textit{@xml:space})  (att.global.rendition (\textit{@rend}, \textit{@style}, \textit{@rendition})) (att.global.linking (\textit{@corresp}, \textit{@synch}, \textit{@sameAs}, \textit{@copyOf}, \textit{@next}, \textit{@prev}, \textit{@exclude}, \textit{@select})) (att.global.analytic (\textit{@ana})) (att.global.facs (\textit{@facs})) (att.global.change (\textit{@change})) (att.global.responsibility (\textit{@cert}, \textit{@resp})) (att.global.source (\textit{@source})) att.personal (\textit{@full}, \textit{@sort})  (att.naming (\textit{@role}, \textit{@nymRef}) (att.canonical (\textit{@key}, \textit{@ref})) ) att.typed (\textit{@type}, \textit{@subtype}) 
    \item[{Member of}]
  model.persNamePart
    \item[{Contained by}]
  
    \item[analysis: ]
   cl phr s span\par 
    \item[core: ]
   abbr add addrLine address author bibl biblScope citedRange corr date del desc distinct editor email emph expan foreign gloss head headItem headLabel hi item l label measure meeting mentioned name note num orig p pubPlace publisher q quote ref reg resp rs said sic soCalled speaker stage street term textLang time title unclear\par 
    \item[figures: ]
   cell figDesc\par 
    \item[header: ]
   authority catDesc change classCode correspAction creation distributor edition extent funder geoDecl handNote language licence principal rendition scriptNote sponsor tagUsage typeNote\par 
    \item[linking: ]
   ab seg\par 
    \item[msdescription: ]
   accMat acquisition additions catchwords collation colophon condition custEvent decoNote explicit filiation finalRubric foliation heraldry incipit layout material musicNotation objectType origDate origPlace origin provenance rubric secFol signatures source stamp summary support surrogates watermark\par 
    \item[namesdates: ]
   addName affiliation age birth bloc country death district education faith floruit forename genName geogFeat geogName langKnown nameLink nationality occupation offset org orgName persName placeName region residence roleName settlement sex socecStatus surname\par 
    \item[textcrit: ]
   lem rdg wit witDetail witness\par 
    \item[textstructure: ]
   byline closer dateline docAuthor docDate docEdition docImprint imprimatur opener salute signed titlePart trailer\par 
    \item[transcr: ]
   damage fw metamark mod restore retrace secl supplied surplus
    \item[{May contain}]
  
    \item[analysis: ]
   c cl interp interpGrp m pc phr s span spanGrp w\par 
    \item[core: ]
   abbr add address cb choice corr date del distinct email emph expan foreign gap gb gloss graphic hi index lb measure measureGrp media mentioned milestone name note num orig pb ptr ref reg rs sic soCalled term time title unclear\par 
    \item[figures: ]
   figure formula notatedMusic\par 
    \item[gaiji: ]
   g\par 
    \item[header: ]
   idno\par 
    \item[linking: ]
   alt altGrp anchor join joinGrp link linkGrp seg timeline\par 
    \item[msdescription: ]
   catchwords depth dim dimensions height heraldry locus locusGrp material objectType origDate origPlace secFol signatures stamp watermark width\par 
    \item[namesdates: ]
   addName affiliation bloc climate country district forename genName geo geogFeat geogName location nameLink offset orgName persName placeName population region roleName settlement state surname terrain trait\par 
    \item[textcrit: ]
   app witDetail\par 
    \item[transcr: ]
   addSpan am damage damageSpan delSpan ex fw handShift listTranspose metamark mod redo restore retrace secl space subst substJoin supplied surplus undo\par character data
    \item[{Example}]
  \leavevmode\bgroup\exampleFont \begin{shaded}\noindent\mbox{}{<\textbf{persName}>}\mbox{}\newline 
\hspace*{6pt}{<\textbf{forename}>}Frederick{</\textbf{forename}>}\mbox{}\newline 
\hspace*{6pt}{<\textbf{addName}\hspace*{6pt}{type}="{epithet}">}the Great{</\textbf{addName}>}\mbox{}\newline 
\hspace*{6pt}{<\textbf{roleName}>}Emperor of Prussia{</\textbf{roleName}>}\mbox{}\newline 
{</\textbf{persName}>}\end{shaded}\egroup 


    \item[{Content model}]
  \mbox{}\hfill\\[-10pt]\begin{Verbatim}[fontsize=\small]
<content>
 <macroRef key="macro.phraseSeq"/>
</content>
    
\end{Verbatim}

    \item[{Schema Declaration}]
  \mbox{}\hfill\\[-10pt]\begin{Verbatim}[fontsize=\small]
element addName
{
   att.global.attributes,
   att.personal.attributes,
   att.typed.attributes,
   macro.phraseSeq}
\end{Verbatim}

\end{reflist}  \index{addSpan=<addSpan>|oddindex}
\begin{reflist}
\item[]\begin{specHead}{TEI.addSpan}{<addSpan> }(added span of text) marks the beginning of a longer sequence of text added by an author, scribe, annotator or corrector (see also <add>). [\xref{http://www.tei-c.org/release/doc/tei-p5-doc/en/html/PH.html\#PHAD}{11.3.1.4. Additions and Deletions}]\end{specHead} 
    \item[{Module}]
  transcr
    \item[{Attributes}]
  Attributes att.global (\textit{@xml:id}, \textit{@n}, \textit{@xml:lang}, \textit{@xml:base}, \textit{@xml:space})  (att.global.rendition (\textit{@rend}, \textit{@style}, \textit{@rendition})) (att.global.linking (\textit{@corresp}, \textit{@synch}, \textit{@sameAs}, \textit{@copyOf}, \textit{@next}, \textit{@prev}, \textit{@exclude}, \textit{@select})) (att.global.analytic (\textit{@ana})) (att.global.facs (\textit{@facs})) (att.global.change (\textit{@change})) (att.global.responsibility (\textit{@cert}, \textit{@resp})) (att.global.source (\textit{@source})) att.transcriptional (\textit{@status}, \textit{@cause}, \textit{@seq})  (att.editLike (\textit{@evidence}, \textit{@instant}) (att.dimensions (\textit{@unit}, \textit{@quantity}, \textit{@extent}, \textit{@precision}, \textit{@scope}) (att.ranging (\textit{@atLeast}, \textit{@atMost}, \textit{@min}, \textit{@max}, \textit{@confidence})) ) ) (att.written (\textit{@hand})) att.placement (\textit{@place}) att.typed (\textit{@type}, \textit{@subtype}) att.spanning (\textit{@spanTo}) 
    \item[{Member of}]
  model.global.edit
    \item[{Contained by}]
  
    \item[analysis: ]
   cl m phr s span w\par 
    \item[core: ]
   abbr add addrLine address author bibl biblScope cit citedRange corr date del distinct editor email emph expan foreign gloss head headItem headLabel hi imprint item l label lg list measure mentioned name note num orig p pubPlace publisher q quote ref reg resp rs said series sic soCalled sp speaker stage street term textLang time title unclear\par 
    \item[figures: ]
   cell figure table\par 
    \item[header: ]
   authority change classCode distributor edition extent funder geoDecl handNote language licence principal scriptNote sponsor typeNote\par 
    \item[linking: ]
   ab seg\par 
    \item[msdescription: ]
   accMat acquisition additions catchwords collation colophon condition custEvent decoNote explicit filiation finalRubric foliation heraldry incipit layout material msItem musicNotation objectType origDate origPlace origin provenance rubric secFol signatures source stamp summary support surrogates watermark\par 
    \item[namesdates: ]
   addName affiliation age birth bloc country death district education faith floruit forename genName geogFeat geogName langKnown nameLink nationality occupation offset orgName persName person personGrp placeName region residence roleName settlement sex socecStatus surname\par 
    \item[textcrit: ]
   lem rdg wit witDetail\par 
    \item[textstructure: ]
   argument back body byline closer dateline div docAuthor docDate docEdition docImprint docTitle epigraph floatingText front group imprimatur opener postscript salute signed text titlePage titlePart trailer\par 
    \item[transcr: ]
   damage fw line metamark mod restore retrace secl sourceDoc supplied surface surfaceGrp surplus zone
    \item[{May contain}]
  Empty element
    \item[{Note}]
  \par
Both the beginning and the end of the added material must be marked; the beginning by the <addSpan> element itself, the end by the {\itshape spanTo} attribute.
    \item[{Example}]
  \leavevmode\bgroup\exampleFont \begin{shaded}\noindent\mbox{}{<\textbf{handNote}\hspace*{6pt}{scribe}="{HelgiÓlafsson}"\mbox{}\newline 
\hspace*{6pt}{xml:id}="{HEOL}"/>}\mbox{}\newline 
\textit{<!-- ... -->}\mbox{}\newline 
{<\textbf{body}>}\mbox{}\newline 
\hspace*{6pt}{<\textbf{div}>}\mbox{}\newline 
\textit{<!-- text here -->}\mbox{}\newline 
\hspace*{6pt}{</\textbf{div}>}\mbox{}\newline 
\hspace*{6pt}{<\textbf{addSpan}\hspace*{6pt}{hand}="{\#HEOL}"\hspace*{6pt}{n}="{added gathering}"\mbox{}\newline 
\hspace*{6pt}\hspace*{6pt}{spanTo}="{\#P025}"/>}\mbox{}\newline 
\hspace*{6pt}{<\textbf{div}>}\mbox{}\newline 
\textit{<!-- text of first added poem here -->}\mbox{}\newline 
\hspace*{6pt}{</\textbf{div}>}\mbox{}\newline 
\hspace*{6pt}{<\textbf{div}>}\mbox{}\newline 
\textit{<!-- text of second added poem here -->}\mbox{}\newline 
\hspace*{6pt}{</\textbf{div}>}\mbox{}\newline 
\hspace*{6pt}{<\textbf{div}>}\mbox{}\newline 
\textit{<!-- text of third added poem here -->}\mbox{}\newline 
\hspace*{6pt}{</\textbf{div}>}\mbox{}\newline 
\hspace*{6pt}{<\textbf{div}>}\mbox{}\newline 
\textit{<!-- text of fourth added poem here -->}\mbox{}\newline 
\hspace*{6pt}{</\textbf{div}>}\mbox{}\newline 
\hspace*{6pt}{<\textbf{anchor}\hspace*{6pt}{xml:id}="{P025}"/>}\mbox{}\newline 
\hspace*{6pt}{<\textbf{div}>}\mbox{}\newline 
\textit{<!-- more text here -->}\mbox{}\newline 
\hspace*{6pt}{</\textbf{div}>}\mbox{}\newline 
{</\textbf{body}>}\end{shaded}\egroup 


    \item[{Schematron}]
   <sch:assert test="@spanTo">The @spanTo attribute of <sch:name/> is required.</sch:assert>
    \item[{Schematron}]
   <sch:assert test="@spanTo">L'attribut spanTo est requis.</sch:assert>
    \item[{Content model}]
  \fbox{\ttfamily <content/>\newline
    } 
    \item[{Schema Declaration}]
  \mbox{}\hfill\\[-10pt]\begin{Verbatim}[fontsize=\small]
element addSpan
{
   att.global.attributes,
   att.transcriptional.attributes,
   att.placement.attributes,
   att.typed.attributes,
   att.spanning.attributes,
   empty
}
\end{Verbatim}

\end{reflist}  \index{additional=<additional>|oddindex}
\begin{reflist}
\item[]\begin{specHead}{TEI.additional}{<additional> }groups additional information, combining bibliographic information about a manuscript, or surrogate copies of it with curatorial or administrative information. [\xref{http://www.tei-c.org/release/doc/tei-p5-doc/en/html/MS.html\#msad}{10.9. Additional Information}]\end{specHead} 
    \item[{Module}]
  msdescription
    \item[{Attributes}]
  Attributes att.global (\textit{@xml:id}, \textit{@n}, \textit{@xml:lang}, \textit{@xml:base}, \textit{@xml:space})  (att.global.rendition (\textit{@rend}, \textit{@style}, \textit{@rendition})) (att.global.linking (\textit{@corresp}, \textit{@synch}, \textit{@sameAs}, \textit{@copyOf}, \textit{@next}, \textit{@prev}, \textit{@exclude}, \textit{@select})) (att.global.analytic (\textit{@ana})) (att.global.facs (\textit{@facs})) (att.global.change (\textit{@change})) (att.global.responsibility (\textit{@cert}, \textit{@resp})) (att.global.source (\textit{@source}))
    \item[{Contained by}]
  
    \item[msdescription: ]
   msDesc msFrag msPart
    \item[{May contain}]
  
    \item[core: ]
   listBibl\par 
    \item[msdescription: ]
   adminInfo surrogates
    \item[{Example}]
  \leavevmode\bgroup\exampleFont \begin{shaded}\noindent\mbox{}{<\textbf{additional}>}\mbox{}\newline 
\hspace*{6pt}{<\textbf{adminInfo}>}\mbox{}\newline 
\hspace*{6pt}\hspace*{6pt}{<\textbf{recordHist}>}\mbox{}\newline 
\hspace*{6pt}\hspace*{6pt}\hspace*{6pt}{<\textbf{p}>}\mbox{}\newline 
\textit{<!-- record history here -->}\mbox{}\newline 
\hspace*{6pt}\hspace*{6pt}\hspace*{6pt}{</\textbf{p}>}\mbox{}\newline 
\hspace*{6pt}\hspace*{6pt}{</\textbf{recordHist}>}\mbox{}\newline 
\hspace*{6pt}\hspace*{6pt}{<\textbf{custodialHist}>}\mbox{}\newline 
\hspace*{6pt}\hspace*{6pt}\hspace*{6pt}{<\textbf{p}>}\mbox{}\newline 
\textit{<!-- custodial history here -->}\mbox{}\newline 
\hspace*{6pt}\hspace*{6pt}\hspace*{6pt}{</\textbf{p}>}\mbox{}\newline 
\hspace*{6pt}\hspace*{6pt}{</\textbf{custodialHist}>}\mbox{}\newline 
\hspace*{6pt}{</\textbf{adminInfo}>}\mbox{}\newline 
\hspace*{6pt}{<\textbf{surrogates}>}\mbox{}\newline 
\hspace*{6pt}\hspace*{6pt}{<\textbf{p}>}\mbox{}\newline 
\textit{<!-- information about surrogates here -->}\mbox{}\newline 
\hspace*{6pt}\hspace*{6pt}{</\textbf{p}>}\mbox{}\newline 
\hspace*{6pt}{</\textbf{surrogates}>}\mbox{}\newline 
\hspace*{6pt}{<\textbf{listBibl}>}\mbox{}\newline 
\hspace*{6pt}\hspace*{6pt}{<\textbf{bibl}>}\mbox{}\newline 
\textit{<!-- ... -->}\mbox{}\newline 
\hspace*{6pt}\hspace*{6pt}{</\textbf{bibl}>}\mbox{}\newline 
\textit{<!-- full bibliography here -->}\mbox{}\newline 
\hspace*{6pt}{</\textbf{listBibl}>}\mbox{}\newline 
{</\textbf{additional}>}\end{shaded}\egroup 


    \item[{Content model}]
  \mbox{}\hfill\\[-10pt]\begin{Verbatim}[fontsize=\small]
<content>
 <sequence>
  <elementRef key="adminInfo" minOccurs="0"/>
  <elementRef key="surrogates"
   minOccurs="0"/>
  <elementRef key="listBibl" minOccurs="0"/>
 </sequence>
</content>
    
\end{Verbatim}

    \item[{Schema Declaration}]
  \mbox{}\hfill\\[-10pt]\begin{Verbatim}[fontsize=\small]
element additional
{
   att.global.attributes,
   ( adminInfo?, surrogates?, listBibl? )
}
\end{Verbatim}

\end{reflist}  \index{additions=<additions>|oddindex}
\begin{reflist}
\item[]\begin{specHead}{TEI.additions}{<additions> }contains a description of any significant additions found within a manuscript, such as marginalia or other annotations. [\xref{http://www.tei-c.org/release/doc/tei-p5-doc/en/html/MS.html\#msph2}{10.7.2. Writing, Decoration, and Other Notations}]\end{specHead} 
    \item[{Module}]
  msdescription
    \item[{Attributes}]
  Attributes att.global (\textit{@xml:id}, \textit{@n}, \textit{@xml:lang}, \textit{@xml:base}, \textit{@xml:space})  (att.global.rendition (\textit{@rend}, \textit{@style}, \textit{@rendition})) (att.global.linking (\textit{@corresp}, \textit{@synch}, \textit{@sameAs}, \textit{@copyOf}, \textit{@next}, \textit{@prev}, \textit{@exclude}, \textit{@select})) (att.global.analytic (\textit{@ana})) (att.global.facs (\textit{@facs})) (att.global.change (\textit{@change})) (att.global.responsibility (\textit{@cert}, \textit{@resp})) (att.global.source (\textit{@source}))
    \item[{Member of}]
  model.physDescPart
    \item[{Contained by}]
  
    \item[msdescription: ]
   physDesc
    \item[{May contain}]
  
    \item[analysis: ]
   c cl interp interpGrp m pc phr s span spanGrp w\par 
    \item[core: ]
   abbr add address bibl biblStruct cb choice cit corr date del desc distinct email emph expan foreign gap gb gloss graphic hi index l label lb lg list listBibl measure measureGrp media mentioned milestone name note num orig p pb ptr q quote ref reg rs said sic soCalled sp stage term time title unclear\par 
    \item[figures: ]
   figure formula notatedMusic table\par 
    \item[gaiji: ]
   g\par 
    \item[header: ]
   biblFull idno\par 
    \item[linking: ]
   ab alt altGrp anchor join joinGrp link linkGrp seg timeline\par 
    \item[msdescription: ]
   catchwords depth dim dimensions height heraldry locus locusGrp material msDesc objectType origDate origPlace secFol signatures stamp watermark width\par 
    \item[namesdates: ]
   addName affiliation bloc climate country district forename genName geo geogFeat geogName listEvent listNym listOrg listPerson listPlace location nameLink offset orgName persName placeName population region roleName settlement state surname terrain trait\par 
    \item[textcrit: ]
   app listApp listWit witDetail\par 
    \item[textstructure: ]
   floatingText\par 
    \item[transcr: ]
   addSpan am damage damageSpan delSpan ex fw handShift listTranspose metamark mod redo restore retrace secl space subst substJoin supplied surplus undo\par character data
    \item[{Example}]
  \leavevmode\bgroup\exampleFont \begin{shaded}\noindent\mbox{}{<\textbf{additions}>}\mbox{}\newline 
\hspace*{6pt}{<\textbf{p}>}There are several marginalia in this manuscript. Some consist of\mbox{}\newline 
\hspace*{6pt}\hspace*{6pt} single characters and others are figurative. On 8v is to be found a drawing of\mbox{}\newline 
\hspace*{6pt}\hspace*{6pt} a mans head wearing a hat. At times sentences occurs: On 5v:\mbox{}\newline 
\hspace*{6pt}{<\textbf{q}\hspace*{6pt}{xml:lang}="{is}">}Her er skrif andres isslendin{</\textbf{q}>},\mbox{}\newline 
\hspace*{6pt}\hspace*{6pt} on 19r: {<\textbf{q}\hspace*{6pt}{xml:lang}="{is}">}þeim go{</\textbf{q}>},\mbox{}\newline 
\hspace*{6pt}\hspace*{6pt} on 21r: {<\textbf{q}\hspace*{6pt}{xml:lang}="{is}">}amen med aund ok munn halla rei knar hofud summu all huad\mbox{}\newline 
\hspace*{6pt}\hspace*{6pt}\hspace*{6pt}\hspace*{6pt} batar þad mælgi ok mal{</\textbf{q}>},\mbox{}\newline 
\hspace*{6pt}\hspace*{6pt} On 21v: some runic letters and the sentence {<\textbf{q}\hspace*{6pt}{xml:lang}="{la}">}aue maria gracia plena dominus{</\textbf{q}>}.{</\textbf{p}>}\mbox{}\newline 
{</\textbf{additions}>}\end{shaded}\egroup 


    \item[{Content model}]
  \mbox{}\hfill\\[-10pt]\begin{Verbatim}[fontsize=\small]
<content>
 <macroRef key="macro.specialPara"/>
</content>
    
\end{Verbatim}

    \item[{Schema Declaration}]
  \mbox{}\hfill\\[-10pt]\begin{Verbatim}[fontsize=\small]
element additions { att.global.attributes, macro.specialPara }
\end{Verbatim}

\end{reflist}  \index{addrLine=<addrLine>|oddindex}
\begin{reflist}
\item[]\begin{specHead}{TEI.addrLine}{<addrLine> }(address line) contains one line of a postal address. [\xref{http://www.tei-c.org/release/doc/tei-p5-doc/en/html/CO.html\#CONAAD}{3.5.2. Addresses} \xref{http://www.tei-c.org/release/doc/tei-p5-doc/en/html/HD.html\#HD24}{2.2.4. Publication, Distribution, Licensing, etc.} \xref{http://www.tei-c.org/release/doc/tei-p5-doc/en/html/CO.html\#COBICOI}{3.11.2.4. Imprint, Size of a Document, and Reprint Information}]\end{specHead} 
    \item[{Module}]
  core
    \item[{Attributes}]
  Attributes att.global (\textit{@xml:id}, \textit{@n}, \textit{@xml:lang}, \textit{@xml:base}, \textit{@xml:space})  (att.global.rendition (\textit{@rend}, \textit{@style}, \textit{@rendition})) (att.global.linking (\textit{@corresp}, \textit{@synch}, \textit{@sameAs}, \textit{@copyOf}, \textit{@next}, \textit{@prev}, \textit{@exclude}, \textit{@select})) (att.global.analytic (\textit{@ana})) (att.global.facs (\textit{@facs})) (att.global.change (\textit{@change})) (att.global.responsibility (\textit{@cert}, \textit{@resp})) (att.global.source (\textit{@source}))
    \item[{Member of}]
  model.addrPart
    \item[{Contained by}]
  
    \item[core: ]
   address
    \item[{May contain}]
  
    \item[analysis: ]
   c cl interp interpGrp m pc phr s span spanGrp w\par 
    \item[core: ]
   abbr add address cb choice corr date del distinct email emph expan foreign gap gb gloss graphic hi index lb measure measureGrp media mentioned milestone name note num orig pb ptr ref reg rs sic soCalled term time title unclear\par 
    \item[figures: ]
   figure formula notatedMusic\par 
    \item[gaiji: ]
   g\par 
    \item[header: ]
   idno\par 
    \item[linking: ]
   alt altGrp anchor join joinGrp link linkGrp seg timeline\par 
    \item[msdescription: ]
   catchwords depth dim dimensions height heraldry locus locusGrp material objectType origDate origPlace secFol signatures stamp watermark width\par 
    \item[namesdates: ]
   addName affiliation bloc climate country district forename genName geo geogFeat geogName location nameLink offset orgName persName placeName population region roleName settlement state surname terrain trait\par 
    \item[textcrit: ]
   app witDetail\par 
    \item[transcr: ]
   addSpan am damage damageSpan delSpan ex fw handShift listTranspose metamark mod redo restore retrace secl space subst substJoin supplied surplus undo\par character data
    \item[{Note}]
  \par
Addresses may be encoded either as a sequence of lines, or using any sequence of component elements from the \textsf{model.addrPart} class. Other non-postal forms of address, such as telephone numbers or email, should not be included within an <address> element directly but may be wrapped within an <addrLine> if they form part of the printed address in some source text.
    \item[{Example}]
  \leavevmode\bgroup\exampleFont \begin{shaded}\noindent\mbox{}{<\textbf{address}>}\mbox{}\newline 
\hspace*{6pt}{<\textbf{addrLine}>}Computing Center, MC 135{</\textbf{addrLine}>}\mbox{}\newline 
\hspace*{6pt}{<\textbf{addrLine}>}P.O. Box 6998{</\textbf{addrLine}>}\mbox{}\newline 
\hspace*{6pt}{<\textbf{addrLine}>}Chicago, IL{</\textbf{addrLine}>}\mbox{}\newline 
\hspace*{6pt}{<\textbf{addrLine}>}60680 USA{</\textbf{addrLine}>}\mbox{}\newline 
{</\textbf{address}>}\end{shaded}\egroup 


    \item[{Example}]
  \leavevmode\bgroup\exampleFont \begin{shaded}\noindent\mbox{}{<\textbf{addrLine}>}\mbox{}\newline 
\hspace*{6pt}{<\textbf{ref}\hspace*{6pt}{target}="{tel:+1-201-555-0123}">}(201) 555 0123{</\textbf{ref}>}\mbox{}\newline 
{</\textbf{addrLine}>}\end{shaded}\egroup 


    \item[{Content model}]
  \mbox{}\hfill\\[-10pt]\begin{Verbatim}[fontsize=\small]
<content>
 <macroRef key="macro.phraseSeq"/>
</content>
    
\end{Verbatim}

    \item[{Schema Declaration}]
  \mbox{}\hfill\\[-10pt]\begin{Verbatim}[fontsize=\small]
element addrLine { att.global.attributes, macro.phraseSeq }
\end{Verbatim}

\end{reflist}  \index{address=<address>|oddindex}
\begin{reflist}
\item[]\begin{specHead}{TEI.address}{<address> }contains a postal address, for example of a publisher, an organization, or an individual. [\xref{http://www.tei-c.org/release/doc/tei-p5-doc/en/html/CO.html\#CONAAD}{3.5.2. Addresses} \xref{http://www.tei-c.org/release/doc/tei-p5-doc/en/html/HD.html\#HD24}{2.2.4. Publication, Distribution, Licensing, etc.} \xref{http://www.tei-c.org/release/doc/tei-p5-doc/en/html/CO.html\#COBICOI}{3.11.2.4. Imprint, Size of a Document, and Reprint Information}]\end{specHead} 
    \item[{Module}]
  core
    \item[{Attributes}]
  Attributes att.global (\textit{@xml:id}, \textit{@n}, \textit{@xml:lang}, \textit{@xml:base}, \textit{@xml:space})  (att.global.rendition (\textit{@rend}, \textit{@style}, \textit{@rendition})) (att.global.linking (\textit{@corresp}, \textit{@synch}, \textit{@sameAs}, \textit{@copyOf}, \textit{@next}, \textit{@prev}, \textit{@exclude}, \textit{@select})) (att.global.analytic (\textit{@ana})) (att.global.facs (\textit{@facs})) (att.global.change (\textit{@change})) (att.global.responsibility (\textit{@cert}, \textit{@resp})) (att.global.source (\textit{@source}))
    \item[{Member of}]
  model.addressLike model.publicationStmtPart.detail
    \item[{Contained by}]
  
    \item[analysis: ]
   cl phr s span\par 
    \item[core: ]
   abbr add addrLine author bibl biblScope citedRange corr date del desc distinct editor email emph expan foreign gloss head headItem headLabel hi item l label measure meeting mentioned name note num orig p pubPlace publisher q quote ref reg resp rs said sic soCalled speaker stage street term textLang time title unclear\par 
    \item[figures: ]
   cell figDesc\par 
    \item[header: ]
   authority catDesc change classCode correspAction creation distributor edition extent funder geoDecl handNote language licence principal publicationStmt rendition scriptNote sponsor tagUsage typeNote\par 
    \item[linking: ]
   ab seg\par 
    \item[msdescription: ]
   accMat acquisition additions catchwords collation colophon condition custEvent decoNote explicit filiation finalRubric foliation heraldry incipit layout material musicNotation objectType origDate origPlace origin provenance rubric secFol signatures source stamp summary support surrogates watermark\par 
    \item[namesdates: ]
   addName affiliation age birth bloc country death district education faith floruit forename genName geogFeat geogName langKnown location nameLink nationality occupation offset orgName persName placeName region residence roleName settlement sex socecStatus surname\par 
    \item[textcrit: ]
   lem rdg wit witDetail witness\par 
    \item[textstructure: ]
   byline closer dateline docAuthor docDate docEdition docImprint imprimatur opener salute signed titlePart trailer\par 
    \item[transcr: ]
   damage fw metamark mod restore retrace secl supplied surplus
    \item[{May contain}]
  
    \item[analysis: ]
   interp interpGrp span spanGrp\par 
    \item[core: ]
   addrLine cb gap gb index lb milestone name note pb rs street\par 
    \item[figures: ]
   figure notatedMusic\par 
    \item[header: ]
   idno\par 
    \item[linking: ]
   alt altGrp anchor join joinGrp link linkGrp timeline\par 
    \item[namesdates: ]
   addName bloc climate country district forename genName geogFeat geogName location nameLink offset orgName persName placeName population region roleName settlement state surname terrain trait\par 
    \item[textcrit: ]
   app witDetail\par 
    \item[transcr: ]
   addSpan damageSpan delSpan fw listTranspose metamark space substJoin
    \item[{Note}]
  \par
This element should be used for postal addresses only. Within it, the generic element <addrLine> may be used as an alternative to any of the more specialized elements available from the \textsf{model.addrPart} class, such as <street>, \texttt{<postCode>} etc.
    \item[{Example}]
  \leavevmode\bgroup\exampleFont \begin{shaded}\noindent\mbox{}{<\textbf{address}>}\mbox{}\newline 
\hspace*{6pt}{<\textbf{street}>}via Marsala 24{</\textbf{street}>}\mbox{}\newline 
\hspace*{6pt}{<\textbf{postCode}>}40126{</\textbf{postCode}>}\mbox{}\newline 
\hspace*{6pt}{<\textbf{name}>}Bologna{</\textbf{name}>}\mbox{}\newline 
\hspace*{6pt}{<\textbf{name}\hspace*{6pt}{n}="{I}">}Italy{</\textbf{name}>}\mbox{}\newline 
{</\textbf{address}>}\end{shaded}\egroup 


    \item[{Example}]
  \leavevmode\bgroup\exampleFont \begin{shaded}\noindent\mbox{}{<\textbf{address}>}\mbox{}\newline 
\hspace*{6pt}{<\textbf{addrLine}>}Computing Center, MC 135{</\textbf{addrLine}>}\mbox{}\newline 
\hspace*{6pt}{<\textbf{addrLine}>}P.O. Box 6998{</\textbf{addrLine}>}\mbox{}\newline 
\hspace*{6pt}{<\textbf{addrLine}>}Chicago, IL 60680{</\textbf{addrLine}>}\mbox{}\newline 
\hspace*{6pt}{<\textbf{addrLine}>}USA{</\textbf{addrLine}>}\mbox{}\newline 
{</\textbf{address}>}\end{shaded}\egroup 


    \item[{Example}]
  \leavevmode\bgroup\exampleFont \begin{shaded}\noindent\mbox{}{<\textbf{address}>}\mbox{}\newline 
\hspace*{6pt}{<\textbf{country}\hspace*{6pt}{key}="{FR}"/>}\mbox{}\newline 
\hspace*{6pt}{<\textbf{settlement}\hspace*{6pt}{type}="{city}">}Lyon{</\textbf{settlement}>}\mbox{}\newline 
\hspace*{6pt}{<\textbf{postCode}>}69002{</\textbf{postCode}>}\mbox{}\newline 
\hspace*{6pt}{<\textbf{district}\hspace*{6pt}{type}="{arrondissement}">}IIème{</\textbf{district}>}\mbox{}\newline 
\hspace*{6pt}{<\textbf{district}\hspace*{6pt}{type}="{quartier}">}Perrache{</\textbf{district}>}\mbox{}\newline 
\hspace*{6pt}{<\textbf{street}>}\mbox{}\newline 
\hspace*{6pt}\hspace*{6pt}{<\textbf{num}>}30{</\textbf{num}>}, Cours de Verdun{</\textbf{street}>}\mbox{}\newline 
{</\textbf{address}>}\end{shaded}\egroup 


    \item[{Content model}]
  \mbox{}\hfill\\[-10pt]\begin{Verbatim}[fontsize=\small]
<content>
 <sequence>
  <classRef key="model.global"
   maxOccurs="unbounded" minOccurs="0"/>
  <sequence maxOccurs="unbounded"
   minOccurs="1">
   <classRef key="model.addrPart"/>
   <classRef key="model.global"
    maxOccurs="unbounded" minOccurs="0"/>
  </sequence>
 </sequence>
</content>
    
\end{Verbatim}

    \item[{Schema Declaration}]
  \mbox{}\hfill\\[-10pt]\begin{Verbatim}[fontsize=\small]
element address
{
   att.global.attributes,
   ( model.global*, ( model.addrPart, model.global* )+ )
}
\end{Verbatim}

\end{reflist}  \index{adminInfo=<adminInfo>|oddindex}
\begin{reflist}
\item[]\begin{specHead}{TEI.adminInfo}{<adminInfo> }(administrative information) contains information about the present custody and availability of the manuscript, and also about the record description itself. [\xref{http://www.tei-c.org/release/doc/tei-p5-doc/en/html/MS.html\#msadad}{10.9.1. Administrative Information}]\end{specHead} 
    \item[{Module}]
  msdescription
    \item[{Attributes}]
  Attributes att.global (\textit{@xml:id}, \textit{@n}, \textit{@xml:lang}, \textit{@xml:base}, \textit{@xml:space})  (att.global.rendition (\textit{@rend}, \textit{@style}, \textit{@rendition})) (att.global.linking (\textit{@corresp}, \textit{@synch}, \textit{@sameAs}, \textit{@copyOf}, \textit{@next}, \textit{@prev}, \textit{@exclude}, \textit{@select})) (att.global.analytic (\textit{@ana})) (att.global.facs (\textit{@facs})) (att.global.change (\textit{@change})) (att.global.responsibility (\textit{@cert}, \textit{@resp})) (att.global.source (\textit{@source}))
    \item[{Contained by}]
  
    \item[msdescription: ]
   additional
    \item[{May contain}]
  
    \item[core: ]
   note\par 
    \item[header: ]
   availability\par 
    \item[msdescription: ]
   custodialHist recordHist\par 
    \item[textcrit: ]
   witDetail
    \item[{Example}]
  \leavevmode\bgroup\exampleFont \begin{shaded}\noindent\mbox{}{<\textbf{adminInfo}>}\mbox{}\newline 
\hspace*{6pt}{<\textbf{recordHist}>}\mbox{}\newline 
\hspace*{6pt}\hspace*{6pt}{<\textbf{source}>}Record created {<\textbf{date}>}1 Aug 2004{</\textbf{date}>}\mbox{}\newline 
\hspace*{6pt}\hspace*{6pt}{</\textbf{source}>}\mbox{}\newline 
\hspace*{6pt}{</\textbf{recordHist}>}\mbox{}\newline 
\hspace*{6pt}{<\textbf{availability}>}\mbox{}\newline 
\hspace*{6pt}\hspace*{6pt}{<\textbf{p}>}Until 2015 permission to photocopy some materials from this\mbox{}\newline 
\hspace*{6pt}\hspace*{6pt}\hspace*{6pt}\hspace*{6pt} collection has been limited at the request of the donor. Please ask repository staff for details\mbox{}\newline 
\hspace*{6pt}\hspace*{6pt}\hspace*{6pt}\hspace*{6pt} if you are interested in obtaining photocopies from Series 1:\mbox{}\newline 
\hspace*{6pt}\hspace*{6pt}\hspace*{6pt}\hspace*{6pt} Correspondence.{</\textbf{p}>}\mbox{}\newline 
\hspace*{6pt}{</\textbf{availability}>}\mbox{}\newline 
\hspace*{6pt}{<\textbf{custodialHist}>}\mbox{}\newline 
\hspace*{6pt}\hspace*{6pt}{<\textbf{p}>}Collection donated to the Manuscript Library by the Estate of\mbox{}\newline 
\hspace*{6pt}\hspace*{6pt}\hspace*{6pt}\hspace*{6pt} Edgar Holden in 1993. Donor number: 1993-034.{</\textbf{p}>}\mbox{}\newline 
\hspace*{6pt}{</\textbf{custodialHist}>}\mbox{}\newline 
{</\textbf{adminInfo}>}\end{shaded}\egroup 


    \item[{Content model}]
  \mbox{}\hfill\\[-10pt]\begin{Verbatim}[fontsize=\small]
<content>
 <sequence>
  <elementRef key="recordHist"
   minOccurs="0"/>
  <elementRef key="availability"
   minOccurs="0"/>
  <elementRef key="custodialHist"
   minOccurs="0"/>
  <classRef key="model.noteLike"
   minOccurs="0"/>
 </sequence>
</content>
    
\end{Verbatim}

    \item[{Schema Declaration}]
  \mbox{}\hfill\\[-10pt]\begin{Verbatim}[fontsize=\small]
element adminInfo
{
   att.global.attributes,
   ( recordHist?, availability?, custodialHist?, model.noteLike? )
}
\end{Verbatim}

\end{reflist}  \index{affiliation=<affiliation>|oddindex}
\begin{reflist}
\item[]\begin{specHead}{TEI.affiliation}{<affiliation> }contains an informal description of a person's present or past affiliation with some organization, for example an employer or sponsor. [\xref{http://www.tei-c.org/release/doc/tei-p5-doc/en/html/CC.html\#CCAHPA}{15.2.2. The Participant Description}]\end{specHead} 
    \item[{Module}]
  namesdates
    \item[{Attributes}]
  Attributes att.global (\textit{@xml:id}, \textit{@n}, \textit{@xml:lang}, \textit{@xml:base}, \textit{@xml:space})  (att.global.rendition (\textit{@rend}, \textit{@style}, \textit{@rendition})) (att.global.linking (\textit{@corresp}, \textit{@synch}, \textit{@sameAs}, \textit{@copyOf}, \textit{@next}, \textit{@prev}, \textit{@exclude}, \textit{@select})) (att.global.analytic (\textit{@ana})) (att.global.facs (\textit{@facs})) (att.global.change (\textit{@change})) (att.global.responsibility (\textit{@cert}, \textit{@resp})) (att.global.source (\textit{@source})) att.editLike (\textit{@evidence}, \textit{@instant})  (att.dimensions (\textit{@unit}, \textit{@quantity}, \textit{@extent}, \textit{@precision}, \textit{@scope}) (att.ranging (\textit{@atLeast}, \textit{@atMost}, \textit{@min}, \textit{@max}, \textit{@confidence})) ) att.datable (\textit{@calendar}, \textit{@period})  (att.datable.w3c (\textit{@when}, \textit{@notBefore}, \textit{@notAfter}, \textit{@from}, \textit{@to})) (att.datable.iso (\textit{@when-iso}, \textit{@notBefore-iso}, \textit{@notAfter-iso}, \textit{@from-iso}, \textit{@to-iso})) (att.datable.custom (\textit{@when-custom}, \textit{@notBefore-custom}, \textit{@notAfter-custom}, \textit{@from-custom}, \textit{@to-custom}, \textit{@datingPoint}, \textit{@datingMethod})) att.naming (\textit{@role}, \textit{@nymRef})  (att.canonical (\textit{@key}, \textit{@ref}))
    \item[{Member of}]
  model.addressLike model.persStateLike
    \item[{Contained by}]
  
    \item[analysis: ]
   cl phr s span\par 
    \item[core: ]
   abbr add addrLine author bibl biblScope citedRange corr date del desc distinct editor email emph expan foreign gloss head headItem headLabel hi item l label measure meeting mentioned name note num orig p pubPlace publisher q quote ref reg resp rs said sic soCalled speaker stage street term textLang time title unclear\par 
    \item[figures: ]
   cell figDesc\par 
    \item[header: ]
   authority catDesc change classCode correspAction creation distributor edition extent funder geoDecl handNote language licence principal rendition scriptNote sponsor tagUsage typeNote\par 
    \item[linking: ]
   ab seg\par 
    \item[msdescription: ]
   accMat acquisition additions catchwords collation colophon condition custEvent decoNote explicit filiation finalRubric foliation heraldry incipit layout material musicNotation objectType origDate origPlace origin provenance rubric secFol signatures source stamp summary support surrogates watermark\par 
    \item[namesdates: ]
   addName affiliation age birth bloc country death district education faith floruit forename genName geogFeat geogName langKnown location nameLink nationality occupation offset orgName persName person personGrp placeName region residence roleName settlement sex socecStatus surname\par 
    \item[textcrit: ]
   lem rdg wit witDetail witness\par 
    \item[textstructure: ]
   byline closer dateline docAuthor docDate docEdition docImprint imprimatur opener salute signed titlePart trailer\par 
    \item[transcr: ]
   damage fw metamark mod restore retrace secl supplied surplus
    \item[{May contain}]
  
    \item[analysis: ]
   c cl interp interpGrp m pc phr s span spanGrp w\par 
    \item[core: ]
   abbr add address cb choice corr date del distinct email emph expan foreign gap gb gloss graphic hi index lb measure measureGrp media mentioned milestone name note num orig pb ptr ref reg rs sic soCalled term time title unclear\par 
    \item[figures: ]
   figure formula notatedMusic\par 
    \item[gaiji: ]
   g\par 
    \item[header: ]
   idno\par 
    \item[linking: ]
   alt altGrp anchor join joinGrp link linkGrp seg timeline\par 
    \item[msdescription: ]
   catchwords depth dim dimensions height heraldry locus locusGrp material objectType origDate origPlace secFol signatures stamp watermark width\par 
    \item[namesdates: ]
   addName affiliation bloc climate country district forename genName geo geogFeat geogName location nameLink offset orgName persName placeName population region roleName settlement state surname terrain trait\par 
    \item[textcrit: ]
   app witDetail\par 
    \item[transcr: ]
   addSpan am damage damageSpan delSpan ex fw handShift listTranspose metamark mod redo restore retrace secl space subst substJoin supplied surplus undo\par character data
    \item[{Note}]
  \par
If included, the name of an organization may be tagged using either the <name> element as above, or the more specific <orgName> element.
    \item[{Example}]
  \leavevmode\bgroup\exampleFont \begin{shaded}\noindent\mbox{}{<\textbf{affiliation}>}Junior project officer for the US {<\textbf{name}\hspace*{6pt}{type}="{org}">}National Endowment for\mbox{}\newline 
\hspace*{6pt}\hspace*{6pt} the Humanities{</\textbf{name}>}\mbox{}\newline 
{</\textbf{affiliation}>}\mbox{}\newline 
{<\textbf{affiliation}\hspace*{6pt}{notAfter}="{1960-01-01}"\mbox{}\newline 
\hspace*{6pt}{notBefore}="{1957-02-28}">}Paid up member of the\mbox{}\newline 
{<\textbf{orgName}>}Australian Journalists Association{</\textbf{orgName}>}\mbox{}\newline 
{</\textbf{affiliation}>}\end{shaded}\egroup 


    \item[{Content model}]
  \mbox{}\hfill\\[-10pt]\begin{Verbatim}[fontsize=\small]
<content>
 <macroRef key="macro.phraseSeq"/>
</content>
    
\end{Verbatim}

    \item[{Schema Declaration}]
  \mbox{}\hfill\\[-10pt]\begin{Verbatim}[fontsize=\small]
element affiliation
{
   att.global.attributes,
   att.editLike.attributes,
   att.datable.attributes,
   att.naming.attributes,
   macro.phraseSeq}
\end{Verbatim}

\end{reflist}  \index{age=<age>|oddindex}\index{value=@value!<age>|oddindex}
\begin{reflist}
\item[]\begin{specHead}{TEI.age}{<age> }specifies the age of a person. [\xref{http://www.tei-c.org/release/doc/tei-p5-doc/en/html/ND.html\#NDPERSEpc}{13.3.2.1. Personal Characteristics}]\end{specHead} 
    \item[{Module}]
  namesdates
    \item[{Attributes}]
  Attributes att.global (\textit{@xml:id}, \textit{@n}, \textit{@xml:lang}, \textit{@xml:base}, \textit{@xml:space})  (att.global.rendition (\textit{@rend}, \textit{@style}, \textit{@rendition})) (att.global.linking (\textit{@corresp}, \textit{@synch}, \textit{@sameAs}, \textit{@copyOf}, \textit{@next}, \textit{@prev}, \textit{@exclude}, \textit{@select})) (att.global.analytic (\textit{@ana})) (att.global.facs (\textit{@facs})) (att.global.change (\textit{@change})) (att.global.responsibility (\textit{@cert}, \textit{@resp})) (att.global.source (\textit{@source})) att.editLike (\textit{@evidence}, \textit{@instant})  (att.dimensions (\textit{@unit}, \textit{@quantity}, \textit{@extent}, \textit{@precision}, \textit{@scope}) (att.ranging (\textit{@atLeast}, \textit{@atMost}, \textit{@min}, \textit{@max}, \textit{@confidence})) ) att.datable (\textit{@calendar}, \textit{@period})  (att.datable.w3c (\textit{@when}, \textit{@notBefore}, \textit{@notAfter}, \textit{@from}, \textit{@to})) (att.datable.iso (\textit{@when-iso}, \textit{@notBefore-iso}, \textit{@notAfter-iso}, \textit{@from-iso}, \textit{@to-iso})) (att.datable.custom (\textit{@when-custom}, \textit{@notBefore-custom}, \textit{@notAfter-custom}, \textit{@from-custom}, \textit{@to-custom}, \textit{@datingPoint}, \textit{@datingMethod})) \hfil\\[-10pt]\begin{sansreflist}
    \item[@value]
  supplies a numeric code representing the age or age group
\begin{reflist}
    \item[{Status}]
  Optional
    \item[{Datatype}]
  teidata.count
    \item[{Note}]
  \par
This attribute may be used to complement a more detailed discussion of a person's age in the content of the element
\end{reflist}  
\end{sansreflist}  
    \item[{Member of}]
  model.persStateLike
    \item[{Contained by}]
  
    \item[namesdates: ]
   person personGrp
    \item[{May contain}]
  
    \item[analysis: ]
   interp interpGrp span spanGrp\par 
    \item[core: ]
   abbr address cb choice date distinct email emph expan foreign gap gb gloss hi index lb measure measureGrp mentioned milestone name note num pb ptr ref rs soCalled term time title\par 
    \item[figures: ]
   figure notatedMusic\par 
    \item[header: ]
   idno\par 
    \item[linking: ]
   alt altGrp anchor join joinGrp link linkGrp timeline\par 
    \item[msdescription: ]
   catchwords depth dim dimensions height heraldry locus locusGrp material objectType origDate origPlace secFol signatures stamp watermark width\par 
    \item[namesdates: ]
   addName affiliation bloc climate country district forename genName geo geogFeat geogName location nameLink offset orgName persName placeName population region roleName settlement state surname terrain trait\par 
    \item[textcrit: ]
   app witDetail\par 
    \item[transcr: ]
   addSpan am damageSpan delSpan ex fw listTranspose metamark space subst substJoin\par character data
    \item[{Note}]
  \par
As with other culturally-constructed traits such as sex, the way in which this concept is described in different cultural contexts may vary. The normalizing attributes are provided as a means of simplifying that variety to Western European norms and should not be used where that is inappropriate. The content of the element may be used to describe the intended concept in more detail, using plain text. 
    \item[{Example}]
  \leavevmode\bgroup\exampleFont \begin{shaded}\noindent\mbox{}{<\textbf{age}\hspace*{6pt}{notAfter}="{1986}"\hspace*{6pt}{value}="{2}">}under 20 in the early eighties{</\textbf{age}>}\end{shaded}\egroup 


    \item[{Content model}]
  \mbox{}\hfill\\[-10pt]\begin{Verbatim}[fontsize=\small]
<content>
 <macroRef key="macro.phraseSeq.limited"/>
</content>
    
\end{Verbatim}

    \item[{Schema Declaration}]
  \mbox{}\hfill\\[-10pt]\begin{Verbatim}[fontsize=\small]
element age
{
   att.global.attributes,
   att.editLike.attributes,
   att.datable.attributes,
   attribute value { text }?,
   macro.phraseSeq.limited}
\end{Verbatim}

\end{reflist}  \index{alt=<alt>|oddindex}\index{target=@target!<alt>|oddindex}\index{mode=@mode!<alt>|oddindex}\index{weights=@weights!<alt>|oddindex}
\begin{reflist}
\item[]\begin{specHead}{TEI.alt}{<alt> }(alternation) identifies an alternation or a set of choices among elements or passages. [\xref{http://www.tei-c.org/release/doc/tei-p5-doc/en/html/SA.html\#SAAT}{16.8. Alternation}]\end{specHead} 
    \item[{Module}]
  linking
    \item[{Attributes}]
  Attributes att.global (\textit{@xml:id}, \textit{@n}, \textit{@xml:lang}, \textit{@xml:base}, \textit{@xml:space})  (att.global.rendition (\textit{@rend}, \textit{@style}, \textit{@rendition})) (att.global.linking (\textit{@corresp}, \textit{@synch}, \textit{@sameAs}, \textit{@copyOf}, \textit{@next}, \textit{@prev}, \textit{@exclude}, \textit{@select})) (att.global.analytic (\textit{@ana})) (att.global.facs (\textit{@facs})) (att.global.change (\textit{@change})) (att.global.responsibility (\textit{@cert}, \textit{@resp})) (att.global.source (\textit{@source})) att.typed (\textit{@type}, \textit{@subtype}) att.pointing (\unusedattribute{target}, @targetLang, @evaluate) \hfil\\[-10pt]\begin{sansreflist}
    \item[@target]
  specifies the destination of the reference by supplying one or more URI References
\begin{reflist}
    \item[{Derived from}]
  att.pointing
    \item[{Status}]
  Optional
    \item[{Datatype}]
  2–∞ occurrences of teidata.pointer separated by whitespace
\end{reflist}  
    \item[@mode]
  states whether the alternations gathered in this collection are exclusive or inclusive.
\begin{reflist}
    \item[{Status}]
  Recommended
    \item[{Datatype}]
  teidata.enumerated
    \item[{Legal values are:}]
  \begin{description}

\item[{excl}](exclusive) indicates that the alternation is exclusive, i.e. that at most one of the alternatives occurs.
\item[{incl}](inclusive) indicates that the alternation is not exclusive, i.e. that one or more of the alternatives occur.
\end{description} 
\end{reflist}  
    \item[@weights]
  If {\itshape mode} is \texttt{excl}, each weight states the probability that the corresponding alternative occurs. If {\itshape mode} is incl each weight states the probability that the corresponding alternative occurs given that at least one of the other alternatives occurs.
\begin{reflist}
    \item[{Status}]
  Optional
    \item[{Datatype}]
  2–∞ occurrences of teidata.probability separated by whitespace
    \item[{Note}]
  \par
If {\itshape mode} is \texttt{excl}, the sum of weights must be 1. If {\itshape mode} is \texttt{incl}, the sum of weights must be in the range from 0 to the number of alternants.
\end{reflist}  
\end{sansreflist}  
    \item[{Member of}]
  model.global.meta
    \item[{Contained by}]
  
    \item[analysis: ]
   cl m phr s span w\par 
    \item[core: ]
   abbr add addrLine address author bibl biblScope cit citedRange corr date del distinct editor email emph expan foreign gloss head headItem headLabel hi imprint item l label lg list measure mentioned name note num orig p pubPlace publisher q quote ref reg resp rs said series sic soCalled sp speaker stage street term textLang time title unclear\par 
    \item[figures: ]
   cell figure table\par 
    \item[header: ]
   authority change classCode distributor edition extent funder geoDecl handNote language licence principal scriptNote sponsor typeNote\par 
    \item[linking: ]
   ab altGrp seg\par 
    \item[msdescription: ]
   accMat acquisition additions catchwords collation colophon condition custEvent decoNote explicit filiation finalRubric foliation heraldry incipit layout material msItem musicNotation objectType origDate origPlace origin provenance rubric secFol signatures source stamp summary support surrogates watermark\par 
    \item[namesdates: ]
   addName affiliation age birth bloc country death district education faith floruit forename genName geogFeat geogName langKnown nameLink nationality occupation offset orgName persName person personGrp placeName region residence roleName settlement sex socecStatus surname\par 
    \item[textcrit: ]
   lem rdg wit witDetail\par 
    \item[textstructure: ]
   argument back body byline closer dateline div docAuthor docDate docEdition docImprint docTitle epigraph floatingText front group imprimatur opener postscript salute signed text titlePage titlePart trailer\par 
    \item[transcr: ]
   damage fw line metamark mod restore retrace secl sourceDoc supplied surface surfaceGrp surplus zone
    \item[{May contain}]
  Empty element
    \item[{Example}]
  \leavevmode\bgroup\exampleFont \begin{shaded}\noindent\mbox{}{<\textbf{alt}\hspace*{6pt}{mode}="{excl}"\hspace*{6pt}{target}="{\#we.fun \#we.sun}"\mbox{}\newline 
\hspace*{6pt}{weights}="{0.5 0.5}"/>}\end{shaded}\egroup 


    \item[{Content model}]
  \fbox{\ttfamily <content>\newline
</content>\newline
    } 
    \item[{Schema Declaration}]
  \mbox{}\hfill\\[-10pt]\begin{Verbatim}[fontsize=\small]
element alt
{
   att.global.attributes,
   att.pointing.attribute.targetLang,
   att.pointing.attribute.evaluate,
   att.typed.attributes,
   attribute target { list { * } }?,
   attribute mode { "excl" | "incl" }?,
   attribute weights { list { * } }?,
   empty
}
\end{Verbatim}

\end{reflist}  \index{altGrp=<altGrp>|oddindex}\index{mode=@mode!<altGrp>|oddindex}
\begin{reflist}
\item[]\begin{specHead}{TEI.altGrp}{<altGrp> }(alternation group) groups a collection of <alt> elements and possibly pointers. [\xref{http://www.tei-c.org/release/doc/tei-p5-doc/en/html/SA.html\#SAAT}{16.8. Alternation}]\end{specHead} 
    \item[{Module}]
  linking
    \item[{Attributes}]
  Attributes att.global (\textit{@xml:id}, \textit{@n}, \textit{@xml:lang}, \textit{@xml:base}, \textit{@xml:space})  (att.global.rendition (\textit{@rend}, \textit{@style}, \textit{@rendition})) (att.global.linking (\textit{@corresp}, \textit{@synch}, \textit{@sameAs}, \textit{@copyOf}, \textit{@next}, \textit{@prev}, \textit{@exclude}, \textit{@select})) (att.global.analytic (\textit{@ana})) (att.global.facs (\textit{@facs})) (att.global.change (\textit{@change})) (att.global.responsibility (\textit{@cert}, \textit{@resp})) (att.global.source (\textit{@source})) att.pointing.group (\textit{@domains}, \textit{@targFunc})  (att.pointing (\textit{@targetLang}, \textit{@target}, \textit{@evaluate})) (att.typed (\textit{@type}, \textit{@subtype})) \hfil\\[-10pt]\begin{sansreflist}
    \item[@mode]
  states whether the alternations gathered in this collection are exclusive or inclusive.
\begin{reflist}
    \item[{Status}]
  Optional
    \item[{Datatype}]
  teidata.enumerated
    \item[{Legal values are:}]
  \begin{description}

\item[{excl}](exclusive) indicates that the alternation is exclusive, i.e. that at most one of the alternatives occurs.{[Default] }
\item[{incl}](inclusive) indicates that the alternation is not exclusive, i.e. that one or more of the alternatives occur.
\end{description} 
\end{reflist}  
\end{sansreflist}  
    \item[{Member of}]
  model.global.meta
    \item[{Contained by}]
  
    \item[analysis: ]
   cl m phr s span w\par 
    \item[core: ]
   abbr add addrLine address author bibl biblScope cit citedRange corr date del distinct editor email emph expan foreign gloss head headItem headLabel hi imprint item l label lg list measure mentioned name note num orig p pubPlace publisher q quote ref reg resp rs said series sic soCalled sp speaker stage street term textLang time title unclear\par 
    \item[figures: ]
   cell figure table\par 
    \item[header: ]
   authority change classCode distributor edition extent funder geoDecl handNote language licence principal scriptNote sponsor typeNote\par 
    \item[linking: ]
   ab seg\par 
    \item[msdescription: ]
   accMat acquisition additions catchwords collation colophon condition custEvent decoNote explicit filiation finalRubric foliation heraldry incipit layout material msItem musicNotation objectType origDate origPlace origin provenance rubric secFol signatures source stamp summary support surrogates watermark\par 
    \item[namesdates: ]
   addName affiliation age birth bloc country death district education faith floruit forename genName geogFeat geogName langKnown nameLink nationality occupation offset orgName persName person personGrp placeName region residence roleName settlement sex socecStatus surname\par 
    \item[textcrit: ]
   lem rdg wit witDetail\par 
    \item[textstructure: ]
   argument back body byline closer dateline div docAuthor docDate docEdition docImprint docTitle epigraph floatingText front group imprimatur opener postscript salute signed text titlePage titlePart trailer\par 
    \item[transcr: ]
   damage fw line metamark mod restore retrace secl sourceDoc supplied surface surfaceGrp surplus zone
    \item[{May contain}]
  
    \item[core: ]
   ptr\par 
    \item[linking: ]
   alt
    \item[{Note}]
  \par
Any number of alternations, pointers or extended pointers.
    \item[{Example}]
  \leavevmode\bgroup\exampleFont \begin{shaded}\noindent\mbox{}{<\textbf{altGrp}\hspace*{6pt}{mode}="{excl}">}\mbox{}\newline 
\hspace*{6pt}{<\textbf{alt}\hspace*{6pt}{target}="{\#dm \#lt \#bb}"\mbox{}\newline 
\hspace*{6pt}\hspace*{6pt}{weights}="{0.5 0.25 0.25}"/>}\mbox{}\newline 
\hspace*{6pt}{<\textbf{alt}\hspace*{6pt}{target}="{\#rl \#db}"\hspace*{6pt}{weights}="{0.5 0.5}"/>}\mbox{}\newline 
{</\textbf{altGrp}>}\end{shaded}\egroup 


    \item[{Example}]
  \leavevmode\bgroup\exampleFont \begin{shaded}\noindent\mbox{}{<\textbf{altGrp}\hspace*{6pt}{mode}="{incl}">}\mbox{}\newline 
\hspace*{6pt}{<\textbf{alt}\hspace*{6pt}{target}="{\#dm \#rl}"\hspace*{6pt}{weights}="{0.90 0.90}"/>}\mbox{}\newline 
\hspace*{6pt}{<\textbf{alt}\hspace*{6pt}{target}="{\#lt \#rl}"\hspace*{6pt}{weights}="{0.5 0.5}"/>}\mbox{}\newline 
\hspace*{6pt}{<\textbf{alt}\hspace*{6pt}{target}="{\#bb \#rl}"\hspace*{6pt}{weights}="{0.5 0.5}"/>}\mbox{}\newline 
\hspace*{6pt}{<\textbf{alt}\hspace*{6pt}{target}="{\#dm \#db}"\hspace*{6pt}{weights}="{0.10 0.10}"/>}\mbox{}\newline 
\hspace*{6pt}{<\textbf{alt}\hspace*{6pt}{target}="{\#lt \#db}"\hspace*{6pt}{weights}="{0.45 0.90}"/>}\mbox{}\newline 
\hspace*{6pt}{<\textbf{alt}\hspace*{6pt}{target}="{\#bb \#db}"\hspace*{6pt}{weights}="{0.45 0.90}"/>}\mbox{}\newline 
{</\textbf{altGrp}>}\end{shaded}\egroup 


    \item[{Content model}]
  \mbox{}\hfill\\[-10pt]\begin{Verbatim}[fontsize=\small]
<content>
 <alternate maxOccurs="unbounded"
  minOccurs="0">
  <elementRef key="alt"/>
  <elementRef key="ptr"/>
 </alternate>
</content>
    
\end{Verbatim}

    \item[{Schema Declaration}]
  \mbox{}\hfill\\[-10pt]\begin{Verbatim}[fontsize=\small]
element altGrp
{
   att.global.attributes,
   att.pointing.group.attributes,
   attribute mode { "excl" | "incl" }?,
   ( alt | ptr )*
}
\end{Verbatim}

\end{reflist}  \index{altIdentifier=<altIdentifier>|oddindex}
\begin{reflist}
\item[]\begin{specHead}{TEI.altIdentifier}{<altIdentifier> }(alternative identifier) contains an alternative or former structured identifier used for a manuscript, such as a former catalogue number. [\xref{http://www.tei-c.org/release/doc/tei-p5-doc/en/html/MS.html\#msid}{10.4. The Manuscript Identifier}]\end{specHead} 
    \item[{Module}]
  msdescription
    \item[{Attributes}]
  Attributes att.global (\textit{@xml:id}, \textit{@n}, \textit{@xml:lang}, \textit{@xml:base}, \textit{@xml:space})  (att.global.rendition (\textit{@rend}, \textit{@style}, \textit{@rendition})) (att.global.linking (\textit{@corresp}, \textit{@synch}, \textit{@sameAs}, \textit{@copyOf}, \textit{@next}, \textit{@prev}, \textit{@exclude}, \textit{@select})) (att.global.analytic (\textit{@ana})) (att.global.facs (\textit{@facs})) (att.global.change (\textit{@change})) (att.global.responsibility (\textit{@cert}, \textit{@resp})) (att.global.source (\textit{@source})) att.typed (\textit{@type}, \textit{@subtype}) 
    \item[{Contained by}]
  
    \item[msdescription: ]
   msFrag msIdentifier
    \item[{May contain}]
  
    \item[core: ]
   note\par 
    \item[header: ]
   idno\par 
    \item[msdescription: ]
   collection institution repository\par 
    \item[namesdates: ]
   bloc country district geogName placeName region settlement
    \item[{Note}]
  \par
An identifying number of some kind must be supplied if known; if it is not known, this should be stated. 
    \item[{Example}]
  \leavevmode\bgroup\exampleFont \begin{shaded}\noindent\mbox{}{<\textbf{altIdentifier}>}\mbox{}\newline 
\hspace*{6pt}{<\textbf{settlement}>}San Marino{</\textbf{settlement}>}\mbox{}\newline 
\hspace*{6pt}{<\textbf{repository}>}Huntington Library{</\textbf{repository}>}\mbox{}\newline 
\hspace*{6pt}{<\textbf{idno}>}MS.El.26.C.9{</\textbf{idno}>}\mbox{}\newline 
{</\textbf{altIdentifier}>}\end{shaded}\egroup 


    \item[{Content model}]
  \mbox{}\hfill\\[-10pt]\begin{Verbatim}[fontsize=\small]
<content>
 <sequence>
  <classRef expand="sequenceOptional"
   key="model.placeNamePart"/>
  <elementRef key="institution"
   minOccurs="0"/>
  <elementRef key="repository"
   minOccurs="0"/>
  <elementRef key="collection"
   minOccurs="0"/>
  <elementRef key="idno"/>
  <elementRef key="note" minOccurs="0"/>
 </sequence>
</content>
    
\end{Verbatim}

    \item[{Schema Declaration}]
  \mbox{}\hfill\\[-10pt]\begin{Verbatim}[fontsize=\small]
element altIdentifier
{
   att.global.attributes,
   att.typed.attributes,
   (
      placeName?,
      bloc?,
      country?,
      region?,
      district?,
      settlement?,
      geogName?,
      institution?,
      repository?,
      collection?,
      idno,
      note?
   )
}
\end{Verbatim}

\end{reflist}  \index{am=<am>|oddindex}
\begin{reflist}
\item[]\begin{specHead}{TEI.am}{<am> }(abbreviation marker) contains a sequence of letters or signs present in an abbreviation which are omitted or replaced in the expanded form of the abbreviation. [\xref{http://www.tei-c.org/release/doc/tei-p5-doc/en/html/PH.html\#PHAB}{11.3.1.2. Abbreviation and Expansion}]\end{specHead} 
    \item[{Module}]
  transcr
    \item[{Attributes}]
  Attributes att.global (\textit{@xml:id}, \textit{@n}, \textit{@xml:lang}, \textit{@xml:base}, \textit{@xml:space})  (att.global.rendition (\textit{@rend}, \textit{@style}, \textit{@rendition})) (att.global.linking (\textit{@corresp}, \textit{@synch}, \textit{@sameAs}, \textit{@copyOf}, \textit{@next}, \textit{@prev}, \textit{@exclude}, \textit{@select})) (att.global.analytic (\textit{@ana})) (att.global.facs (\textit{@facs})) (att.global.change (\textit{@change})) (att.global.responsibility (\textit{@cert}, \textit{@resp})) (att.global.source (\textit{@source})) att.typed (\textit{@type}, \textit{@subtype}) att.editLike (\textit{@evidence}, \textit{@instant})  (att.dimensions (\textit{@unit}, \textit{@quantity}, \textit{@extent}, \textit{@precision}, \textit{@scope}) (att.ranging (\textit{@atLeast}, \textit{@atMost}, \textit{@min}, \textit{@max}, \textit{@confidence})) )
    \item[{Member of}]
  model.choicePart model.pPart.editorial
    \item[{Contained by}]
  
    \item[analysis: ]
   cl pc phr s span w\par 
    \item[core: ]
   abbr add addrLine author bibl biblScope choice citedRange corr date del desc distinct editor email emph expan foreign gloss head headItem headLabel hi item l label measure meeting mentioned name note num orig p pubPlace publisher q quote ref reg resp rs said sic soCalled speaker stage street term textLang time title unclear\par 
    \item[figures: ]
   cell figDesc\par 
    \item[header: ]
   authority catDesc change classCode creation distributor edition extent funder geoDecl handNote language licence principal rendition scriptNote sponsor tagUsage typeNote\par 
    \item[linking: ]
   ab seg\par 
    \item[msdescription: ]
   accMat acquisition additions catchwords collation colophon condition custEvent decoNote explicit filiation finalRubric foliation heraldry incipit layout material musicNotation objectType origDate origPlace origin provenance rubric secFol signatures source stamp summary support surrogates watermark\par 
    \item[namesdates: ]
   addName affiliation age birth bloc country death district education faith floruit forename genName geogFeat geogName langKnown nameLink nationality occupation offset orgName persName placeName region residence roleName settlement sex socecStatus surname\par 
    \item[textcrit: ]
   lem rdg wit witDetail witness\par 
    \item[textstructure: ]
   byline closer dateline docAuthor docDate docEdition docImprint imprimatur opener salute signed titlePart trailer\par 
    \item[transcr: ]
   damage fw metamark mod restore retrace secl supplied surplus
    \item[{May contain}]
  
    \item[core: ]
   add corr del orig reg sic unclear\par 
    \item[gaiji: ]
   g\par 
    \item[transcr: ]
   damage handShift mod redo restore retrace secl supplied surplus undo\par character data
    \item[{Example}]
  \leavevmode\bgroup\exampleFont \begin{shaded}\noindent\mbox{} do you {<\textbf{abbr}>}Mr{<\textbf{am}>}.{</\textbf{am}>}\mbox{}\newline 
{</\textbf{abbr}>} Jones?\mbox{}\newline 
\end{shaded}\egroup 


    \item[{Example}]
  \leavevmode\bgroup\exampleFont \begin{shaded}\noindent\mbox{}{<\textbf{choice}>}\mbox{}\newline 
\hspace*{6pt}{<\textbf{abbr}>}Aug{<\textbf{am}>}g{</\textbf{am}>}\mbox{}\newline 
\hspace*{6pt}{</\textbf{abbr}>}\mbox{}\newline 
\hspace*{6pt}{<\textbf{expan}>}Aug{<\textbf{ex}>}ustorum duo{</\textbf{ex}>}\mbox{}\newline 
\hspace*{6pt}{</\textbf{expan}>}\mbox{}\newline 
{</\textbf{choice}>}\end{shaded}\egroup 


    \item[{Example}]
  \leavevmode\bgroup\exampleFont \begin{shaded}\noindent\mbox{}{<\textbf{abbr}>}eu{<\textbf{am}>}\mbox{}\newline 
\hspace*{6pt}\hspace*{6pt}{<\textbf{g}\hspace*{6pt}{ref}="{\#b-er}"/>}\mbox{}\newline 
\hspace*{6pt}{</\textbf{am}>}y{</\textbf{abbr}>}\mbox{}\newline 
{<\textbf{abbr}>}\mbox{}\newline 
\hspace*{6pt}{<\textbf{am}>}\mbox{}\newline 
\hspace*{6pt}\hspace*{6pt}{<\textbf{g}\hspace*{6pt}{ref}="{\#b-per}"/>}\mbox{}\newline 
\hspace*{6pt}{</\textbf{am}>}sone\mbox{}\newline 
\mbox{}\newline 
{</\textbf{abbr}>} ...\mbox{}\newline 
\end{shaded}\egroup 


    \item[{Content model}]
  \mbox{}\hfill\\[-10pt]\begin{Verbatim}[fontsize=\small]
<content>
 <alternate maxOccurs="unbounded"
  minOccurs="0">
  <textNode/>
  <classRef key="model.gLike"/>
  <classRef key="model.pPart.transcriptional"/>
 </alternate>
</content>
    
\end{Verbatim}

    \item[{Schema Declaration}]
  \mbox{}\hfill\\[-10pt]\begin{Verbatim}[fontsize=\small]
element am
{
   att.global.attributes,
   att.typed.attributes,
   att.editLike.attributes,
   ( text | model.gLike | model.pPart.transcriptional )*
}
\end{Verbatim}

\end{reflist}  \index{analytic=<analytic>|oddindex}
\begin{reflist}
\item[]\begin{specHead}{TEI.analytic}{<analytic> }(analytic level) contains bibliographic elements describing an item (e.g. an article or poem) published within a monograph or journal and not as an independent publication. [\xref{http://www.tei-c.org/release/doc/tei-p5-doc/en/html/CO.html\#COBICOL}{3.11.2.1. Analytic, Monographic, and Series Levels}]\end{specHead} 
    \item[{Module}]
  core
    \item[{Attributes}]
  Attributes att.global (\textit{@xml:id}, \textit{@n}, \textit{@xml:lang}, \textit{@xml:base}, \textit{@xml:space})  (att.global.rendition (\textit{@rend}, \textit{@style}, \textit{@rendition})) (att.global.linking (\textit{@corresp}, \textit{@synch}, \textit{@sameAs}, \textit{@copyOf}, \textit{@next}, \textit{@prev}, \textit{@exclude}, \textit{@select})) (att.global.analytic (\textit{@ana})) (att.global.facs (\textit{@facs})) (att.global.change (\textit{@change})) (att.global.responsibility (\textit{@cert}, \textit{@resp})) (att.global.source (\textit{@source}))
    \item[{Contained by}]
  
    \item[core: ]
   biblStruct
    \item[{May contain}]
  
    \item[core: ]
   author date editor ptr ref respStmt textLang title\par 
    \item[header: ]
   availability idno
    \item[{Note}]
  \par
May contain titles and statements of responsibility (author, editor, or other), in any order.\par
The <analytic> element may only occur within a <biblStruct>, where its use is mandatory for the description of an analytic level bibliographic item.
    \item[{Example}]
  \leavevmode\bgroup\exampleFont \begin{shaded}\noindent\mbox{}{<\textbf{biblStruct}>}\mbox{}\newline 
\hspace*{6pt}{<\textbf{analytic}>}\mbox{}\newline 
\hspace*{6pt}\hspace*{6pt}{<\textbf{author}>}Chesnutt, David{</\textbf{author}>}\mbox{}\newline 
\hspace*{6pt}\hspace*{6pt}{<\textbf{title}>}Historical Editions in the States{</\textbf{title}>}\mbox{}\newline 
\hspace*{6pt}{</\textbf{analytic}>}\mbox{}\newline 
\hspace*{6pt}{<\textbf{monogr}>}\mbox{}\newline 
\hspace*{6pt}\hspace*{6pt}{<\textbf{title}\hspace*{6pt}{level}="{j}">}Computers and the Humanities{</\textbf{title}>}\mbox{}\newline 
\hspace*{6pt}\hspace*{6pt}{<\textbf{imprint}>}\mbox{}\newline 
\hspace*{6pt}\hspace*{6pt}\hspace*{6pt}{<\textbf{date}\hspace*{6pt}{when}="{1991-12}">}(December, 1991):{</\textbf{date}>}\mbox{}\newline 
\hspace*{6pt}\hspace*{6pt}{</\textbf{imprint}>}\mbox{}\newline 
\hspace*{6pt}\hspace*{6pt}{<\textbf{biblScope}>}25.6{</\textbf{biblScope}>}\mbox{}\newline 
\hspace*{6pt}\hspace*{6pt}{<\textbf{biblScope}>}377–380{</\textbf{biblScope}>}\mbox{}\newline 
\hspace*{6pt}{</\textbf{monogr}>}\mbox{}\newline 
{</\textbf{biblStruct}>}\end{shaded}\egroup 


    \item[{Content model}]
  \mbox{}\hfill\\[-10pt]\begin{Verbatim}[fontsize=\small]
<content>
 <alternate maxOccurs="unbounded"
  minOccurs="0">
  <elementRef key="author"/>
  <elementRef key="editor"/>
  <elementRef key="respStmt"/>
  <elementRef key="title"/>
  <classRef key="model.ptrLike"/>
  <elementRef key="date"/>
  <elementRef key="textLang"/>
  <elementRef key="idno"/>
  <elementRef key="availability"/>
 </alternate>
</content>
    
\end{Verbatim}

    \item[{Schema Declaration}]
  \mbox{}\hfill\\[-10pt]\begin{Verbatim}[fontsize=\small]
element analytic
{
   att.global.attributes,
   (
      author    | editor    | respStmt    | title    | model.ptrLike    | date    | textLang    | idno    | availability   )*
}
\end{Verbatim}

\end{reflist}  \index{anchor=<anchor>|oddindex}
\begin{reflist}
\item[]\begin{specHead}{TEI.anchor}{<anchor> }(anchor point) attaches an identifier to a point within a text, whether or not it corresponds with a textual element. [\xref{http://www.tei-c.org/release/doc/tei-p5-doc/en/html/TS.html\#TSSAPA}{8.4.2. Synchronization and Overlap} \xref{http://www.tei-c.org/release/doc/tei-p5-doc/en/html/SA.html\#SACS}{16.5. Correspondence and Alignment}]\end{specHead} 
    \item[{Module}]
  linking
    \item[{Attributes}]
  Attributes att.global (\textit{@xml:id}, \textit{@n}, \textit{@xml:lang}, \textit{@xml:base}, \textit{@xml:space})  (att.global.rendition (\textit{@rend}, \textit{@style}, \textit{@rendition})) (att.global.linking (\textit{@corresp}, \textit{@synch}, \textit{@sameAs}, \textit{@copyOf}, \textit{@next}, \textit{@prev}, \textit{@exclude}, \textit{@select})) (att.global.analytic (\textit{@ana})) (att.global.facs (\textit{@facs})) (att.global.change (\textit{@change})) (att.global.responsibility (\textit{@cert}, \textit{@resp})) (att.global.source (\textit{@source})) att.typed (\textit{@type}, \textit{@subtype}) 
    \item[{Member of}]
  model.milestoneLike
    \item[{Contained by}]
  
    \item[analysis: ]
   cl m phr s span w\par 
    \item[core: ]
   abbr add addrLine address author bibl biblScope cit citedRange corr date del distinct editor email emph expan foreign gloss head headItem headLabel hi imprint item l label lg list listBibl measure mentioned name note num orig p pubPlace publisher q quote ref reg resp rs said series sic soCalled sp speaker stage street term textLang time title unclear\par 
    \item[figures: ]
   cell figure table\par 
    \item[header: ]
   authority change classCode distributor edition extent funder geoDecl handNote language licence principal scriptNote sponsor typeNote\par 
    \item[linking: ]
   ab seg\par 
    \item[msdescription: ]
   accMat acquisition additions catchwords collation colophon condition custEvent decoNote explicit filiation finalRubric foliation heraldry incipit layout material msItem musicNotation objectType origDate origPlace origin provenance rubric secFol signatures source stamp summary support surrogates watermark\par 
    \item[namesdates: ]
   addName affiliation age birth bloc country death district education faith floruit forename genName geogFeat geogName langKnown nameLink nationality occupation offset org orgName persName person personGrp placeName region residence roleName settlement sex socecStatus surname\par 
    \item[textcrit: ]
   lem rdg wit witDetail\par 
    \item[textstructure: ]
   argument back body byline closer dateline div docAuthor docDate docEdition docImprint docTitle epigraph floatingText front group imprimatur opener postscript salute signed text titlePage titlePart trailer\par 
    \item[transcr: ]
   damage fw line metamark mod restore retrace secl sourceDoc subst supplied surface surfaceGrp surplus zone
    \item[{May contain}]
  Empty element
    \item[{Note}]
  \par
On this element, the global {\itshape xml:id} attribute must be supplied to specify an identifier for the point at which this element occurs within a document. The value used may be chosen freely provided that it is unique within the document and is a syntactically valid name. There is no requirement for values containing numbers to be in sequence.
    \item[{Example}]
  \leavevmode\bgroup\exampleFont \begin{shaded}\noindent\mbox{}{<\textbf{s}>}The anchor is he{<\textbf{anchor}\hspace*{6pt}{xml:id}="{A234}"/>}re somewhere.{</\textbf{s}>}\mbox{}\newline 
{<\textbf{s}>}Help me find it.{<\textbf{ptr}\hspace*{6pt}{target}="{\#A234}"/>}\mbox{}\newline 
{</\textbf{s}>}\end{shaded}\egroup 


    \item[{Content model}]
  \fbox{\ttfamily <content>\newline
</content>\newline
    } 
    \item[{Schema Declaration}]
  \mbox{}\hfill\\[-10pt]\begin{Verbatim}[fontsize=\small]
element anchor { att.global.attributes, att.typed.attributes, empty }
\end{Verbatim}

\end{reflist}  \index{app=<app>|oddindex}\index{type=@type!<app>|oddindex}\index{from=@from!<app>|oddindex}\index{to=@to!<app>|oddindex}\index{loc=@loc!<app>|oddindex}
\begin{reflist}
\item[]\begin{specHead}{TEI.app}{<app> }(apparatus entry) contains one entry in a critical apparatus, with an optional lemma and usually one or more readings or notes on the relevant passage. [\xref{http://www.tei-c.org/release/doc/tei-p5-doc/en/html/TC.html\#TCAPEN}{12.1.1. The Apparatus Entry}]\end{specHead} 
    \item[{Module}]
  textcrit
    \item[{Attributes}]
  Attributes att.global (\textit{@xml:id}, \textit{@n}, \textit{@xml:lang}, \textit{@xml:base}, \textit{@xml:space})  (att.global.rendition (\textit{@rend}, \textit{@style}, \textit{@rendition})) (att.global.linking (\textit{@corresp}, \textit{@synch}, \textit{@sameAs}, \textit{@copyOf}, \textit{@next}, \textit{@prev}, \textit{@exclude}, \textit{@select})) (att.global.analytic (\textit{@ana})) (att.global.facs (\textit{@facs})) (att.global.change (\textit{@change})) (att.global.responsibility (\textit{@cert}, \textit{@resp})) (att.global.source (\textit{@source})) \hfil\\[-10pt]\begin{sansreflist}
    \item[@type]
  classifies the variation contained in this element according to some convenient typology.
\begin{reflist}
    \item[{Status}]
  Optional
    \item[{Datatype}]
  teidata.enumerated
\end{reflist}  
    \item[@from]
  identifies the beginning of the lemma in the base text.
\begin{reflist}
    \item[{Status}]
  Optional
    \item[{Datatype}]
  teidata.pointer
    \item[{Note}]
  \par
This attribute should be used when either the double-end point method of apparatus markup, or the location-referenced method with a URL rather than canonical reference, are used.
\end{reflist}  
    \item[@to]
  identifies the endpoint of the lemma in the base text.
\begin{reflist}
    \item[{Status}]
  Optional
    \item[{Datatype}]
  teidata.pointer
    \item[{Note}]
  \par
This attribute is only used when the double-end point method of apparatus markup is used, when the encoded apparatus is not embedded \textit{in-line} in the base-text.
\end{reflist}  
    \item[@loc]
  (location) indicates the location of the variation, when the location-referenced method of apparatus markup is used.
\begin{reflist}
    \item[{Status}]
  Optional
    \item[{Datatype}]
  1–∞ occurrences of teidata.word separated by whitespace
    \item[{Note}]
  \par
This attribute is used only when the location-referenced encoding method is used. It supplies a string containing a canonical reference for the passage to which the variation applies.
\end{reflist}  
\end{sansreflist}  
    \item[{Member of}]
  model.global.edit
    \item[{Contained by}]
  
    \item[analysis: ]
   cl m phr s span w\par 
    \item[core: ]
   abbr add addrLine address author bibl biblScope cit citedRange corr date del distinct editor email emph expan foreign gloss head headItem headLabel hi imprint item l label lg list measure mentioned name note num orig p pubPlace publisher q quote ref reg resp rs said series sic soCalled sp speaker stage street term textLang time title unclear\par 
    \item[figures: ]
   cell figure table\par 
    \item[header: ]
   authority change classCode distributor edition extent funder geoDecl handNote language licence principal scriptNote sponsor typeNote\par 
    \item[linking: ]
   ab seg\par 
    \item[msdescription: ]
   accMat acquisition additions catchwords collation colophon condition custEvent decoNote explicit filiation finalRubric foliation heraldry incipit layout material msItem musicNotation objectType origDate origPlace origin provenance rubric secFol signatures source stamp summary support surrogates watermark\par 
    \item[namesdates: ]
   addName affiliation age birth bloc country death district education faith floruit forename genName geogFeat geogName langKnown nameLink nationality occupation offset orgName persName person personGrp placeName region residence roleName settlement sex socecStatus surname\par 
    \item[textcrit: ]
   lem listApp rdg wit witDetail\par 
    \item[textstructure: ]
   argument back body byline closer dateline div docAuthor docDate docEdition docImprint docTitle epigraph floatingText front group imprimatur opener postscript salute signed text titlePage titlePart trailer\par 
    \item[transcr: ]
   damage fw line metamark mod restore retrace secl sourceDoc supplied surface surfaceGrp surplus zone
    \item[{May contain}]
  
    \item[core: ]
   note\par 
    \item[textcrit: ]
   lem rdg rdgGrp wit witDetail
    \item[{Example}]
  \leavevmode\bgroup\exampleFont \begin{shaded}\noindent\mbox{}{<\textbf{app}>}\mbox{}\newline 
\hspace*{6pt}{<\textbf{lem}\hspace*{6pt}{wit}="{\#El \#Hg}">}Experience{</\textbf{lem}>}\mbox{}\newline 
\hspace*{6pt}{<\textbf{rdg}\hspace*{6pt}{type}="{substantive}"\hspace*{6pt}{wit}="{\#La}">}Experiment{</\textbf{rdg}>}\mbox{}\newline 
\hspace*{6pt}{<\textbf{rdg}\hspace*{6pt}{type}="{substantive}"\hspace*{6pt}{wit}="{\#Ra2}">}Eryment{</\textbf{rdg}>}\mbox{}\newline 
{</\textbf{app}>}\end{shaded}\egroup 


    \item[{Example}]
  \leavevmode\bgroup\exampleFont \begin{shaded}\noindent\mbox{}{<\textbf{app}\hspace*{6pt}{type}="{substantive}">}\mbox{}\newline 
\hspace*{6pt}{<\textbf{rdgGrp}\hspace*{6pt}{type}="{subvariants}">}\mbox{}\newline 
\hspace*{6pt}\hspace*{6pt}{<\textbf{lem}\hspace*{6pt}{wit}="{\#El \#Hg}">}Experience{</\textbf{lem}>}\mbox{}\newline 
\hspace*{6pt}\hspace*{6pt}{<\textbf{rdg}\hspace*{6pt}{wit}="{\#Ha4}">}Experiens{</\textbf{rdg}>}\mbox{}\newline 
\hspace*{6pt}{</\textbf{rdgGrp}>}\mbox{}\newline 
\hspace*{6pt}{<\textbf{rdgGrp}\hspace*{6pt}{type}="{subvariants}">}\mbox{}\newline 
\hspace*{6pt}\hspace*{6pt}{<\textbf{lem}\hspace*{6pt}{wit}="{\#Cp \#Ld1}">}Experiment{</\textbf{lem}>}\mbox{}\newline 
\hspace*{6pt}\hspace*{6pt}{<\textbf{rdg}\hspace*{6pt}{wit}="{\#La}">}Ex{<\textbf{g}\hspace*{6pt}{ref}="{\#per}"/>}iment{</\textbf{rdg}>}\mbox{}\newline 
\hspace*{6pt}{</\textbf{rdgGrp}>}\mbox{}\newline 
\hspace*{6pt}{<\textbf{rdgGrp}\hspace*{6pt}{type}="{subvariants}">}\mbox{}\newline 
\hspace*{6pt}\hspace*{6pt}{<\textbf{lem}\hspace*{6pt}{resp}="{\#ed2013}">}Eriment{</\textbf{lem}>}\mbox{}\newline 
\hspace*{6pt}\hspace*{6pt}{<\textbf{rdg}\hspace*{6pt}{wit}="{\#Ra2}">}Eryment{</\textbf{rdg}>}\mbox{}\newline 
\hspace*{6pt}{</\textbf{rdgGrp}>}\mbox{}\newline 
{</\textbf{app}>}\end{shaded}\egroup 


    \item[{Example}]
  \leavevmode\bgroup\exampleFont \begin{shaded}\noindent\mbox{}{<\textbf{app}\hspace*{6pt}{loc}="{1}">}\mbox{}\newline 
\hspace*{6pt}{<\textbf{rdg}\hspace*{6pt}{resp}="{\#SEG}">}TIMΩΔA{</\textbf{rdg}>}\mbox{}\newline 
{</\textbf{app}>}\end{shaded}\egroup 


    \item[{Example}]
  \leavevmode\bgroup\exampleFont \begin{shaded}\noindent\mbox{}{<\textbf{app}\hspace*{6pt}{loc}="{1-6}">}\mbox{}\newline 
\hspace*{6pt}{<\textbf{note}>}Too badly worn to yield a text{</\textbf{note}>}\mbox{}\newline 
{</\textbf{app}>}\end{shaded}\egroup 


    \item[{Example}]
  \leavevmode\bgroup\exampleFont \begin{shaded}\noindent\mbox{}{<\textbf{choice}\hspace*{6pt}{xml:id}="{choice3}">}\mbox{}\newline 
\hspace*{6pt}{<\textbf{reg}>}σύμπαντα{</\textbf{reg}>}\mbox{}\newline 
\hspace*{6pt}{<\textbf{orig}>}ΣΙΝΠΑΤΑΝ{</\textbf{orig}>}\mbox{}\newline 
{</\textbf{choice}>}\mbox{}\newline 
\textit{<!-- ... -->}\mbox{}\newline 
{<\textbf{app}\hspace*{6pt}{from}="{\#choice3}">}\mbox{}\newline 
\hspace*{6pt}{<\textbf{note}>}Mommsen's fanciful normalization, reproduced here, has not been accepted by all recent editions{</\textbf{note}>}\mbox{}\newline 
{</\textbf{app}>}\end{shaded}\egroup 


    \item[{Content model}]
  \mbox{}\hfill\\[-10pt]\begin{Verbatim}[fontsize=\small]
<content>
 <sequence>
  <elementRef key="lem" minOccurs="0"/>
  <alternate maxOccurs="unbounded"
   minOccurs="0">
   <classRef key="model.rdgLike"/>
   <classRef key="model.noteLike"/>
   <elementRef key="wit"/>
   <elementRef key="rdgGrp"/>
  </alternate>
 </sequence>
</content>
    
\end{Verbatim}

    \item[{Schema Declaration}]
  \mbox{}\hfill\\[-10pt]\begin{Verbatim}[fontsize=\small]
element app
{
   att.global.attributes,
   attribute type { text }?,
   attribute from { text }?,
   attribute to { text }?,
   attribute loc { list { + } }?,
   ( lem?, ( model.rdgLike | model.noteLike | wit | rdgGrp )* )
}
\end{Verbatim}

\end{reflist}  \index{appInfo=<appInfo>|oddindex}
\begin{reflist}
\item[]\begin{specHead}{TEI.appInfo}{<appInfo> }(application information) records information about an application which has edited the TEI file. [\xref{http://www.tei-c.org/release/doc/tei-p5-doc/en/html/HD.html\#HDAPP}{2.3.10. The Application Information Element}]\end{specHead} 
    \item[{Module}]
  header
    \item[{Attributes}]
  Attributes att.global (\textit{@xml:id}, \textit{@n}, \textit{@xml:lang}, \textit{@xml:base}, \textit{@xml:space})  (att.global.rendition (\textit{@rend}, \textit{@style}, \textit{@rendition})) (att.global.linking (\textit{@corresp}, \textit{@synch}, \textit{@sameAs}, \textit{@copyOf}, \textit{@next}, \textit{@prev}, \textit{@exclude}, \textit{@select})) (att.global.analytic (\textit{@ana})) (att.global.facs (\textit{@facs})) (att.global.change (\textit{@change})) (att.global.responsibility (\textit{@cert}, \textit{@resp})) (att.global.source (\textit{@source}))
    \item[{Member of}]
  model.encodingDescPart
    \item[{Contained by}]
  
    \item[header: ]
   encodingDesc
    \item[{May contain}]
  
    \item[header: ]
   application
    \item[{Example}]
  \leavevmode\bgroup\exampleFont \begin{shaded}\noindent\mbox{}{<\textbf{appInfo}>}\mbox{}\newline 
\hspace*{6pt}{<\textbf{application}\hspace*{6pt}{ident}="{Xaira}"\hspace*{6pt}{version}="{1.24}">}\mbox{}\newline 
\hspace*{6pt}\hspace*{6pt}{<\textbf{label}>}XAIRA Indexer{</\textbf{label}>}\mbox{}\newline 
\hspace*{6pt}\hspace*{6pt}{<\textbf{ptr}\hspace*{6pt}{target}="{\#P1}"/>}\mbox{}\newline 
\hspace*{6pt}{</\textbf{application}>}\mbox{}\newline 
{</\textbf{appInfo}>}\end{shaded}\egroup 


    \item[{Content model}]
  \mbox{}\hfill\\[-10pt]\begin{Verbatim}[fontsize=\small]
<content>
 <classRef key="model.applicationLike"
  maxOccurs="unbounded" minOccurs="1"/>
</content>
    
\end{Verbatim}

    \item[{Schema Declaration}]
  \mbox{}\hfill\\[-10pt]\begin{Verbatim}[fontsize=\small]
element appInfo { att.global.attributes, model.applicationLike+ }
\end{Verbatim}

\end{reflist}  \index{application=<application>|oddindex}\index{ident=@ident!<application>|oddindex}\index{version=@version!<application>|oddindex}
\begin{reflist}
\item[]\begin{specHead}{TEI.application}{<application> }provides information about an application which has acted upon the document. [\xref{http://www.tei-c.org/release/doc/tei-p5-doc/en/html/HD.html\#HDAPP}{2.3.10. The Application Information Element}]\end{specHead} 
    \item[{Module}]
  header
    \item[{Attributes}]
  Attributes att.global (\textit{@xml:id}, \textit{@n}, \textit{@xml:lang}, \textit{@xml:base}, \textit{@xml:space})  (att.global.rendition (\textit{@rend}, \textit{@style}, \textit{@rendition})) (att.global.linking (\textit{@corresp}, \textit{@synch}, \textit{@sameAs}, \textit{@copyOf}, \textit{@next}, \textit{@prev}, \textit{@exclude}, \textit{@select})) (att.global.analytic (\textit{@ana})) (att.global.facs (\textit{@facs})) (att.global.change (\textit{@change})) (att.global.responsibility (\textit{@cert}, \textit{@resp})) (att.global.source (\textit{@source})) att.typed (\textit{@type}, \textit{@subtype}) att.datable (\textit{@calendar}, \textit{@period})  (att.datable.w3c (\textit{@when}, \textit{@notBefore}, \textit{@notAfter}, \textit{@from}, \textit{@to})) (att.datable.iso (\textit{@when-iso}, \textit{@notBefore-iso}, \textit{@notAfter-iso}, \textit{@from-iso}, \textit{@to-iso})) (att.datable.custom (\textit{@when-custom}, \textit{@notBefore-custom}, \textit{@notAfter-custom}, \textit{@from-custom}, \textit{@to-custom}, \textit{@datingPoint}, \textit{@datingMethod})) \hfil\\[-10pt]\begin{sansreflist}
    \item[@ident]
  supplies an identifier for the application, independent of its version number or display name.
\begin{reflist}
    \item[{Status}]
  Required
    \item[{Datatype}]
  teidata.name
\end{reflist}  
    \item[@version]
  supplies a version number for the application, independent of its identifier or display name.
\begin{reflist}
    \item[{Status}]
  Required
    \item[{Datatype}]
  teidata.versionNumber
\end{reflist}  
\end{sansreflist}  
    \item[{Member of}]
  model.applicationLike
    \item[{Contained by}]
  
    \item[header: ]
   appInfo
    \item[{May contain}]
  
    \item[core: ]
   desc label p ptr ref\par 
    \item[linking: ]
   ab
    \item[{Example}]
  \leavevmode\bgroup\exampleFont \begin{shaded}\noindent\mbox{}{<\textbf{appInfo}>}\mbox{}\newline 
\hspace*{6pt}{<\textbf{application}\hspace*{6pt}{ident}="{ImageMarkupTool1}"\mbox{}\newline 
\hspace*{6pt}\hspace*{6pt}{notAfter}="{2006-06-01}"\hspace*{6pt}{version}="{1.5}">}\mbox{}\newline 
\hspace*{6pt}\hspace*{6pt}{<\textbf{label}>}Image Markup Tool{</\textbf{label}>}\mbox{}\newline 
\hspace*{6pt}\hspace*{6pt}{<\textbf{ptr}\hspace*{6pt}{target}="{\#P1}"/>}\mbox{}\newline 
\hspace*{6pt}\hspace*{6pt}{<\textbf{ptr}\hspace*{6pt}{target}="{\#P2}"/>}\mbox{}\newline 
\hspace*{6pt}{</\textbf{application}>}\mbox{}\newline 
{</\textbf{appInfo}>}\end{shaded}\egroup 

This example shows an appInfo element documenting the fact that version 1.5 of the Image Markup Tool1 application has an interest in two parts of a document which was last saved on June 6 2006. The parts concerned are accessible at the URLs given as target for the two <ptr> elements.
    \item[{Content model}]
  \mbox{}\hfill\\[-10pt]\begin{Verbatim}[fontsize=\small]
<content>
 <sequence>
  <classRef key="model.labelLike"
   maxOccurs="unbounded" minOccurs="1"/>
  <alternate>
   <classRef key="model.ptrLike"
    maxOccurs="unbounded" minOccurs="0"/>
   <classRef key="model.pLike"
    maxOccurs="unbounded" minOccurs="0"/>
  </alternate>
 </sequence>
</content>
    
\end{Verbatim}

    \item[{Schema Declaration}]
  \mbox{}\hfill\\[-10pt]\begin{Verbatim}[fontsize=\small]
element application
{
   att.global.attributes,
   att.typed.attributes,
   att.datable.attributes,
   attribute ident { text },
   attribute version { text },
   ( model.labelLike+, ( model.ptrLike* | model.pLike* ) )
}
\end{Verbatim}

\end{reflist}  \index{argument=<argument>|oddindex}
\begin{reflist}
\item[]\begin{specHead}{TEI.argument}{<argument> }contains a formal list or prose description of the topics addressed by a subdivision of a text. [\xref{http://www.tei-c.org/release/doc/tei-p5-doc/en/html/DS.html\#DSDTB}{4.2. Elements Common to All Divisions} \xref{http://www.tei-c.org/release/doc/tei-p5-doc/en/html/DS.html\#DSTITL}{4.6. Title Pages}]\end{specHead} 
    \item[{Module}]
  textstructure
    \item[{Attributes}]
  Attributes att.global (\textit{@xml:id}, \textit{@n}, \textit{@xml:lang}, \textit{@xml:base}, \textit{@xml:space})  (att.global.rendition (\textit{@rend}, \textit{@style}, \textit{@rendition})) (att.global.linking (\textit{@corresp}, \textit{@synch}, \textit{@sameAs}, \textit{@copyOf}, \textit{@next}, \textit{@prev}, \textit{@exclude}, \textit{@select})) (att.global.analytic (\textit{@ana})) (att.global.facs (\textit{@facs})) (att.global.change (\textit{@change})) (att.global.responsibility (\textit{@cert}, \textit{@resp})) (att.global.source (\textit{@source}))
    \item[{Member of}]
  model.divWrapper model.pLike.front model.titlepagePart 
    \item[{Contained by}]
  
    \item[core: ]
   lg list\par 
    \item[figures: ]
   figure table\par 
    \item[msdescription: ]
   msItem\par 
    \item[textstructure: ]
   back body div front group opener titlePage
    \item[{May contain}]
  
    \item[analysis: ]
   interp interpGrp span spanGrp\par 
    \item[core: ]
   bibl biblStruct cb cit desc gap gb head index l label lb lg list listBibl milestone note p pb q quote said sp stage\par 
    \item[figures: ]
   figure notatedMusic table\par 
    \item[header: ]
   biblFull\par 
    \item[linking: ]
   ab alt altGrp anchor join joinGrp link linkGrp timeline\par 
    \item[msdescription: ]
   msDesc\par 
    \item[namesdates: ]
   listEvent listNym listOrg listPerson listPlace\par 
    \item[textcrit: ]
   app listApp listWit witDetail\par 
    \item[textstructure: ]
   floatingText\par 
    \item[transcr: ]
   addSpan damageSpan delSpan fw listTranspose metamark space substJoin
    \item[{Note}]
  \par
Often contains either a list or a paragraph
    \item[{Example}]
  \leavevmode\bgroup\exampleFont \begin{shaded}\noindent\mbox{}{<\textbf{argument}>}\mbox{}\newline 
\hspace*{6pt}{<\textbf{p}>}Monte Video — Maldonado — Excursion\mbox{}\newline 
\hspace*{6pt}\hspace*{6pt} to R Polanco — Lazo and Bolas — Partridges —\mbox{}\newline 
\hspace*{6pt}\hspace*{6pt} Absence of Trees — Deer — Capybara, or River Hog —\mbox{}\newline 
\hspace*{6pt}\hspace*{6pt} Tucutuco — Molothrus, cuckoo-like habits — Tyrant\mbox{}\newline 
\hspace*{6pt}\hspace*{6pt} Flycatcher — Mocking-bird — Carrion Hawks —\mbox{}\newline 
\hspace*{6pt}\hspace*{6pt} Tubes formed by Lightning — House struck{</\textbf{p}>}\mbox{}\newline 
{</\textbf{argument}>}\end{shaded}\egroup 


    \item[{Content model}]
  \mbox{}\hfill\\[-10pt]\begin{Verbatim}[fontsize=\small]
<content>
 <sequence>
  <alternate maxOccurs="unbounded"
   minOccurs="0">
   <classRef key="model.global"/>
   <classRef key="model.headLike"/>
  </alternate>
  <sequence maxOccurs="unbounded"
   minOccurs="1">
   <classRef key="model.common"/>
   <classRef key="model.global"
    maxOccurs="unbounded" minOccurs="0"/>
  </sequence>
 </sequence>
</content>
    
\end{Verbatim}

    \item[{Schema Declaration}]
  \mbox{}\hfill\\[-10pt]\begin{Verbatim}[fontsize=\small]
element argument
{
   att.global.attributes,
   ( ( model.global | model.headLike )*, ( model.common, model.global* )+ )
}
\end{Verbatim}

\end{reflist}  \index{author=<author>|oddindex}
\begin{reflist}
\item[]\begin{specHead}{TEI.author}{<author> }in a bibliographic reference, contains the name(s) of an author, personal or corporate, of a work; for example in the same form as that provided by a recognized bibliographic name authority. [\xref{http://www.tei-c.org/release/doc/tei-p5-doc/en/html/CO.html\#COBICOR}{3.11.2.2. Titles, Authors, and Editors} \xref{http://www.tei-c.org/release/doc/tei-p5-doc/en/html/HD.html\#HD21}{2.2.1. The Title Statement}]\end{specHead} 
    \item[{Module}]
  core
    \item[{Attributes}]
  Attributes att.global (\textit{@xml:id}, \textit{@n}, \textit{@xml:lang}, \textit{@xml:base}, \textit{@xml:space})  (att.global.rendition (\textit{@rend}, \textit{@style}, \textit{@rendition})) (att.global.linking (\textit{@corresp}, \textit{@synch}, \textit{@sameAs}, \textit{@copyOf}, \textit{@next}, \textit{@prev}, \textit{@exclude}, \textit{@select})) (att.global.analytic (\textit{@ana})) (att.global.facs (\textit{@facs})) (att.global.change (\textit{@change})) (att.global.responsibility (\textit{@cert}, \textit{@resp})) (att.global.source (\textit{@source})) att.naming (\textit{@role}, \textit{@nymRef})  (att.canonical (\textit{@key}, \textit{@ref}))
    \item[{Member of}]
  model.respLike 
    \item[{Contained by}]
  
    \item[core: ]
   analytic bibl monogr\par 
    \item[header: ]
   editionStmt titleStmt\par 
    \item[msdescription: ]
   msItem msItemStruct
    \item[{May contain}]
  
    \item[analysis: ]
   c cl interp interpGrp m pc phr s span spanGrp w\par 
    \item[core: ]
   abbr add address cb choice corr date del distinct email emph expan foreign gap gb gloss graphic hi index lb measure measureGrp media mentioned milestone name note num orig pb ptr ref reg rs sic soCalled term time title unclear\par 
    \item[figures: ]
   figure formula notatedMusic\par 
    \item[gaiji: ]
   g\par 
    \item[header: ]
   idno\par 
    \item[linking: ]
   alt altGrp anchor join joinGrp link linkGrp seg timeline\par 
    \item[msdescription: ]
   catchwords depth dim dimensions height heraldry locus locusGrp material objectType origDate origPlace secFol signatures stamp watermark width\par 
    \item[namesdates: ]
   addName affiliation bloc climate country district forename genName geo geogFeat geogName location nameLink offset orgName persName placeName population region roleName settlement state surname terrain trait\par 
    \item[textcrit: ]
   app witDetail\par 
    \item[transcr: ]
   addSpan am damage damageSpan delSpan ex fw handShift listTranspose metamark mod redo restore retrace secl space subst substJoin supplied surplus undo\par character data
    \item[{Note}]
  \par
Particularly where cataloguing is likely to be based on the content of the header, it is advisable to use a generally recognized name authority file to supply the content for this element. The attributes {\itshape key} or {\itshape ref} may also be used to reference canonical information about the author(s) intended from any appropriate authority, such as a library catalogue or online resource.\par
In the case of a broadcast, use this element for the name of the company or network responsible for making the broadcast.\par
Where an author is unknown or unspecified, this element may contain text such as \textit{Unknown} or \textit{Anonymous}. When the appropriate TEI modules are in use, it may also contain detailed tagging of the names used for people, organizations or places, in particular where multiple names are given.
    \item[{Example}]
  \leavevmode\bgroup\exampleFont \begin{shaded}\noindent\mbox{}{<\textbf{author}>}British Broadcasting Corporation{</\textbf{author}>}\mbox{}\newline 
{<\textbf{author}>}La Fayette, Marie Madeleine Pioche de la Vergne, comtesse de (1634–1693){</\textbf{author}>}\mbox{}\newline 
{<\textbf{author}>}Anonymous{</\textbf{author}>}\mbox{}\newline 
{<\textbf{author}>}Bill and Melinda Gates Foundation{</\textbf{author}>}\mbox{}\newline 
{<\textbf{author}>}\mbox{}\newline 
\hspace*{6pt}{<\textbf{persName}>}Beaumont, Francis{</\textbf{persName}>} and\mbox{}\newline 
{<\textbf{persName}>}John Fletcher{</\textbf{persName}>}\mbox{}\newline 
{</\textbf{author}>}\mbox{}\newline 
{<\textbf{author}>}\mbox{}\newline 
\hspace*{6pt}{<\textbf{orgName}\hspace*{6pt}{key}="{BBC}">}British Broadcasting\mbox{}\newline 
\hspace*{6pt}\hspace*{6pt} Corporation{</\textbf{orgName}>}: Radio 3 Network\mbox{}\newline 
{</\textbf{author}>}\end{shaded}\egroup 


    \item[{Content model}]
  \mbox{}\hfill\\[-10pt]\begin{Verbatim}[fontsize=\small]
<content>
 <macroRef key="macro.phraseSeq"/>
</content>
    
\end{Verbatim}

    \item[{Schema Declaration}]
  \mbox{}\hfill\\[-10pt]\begin{Verbatim}[fontsize=\small]
element author
{
   att.global.attributes,
   att.naming.attributes,
   macro.phraseSeq}
\end{Verbatim}

\end{reflist}  \index{authority=<authority>|oddindex}
\begin{reflist}
\item[]\begin{specHead}{TEI.authority}{<authority> }(release authority) supplies the name of a person or other agency responsible for making a work available, other than a publisher or distributor. [\xref{http://www.tei-c.org/release/doc/tei-p5-doc/en/html/HD.html\#HD24}{2.2.4. Publication, Distribution, Licensing, etc.}]\end{specHead} 
    \item[{Module}]
  header
    \item[{Attributes}]
  Attributes att.global (\textit{@xml:id}, \textit{@n}, \textit{@xml:lang}, \textit{@xml:base}, \textit{@xml:space})  (att.global.rendition (\textit{@rend}, \textit{@style}, \textit{@rendition})) (att.global.linking (\textit{@corresp}, \textit{@synch}, \textit{@sameAs}, \textit{@copyOf}, \textit{@next}, \textit{@prev}, \textit{@exclude}, \textit{@select})) (att.global.analytic (\textit{@ana})) (att.global.facs (\textit{@facs})) (att.global.change (\textit{@change})) (att.global.responsibility (\textit{@cert}, \textit{@resp})) (att.global.source (\textit{@source}))
    \item[{Member of}]
  model.publicationStmtPart.agency 
    \item[{Contained by}]
  
    \item[core: ]
   monogr\par 
    \item[header: ]
   publicationStmt
    \item[{May contain}]
  
    \item[analysis: ]
   interp interpGrp span spanGrp\par 
    \item[core: ]
   abbr address cb choice date distinct email emph expan foreign gap gb gloss hi index lb measure measureGrp mentioned milestone name note num pb ptr ref rs soCalled term time title\par 
    \item[figures: ]
   figure notatedMusic\par 
    \item[header: ]
   idno\par 
    \item[linking: ]
   alt altGrp anchor join joinGrp link linkGrp timeline\par 
    \item[msdescription: ]
   catchwords depth dim dimensions height heraldry locus locusGrp material objectType origDate origPlace secFol signatures stamp watermark width\par 
    \item[namesdates: ]
   addName affiliation bloc climate country district forename genName geo geogFeat geogName location nameLink offset orgName persName placeName population region roleName settlement state surname terrain trait\par 
    \item[textcrit: ]
   app witDetail\par 
    \item[transcr: ]
   addSpan am damageSpan delSpan ex fw listTranspose metamark space subst substJoin\par character data
    \item[{Example}]
  \leavevmode\bgroup\exampleFont \begin{shaded}\noindent\mbox{}{<\textbf{authority}>}John Smith{</\textbf{authority}>}\end{shaded}\egroup 


    \item[{Content model}]
  \mbox{}\hfill\\[-10pt]\begin{Verbatim}[fontsize=\small]
<content>
 <macroRef key="macro.phraseSeq.limited"/>
</content>
    
\end{Verbatim}

    \item[{Schema Declaration}]
  \mbox{}\hfill\\[-10pt]\begin{Verbatim}[fontsize=\small]
element authority { att.global.attributes, macro.phraseSeq.limited }
\end{Verbatim}

\end{reflist}  \index{availability=<availability>|oddindex}\index{status=@status!<availability>|oddindex}
\begin{reflist}
\item[]\begin{specHead}{TEI.availability}{<availability> }supplies information about the availability of a text, for example any restrictions on its use or distribution, its copyright status, any licence applying to it, etc. [\xref{http://www.tei-c.org/release/doc/tei-p5-doc/en/html/HD.html\#HD24}{2.2.4. Publication, Distribution, Licensing, etc.}]\end{specHead} 
    \item[{Module}]
  header
    \item[{Attributes}]
  Attributes att.global (\textit{@xml:id}, \textit{@n}, \textit{@xml:lang}, \textit{@xml:base}, \textit{@xml:space})  (att.global.rendition (\textit{@rend}, \textit{@style}, \textit{@rendition})) (att.global.linking (\textit{@corresp}, \textit{@synch}, \textit{@sameAs}, \textit{@copyOf}, \textit{@next}, \textit{@prev}, \textit{@exclude}, \textit{@select})) (att.global.analytic (\textit{@ana})) (att.global.facs (\textit{@facs})) (att.global.change (\textit{@change})) (att.global.responsibility (\textit{@cert}, \textit{@resp})) (att.global.source (\textit{@source})) att.declarable (\textit{@default}) \hfil\\[-10pt]\begin{sansreflist}
    \item[@status]
  supplies a code identifying the current availability of the text.
\begin{reflist}
    \item[{Status}]
  Optional
    \item[{Datatype}]
  teidata.enumerated
    \item[{Legal values are:}]
  \begin{description}

\item[{free}]the text is freely available.
\item[{unknown}]the status of the text is unknown.{[Default] \xref{http://www.tei-c.org/Activities/Council/Working/tcw27.xml}{Deprecated}. The value will no longer be a default after 2017-09-05.}
\item[{restricted}]the text is not freely available.
\end{description} 
\end{reflist}  
\end{sansreflist}  
    \item[{Member of}]
  model.biblPart model.publicationStmtPart.detail 
    \item[{Contained by}]
  
    \item[core: ]
   analytic bibl monogr series\par 
    \item[header: ]
   publicationStmt\par 
    \item[msdescription: ]
   adminInfo
    \item[{May contain}]
  
    \item[core: ]
   p\par 
    \item[header: ]
   licence\par 
    \item[linking: ]
   ab
    \item[{Note}]
  \par
A consistent format should be adopted
    \item[{Example}]
  \leavevmode\bgroup\exampleFont \begin{shaded}\noindent\mbox{}{<\textbf{availability}\hspace*{6pt}{status}="{restricted}">}\mbox{}\newline 
\hspace*{6pt}{<\textbf{p}>}Available for academic research purposes only.{</\textbf{p}>}\mbox{}\newline 
{</\textbf{availability}>}\mbox{}\newline 
{<\textbf{availability}\hspace*{6pt}{status}="{free}">}\mbox{}\newline 
\hspace*{6pt}{<\textbf{p}>}In the public domain{</\textbf{p}>}\mbox{}\newline 
{</\textbf{availability}>}\mbox{}\newline 
{<\textbf{availability}\hspace*{6pt}{status}="{restricted}">}\mbox{}\newline 
\hspace*{6pt}{<\textbf{p}>}Available under licence from the publishers.{</\textbf{p}>}\mbox{}\newline 
{</\textbf{availability}>}\end{shaded}\egroup 


    \item[{Example}]
  \leavevmode\bgroup\exampleFont \begin{shaded}\noindent\mbox{}{<\textbf{availability}>}\mbox{}\newline 
\hspace*{6pt}{<\textbf{licence}\hspace*{6pt}{target}="{http://opensource.org/licenses/MIT}">}\mbox{}\newline 
\hspace*{6pt}\hspace*{6pt}{<\textbf{p}>}The MIT License\mbox{}\newline 
\hspace*{6pt}\hspace*{6pt}\hspace*{6pt}\hspace*{6pt} applies to this document.{</\textbf{p}>}\mbox{}\newline 
\hspace*{6pt}\hspace*{6pt}{<\textbf{p}>}Copyright (C) 2011 by The University of Victoria{</\textbf{p}>}\mbox{}\newline 
\hspace*{6pt}\hspace*{6pt}{<\textbf{p}>}Permission is hereby granted, free of charge, to any person obtaining a copy\mbox{}\newline 
\hspace*{6pt}\hspace*{6pt}\hspace*{6pt}\hspace*{6pt} of this software and associated documentation files (the "Software"), to deal\mbox{}\newline 
\hspace*{6pt}\hspace*{6pt}\hspace*{6pt}\hspace*{6pt} in the Software without restriction, including without limitation the rights\mbox{}\newline 
\hspace*{6pt}\hspace*{6pt}\hspace*{6pt}\hspace*{6pt} to use, copy, modify, merge, publish, distribute, sublicense, and/or sell\mbox{}\newline 
\hspace*{6pt}\hspace*{6pt}\hspace*{6pt}\hspace*{6pt} copies of the Software, and to permit persons to whom the Software is\mbox{}\newline 
\hspace*{6pt}\hspace*{6pt}\hspace*{6pt}\hspace*{6pt} furnished to do so, subject to the following conditions:{</\textbf{p}>}\mbox{}\newline 
\hspace*{6pt}\hspace*{6pt}{<\textbf{p}>}The above copyright notice and this permission notice shall be included in\mbox{}\newline 
\hspace*{6pt}\hspace*{6pt}\hspace*{6pt}\hspace*{6pt} all copies or substantial portions of the Software.{</\textbf{p}>}\mbox{}\newline 
\hspace*{6pt}\hspace*{6pt}{<\textbf{p}>}THE SOFTWARE IS PROVIDED "AS IS", WITHOUT WARRANTY OF ANY KIND, EXPRESS OR\mbox{}\newline 
\hspace*{6pt}\hspace*{6pt}\hspace*{6pt}\hspace*{6pt} IMPLIED, INCLUDING BUT NOT LIMITED TO THE WARRANTIES OF MERCHANTABILITY,\mbox{}\newline 
\hspace*{6pt}\hspace*{6pt}\hspace*{6pt}\hspace*{6pt} FITNESS FOR A PARTICULAR PURPOSE AND NONINFRINGEMENT. IN NO EVENT SHALL THE\mbox{}\newline 
\hspace*{6pt}\hspace*{6pt}\hspace*{6pt}\hspace*{6pt} AUTHORS OR COPYRIGHT HOLDERS BE LIABLE FOR ANY CLAIM, DAMAGES OR OTHER\mbox{}\newline 
\hspace*{6pt}\hspace*{6pt}\hspace*{6pt}\hspace*{6pt} LIABILITY, WHETHER IN AN ACTION OF CONTRACT, TORT OR OTHERWISE, ARISING FROM,\mbox{}\newline 
\hspace*{6pt}\hspace*{6pt}\hspace*{6pt}\hspace*{6pt} OUT OF OR IN CONNECTION WITH THE SOFTWARE OR THE USE OR OTHER DEALINGS IN\mbox{}\newline 
\hspace*{6pt}\hspace*{6pt}\hspace*{6pt}\hspace*{6pt} THE SOFTWARE.{</\textbf{p}>}\mbox{}\newline 
\hspace*{6pt}{</\textbf{licence}>}\mbox{}\newline 
{</\textbf{availability}>}\end{shaded}\egroup 


    \item[{Content model}]
  \mbox{}\hfill\\[-10pt]\begin{Verbatim}[fontsize=\small]
<content>
 <alternate maxOccurs="unbounded"
  minOccurs="1">
  <classRef key="model.availabilityPart"/>
  <classRef key="model.pLike"/>
 </alternate>
</content>
    
\end{Verbatim}

    \item[{Schema Declaration}]
  \mbox{}\hfill\\[-10pt]\begin{Verbatim}[fontsize=\small]
element availability
{
   att.global.attributes,
   att.declarable.attributes,
   attribute status { "free" | "unknown" | "restricted" }?,
   ( model.availabilityPart | model.pLike )+
}
\end{Verbatim}

\end{reflist}  \index{back=<back>|oddindex}
\begin{reflist}
\item[]\begin{specHead}{TEI.back}{<back> }(back matter) contains any appendixes, etc. following the main part of a text. [\xref{http://www.tei-c.org/release/doc/tei-p5-doc/en/html/DS.html\#DSBACK}{4.7. Back Matter} \xref{http://www.tei-c.org/release/doc/tei-p5-doc/en/html/DS.html\#DS}{4. Default Text Structure}]\end{specHead} 
    \item[{Module}]
  textstructure
    \item[{Attributes}]
  Attributes att.global (\textit{@xml:id}, \textit{@n}, \textit{@xml:lang}, \textit{@xml:base}, \textit{@xml:space})  (att.global.rendition (\textit{@rend}, \textit{@style}, \textit{@rendition})) (att.global.linking (\textit{@corresp}, \textit{@synch}, \textit{@sameAs}, \textit{@copyOf}, \textit{@next}, \textit{@prev}, \textit{@exclude}, \textit{@select})) (att.global.analytic (\textit{@ana})) (att.global.facs (\textit{@facs})) (att.global.change (\textit{@change})) (att.global.responsibility (\textit{@cert}, \textit{@resp})) (att.global.source (\textit{@source})) att.declaring (\textit{@decls}) 
    \item[{Contained by}]
  
    \item[textstructure: ]
   floatingText text\par 
    \item[transcr: ]
   facsimile
    \item[{May contain}]
  
    \item[analysis: ]
   interp interpGrp span spanGrp\par 
    \item[core: ]
   cb divGen gap gb head index lb list listBibl milestone note p pb\par 
    \item[figures: ]
   figure notatedMusic table\par 
    \item[linking: ]
   ab alt altGrp anchor join joinGrp link linkGrp timeline\par 
    \item[namesdates: ]
   listEvent listNym listOrg listPerson listPlace\par 
    \item[textcrit: ]
   app listApp listWit witDetail\par 
    \item[textstructure: ]
   argument byline closer div docAuthor docDate docEdition docImprint docTitle epigraph postscript signed titlePage titlePart trailer\par 
    \item[transcr: ]
   addSpan damageSpan delSpan fw listTranspose metamark space substJoin
    \item[{Note}]
  \par
Because cultural conventions differ as to which elements are grouped as back matter and which as front matter, the content models for the <back> and <front> elements are identical.
    \item[{Example}]
  \leavevmode\bgroup\exampleFont \begin{shaded}\noindent\mbox{}{<\textbf{back}>}\mbox{}\newline 
\hspace*{6pt}{<\textbf{div}\hspace*{6pt}{type}="{appendix}">}\mbox{}\newline 
\hspace*{6pt}\hspace*{6pt}{<\textbf{head}>}The Golden Dream or, the Ingenuous Confession{</\textbf{head}>}\mbox{}\newline 
\hspace*{6pt}\hspace*{6pt}{<\textbf{p}>}TO shew the Depravity of human Nature, and how apt the Mind is to be misled by Trinkets\mbox{}\newline 
\hspace*{6pt}\hspace*{6pt}\hspace*{6pt}\hspace*{6pt} and false Appearances, Mrs. Two-Shoes does acknowledge, that after she became rich, she\mbox{}\newline 
\hspace*{6pt}\hspace*{6pt}\hspace*{6pt}\hspace*{6pt} had like to have been, too fond of Money \mbox{}\newline 
\textit{<!-- .... -->}\mbox{}\newline 
\hspace*{6pt}\hspace*{6pt}{</\textbf{p}>}\mbox{}\newline 
\hspace*{6pt}{</\textbf{div}>}\mbox{}\newline 
\textit{<!-- ... -->}\mbox{}\newline 
\hspace*{6pt}{<\textbf{div}\hspace*{6pt}{type}="{epistle}">}\mbox{}\newline 
\hspace*{6pt}\hspace*{6pt}{<\textbf{head}>}A letter from the Printer, which he desires may be inserted{</\textbf{head}>}\mbox{}\newline 
\hspace*{6pt}\hspace*{6pt}{<\textbf{salute}>}Sir.{</\textbf{salute}>}\mbox{}\newline 
\hspace*{6pt}\hspace*{6pt}{<\textbf{p}>}I have done with your Copy, so you may return it to the Vatican, if you please;\mbox{}\newline 
\hspace*{6pt}\hspace*{6pt}\mbox{}\newline 
\textit{<!-- ... -->}\mbox{}\newline 
\hspace*{6pt}\hspace*{6pt}{</\textbf{p}>}\mbox{}\newline 
\hspace*{6pt}{</\textbf{div}>}\mbox{}\newline 
\hspace*{6pt}{<\textbf{div}\hspace*{6pt}{type}="{advert}">}\mbox{}\newline 
\hspace*{6pt}\hspace*{6pt}{<\textbf{head}>}The Books usually read by the Scholars of Mrs Two-Shoes are these and are sold at Mr\mbox{}\newline 
\hspace*{6pt}\hspace*{6pt}\hspace*{6pt}\hspace*{6pt} Newbery's at the Bible and Sun in St Paul's Church-yard.{</\textbf{head}>}\mbox{}\newline 
\hspace*{6pt}\hspace*{6pt}{<\textbf{list}>}\mbox{}\newline 
\hspace*{6pt}\hspace*{6pt}\hspace*{6pt}{<\textbf{item}\hspace*{6pt}{n}="{1}">}The Christmas Box, Price 1d.{</\textbf{item}>}\mbox{}\newline 
\hspace*{6pt}\hspace*{6pt}\hspace*{6pt}{<\textbf{item}\hspace*{6pt}{n}="{2}">}The History of Giles Gingerbread, 1d.{</\textbf{item}>}\mbox{}\newline 
\textit{<!-- ... -->}\mbox{}\newline 
\hspace*{6pt}\hspace*{6pt}\hspace*{6pt}{<\textbf{item}\hspace*{6pt}{n}="{42}">}A Curious Collection of Travels, selected from the Writers of all Nations,\mbox{}\newline 
\hspace*{6pt}\hspace*{6pt}\hspace*{6pt}\hspace*{6pt}\hspace*{6pt}\hspace*{6pt} 10 Vol, Pr. bound 1l.{</\textbf{item}>}\mbox{}\newline 
\hspace*{6pt}\hspace*{6pt}{</\textbf{list}>}\mbox{}\newline 
\hspace*{6pt}{</\textbf{div}>}\mbox{}\newline 
\hspace*{6pt}{<\textbf{div}\hspace*{6pt}{type}="{advert}">}\mbox{}\newline 
\hspace*{6pt}\hspace*{6pt}{<\textbf{head}>}By the KING's Royal Patent, Are sold by J. NEWBERY, at the Bible and Sun in St.\mbox{}\newline 
\hspace*{6pt}\hspace*{6pt}\hspace*{6pt}\hspace*{6pt} Paul's Church-Yard.{</\textbf{head}>}\mbox{}\newline 
\hspace*{6pt}\hspace*{6pt}{<\textbf{list}>}\mbox{}\newline 
\hspace*{6pt}\hspace*{6pt}\hspace*{6pt}{<\textbf{item}\hspace*{6pt}{n}="{1}">}Dr. James's Powders for Fevers, the Small-Pox, Measles, Colds, \&c. 2s.\mbox{}\newline 
\hspace*{6pt}\hspace*{6pt}\hspace*{6pt}\hspace*{6pt}\hspace*{6pt}\hspace*{6pt} 6d{</\textbf{item}>}\mbox{}\newline 
\hspace*{6pt}\hspace*{6pt}\hspace*{6pt}{<\textbf{item}\hspace*{6pt}{n}="{2}">}Dr. Hooper's Female Pills, 1s.{</\textbf{item}>}\mbox{}\newline 
\textit{<!-- ... -->}\mbox{}\newline 
\hspace*{6pt}\hspace*{6pt}{</\textbf{list}>}\mbox{}\newline 
\hspace*{6pt}{</\textbf{div}>}\mbox{}\newline 
{</\textbf{back}>}\end{shaded}\egroup 


    \item[{Content model}]
  \mbox{}\hfill\\[-10pt]\begin{Verbatim}[fontsize=\small]
<content>
 <sequence>
  <alternate maxOccurs="unbounded"
   minOccurs="0">
   <classRef key="model.frontPart"/>
   <classRef key="model.pLike.front"/>
   <classRef key="model.pLike"/>
   <classRef key="model.listLike"/>
   <classRef key="model.global"/>
  </alternate>
  <alternate minOccurs="0">
   <sequence>
    <classRef key="model.div1Like"/>
    <alternate maxOccurs="unbounded"
     minOccurs="0">
     <classRef key="model.frontPart"/>
     <classRef key="model.div1Like"/>
     <classRef key="model.global"/>
    </alternate>
   </sequence>
   <sequence>
    <classRef key="model.divLike"/>
    <alternate maxOccurs="unbounded"
     minOccurs="0">
     <classRef key="model.frontPart"/>
     <classRef key="model.divLike"/>
     <classRef key="model.global"/>
    </alternate>
   </sequence>
  </alternate>
  <sequence minOccurs="0">
   <classRef key="model.divBottomPart"/>
   <alternate maxOccurs="unbounded"
    minOccurs="0">
    <classRef key="model.divBottomPart"/>
    <classRef key="model.global"/>
   </alternate>
  </sequence>
 </sequence>
</content>
    
\end{Verbatim}

    \item[{Schema Declaration}]
  \mbox{}\hfill\\[-10pt]\begin{Verbatim}[fontsize=\small]
element back
{
   att.global.attributes,
   att.declaring.attributes,
   (
      (
         model.frontPart       | model.pLike.front       | model.pLike       | model.listLike       | model.global      )*,
      (
         (
            model.div1Like,
            ( model.frontPart | model.div1Like | model.global )*
         )
       | ( model.divLike, ( model.frontPart | model.divLike | model.global )* )
      )?,
      ( model.divBottomPart, ( model.divBottomPart | model.global )* )?
   )
}
\end{Verbatim}

\end{reflist}  \index{bibl=<bibl>|oddindex}
\begin{reflist}
\item[]\begin{specHead}{TEI.bibl}{<bibl> }(bibliographic citation) contains a loosely-structured bibliographic citation of which the sub-components may or may not be explicitly tagged. [\xref{http://www.tei-c.org/release/doc/tei-p5-doc/en/html/CO.html\#COBITY}{3.11.1. Methods of Encoding Bibliographic References and Lists of References} \xref{http://www.tei-c.org/release/doc/tei-p5-doc/en/html/HD.html\#HD3}{2.2.7. The Source Description} \xref{http://www.tei-c.org/release/doc/tei-p5-doc/en/html/CC.html\#CCAS2}{15.3.2. Declarable Elements}]\end{specHead} 
    \item[{Module}]
  core
    \item[{Attributes}]
  Attributes att.global (\textit{@xml:id}, \textit{@n}, \textit{@xml:lang}, \textit{@xml:base}, \textit{@xml:space})  (att.global.rendition (\textit{@rend}, \textit{@style}, \textit{@rendition})) (att.global.linking (\textit{@corresp}, \textit{@synch}, \textit{@sameAs}, \textit{@copyOf}, \textit{@next}, \textit{@prev}, \textit{@exclude}, \textit{@select})) (att.global.analytic (\textit{@ana})) (att.global.facs (\textit{@facs})) (att.global.change (\textit{@change})) (att.global.responsibility (\textit{@cert}, \textit{@resp})) (att.global.source (\textit{@source})) att.declarable (\textit{@default}) att.typed (\textit{@type}, \textit{@subtype}) att.sortable (\textit{@sortKey}) att.docStatus (\textit{@status}) 
    \item[{Member of}]
  model.biblLike model.biblPart 
    \item[{Contained by}]
  
    \item[core: ]
   add bibl cit corr del desc emph head hi item l listBibl meeting note orig p q quote ref reg relatedItem said sic stage title unclear\par 
    \item[figures: ]
   cell figDesc figure\par 
    \item[header: ]
   change handNote licence rendition scriptNote sourceDesc tagUsage taxonomy typeNote\par 
    \item[linking: ]
   ab seg\par 
    \item[msdescription: ]
   accMat acquisition additions collation condition custEvent decoNote filiation foliation layout msItem msItemStruct musicNotation origin provenance signatures source summary support surrogates\par 
    \item[namesdates: ]
   climate event location occupation org person personGrp place population state terrain trait\par 
    \item[textcrit: ]
   lem rdg witness\par 
    \item[textstructure: ]
   argument body div docEdition epigraph imprimatur postscript salute signed titlePart trailer\par 
    \item[transcr: ]
   damage metamark mod restore retrace secl supplied surplus
    \item[{May contain}]
  
    \item[analysis: ]
   c cl interp interpGrp m pc phr s span spanGrp w\par 
    \item[core: ]
   abbr add address author bibl biblScope cb choice citedRange corr date del distinct editor email emph expan foreign gap gb gloss hi index lb measure measureGrp meeting mentioned milestone name note num orig pb ptr pubPlace publisher ref reg relatedItem respStmt rs series sic soCalled term textLang time title unclear\par 
    \item[figures: ]
   figure notatedMusic\par 
    \item[gaiji: ]
   g\par 
    \item[header: ]
   availability distributor edition extent funder idno principal sponsor\par 
    \item[linking: ]
   alt altGrp anchor join joinGrp link linkGrp seg timeline\par 
    \item[msdescription: ]
   depth dim height msIdentifier width\par 
    \item[namesdates: ]
   addName affiliation bloc climate country district forename genName geo geogFeat geogName listRelation location nameLink offset orgName persName placeName population region roleName settlement state surname terrain trait\par 
    \item[textcrit: ]
   app witDetail\par 
    \item[transcr: ]
   addSpan am damage damageSpan delSpan ex fw handShift listTranspose metamark mod redo restore retrace secl space subst substJoin supplied surplus undo\par character data
    \item[{Note}]
  \par
Contains phrase-level elements, together with any combination of elements from the \textit{biblPart} class
    \item[{Example}]
  \leavevmode\bgroup\exampleFont \begin{shaded}\noindent\mbox{}{<\textbf{bibl}>}Blain, Clements and Grundy: Feminist Companion to Literature in English (Yale,\mbox{}\newline 
 1990){</\textbf{bibl}>}\end{shaded}\egroup 


    \item[{Example}]
  \leavevmode\bgroup\exampleFont \begin{shaded}\noindent\mbox{}{<\textbf{bibl}>}\mbox{}\newline 
\hspace*{6pt}{<\textbf{title}\hspace*{6pt}{level}="{a}">}The Interesting story of the Children in the Wood{</\textbf{title}>}. In\mbox{}\newline 
{<\textbf{author}>}Victor E Neuberg{</\textbf{author}>}, {<\textbf{title}>}The Penny Histories{</\textbf{title}>}.\mbox{}\newline 
{<\textbf{publisher}>}OUP{</\textbf{publisher}>}\mbox{}\newline 
\hspace*{6pt}{<\textbf{date}>}1968{</\textbf{date}>}. \mbox{}\newline 
{</\textbf{bibl}>}\end{shaded}\egroup 


    \item[{Example}]
  \leavevmode\bgroup\exampleFont \begin{shaded}\noindent\mbox{}{<\textbf{bibl}\hspace*{6pt}{subtype}="{book\textunderscore chapter}"\hspace*{6pt}{type}="{article}"\mbox{}\newline 
\hspace*{6pt}{xml:id}="{carlin\textunderscore 2003}">}\mbox{}\newline 
\hspace*{6pt}{<\textbf{author}>}\mbox{}\newline 
\hspace*{6pt}\hspace*{6pt}{<\textbf{name}>}\mbox{}\newline 
\hspace*{6pt}\hspace*{6pt}\hspace*{6pt}{<\textbf{surname}>}Carlin{</\textbf{surname}>}\mbox{}\newline 
\hspace*{6pt}\hspace*{6pt}\hspace*{6pt}\hspace*{6pt} ({<\textbf{forename}>}Claire{</\textbf{forename}>}){</\textbf{name}>}\mbox{}\newline 
\hspace*{6pt}{</\textbf{author}>},\mbox{}\newline 
{<\textbf{title}\hspace*{6pt}{level}="{a}">}The Staging of Impotence : France’s last\mbox{}\newline 
\hspace*{6pt}\hspace*{6pt} congrès{</\textbf{title}>} dans\mbox{}\newline 
{<\textbf{bibl}\hspace*{6pt}{type}="{monogr}">}\mbox{}\newline 
\hspace*{6pt}\hspace*{6pt}{<\textbf{title}\hspace*{6pt}{level}="{m}">}Theatrum mundi : studies in honor of Ronald W.\mbox{}\newline 
\hspace*{6pt}\hspace*{6pt}\hspace*{6pt}\hspace*{6pt} Tobin{</\textbf{title}>}, éd.\mbox{}\newline 
\hspace*{6pt}{<\textbf{editor}>}\mbox{}\newline 
\hspace*{6pt}\hspace*{6pt}\hspace*{6pt}{<\textbf{name}>}\mbox{}\newline 
\hspace*{6pt}\hspace*{6pt}\hspace*{6pt}\hspace*{6pt}{<\textbf{forename}>}Claire{</\textbf{forename}>}\mbox{}\newline 
\hspace*{6pt}\hspace*{6pt}\hspace*{6pt}\hspace*{6pt}{<\textbf{surname}>}Carlin{</\textbf{surname}>}\mbox{}\newline 
\hspace*{6pt}\hspace*{6pt}\hspace*{6pt}{</\textbf{name}>}\mbox{}\newline 
\hspace*{6pt}\hspace*{6pt}{</\textbf{editor}>} et\mbox{}\newline 
\hspace*{6pt}{<\textbf{editor}>}\mbox{}\newline 
\hspace*{6pt}\hspace*{6pt}\hspace*{6pt}{<\textbf{name}>}\mbox{}\newline 
\hspace*{6pt}\hspace*{6pt}\hspace*{6pt}\hspace*{6pt}{<\textbf{forename}>}Kathleen{</\textbf{forename}>}\mbox{}\newline 
\hspace*{6pt}\hspace*{6pt}\hspace*{6pt}\hspace*{6pt}{<\textbf{surname}>}Wine{</\textbf{surname}>}\mbox{}\newline 
\hspace*{6pt}\hspace*{6pt}\hspace*{6pt}{</\textbf{name}>}\mbox{}\newline 
\hspace*{6pt}\hspace*{6pt}{</\textbf{editor}>},\mbox{}\newline 
\hspace*{6pt}{<\textbf{pubPlace}>}Charlottesville, Va.{</\textbf{pubPlace}>},\mbox{}\newline 
\hspace*{6pt}{<\textbf{publisher}>}Rookwood Press{</\textbf{publisher}>},\mbox{}\newline 
\hspace*{6pt}{<\textbf{date}\hspace*{6pt}{when}="{2003}">}2003{</\textbf{date}>}.\mbox{}\newline 
\hspace*{6pt}{</\textbf{bibl}>}\mbox{}\newline 
{</\textbf{bibl}>}\end{shaded}\egroup 


    \item[{Content model}]
  \mbox{}\hfill\\[-10pt]\begin{Verbatim}[fontsize=\small]
<content>
 <alternate maxOccurs="unbounded"
  minOccurs="0">
  <textNode/>
  <classRef key="model.gLike"/>
  <classRef key="model.highlighted"/>
  <classRef key="model.pPart.data"/>
  <classRef key="model.pPart.edit"/>
  <classRef key="model.segLike"/>
  <classRef key="model.ptrLike"/>
  <classRef key="model.biblPart"/>
  <classRef key="model.global"/>
 </alternate>
</content>
    
\end{Verbatim}

    \item[{Schema Declaration}]
  \mbox{}\hfill\\[-10pt]\begin{Verbatim}[fontsize=\small]
element bibl
{
   att.global.attributes,
   att.declarable.attributes,
   att.typed.attributes,
   att.sortable.attributes,
   att.docStatus.attributes,
   (
      text
    | model.gLike    | model.highlighted    | model.pPart.data    | model.pPart.edit    | model.segLike    | model.ptrLike    | model.biblPart    | model.global   )*
}
\end{Verbatim}

\end{reflist}  \index{biblFull=<biblFull>|oddindex}
\begin{reflist}
\item[]\begin{specHead}{TEI.biblFull}{<biblFull> }(fully-structured bibliographic citation) contains a fully-structured bibliographic citation, in which all components of the TEI file description are present. [\xref{http://www.tei-c.org/release/doc/tei-p5-doc/en/html/CO.html\#COBITY}{3.11.1. Methods of Encoding Bibliographic References and Lists of References} \xref{http://www.tei-c.org/release/doc/tei-p5-doc/en/html/HD.html\#HD2}{2.2. The File Description} \xref{http://www.tei-c.org/release/doc/tei-p5-doc/en/html/HD.html\#HD3}{2.2.7. The Source Description} \xref{http://www.tei-c.org/release/doc/tei-p5-doc/en/html/CC.html\#CCAS2}{15.3.2. Declarable Elements}]\end{specHead} 
    \item[{Module}]
  header
    \item[{Attributes}]
  Attributes att.global (\textit{@xml:id}, \textit{@n}, \textit{@xml:lang}, \textit{@xml:base}, \textit{@xml:space})  (att.global.rendition (\textit{@rend}, \textit{@style}, \textit{@rendition})) (att.global.linking (\textit{@corresp}, \textit{@synch}, \textit{@sameAs}, \textit{@copyOf}, \textit{@next}, \textit{@prev}, \textit{@exclude}, \textit{@select})) (att.global.analytic (\textit{@ana})) (att.global.facs (\textit{@facs})) (att.global.change (\textit{@change})) (att.global.responsibility (\textit{@cert}, \textit{@resp})) (att.global.source (\textit{@source})) att.declarable (\textit{@default}) att.sortable (\textit{@sortKey}) att.docStatus (\textit{@status}) 
    \item[{Member of}]
  model.biblLike
    \item[{Contained by}]
  
    \item[core: ]
   add cit corr del desc emph head hi item l listBibl meeting note orig p q quote ref reg relatedItem said sic stage title unclear\par 
    \item[figures: ]
   cell figDesc figure\par 
    \item[header: ]
   change handNote licence rendition scriptNote sourceDesc tagUsage taxonomy typeNote\par 
    \item[linking: ]
   ab seg\par 
    \item[msdescription: ]
   accMat acquisition additions collation condition custEvent decoNote filiation foliation layout msItem musicNotation origin provenance signatures source summary support surrogates\par 
    \item[namesdates: ]
   climate event location occupation org person personGrp place population state terrain trait\par 
    \item[textcrit: ]
   lem rdg witness\par 
    \item[textstructure: ]
   argument body div docEdition epigraph imprimatur postscript salute signed titlePart trailer\par 
    \item[transcr: ]
   damage metamark mod restore retrace secl supplied surplus
    \item[{May contain}]
  
    \item[header: ]
   editionStmt extent fileDesc notesStmt profileDesc publicationStmt seriesStmt sourceDesc titleStmt
    \item[{Example}]
  \leavevmode\bgroup\exampleFont \begin{shaded}\noindent\mbox{}{<\textbf{biblFull}>}\mbox{}\newline 
\hspace*{6pt}{<\textbf{titleStmt}>}\mbox{}\newline 
\hspace*{6pt}\hspace*{6pt}{<\textbf{title}>}The Feminist Companion to Literature in English: women writers from the middle ages\mbox{}\newline 
\hspace*{6pt}\hspace*{6pt}\hspace*{6pt}\hspace*{6pt} to the present{</\textbf{title}>}\mbox{}\newline 
\hspace*{6pt}\hspace*{6pt}{<\textbf{author}>}Blain, Virginia{</\textbf{author}>}\mbox{}\newline 
\hspace*{6pt}\hspace*{6pt}{<\textbf{author}>}Clements, Patricia{</\textbf{author}>}\mbox{}\newline 
\hspace*{6pt}\hspace*{6pt}{<\textbf{author}>}Grundy, Isobel{</\textbf{author}>}\mbox{}\newline 
\hspace*{6pt}{</\textbf{titleStmt}>}\mbox{}\newline 
\hspace*{6pt}{<\textbf{editionStmt}>}\mbox{}\newline 
\hspace*{6pt}\hspace*{6pt}{<\textbf{edition}>}UK edition{</\textbf{edition}>}\mbox{}\newline 
\hspace*{6pt}{</\textbf{editionStmt}>}\mbox{}\newline 
\hspace*{6pt}{<\textbf{extent}>}1231 pp{</\textbf{extent}>}\mbox{}\newline 
\hspace*{6pt}{<\textbf{publicationStmt}>}\mbox{}\newline 
\hspace*{6pt}\hspace*{6pt}{<\textbf{publisher}>}Yale University Press{</\textbf{publisher}>}\mbox{}\newline 
\hspace*{6pt}\hspace*{6pt}{<\textbf{pubPlace}>}New Haven and London{</\textbf{pubPlace}>}\mbox{}\newline 
\hspace*{6pt}\hspace*{6pt}{<\textbf{date}>}1990{</\textbf{date}>}\mbox{}\newline 
\hspace*{6pt}{</\textbf{publicationStmt}>}\mbox{}\newline 
\hspace*{6pt}{<\textbf{sourceDesc}>}\mbox{}\newline 
\hspace*{6pt}\hspace*{6pt}{<\textbf{p}>}No source: this is an original work{</\textbf{p}>}\mbox{}\newline 
\hspace*{6pt}{</\textbf{sourceDesc}>}\mbox{}\newline 
{</\textbf{biblFull}>}\end{shaded}\egroup 


    \item[{Content model}]
  \mbox{}\hfill\\[-10pt]\begin{Verbatim}[fontsize=\small]
<content>
 <alternate>
  <sequence>
   <sequence>
    <elementRef key="titleStmt"/>
    <elementRef key="editionStmt"
     minOccurs="0"/>
    <elementRef key="extent" minOccurs="0"/>
    <elementRef key="publicationStmt"/>
    <elementRef key="seriesStmt"
     minOccurs="0"/>
    <elementRef key="notesStmt"
     minOccurs="0"/>
   </sequence>
   <elementRef key="sourceDesc"
    maxOccurs="unbounded" minOccurs="0"/>
  </sequence>
  <sequence>
   <elementRef key="fileDesc"/>
   <elementRef key="profileDesc"/>
  </sequence>
 </alternate>
</content>
    
\end{Verbatim}

    \item[{Schema Declaration}]
  \mbox{}\hfill\\[-10pt]\begin{Verbatim}[fontsize=\small]
element biblFull
{
   att.global.attributes,
   att.declarable.attributes,
   att.sortable.attributes,
   att.docStatus.attributes,
   (
      (
         (
            titleStmt,
            editionStmt?,
            extent?,
            publicationStmt,
            seriesStmt?,
            notesStmt?
         ),
         sourceDesc*
      )
    | ( fileDesc, profileDesc )
   )
}
\end{Verbatim}

\end{reflist}  \index{biblScope=<biblScope>|oddindex}
\begin{reflist}
\item[]\begin{specHead}{TEI.biblScope}{<biblScope> }(scope of bibliographic reference) defines the scope of a bibliographic reference, for example as a list of page numbers, or a named subdivision of a larger work. [\xref{http://www.tei-c.org/release/doc/tei-p5-doc/en/html/CO.html\#COBICOB}{3.11.2.5. Scopes and Ranges in Bibliographic Citations}]\end{specHead} 
    \item[{Module}]
  core
    \item[{Attributes}]
  Attributes att.global (\textit{@xml:id}, \textit{@n}, \textit{@xml:lang}, \textit{@xml:base}, \textit{@xml:space})  (att.global.rendition (\textit{@rend}, \textit{@style}, \textit{@rendition})) (att.global.linking (\textit{@corresp}, \textit{@synch}, \textit{@sameAs}, \textit{@copyOf}, \textit{@next}, \textit{@prev}, \textit{@exclude}, \textit{@select})) (att.global.analytic (\textit{@ana})) (att.global.facs (\textit{@facs})) (att.global.change (\textit{@change})) (att.global.responsibility (\textit{@cert}, \textit{@resp})) (att.global.source (\textit{@source})) att.citing (\textit{@unit}, \textit{@from}, \textit{@to}) 
    \item[{Member of}]
  model.imprintPart 
    \item[{Contained by}]
  
    \item[core: ]
   bibl imprint monogr series\par 
    \item[header: ]
   seriesStmt
    \item[{May contain}]
  
    \item[analysis: ]
   c cl interp interpGrp m pc phr s span spanGrp w\par 
    \item[core: ]
   abbr add address cb choice corr date del distinct email emph expan foreign gap gb gloss graphic hi index lb measure measureGrp media mentioned milestone name note num orig pb ptr ref reg rs sic soCalled term time title unclear\par 
    \item[figures: ]
   figure formula notatedMusic\par 
    \item[gaiji: ]
   g\par 
    \item[header: ]
   idno\par 
    \item[linking: ]
   alt altGrp anchor join joinGrp link linkGrp seg timeline\par 
    \item[msdescription: ]
   catchwords depth dim dimensions height heraldry locus locusGrp material objectType origDate origPlace secFol signatures stamp watermark width\par 
    \item[namesdates: ]
   addName affiliation bloc climate country district forename genName geo geogFeat geogName location nameLink offset orgName persName placeName population region roleName settlement state surname terrain trait\par 
    \item[textcrit: ]
   app witDetail\par 
    \item[transcr: ]
   addSpan am damage damageSpan delSpan ex fw handShift listTranspose metamark mod redo restore retrace secl space subst substJoin supplied surplus undo\par character data
    \item[{Note}]
  \par
When a single page is being cited, use the {\itshape from} and {\itshape to} attributes with an identical value. When no clear endpoint is provided, the {\itshape from} attribute may be used without {\itshape to}; for example a citation such as ‘p. 3ff’ might be encoded \texttt{<biblScope from="3">p. 2ff<biblScope>}.\par
It is now considered good practice to supply this element as a sibling (rather than a child) of <imprint>, since it supplies information which does not constitute part of the imprint.
    \item[{Example}]
  \leavevmode\bgroup\exampleFont \begin{shaded}\noindent\mbox{}{<\textbf{biblScope}>}pp 12–34{</\textbf{biblScope}>}\mbox{}\newline 
{<\textbf{biblScope}\hspace*{6pt}{from}="{12}"\hspace*{6pt}{to}="{34}"\hspace*{6pt}{unit}="{page}"/>}\mbox{}\newline 
{<\textbf{biblScope}\hspace*{6pt}{unit}="{volume}">}II{</\textbf{biblScope}>}\mbox{}\newline 
{<\textbf{biblScope}\hspace*{6pt}{unit}="{page}">}12{</\textbf{biblScope}>}\end{shaded}\egroup 


    \item[{Content model}]
  \mbox{}\hfill\\[-10pt]\begin{Verbatim}[fontsize=\small]
<content>
 <macroRef key="macro.phraseSeq"/>
</content>
    
\end{Verbatim}

    \item[{Schema Declaration}]
  \mbox{}\hfill\\[-10pt]\begin{Verbatim}[fontsize=\small]
element biblScope
{
   att.global.attributes,
   att.citing.attributes,
   macro.phraseSeq}
\end{Verbatim}

\end{reflist}  \index{biblStruct=<biblStruct>|oddindex}
\begin{reflist}
\item[]\begin{specHead}{TEI.biblStruct}{<biblStruct> }(structured bibliographic citation) contains a structured bibliographic citation, in which only bibliographic sub-elements appear and in a specified order. [\xref{http://www.tei-c.org/release/doc/tei-p5-doc/en/html/CO.html\#COBITY}{3.11.1. Methods of Encoding Bibliographic References and Lists of References} \xref{http://www.tei-c.org/release/doc/tei-p5-doc/en/html/HD.html\#HD3}{2.2.7. The Source Description} \xref{http://www.tei-c.org/release/doc/tei-p5-doc/en/html/CC.html\#CCAS2}{15.3.2. Declarable Elements}]\end{specHead} 
    \item[{Module}]
  core
    \item[{Attributes}]
  Attributes att.global (\textit{@xml:id}, \textit{@n}, \textit{@xml:lang}, \textit{@xml:base}, \textit{@xml:space})  (att.global.rendition (\textit{@rend}, \textit{@style}, \textit{@rendition})) (att.global.linking (\textit{@corresp}, \textit{@synch}, \textit{@sameAs}, \textit{@copyOf}, \textit{@next}, \textit{@prev}, \textit{@exclude}, \textit{@select})) (att.global.analytic (\textit{@ana})) (att.global.facs (\textit{@facs})) (att.global.change (\textit{@change})) (att.global.responsibility (\textit{@cert}, \textit{@resp})) (att.global.source (\textit{@source})) att.declarable (\textit{@default}) att.typed (\textit{@type}, \textit{@subtype}) att.sortable (\textit{@sortKey}) att.docStatus (\textit{@status}) 
    \item[{Member of}]
  model.biblLike 
    \item[{Contained by}]
  
    \item[core: ]
   add cit corr del desc emph head hi item l listBibl meeting note orig p q quote ref reg relatedItem said sic stage title unclear\par 
    \item[figures: ]
   cell figDesc figure\par 
    \item[header: ]
   change handNote licence rendition scriptNote sourceDesc tagUsage taxonomy typeNote\par 
    \item[linking: ]
   ab seg\par 
    \item[msdescription: ]
   accMat acquisition additions collation condition custEvent decoNote filiation foliation layout msItem msItemStruct musicNotation origin provenance signatures source summary support surrogates\par 
    \item[namesdates: ]
   climate event location occupation org person personGrp place population state terrain trait\par 
    \item[textcrit: ]
   lem rdg witness\par 
    \item[textstructure: ]
   argument body div docEdition epigraph imprimatur postscript salute signed titlePart trailer\par 
    \item[transcr: ]
   damage metamark mod restore retrace secl supplied surplus
    \item[{May contain}]
  
    \item[core: ]
   analytic citedRange monogr note ptr ref relatedItem series\par 
    \item[textcrit: ]
   witDetail
    \item[{Example}]
  \leavevmode\bgroup\exampleFont \begin{shaded}\noindent\mbox{}{<\textbf{biblStruct}>}\mbox{}\newline 
\hspace*{6pt}{<\textbf{monogr}>}\mbox{}\newline 
\hspace*{6pt}\hspace*{6pt}{<\textbf{author}>}Blain, Virginia{</\textbf{author}>}\mbox{}\newline 
\hspace*{6pt}\hspace*{6pt}{<\textbf{author}>}Clements, Patricia{</\textbf{author}>}\mbox{}\newline 
\hspace*{6pt}\hspace*{6pt}{<\textbf{author}>}Grundy, Isobel{</\textbf{author}>}\mbox{}\newline 
\hspace*{6pt}\hspace*{6pt}{<\textbf{title}>}The Feminist Companion to Literature in English: women writers from the middle ages\mbox{}\newline 
\hspace*{6pt}\hspace*{6pt}\hspace*{6pt}\hspace*{6pt} to the present{</\textbf{title}>}\mbox{}\newline 
\hspace*{6pt}\hspace*{6pt}{<\textbf{edition}>}first edition{</\textbf{edition}>}\mbox{}\newline 
\hspace*{6pt}\hspace*{6pt}{<\textbf{imprint}>}\mbox{}\newline 
\hspace*{6pt}\hspace*{6pt}\hspace*{6pt}{<\textbf{publisher}>}Yale University Press{</\textbf{publisher}>}\mbox{}\newline 
\hspace*{6pt}\hspace*{6pt}\hspace*{6pt}{<\textbf{pubPlace}>}New Haven and London{</\textbf{pubPlace}>}\mbox{}\newline 
\hspace*{6pt}\hspace*{6pt}\hspace*{6pt}{<\textbf{date}>}1990{</\textbf{date}>}\mbox{}\newline 
\hspace*{6pt}\hspace*{6pt}{</\textbf{imprint}>}\mbox{}\newline 
\hspace*{6pt}{</\textbf{monogr}>}\mbox{}\newline 
{</\textbf{biblStruct}>}\end{shaded}\egroup 


    \item[{Content model}]
  \mbox{}\hfill\\[-10pt]\begin{Verbatim}[fontsize=\small]
<content>
 <sequence>
  <elementRef key="analytic"
   maxOccurs="unbounded" minOccurs="0"/>
  <sequence maxOccurs="unbounded"
   minOccurs="1">
   <elementRef key="monogr"/>
   <elementRef key="series"
    maxOccurs="unbounded" minOccurs="0"/>
  </sequence>
  <alternate maxOccurs="unbounded"
   minOccurs="0">
   <classRef key="model.noteLike"/>
   <classRef key="model.ptrLike"/>
   <elementRef key="relatedItem"/>
   <elementRef key="citedRange"/>
  </alternate>
 </sequence>
</content>
    
\end{Verbatim}

    \item[{Schema Declaration}]
  \mbox{}\hfill\\[-10pt]\begin{Verbatim}[fontsize=\small]
element biblStruct
{
   att.global.attributes,
   att.declarable.attributes,
   att.typed.attributes,
   att.sortable.attributes,
   att.docStatus.attributes,
   (
      analytic*,
      ( monogr, series* )+,
      ( model.noteLike | model.ptrLike | relatedItem | citedRange )*
   )
}
\end{Verbatim}

\end{reflist}  \index{binding=<binding>|oddindex}\index{contemporary=@contemporary!<binding>|oddindex}
\begin{reflist}
\item[]\begin{specHead}{TEI.binding}{<binding> }contains a description of one binding, i.e. type of covering, boards, etc. applied to a manuscript. [\xref{http://www.tei-c.org/release/doc/tei-p5-doc/en/html/MS.html\#msphbi}{10.7.3.1. Binding Descriptions}]\end{specHead} 
    \item[{Module}]
  msdescription
    \item[{Attributes}]
  Attributes att.global (\textit{@xml:id}, \textit{@n}, \textit{@xml:lang}, \textit{@xml:base}, \textit{@xml:space})  (att.global.rendition (\textit{@rend}, \textit{@style}, \textit{@rendition})) (att.global.linking (\textit{@corresp}, \textit{@synch}, \textit{@sameAs}, \textit{@copyOf}, \textit{@next}, \textit{@prev}, \textit{@exclude}, \textit{@select})) (att.global.analytic (\textit{@ana})) (att.global.facs (\textit{@facs})) (att.global.change (\textit{@change})) (att.global.responsibility (\textit{@cert}, \textit{@resp})) (att.global.source (\textit{@source})) att.datable (\textit{@calendar}, \textit{@period})  (att.datable.w3c (\textit{@when}, \textit{@notBefore}, \textit{@notAfter}, \textit{@from}, \textit{@to})) (att.datable.iso (\textit{@when-iso}, \textit{@notBefore-iso}, \textit{@notAfter-iso}, \textit{@from-iso}, \textit{@to-iso})) (att.datable.custom (\textit{@when-custom}, \textit{@notBefore-custom}, \textit{@notAfter-custom}, \textit{@from-custom}, \textit{@to-custom}, \textit{@datingPoint}, \textit{@datingMethod})) \hfil\\[-10pt]\begin{sansreflist}
    \item[@contemporary]
  specifies whether or not the binding is contemporary with the majority of its contents
\begin{reflist}
    \item[{Status}]
  Optional
    \item[{Datatype}]
  teidata.xTruthValue
    \item[{Note}]
  \par
The value true indicates that the binding is contemporaneous with its contents; the value false that it is not. The value unknown should be used when the date of either binding or manuscript is unknown
\end{reflist}  
\end{sansreflist}  
    \item[{Contained by}]
  
    \item[msdescription: ]
   bindingDesc
    \item[{May contain}]
  
    \item[core: ]
   p\par 
    \item[linking: ]
   ab\par 
    \item[msdescription: ]
   condition decoNote
    \item[{Example}]
  \leavevmode\bgroup\exampleFont \begin{shaded}\noindent\mbox{}{<\textbf{binding}\hspace*{6pt}{contemporary}="{true}">}\mbox{}\newline 
\hspace*{6pt}{<\textbf{p}>}Contemporary blind stamped leather over wooden boards with evidence of a fore edge clasp\mbox{}\newline 
\hspace*{6pt}\hspace*{6pt} closing to the back cover.{</\textbf{p}>}\mbox{}\newline 
{</\textbf{binding}>}\end{shaded}\egroup 


    \item[{Example}]
  \leavevmode\bgroup\exampleFont \begin{shaded}\noindent\mbox{}{<\textbf{bindingDesc}>}\mbox{}\newline 
\hspace*{6pt}{<\textbf{binding}\hspace*{6pt}{contemporary}="{false}">}\mbox{}\newline 
\hspace*{6pt}\hspace*{6pt}{<\textbf{p}>}Quarter bound by the Phillipps' binder, Bretherton, with his sticker on the front\mbox{}\newline 
\hspace*{6pt}\hspace*{6pt}\hspace*{6pt}\hspace*{6pt} pastedown.{</\textbf{p}>}\mbox{}\newline 
\hspace*{6pt}{</\textbf{binding}>}\mbox{}\newline 
\hspace*{6pt}{<\textbf{binding}\hspace*{6pt}{contemporary}="{false}">}\mbox{}\newline 
\hspace*{6pt}\hspace*{6pt}{<\textbf{p}>}Rebound by an unknown 19th c. company; edges cropped and gilt.{</\textbf{p}>}\mbox{}\newline 
\hspace*{6pt}{</\textbf{binding}>}\mbox{}\newline 
{</\textbf{bindingDesc}>}\end{shaded}\egroup 


    \item[{Content model}]
  \mbox{}\hfill\\[-10pt]\begin{Verbatim}[fontsize=\small]
<content>
 <alternate maxOccurs="unbounded"
  minOccurs="1">
  <classRef key="model.pLike"/>
  <elementRef key="condition"/>
  <elementRef key="decoNote"/>
 </alternate>
</content>
    
\end{Verbatim}

    \item[{Schema Declaration}]
  \mbox{}\hfill\\[-10pt]\begin{Verbatim}[fontsize=\small]
element binding
{
   att.global.attributes,
   att.datable.attributes,
   attribute contemporary { text }?,
   ( model.pLike | condition | decoNote )+
}
\end{Verbatim}

\end{reflist}  \index{bindingDesc=<bindingDesc>|oddindex}
\begin{reflist}
\item[]\begin{specHead}{TEI.bindingDesc}{<bindingDesc> }(binding description) describes the present and former bindings of a manuscript, either as a series of paragraphs or as a series of distinct <binding> elements, one for each binding of the manuscript. [\xref{http://www.tei-c.org/release/doc/tei-p5-doc/en/html/MS.html\#msphbi}{10.7.3.1. Binding Descriptions}]\end{specHead} 
    \item[{Module}]
  msdescription
    \item[{Attributes}]
  Attributes att.global (\textit{@xml:id}, \textit{@n}, \textit{@xml:lang}, \textit{@xml:base}, \textit{@xml:space})  (att.global.rendition (\textit{@rend}, \textit{@style}, \textit{@rendition})) (att.global.linking (\textit{@corresp}, \textit{@synch}, \textit{@sameAs}, \textit{@copyOf}, \textit{@next}, \textit{@prev}, \textit{@exclude}, \textit{@select})) (att.global.analytic (\textit{@ana})) (att.global.facs (\textit{@facs})) (att.global.change (\textit{@change})) (att.global.responsibility (\textit{@cert}, \textit{@resp})) (att.global.source (\textit{@source}))
    \item[{Member of}]
  model.physDescPart
    \item[{Contained by}]
  
    \item[msdescription: ]
   physDesc
    \item[{May contain}]
  
    \item[core: ]
   p\par 
    \item[linking: ]
   ab\par 
    \item[msdescription: ]
   binding condition decoNote
    \item[{Example}]
  \leavevmode\bgroup\exampleFont \begin{shaded}\noindent\mbox{}{<\textbf{bindingDesc}>}\mbox{}\newline 
\hspace*{6pt}{<\textbf{p}>}Sewing not visible; tightly rebound over\mbox{}\newline 
\hspace*{6pt}\hspace*{6pt} 19th-cent. pasteboards, reusing panels of 16th-cent. brown leather with\mbox{}\newline 
\hspace*{6pt}\hspace*{6pt} gilt tooling à la fanfare, Paris c. 1580-90, the centre of each\mbox{}\newline 
\hspace*{6pt}\hspace*{6pt} cover inlaid with a 17th-cent. oval medallion of red morocco tooled in\mbox{}\newline 
\hspace*{6pt}\hspace*{6pt} gilt (perhaps replacing the identifying mark of a previous owner); the\mbox{}\newline 
\hspace*{6pt}\hspace*{6pt} spine similarly tooled, without raised bands or title-piece; coloured\mbox{}\newline 
\hspace*{6pt}\hspace*{6pt} endbands; the edges of the leaves and boards gilt.Boxed.{</\textbf{p}>}\mbox{}\newline 
{</\textbf{bindingDesc}>}\end{shaded}\egroup 


    \item[{Content model}]
  \mbox{}\hfill\\[-10pt]\begin{Verbatim}[fontsize=\small]
<content>
 <alternate>
  <alternate maxOccurs="unbounded"
   minOccurs="1">
   <classRef key="model.pLike"/>
   <elementRef key="decoNote"/>
   <elementRef key="condition"/>
  </alternate>
  <elementRef key="binding"
   maxOccurs="unbounded" minOccurs="1"/>
 </alternate>
</content>
    
\end{Verbatim}

    \item[{Schema Declaration}]
  \mbox{}\hfill\\[-10pt]\begin{Verbatim}[fontsize=\small]
element bindingDesc
{
   att.global.attributes,
   ( ( model.pLike | decoNote | condition )+ | binding+ )
}
\end{Verbatim}

\end{reflist}  \index{birth=<birth>|oddindex}
\begin{reflist}
\item[]\begin{specHead}{TEI.birth}{<birth> }contains information about a person's birth, such as its date and place. [\xref{http://www.tei-c.org/release/doc/tei-p5-doc/en/html/CC.html\#CCAHPA}{15.2.2. The Participant Description}]\end{specHead} 
    \item[{Module}]
  namesdates
    \item[{Attributes}]
  Attributes att.global (\textit{@xml:id}, \textit{@n}, \textit{@xml:lang}, \textit{@xml:base}, \textit{@xml:space})  (att.global.rendition (\textit{@rend}, \textit{@style}, \textit{@rendition})) (att.global.linking (\textit{@corresp}, \textit{@synch}, \textit{@sameAs}, \textit{@copyOf}, \textit{@next}, \textit{@prev}, \textit{@exclude}, \textit{@select})) (att.global.analytic (\textit{@ana})) (att.global.facs (\textit{@facs})) (att.global.change (\textit{@change})) (att.global.responsibility (\textit{@cert}, \textit{@resp})) (att.global.source (\textit{@source})) att.editLike (\textit{@evidence}, \textit{@instant})  (att.dimensions (\textit{@unit}, \textit{@quantity}, \textit{@extent}, \textit{@precision}, \textit{@scope}) (att.ranging (\textit{@atLeast}, \textit{@atMost}, \textit{@min}, \textit{@max}, \textit{@confidence})) ) att.datable (\textit{@calendar}, \textit{@period})  (att.datable.w3c (\textit{@when}, \textit{@notBefore}, \textit{@notAfter}, \textit{@from}, \textit{@to})) (att.datable.iso (\textit{@when-iso}, \textit{@notBefore-iso}, \textit{@notAfter-iso}, \textit{@from-iso}, \textit{@to-iso})) (att.datable.custom (\textit{@when-custom}, \textit{@notBefore-custom}, \textit{@notAfter-custom}, \textit{@from-custom}, \textit{@to-custom}, \textit{@datingPoint}, \textit{@datingMethod})) att.naming (\textit{@role}, \textit{@nymRef})  (att.canonical (\textit{@key}, \textit{@ref}))
    \item[{Member of}]
  model.personPart
    \item[{Contained by}]
  
    \item[namesdates: ]
   person personGrp
    \item[{May contain}]
  
    \item[analysis: ]
   c cl interp interpGrp m pc phr s span spanGrp w\par 
    \item[core: ]
   abbr add address cb choice corr date del distinct email emph expan foreign gap gb gloss graphic hi index lb measure measureGrp media mentioned milestone name note num orig pb ptr ref reg rs sic soCalled term time title unclear\par 
    \item[figures: ]
   figure formula notatedMusic\par 
    \item[gaiji: ]
   g\par 
    \item[header: ]
   idno\par 
    \item[linking: ]
   alt altGrp anchor join joinGrp link linkGrp seg timeline\par 
    \item[msdescription: ]
   catchwords depth dim dimensions height heraldry locus locusGrp material objectType origDate origPlace secFol signatures stamp watermark width\par 
    \item[namesdates: ]
   addName affiliation bloc climate country district forename genName geo geogFeat geogName location nameLink offset orgName persName placeName population region roleName settlement state surname terrain trait\par 
    \item[textcrit: ]
   app witDetail\par 
    \item[transcr: ]
   addSpan am damage damageSpan delSpan ex fw handShift listTranspose metamark mod redo restore retrace secl space subst substJoin supplied surplus undo\par character data
    \item[{Example}]
  \leavevmode\bgroup\exampleFont \begin{shaded}\noindent\mbox{}{<\textbf{birth}>}Before 1920, Midlands region.{</\textbf{birth}>}\end{shaded}\egroup 


    \item[{Example}]
  \leavevmode\bgroup\exampleFont \begin{shaded}\noindent\mbox{}{<\textbf{birth}\hspace*{6pt}{when}="{1960-12-10}">}In a small cottage near {<\textbf{name}\hspace*{6pt}{type}="{place}">}Aix-la-Chapelle{</\textbf{name}>},\mbox{}\newline 
 early in the morning of {<\textbf{date}>}10 Dec 1960{</\textbf{date}>}\mbox{}\newline 
{</\textbf{birth}>}\end{shaded}\egroup 


    \item[{Content model}]
  \mbox{}\hfill\\[-10pt]\begin{Verbatim}[fontsize=\small]
<content>
 <macroRef key="macro.phraseSeq"/>
</content>
    
\end{Verbatim}

    \item[{Schema Declaration}]
  \mbox{}\hfill\\[-10pt]\begin{Verbatim}[fontsize=\small]
element birth
{
   att.global.attributes,
   att.editLike.attributes,
   att.datable.attributes,
   att.naming.attributes,
   macro.phraseSeq}
\end{Verbatim}

\end{reflist}  \index{bloc=<bloc>|oddindex}
\begin{reflist}
\item[]\begin{specHead}{TEI.bloc}{<bloc> }contains the name of a geo-political unit consisting of two or more nation states or countries. [\xref{http://www.tei-c.org/release/doc/tei-p5-doc/en/html/ND.html\#NDPLAC}{13.2.3. Place Names}]\end{specHead} 
    \item[{Module}]
  namesdates
    \item[{Attributes}]
  Attributes att.global (\textit{@xml:id}, \textit{@n}, \textit{@xml:lang}, \textit{@xml:base}, \textit{@xml:space})  (att.global.rendition (\textit{@rend}, \textit{@style}, \textit{@rendition})) (att.global.linking (\textit{@corresp}, \textit{@synch}, \textit{@sameAs}, \textit{@copyOf}, \textit{@next}, \textit{@prev}, \textit{@exclude}, \textit{@select})) (att.global.analytic (\textit{@ana})) (att.global.facs (\textit{@facs})) (att.global.change (\textit{@change})) (att.global.responsibility (\textit{@cert}, \textit{@resp})) (att.global.source (\textit{@source})) att.naming (\textit{@role}, \textit{@nymRef})  (att.canonical (\textit{@key}, \textit{@ref})) att.typed (\textit{@type}, \textit{@subtype}) att.datable (\textit{@calendar}, \textit{@period})  (att.datable.w3c (\textit{@when}, \textit{@notBefore}, \textit{@notAfter}, \textit{@from}, \textit{@to})) (att.datable.iso (\textit{@when-iso}, \textit{@notBefore-iso}, \textit{@notAfter-iso}, \textit{@from-iso}, \textit{@to-iso})) (att.datable.custom (\textit{@when-custom}, \textit{@notBefore-custom}, \textit{@notAfter-custom}, \textit{@from-custom}, \textit{@to-custom}, \textit{@datingPoint}, \textit{@datingMethod}))
    \item[{Member of}]
  model.placeNamePart
    \item[{Contained by}]
  
    \item[analysis: ]
   cl phr s span\par 
    \item[core: ]
   abbr add addrLine address author bibl biblScope citedRange corr date del desc distinct editor email emph expan foreign gloss head headItem headLabel hi item l label measure meeting mentioned name note num orig p pubPlace publisher q quote ref reg resp rs said sic soCalled speaker stage street term textLang time title unclear\par 
    \item[figures: ]
   cell figDesc\par 
    \item[header: ]
   authority catDesc change classCode correspAction creation distributor edition extent funder geoDecl handNote language licence principal rendition scriptNote sponsor tagUsage typeNote\par 
    \item[linking: ]
   ab seg\par 
    \item[msdescription: ]
   accMat acquisition additions altIdentifier catchwords collation colophon condition custEvent decoNote explicit filiation finalRubric foliation heraldry incipit layout material msIdentifier musicNotation objectType origDate origPlace origin provenance rubric secFol signatures source stamp summary support surrogates watermark\par 
    \item[namesdates: ]
   addName affiliation age birth bloc country death district education faith floruit forename genName geogFeat geogName langKnown location nameLink nationality occupation offset org orgName persName place placeName region residence roleName settlement sex socecStatus surname\par 
    \item[textcrit: ]
   lem rdg wit witDetail witness\par 
    \item[textstructure: ]
   byline closer dateline docAuthor docDate docEdition docImprint imprimatur opener salute signed titlePart trailer\par 
    \item[transcr: ]
   damage fw metamark mod restore retrace secl supplied surplus
    \item[{May contain}]
  
    \item[analysis: ]
   c cl interp interpGrp m pc phr s span spanGrp w\par 
    \item[core: ]
   abbr add address cb choice corr date del distinct email emph expan foreign gap gb gloss graphic hi index lb measure measureGrp media mentioned milestone name note num orig pb ptr ref reg rs sic soCalled term time title unclear\par 
    \item[figures: ]
   figure formula notatedMusic\par 
    \item[gaiji: ]
   g\par 
    \item[header: ]
   idno\par 
    \item[linking: ]
   alt altGrp anchor join joinGrp link linkGrp seg timeline\par 
    \item[msdescription: ]
   catchwords depth dim dimensions height heraldry locus locusGrp material objectType origDate origPlace secFol signatures stamp watermark width\par 
    \item[namesdates: ]
   addName affiliation bloc climate country district forename genName geo geogFeat geogName location nameLink offset orgName persName placeName population region roleName settlement state surname terrain trait\par 
    \item[textcrit: ]
   app witDetail\par 
    \item[transcr: ]
   addSpan am damage damageSpan delSpan ex fw handShift listTranspose metamark mod redo restore retrace secl space subst substJoin supplied surplus undo\par character data
    \item[{Example}]
  \leavevmode\bgroup\exampleFont \begin{shaded}\noindent\mbox{}{<\textbf{bloc}\hspace*{6pt}{type}="{union}">}the European Union{</\textbf{bloc}>}\mbox{}\newline 
{<\textbf{bloc}\hspace*{6pt}{type}="{continent}">}Africa{</\textbf{bloc}>}\end{shaded}\egroup 


    \item[{Content model}]
  \mbox{}\hfill\\[-10pt]\begin{Verbatim}[fontsize=\small]
<content>
 <macroRef key="macro.phraseSeq"/>
</content>
    
\end{Verbatim}

    \item[{Schema Declaration}]
  \mbox{}\hfill\\[-10pt]\begin{Verbatim}[fontsize=\small]
element bloc
{
   att.global.attributes,
   att.naming.attributes,
   att.typed.attributes,
   att.datable.attributes,
   macro.phraseSeq}
\end{Verbatim}

\end{reflist}  \index{body=<body>|oddindex}
\begin{reflist}
\item[]\begin{specHead}{TEI.body}{<body> }(text body) contains the whole body of a single unitary text, excluding any front or back matter. [\xref{http://www.tei-c.org/release/doc/tei-p5-doc/en/html/DS.html\#DS}{4. Default Text Structure}]\end{specHead} 
    \item[{Module}]
  textstructure
    \item[{Attributes}]
  Attributes att.global (\textit{@xml:id}, \textit{@n}, \textit{@xml:lang}, \textit{@xml:base}, \textit{@xml:space})  (att.global.rendition (\textit{@rend}, \textit{@style}, \textit{@rendition})) (att.global.linking (\textit{@corresp}, \textit{@synch}, \textit{@sameAs}, \textit{@copyOf}, \textit{@next}, \textit{@prev}, \textit{@exclude}, \textit{@select})) (att.global.analytic (\textit{@ana})) (att.global.facs (\textit{@facs})) (att.global.change (\textit{@change})) (att.global.responsibility (\textit{@cert}, \textit{@resp})) (att.global.source (\textit{@source})) att.declaring (\textit{@decls}) 
    \item[{Contained by}]
  
    \item[textstructure: ]
   floatingText text
    \item[{May contain}]
  
    \item[analysis: ]
   interp interpGrp span spanGrp\par 
    \item[core: ]
   bibl biblStruct cb cit desc divGen gap gb head index l label lb lg list listBibl meeting milestone note p pb q quote said sp stage\par 
    \item[figures: ]
   figure notatedMusic table\par 
    \item[header: ]
   biblFull\par 
    \item[linking: ]
   ab alt altGrp anchor join joinGrp link linkGrp timeline\par 
    \item[msdescription: ]
   msDesc\par 
    \item[namesdates: ]
   listEvent listNym listOrg listPerson listPlace\par 
    \item[textcrit: ]
   app listApp listWit witDetail\par 
    \item[textstructure: ]
   argument byline closer dateline div docAuthor docDate epigraph floatingText opener postscript salute signed trailer\par 
    \item[transcr: ]
   addSpan damageSpan delSpan fw listTranspose metamark space substJoin
    \item[{Example}]
  \leavevmode\bgroup\exampleFont \begin{shaded}\noindent\mbox{}{<\textbf{body}>}\mbox{}\newline 
\hspace*{6pt}{<\textbf{l}>}Nu scylun hergan hefaenricaes uard{</\textbf{l}>}\mbox{}\newline 
\hspace*{6pt}{<\textbf{l}>}metudæs maecti end his modgidanc{</\textbf{l}>}\mbox{}\newline 
\hspace*{6pt}{<\textbf{l}>}uerc uuldurfadur sue he uundra gihuaes{</\textbf{l}>}\mbox{}\newline 
\hspace*{6pt}{<\textbf{l}>}eci dryctin or astelidæ{</\textbf{l}>}\mbox{}\newline 
\hspace*{6pt}{<\textbf{l}>}he aerist scop aelda barnum{</\textbf{l}>}\mbox{}\newline 
\hspace*{6pt}{<\textbf{l}>}heben til hrofe haleg scepen.{</\textbf{l}>}\mbox{}\newline 
\hspace*{6pt}{<\textbf{l}>}tha middungeard moncynnæs uard{</\textbf{l}>}\mbox{}\newline 
\hspace*{6pt}{<\textbf{l}>}eci dryctin æfter tiadæ{</\textbf{l}>}\mbox{}\newline 
\hspace*{6pt}{<\textbf{l}>}firum foldu frea allmectig{</\textbf{l}>}\mbox{}\newline 
\hspace*{6pt}{<\textbf{trailer}>}primo cantauit Cædmon istud carmen.{</\textbf{trailer}>}\mbox{}\newline 
{</\textbf{body}>}\end{shaded}\egroup 


    \item[{Content model}]
  \mbox{}\hfill\\[-10pt]\begin{Verbatim}[fontsize=\small]
<content>
 <sequence>
  <classRef key="model.global"
   maxOccurs="unbounded" minOccurs="0"/>
  <sequence minOccurs="0">
   <classRef key="model.divTop"/>
   <alternate maxOccurs="unbounded"
    minOccurs="0">
    <classRef key="model.global"/>
    <classRef key="model.divTop"/>
   </alternate>
  </sequence>
  <sequence minOccurs="0">
   <classRef key="model.divGenLike"/>
   <alternate maxOccurs="unbounded"
    minOccurs="0">
    <classRef key="model.global"/>
    <classRef key="model.divGenLike"/>
   </alternate>
  </sequence>
  <alternate>
   <sequence maxOccurs="unbounded"
    minOccurs="1">
    <classRef key="model.divLike"/>
    <alternate maxOccurs="unbounded"
     minOccurs="0">
     <classRef key="model.global"/>
     <classRef key="model.divGenLike"/>
    </alternate>
   </sequence>
   <sequence maxOccurs="unbounded"
    minOccurs="1">
    <classRef key="model.div1Like"/>
    <alternate maxOccurs="unbounded"
     minOccurs="0">
     <classRef key="model.global"/>
     <classRef key="model.divGenLike"/>
    </alternate>
   </sequence>
   <sequence>
    <sequence maxOccurs="unbounded"
     minOccurs="1">
     <classRef key="model.common"/>
     <classRef key="model.global"
      maxOccurs="unbounded" minOccurs="0"/>
    </sequence>
    <alternate minOccurs="0">
     <sequence maxOccurs="unbounded"
      minOccurs="1">
      <classRef key="model.divLike"/>
      <alternate maxOccurs="unbounded"
       minOccurs="0">
       <classRef key="model.global"/>
       <classRef key="model.divGenLike"/>
      </alternate>
     </sequence>
     <sequence maxOccurs="unbounded"
      minOccurs="1">
      <classRef key="model.div1Like"/>
      <alternate maxOccurs="unbounded"
       minOccurs="0">
       <classRef key="model.global"/>
       <classRef key="model.divGenLike"/>
      </alternate>
     </sequence>
    </alternate>
   </sequence>
  </alternate>
  <sequence maxOccurs="unbounded"
   minOccurs="0">
   <classRef key="model.divBottom"/>
   <classRef key="model.global"
    maxOccurs="unbounded" minOccurs="0"/>
  </sequence>
 </sequence>
</content>
    
\end{Verbatim}

    \item[{Schema Declaration}]
  \mbox{}\hfill\\[-10pt]\begin{Verbatim}[fontsize=\small]
element body
{
   att.global.attributes,
   att.declaring.attributes,
   (
      model.global*,
      ( model.divTop, ( model.global | model.divTop )* )?,
      ( model.divGenLike, ( model.global | model.divGenLike )* )?,
      (
         ( model.divLike, ( model.global | model.divGenLike )* )+
       | ( model.div1Like, ( model.global | model.divGenLike )* )+
       | (
            ( model.common, model.global* )+,
            (
               ( model.divLike, ( model.global | model.divGenLike )* )+
             | ( model.div1Like, ( model.global | model.divGenLike )* )+
            )?
         )
      ),
      ( model.divBottom, model.global* )*
   )
}
\end{Verbatim}

\end{reflist}  \index{byline=<byline>|oddindex}
\begin{reflist}
\item[]\begin{specHead}{TEI.byline}{<byline> }contains the primary statement of responsibility given for a work on its title page or at the head or end of the work. [\xref{http://www.tei-c.org/release/doc/tei-p5-doc/en/html/DS.html\#DSOC}{4.2.2. Openers and Closers} \xref{http://www.tei-c.org/release/doc/tei-p5-doc/en/html/DS.html\#DSFRONT}{4.5. Front Matter}]\end{specHead} 
    \item[{Module}]
  textstructure
    \item[{Attributes}]
  Attributes att.global (\textit{@xml:id}, \textit{@n}, \textit{@xml:lang}, \textit{@xml:base}, \textit{@xml:space})  (att.global.rendition (\textit{@rend}, \textit{@style}, \textit{@rendition})) (att.global.linking (\textit{@corresp}, \textit{@synch}, \textit{@sameAs}, \textit{@copyOf}, \textit{@next}, \textit{@prev}, \textit{@exclude}, \textit{@select})) (att.global.analytic (\textit{@ana})) (att.global.facs (\textit{@facs})) (att.global.change (\textit{@change})) (att.global.responsibility (\textit{@cert}, \textit{@resp})) (att.global.source (\textit{@source}))
    \item[{Member of}]
  model.divWrapper model.pLike.front model.titlepagePart 
    \item[{Contained by}]
  
    \item[core: ]
   lg list\par 
    \item[figures: ]
   figure table\par 
    \item[msdescription: ]
   msItem\par 
    \item[textstructure: ]
   back body div front group opener titlePage
    \item[{May contain}]
  
    \item[analysis: ]
   c cl interp interpGrp m pc phr s span spanGrp w\par 
    \item[core: ]
   abbr add address cb choice corr date del distinct email emph expan foreign gap gb gloss graphic hi index lb measure measureGrp media mentioned milestone name note num orig pb ptr ref reg rs sic soCalled term time title unclear\par 
    \item[figures: ]
   figure formula notatedMusic\par 
    \item[gaiji: ]
   g\par 
    \item[header: ]
   idno\par 
    \item[linking: ]
   alt altGrp anchor join joinGrp link linkGrp seg timeline\par 
    \item[msdescription: ]
   catchwords depth dim dimensions height heraldry locus locusGrp material objectType origDate origPlace secFol signatures stamp watermark width\par 
    \item[namesdates: ]
   addName affiliation bloc climate country district forename genName geo geogFeat geogName location nameLink offset orgName persName placeName population region roleName settlement state surname terrain trait\par 
    \item[textcrit: ]
   app witDetail\par 
    \item[textstructure: ]
   docAuthor\par 
    \item[transcr: ]
   addSpan am damage damageSpan delSpan ex fw handShift listTranspose metamark mod redo restore retrace secl space subst substJoin supplied surplus undo\par character data
    \item[{Note}]
  \par
The byline on a title page may include either the name or a description for the document's author. Where the name is included, it may optionally be tagged using the <docAuthor> element.
    \item[{Example}]
  \leavevmode\bgroup\exampleFont \begin{shaded}\noindent\mbox{}{<\textbf{byline}>}Written by a CITIZEN who continued all the\mbox{}\newline 
 while in London. Never made publick before.{</\textbf{byline}>}\end{shaded}\egroup 


    \item[{Example}]
  \leavevmode\bgroup\exampleFont \begin{shaded}\noindent\mbox{}{<\textbf{byline}>}Written from her own MEMORANDUMS{</\textbf{byline}>}\end{shaded}\egroup 


    \item[{Example}]
  \leavevmode\bgroup\exampleFont \begin{shaded}\noindent\mbox{}{<\textbf{byline}>}By George Jones, Political Editor, in Washington{</\textbf{byline}>}\end{shaded}\egroup 


    \item[{Example}]
  \leavevmode\bgroup\exampleFont \begin{shaded}\noindent\mbox{}{<\textbf{byline}>}BY\mbox{}\newline 
{<\textbf{docAuthor}>}THOMAS PHILIPOTT,{</\textbf{docAuthor}>}\mbox{}\newline 
 Master of Arts,\mbox{}\newline 
 (Somtimes)\mbox{}\newline 
 Of Clare-Hall in Cambridge.{</\textbf{byline}>}\end{shaded}\egroup 


    \item[{Content model}]
  \mbox{}\hfill\\[-10pt]\begin{Verbatim}[fontsize=\small]
<content>
 <alternate maxOccurs="unbounded"
  minOccurs="0">
  <textNode/>
  <classRef key="model.gLike"/>
  <classRef key="model.phrase"/>
  <elementRef key="docAuthor"/>
  <classRef key="model.global"/>
 </alternate>
</content>
    
\end{Verbatim}

    \item[{Schema Declaration}]
  \mbox{}\hfill\\[-10pt]\begin{Verbatim}[fontsize=\small]
element byline
{
   att.global.attributes,
   ( text | model.gLike | model.phrase | docAuthor | model.global )*
}
\end{Verbatim}

\end{reflist}  \index{c=<c>|oddindex}
\begin{reflist}
\item[]\begin{specHead}{TEI.c}{<c> }(character) represents a character. [\xref{http://www.tei-c.org/release/doc/tei-p5-doc/en/html/AI.html\#AILC}{17.1. Linguistic Segment Categories}]\end{specHead} 
    \item[{Module}]
  analysis
    \item[{Attributes}]
  Attributes att.global (\textit{@xml:id}, \textit{@n}, \textit{@xml:lang}, \textit{@xml:base}, \textit{@xml:space})  (att.global.rendition (\textit{@rend}, \textit{@style}, \textit{@rendition})) (att.global.linking (\textit{@corresp}, \textit{@synch}, \textit{@sameAs}, \textit{@copyOf}, \textit{@next}, \textit{@prev}, \textit{@exclude}, \textit{@select})) (att.global.analytic (\textit{@ana})) (att.global.facs (\textit{@facs})) (att.global.change (\textit{@change})) (att.global.responsibility (\textit{@cert}, \textit{@resp})) (att.global.source (\textit{@source})) att.segLike (\textit{@function})  (att.datcat (\textit{@datcat}, \textit{@valueDatcat})) (att.fragmentable (\textit{@part})) att.typed (\textit{@type}, \textit{@subtype}) 
    \item[{Member of}]
  model.linePart model.segLike 
    \item[{Contained by}]
  
    \item[analysis: ]
   cl m pc phr s w\par 
    \item[core: ]
   abbr add addrLine author bibl biblScope citedRange corr date del distinct editor email emph expan foreign gloss head headItem headLabel hi item l label measure mentioned name note num orig p pubPlace publisher q quote ref reg rs said sic soCalled speaker stage street term textLang time title unclear\par 
    \item[figures: ]
   cell\par 
    \item[header: ]
   change distributor edition extent geoDecl handNote licence scriptNote typeNote\par 
    \item[linking: ]
   ab seg\par 
    \item[msdescription: ]
   accMat acquisition additions catchwords collation colophon condition custEvent decoNote explicit filiation finalRubric foliation heraldry incipit layout material musicNotation objectType origDate origPlace origin provenance rubric secFol signatures source stamp summary support surrogates watermark\par 
    \item[namesdates: ]
   addName affiliation birth bloc country death district education faith floruit forename genName geogFeat geogName nameLink nationality occupation offset orgName persName placeName region residence roleName settlement sex socecStatus surname\par 
    \item[textcrit: ]
   lem rdg wit witDetail\par 
    \item[textstructure: ]
   byline closer dateline docAuthor docDate docEdition docImprint imprimatur opener salute signed titlePart trailer\par 
    \item[transcr: ]
   damage fw line metamark mod restore retrace secl supplied surplus zone
    \item[{May contain}]
  
    \item[gaiji: ]
   g\par character data
    \item[{Note}]
  \par
Contains a single character, a <g> element, or a sequence of graphemes to be treated as a single character. The {\itshape type} attribute is used to indicate the function of this segmentation, taking values such as letter, punctuation, or digit etc.
    \item[{Example}]
  \leavevmode\bgroup\exampleFont \begin{shaded}\noindent\mbox{}{<\textbf{phr}>}\mbox{}\newline 
\hspace*{6pt}{<\textbf{c}>}M{</\textbf{c}>}\mbox{}\newline 
\hspace*{6pt}{<\textbf{c}>}O{</\textbf{c}>}\mbox{}\newline 
\hspace*{6pt}{<\textbf{c}>}A{</\textbf{c}>}\mbox{}\newline 
\hspace*{6pt}{<\textbf{c}>}I{</\textbf{c}>}\mbox{}\newline 
\hspace*{6pt}{<\textbf{w}>}doth{</\textbf{w}>}\mbox{}\newline 
\hspace*{6pt}{<\textbf{w}>}sway{</\textbf{w}>}\mbox{}\newline 
\hspace*{6pt}{<\textbf{w}>}my{</\textbf{w}>}\mbox{}\newline 
\hspace*{6pt}{<\textbf{w}>}life{</\textbf{w}>}\mbox{}\newline 
{</\textbf{phr}>}\end{shaded}\egroup 


    \item[{Content model}]
  \fbox{\ttfamily <content>\newline
 <macroRef key="macro.xtext"/>\newline
</content>\newline
    } 
    \item[{Schema Declaration}]
  \mbox{}\hfill\\[-10pt]\begin{Verbatim}[fontsize=\small]
element c
{
   att.global.attributes,
   att.segLike.attributes,
   att.typed.attributes,
   macro.xtext}
\end{Verbatim}

\end{reflist}  \index{cRefPattern=<cRefPattern>|oddindex}
\begin{reflist}
\item[]\begin{specHead}{TEI.cRefPattern}{<cRefPattern> }(canonical reference pattern) specifies an expression and replacement pattern for transforming a canonical reference into a URI. [\xref{http://www.tei-c.org/release/doc/tei-p5-doc/en/html/HD.html\#HD54M}{2.3.6.3. Milestone Method} \xref{http://www.tei-c.org/release/doc/tei-p5-doc/en/html/HD.html\#HD54}{2.3.6. The Reference System Declaration} \xref{http://www.tei-c.org/release/doc/tei-p5-doc/en/html/HD.html\#HD54S}{2.3.6.2. Search-and-Replace Method}]\end{specHead} 
    \item[{Module}]
  header
    \item[{Attributes}]
  Attributes att.global (\textit{@xml:id}, \textit{@n}, \textit{@xml:lang}, \textit{@xml:base}, \textit{@xml:space})  (att.global.rendition (\textit{@rend}, \textit{@style}, \textit{@rendition})) (att.global.linking (\textit{@corresp}, \textit{@synch}, \textit{@sameAs}, \textit{@copyOf}, \textit{@next}, \textit{@prev}, \textit{@exclude}, \textit{@select})) (att.global.analytic (\textit{@ana})) (att.global.facs (\textit{@facs})) (att.global.change (\textit{@change})) (att.global.responsibility (\textit{@cert}, \textit{@resp})) (att.global.source (\textit{@source})) att.patternReplacement (\textit{@matchPattern}, \textit{@replacementPattern}) 
    \item[{Contained by}]
  
    \item[header: ]
   refsDecl
    \item[{May contain}]
  
    \item[core: ]
   p\par 
    \item[linking: ]
   ab
    \item[{Note}]
  \par
The result of the substitution may be either an absolute or a relative URI reference. In the latter case it is combined with the value of {\itshape xml:base} in force at the place where the {\itshape cRef} attribute occurs to form an absolute URI in the usual manner as prescribed by \xref{http://www.w3.org/TR/xmlbase/}{XML Base}.
    \item[{Example}]
  \leavevmode\bgroup\exampleFont \begin{shaded}\noindent\mbox{}{<\textbf{cRefPattern}\hspace*{6pt}{matchPattern}="{([1-9A-Za-z]+)⃥s+([0-9]+):([0-9]+)}"\mbox{}\newline 
\hspace*{6pt}{replacementPattern}="{\#xpath(//div[@type='book'][@n='\$1']/div[@type='chap'][@n='\$2']/div[@type='verse'][@n='\$3'])}"/>}\end{shaded}\egroup 


    \item[{Content model}]
  \mbox{}\hfill\\[-10pt]\begin{Verbatim}[fontsize=\small]
<content>
 <classRef key="model.pLike"
  maxOccurs="unbounded" minOccurs="0"/>
</content>
    
\end{Verbatim}

    \item[{Schema Declaration}]
  \mbox{}\hfill\\[-10pt]\begin{Verbatim}[fontsize=\small]
element cRefPattern
{
   att.global.attributes,
   att.patternReplacement.attributes,
   model.pLike*
}
\end{Verbatim}

\end{reflist}  \index{calendar=<calendar>|oddindex}
\begin{reflist}
\item[]\begin{specHead}{TEI.calendar}{<calendar> }describes a calendar or dating system used in a dating formula in the text. [\xref{http://www.tei-c.org/release/doc/tei-p5-doc/en/html/HD.html\#HD44}{2.4.5. Calendar Description}]\end{specHead} 
    \item[{Module}]
  header
    \item[{Attributes}]
  Attributes att.global (\textit{@xml:id}, \textit{@n}, \textit{@xml:lang}, \textit{@xml:base}, \textit{@xml:space})  (att.global.rendition (\textit{@rend}, \textit{@style}, \textit{@rendition})) (att.global.linking (\textit{@corresp}, \textit{@synch}, \textit{@sameAs}, \textit{@copyOf}, \textit{@next}, \textit{@prev}, \textit{@exclude}, \textit{@select})) (att.global.analytic (\textit{@ana})) (att.global.facs (\textit{@facs})) (att.global.change (\textit{@change})) (att.global.responsibility (\textit{@cert}, \textit{@resp})) (att.global.source (\textit{@source})) att.pointing (\textit{@targetLang}, \textit{@target}, \textit{@evaluate}) 
    \item[{Contained by}]
  
    \item[header: ]
   calendarDesc
    \item[{May contain}]
  
    \item[core: ]
   p\par 
    \item[linking: ]
   ab
    \item[{Example}]
  \leavevmode\bgroup\exampleFont \begin{shaded}\noindent\mbox{}{<\textbf{calendarDesc}>}\mbox{}\newline 
\hspace*{6pt}{<\textbf{calendar}\hspace*{6pt}{xml:id}="{julianEngland}">}\mbox{}\newline 
\hspace*{6pt}\hspace*{6pt}{<\textbf{p}>}Julian Calendar (including proleptic){</\textbf{p}>}\mbox{}\newline 
\hspace*{6pt}{</\textbf{calendar}>}\mbox{}\newline 
{</\textbf{calendarDesc}>}\end{shaded}\egroup 


    \item[{Example}]
  \leavevmode\bgroup\exampleFont \begin{shaded}\noindent\mbox{}{<\textbf{calendarDesc}>}\mbox{}\newline 
\hspace*{6pt}{<\textbf{calendar}\hspace*{6pt}{target}="{http://en.wikipedia.org/wiki/Egyptian\textunderscore calendar}"\mbox{}\newline 
\hspace*{6pt}\hspace*{6pt}{xml:id}="{egyptian}">}\mbox{}\newline 
\hspace*{6pt}\hspace*{6pt}{<\textbf{p}>}Egyptian calendar (as defined by Wikipedia){</\textbf{p}>}\mbox{}\newline 
\hspace*{6pt}{</\textbf{calendar}>}\mbox{}\newline 
{</\textbf{calendarDesc}>}\end{shaded}\egroup 


    \item[{Content model}]
  \mbox{}\hfill\\[-10pt]\begin{Verbatim}[fontsize=\small]
<content>
 <classRef key="model.pLike"
  maxOccurs="unbounded" minOccurs="1"/>
</content>
    
\end{Verbatim}

    \item[{Schema Declaration}]
  \mbox{}\hfill\\[-10pt]\begin{Verbatim}[fontsize=\small]
element calendar
{
   att.global.attributes,
   att.pointing.attributes,
   model.pLike+
}
\end{Verbatim}

\end{reflist}  \index{calendarDesc=<calendarDesc>|oddindex}
\begin{reflist}
\item[]\begin{specHead}{TEI.calendarDesc}{<calendarDesc> }(calendar description) contains a description of the calendar system used in any dating expression found in the text. [\xref{http://www.tei-c.org/release/doc/tei-p5-doc/en/html/HD.html\#HD4}{2.4. The Profile Description} \xref{http://www.tei-c.org/release/doc/tei-p5-doc/en/html/HD.html\#HD44}{2.4.5. Calendar Description}]\end{specHead} 
    \item[{Module}]
  header
    \item[{Attributes}]
  Attributes att.global (\textit{@xml:id}, \textit{@n}, \textit{@xml:lang}, \textit{@xml:base}, \textit{@xml:space})  (att.global.rendition (\textit{@rend}, \textit{@style}, \textit{@rendition})) (att.global.linking (\textit{@corresp}, \textit{@synch}, \textit{@sameAs}, \textit{@copyOf}, \textit{@next}, \textit{@prev}, \textit{@exclude}, \textit{@select})) (att.global.analytic (\textit{@ana})) (att.global.facs (\textit{@facs})) (att.global.change (\textit{@change})) (att.global.responsibility (\textit{@cert}, \textit{@resp})) (att.global.source (\textit{@source}))
    \item[{Member of}]
  model.profileDescPart
    \item[{Contained by}]
  
    \item[header: ]
   profileDesc
    \item[{May contain}]
  
    \item[header: ]
   calendar
    \item[{Note}]
  \par
In the first example above, calendars and short codes for {\itshape xml:id}s are from W3 guidelines at  {\ref http://www.w3.org/TR/xpath-functions-11/\#lang-cal-country}
    \item[{Example}]
  \leavevmode\bgroup\exampleFont \begin{shaded}\noindent\mbox{}{<\textbf{calendarDesc}>}\mbox{}\newline 
\hspace*{6pt}{<\textbf{calendar}\hspace*{6pt}{xml:id}="{cal\textunderscore AD}">}\mbox{}\newline 
\hspace*{6pt}\hspace*{6pt}{<\textbf{p}>}Anno Domini (Christian Era){</\textbf{p}>}\mbox{}\newline 
\hspace*{6pt}{</\textbf{calendar}>}\mbox{}\newline 
\hspace*{6pt}{<\textbf{calendar}\hspace*{6pt}{xml:id}="{cal\textunderscore AH}">}\mbox{}\newline 
\hspace*{6pt}\hspace*{6pt}{<\textbf{p}>}Anno Hegirae (Muhammedan Era){</\textbf{p}>}\mbox{}\newline 
\hspace*{6pt}{</\textbf{calendar}>}\mbox{}\newline 
\hspace*{6pt}{<\textbf{calendar}\hspace*{6pt}{xml:id}="{cal\textunderscore AME}">}\mbox{}\newline 
\hspace*{6pt}\hspace*{6pt}{<\textbf{p}>}Mauludi Era (solar years since Mohammed's birth){</\textbf{p}>}\mbox{}\newline 
\hspace*{6pt}{</\textbf{calendar}>}\mbox{}\newline 
\hspace*{6pt}{<\textbf{calendar}\hspace*{6pt}{xml:id}="{cal\textunderscore AM}">}\mbox{}\newline 
\hspace*{6pt}\hspace*{6pt}{<\textbf{p}>}Anno Mundi (Jewish Calendar){</\textbf{p}>}\mbox{}\newline 
\hspace*{6pt}{</\textbf{calendar}>}\mbox{}\newline 
\hspace*{6pt}{<\textbf{calendar}\hspace*{6pt}{xml:id}="{cal\textunderscore AP}">}\mbox{}\newline 
\hspace*{6pt}\hspace*{6pt}{<\textbf{p}>}Anno Persici{</\textbf{p}>}\mbox{}\newline 
\hspace*{6pt}{</\textbf{calendar}>}\mbox{}\newline 
\hspace*{6pt}{<\textbf{calendar}\hspace*{6pt}{xml:id}="{cal\textunderscore AS}">}\mbox{}\newline 
\hspace*{6pt}\hspace*{6pt}{<\textbf{p}>}Aji Saka Era (Java){</\textbf{p}>}\mbox{}\newline 
\hspace*{6pt}{</\textbf{calendar}>}\mbox{}\newline 
\hspace*{6pt}{<\textbf{calendar}\hspace*{6pt}{xml:id}="{cal\textunderscore BE}">}\mbox{}\newline 
\hspace*{6pt}\hspace*{6pt}{<\textbf{p}>}Buddhist Era{</\textbf{p}>}\mbox{}\newline 
\hspace*{6pt}{</\textbf{calendar}>}\mbox{}\newline 
\hspace*{6pt}{<\textbf{calendar}\hspace*{6pt}{xml:id}="{cal\textunderscore CB}">}\mbox{}\newline 
\hspace*{6pt}\hspace*{6pt}{<\textbf{p}>}Cooch Behar Era{</\textbf{p}>}\mbox{}\newline 
\hspace*{6pt}{</\textbf{calendar}>}\mbox{}\newline 
\hspace*{6pt}{<\textbf{calendar}\hspace*{6pt}{xml:id}="{cal\textunderscore CE}">}\mbox{}\newline 
\hspace*{6pt}\hspace*{6pt}{<\textbf{p}>}Common Era{</\textbf{p}>}\mbox{}\newline 
\hspace*{6pt}{</\textbf{calendar}>}\mbox{}\newline 
\hspace*{6pt}{<\textbf{calendar}\hspace*{6pt}{xml:id}="{cal\textunderscore CL}">}\mbox{}\newline 
\hspace*{6pt}\hspace*{6pt}{<\textbf{p}>}Chinese Lunar Era{</\textbf{p}>}\mbox{}\newline 
\hspace*{6pt}{</\textbf{calendar}>}\mbox{}\newline 
\hspace*{6pt}{<\textbf{calendar}\hspace*{6pt}{xml:id}="{cal\textunderscore CS}">}\mbox{}\newline 
\hspace*{6pt}\hspace*{6pt}{<\textbf{p}>}Chula Sakarat Era{</\textbf{p}>}\mbox{}\newline 
\hspace*{6pt}{</\textbf{calendar}>}\mbox{}\newline 
\hspace*{6pt}{<\textbf{calendar}\hspace*{6pt}{xml:id}="{cal\textunderscore EE}">}\mbox{}\newline 
\hspace*{6pt}\hspace*{6pt}{<\textbf{p}>}Ethiopian Era{</\textbf{p}>}\mbox{}\newline 
\hspace*{6pt}{</\textbf{calendar}>}\mbox{}\newline 
\hspace*{6pt}{<\textbf{calendar}\hspace*{6pt}{xml:id}="{cal\textunderscore FE}">}\mbox{}\newline 
\hspace*{6pt}\hspace*{6pt}{<\textbf{p}>}Fasli Era{</\textbf{p}>}\mbox{}\newline 
\hspace*{6pt}{</\textbf{calendar}>}\mbox{}\newline 
\hspace*{6pt}{<\textbf{calendar}\hspace*{6pt}{xml:id}="{cal\textunderscore ISO}">}\mbox{}\newline 
\hspace*{6pt}\hspace*{6pt}{<\textbf{p}>}ISO 8601 calendar{</\textbf{p}>}\mbox{}\newline 
\hspace*{6pt}{</\textbf{calendar}>}\mbox{}\newline 
\hspace*{6pt}{<\textbf{calendar}\hspace*{6pt}{xml:id}="{cal\textunderscore JE}">}\mbox{}\newline 
\hspace*{6pt}\hspace*{6pt}{<\textbf{p}>}Japanese Calendar{</\textbf{p}>}\mbox{}\newline 
\hspace*{6pt}{</\textbf{calendar}>}\mbox{}\newline 
\hspace*{6pt}{<\textbf{calendar}\hspace*{6pt}{xml:id}="{cal\textunderscore KE}">}\mbox{}\newline 
\hspace*{6pt}\hspace*{6pt}{<\textbf{p}>}Khalsa Era (Sikh calendar){</\textbf{p}>}\mbox{}\newline 
\hspace*{6pt}{</\textbf{calendar}>}\mbox{}\newline 
\hspace*{6pt}{<\textbf{calendar}\hspace*{6pt}{xml:id}="{cal\textunderscore KY}">}\mbox{}\newline 
\hspace*{6pt}\hspace*{6pt}{<\textbf{p}>}Kali Yuga{</\textbf{p}>}\mbox{}\newline 
\hspace*{6pt}{</\textbf{calendar}>}\mbox{}\newline 
\hspace*{6pt}{<\textbf{calendar}\hspace*{6pt}{xml:id}="{cal\textunderscore ME}">}\mbox{}\newline 
\hspace*{6pt}\hspace*{6pt}{<\textbf{p}>}Malabar Era{</\textbf{p}>}\mbox{}\newline 
\hspace*{6pt}{</\textbf{calendar}>}\mbox{}\newline 
\hspace*{6pt}{<\textbf{calendar}\hspace*{6pt}{xml:id}="{cal\textunderscore MS}">}\mbox{}\newline 
\hspace*{6pt}\hspace*{6pt}{<\textbf{p}>}Monarchic Solar Era{</\textbf{p}>}\mbox{}\newline 
\hspace*{6pt}{</\textbf{calendar}>}\mbox{}\newline 
\hspace*{6pt}{<\textbf{calendar}\hspace*{6pt}{xml:id}="{cal\textunderscore NS}">}\mbox{}\newline 
\hspace*{6pt}\hspace*{6pt}{<\textbf{p}>}Nepal Samwat Era{</\textbf{p}>}\mbox{}\newline 
\hspace*{6pt}{</\textbf{calendar}>}\mbox{}\newline 
\hspace*{6pt}{<\textbf{calendar}\hspace*{6pt}{xml:id}="{cal\textunderscore OS}">}\mbox{}\newline 
\hspace*{6pt}\hspace*{6pt}{<\textbf{p}>}Old Style (Julian Calendar){</\textbf{p}>}\mbox{}\newline 
\hspace*{6pt}{</\textbf{calendar}>}\mbox{}\newline 
\hspace*{6pt}{<\textbf{calendar}\hspace*{6pt}{xml:id}="{cal\textunderscore RS}">}\mbox{}\newline 
\hspace*{6pt}\hspace*{6pt}{<\textbf{p}>}Rattanakosin (Bangkok) Era{</\textbf{p}>}\mbox{}\newline 
\hspace*{6pt}{</\textbf{calendar}>}\mbox{}\newline 
\hspace*{6pt}{<\textbf{calendar}\hspace*{6pt}{xml:id}="{cal\textunderscore SE}">}\mbox{}\newline 
\hspace*{6pt}\hspace*{6pt}{<\textbf{p}>}Saka Era{</\textbf{p}>}\mbox{}\newline 
\hspace*{6pt}{</\textbf{calendar}>}\mbox{}\newline 
\hspace*{6pt}{<\textbf{calendar}\hspace*{6pt}{xml:id}="{cal\textunderscore SH}">}\mbox{}\newline 
\hspace*{6pt}\hspace*{6pt}{<\textbf{p}>}Mohammedan Solar Era (Iran){</\textbf{p}>}\mbox{}\newline 
\hspace*{6pt}{</\textbf{calendar}>}\mbox{}\newline 
\hspace*{6pt}{<\textbf{calendar}\hspace*{6pt}{xml:id}="{cal\textunderscore SS}">}\mbox{}\newline 
\hspace*{6pt}\hspace*{6pt}{<\textbf{p}>}Saka Samvat{</\textbf{p}>}\mbox{}\newline 
\hspace*{6pt}{</\textbf{calendar}>}\mbox{}\newline 
\hspace*{6pt}{<\textbf{calendar}\hspace*{6pt}{xml:id}="{cal\textunderscore TE}">}\mbox{}\newline 
\hspace*{6pt}\hspace*{6pt}{<\textbf{p}>}Tripurabda Era{</\textbf{p}>}\mbox{}\newline 
\hspace*{6pt}{</\textbf{calendar}>}\mbox{}\newline 
\hspace*{6pt}{<\textbf{calendar}\hspace*{6pt}{xml:id}="{cal\textunderscore VE}">}\mbox{}\newline 
\hspace*{6pt}\hspace*{6pt}{<\textbf{p}>}Vikrama Era{</\textbf{p}>}\mbox{}\newline 
\hspace*{6pt}{</\textbf{calendar}>}\mbox{}\newline 
\hspace*{6pt}{<\textbf{calendar}\hspace*{6pt}{xml:id}="{cal\textunderscore VS}">}\mbox{}\newline 
\hspace*{6pt}\hspace*{6pt}{<\textbf{p}>}Vikrama Samvat Era{</\textbf{p}>}\mbox{}\newline 
\hspace*{6pt}{</\textbf{calendar}>}\mbox{}\newline 
{</\textbf{calendarDesc}>}\end{shaded}\egroup 


    \item[{Example}]
  \leavevmode\bgroup\exampleFont \begin{shaded}\noindent\mbox{}{<\textbf{calendarDesc}>}\mbox{}\newline 
\hspace*{6pt}{<\textbf{calendar}\hspace*{6pt}{xml:id}="{cal\textunderscore Gregorian}">}\mbox{}\newline 
\hspace*{6pt}\hspace*{6pt}{<\textbf{p}>}Gregorian calendar{</\textbf{p}>}\mbox{}\newline 
\hspace*{6pt}{</\textbf{calendar}>}\mbox{}\newline 
\hspace*{6pt}{<\textbf{calendar}\hspace*{6pt}{xml:id}="{cal\textunderscore Julian}">}\mbox{}\newline 
\hspace*{6pt}\hspace*{6pt}{<\textbf{p}>}Julian calendar{</\textbf{p}>}\mbox{}\newline 
\hspace*{6pt}{</\textbf{calendar}>}\mbox{}\newline 
\hspace*{6pt}{<\textbf{calendar}\hspace*{6pt}{xml:id}="{cal\textunderscore Islamic}">}\mbox{}\newline 
\hspace*{6pt}\hspace*{6pt}{<\textbf{p}>}Islamic or Muslim (hijri) lunar calendar{</\textbf{p}>}\mbox{}\newline 
\hspace*{6pt}{</\textbf{calendar}>}\mbox{}\newline 
\hspace*{6pt}{<\textbf{calendar}\hspace*{6pt}{xml:id}="{cal\textunderscore Hebrew}">}\mbox{}\newline 
\hspace*{6pt}\hspace*{6pt}{<\textbf{p}>}Hebrew or Jewish lunisolar calendar{</\textbf{p}>}\mbox{}\newline 
\hspace*{6pt}{</\textbf{calendar}>}\mbox{}\newline 
\hspace*{6pt}{<\textbf{calendar}\hspace*{6pt}{xml:id}="{cal\textunderscore Revolutionary}">}\mbox{}\newline 
\hspace*{6pt}\hspace*{6pt}{<\textbf{p}>}French Revolutionary calendar{</\textbf{p}>}\mbox{}\newline 
\hspace*{6pt}{</\textbf{calendar}>}\mbox{}\newline 
\hspace*{6pt}{<\textbf{calendar}\hspace*{6pt}{xml:id}="{cal\textunderscore Iranian}">}\mbox{}\newline 
\hspace*{6pt}\hspace*{6pt}{<\textbf{p}>}Iranian or Persian (Jalaali) solar calendar{</\textbf{p}>}\mbox{}\newline 
\hspace*{6pt}{</\textbf{calendar}>}\mbox{}\newline 
\hspace*{6pt}{<\textbf{calendar}\hspace*{6pt}{xml:id}="{cal\textunderscore Coptic}">}\mbox{}\newline 
\hspace*{6pt}\hspace*{6pt}{<\textbf{p}>}Coptic or Alexandrian calendar{</\textbf{p}>}\mbox{}\newline 
\hspace*{6pt}{</\textbf{calendar}>}\mbox{}\newline 
\hspace*{6pt}{<\textbf{calendar}\hspace*{6pt}{xml:id}="{cal\textunderscore Chinese}">}\mbox{}\newline 
\hspace*{6pt}\hspace*{6pt}{<\textbf{p}>}Chinese lunisolar calendar{</\textbf{p}>}\mbox{}\newline 
\hspace*{6pt}{</\textbf{calendar}>}\mbox{}\newline 
{</\textbf{calendarDesc}>}\end{shaded}\egroup 


    \item[{Example}]
  \leavevmode\bgroup\exampleFont \begin{shaded}\noindent\mbox{}{<\textbf{calendarDesc}>}\mbox{}\newline 
\hspace*{6pt}{<\textbf{calendar}\hspace*{6pt}{target}="{http://en.wikipedia.org/wiki/Egyptian\textunderscore calendar}"\mbox{}\newline 
\hspace*{6pt}\hspace*{6pt}{xml:id}="{cal\textunderscore Egyptian}">}\mbox{}\newline 
\hspace*{6pt}\hspace*{6pt}{<\textbf{p}>}Egyptian calendar (as defined by Wikipedia){</\textbf{p}>}\mbox{}\newline 
\hspace*{6pt}{</\textbf{calendar}>}\mbox{}\newline 
{</\textbf{calendarDesc}>}\end{shaded}\egroup 


    \item[{Content model}]
  \mbox{}\hfill\\[-10pt]\begin{Verbatim}[fontsize=\small]
<content>
 <elementRef key="calendar"
  maxOccurs="unbounded" minOccurs="1"/>
</content>
    
\end{Verbatim}

    \item[{Schema Declaration}]
  \mbox{}\hfill\\[-10pt]\begin{Verbatim}[fontsize=\small]
element calendarDesc { att.global.attributes, calendar+ }
\end{Verbatim}

\end{reflist}  \index{catDesc=<catDesc>|oddindex}
\begin{reflist}
\item[]\begin{specHead}{TEI.catDesc}{<catDesc> }(category description) describes some category within a taxonomy or text typology, either in the form of a brief prose description or in terms of the situational parameters used by the TEI formal \texttt{<textDesc>}. [\xref{http://www.tei-c.org/release/doc/tei-p5-doc/en/html/HD.html\#HD55}{2.3.7. The Classification Declaration}]\end{specHead} 
    \item[{Module}]
  header
    \item[{Attributes}]
  Attributes att.global (\textit{@xml:id}, \textit{@n}, \textit{@xml:lang}, \textit{@xml:base}, \textit{@xml:space})  (att.global.rendition (\textit{@rend}, \textit{@style}, \textit{@rendition})) (att.global.linking (\textit{@corresp}, \textit{@synch}, \textit{@sameAs}, \textit{@copyOf}, \textit{@next}, \textit{@prev}, \textit{@exclude}, \textit{@select})) (att.global.analytic (\textit{@ana})) (att.global.facs (\textit{@facs})) (att.global.change (\textit{@change})) (att.global.responsibility (\textit{@cert}, \textit{@resp})) (att.global.source (\textit{@source}))
    \item[{Contained by}]
  
    \item[header: ]
   category
    \item[{May contain}]
  
    \item[core: ]
   abbr address choice date distinct email emph expan foreign gloss hi measure measureGrp mentioned name num ptr ref rs soCalled term time title\par 
    \item[header: ]
   idno\par 
    \item[msdescription: ]
   catchwords depth dim dimensions height heraldry locus locusGrp material objectType origDate origPlace secFol signatures stamp watermark width\par 
    \item[namesdates: ]
   addName affiliation bloc climate country district forename genName geo geogFeat geogName location nameLink offset orgName persName placeName population region roleName settlement state surname terrain trait\par 
    \item[transcr: ]
   am ex subst\par character data
    \item[{Example}]
  \leavevmode\bgroup\exampleFont \begin{shaded}\noindent\mbox{}{<\textbf{catDesc}>}Prose reportage{</\textbf{catDesc}>}\end{shaded}\egroup 


    \item[{Example}]
  \leavevmode\bgroup\exampleFont \begin{shaded}\noindent\mbox{}{<\textbf{catDesc}>}\mbox{}\newline 
\hspace*{6pt}{<\textbf{textDesc}\hspace*{6pt}{n}="{novel}">}\mbox{}\newline 
\hspace*{6pt}\hspace*{6pt}{<\textbf{channel}\hspace*{6pt}{mode}="{w}">}print; part issues{</\textbf{channel}>}\mbox{}\newline 
\hspace*{6pt}\hspace*{6pt}{<\textbf{constitution}\hspace*{6pt}{type}="{single}"/>}\mbox{}\newline 
\hspace*{6pt}\hspace*{6pt}{<\textbf{derivation}\hspace*{6pt}{type}="{original}"/>}\mbox{}\newline 
\hspace*{6pt}\hspace*{6pt}{<\textbf{domain}\hspace*{6pt}{type}="{art}"/>}\mbox{}\newline 
\hspace*{6pt}\hspace*{6pt}{<\textbf{factuality}\hspace*{6pt}{type}="{fiction}"/>}\mbox{}\newline 
\hspace*{6pt}\hspace*{6pt}{<\textbf{interaction}\hspace*{6pt}{type}="{none}"/>}\mbox{}\newline 
\hspace*{6pt}\hspace*{6pt}{<\textbf{preparedness}\hspace*{6pt}{type}="{prepared}"/>}\mbox{}\newline 
\hspace*{6pt}\hspace*{6pt}{<\textbf{purpose}\hspace*{6pt}{degree}="{high}"\hspace*{6pt}{type}="{entertain}"/>}\mbox{}\newline 
\hspace*{6pt}\hspace*{6pt}{<\textbf{purpose}\hspace*{6pt}{degree}="{medium}"\hspace*{6pt}{type}="{inform}"/>}\mbox{}\newline 
\hspace*{6pt}{</\textbf{textDesc}>}\mbox{}\newline 
{</\textbf{catDesc}>}\end{shaded}\egroup 


    \item[{Content model}]
  \mbox{}\hfill\\[-10pt]\begin{Verbatim}[fontsize=\small]
<content>
 <alternate maxOccurs="unbounded"
  minOccurs="0">
  <textNode/>
  <classRef key="model.limitedPhrase"/>
  <classRef key="model.catDescPart"/>
 </alternate>
</content>
    
\end{Verbatim}

    \item[{Schema Declaration}]
  \mbox{}\hfill\\[-10pt]\begin{Verbatim}[fontsize=\small]
element catDesc
{
   att.global.attributes,
   ( text | model.limitedPhrase | model.catDescPart )*
}
\end{Verbatim}

\end{reflist}  \index{catRef=<catRef>|oddindex}\index{scheme=@scheme!<catRef>|oddindex}
\begin{reflist}
\item[]\begin{specHead}{TEI.catRef}{<catRef> }(category reference) specifies one or more defined categories within some taxonomy or text typology. [\xref{http://www.tei-c.org/release/doc/tei-p5-doc/en/html/HD.html\#HD43}{2.4.3. The Text Classification}]\end{specHead} 
    \item[{Module}]
  header
    \item[{Attributes}]
  Attributes att.global (\textit{@xml:id}, \textit{@n}, \textit{@xml:lang}, \textit{@xml:base}, \textit{@xml:space})  (att.global.rendition (\textit{@rend}, \textit{@style}, \textit{@rendition})) (att.global.linking (\textit{@corresp}, \textit{@synch}, \textit{@sameAs}, \textit{@copyOf}, \textit{@next}, \textit{@prev}, \textit{@exclude}, \textit{@select})) (att.global.analytic (\textit{@ana})) (att.global.facs (\textit{@facs})) (att.global.change (\textit{@change})) (att.global.responsibility (\textit{@cert}, \textit{@resp})) (att.global.source (\textit{@source})) att.pointing (\textit{@targetLang}, \textit{@target}, \textit{@evaluate}) \hfil\\[-10pt]\begin{sansreflist}
    \item[@scheme]
  identifies the classification scheme within which the set of categories concerned is defined, for example by a <taxonomy> element, or by some other resource.
\begin{reflist}
    \item[{Status}]
  Optional
    \item[{Datatype}]
  teidata.pointer
\end{reflist}  
\end{sansreflist}  
    \item[{Contained by}]
  
    \item[core: ]
   imprint\par 
    \item[header: ]
   textClass
    \item[{May contain}]
  Empty element
    \item[{Note}]
  \par
The {\itshape scheme} attribute needs to be supplied only if more than one taxonomy has been declared.
    \item[{Example}]
  \leavevmode\bgroup\exampleFont \begin{shaded}\noindent\mbox{}{<\textbf{catRef}\hspace*{6pt}{scheme}="{\#myTopics}"\mbox{}\newline 
\hspace*{6pt}{target}="{\#news \#prov \#sales2}"/>}\mbox{}\newline 
\textit{<!-- elsewhere -->}\mbox{}\newline 
{<\textbf{taxonomy}\hspace*{6pt}{xml:id}="{myTopics}">}\mbox{}\newline 
\hspace*{6pt}{<\textbf{category}\hspace*{6pt}{xml:id}="{news}">}\mbox{}\newline 
\hspace*{6pt}\hspace*{6pt}{<\textbf{catDesc}>}Newspapers{</\textbf{catDesc}>}\mbox{}\newline 
\hspace*{6pt}{</\textbf{category}>}\mbox{}\newline 
\hspace*{6pt}{<\textbf{category}\hspace*{6pt}{xml:id}="{prov}">}\mbox{}\newline 
\hspace*{6pt}\hspace*{6pt}{<\textbf{catDesc}>}Provincial{</\textbf{catDesc}>}\mbox{}\newline 
\hspace*{6pt}{</\textbf{category}>}\mbox{}\newline 
\hspace*{6pt}{<\textbf{category}\hspace*{6pt}{xml:id}="{sales2}">}\mbox{}\newline 
\hspace*{6pt}\hspace*{6pt}{<\textbf{catDesc}>}Low to average annual sales{</\textbf{catDesc}>}\mbox{}\newline 
\hspace*{6pt}{</\textbf{category}>}\mbox{}\newline 
{</\textbf{taxonomy}>}\end{shaded}\egroup 


    \item[{Content model}]
  \fbox{\ttfamily <content>\newline
</content>\newline
    } 
    \item[{Schema Declaration}]
  \mbox{}\hfill\\[-10pt]\begin{Verbatim}[fontsize=\small]
element catRef
{
   att.global.attributes,
   att.pointing.attributes,
   attribute scheme { text }?,
   empty
}
\end{Verbatim}

\end{reflist}  \index{catchwords=<catchwords>|oddindex}
\begin{reflist}
\item[]\begin{specHead}{TEI.catchwords}{<catchwords> }describes the system used to ensure correct ordering of the quires making up a codex or incunable, typically by means of annotations at the foot of the page. [\xref{http://www.tei-c.org/release/doc/tei-p5-doc/en/html/MS.html\#msmisc}{10.3.7. Catchwords, Signatures, Secundo Folio}]\end{specHead} 
    \item[{Module}]
  msdescription
    \item[{Attributes}]
  Attributes att.global (\textit{@xml:id}, \textit{@n}, \textit{@xml:lang}, \textit{@xml:base}, \textit{@xml:space})  (att.global.rendition (\textit{@rend}, \textit{@style}, \textit{@rendition})) (att.global.linking (\textit{@corresp}, \textit{@synch}, \textit{@sameAs}, \textit{@copyOf}, \textit{@next}, \textit{@prev}, \textit{@exclude}, \textit{@select})) (att.global.analytic (\textit{@ana})) (att.global.facs (\textit{@facs})) (att.global.change (\textit{@change})) (att.global.responsibility (\textit{@cert}, \textit{@resp})) (att.global.source (\textit{@source}))
    \item[{Member of}]
  model.pPart.msdesc
    \item[{Contained by}]
  
    \item[analysis: ]
   cl phr s span\par 
    \item[core: ]
   abbr add addrLine author biblScope citedRange corr date del desc distinct editor email emph expan foreign gloss head headItem headLabel hi item l label measure meeting mentioned name note num orig p pubPlace publisher q quote ref reg resp rs said sic soCalled speaker stage street term textLang time title unclear\par 
    \item[figures: ]
   cell figDesc\par 
    \item[header: ]
   authority catDesc change classCode creation distributor edition extent funder geoDecl handNote language licence principal rendition scriptNote sponsor tagUsage typeNote\par 
    \item[linking: ]
   ab seg\par 
    \item[msdescription: ]
   accMat acquisition additions catchwords collation colophon condition custEvent decoNote explicit filiation finalRubric foliation heraldry incipit layout material musicNotation objectType origDate origPlace origin provenance rubric secFol signatures source stamp summary support surrogates watermark\par 
    \item[namesdates: ]
   addName affiliation age birth bloc country death district education faith floruit forename genName geogFeat geogName langKnown nameLink nationality occupation offset orgName persName placeName region residence roleName settlement sex socecStatus surname\par 
    \item[textcrit: ]
   lem rdg wit witDetail witness\par 
    \item[textstructure: ]
   byline closer dateline docAuthor docDate docEdition docImprint imprimatur opener salute signed titlePart trailer\par 
    \item[transcr: ]
   damage fw metamark mod restore retrace secl supplied surplus
    \item[{May contain}]
  
    \item[analysis: ]
   c cl interp interpGrp m pc phr s span spanGrp w\par 
    \item[core: ]
   abbr add address cb choice corr date del distinct email emph expan foreign gap gb gloss graphic hi index lb measure measureGrp media mentioned milestone name note num orig pb ptr ref reg rs sic soCalled term time title unclear\par 
    \item[figures: ]
   figure formula notatedMusic\par 
    \item[gaiji: ]
   g\par 
    \item[header: ]
   idno\par 
    \item[linking: ]
   alt altGrp anchor join joinGrp link linkGrp seg timeline\par 
    \item[msdescription: ]
   catchwords depth dim dimensions height heraldry locus locusGrp material objectType origDate origPlace secFol signatures stamp watermark width\par 
    \item[namesdates: ]
   addName affiliation bloc climate country district forename genName geo geogFeat geogName location nameLink offset orgName persName placeName population region roleName settlement state surname terrain trait\par 
    \item[textcrit: ]
   app witDetail\par 
    \item[transcr: ]
   addSpan am damage damageSpan delSpan ex fw handShift listTranspose metamark mod redo restore retrace secl space subst substJoin supplied surplus undo\par character data
    \item[{Example}]
  \leavevmode\bgroup\exampleFont \begin{shaded}\noindent\mbox{}{<\textbf{catchwords}>}Vertical catchwords in the hand of the scribe placed along\mbox{}\newline 
 the inner bounding line, reading from top to bottom.{</\textbf{catchwords}>}\end{shaded}\egroup 


    \item[{Schematron}]
   <sch:assert role="nonfatal"  test="ancestor::tei:msDesc">WARNING: deprecated use of element — The <sch:name/> element will not be allowed outside of msDesc as of 2018-10-01.</sch:assert>
    \item[{Content model}]
  \mbox{}\hfill\\[-10pt]\begin{Verbatim}[fontsize=\small]
<content>
 <macroRef key="macro.phraseSeq"/>
</content>
    
\end{Verbatim}

    \item[{Schema Declaration}]
  \mbox{}\hfill\\[-10pt]\begin{Verbatim}[fontsize=\small]
element catchwords { att.global.attributes, macro.phraseSeq }
\end{Verbatim}

\end{reflist}  \index{category=<category>|oddindex}
\begin{reflist}
\item[]\begin{specHead}{TEI.category}{<category> }contains an individual descriptive category, possibly nested within a superordinate category, within a user-defined taxonomy. [\xref{http://www.tei-c.org/release/doc/tei-p5-doc/en/html/HD.html\#HD55}{2.3.7. The Classification Declaration}]\end{specHead} 
    \item[{Module}]
  header
    \item[{Attributes}]
  Attributes att.global (\textit{@xml:id}, \textit{@n}, \textit{@xml:lang}, \textit{@xml:base}, \textit{@xml:space})  (att.global.rendition (\textit{@rend}, \textit{@style}, \textit{@rendition})) (att.global.linking (\textit{@corresp}, \textit{@synch}, \textit{@sameAs}, \textit{@copyOf}, \textit{@next}, \textit{@prev}, \textit{@exclude}, \textit{@select})) (att.global.analytic (\textit{@ana})) (att.global.facs (\textit{@facs})) (att.global.change (\textit{@change})) (att.global.responsibility (\textit{@cert}, \textit{@resp})) (att.global.source (\textit{@source}))
    \item[{Contained by}]
  
    \item[header: ]
   category taxonomy
    \item[{May contain}]
  
    \item[core: ]
   desc gloss\par 
    \item[header: ]
   catDesc category
    \item[{Example}]
  \leavevmode\bgroup\exampleFont \begin{shaded}\noindent\mbox{}{<\textbf{category}\hspace*{6pt}{xml:id}="{b1}">}\mbox{}\newline 
\hspace*{6pt}{<\textbf{catDesc}>}Prose reportage{</\textbf{catDesc}>}\mbox{}\newline 
{</\textbf{category}>}\end{shaded}\egroup 


    \item[{Example}]
  \leavevmode\bgroup\exampleFont \begin{shaded}\noindent\mbox{}{<\textbf{category}\hspace*{6pt}{xml:id}="{b2}">}\mbox{}\newline 
\hspace*{6pt}{<\textbf{catDesc}>}Prose {</\textbf{catDesc}>}\mbox{}\newline 
\hspace*{6pt}{<\textbf{category}\hspace*{6pt}{xml:id}="{b11}">}\mbox{}\newline 
\hspace*{6pt}\hspace*{6pt}{<\textbf{catDesc}>}journalism{</\textbf{catDesc}>}\mbox{}\newline 
\hspace*{6pt}{</\textbf{category}>}\mbox{}\newline 
\hspace*{6pt}{<\textbf{category}\hspace*{6pt}{xml:id}="{b12}">}\mbox{}\newline 
\hspace*{6pt}\hspace*{6pt}{<\textbf{catDesc}>}fiction{</\textbf{catDesc}>}\mbox{}\newline 
\hspace*{6pt}{</\textbf{category}>}\mbox{}\newline 
{</\textbf{category}>}\end{shaded}\egroup 


    \item[{Example}]
  \leavevmode\bgroup\exampleFont \begin{shaded}\noindent\mbox{}{<\textbf{category}\hspace*{6pt}{xml:id}="{LIT}">}\mbox{}\newline 
\hspace*{6pt}{<\textbf{catDesc}\hspace*{6pt}{xml:lang}="{pl}">}literatura piękna{</\textbf{catDesc}>}\mbox{}\newline 
\hspace*{6pt}{<\textbf{catDesc}\hspace*{6pt}{xml:lang}="{en}">}fiction{</\textbf{catDesc}>}\mbox{}\newline 
\hspace*{6pt}{<\textbf{category}\hspace*{6pt}{xml:id}="{LPROSE}">}\mbox{}\newline 
\hspace*{6pt}\hspace*{6pt}{<\textbf{catDesc}\hspace*{6pt}{xml:lang}="{pl}">}proza{</\textbf{catDesc}>}\mbox{}\newline 
\hspace*{6pt}\hspace*{6pt}{<\textbf{catDesc}\hspace*{6pt}{xml:lang}="{en}">}prose{</\textbf{catDesc}>}\mbox{}\newline 
\hspace*{6pt}{</\textbf{category}>}\mbox{}\newline 
\hspace*{6pt}{<\textbf{category}\hspace*{6pt}{xml:id}="{LPOETRY}">}\mbox{}\newline 
\hspace*{6pt}\hspace*{6pt}{<\textbf{catDesc}\hspace*{6pt}{xml:lang}="{pl}">}poezja{</\textbf{catDesc}>}\mbox{}\newline 
\hspace*{6pt}\hspace*{6pt}{<\textbf{catDesc}\hspace*{6pt}{xml:lang}="{en}">}poetry{</\textbf{catDesc}>}\mbox{}\newline 
\hspace*{6pt}{</\textbf{category}>}\mbox{}\newline 
\hspace*{6pt}{<\textbf{category}\hspace*{6pt}{xml:id}="{LDRAMA}">}\mbox{}\newline 
\hspace*{6pt}\hspace*{6pt}{<\textbf{catDesc}\hspace*{6pt}{xml:lang}="{pl}">}dramat{</\textbf{catDesc}>}\mbox{}\newline 
\hspace*{6pt}\hspace*{6pt}{<\textbf{catDesc}\hspace*{6pt}{xml:lang}="{en}">}drama{</\textbf{catDesc}>}\mbox{}\newline 
\hspace*{6pt}{</\textbf{category}>}\mbox{}\newline 
{</\textbf{category}>}\end{shaded}\egroup 


    \item[{Content model}]
  \mbox{}\hfill\\[-10pt]\begin{Verbatim}[fontsize=\small]
<content>
 <sequence>
  <alternate>
   <elementRef key="catDesc"
    maxOccurs="unbounded" minOccurs="1"/>
   <alternate maxOccurs="unbounded"
    minOccurs="0">
    <classRef key="model.descLike"/>
    <classRef key="model.glossLike"/>
   </alternate>
  </alternate>
  <elementRef key="category"
   maxOccurs="unbounded" minOccurs="0"/>
 </sequence>
</content>
    
\end{Verbatim}

    \item[{Schema Declaration}]
  \mbox{}\hfill\\[-10pt]\begin{Verbatim}[fontsize=\small]
element category
{
   att.global.attributes,
   ( ( catDesc+ | ( model.descLike | model.glossLike )* ), category* )
}
\end{Verbatim}

\end{reflist}  \index{cb=<cb>|oddindex}
\begin{reflist}
\item[]\begin{specHead}{TEI.cb}{<cb> }(column break) marks the beginning of a new column of a text on a multi-column page. [\xref{http://www.tei-c.org/release/doc/tei-p5-doc/en/html/CO.html\#CORS5}{3.10.3. Milestone Elements}]\end{specHead} 
    \item[{Module}]
  core
    \item[{Attributes}]
  Attributes att.global (\textit{@xml:id}, \textit{@n}, \textit{@xml:lang}, \textit{@xml:base}, \textit{@xml:space})  (att.global.rendition (\textit{@rend}, \textit{@style}, \textit{@rendition})) (att.global.linking (\textit{@corresp}, \textit{@synch}, \textit{@sameAs}, \textit{@copyOf}, \textit{@next}, \textit{@prev}, \textit{@exclude}, \textit{@select})) (att.global.analytic (\textit{@ana})) (att.global.facs (\textit{@facs})) (att.global.change (\textit{@change})) (att.global.responsibility (\textit{@cert}, \textit{@resp})) (att.global.source (\textit{@source})) att.typed (\textit{@type}, \textit{@subtype}) att.edition (\textit{@ed}, \textit{@edRef}) att.spanning (\textit{@spanTo}) att.breaking (\textit{@break}) 
    \item[{Member of}]
  model.milestoneLike
    \item[{Contained by}]
  
    \item[analysis: ]
   cl m phr s span w\par 
    \item[core: ]
   abbr add addrLine address author bibl biblScope cit citedRange corr date del distinct editor email emph expan foreign gloss head headItem headLabel hi imprint item l label lg list listBibl measure mentioned name note num orig p pubPlace publisher q quote ref reg resp rs said series sic soCalled sp speaker stage street term textLang time title unclear\par 
    \item[figures: ]
   cell figure table\par 
    \item[header: ]
   authority change classCode distributor edition extent funder geoDecl handNote language licence principal scriptNote sponsor typeNote\par 
    \item[linking: ]
   ab seg\par 
    \item[msdescription: ]
   accMat acquisition additions catchwords collation colophon condition custEvent decoNote explicit filiation finalRubric foliation heraldry incipit layout material msItem musicNotation objectType origDate origPlace origin provenance rubric secFol signatures source stamp summary support surrogates watermark\par 
    \item[namesdates: ]
   addName affiliation age birth bloc country death district education faith floruit forename genName geogFeat geogName langKnown nameLink nationality occupation offset org orgName persName person personGrp placeName region residence roleName settlement sex socecStatus surname\par 
    \item[textcrit: ]
   lem rdg wit witDetail\par 
    \item[textstructure: ]
   argument back body byline closer dateline div docAuthor docDate docEdition docImprint docTitle epigraph floatingText front group imprimatur opener postscript salute signed text titlePage titlePart trailer\par 
    \item[transcr: ]
   damage fw line metamark mod restore retrace secl sourceDoc subst supplied surface surfaceGrp surplus zone
    \item[{May contain}]
  Empty element
    \item[{Note}]
  \par
On this element, the global {\itshape n} attribute indicates the number or other value associated with the column which follows the point of insertion of this <cb> element. Encoders should adopt a clear and consistent policy as to whether the numbers associated with column breaks relate to the physical sequence number of the column in the whole text, or whether columns are numbered within the page. The <cb> element is placed at the head of the column to which it refers.
    \item[{Example}]
  Markup of an early English dictionary printed in two columns:\leavevmode\bgroup\exampleFont \begin{shaded}\noindent\mbox{}{<\textbf{pb}/>}\mbox{}\newline 
{<\textbf{cb}\hspace*{6pt}{n}="{1}"/>}\mbox{}\newline 
{<\textbf{entryFree}>}\mbox{}\newline 
\hspace*{6pt}{<\textbf{form}>}Well{</\textbf{form}>}, {<\textbf{sense}>}a Pit to hold Spring-Water{</\textbf{sense}>}:\mbox{}\newline 
{<\textbf{sense}>}In the Art of {<\textbf{hi}\hspace*{6pt}{rend}="{italic}">}War{</\textbf{hi}>}, a Depth the Miner\mbox{}\newline 
\hspace*{6pt}\hspace*{6pt} sinks into the Ground, to find out and disappoint the Enemies Mines,\mbox{}\newline 
\hspace*{6pt}\hspace*{6pt} or to prepare one{</\textbf{sense}>}.\mbox{}\newline 
{</\textbf{entryFree}>}\mbox{}\newline 
{<\textbf{entryFree}>}To {<\textbf{form}>}Welter{</\textbf{form}>}, {<\textbf{sense}>}to wallow{</\textbf{sense}>}, or\mbox{}\newline 
{<\textbf{sense}>}lie groveling{</\textbf{sense}>}.{</\textbf{entryFree}>}\mbox{}\newline 
\textit{<!-- remainder of column -->}\mbox{}\newline 
{<\textbf{cb}\hspace*{6pt}{n}="{2}"/>}\mbox{}\newline 
{<\textbf{entryFree}>}\mbox{}\newline 
\hspace*{6pt}{<\textbf{form}>}Wey{</\textbf{form}>}, {<\textbf{sense}>}the greatest Measure for dry Things,\mbox{}\newline 
\hspace*{6pt}\hspace*{6pt} containing five Chaldron{</\textbf{sense}>}.\mbox{}\newline 
{</\textbf{entryFree}>}\mbox{}\newline 
{<\textbf{entryFree}>}\mbox{}\newline 
\hspace*{6pt}{<\textbf{form}>}Whale{</\textbf{form}>}, {<\textbf{sense}>}the greatest of\mbox{}\newline 
\hspace*{6pt}\hspace*{6pt} Sea-Fishes{</\textbf{sense}>}.\mbox{}\newline 
{</\textbf{entryFree}>}\end{shaded}\egroup 


    \item[{Content model}]
  \fbox{\ttfamily <content>\newline
</content>\newline
    } 
    \item[{Schema Declaration}]
  \mbox{}\hfill\\[-10pt]\begin{Verbatim}[fontsize=\small]
element cb
{
   att.global.attributes,
   att.typed.attributes,
   att.edition.attributes,
   att.spanning.attributes,
   att.breaking.attributes,
   empty
}
\end{Verbatim}

\end{reflist}  \index{cell=<cell>|oddindex}
\begin{reflist}
\item[]\begin{specHead}{TEI.cell}{<cell> }contains one cell of a table. [\xref{http://www.tei-c.org/release/doc/tei-p5-doc/en/html/FT.html\#FTTAB1}{14.1.1. TEI Tables}]\end{specHead} 
    \item[{Module}]
  figures
    \item[{Attributes}]
  Attributes att.global (\textit{@xml:id}, \textit{@n}, \textit{@xml:lang}, \textit{@xml:base}, \textit{@xml:space})  (att.global.rendition (\textit{@rend}, \textit{@style}, \textit{@rendition})) (att.global.linking (\textit{@corresp}, \textit{@synch}, \textit{@sameAs}, \textit{@copyOf}, \textit{@next}, \textit{@prev}, \textit{@exclude}, \textit{@select})) (att.global.analytic (\textit{@ana})) (att.global.facs (\textit{@facs})) (att.global.change (\textit{@change})) (att.global.responsibility (\textit{@cert}, \textit{@resp})) (att.global.source (\textit{@source})) att.tableDecoration (\textit{@role}, \textit{@rows}, \textit{@cols}) 
    \item[{Contained by}]
  
    \item[figures: ]
   row
    \item[{May contain}]
  
    \item[analysis: ]
   c cl interp interpGrp m pc phr s span spanGrp w\par 
    \item[core: ]
   abbr add address bibl biblStruct cb choice cit corr date del desc distinct email emph expan foreign gap gb gloss graphic hi index l label lb lg list listBibl measure measureGrp media mentioned milestone name note num orig p pb ptr q quote ref reg rs said sic soCalled sp stage term time title unclear\par 
    \item[figures: ]
   figure formula notatedMusic table\par 
    \item[gaiji: ]
   g\par 
    \item[header: ]
   biblFull idno\par 
    \item[linking: ]
   ab alt altGrp anchor join joinGrp link linkGrp seg timeline\par 
    \item[msdescription: ]
   catchwords depth dim dimensions height heraldry locus locusGrp material msDesc objectType origDate origPlace secFol signatures stamp watermark width\par 
    \item[namesdates: ]
   addName affiliation bloc climate country district forename genName geo geogFeat geogName listEvent listNym listOrg listPerson listPlace location nameLink offset orgName persName placeName population region roleName settlement state surname terrain trait\par 
    \item[textcrit: ]
   app listApp listWit witDetail\par 
    \item[textstructure: ]
   floatingText\par 
    \item[transcr: ]
   addSpan am damage damageSpan delSpan ex fw handShift listTranspose metamark mod redo restore retrace secl space subst substJoin supplied surplus undo\par character data
    \item[{Example}]
  \leavevmode\bgroup\exampleFont \begin{shaded}\noindent\mbox{}{<\textbf{row}>}\mbox{}\newline 
\hspace*{6pt}{<\textbf{cell}\hspace*{6pt}{role}="{label}">}General conduct{</\textbf{cell}>}\mbox{}\newline 
\hspace*{6pt}{<\textbf{cell}\hspace*{6pt}{role}="{data}">}Not satisfactory, on account of his great unpunctuality\mbox{}\newline 
\hspace*{6pt}\hspace*{6pt} and inattention to duties{</\textbf{cell}>}\mbox{}\newline 
{</\textbf{row}>}\end{shaded}\egroup 


    \item[{Content model}]
  \mbox{}\hfill\\[-10pt]\begin{Verbatim}[fontsize=\small]
<content>
 <macroRef key="macro.specialPara"/>
</content>
    
\end{Verbatim}

    \item[{Schema Declaration}]
  \mbox{}\hfill\\[-10pt]\begin{Verbatim}[fontsize=\small]
element cell
{
   att.global.attributes,
   att.tableDecoration.attributes,
   macro.specialPara}
\end{Verbatim}

\end{reflist}  \index{change=<change>|oddindex}\index{target=@target!<change>|oddindex}
\begin{reflist}
\item[]\begin{specHead}{TEI.change}{<change> }documents a change or set of changes made during the production of a source document, or during the revision of an electronic file. [\xref{http://www.tei-c.org/release/doc/tei-p5-doc/en/html/HD.html\#HD6}{2.6. The Revision Description} \xref{http://www.tei-c.org/release/doc/tei-p5-doc/en/html/HD.html\#HD4C}{2.4.1. Creation} \xref{http://www.tei-c.org/release/doc/tei-p5-doc/en/html/PH.html\#PH-changes}{11.7. Identifying Changes and Revisions}]\end{specHead} 
    \item[{Module}]
  header
    \item[{Attributes}]
  Attributes att.ascribed (\textit{@who}) att.datable (\textit{@calendar}, \textit{@period})  (att.datable.w3c (\textit{@when}, \textit{@notBefore}, \textit{@notAfter}, \textit{@from}, \textit{@to})) (att.datable.iso (\textit{@when-iso}, \textit{@notBefore-iso}, \textit{@notAfter-iso}, \textit{@from-iso}, \textit{@to-iso})) (att.datable.custom (\textit{@when-custom}, \textit{@notBefore-custom}, \textit{@notAfter-custom}, \textit{@from-custom}, \textit{@to-custom}, \textit{@datingPoint}, \textit{@datingMethod})) att.docStatus (\textit{@status}) att.global (\textit{@xml:id}, \textit{@n}, \textit{@xml:lang}, \textit{@xml:base}, \textit{@xml:space})  (att.global.rendition (\textit{@rend}, \textit{@style}, \textit{@rendition})) (att.global.linking (\textit{@corresp}, \textit{@synch}, \textit{@sameAs}, \textit{@copyOf}, \textit{@next}, \textit{@prev}, \textit{@exclude}, \textit{@select})) (att.global.analytic (\textit{@ana})) (att.global.facs (\textit{@facs})) (att.global.change (\textit{@change})) (att.global.responsibility (\textit{@cert}, \textit{@resp})) (att.global.source (\textit{@source})) att.typed (\textit{@type}, \textit{@subtype}) \hfil\\[-10pt]\begin{sansreflist}
    \item[@target]
  points to one or more elements that belong to this change.
\begin{reflist}
    \item[{Status}]
  Optional
    \item[{Datatype}]
  1–∞ occurrences of teidata.pointer separated by whitespace
\end{reflist}  
\end{sansreflist}  
    \item[{Contained by}]
  
    \item[header: ]
   listChange revisionDesc\par 
    \item[msdescription: ]
   recordHist
    \item[{May contain}]
  
    \item[analysis: ]
   c cl interp interpGrp m pc phr s span spanGrp w\par 
    \item[core: ]
   abbr add address bibl biblStruct cb choice cit corr date del desc distinct email emph expan foreign gap gb gloss graphic hi index l label lb lg list listBibl measure measureGrp media mentioned milestone name note num orig p pb ptr q quote ref reg rs said sic soCalled sp stage term time title unclear\par 
    \item[figures: ]
   figure formula notatedMusic table\par 
    \item[gaiji: ]
   g\par 
    \item[header: ]
   biblFull idno\par 
    \item[linking: ]
   ab alt altGrp anchor join joinGrp link linkGrp seg timeline\par 
    \item[msdescription: ]
   catchwords depth dim dimensions height heraldry locus locusGrp material msDesc objectType origDate origPlace secFol signatures stamp watermark width\par 
    \item[namesdates: ]
   addName affiliation bloc climate country district forename genName geo geogFeat geogName listEvent listNym listOrg listPerson listPlace location nameLink offset orgName persName placeName population region roleName settlement state surname terrain trait\par 
    \item[textcrit: ]
   app listApp listWit witDetail\par 
    \item[textstructure: ]
   floatingText\par 
    \item[transcr: ]
   addSpan am damage damageSpan delSpan ex fw handShift listTranspose metamark mod redo restore retrace secl space subst substJoin supplied surplus undo\par character data
    \item[{Note}]
  \par
The {\itshape who} attribute may be used to point to any other element, but will typically specify a <respStmt> or <person> element elsewhere in the header, identifying the person responsible for the change and their role in making it.\par
It is recommended that changes be recorded with the most recent first. The {\itshape status} attribute may be used to indicate the status of a document following the change documented.
    \item[{Example}]
  \leavevmode\bgroup\exampleFont \begin{shaded}\noindent\mbox{}{<\textbf{titleStmt}>}\mbox{}\newline 
\hspace*{6pt}{<\textbf{title}>} ... {</\textbf{title}>}\mbox{}\newline 
\hspace*{6pt}{<\textbf{editor}\hspace*{6pt}{xml:id}="{LDB}">}Lou Burnard{</\textbf{editor}>}\mbox{}\newline 
\hspace*{6pt}{<\textbf{respStmt}\hspace*{6pt}{xml:id}="{BZ}">}\mbox{}\newline 
\hspace*{6pt}\hspace*{6pt}{<\textbf{resp}>}copy editing{</\textbf{resp}>}\mbox{}\newline 
\hspace*{6pt}\hspace*{6pt}{<\textbf{name}>}Brett Zamir{</\textbf{name}>}\mbox{}\newline 
\hspace*{6pt}{</\textbf{respStmt}>}\mbox{}\newline 
{</\textbf{titleStmt}>}\mbox{}\newline 
\textit{<!-- ... -->}\mbox{}\newline 
{<\textbf{revisionDesc}\hspace*{6pt}{status}="{published}">}\mbox{}\newline 
\hspace*{6pt}{<\textbf{change}\hspace*{6pt}{status}="{public}"\hspace*{6pt}{when}="{2008-02-02}"\mbox{}\newline 
\hspace*{6pt}\hspace*{6pt}{who}="{\#BZ}">}Finished chapter 23{</\textbf{change}>}\mbox{}\newline 
\hspace*{6pt}{<\textbf{change}\hspace*{6pt}{status}="{draft}"\hspace*{6pt}{when}="{2008-01-02}"\mbox{}\newline 
\hspace*{6pt}\hspace*{6pt}{who}="{\#BZ}">}Finished chapter 2{</\textbf{change}>}\mbox{}\newline 
\hspace*{6pt}{<\textbf{change}\hspace*{6pt}{n}="{P2.2}"\hspace*{6pt}{when}="{1991-12-21}"\mbox{}\newline 
\hspace*{6pt}\hspace*{6pt}{who}="{\#LDB}">}Added examples to section 3{</\textbf{change}>}\mbox{}\newline 
\hspace*{6pt}{<\textbf{change}\hspace*{6pt}{when}="{1991-11-11}"\hspace*{6pt}{who}="{\#MSM}">}Deleted chapter 10{</\textbf{change}>}\mbox{}\newline 
{</\textbf{revisionDesc}>}\end{shaded}\egroup 


    \item[{Example}]
  \leavevmode\bgroup\exampleFont \begin{shaded}\noindent\mbox{}{<\textbf{profileDesc}>}\mbox{}\newline 
\hspace*{6pt}{<\textbf{creation}>}\mbox{}\newline 
\hspace*{6pt}\hspace*{6pt}{<\textbf{listChange}>}\mbox{}\newline 
\hspace*{6pt}\hspace*{6pt}\hspace*{6pt}{<\textbf{change}\hspace*{6pt}{xml:id}="{DRAFT1}">}First draft in pencil{</\textbf{change}>}\mbox{}\newline 
\hspace*{6pt}\hspace*{6pt}\hspace*{6pt}{<\textbf{change}\hspace*{6pt}{notBefore}="{1880-12-09}"\mbox{}\newline 
\hspace*{6pt}\hspace*{6pt}\hspace*{6pt}\hspace*{6pt}{xml:id}="{DRAFT2}">}First revision, mostly\mbox{}\newline 
\hspace*{6pt}\hspace*{6pt}\hspace*{6pt}\hspace*{6pt}\hspace*{6pt}\hspace*{6pt} using green ink{</\textbf{change}>}\mbox{}\newline 
\hspace*{6pt}\hspace*{6pt}\hspace*{6pt}{<\textbf{change}\hspace*{6pt}{notBefore}="{1881-02-13}"\mbox{}\newline 
\hspace*{6pt}\hspace*{6pt}\hspace*{6pt}\hspace*{6pt}{xml:id}="{DRAFT3}">}Final corrections as\mbox{}\newline 
\hspace*{6pt}\hspace*{6pt}\hspace*{6pt}\hspace*{6pt}\hspace*{6pt}\hspace*{6pt} supplied to printer.{</\textbf{change}>}\mbox{}\newline 
\hspace*{6pt}\hspace*{6pt}{</\textbf{listChange}>}\mbox{}\newline 
\hspace*{6pt}{</\textbf{creation}>}\mbox{}\newline 
{</\textbf{profileDesc}>}\end{shaded}\egroup 


    \item[{Content model}]
  \mbox{}\hfill\\[-10pt]\begin{Verbatim}[fontsize=\small]
<content>
 <macroRef key="macro.specialPara"/>
</content>
    
\end{Verbatim}

    \item[{Schema Declaration}]
  \mbox{}\hfill\\[-10pt]\begin{Verbatim}[fontsize=\small]
element change
{
   att.ascribed.attributes,
   att.datable.attributes,
   att.docStatus.attributes,
   att.global.attributes,
   att.typed.attributes,
   attribute target { list { + } }?,
   macro.specialPara}
\end{Verbatim}

\end{reflist}  \index{char=<char>|oddindex}
\begin{reflist}
\item[]\begin{specHead}{TEI.char}{<char> }(character) provides descriptive information about a character. [\xref{http://www.tei-c.org/release/doc/tei-p5-doc/en/html/WD.html\#D25-20}{5.2. Markup Constructs for Representation of Characters and Glyphs}]\end{specHead} 
    \item[{Module}]
  gaiji
    \item[{Attributes}]
  Attributes att.global (\textit{@xml:id}, \textit{@n}, \textit{@xml:lang}, \textit{@xml:base}, \textit{@xml:space})  (att.global.rendition (\textit{@rend}, \textit{@style}, \textit{@rendition})) (att.global.linking (\textit{@corresp}, \textit{@synch}, \textit{@sameAs}, \textit{@copyOf}, \textit{@next}, \textit{@prev}, \textit{@exclude}, \textit{@select})) (att.global.analytic (\textit{@ana})) (att.global.facs (\textit{@facs})) (att.global.change (\textit{@change})) (att.global.responsibility (\textit{@cert}, \textit{@resp})) (att.global.source (\textit{@source}))
    \item[{Contained by}]
  
    \item[gaiji: ]
   charDecl
    \item[{May contain}]
  
    \item[core: ]
   desc graphic media note\par 
    \item[figures: ]
   figure formula\par 
    \item[gaiji: ]
   charName charProp mapping\par 
    \item[textcrit: ]
   witDetail
    \item[{Example}]
  \leavevmode\bgroup\exampleFont \begin{shaded}\noindent\mbox{}{<\textbf{char}\hspace*{6pt}{xml:id}="{circledU4EBA}">}\mbox{}\newline 
\hspace*{6pt}{<\textbf{charName}>}CIRCLED IDEOGRAPH 4EBA{</\textbf{charName}>}\mbox{}\newline 
\hspace*{6pt}{<\textbf{charProp}>}\mbox{}\newline 
\hspace*{6pt}\hspace*{6pt}{<\textbf{unicodeName}>}character-decomposition-mapping{</\textbf{unicodeName}>}\mbox{}\newline 
\hspace*{6pt}\hspace*{6pt}{<\textbf{value}>}circle{</\textbf{value}>}\mbox{}\newline 
\hspace*{6pt}{</\textbf{charProp}>}\mbox{}\newline 
\hspace*{6pt}{<\textbf{charProp}>}\mbox{}\newline 
\hspace*{6pt}\hspace*{6pt}{<\textbf{localName}>}daikanwa{</\textbf{localName}>}\mbox{}\newline 
\hspace*{6pt}\hspace*{6pt}{<\textbf{value}>}36{</\textbf{value}>}\mbox{}\newline 
\hspace*{6pt}{</\textbf{charProp}>}\mbox{}\newline 
\hspace*{6pt}{<\textbf{mapping}\hspace*{6pt}{type}="{standard}">}人{</\textbf{mapping}>}\mbox{}\newline 
{</\textbf{char}>}\end{shaded}\egroup 


    \item[{Content model}]
  \mbox{}\hfill\\[-10pt]\begin{Verbatim}[fontsize=\small]
<content>
 <sequence>
  <elementRef key="charName" minOccurs="0"/>
  <classRef key="model.descLike"
   maxOccurs="unbounded" minOccurs="0"/>
  <elementRef key="charProp"
   maxOccurs="unbounded" minOccurs="0"/>
  <elementRef key="mapping"
   maxOccurs="unbounded" minOccurs="0"/>
  <elementRef key="figure"
   maxOccurs="unbounded" minOccurs="0"/>
  <classRef key="model.graphicLike"
   maxOccurs="unbounded" minOccurs="0"/>
  <classRef key="model.noteLike"
   maxOccurs="unbounded" minOccurs="0"/>
 </sequence>
</content>
    
\end{Verbatim}

    \item[{Schema Declaration}]
  \mbox{}\hfill\\[-10pt]\begin{Verbatim}[fontsize=\small]
element char
{
   att.global.attributes,
   (
      charName?,
      model.descLike*,
      charProp*,
      mapping*,
      figure*,
      model.graphicLike*,
      model.noteLike*
   )
}
\end{Verbatim}

\end{reflist}  \index{charDecl=<charDecl>|oddindex}
\begin{reflist}
\item[]\begin{specHead}{TEI.charDecl}{<charDecl> }(character declarations) provides information about nonstandard characters and glyphs. [\xref{http://www.tei-c.org/release/doc/tei-p5-doc/en/html/WD.html\#D25-20}{5.2. Markup Constructs for Representation of Characters and Glyphs}]\end{specHead} 
    \item[{Module}]
  gaiji
    \item[{Attributes}]
  Attributes att.global (\textit{@xml:id}, \textit{@n}, \textit{@xml:lang}, \textit{@xml:base}, \textit{@xml:space})  (att.global.rendition (\textit{@rend}, \textit{@style}, \textit{@rendition})) (att.global.linking (\textit{@corresp}, \textit{@synch}, \textit{@sameAs}, \textit{@copyOf}, \textit{@next}, \textit{@prev}, \textit{@exclude}, \textit{@select})) (att.global.analytic (\textit{@ana})) (att.global.facs (\textit{@facs})) (att.global.change (\textit{@change})) (att.global.responsibility (\textit{@cert}, \textit{@resp})) (att.global.source (\textit{@source}))
    \item[{Member of}]
  model.encodingDescPart
    \item[{Contained by}]
  
    \item[header: ]
   encodingDesc
    \item[{May contain}]
  
    \item[core: ]
   desc\par 
    \item[gaiji: ]
   char glyph
    \item[{Example}]
  \leavevmode\bgroup\exampleFont \begin{shaded}\noindent\mbox{}{<\textbf{charDecl}>}\mbox{}\newline 
\hspace*{6pt}{<\textbf{char}\hspace*{6pt}{xml:id}="{aENL}">}\mbox{}\newline 
\hspace*{6pt}\hspace*{6pt}{<\textbf{charName}>}LATIN LETTER ENLARGED SMALL A{</\textbf{charName}>}\mbox{}\newline 
\hspace*{6pt}\hspace*{6pt}{<\textbf{mapping}\hspace*{6pt}{type}="{standard}">}a{</\textbf{mapping}>}\mbox{}\newline 
\hspace*{6pt}{</\textbf{char}>}\mbox{}\newline 
{</\textbf{charDecl}>}\end{shaded}\egroup 


    \item[{Content model}]
  \mbox{}\hfill\\[-10pt]\begin{Verbatim}[fontsize=\small]
<content>
 <sequence>
  <elementRef key="desc" minOccurs="0"/>
  <alternate maxOccurs="unbounded"
   minOccurs="1">
   <elementRef key="char"/>
   <elementRef key="glyph"/>
  </alternate>
 </sequence>
</content>
    
\end{Verbatim}

    \item[{Schema Declaration}]
  \mbox{}\hfill\\[-10pt]\begin{Verbatim}[fontsize=\small]
element charDecl { att.global.attributes, ( desc?, ( char | glyph )+ ) }
\end{Verbatim}

\end{reflist}  \index{charName=<charName>|oddindex}
\begin{reflist}
\item[]\begin{specHead}{TEI.charName}{<charName> }(character name) contains the name of a character, expressed following Unicode conventions. [\xref{http://www.tei-c.org/release/doc/tei-p5-doc/en/html/WD.html\#D25-20}{5.2. Markup Constructs for Representation of Characters and Glyphs}]\end{specHead} 
    \item[{Module}]
  gaiji
    \item[{Attributes}]
  Attributes att.global (\textit{@xml:id}, \textit{@n}, \textit{@xml:lang}, \textit{@xml:base}, \textit{@xml:space})  (att.global.rendition (\textit{@rend}, \textit{@style}, \textit{@rendition})) (att.global.linking (\textit{@corresp}, \textit{@synch}, \textit{@sameAs}, \textit{@copyOf}, \textit{@next}, \textit{@prev}, \textit{@exclude}, \textit{@select})) (att.global.analytic (\textit{@ana})) (att.global.facs (\textit{@facs})) (att.global.change (\textit{@change})) (att.global.responsibility (\textit{@cert}, \textit{@resp})) (att.global.source (\textit{@source}))
    \item[{Contained by}]
  
    \item[gaiji: ]
   char
    \item[{May contain}]
  Character data only
    \item[{Note}]
  \par
The name must follow Unicode conventions for character naming. Projects working in similar fields are recommended to coordinate and publish their list of <charName>s to facilitate data exchange.
    \item[{Example}]
  \leavevmode\bgroup\exampleFont \begin{shaded}\noindent\mbox{}{<\textbf{charName}>}CIRCLED IDEOGRAPH 4EBA{</\textbf{charName}>}\end{shaded}\egroup 


    \item[{Content model}]
  \fbox{\ttfamily <content>\newline
 <textNode/>\newline
</content>\newline
    } 
    \item[{Schema Declaration}]
  \fbox{\ttfamily element charName ❴ att.global.attributes, text ❵} 
\end{reflist}  \index{charProp=<charProp>|oddindex}
\begin{reflist}
\item[]\begin{specHead}{TEI.charProp}{<charProp> }(character property) provides a name and value for some property of the parent character or glyph. [\xref{http://www.tei-c.org/release/doc/tei-p5-doc/en/html/WD.html\#D25-20}{5.2. Markup Constructs for Representation of Characters and Glyphs}]\end{specHead} 
    \item[{Module}]
  gaiji
    \item[{Attributes}]
  Attributes att.global (\textit{@xml:id}, \textit{@n}, \textit{@xml:lang}, \textit{@xml:base}, \textit{@xml:space})  (att.global.rendition (\textit{@rend}, \textit{@style}, \textit{@rendition})) (att.global.linking (\textit{@corresp}, \textit{@synch}, \textit{@sameAs}, \textit{@copyOf}, \textit{@next}, \textit{@prev}, \textit{@exclude}, \textit{@select})) (att.global.analytic (\textit{@ana})) (att.global.facs (\textit{@facs})) (att.global.change (\textit{@change})) (att.global.responsibility (\textit{@cert}, \textit{@resp})) (att.global.source (\textit{@source})) att.typed (\textit{@type}, \textit{@subtype}) 
    \item[{Contained by}]
  
    \item[gaiji: ]
   char glyph
    \item[{May contain}]
  
    \item[gaiji: ]
   localName unicodeName value
    \item[{Note}]
  \par
If the property is a Unicode Normative Property, then its <unicodeName> must be supplied. Otherwise, its name must be specied by means of a <localName>.\par
At a later release, additional constraints will be defined on possible value/name combinations using Schematron rules
    \item[{Example}]
  \leavevmode\bgroup\exampleFont \begin{shaded}\noindent\mbox{}{<\textbf{charProp}>}\mbox{}\newline 
\hspace*{6pt}{<\textbf{unicodeName}>}character-decomposition-mapping{</\textbf{unicodeName}>}\mbox{}\newline 
\hspace*{6pt}{<\textbf{value}>}circle{</\textbf{value}>}\mbox{}\newline 
{</\textbf{charProp}>}\mbox{}\newline 
{<\textbf{charProp}>}\mbox{}\newline 
\hspace*{6pt}{<\textbf{localName}>}daikanwa{</\textbf{localName}>}\mbox{}\newline 
\hspace*{6pt}{<\textbf{value}>}36{</\textbf{value}>}\mbox{}\newline 
{</\textbf{charProp}>}\end{shaded}\egroup 


    \item[{Content model}]
  \mbox{}\hfill\\[-10pt]\begin{Verbatim}[fontsize=\small]
<content>
 <sequence>
  <alternate>
   <elementRef key="unicodeName"/>
   <elementRef key="localName"/>
  </alternate>
  <elementRef key="value"/>
 </sequence>
</content>
    
\end{Verbatim}

    \item[{Schema Declaration}]
  \mbox{}\hfill\\[-10pt]\begin{Verbatim}[fontsize=\small]
element charProp
{
   att.global.attributes,
   att.typed.attributes,
   ( ( unicodeName | localName ), value )
}
\end{Verbatim}

\end{reflist}  \index{choice=<choice>|oddindex}
\begin{reflist}
\item[]\begin{specHead}{TEI.choice}{<choice> }groups a number of alternative encodings for the same point in a text. [\xref{http://www.tei-c.org/release/doc/tei-p5-doc/en/html/CO.html\#COED}{3.4. Simple Editorial Changes}]\end{specHead} 
    \item[{Module}]
  core
    \item[{Attributes}]
  Attributes att.global (\textit{@xml:id}, \textit{@n}, \textit{@xml:lang}, \textit{@xml:base}, \textit{@xml:space})  (att.global.rendition (\textit{@rend}, \textit{@style}, \textit{@rendition})) (att.global.linking (\textit{@corresp}, \textit{@synch}, \textit{@sameAs}, \textit{@copyOf}, \textit{@next}, \textit{@prev}, \textit{@exclude}, \textit{@select})) (att.global.analytic (\textit{@ana})) (att.global.facs (\textit{@facs})) (att.global.change (\textit{@change})) (att.global.responsibility (\textit{@cert}, \textit{@resp})) (att.global.source (\textit{@source}))
    \item[{Member of}]
  model.linePart model.pPart.editorial
    \item[{Contained by}]
  
    \item[analysis: ]
   cl pc phr s span w\par 
    \item[core: ]
   abbr add addrLine author bibl biblScope choice citedRange corr date del desc distinct editor email emph expan foreign gloss head headItem headLabel hi item l label measure meeting mentioned name note num orig p pubPlace publisher q quote ref reg resp rs said sic soCalled speaker stage street term textLang time title unclear\par 
    \item[figures: ]
   cell figDesc\par 
    \item[header: ]
   authority catDesc change classCode creation distributor edition extent funder geoDecl handNote language licence principal rendition scriptNote sponsor tagUsage typeNote\par 
    \item[linking: ]
   ab seg\par 
    \item[msdescription: ]
   accMat acquisition additions catchwords collation colophon condition custEvent decoNote explicit filiation finalRubric foliation heraldry incipit layout material musicNotation objectType origDate origPlace origin provenance rubric secFol signatures source stamp summary support surrogates watermark\par 
    \item[namesdates: ]
   addName affiliation age birth bloc country death district education faith floruit forename genName geogFeat geogName langKnown nameLink nationality occupation offset orgName persName placeName region residence roleName settlement sex socecStatus surname\par 
    \item[textcrit: ]
   lem rdg wit witDetail witness\par 
    \item[textstructure: ]
   byline closer dateline docAuthor docDate docEdition docImprint imprimatur opener salute signed titlePart trailer\par 
    \item[transcr: ]
   damage fw line metamark mod restore retrace secl supplied surplus zone
    \item[{May contain}]
  
    \item[core: ]
   abbr choice corr expan orig reg sic unclear\par 
    \item[linking: ]
   seg\par 
    \item[transcr: ]
   am ex supplied
    \item[{Note}]
  \par
Because the children of a <choice> element all represent alternative ways of encoding the same sequence, it is natural to think of them as mutually exclusive. However, there may be cases where a full representation of a text requires the alternative encodings to be considered as parallel.\par
Note also that <choice> elements may self-nest.\par
Where the purpose of an encoding is to record multiple witnesses of a single work, rather than to identify multiple possible encoding decisions at a given point, the <app> element and associated elements discussed in section \xref{http://www.tei-c.org/release/doc/tei-p5-doc/en/html/TC.html\#TCAPLL}{12.1. The Apparatus Entry, Readings, and Witnesses} should be preferred.
    \item[{Example}]
  An American encoding of \textit{Gulliver's Travels} which retains the British spelling but also provides a version regularized to American spelling might be encoded as follows.\leavevmode\bgroup\exampleFont \begin{shaded}\noindent\mbox{}{<\textbf{p}>}Lastly, That, upon his solemn oath to observe all the above\mbox{}\newline 
 articles, the said man-mountain shall have a daily allowance of\mbox{}\newline 
 meat and drink sufficient for the support of {<\textbf{choice}>}\mbox{}\newline 
\hspace*{6pt}\hspace*{6pt}{<\textbf{sic}>}1724{</\textbf{sic}>}\mbox{}\newline 
\hspace*{6pt}\hspace*{6pt}{<\textbf{corr}>}1728{</\textbf{corr}>}\mbox{}\newline 
\hspace*{6pt}{</\textbf{choice}>} of our subjects,\mbox{}\newline 
 with free access to our royal person, and other marks of our\mbox{}\newline 
{<\textbf{choice}>}\mbox{}\newline 
\hspace*{6pt}\hspace*{6pt}{<\textbf{orig}>}favour{</\textbf{orig}>}\mbox{}\newline 
\hspace*{6pt}\hspace*{6pt}{<\textbf{reg}>}favor{</\textbf{reg}>}\mbox{}\newline 
\hspace*{6pt}{</\textbf{choice}>}.{</\textbf{p}>}\end{shaded}\egroup 


    \item[{Content model}]
  \mbox{}\hfill\\[-10pt]\begin{Verbatim}[fontsize=\small]
<content>
 <alternate maxOccurs="unbounded"
  minOccurs="0">
  <classRef key="model.choicePart"/>
  <elementRef key="choice"/>
 </alternate>
</content>
    
\end{Verbatim}

    \item[{Schema Declaration}]
  \mbox{}\hfill\\[-10pt]\begin{Verbatim}[fontsize=\small]
element choice { att.global.attributes, ( model.choicePart | choice )* }
\end{Verbatim}

\end{reflist}  \index{cit=<cit>|oddindex}
\begin{reflist}
\item[]\begin{specHead}{TEI.cit}{<cit> }(cited quotation) contains a quotation from some other document, together with a bibliographic reference to its source. In a dictionary it may contain an example text with at least one occurrence of the word form, used in the sense being described, or a translation of the headword, or an example. [\xref{http://www.tei-c.org/release/doc/tei-p5-doc/en/html/CO.html\#COHQQ}{3.3.3. Quotation} \xref{http://www.tei-c.org/release/doc/tei-p5-doc/en/html/DS.html\#DSGRP}{4.3.1. Grouped Texts} \xref{http://www.tei-c.org/release/doc/tei-p5-doc/en/html/DI.html\#DITPEG}{9.3.5.1. Examples}]\end{specHead} 
    \item[{Module}]
  core
    \item[{Attributes}]
  Attributes att.global (\textit{@xml:id}, \textit{@n}, \textit{@xml:lang}, \textit{@xml:base}, \textit{@xml:space})  (att.global.rendition (\textit{@rend}, \textit{@style}, \textit{@rendition})) (att.global.linking (\textit{@corresp}, \textit{@synch}, \textit{@sameAs}, \textit{@copyOf}, \textit{@next}, \textit{@prev}, \textit{@exclude}, \textit{@select})) (att.global.analytic (\textit{@ana})) (att.global.facs (\textit{@facs})) (att.global.change (\textit{@change})) (att.global.responsibility (\textit{@cert}, \textit{@resp})) (att.global.source (\textit{@source})) att.typed (\textit{@type}, \textit{@subtype}) 
    \item[{Member of}]
  model.quoteLike
    \item[{Contained by}]
  
    \item[core: ]
   add cit corr del desc emph head hi item l meeting note orig p q quote ref reg said sic sp stage title unclear\par 
    \item[figures: ]
   cell figDesc figure\par 
    \item[header: ]
   change handNote licence rendition scriptNote tagUsage typeNote\par 
    \item[linking: ]
   ab seg\par 
    \item[msdescription: ]
   accMat acquisition additions collation condition custEvent decoNote filiation foliation layout msItem musicNotation origin provenance signatures source summary support surrogates\par 
    \item[namesdates: ]
   occupation\par 
    \item[textcrit: ]
   lem rdg witness\par 
    \item[textstructure: ]
   argument body div docEdition epigraph imprimatur postscript salute signed titlePart trailer\par 
    \item[transcr: ]
   damage metamark mod restore retrace secl supplied surplus
    \item[{May contain}]
  
    \item[analysis: ]
   interp interpGrp span spanGrp\par 
    \item[core: ]
   bibl biblStruct cb cit gap gb index lb listBibl milestone note pb ptr q quote ref said\par 
    \item[figures: ]
   figure notatedMusic\par 
    \item[header: ]
   biblFull\par 
    \item[linking: ]
   alt altGrp anchor join joinGrp link linkGrp timeline\par 
    \item[msdescription: ]
   msDesc\par 
    \item[textcrit: ]
   app witDetail\par 
    \item[textstructure: ]
   floatingText\par 
    \item[transcr: ]
   addSpan damageSpan delSpan fw listTranspose metamark space substJoin
    \item[{Example}]
  \leavevmode\bgroup\exampleFont \begin{shaded}\noindent\mbox{}{<\textbf{cit}>}\mbox{}\newline 
\hspace*{6pt}{<\textbf{quote}>}and the breath of the whale is frequently attended with such an insupportable smell,\mbox{}\newline 
\hspace*{6pt}\hspace*{6pt} as to bring on disorder of the brain.{</\textbf{quote}>}\mbox{}\newline 
\hspace*{6pt}{<\textbf{bibl}>}Ulloa's South America{</\textbf{bibl}>}\mbox{}\newline 
{</\textbf{cit}>}\end{shaded}\egroup 


    \item[{Example}]
  \leavevmode\bgroup\exampleFont \begin{shaded}\noindent\mbox{}{<\textbf{entry}>}\mbox{}\newline 
\hspace*{6pt}{<\textbf{form}>}\mbox{}\newline 
\hspace*{6pt}\hspace*{6pt}{<\textbf{orth}>}horrifier{</\textbf{orth}>}\mbox{}\newline 
\hspace*{6pt}{</\textbf{form}>}\mbox{}\newline 
\hspace*{6pt}{<\textbf{cit}\hspace*{6pt}{type}="{translation}"\hspace*{6pt}{xml:lang}="{en}">}\mbox{}\newline 
\hspace*{6pt}\hspace*{6pt}{<\textbf{quote}>}to horrify{</\textbf{quote}>}\mbox{}\newline 
\hspace*{6pt}{</\textbf{cit}>}\mbox{}\newline 
\hspace*{6pt}{<\textbf{cit}\hspace*{6pt}{type}="{example}">}\mbox{}\newline 
\hspace*{6pt}\hspace*{6pt}{<\textbf{quote}>}elle était horrifiée par la dépense{</\textbf{quote}>}\mbox{}\newline 
\hspace*{6pt}\hspace*{6pt}{<\textbf{cit}\hspace*{6pt}{type}="{translation}"\hspace*{6pt}{xml:lang}="{en}">}\mbox{}\newline 
\hspace*{6pt}\hspace*{6pt}\hspace*{6pt}{<\textbf{quote}>}she was horrified at the expense.{</\textbf{quote}>}\mbox{}\newline 
\hspace*{6pt}\hspace*{6pt}{</\textbf{cit}>}\mbox{}\newline 
\hspace*{6pt}{</\textbf{cit}>}\mbox{}\newline 
{</\textbf{entry}>}\end{shaded}\egroup 


    \item[{Content model}]
  \mbox{}\hfill\\[-10pt]\begin{Verbatim}[fontsize=\small]
<content>
 <alternate maxOccurs="unbounded"
  minOccurs="1">
  <classRef key="model.qLike"/>
  <classRef key="model.egLike"/>
  <classRef key="model.biblLike"/>
  <classRef key="model.ptrLike"/>
  <classRef key="model.global"/>
  <classRef key="model.entryPart"/>
 </alternate>
</content>
    
\end{Verbatim}

    \item[{Schema Declaration}]
  \mbox{}\hfill\\[-10pt]\begin{Verbatim}[fontsize=\small]
element cit
{
   att.global.attributes,
   att.typed.attributes,
   (
      model.qLike    | model.egLike    | model.biblLike    | model.ptrLike    | model.global    | model.entryPart   )+
}
\end{Verbatim}

\end{reflist}  \index{citedRange=<citedRange>|oddindex}
\begin{reflist}
\item[]\begin{specHead}{TEI.citedRange}{<citedRange> }(cited range) defines the range of cited content, often represented by pages or other units [\xref{http://www.tei-c.org/release/doc/tei-p5-doc/en/html/CO.html\#COBICOB}{3.11.2.5. Scopes and Ranges in Bibliographic Citations}]\end{specHead} 
    \item[{Module}]
  core
    \item[{Attributes}]
  Attributes att.global (\textit{@xml:id}, \textit{@n}, \textit{@xml:lang}, \textit{@xml:base}, \textit{@xml:space})  (att.global.rendition (\textit{@rend}, \textit{@style}, \textit{@rendition})) (att.global.linking (\textit{@corresp}, \textit{@synch}, \textit{@sameAs}, \textit{@copyOf}, \textit{@next}, \textit{@prev}, \textit{@exclude}, \textit{@select})) (att.global.analytic (\textit{@ana})) (att.global.facs (\textit{@facs})) (att.global.change (\textit{@change})) (att.global.responsibility (\textit{@cert}, \textit{@resp})) (att.global.source (\textit{@source})) att.pointing (\textit{@targetLang}, \textit{@target}, \textit{@evaluate}) att.citing (\textit{@unit}, \textit{@from}, \textit{@to}) 
    \item[{Member of}]
  model.biblPart
    \item[{Contained by}]
  
    \item[core: ]
   bibl biblStruct
    \item[{May contain}]
  
    \item[analysis: ]
   c cl interp interpGrp m pc phr s span spanGrp w\par 
    \item[core: ]
   abbr add address cb choice corr date del distinct email emph expan foreign gap gb gloss graphic hi index lb measure measureGrp media mentioned milestone name note num orig pb ptr ref reg rs sic soCalled term time title unclear\par 
    \item[figures: ]
   figure formula notatedMusic\par 
    \item[gaiji: ]
   g\par 
    \item[header: ]
   idno\par 
    \item[linking: ]
   alt altGrp anchor join joinGrp link linkGrp seg timeline\par 
    \item[msdescription: ]
   catchwords depth dim dimensions height heraldry locus locusGrp material objectType origDate origPlace secFol signatures stamp watermark width\par 
    \item[namesdates: ]
   addName affiliation bloc climate country district forename genName geo geogFeat geogName location nameLink offset orgName persName placeName population region roleName settlement state surname terrain trait\par 
    \item[textcrit: ]
   app witDetail\par 
    \item[transcr: ]
   addSpan am damage damageSpan delSpan ex fw handShift listTranspose metamark mod redo restore retrace secl space subst substJoin supplied surplus undo\par character data
    \item[{Note}]
  \par
When the range cited consists of a single page or other unit, use the {\itshape from} and {\itshape to} attributes with an identical value. When no clear endpoint is given the {\itshape from} attribute should be used without {\itshape to}. For example, if the citation has ‘p. 3ff’ as a page reference.
    \item[{Example}]
  \leavevmode\bgroup\exampleFont \begin{shaded}\noindent\mbox{}{<\textbf{citedRange}>}pp 12–13{</\textbf{citedRange}>}\mbox{}\newline 
{<\textbf{citedRange}\hspace*{6pt}{from}="{12}"\hspace*{6pt}{to}="{13}"\hspace*{6pt}{unit}="{page}"/>}\mbox{}\newline 
{<\textbf{citedRange}\hspace*{6pt}{unit}="{volume}">}II{</\textbf{citedRange}>}\mbox{}\newline 
{<\textbf{citedRange}\hspace*{6pt}{unit}="{page}">}12{</\textbf{citedRange}>}\end{shaded}\egroup 


    \item[{Example}]
  \leavevmode\bgroup\exampleFont \begin{shaded}\noindent\mbox{}{<\textbf{bibl}>}\mbox{}\newline 
\hspace*{6pt}{<\textbf{ptr}\hspace*{6pt}{target}="{\#mueller01}"/>}, {<\textbf{citedRange}\hspace*{6pt}{target}="{http://example.com/mueller3.xml\#page4}">}vol. 3, pp.\mbox{}\newline 
\hspace*{6pt}\hspace*{6pt} 4-5{</\textbf{citedRange}>}\mbox{}\newline 
{</\textbf{bibl}>}\end{shaded}\egroup 


    \item[{Content model}]
  \mbox{}\hfill\\[-10pt]\begin{Verbatim}[fontsize=\small]
<content>
 <macroRef key="macro.phraseSeq"/>
</content>
    
\end{Verbatim}

    \item[{Schema Declaration}]
  \mbox{}\hfill\\[-10pt]\begin{Verbatim}[fontsize=\small]
element citedRange
{
   att.global.attributes,
   att.pointing.attributes,
   att.citing.attributes,
   macro.phraseSeq}
\end{Verbatim}

\end{reflist}  \index{cl=<cl>|oddindex}
\begin{reflist}
\item[]\begin{specHead}{TEI.cl}{<cl> }(clause) represents a grammatical clause. [\xref{http://www.tei-c.org/release/doc/tei-p5-doc/en/html/AI.html\#AILC}{17.1. Linguistic Segment Categories}]\end{specHead} 
    \item[{Module}]
  analysis
    \item[{Attributes}]
  Attributes att.global (\textit{@xml:id}, \textit{@n}, \textit{@xml:lang}, \textit{@xml:base}, \textit{@xml:space})  (att.global.rendition (\textit{@rend}, \textit{@style}, \textit{@rendition})) (att.global.linking (\textit{@corresp}, \textit{@synch}, \textit{@sameAs}, \textit{@copyOf}, \textit{@next}, \textit{@prev}, \textit{@exclude}, \textit{@select})) (att.global.analytic (\textit{@ana})) (att.global.facs (\textit{@facs})) (att.global.change (\textit{@change})) (att.global.responsibility (\textit{@cert}, \textit{@resp})) (att.global.source (\textit{@source})) att.segLike (\textit{@function})  (att.datcat (\textit{@datcat}, \textit{@valueDatcat})) (att.fragmentable (\textit{@part})) att.typed (\textit{@type}, \textit{@subtype}) 
    \item[{Member of}]
  model.segLike
    \item[{Contained by}]
  
    \item[analysis: ]
   cl phr s\par 
    \item[core: ]
   abbr add addrLine author bibl biblScope citedRange corr date del distinct editor email emph expan foreign gloss head headItem headLabel hi item l label measure mentioned name note num orig p pubPlace publisher q quote ref reg rs said sic soCalled speaker stage street term textLang time title unclear\par 
    \item[figures: ]
   cell\par 
    \item[header: ]
   change distributor edition extent geoDecl handNote licence scriptNote typeNote\par 
    \item[linking: ]
   ab seg\par 
    \item[msdescription: ]
   accMat acquisition additions catchwords collation colophon condition custEvent decoNote explicit filiation finalRubric foliation heraldry incipit layout material musicNotation objectType origDate origPlace origin provenance rubric secFol signatures source stamp summary support surrogates watermark\par 
    \item[namesdates: ]
   addName affiliation birth bloc country death district education faith floruit forename genName geogFeat geogName nameLink nationality occupation offset orgName persName placeName region residence roleName settlement sex socecStatus surname\par 
    \item[textcrit: ]
   lem rdg wit witDetail\par 
    \item[textstructure: ]
   byline closer dateline docAuthor docDate docEdition docImprint imprimatur opener salute signed titlePart trailer\par 
    \item[transcr: ]
   damage fw metamark mod restore retrace secl supplied surplus
    \item[{May contain}]
  
    \item[analysis: ]
   c cl interp interpGrp m pc phr s span spanGrp w\par 
    \item[core: ]
   abbr add address cb choice corr date del distinct email emph expan foreign gap gb gloss graphic hi index lb measure measureGrp media mentioned milestone name note num orig pb ptr ref reg rs sic soCalled term time title unclear\par 
    \item[figures: ]
   figure formula notatedMusic\par 
    \item[gaiji: ]
   g\par 
    \item[header: ]
   idno\par 
    \item[linking: ]
   alt altGrp anchor join joinGrp link linkGrp seg timeline\par 
    \item[msdescription: ]
   catchwords depth dim dimensions height heraldry locus locusGrp material objectType origDate origPlace secFol signatures stamp watermark width\par 
    \item[namesdates: ]
   addName affiliation bloc climate country district forename genName geo geogFeat geogName location nameLink offset orgName persName placeName population region roleName settlement state surname terrain trait\par 
    \item[textcrit: ]
   app witDetail\par 
    \item[transcr: ]
   addSpan am damage damageSpan delSpan ex fw handShift listTranspose metamark mod redo restore retrace secl space subst substJoin supplied surplus undo\par character data
    \item[{Note}]
  \par
The {\itshape type} attribute may be used to indicate the type of clause, taking values such as finite, nonfinite, declarative, interrogative, relative etc. as appropriate.
    \item[{Example}]
  \leavevmode\bgroup\exampleFont \begin{shaded}\noindent\mbox{}{<\textbf{cl}\hspace*{6pt}{function}="{clause\textunderscore modifier}"\mbox{}\newline 
\hspace*{6pt}{type}="{relative}">}Which frightened\mbox{}\newline 
 both the heroes so,{<\textbf{cl}>}They quite forgot their quarrel.{</\textbf{cl}>}\mbox{}\newline 
{</\textbf{cl}>}\end{shaded}\egroup 


    \item[{Content model}]
  \mbox{}\hfill\\[-10pt]\begin{Verbatim}[fontsize=\small]
<content>
 <macroRef key="macro.phraseSeq"/>
</content>
    
\end{Verbatim}

    \item[{Schema Declaration}]
  \mbox{}\hfill\\[-10pt]\begin{Verbatim}[fontsize=\small]
element cl
{
   att.global.attributes,
   att.segLike.attributes,
   att.typed.attributes,
   macro.phraseSeq}
\end{Verbatim}

\end{reflist}  \index{classCode=<classCode>|oddindex}\index{scheme=@scheme!<classCode>|oddindex}
\begin{reflist}
\item[]\begin{specHead}{TEI.classCode}{<classCode> }(classification code) contains the classification code used for this text in some standard classification system. [\xref{http://www.tei-c.org/release/doc/tei-p5-doc/en/html/HD.html\#HD43}{2.4.3. The Text Classification}]\end{specHead} 
    \item[{Module}]
  header
    \item[{Attributes}]
  Attributes att.global (\textit{@xml:id}, \textit{@n}, \textit{@xml:lang}, \textit{@xml:base}, \textit{@xml:space})  (att.global.rendition (\textit{@rend}, \textit{@style}, \textit{@rendition})) (att.global.linking (\textit{@corresp}, \textit{@synch}, \textit{@sameAs}, \textit{@copyOf}, \textit{@next}, \textit{@prev}, \textit{@exclude}, \textit{@select})) (att.global.analytic (\textit{@ana})) (att.global.facs (\textit{@facs})) (att.global.change (\textit{@change})) (att.global.responsibility (\textit{@cert}, \textit{@resp})) (att.global.source (\textit{@source})) \hfil\\[-10pt]\begin{sansreflist}
    \item[@scheme]
  identifies the classification system in use, as defined by, e.g. a <taxonomy> element, or some other resource.
\begin{reflist}
    \item[{Status}]
  Required
    \item[{Datatype}]
  teidata.pointer
\end{reflist}  
\end{sansreflist}  
    \item[{Contained by}]
  
    \item[core: ]
   imprint\par 
    \item[header: ]
   textClass
    \item[{May contain}]
  
    \item[analysis: ]
   interp interpGrp span spanGrp\par 
    \item[core: ]
   abbr address cb choice date distinct email emph expan foreign gap gb gloss hi index lb measure measureGrp mentioned milestone name note num pb ptr ref rs soCalled term time title\par 
    \item[figures: ]
   figure notatedMusic\par 
    \item[header: ]
   idno\par 
    \item[linking: ]
   alt altGrp anchor join joinGrp link linkGrp timeline\par 
    \item[msdescription: ]
   catchwords depth dim dimensions height heraldry locus locusGrp material objectType origDate origPlace secFol signatures stamp watermark width\par 
    \item[namesdates: ]
   addName affiliation bloc climate country district forename genName geo geogFeat geogName location nameLink offset orgName persName placeName population region roleName settlement state surname terrain trait\par 
    \item[textcrit: ]
   app witDetail\par 
    \item[transcr: ]
   addSpan am damageSpan delSpan ex fw listTranspose metamark space subst substJoin\par character data
    \item[{Example}]
  \leavevmode\bgroup\exampleFont \begin{shaded}\noindent\mbox{}{<\textbf{classCode}\hspace*{6pt}{scheme}="{http://www.udc.org}">}410{</\textbf{classCode}>}\end{shaded}\egroup 


    \item[{Content model}]
  \mbox{}\hfill\\[-10pt]\begin{Verbatim}[fontsize=\small]
<content>
 <macroRef key="macro.phraseSeq.limited"/>
</content>
    
\end{Verbatim}

    \item[{Schema Declaration}]
  \mbox{}\hfill\\[-10pt]\begin{Verbatim}[fontsize=\small]
element classCode
{
   att.global.attributes,
   attribute scheme { text },
   macro.phraseSeq.limited}
\end{Verbatim}

\end{reflist}  \index{classDecl=<classDecl>|oddindex}
\begin{reflist}
\item[]\begin{specHead}{TEI.classDecl}{<classDecl> }(classification declarations) contains one or more taxonomies defining any classificatory codes used elsewhere in the text. [\xref{http://www.tei-c.org/release/doc/tei-p5-doc/en/html/HD.html\#HD55}{2.3.7. The Classification Declaration} \xref{http://www.tei-c.org/release/doc/tei-p5-doc/en/html/HD.html\#HD5}{2.3. The Encoding Description}]\end{specHead} 
    \item[{Module}]
  header
    \item[{Attributes}]
  Attributes att.global (\textit{@xml:id}, \textit{@n}, \textit{@xml:lang}, \textit{@xml:base}, \textit{@xml:space})  (att.global.rendition (\textit{@rend}, \textit{@style}, \textit{@rendition})) (att.global.linking (\textit{@corresp}, \textit{@synch}, \textit{@sameAs}, \textit{@copyOf}, \textit{@next}, \textit{@prev}, \textit{@exclude}, \textit{@select})) (att.global.analytic (\textit{@ana})) (att.global.facs (\textit{@facs})) (att.global.change (\textit{@change})) (att.global.responsibility (\textit{@cert}, \textit{@resp})) (att.global.source (\textit{@source}))
    \item[{Member of}]
  model.encodingDescPart
    \item[{Contained by}]
  
    \item[header: ]
   encodingDesc
    \item[{May contain}]
  
    \item[header: ]
   taxonomy
    \item[{Example}]
  \leavevmode\bgroup\exampleFont \begin{shaded}\noindent\mbox{}{<\textbf{classDecl}>}\mbox{}\newline 
\hspace*{6pt}{<\textbf{taxonomy}\hspace*{6pt}{xml:id}="{LCSH}">}\mbox{}\newline 
\hspace*{6pt}\hspace*{6pt}{<\textbf{bibl}>}Library of Congress Subject Headings{</\textbf{bibl}>}\mbox{}\newline 
\hspace*{6pt}{</\textbf{taxonomy}>}\mbox{}\newline 
{</\textbf{classDecl}>}\mbox{}\newline 
\textit{<!-- ... -->}\mbox{}\newline 
{<\textbf{textClass}>}\mbox{}\newline 
\hspace*{6pt}{<\textbf{keywords}\hspace*{6pt}{scheme}="{\#LCSH}">}\mbox{}\newline 
\hspace*{6pt}\hspace*{6pt}{<\textbf{term}>}Political science{</\textbf{term}>}\mbox{}\newline 
\hspace*{6pt}\hspace*{6pt}{<\textbf{term}>}United States -- Politics and government —\mbox{}\newline 
\hspace*{6pt}\hspace*{6pt}\hspace*{6pt}\hspace*{6pt} Revolution, 1775-1783{</\textbf{term}>}\mbox{}\newline 
\hspace*{6pt}{</\textbf{keywords}>}\mbox{}\newline 
{</\textbf{textClass}>}\end{shaded}\egroup 


    \item[{Content model}]
  \mbox{}\hfill\\[-10pt]\begin{Verbatim}[fontsize=\small]
<content>
 <elementRef key="taxonomy"
  maxOccurs="unbounded" minOccurs="1"/>
</content>
    
\end{Verbatim}

    \item[{Schema Declaration}]
  \mbox{}\hfill\\[-10pt]\begin{Verbatim}[fontsize=\small]
element classDecl { att.global.attributes, taxonomy+ }
\end{Verbatim}

\end{reflist}  \index{climate=<climate>|oddindex}
\begin{reflist}
\item[]\begin{specHead}{TEI.climate}{<climate> }contains information about the physical climate of a place. [\xref{http://www.tei-c.org/release/doc/tei-p5-doc/en/html/ND.html\#NDGEOGste}{13.3.4.3. States, Traits, and Events}]\end{specHead} 
    \item[{Module}]
  namesdates
    \item[{Attributes}]
  Attributes att.global (\textit{@xml:id}, \textit{@n}, \textit{@xml:lang}, \textit{@xml:base}, \textit{@xml:space})  (att.global.rendition (\textit{@rend}, \textit{@style}, \textit{@rendition})) (att.global.linking (\textit{@corresp}, \textit{@synch}, \textit{@sameAs}, \textit{@copyOf}, \textit{@next}, \textit{@prev}, \textit{@exclude}, \textit{@select})) (att.global.analytic (\textit{@ana})) (att.global.facs (\textit{@facs})) (att.global.change (\textit{@change})) (att.global.responsibility (\textit{@cert}, \textit{@resp})) (att.global.source (\textit{@source})) att.datable (\textit{@calendar}, \textit{@period})  (att.datable.w3c (\textit{@when}, \textit{@notBefore}, \textit{@notAfter}, \textit{@from}, \textit{@to})) (att.datable.iso (\textit{@when-iso}, \textit{@notBefore-iso}, \textit{@notAfter-iso}, \textit{@from-iso}, \textit{@to-iso})) (att.datable.custom (\textit{@when-custom}, \textit{@notBefore-custom}, \textit{@notAfter-custom}, \textit{@from-custom}, \textit{@to-custom}, \textit{@datingPoint}, \textit{@datingMethod})) att.editLike (\textit{@evidence}, \textit{@instant})  (att.dimensions (\textit{@unit}, \textit{@quantity}, \textit{@extent}, \textit{@precision}, \textit{@scope}) (att.ranging (\textit{@atLeast}, \textit{@atMost}, \textit{@min}, \textit{@max}, \textit{@confidence})) ) att.naming (\textit{@role}, \textit{@nymRef})  (att.canonical (\textit{@key}, \textit{@ref})) att.typed (\textit{@type}, \textit{@subtype}) 
    \item[{Member of}]
  model.placeStateLike
    \item[{Contained by}]
  
    \item[analysis: ]
   cl phr s span\par 
    \item[core: ]
   abbr add addrLine address author bibl biblScope citedRange corr date del desc distinct editor email emph expan foreign gloss head headItem headLabel hi item l label measure meeting mentioned name note num orig p pubPlace publisher q quote ref reg resp rs said sic soCalled speaker stage street term textLang time title unclear\par 
    \item[figures: ]
   cell figDesc\par 
    \item[header: ]
   authority catDesc change classCode correspAction creation distributor edition extent funder geoDecl handNote language licence principal rendition scriptNote sponsor tagUsage typeNote\par 
    \item[linking: ]
   ab seg\par 
    \item[msdescription: ]
   accMat acquisition additions catchwords collation colophon condition custEvent decoNote explicit filiation finalRubric foliation heraldry incipit layout material musicNotation objectType origDate origPlace origin provenance rubric secFol signatures source stamp summary support surrogates watermark\par 
    \item[namesdates: ]
   addName affiliation age birth bloc climate country death district education faith floruit forename genName geogFeat geogName langKnown nameLink nationality occupation offset org orgName persName place placeName region residence roleName settlement sex socecStatus surname\par 
    \item[textcrit: ]
   lem rdg wit witDetail witness\par 
    \item[textstructure: ]
   byline closer dateline docAuthor docDate docEdition docImprint imprimatur opener salute signed titlePart trailer\par 
    \item[transcr: ]
   damage fw metamark mod restore retrace secl supplied surplus
    \item[{May contain}]
  
    \item[core: ]
   bibl biblStruct desc head label listBibl note p\par 
    \item[header: ]
   biblFull\par 
    \item[linking: ]
   ab\par 
    \item[msdescription: ]
   msDesc\par 
    \item[namesdates: ]
   climate\par 
    \item[textcrit: ]
   witDetail
    \item[{Example}]
  \leavevmode\bgroup\exampleFont \begin{shaded}\noindent\mbox{}{<\textbf{place}\hspace*{6pt}{xml:id}="{ROMA}">}\mbox{}\newline 
\hspace*{6pt}{<\textbf{placeName}>}Rome{</\textbf{placeName}>}\mbox{}\newline 
\textit{<!-- ... -->}\mbox{}\newline 
\hspace*{6pt}{<\textbf{climate}>}\mbox{}\newline 
\hspace*{6pt}\hspace*{6pt}{<\textbf{ab}>}\mbox{}\newline 
\hspace*{6pt}\hspace*{6pt}\hspace*{6pt}{<\textbf{table}>}\mbox{}\newline 
\hspace*{6pt}\hspace*{6pt}\hspace*{6pt}\hspace*{6pt}{<\textbf{head}>}24-hr Average Temperature{</\textbf{head}>}\mbox{}\newline 
\hspace*{6pt}\hspace*{6pt}\hspace*{6pt}\hspace*{6pt}{<\textbf{row}>}\mbox{}\newline 
\hspace*{6pt}\hspace*{6pt}\hspace*{6pt}\hspace*{6pt}\hspace*{6pt}{<\textbf{cell}/>}\mbox{}\newline 
\hspace*{6pt}\hspace*{6pt}\hspace*{6pt}\hspace*{6pt}\hspace*{6pt}{<\textbf{cell}\hspace*{6pt}{role}="{label}">}Jan{</\textbf{cell}>}\mbox{}\newline 
\hspace*{6pt}\hspace*{6pt}\hspace*{6pt}\hspace*{6pt}\hspace*{6pt}{<\textbf{cell}\hspace*{6pt}{role}="{label}">}Jun{</\textbf{cell}>}\mbox{}\newline 
\hspace*{6pt}\hspace*{6pt}\hspace*{6pt}\hspace*{6pt}\hspace*{6pt}{<\textbf{cell}\hspace*{6pt}{role}="{label}">}Dec{</\textbf{cell}>}\mbox{}\newline 
\hspace*{6pt}\hspace*{6pt}\hspace*{6pt}\hspace*{6pt}{</\textbf{row}>}\mbox{}\newline 
\hspace*{6pt}\hspace*{6pt}\hspace*{6pt}\hspace*{6pt}{<\textbf{row}>}\mbox{}\newline 
\hspace*{6pt}\hspace*{6pt}\hspace*{6pt}\hspace*{6pt}\hspace*{6pt}{<\textbf{cell}\hspace*{6pt}{role}="{label}">}°C{</\textbf{cell}>}\mbox{}\newline 
\hspace*{6pt}\hspace*{6pt}\hspace*{6pt}\hspace*{6pt}\hspace*{6pt}{<\textbf{cell}\hspace*{6pt}{role}="{data}">}7.1{</\textbf{cell}>}\mbox{}\newline 
\hspace*{6pt}\hspace*{6pt}\hspace*{6pt}\hspace*{6pt}\hspace*{6pt}{<\textbf{cell}\hspace*{6pt}{role}="{data}">}21.7{</\textbf{cell}>}\mbox{}\newline 
\hspace*{6pt}\hspace*{6pt}\hspace*{6pt}\hspace*{6pt}\hspace*{6pt}{<\textbf{cell}\hspace*{6pt}{role}="{data}">}8.3{</\textbf{cell}>}\mbox{}\newline 
\hspace*{6pt}\hspace*{6pt}\hspace*{6pt}\hspace*{6pt}{</\textbf{row}>}\mbox{}\newline 
\hspace*{6pt}\hspace*{6pt}\hspace*{6pt}\hspace*{6pt}{<\textbf{row}>}\mbox{}\newline 
\hspace*{6pt}\hspace*{6pt}\hspace*{6pt}\hspace*{6pt}\hspace*{6pt}{<\textbf{cell}\hspace*{6pt}{role}="{label}">}°F{</\textbf{cell}>}\mbox{}\newline 
\hspace*{6pt}\hspace*{6pt}\hspace*{6pt}\hspace*{6pt}\hspace*{6pt}{<\textbf{cell}\hspace*{6pt}{role}="{data}">}44.8{</\textbf{cell}>}\mbox{}\newline 
\hspace*{6pt}\hspace*{6pt}\hspace*{6pt}\hspace*{6pt}\hspace*{6pt}{<\textbf{cell}\hspace*{6pt}{role}="{data}">}71.1{</\textbf{cell}>}\mbox{}\newline 
\hspace*{6pt}\hspace*{6pt}\hspace*{6pt}\hspace*{6pt}\hspace*{6pt}{<\textbf{cell}\hspace*{6pt}{role}="{data}">}46.9{</\textbf{cell}>}\mbox{}\newline 
\hspace*{6pt}\hspace*{6pt}\hspace*{6pt}\hspace*{6pt}{</\textbf{row}>}\mbox{}\newline 
\hspace*{6pt}\hspace*{6pt}\hspace*{6pt}{</\textbf{table}>}\mbox{}\newline 
\hspace*{6pt}\hspace*{6pt}{</\textbf{ab}>}\mbox{}\newline 
\hspace*{6pt}\hspace*{6pt}{<\textbf{note}>}Taken from {<\textbf{bibl}>}\mbox{}\newline 
\hspace*{6pt}\hspace*{6pt}\hspace*{6pt}\hspace*{6pt}{<\textbf{abbr}>}GHCN 2 Beta{</\textbf{abbr}>}: The Global Historical Climatology Network,\mbox{}\newline 
\hspace*{6pt}\hspace*{6pt}\hspace*{6pt}\hspace*{6pt}\hspace*{6pt}\hspace*{6pt} version 2 beta, 1904 months between 1811 and 1980. {<\textbf{ptr}\hspace*{6pt}{target}="{http://www.worldclimate.com/cgi-bin/data.pl?ref=N41E012+1202+0004058G2}"/>}\mbox{}\newline 
\hspace*{6pt}\hspace*{6pt}\hspace*{6pt}{</\textbf{bibl}>}\mbox{}\newline 
\hspace*{6pt}\hspace*{6pt}{</\textbf{note}>}\mbox{}\newline 
\hspace*{6pt}{</\textbf{climate}>}\mbox{}\newline 
{</\textbf{place}>}\end{shaded}\egroup 


    \item[{Content model}]
  \mbox{}\hfill\\[-10pt]\begin{Verbatim}[fontsize=\small]
<content>
 <sequence>
  <elementRef key="precision"
   maxOccurs="unbounded" minOccurs="0"/>
  <classRef key="model.headLike"
   maxOccurs="unbounded" minOccurs="0"/>
  <alternate>
   <classRef key="model.pLike"
    maxOccurs="unbounded" minOccurs="1"/>
   <classRef key="model.labelLike"
    maxOccurs="unbounded" minOccurs="1"/>
  </alternate>
  <alternate maxOccurs="unbounded"
   minOccurs="0">
   <classRef key="model.noteLike"/>
   <classRef key="model.biblLike"/>
  </alternate>
  <elementRef key="climate"
   maxOccurs="unbounded" minOccurs="0"/>
 </sequence>
</content>
    
\end{Verbatim}

    \item[{Schema Declaration}]
  \mbox{}\hfill\\[-10pt]\begin{Verbatim}[fontsize=\small]
element climate
{
   att.global.attributes,
   att.datable.attributes,
   att.editLike.attributes,
   att.naming.attributes,
   att.typed.attributes,
   (
      precision*,
      model.headLike*,
      ( model.pLike+ | model.labelLike+ ),
      ( model.noteLike | model.biblLike )*,
      climate*
   )
}
\end{Verbatim}

\end{reflist}  \index{closer=<closer>|oddindex}
\begin{reflist}
\item[]\begin{specHead}{TEI.closer}{<closer> }groups together salutations, datelines, and similar phrases appearing as a final group at the end of a division, especially of a letter. [\xref{http://www.tei-c.org/release/doc/tei-p5-doc/en/html/DS.html\#DSOC}{4.2.2. Openers and Closers} \xref{http://www.tei-c.org/release/doc/tei-p5-doc/en/html/DS.html\#DSDTB}{4.2. Elements Common to All Divisions}]\end{specHead} 
    \item[{Module}]
  textstructure
    \item[{Attributes}]
  Attributes att.global (\textit{@xml:id}, \textit{@n}, \textit{@xml:lang}, \textit{@xml:base}, \textit{@xml:space})  (att.global.rendition (\textit{@rend}, \textit{@style}, \textit{@rendition})) (att.global.linking (\textit{@corresp}, \textit{@synch}, \textit{@sameAs}, \textit{@copyOf}, \textit{@next}, \textit{@prev}, \textit{@exclude}, \textit{@select})) (att.global.analytic (\textit{@ana})) (att.global.facs (\textit{@facs})) (att.global.change (\textit{@change})) (att.global.responsibility (\textit{@cert}, \textit{@resp})) (att.global.source (\textit{@source})) att.written (\textit{@hand}) 
    \item[{Member of}]
  model.divBottomPart
    \item[{Contained by}]
  
    \item[core: ]
   lg list\par 
    \item[figures: ]
   figure table\par 
    \item[textstructure: ]
   back body div front group postscript
    \item[{May contain}]
  
    \item[analysis: ]
   c cl interp interpGrp m pc phr s span spanGrp w\par 
    \item[core: ]
   abbr add address cb choice corr date del distinct email emph expan foreign gap gb gloss graphic hi index lb measure measureGrp media mentioned milestone name note num orig pb ptr ref reg rs sic soCalled term time title unclear\par 
    \item[figures: ]
   figure formula notatedMusic\par 
    \item[gaiji: ]
   g\par 
    \item[header: ]
   idno\par 
    \item[linking: ]
   alt altGrp anchor join joinGrp link linkGrp seg timeline\par 
    \item[msdescription: ]
   catchwords depth dim dimensions height heraldry locus locusGrp material objectType origDate origPlace secFol signatures stamp watermark width\par 
    \item[namesdates: ]
   addName affiliation bloc climate country district forename genName geo geogFeat geogName location nameLink offset orgName persName placeName population region roleName settlement state surname terrain trait\par 
    \item[textcrit: ]
   app witDetail\par 
    \item[textstructure: ]
   dateline salute signed\par 
    \item[transcr: ]
   addSpan am damage damageSpan delSpan ex fw handShift listTranspose metamark mod redo restore retrace secl space subst substJoin supplied surplus undo\par character data
    \item[{Example}]
  \leavevmode\bgroup\exampleFont \begin{shaded}\noindent\mbox{}{<\textbf{div}\hspace*{6pt}{type}="{letter}">}\mbox{}\newline 
\hspace*{6pt}{<\textbf{p}>} perhaps you will favour me with a sight of it when convenient.{</\textbf{p}>}\mbox{}\newline 
\hspace*{6pt}{<\textbf{closer}>}\mbox{}\newline 
\hspace*{6pt}\hspace*{6pt}{<\textbf{salute}>}I remain, \&c. \&c.{</\textbf{salute}>}\mbox{}\newline 
\hspace*{6pt}\hspace*{6pt}{<\textbf{signed}>}H. Colburn{</\textbf{signed}>}\mbox{}\newline 
\hspace*{6pt}{</\textbf{closer}>}\mbox{}\newline 
{</\textbf{div}>}\end{shaded}\egroup 


    \item[{Example}]
  \leavevmode\bgroup\exampleFont \begin{shaded}\noindent\mbox{}{<\textbf{div}\hspace*{6pt}{type}="{chapter}">}\mbox{}\newline 
\hspace*{6pt}{<\textbf{p}>}\mbox{}\newline 
\textit{<!-- ... -->} and his heart was going like mad and yes I said yes I will Yes.{</\textbf{p}>}\mbox{}\newline 
\hspace*{6pt}{<\textbf{closer}>}\mbox{}\newline 
\hspace*{6pt}\hspace*{6pt}{<\textbf{dateline}>}\mbox{}\newline 
\hspace*{6pt}\hspace*{6pt}\hspace*{6pt}{<\textbf{name}\hspace*{6pt}{type}="{place}">}Trieste-Zürich-Paris,{</\textbf{name}>}\mbox{}\newline 
\hspace*{6pt}\hspace*{6pt}\hspace*{6pt}{<\textbf{date}>}1914–1921{</\textbf{date}>}\mbox{}\newline 
\hspace*{6pt}\hspace*{6pt}{</\textbf{dateline}>}\mbox{}\newline 
\hspace*{6pt}{</\textbf{closer}>}\mbox{}\newline 
{</\textbf{div}>}\end{shaded}\egroup 


    \item[{Content model}]
  \mbox{}\hfill\\[-10pt]\begin{Verbatim}[fontsize=\small]
<content>
 <alternate maxOccurs="unbounded"
  minOccurs="0">
  <textNode/>
  <classRef key="model.gLike"/>
  <elementRef key="signed"/>
  <elementRef key="dateline"/>
  <elementRef key="salute"/>
  <classRef key="model.phrase"/>
  <classRef key="model.global"/>
 </alternate>
</content>
    
\end{Verbatim}

    \item[{Schema Declaration}]
  \mbox{}\hfill\\[-10pt]\begin{Verbatim}[fontsize=\small]
element closer
{
   att.global.attributes,
   att.written.attributes,
   (
      text
    | model.gLike    | signed    | dateline    | salute    | model.phrase    | model.global   )*
}
\end{Verbatim}

\end{reflist}  \index{collation=<collation>|oddindex}
\begin{reflist}
\item[]\begin{specHead}{TEI.collation}{<collation> }contains a description of how the leaves or bifolia are physically arranged. [\xref{http://www.tei-c.org/release/doc/tei-p5-doc/en/html/MS.html\#msph1}{10.7.1. Object Description}]\end{specHead} 
    \item[{Module}]
  msdescription
    \item[{Attributes}]
  Attributes att.global (\textit{@xml:id}, \textit{@n}, \textit{@xml:lang}, \textit{@xml:base}, \textit{@xml:space})  (att.global.rendition (\textit{@rend}, \textit{@style}, \textit{@rendition})) (att.global.linking (\textit{@corresp}, \textit{@synch}, \textit{@sameAs}, \textit{@copyOf}, \textit{@next}, \textit{@prev}, \textit{@exclude}, \textit{@select})) (att.global.analytic (\textit{@ana})) (att.global.facs (\textit{@facs})) (att.global.change (\textit{@change})) (att.global.responsibility (\textit{@cert}, \textit{@resp})) (att.global.source (\textit{@source}))
    \item[{Contained by}]
  
    \item[msdescription: ]
   supportDesc
    \item[{May contain}]
  
    \item[analysis: ]
   c cl interp interpGrp m pc phr s span spanGrp w\par 
    \item[core: ]
   abbr add address bibl biblStruct cb choice cit corr date del desc distinct email emph expan foreign gap gb gloss graphic hi index l label lb lg list listBibl measure measureGrp media mentioned milestone name note num orig p pb ptr q quote ref reg rs said sic soCalled sp stage term time title unclear\par 
    \item[figures: ]
   figure formula notatedMusic table\par 
    \item[gaiji: ]
   g\par 
    \item[header: ]
   biblFull idno\par 
    \item[linking: ]
   ab alt altGrp anchor join joinGrp link linkGrp seg timeline\par 
    \item[msdescription: ]
   catchwords depth dim dimensions height heraldry locus locusGrp material msDesc objectType origDate origPlace secFol signatures stamp watermark width\par 
    \item[namesdates: ]
   addName affiliation bloc climate country district forename genName geo geogFeat geogName listEvent listNym listOrg listPerson listPlace location nameLink offset orgName persName placeName population region roleName settlement state surname terrain trait\par 
    \item[textcrit: ]
   app listApp listWit witDetail\par 
    \item[textstructure: ]
   floatingText\par 
    \item[transcr: ]
   addSpan am damage damageSpan delSpan ex fw handShift listTranspose metamark mod redo restore retrace secl space subst substJoin supplied surplus undo\par character data
    \item[{Example}]
  \leavevmode\bgroup\exampleFont \begin{shaded}\noindent\mbox{}{<\textbf{collation}>}The written leaves preceded by an original flyleaf,\mbox{}\newline 
 conjoint with the pastedown.{</\textbf{collation}>}\end{shaded}\egroup 


    \item[{Example}]
  \leavevmode\bgroup\exampleFont \begin{shaded}\noindent\mbox{}{<\textbf{collation}>}\mbox{}\newline 
\hspace*{6pt}{<\textbf{p}>}\mbox{}\newline 
\hspace*{6pt}\hspace*{6pt}{<\textbf{formula}>}1-5.8 6.6 (catchword, f. 46, does not match following text)\mbox{}\newline 
\hspace*{6pt}\hspace*{6pt}\hspace*{6pt}\hspace*{6pt} 7-8.8 9.10, 11.2 (through f. 82) 12-14.8 15.8(-7){</\textbf{formula}>}\mbox{}\newline 
\hspace*{6pt}\hspace*{6pt}{<\textbf{catchwords}>}Catchwords are written horizontally in center\mbox{}\newline 
\hspace*{6pt}\hspace*{6pt}\hspace*{6pt}\hspace*{6pt} or towards the right lower margin in various manners:\mbox{}\newline 
\hspace*{6pt}\hspace*{6pt}\hspace*{6pt}\hspace*{6pt} in red ink for quires 1-6 (which are also signed in red\mbox{}\newline 
\hspace*{6pt}\hspace*{6pt}\hspace*{6pt}\hspace*{6pt} ink with letters of the alphabet and arabic numerals);\mbox{}\newline 
\hspace*{6pt}\hspace*{6pt}\hspace*{6pt}\hspace*{6pt} quires 7-9 in ink of text within yellow decorated frames;\mbox{}\newline 
\hspace*{6pt}\hspace*{6pt}\hspace*{6pt}\hspace*{6pt} quire 10 in red decorated frame; quire 12 in ink of text;\mbox{}\newline 
\hspace*{6pt}\hspace*{6pt}\hspace*{6pt}\hspace*{6pt} quire 13 with red decorative slashes; quire 14 added in\mbox{}\newline 
\hspace*{6pt}\hspace*{6pt}\hspace*{6pt}\hspace*{6pt} cursive hand.{</\textbf{catchwords}>}\mbox{}\newline 
\hspace*{6pt}{</\textbf{p}>}\mbox{}\newline 
{</\textbf{collation}>}\end{shaded}\egroup 


    \item[{Content model}]
  \mbox{}\hfill\\[-10pt]\begin{Verbatim}[fontsize=\small]
<content>
 <macroRef key="macro.specialPara"/>
</content>
    
\end{Verbatim}

    \item[{Schema Declaration}]
  \mbox{}\hfill\\[-10pt]\begin{Verbatim}[fontsize=\small]
element collation { att.global.attributes, macro.specialPara }
\end{Verbatim}

\end{reflist}  \index{collection=<collection>|oddindex}
\begin{reflist}
\item[]\begin{specHead}{TEI.collection}{<collection> }contains the name of a collection of manuscripts, not necessarily located within a single repository. [\xref{http://www.tei-c.org/release/doc/tei-p5-doc/en/html/MS.html\#msid}{10.4. The Manuscript Identifier}]\end{specHead} 
    \item[{Module}]
  msdescription
    \item[{Attributes}]
  Attributes att.global (\textit{@xml:id}, \textit{@n}, \textit{@xml:lang}, \textit{@xml:base}, \textit{@xml:space})  (att.global.rendition (\textit{@rend}, \textit{@style}, \textit{@rendition})) (att.global.linking (\textit{@corresp}, \textit{@synch}, \textit{@sameAs}, \textit{@copyOf}, \textit{@next}, \textit{@prev}, \textit{@exclude}, \textit{@select})) (att.global.analytic (\textit{@ana})) (att.global.facs (\textit{@facs})) (att.global.change (\textit{@change})) (att.global.responsibility (\textit{@cert}, \textit{@resp})) (att.global.source (\textit{@source})) att.naming (\textit{@role}, \textit{@nymRef})  (att.canonical (\textit{@key}, \textit{@ref})) att.typed (\textit{@type}, \textit{@subtype}) 
    \item[{Contained by}]
  
    \item[msdescription: ]
   altIdentifier msIdentifier
    \item[{May contain}]
  
    \item[gaiji: ]
   g\par character data
    \item[{Example}]
  \leavevmode\bgroup\exampleFont \begin{shaded}\noindent\mbox{}{<\textbf{msIdentifier}>}\mbox{}\newline 
\hspace*{6pt}{<\textbf{country}>}USA{</\textbf{country}>}\mbox{}\newline 
\hspace*{6pt}{<\textbf{region}>}California{</\textbf{region}>}\mbox{}\newline 
\hspace*{6pt}{<\textbf{settlement}>}San Marino{</\textbf{settlement}>}\mbox{}\newline 
\hspace*{6pt}{<\textbf{repository}>}Huntington Library{</\textbf{repository}>}\mbox{}\newline 
\hspace*{6pt}{<\textbf{collection}>}Ellesmere{</\textbf{collection}>}\mbox{}\newline 
\hspace*{6pt}{<\textbf{idno}>}El 26 C 9{</\textbf{idno}>}\mbox{}\newline 
\hspace*{6pt}{<\textbf{msName}>}The Ellesmere Chaucer{</\textbf{msName}>}\mbox{}\newline 
{</\textbf{msIdentifier}>}\end{shaded}\egroup 


    \item[{Content model}]
  \fbox{\ttfamily <content>\newline
 <macroRef key="macro.xtext"/>\newline
</content>\newline
    } 
    \item[{Schema Declaration}]
  \mbox{}\hfill\\[-10pt]\begin{Verbatim}[fontsize=\small]
element collection
{
   att.global.attributes,
   att.naming.attributes,
   att.typed.attributes,
   macro.xtext}
\end{Verbatim}

\end{reflist}  \index{colophon=<colophon>|oddindex}
\begin{reflist}
\item[]\begin{specHead}{TEI.colophon}{<colophon> }contains the \textit{colophon} of a manuscript item: that is, a statement providing information regarding the date, place, agency, or reason for production of the manuscript. [\xref{http://www.tei-c.org/release/doc/tei-p5-doc/en/html/MS.html\#mscoit}{10.6.1. The msItem and msItemStruct Elements}]\end{specHead} 
    \item[{Module}]
  msdescription
    \item[{Attributes}]
  Attributes att.global (\textit{@xml:id}, \textit{@n}, \textit{@xml:lang}, \textit{@xml:base}, \textit{@xml:space})  (att.global.rendition (\textit{@rend}, \textit{@style}, \textit{@rendition})) (att.global.linking (\textit{@corresp}, \textit{@synch}, \textit{@sameAs}, \textit{@copyOf}, \textit{@next}, \textit{@prev}, \textit{@exclude}, \textit{@select})) (att.global.analytic (\textit{@ana})) (att.global.facs (\textit{@facs})) (att.global.change (\textit{@change})) (att.global.responsibility (\textit{@cert}, \textit{@resp})) (att.global.source (\textit{@source}))
    \item[{Member of}]
  model.msQuoteLike 
    \item[{Contained by}]
  
    \item[msdescription: ]
   msItem msItemStruct
    \item[{May contain}]
  
    \item[analysis: ]
   c cl interp interpGrp m pc phr s span spanGrp w\par 
    \item[core: ]
   abbr add address cb choice corr date del distinct email emph expan foreign gap gb gloss graphic hi index lb measure measureGrp media mentioned milestone name note num orig pb ptr ref reg rs sic soCalled term time title unclear\par 
    \item[figures: ]
   figure formula notatedMusic\par 
    \item[gaiji: ]
   g\par 
    \item[header: ]
   idno\par 
    \item[linking: ]
   alt altGrp anchor join joinGrp link linkGrp seg timeline\par 
    \item[msdescription: ]
   catchwords depth dim dimensions height heraldry locus locusGrp material objectType origDate origPlace secFol signatures stamp watermark width\par 
    \item[namesdates: ]
   addName affiliation bloc climate country district forename genName geo geogFeat geogName location nameLink offset orgName persName placeName population region roleName settlement state surname terrain trait\par 
    \item[textcrit: ]
   app witDetail\par 
    \item[transcr: ]
   addSpan am damage damageSpan delSpan ex fw handShift listTranspose metamark mod redo restore retrace secl space subst substJoin supplied surplus undo\par character data
    \item[{Example}]
  \leavevmode\bgroup\exampleFont \begin{shaded}\noindent\mbox{}{<\textbf{colophon}>}Ricardus Franciscus Scripsit Anno Domini\mbox{}\newline 
 1447.{</\textbf{colophon}>}\end{shaded}\egroup 


    \item[{Example}]
  \leavevmode\bgroup\exampleFont \begin{shaded}\noindent\mbox{}{<\textbf{colophon}>}Explicit expliceat/scriptor ludere eat.{</\textbf{colophon}>}\end{shaded}\egroup 


    \item[{Example}]
  \leavevmode\bgroup\exampleFont \begin{shaded}\noindent\mbox{}{<\textbf{colophon}>}Explicit venenum viciorum domini illius, qui comparavit Anno\mbox{}\newline 
 domini Millessimo Trecentesimo nonagesimo primo, Sabbato in festo\mbox{}\newline 
 sancte Marthe virginis gloriose. Laus tibi criste quia finitur\mbox{}\newline 
 libellus iste.{</\textbf{colophon}>}\end{shaded}\egroup 


    \item[{Content model}]
  \mbox{}\hfill\\[-10pt]\begin{Verbatim}[fontsize=\small]
<content>
 <macroRef key="macro.phraseSeq"/>
</content>
    
\end{Verbatim}

    \item[{Schema Declaration}]
  \mbox{}\hfill\\[-10pt]\begin{Verbatim}[fontsize=\small]
element colophon { att.global.attributes, macro.phraseSeq }
\end{Verbatim}

\end{reflist}  \index{condition=<condition>|oddindex}
\begin{reflist}
\item[]\begin{specHead}{TEI.condition}{<condition> }contains a description of the physical condition of the manuscript. [\xref{http://www.tei-c.org/release/doc/tei-p5-doc/en/html/MS.html\#msphco}{10.7.1.5. Condition}]\end{specHead} 
    \item[{Module}]
  msdescription
    \item[{Attributes}]
  Attributes att.global (\textit{@xml:id}, \textit{@n}, \textit{@xml:lang}, \textit{@xml:base}, \textit{@xml:space})  (att.global.rendition (\textit{@rend}, \textit{@style}, \textit{@rendition})) (att.global.linking (\textit{@corresp}, \textit{@synch}, \textit{@sameAs}, \textit{@copyOf}, \textit{@next}, \textit{@prev}, \textit{@exclude}, \textit{@select})) (att.global.analytic (\textit{@ana})) (att.global.facs (\textit{@facs})) (att.global.change (\textit{@change})) (att.global.responsibility (\textit{@cert}, \textit{@resp})) (att.global.source (\textit{@source}))
    \item[{Contained by}]
  
    \item[msdescription: ]
   binding bindingDesc sealDesc supportDesc
    \item[{May contain}]
  
    \item[analysis: ]
   c cl interp interpGrp m pc phr s span spanGrp w\par 
    \item[core: ]
   abbr add address bibl biblStruct cb choice cit corr date del desc distinct email emph expan foreign gap gb gloss graphic hi index l label lb lg list listBibl measure measureGrp media mentioned milestone name note num orig p pb ptr q quote ref reg rs said sic soCalled sp stage term time title unclear\par 
    \item[figures: ]
   figure formula notatedMusic table\par 
    \item[gaiji: ]
   g\par 
    \item[header: ]
   biblFull idno\par 
    \item[linking: ]
   ab alt altGrp anchor join joinGrp link linkGrp seg timeline\par 
    \item[msdescription: ]
   catchwords depth dim dimensions height heraldry locus locusGrp material msDesc objectType origDate origPlace secFol signatures stamp watermark width\par 
    \item[namesdates: ]
   addName affiliation bloc climate country district forename genName geo geogFeat geogName listEvent listNym listOrg listPerson listPlace location nameLink offset orgName persName placeName population region roleName settlement state surname terrain trait\par 
    \item[textcrit: ]
   app listApp listWit witDetail\par 
    \item[textstructure: ]
   floatingText\par 
    \item[transcr: ]
   addSpan am damage damageSpan delSpan ex fw handShift listTranspose metamark mod redo restore retrace secl space subst substJoin supplied surplus undo\par character data
    \item[{Example}]
  \leavevmode\bgroup\exampleFont \begin{shaded}\noindent\mbox{}{<\textbf{condition}>}\mbox{}\newline 
\hspace*{6pt}{<\textbf{p}>}There are lacunae in three places in this\mbox{}\newline 
\hspace*{6pt}\hspace*{6pt} manuscript. After 14v two\mbox{}\newline 
\hspace*{6pt}\hspace*{6pt} leaves has been cut out and narrow strips leaves remains in the spine. After\mbox{}\newline 
\hspace*{6pt}\hspace*{6pt} 68v one gathering is missing and after 101v at least one gathering of 8 leaves\mbox{}\newline 
\hspace*{6pt}\hspace*{6pt} has been lost. {</\textbf{p}>}\mbox{}\newline 
\hspace*{6pt}{<\textbf{p}>}Several leaves are damaged with tears or holes or have a\mbox{}\newline 
\hspace*{6pt}\hspace*{6pt} irregular shape. Some of the damages do not allow the lines to be of full\mbox{}\newline 
\hspace*{6pt}\hspace*{6pt} length and they are apparently older than the script. There are tears on fol.\mbox{}\newline 
\hspace*{6pt}\hspace*{6pt} 2r-v, 9r-v, 10r-v, 15r-18v, 19r-v, 20r-22v, 23r-v, 24r-28v, 30r-v, 32r-35v,\mbox{}\newline 
\hspace*{6pt}\hspace*{6pt} 37r-v, 38r-v, 40r-43v, 45r-47v, 49r-v, 51r-v, 53r-60v, 67r-v, 68r-v, 70r-v,\mbox{}\newline 
\hspace*{6pt}\hspace*{6pt} 74r-80v, 82r-v, 86r-v, 88r-v, 89r-v, 95r-v, 97r-98v 99r-v, 100r-v. On fol. 98\mbox{}\newline 
\hspace*{6pt}\hspace*{6pt} the corner has been torn off. Several leaves are in a bad condition due to\mbox{}\newline 
\hspace*{6pt}\hspace*{6pt} moist and wear, and have become dark, bleached or\mbox{}\newline 
\hspace*{6pt}\hspace*{6pt} wrinkled. {</\textbf{p}>}\mbox{}\newline 
\hspace*{6pt}{<\textbf{p}>}The script has been\mbox{}\newline 
\hspace*{6pt}\hspace*{6pt} touched up in the 17th century with black ink. The touching up on the following\mbox{}\newline 
\hspace*{6pt}\hspace*{6pt} fols. was done by\mbox{}\newline 
\hspace*{6pt}{<\textbf{name}>}Bishop Brynjólf Sveinsson{</\textbf{name}>}: 1v, 3r, 4r, 5r,\mbox{}\newline 
\hspace*{6pt}\hspace*{6pt} 6v, 8v,9r, 10r, 14r, 14v, 22r,30v, 36r-52v, 72v, 77r,78r,103r, 104r,. An\mbox{}\newline 
\hspace*{6pt}\hspace*{6pt} AM-note says according to the lawman\mbox{}\newline 
\hspace*{6pt}{<\textbf{name}>}Sigurður Björnsson{</\textbf{name}>} that the rest of the\mbox{}\newline 
\hspace*{6pt}\hspace*{6pt} touching up was done by himself and another lawman\mbox{}\newline 
\hspace*{6pt}{<\textbf{name}>}Sigurður Jónsson{</\textbf{name}>}.\mbox{}\newline 
\hspace*{6pt}{<\textbf{name}>}Sigurður Björnsson{</\textbf{name}>} did the touching up\mbox{}\newline 
\hspace*{6pt}\hspace*{6pt} on the following fols.: 46v, 47r, 48r, 49r-v, 50r, 52r-v.\mbox{}\newline 
\hspace*{6pt}{<\textbf{name}>}Sigurður Jónsson{</\textbf{name}>} did the rest of the\mbox{}\newline 
\hspace*{6pt}\hspace*{6pt} touching up in the section 36r-59r containing\mbox{}\newline 
\hspace*{6pt}{<\textbf{title}>}Bretasögur{</\textbf{title}>}\mbox{}\newline 
\hspace*{6pt}{</\textbf{p}>}\mbox{}\newline 
{</\textbf{condition}>}\end{shaded}\egroup 


    \item[{Content model}]
  \mbox{}\hfill\\[-10pt]\begin{Verbatim}[fontsize=\small]
<content>
 <macroRef key="macro.specialPara"/>
</content>
    
\end{Verbatim}

    \item[{Schema Declaration}]
  \mbox{}\hfill\\[-10pt]\begin{Verbatim}[fontsize=\small]
element condition { att.global.attributes, macro.specialPara }
\end{Verbatim}

\end{reflist}  \index{corr=<corr>|oddindex}
\begin{reflist}
\item[]\begin{specHead}{TEI.corr}{<corr> }(correction) contains the correct form of a passage apparently erroneous in the copy text. [\xref{http://www.tei-c.org/release/doc/tei-p5-doc/en/html/CO.html\#COEDCOR}{3.4.1. Apparent Errors}]\end{specHead} 
    \item[{Module}]
  core
    \item[{Attributes}]
  Attributes att.global (\textit{@xml:id}, \textit{@n}, \textit{@xml:lang}, \textit{@xml:base}, \textit{@xml:space})  (att.global.rendition (\textit{@rend}, \textit{@style}, \textit{@rendition})) (att.global.linking (\textit{@corresp}, \textit{@synch}, \textit{@sameAs}, \textit{@copyOf}, \textit{@next}, \textit{@prev}, \textit{@exclude}, \textit{@select})) (att.global.analytic (\textit{@ana})) (att.global.facs (\textit{@facs})) (att.global.change (\textit{@change})) (att.global.responsibility (\textit{@cert}, \textit{@resp})) (att.global.source (\textit{@source})) att.editLike (\textit{@evidence}, \textit{@instant})  (att.dimensions (\textit{@unit}, \textit{@quantity}, \textit{@extent}, \textit{@precision}, \textit{@scope}) (att.ranging (\textit{@atLeast}, \textit{@atMost}, \textit{@min}, \textit{@max}, \textit{@confidence})) ) att.typed (\textit{@type}, \textit{@subtype}) 
    \item[{Member of}]
  model.choicePart model.pPart.transcriptional
    \item[{Contained by}]
  
    \item[analysis: ]
   cl pc phr s w\par 
    \item[core: ]
   abbr add addrLine author bibl biblScope choice citedRange corr date del distinct editor email emph expan foreign gloss head headItem headLabel hi item l label measure mentioned name note num orig p pubPlace publisher q quote ref reg rs said sic soCalled speaker stage street term textLang time title unclear\par 
    \item[figures: ]
   cell\par 
    \item[header: ]
   change distributor edition extent geoDecl handNote licence scriptNote typeNote\par 
    \item[linking: ]
   ab seg\par 
    \item[msdescription: ]
   accMat acquisition additions catchwords collation colophon condition custEvent decoNote explicit filiation finalRubric foliation heraldry incipit layout material musicNotation objectType origDate origPlace origin provenance rubric secFol signatures source stamp summary support surrogates watermark\par 
    \item[namesdates: ]
   addName affiliation birth bloc country death district education faith floruit forename genName geogFeat geogName nameLink nationality occupation offset orgName persName placeName region residence roleName settlement sex socecStatus surname\par 
    \item[textcrit: ]
   lem rdg wit witDetail\par 
    \item[textstructure: ]
   byline closer dateline docAuthor docDate docEdition docImprint imprimatur opener salute signed titlePart trailer\par 
    \item[transcr: ]
   am damage fw metamark mod restore retrace secl supplied surplus
    \item[{May contain}]
  
    \item[analysis: ]
   c cl interp interpGrp m pc phr s span spanGrp w\par 
    \item[core: ]
   abbr add address bibl biblStruct cb choice cit corr date del desc distinct email emph expan foreign gap gb gloss graphic hi index l label lb lg list listBibl measure measureGrp media mentioned milestone name note num orig pb ptr q quote ref reg rs said sic soCalled stage term time title unclear\par 
    \item[figures: ]
   figure formula notatedMusic table\par 
    \item[gaiji: ]
   g\par 
    \item[header: ]
   biblFull idno\par 
    \item[linking: ]
   alt altGrp anchor join joinGrp link linkGrp seg timeline\par 
    \item[msdescription: ]
   catchwords depth dim dimensions height heraldry locus locusGrp material msDesc objectType origDate origPlace secFol signatures stamp watermark width\par 
    \item[namesdates: ]
   addName affiliation bloc climate country district forename genName geo geogFeat geogName listEvent listNym listOrg listPerson listPlace location nameLink offset orgName persName placeName population region roleName settlement state surname terrain trait\par 
    \item[textcrit: ]
   app listApp listWit witDetail\par 
    \item[textstructure: ]
   floatingText\par 
    \item[transcr: ]
   addSpan am damage damageSpan delSpan ex fw handShift listTranspose metamark mod redo restore retrace secl space subst substJoin supplied surplus undo\par character data
    \item[{Example}]
  If all that is desired is to call attention to the fact that the copy text has been corrected, <corr> may be used alone:\leavevmode\bgroup\exampleFont \begin{shaded}\noindent\mbox{}I don't know,\mbox{}\newline 
 Juan. It's so far in the past now — how {<\textbf{corr}>}can we{</\textbf{corr}>} prove\mbox{}\newline 
 or disprove anyone's theories?\end{shaded}\egroup 


    \item[{Example}]
  It is also possible, using the <choice> and <sic> elements, to provide an uncorrected reading:\leavevmode\bgroup\exampleFont \begin{shaded}\noindent\mbox{}I don't know, Juan. It's so far in the past now —\mbox{}\newline 
 how {<\textbf{choice}>}\mbox{}\newline 
\hspace*{6pt}{<\textbf{sic}>}we can{</\textbf{sic}>}\mbox{}\newline 
\hspace*{6pt}{<\textbf{corr}>}can we{</\textbf{corr}>}\mbox{}\newline 
{</\textbf{choice}>} prove or\mbox{}\newline 
 disprove anyone's theories?\end{shaded}\egroup 


    \item[{Content model}]
  \mbox{}\hfill\\[-10pt]\begin{Verbatim}[fontsize=\small]
<content>
 <macroRef key="macro.paraContent"/>
</content>
    
\end{Verbatim}

    \item[{Schema Declaration}]
  \mbox{}\hfill\\[-10pt]\begin{Verbatim}[fontsize=\small]
element corr
{
   att.global.attributes,
   att.editLike.attributes,
   att.typed.attributes,
   macro.paraContent}
\end{Verbatim}

\end{reflist}  \index{correction=<correction>|oddindex}\index{status=@status!<correction>|oddindex}\index{method=@method!<correction>|oddindex}
\begin{reflist}
\item[]\begin{specHead}{TEI.correction}{<correction> }(correction principles) states how and under what circumstances corrections have been made in the text. [\xref{http://www.tei-c.org/release/doc/tei-p5-doc/en/html/HD.html\#HD53}{2.3.3. The Editorial Practices Declaration} \xref{http://www.tei-c.org/release/doc/tei-p5-doc/en/html/CC.html\#CCAS2}{15.3.2. Declarable Elements}]\end{specHead} 
    \item[{Module}]
  header
    \item[{Attributes}]
  Attributes att.global (\textit{@xml:id}, \textit{@n}, \textit{@xml:lang}, \textit{@xml:base}, \textit{@xml:space})  (att.global.rendition (\textit{@rend}, \textit{@style}, \textit{@rendition})) (att.global.linking (\textit{@corresp}, \textit{@synch}, \textit{@sameAs}, \textit{@copyOf}, \textit{@next}, \textit{@prev}, \textit{@exclude}, \textit{@select})) (att.global.analytic (\textit{@ana})) (att.global.facs (\textit{@facs})) (att.global.change (\textit{@change})) (att.global.responsibility (\textit{@cert}, \textit{@resp})) (att.global.source (\textit{@source})) att.declarable (\textit{@default}) \hfil\\[-10pt]\begin{sansreflist}
    \item[@status]
  indicates the degree of correction applied to the text.
\begin{reflist}
    \item[{Status}]
  Optional
    \item[{Datatype}]
  teidata.enumerated
    \item[{Legal values are:}]
  \begin{description}

\item[{high}]the text has been thoroughly checked and proofread.
\item[{medium}]the text has been checked at least once.
\item[{low}]the text has not been checked.
\item[{unknown}]the correction status of the text is unknown.{[Default] \xref{http://www.tei-c.org/Activities/Council/Working/tcw27.xml}{Deprecated}. The value will no longer be a default after 2017-09-05.}
\end{description} 
\end{reflist}  
    \item[@method]
  indicates the method adopted to indicate corrections within the text.
\begin{reflist}
    \item[{Status}]
  Optional
    \item[{Datatype}]
  teidata.enumerated
    \item[{Legal values are:}]
  \begin{description}

\item[{silent}]corrections have been made silently{[Default] }
\item[{markup}]corrections have been represented using markup
\end{description} 
\end{reflist}  
\end{sansreflist}  
    \item[{Member of}]
  model.editorialDeclPart
    \item[{Contained by}]
  
    \item[header: ]
   editorialDecl
    \item[{May contain}]
  
    \item[core: ]
   p\par 
    \item[linking: ]
   ab
    \item[{Note}]
  \par
May be used to note the results of proof reading the text against its original, indicating (for example) whether discrepancies have been silently rectified, or recorded using the editorial tags described in section \xref{http://www.tei-c.org/release/doc/tei-p5-doc/en/html/CO.html\#COED}{3.4. Simple Editorial Changes}.
    \item[{Example}]
  \leavevmode\bgroup\exampleFont \begin{shaded}\noindent\mbox{}{<\textbf{correction}>}\mbox{}\newline 
\hspace*{6pt}{<\textbf{p}>}Errors in transcription controlled by using the WordPerfect spelling checker, with a user\mbox{}\newline 
\hspace*{6pt}\hspace*{6pt} defined dictionary of 500 extra words taken from Chambers Twentieth Century\mbox{}\newline 
\hspace*{6pt}\hspace*{6pt} Dictionary.{</\textbf{p}>}\mbox{}\newline 
{</\textbf{correction}>}\end{shaded}\egroup 


    \item[{Content model}]
  \mbox{}\hfill\\[-10pt]\begin{Verbatim}[fontsize=\small]
<content>
 <classRef key="model.pLike"
  maxOccurs="unbounded" minOccurs="1"/>
</content>
    
\end{Verbatim}

    \item[{Schema Declaration}]
  \mbox{}\hfill\\[-10pt]\begin{Verbatim}[fontsize=\small]
element correction
{
   att.global.attributes,
   att.declarable.attributes,
   attribute status { "high" | "medium" | "low" | "unknown" }?,
   attribute method { "silent" | "markup" }?,
   model.pLike+
}
\end{Verbatim}

\end{reflist}  \index{correspAction=<correspAction>|oddindex}\index{type=@type!<correspAction>|oddindex}
\begin{reflist}
\item[]\begin{specHead}{TEI.correspAction}{<correspAction> }(correspondence action) contains a structured description of the place, the name of a person/organization and the date related to the sending/receiving of a message or any other action related to the correspondence. [\xref{http://www.tei-c.org/release/doc/tei-p5-doc/en/html/HD.html\#HD44CD}{2.4.6. Correspondence Description}]\end{specHead} 
    \item[{Module}]
  header
    \item[{Attributes}]
  Attributes att.global (\textit{@xml:id}, \textit{@n}, \textit{@xml:lang}, \textit{@xml:base}, \textit{@xml:space})  (att.global.rendition (\textit{@rend}, \textit{@style}, \textit{@rendition})) (att.global.linking (\textit{@corresp}, \textit{@synch}, \textit{@sameAs}, \textit{@copyOf}, \textit{@next}, \textit{@prev}, \textit{@exclude}, \textit{@select})) (att.global.analytic (\textit{@ana})) (att.global.facs (\textit{@facs})) (att.global.change (\textit{@change})) (att.global.responsibility (\textit{@cert}, \textit{@resp})) (att.global.source (\textit{@source})) att.sortable (\textit{@sortKey}) att.typed (\unusedattribute{type}, @subtype) \hfil\\[-10pt]\begin{sansreflist}
    \item[@type]
  describes the nature of the action.
\begin{reflist}
    \item[{Derived from}]
  att.typed
    \item[{Status}]
  Optional
    \item[{Datatype}]
  teidata.enumerated
    \item[{Suggested values include:}]
  \begin{description}

\item[{sent}]information concerning the sending or dispatch of a message.
\item[{received}]information concerning the receipt of a message.
\item[{transmitted}]information concerning the transmission of a message, i.e. between the dispatch and the next receipt, redirect or forwarding.
\item[{redirected}]information concerning the redirection of an unread message.
\item[{forwarded}]information concerning the forwarding of a message.
\end{description} 
\end{reflist}  
\end{sansreflist}  
    \item[{Member of}]
  model.correspDescPart
    \item[{Contained by}]
  
    \item[header: ]
   correspDesc
    \item[{May contain}]
  
    \item[core: ]
   address date email name note p rs time\par 
    \item[header: ]
   idno\par 
    \item[linking: ]
   ab\par 
    \item[namesdates: ]
   addName affiliation bloc climate country district forename genName geogFeat geogName location nameLink offset orgName persName placeName population region roleName settlement state surname terrain trait
    \item[{Example}]
  \leavevmode\bgroup\exampleFont \begin{shaded}\noindent\mbox{}{<\textbf{correspAction}\hspace*{6pt}{type}="{sent}">}\mbox{}\newline 
\hspace*{6pt}{<\textbf{persName}>}Adelbert von Chamisso{</\textbf{persName}>}\mbox{}\newline 
\hspace*{6pt}{<\textbf{settlement}>}Vertus{</\textbf{settlement}>}\mbox{}\newline 
\hspace*{6pt}{<\textbf{date}\hspace*{6pt}{when}="{1807-01-29}"/>}\mbox{}\newline 
{</\textbf{correspAction}>}\end{shaded}\egroup 


    \item[{Content model}]
  \mbox{}\hfill\\[-10pt]\begin{Verbatim}[fontsize=\small]
<content>
 <alternate>
  <classRef key="model.correspActionPart"
   maxOccurs="unbounded" minOccurs="1"/>
  <classRef key="model.pLike"
   maxOccurs="unbounded" minOccurs="1"/>
 </alternate>
</content>
    
\end{Verbatim}

    \item[{Schema Declaration}]
  \mbox{}\hfill\\[-10pt]\begin{Verbatim}[fontsize=\small]
element correspAction
{
   att.global.attributes,
   att.typed.attribute.subtype,
   att.sortable.attributes,
   attribute type
   {
      "sent" | "received" | "transmitted" | "redirected" | "forwarded"
   }?,
   ( model.correspActionPart+ | model.pLike+ )
}
\end{Verbatim}

\end{reflist}  \index{correspContext=<correspContext>|oddindex}
\begin{reflist}
\item[]\begin{specHead}{TEI.correspContext}{<correspContext> }(correspondence context) provides references to preceding or following correspondence related to this piece of correspondence. [\xref{http://www.tei-c.org/release/doc/tei-p5-doc/en/html/HD.html\#HD44CD}{2.4.6. Correspondence Description}]\end{specHead} 
    \item[{Module}]
  header
    \item[{Attributes}]
  Attributes att.global (\textit{@xml:id}, \textit{@n}, \textit{@xml:lang}, \textit{@xml:base}, \textit{@xml:space})  (att.global.rendition (\textit{@rend}, \textit{@style}, \textit{@rendition})) (att.global.linking (\textit{@corresp}, \textit{@synch}, \textit{@sameAs}, \textit{@copyOf}, \textit{@next}, \textit{@prev}, \textit{@exclude}, \textit{@select})) (att.global.analytic (\textit{@ana})) (att.global.facs (\textit{@facs})) (att.global.change (\textit{@change})) (att.global.responsibility (\textit{@cert}, \textit{@resp})) (att.global.source (\textit{@source}))
    \item[{Member of}]
  model.correspDescPart
    \item[{Contained by}]
  
    \item[header: ]
   correspDesc
    \item[{May contain}]
  
    \item[core: ]
   note p ptr ref\par 
    \item[linking: ]
   ab
    \item[{Example}]
  \leavevmode\bgroup\exampleFont \begin{shaded}\noindent\mbox{}{<\textbf{correspContext}>}\mbox{}\newline 
\hspace*{6pt}{<\textbf{ptr}\hspace*{6pt}{subtype}="{toAuthor}"\mbox{}\newline 
\hspace*{6pt}\hspace*{6pt}{target}="{http://tei.ibi.hu-berlin.de/berliner-intellektuelle/manuscript?Brief101VarnhagenanBoeckh}"\hspace*{6pt}{type}="{next}"/>}\mbox{}\newline 
\hspace*{6pt}{<\textbf{ptr}\hspace*{6pt}{subtype}="{fromAuthor}"\mbox{}\newline 
\hspace*{6pt}\hspace*{6pt}{target}="{http://tei.ibi.hu-berlin.de/berliner-intellektuelle/manuscript?Brief103BoeckhanVarnhagen}"\hspace*{6pt}{type}="{prev}"/>}\mbox{}\newline 
{</\textbf{correspContext}>}\end{shaded}\egroup 


    \item[{Example}]
  \leavevmode\bgroup\exampleFont \begin{shaded}\noindent\mbox{}{<\textbf{correspContext}>}\mbox{}\newline 
\hspace*{6pt}{<\textbf{ref}\hspace*{6pt}{target}="{http://weber-gesamtausgabe.de/A040962}"\mbox{}\newline 
\hspace*{6pt}\hspace*{6pt}{type}="{prev}">} Previous letter of\mbox{}\newline 
\hspace*{6pt}{<\textbf{persName}>}Carl Maria von Weber{</\textbf{persName}>} to\mbox{}\newline 
\hspace*{6pt}{<\textbf{persName}>}Caroline Brandt{</\textbf{persName}>}:\mbox{}\newline 
\hspace*{6pt}{<\textbf{date}\hspace*{6pt}{when}="{1816-12-30}">}December 30, 1816{</\textbf{date}>}\mbox{}\newline 
\hspace*{6pt}{</\textbf{ref}>}\mbox{}\newline 
\hspace*{6pt}{<\textbf{ref}\hspace*{6pt}{target}="{http://weber-gesamtausgabe.de/A041003}"\mbox{}\newline 
\hspace*{6pt}\hspace*{6pt}{type}="{next}">} Next letter of\mbox{}\newline 
\hspace*{6pt}{<\textbf{persName}>}Carl Maria von Weber{</\textbf{persName}>} to\mbox{}\newline 
\hspace*{6pt}{<\textbf{persName}>}Caroline Brandt{</\textbf{persName}>}:\mbox{}\newline 
\hspace*{6pt}{<\textbf{date}\hspace*{6pt}{when}="{1817-01-05}">}January 5, 1817{</\textbf{date}>}\mbox{}\newline 
\hspace*{6pt}{</\textbf{ref}>}\mbox{}\newline 
{</\textbf{correspContext}>}\end{shaded}\egroup 


    \item[{Content model}]
  \mbox{}\hfill\\[-10pt]\begin{Verbatim}[fontsize=\small]
<content>
 <classRef key="model.correspContextPart"
  maxOccurs="unbounded" minOccurs="1"/>
</content>
    
\end{Verbatim}

    \item[{Schema Declaration}]
  \mbox{}\hfill\\[-10pt]\begin{Verbatim}[fontsize=\small]
element correspContext { att.global.attributes, model.correspContextPart+ }
\end{Verbatim}

\end{reflist}  \index{correspDesc=<correspDesc>|oddindex}
\begin{reflist}
\item[]\begin{specHead}{TEI.correspDesc}{<correspDesc> }(correspondence description) contains a description of the actions related to one act of correspondence. [\xref{http://www.tei-c.org/release/doc/tei-p5-doc/en/html/HD.html\#HD44CD}{2.4.6. Correspondence Description}]\end{specHead} 
    \item[{Module}]
  header
    \item[{Attributes}]
  Attributes att.declarable (\textit{@default}) att.canonical (\textit{@key}, \textit{@ref}) att.global (\textit{@xml:id}, \textit{@n}, \textit{@xml:lang}, \textit{@xml:base}, \textit{@xml:space})  (att.global.rendition (\textit{@rend}, \textit{@style}, \textit{@rendition})) (att.global.linking (\textit{@corresp}, \textit{@synch}, \textit{@sameAs}, \textit{@copyOf}, \textit{@next}, \textit{@prev}, \textit{@exclude}, \textit{@select})) (att.global.analytic (\textit{@ana})) (att.global.facs (\textit{@facs})) (att.global.change (\textit{@change})) (att.global.responsibility (\textit{@cert}, \textit{@resp})) (att.global.source (\textit{@source})) att.typed (\textit{@type}, \textit{@subtype}) 
    \item[{Member of}]
  model.profileDescPart
    \item[{Contained by}]
  
    \item[header: ]
   profileDesc
    \item[{May contain}]
  
    \item[core: ]
   note p\par 
    \item[header: ]
   correspAction correspContext\par 
    \item[linking: ]
   ab
    \item[{Example}]
  \leavevmode\bgroup\exampleFont \begin{shaded}\noindent\mbox{}{<\textbf{correspDesc}>}\mbox{}\newline 
\hspace*{6pt}{<\textbf{correspAction}\hspace*{6pt}{type}="{sent}">}\mbox{}\newline 
\hspace*{6pt}\hspace*{6pt}{<\textbf{persName}>}Carl Maria von Weber{</\textbf{persName}>}\mbox{}\newline 
\hspace*{6pt}\hspace*{6pt}{<\textbf{settlement}>}Dresden{</\textbf{settlement}>}\mbox{}\newline 
\hspace*{6pt}\hspace*{6pt}{<\textbf{date}\hspace*{6pt}{when}="{1817-06-23}">}23 June 1817{</\textbf{date}>}\mbox{}\newline 
\hspace*{6pt}{</\textbf{correspAction}>}\mbox{}\newline 
\hspace*{6pt}{<\textbf{correspAction}\hspace*{6pt}{type}="{received}">}\mbox{}\newline 
\hspace*{6pt}\hspace*{6pt}{<\textbf{persName}>}Caroline Brandt{</\textbf{persName}>}\mbox{}\newline 
\hspace*{6pt}\hspace*{6pt}{<\textbf{settlement}>}Prag{</\textbf{settlement}>}\mbox{}\newline 
\hspace*{6pt}{</\textbf{correspAction}>}\mbox{}\newline 
\hspace*{6pt}{<\textbf{correspContext}>}\mbox{}\newline 
\hspace*{6pt}\hspace*{6pt}{<\textbf{ref}\hspace*{6pt}{target}="{http://www.weber-gesamtausgabe.de/A041209}"\mbox{}\newline 
\hspace*{6pt}\hspace*{6pt}\hspace*{6pt}{type}="{prev}">}Previous letter of\mbox{}\newline 
\hspace*{6pt}\hspace*{6pt}{<\textbf{persName}>}Carl Maria von Weber{</\textbf{persName}>} \mbox{}\newline 
\hspace*{6pt}\hspace*{6pt}\hspace*{6pt}\hspace*{6pt} to {<\textbf{persName}>}Caroline Brandt{</\textbf{persName}>}:\mbox{}\newline 
\hspace*{6pt}\hspace*{6pt}{<\textbf{date}\hspace*{6pt}{from}="{1817-06-19}"\hspace*{6pt}{to}="{1817-06-20}">}June 19/20, 1817{</\textbf{date}>}\mbox{}\newline 
\hspace*{6pt}\hspace*{6pt}{</\textbf{ref}>}\mbox{}\newline 
\hspace*{6pt}\hspace*{6pt}{<\textbf{ref}\hspace*{6pt}{target}="{http://www.weber-gesamtausgabe.de/A041217}"\mbox{}\newline 
\hspace*{6pt}\hspace*{6pt}\hspace*{6pt}{type}="{next}">}Next letter of\mbox{}\newline 
\hspace*{6pt}\hspace*{6pt}{<\textbf{persName}>}Carl Maria von Weber{</\textbf{persName}>} to\mbox{}\newline 
\hspace*{6pt}\hspace*{6pt}{<\textbf{persName}>}Caroline Brandt{</\textbf{persName}>}:\mbox{}\newline 
\hspace*{6pt}\hspace*{6pt}{<\textbf{date}\hspace*{6pt}{when}="{1817-06-27}">}June 27, 1817{</\textbf{date}>}\mbox{}\newline 
\hspace*{6pt}\hspace*{6pt}{</\textbf{ref}>}\mbox{}\newline 
\hspace*{6pt}{</\textbf{correspContext}>}\mbox{}\newline 
{</\textbf{correspDesc}>}\end{shaded}\egroup 


    \item[{Content model}]
  \mbox{}\hfill\\[-10pt]\begin{Verbatim}[fontsize=\small]
<content>
 <alternate>
  <classRef key="model.correspDescPart"
   maxOccurs="unbounded" minOccurs="1"/>
  <classRef key="model.pLike"
   maxOccurs="unbounded" minOccurs="1"/>
 </alternate>
</content>
    
\end{Verbatim}

    \item[{Schema Declaration}]
  \mbox{}\hfill\\[-10pt]\begin{Verbatim}[fontsize=\small]
element correspDesc
{
   att.declarable.attributes,
   att.canonical.attributes,
   att.global.attributes,
   att.typed.attributes,
   ( model.correspDescPart+ | model.pLike+ )
}
\end{Verbatim}

\end{reflist}  \index{country=<country>|oddindex}
\begin{reflist}
\item[]\begin{specHead}{TEI.country}{<country> }contains the name of a geo-political unit, such as a nation, country, colony, or commonwealth, larger than or administratively superior to a region and smaller than a bloc. [\xref{http://www.tei-c.org/release/doc/tei-p5-doc/en/html/ND.html\#NDPLAC}{13.2.3. Place Names}]\end{specHead} 
    \item[{Module}]
  namesdates
    \item[{Attributes}]
  Attributes att.global (\textit{@xml:id}, \textit{@n}, \textit{@xml:lang}, \textit{@xml:base}, \textit{@xml:space})  (att.global.rendition (\textit{@rend}, \textit{@style}, \textit{@rendition})) (att.global.linking (\textit{@corresp}, \textit{@synch}, \textit{@sameAs}, \textit{@copyOf}, \textit{@next}, \textit{@prev}, \textit{@exclude}, \textit{@select})) (att.global.analytic (\textit{@ana})) (att.global.facs (\textit{@facs})) (att.global.change (\textit{@change})) (att.global.responsibility (\textit{@cert}, \textit{@resp})) (att.global.source (\textit{@source})) att.naming (\textit{@role}, \textit{@nymRef})  (att.canonical (\textit{@key}, \textit{@ref})) att.typed (\textit{@type}, \textit{@subtype}) att.datable (\textit{@calendar}, \textit{@period})  (att.datable.w3c (\textit{@when}, \textit{@notBefore}, \textit{@notAfter}, \textit{@from}, \textit{@to})) (att.datable.iso (\textit{@when-iso}, \textit{@notBefore-iso}, \textit{@notAfter-iso}, \textit{@from-iso}, \textit{@to-iso})) (att.datable.custom (\textit{@when-custom}, \textit{@notBefore-custom}, \textit{@notAfter-custom}, \textit{@from-custom}, \textit{@to-custom}, \textit{@datingPoint}, \textit{@datingMethod}))
    \item[{Member of}]
  model.placeNamePart
    \item[{Contained by}]
  
    \item[analysis: ]
   cl phr s span\par 
    \item[core: ]
   abbr add addrLine address author bibl biblScope citedRange corr date del desc distinct editor email emph expan foreign gloss head headItem headLabel hi item l label measure meeting mentioned name note num orig p pubPlace publisher q quote ref reg resp rs said sic soCalled speaker stage street term textLang time title unclear\par 
    \item[figures: ]
   cell figDesc\par 
    \item[header: ]
   authority catDesc change classCode correspAction creation distributor edition extent funder geoDecl handNote language licence principal rendition scriptNote sponsor tagUsage typeNote\par 
    \item[linking: ]
   ab seg\par 
    \item[msdescription: ]
   accMat acquisition additions altIdentifier catchwords collation colophon condition custEvent decoNote explicit filiation finalRubric foliation heraldry incipit layout material msIdentifier musicNotation objectType origDate origPlace origin provenance rubric secFol signatures source stamp summary support surrogates watermark\par 
    \item[namesdates: ]
   addName affiliation age birth bloc country death district education faith floruit forename genName geogFeat geogName langKnown location nameLink nationality occupation offset org orgName persName place placeName region residence roleName settlement sex socecStatus surname\par 
    \item[textcrit: ]
   lem rdg wit witDetail witness\par 
    \item[textstructure: ]
   byline closer dateline docAuthor docDate docEdition docImprint imprimatur opener salute signed titlePart trailer\par 
    \item[transcr: ]
   damage fw metamark mod restore retrace secl supplied surplus
    \item[{May contain}]
  
    \item[analysis: ]
   c cl interp interpGrp m pc phr s span spanGrp w\par 
    \item[core: ]
   abbr add address cb choice corr date del distinct email emph expan foreign gap gb gloss graphic hi index lb measure measureGrp media mentioned milestone name note num orig pb ptr ref reg rs sic soCalled term time title unclear\par 
    \item[figures: ]
   figure formula notatedMusic\par 
    \item[gaiji: ]
   g\par 
    \item[header: ]
   idno\par 
    \item[linking: ]
   alt altGrp anchor join joinGrp link linkGrp seg timeline\par 
    \item[msdescription: ]
   catchwords depth dim dimensions height heraldry locus locusGrp material objectType origDate origPlace secFol signatures stamp watermark width\par 
    \item[namesdates: ]
   addName affiliation bloc climate country district forename genName geo geogFeat geogName location nameLink offset orgName persName placeName population region roleName settlement state surname terrain trait\par 
    \item[textcrit: ]
   app witDetail\par 
    \item[transcr: ]
   addSpan am damage damageSpan delSpan ex fw handShift listTranspose metamark mod redo restore retrace secl space subst substJoin supplied surplus undo\par character data
    \item[{Note}]
  \par
The recommended source for codes to represent coded country names is ISO 3166.
    \item[{Example}]
  \leavevmode\bgroup\exampleFont \begin{shaded}\noindent\mbox{}{<\textbf{country}\hspace*{6pt}{key}="{DK}">}Denmark{</\textbf{country}>}\end{shaded}\egroup 


    \item[{Content model}]
  \mbox{}\hfill\\[-10pt]\begin{Verbatim}[fontsize=\small]
<content>
 <macroRef key="macro.phraseSeq"/>
</content>
    
\end{Verbatim}

    \item[{Schema Declaration}]
  \mbox{}\hfill\\[-10pt]\begin{Verbatim}[fontsize=\small]
element country
{
   att.global.attributes,
   att.naming.attributes,
   att.typed.attributes,
   att.datable.attributes,
   macro.phraseSeq}
\end{Verbatim}

\end{reflist}  \index{creation=<creation>|oddindex}
\begin{reflist}
\item[]\begin{specHead}{TEI.creation}{<creation> }contains information about the creation of a text. [\xref{http://www.tei-c.org/release/doc/tei-p5-doc/en/html/HD.html\#HD4C}{2.4.1. Creation} \xref{http://www.tei-c.org/release/doc/tei-p5-doc/en/html/HD.html\#HD4}{2.4. The Profile Description}]\end{specHead} 
    \item[{Module}]
  header
    \item[{Attributes}]
  Attributes att.global (\textit{@xml:id}, \textit{@n}, \textit{@xml:lang}, \textit{@xml:base}, \textit{@xml:space})  (att.global.rendition (\textit{@rend}, \textit{@style}, \textit{@rendition})) (att.global.linking (\textit{@corresp}, \textit{@synch}, \textit{@sameAs}, \textit{@copyOf}, \textit{@next}, \textit{@prev}, \textit{@exclude}, \textit{@select})) (att.global.analytic (\textit{@ana})) (att.global.facs (\textit{@facs})) (att.global.change (\textit{@change})) (att.global.responsibility (\textit{@cert}, \textit{@resp})) (att.global.source (\textit{@source})) att.datable (\textit{@calendar}, \textit{@period})  (att.datable.w3c (\textit{@when}, \textit{@notBefore}, \textit{@notAfter}, \textit{@from}, \textit{@to})) (att.datable.iso (\textit{@when-iso}, \textit{@notBefore-iso}, \textit{@notAfter-iso}, \textit{@from-iso}, \textit{@to-iso})) (att.datable.custom (\textit{@when-custom}, \textit{@notBefore-custom}, \textit{@notAfter-custom}, \textit{@from-custom}, \textit{@to-custom}, \textit{@datingPoint}, \textit{@datingMethod}))
    \item[{Member of}]
  model.profileDescPart
    \item[{Contained by}]
  
    \item[header: ]
   profileDesc
    \item[{May contain}]
  
    \item[core: ]
   abbr address choice date distinct email emph expan foreign gloss hi measure measureGrp mentioned name num ptr ref rs soCalled term time title\par 
    \item[header: ]
   idno listChange\par 
    \item[msdescription: ]
   catchwords depth dim dimensions height heraldry locus locusGrp material objectType origDate origPlace secFol signatures stamp watermark width\par 
    \item[namesdates: ]
   addName affiliation bloc climate country district forename genName geo geogFeat geogName location nameLink offset orgName persName placeName population region roleName settlement state surname terrain trait\par 
    \item[transcr: ]
   am ex subst\par character data
    \item[{Note}]
  \par
The <creation> element may be used to record details of a text's creation, e.g. the date and place it was composed, if these are of interest.\par
It may also contain a more structured account of the various stages or revisions associated with the evolution of a text; this should be encoded using the <listChange> element. It should not be confused with the <publicationStmt> element, which records date and place of publication.
    \item[{Example}]
  \leavevmode\bgroup\exampleFont \begin{shaded}\noindent\mbox{}{<\textbf{creation}>}\mbox{}\newline 
\hspace*{6pt}{<\textbf{date}>}Before 1987{</\textbf{date}>}\mbox{}\newline 
{</\textbf{creation}>}\end{shaded}\egroup 


    \item[{Example}]
  \leavevmode\bgroup\exampleFont \begin{shaded}\noindent\mbox{}{<\textbf{creation}>}\mbox{}\newline 
\hspace*{6pt}{<\textbf{date}\hspace*{6pt}{when}="{1988-07-10}">}10 July 1988{</\textbf{date}>}\mbox{}\newline 
{</\textbf{creation}>}\end{shaded}\egroup 


    \item[{Content model}]
  \mbox{}\hfill\\[-10pt]\begin{Verbatim}[fontsize=\small]
<content>
 <alternate maxOccurs="unbounded"
  minOccurs="0">
  <textNode/>
  <classRef key="model.limitedPhrase"/>
  <elementRef key="listChange"/>
 </alternate>
</content>
    
\end{Verbatim}

    \item[{Schema Declaration}]
  \mbox{}\hfill\\[-10pt]\begin{Verbatim}[fontsize=\small]
element creation
{
   att.global.attributes,
   att.datable.attributes,
   ( text | model.limitedPhrase | listChange )*
}
\end{Verbatim}

\end{reflist}  \index{custEvent=<custEvent>|oddindex}
\begin{reflist}
\item[]\begin{specHead}{TEI.custEvent}{<custEvent> }(custodial event) describes a single event during the custodial history of a manuscript. [\xref{http://www.tei-c.org/release/doc/tei-p5-doc/en/html/MS.html\#msadch}{10.9.1.2. Availability and Custodial History}]\end{specHead} 
    \item[{Module}]
  msdescription
    \item[{Attributes}]
  Attributes att.global (\textit{@xml:id}, \textit{@n}, \textit{@xml:lang}, \textit{@xml:base}, \textit{@xml:space})  (att.global.rendition (\textit{@rend}, \textit{@style}, \textit{@rendition})) (att.global.linking (\textit{@corresp}, \textit{@synch}, \textit{@sameAs}, \textit{@copyOf}, \textit{@next}, \textit{@prev}, \textit{@exclude}, \textit{@select})) (att.global.analytic (\textit{@ana})) (att.global.facs (\textit{@facs})) (att.global.change (\textit{@change})) (att.global.responsibility (\textit{@cert}, \textit{@resp})) (att.global.source (\textit{@source})) att.datable (\textit{@calendar}, \textit{@period})  (att.datable.w3c (\textit{@when}, \textit{@notBefore}, \textit{@notAfter}, \textit{@from}, \textit{@to})) (att.datable.iso (\textit{@when-iso}, \textit{@notBefore-iso}, \textit{@notAfter-iso}, \textit{@from-iso}, \textit{@to-iso})) (att.datable.custom (\textit{@when-custom}, \textit{@notBefore-custom}, \textit{@notAfter-custom}, \textit{@from-custom}, \textit{@to-custom}, \textit{@datingPoint}, \textit{@datingMethod})) att.typed (\textit{@type}, \textit{@subtype}) 
    \item[{Contained by}]
  
    \item[msdescription: ]
   custodialHist
    \item[{May contain}]
  
    \item[analysis: ]
   c cl interp interpGrp m pc phr s span spanGrp w\par 
    \item[core: ]
   abbr add address bibl biblStruct cb choice cit corr date del desc distinct email emph expan foreign gap gb gloss graphic hi index l label lb lg list listBibl measure measureGrp media mentioned milestone name note num orig p pb ptr q quote ref reg rs said sic soCalled sp stage term time title unclear\par 
    \item[figures: ]
   figure formula notatedMusic table\par 
    \item[gaiji: ]
   g\par 
    \item[header: ]
   biblFull idno\par 
    \item[linking: ]
   ab alt altGrp anchor join joinGrp link linkGrp seg timeline\par 
    \item[msdescription: ]
   catchwords depth dim dimensions height heraldry locus locusGrp material msDesc objectType origDate origPlace secFol signatures stamp watermark width\par 
    \item[namesdates: ]
   addName affiliation bloc climate country district forename genName geo geogFeat geogName listEvent listNym listOrg listPerson listPlace location nameLink offset orgName persName placeName population region roleName settlement state surname terrain trait\par 
    \item[textcrit: ]
   app listApp listWit witDetail\par 
    \item[textstructure: ]
   floatingText\par 
    \item[transcr: ]
   addSpan am damage damageSpan delSpan ex fw handShift listTranspose metamark mod redo restore retrace secl space subst substJoin supplied surplus undo\par character data
    \item[{Example}]
  \leavevmode\bgroup\exampleFont \begin{shaded}\noindent\mbox{}{<\textbf{custEvent}\hspace*{6pt}{type}="{photography}">}Photographed by David Cooper on {<\textbf{date}>}12 Dec 1964{</\textbf{date}>}\mbox{}\newline 
{</\textbf{custEvent}>}\end{shaded}\egroup 


    \item[{Content model}]
  \mbox{}\hfill\\[-10pt]\begin{Verbatim}[fontsize=\small]
<content>
 <macroRef key="macro.specialPara"/>
</content>
    
\end{Verbatim}

    \item[{Schema Declaration}]
  \mbox{}\hfill\\[-10pt]\begin{Verbatim}[fontsize=\small]
element custEvent
{
   att.global.attributes,
   att.datable.attributes,
   att.typed.attributes,
   macro.specialPara}
\end{Verbatim}

\end{reflist}  \index{custodialHist=<custodialHist>|oddindex}
\begin{reflist}
\item[]\begin{specHead}{TEI.custodialHist}{<custodialHist> }(custodial history) contains a description of a manuscript's custodial history, either as running prose or as a series of dated custodial events. [\xref{http://www.tei-c.org/release/doc/tei-p5-doc/en/html/MS.html\#msadch}{10.9.1.2. Availability and Custodial History}]\end{specHead} 
    \item[{Module}]
  msdescription
    \item[{Attributes}]
  Attributes att.global (\textit{@xml:id}, \textit{@n}, \textit{@xml:lang}, \textit{@xml:base}, \textit{@xml:space})  (att.global.rendition (\textit{@rend}, \textit{@style}, \textit{@rendition})) (att.global.linking (\textit{@corresp}, \textit{@synch}, \textit{@sameAs}, \textit{@copyOf}, \textit{@next}, \textit{@prev}, \textit{@exclude}, \textit{@select})) (att.global.analytic (\textit{@ana})) (att.global.facs (\textit{@facs})) (att.global.change (\textit{@change})) (att.global.responsibility (\textit{@cert}, \textit{@resp})) (att.global.source (\textit{@source}))
    \item[{Contained by}]
  
    \item[msdescription: ]
   adminInfo
    \item[{May contain}]
  
    \item[core: ]
   p\par 
    \item[linking: ]
   ab\par 
    \item[msdescription: ]
   custEvent
    \item[{Example}]
  \leavevmode\bgroup\exampleFont \begin{shaded}\noindent\mbox{}{<\textbf{custodialHist}>}\mbox{}\newline 
\hspace*{6pt}{<\textbf{custEvent}\hspace*{6pt}{notAfter}="{1963-02}"\mbox{}\newline 
\hspace*{6pt}\hspace*{6pt}{notBefore}="{1961-03}"\hspace*{6pt}{type}="{conservation}">}Conserved between March 1961 and February 1963 at\mbox{}\newline 
\hspace*{6pt}\hspace*{6pt} Birgitte Dalls Konserveringsværksted.{</\textbf{custEvent}>}\mbox{}\newline 
\hspace*{6pt}{<\textbf{custEvent}\hspace*{6pt}{notAfter}="{1988-05-30}"\mbox{}\newline 
\hspace*{6pt}\hspace*{6pt}{notBefore}="{1988-05-01}"\hspace*{6pt}{type}="{photography}">}Photographed in\mbox{}\newline 
\hspace*{6pt}\hspace*{6pt} May 1988 by AMI/FA.{</\textbf{custEvent}>}\mbox{}\newline 
\hspace*{6pt}{<\textbf{custEvent}\hspace*{6pt}{notAfter}="{1989-11-13}"\mbox{}\newline 
\hspace*{6pt}\hspace*{6pt}{notBefore}="{1989-11-13}"\hspace*{6pt}{type}="{transfer-dispatch}">}Dispatched to Iceland\mbox{}\newline 
\hspace*{6pt}\hspace*{6pt} 13 November 1989.{</\textbf{custEvent}>}\mbox{}\newline 
{</\textbf{custodialHist}>}\end{shaded}\egroup 


    \item[{Content model}]
  \mbox{}\hfill\\[-10pt]\begin{Verbatim}[fontsize=\small]
<content>
 <alternate>
  <classRef key="model.pLike"
   maxOccurs="unbounded" minOccurs="1"/>
  <elementRef key="custEvent"
   maxOccurs="unbounded" minOccurs="1"/>
 </alternate>
</content>
    
\end{Verbatim}

    \item[{Schema Declaration}]
  \mbox{}\hfill\\[-10pt]\begin{Verbatim}[fontsize=\small]
element custodialHist { att.global.attributes, ( model.pLike+ | custEvent+ ) }
\end{Verbatim}

\end{reflist}  \index{damage=<damage>|oddindex}
\begin{reflist}
\item[]\begin{specHead}{TEI.damage}{<damage> }contains an area of damage to the text witness. [\xref{http://www.tei-c.org/release/doc/tei-p5-doc/en/html/PH.html\#PHDA}{11.3.3.1. Damage, Illegibility, and Supplied Text}]\end{specHead} 
    \item[{Module}]
  transcr
    \item[{Attributes}]
  Attributes att.global (\textit{@xml:id}, \textit{@n}, \textit{@xml:lang}, \textit{@xml:base}, \textit{@xml:space})  (att.global.rendition (\textit{@rend}, \textit{@style}, \textit{@rendition})) (att.global.linking (\textit{@corresp}, \textit{@synch}, \textit{@sameAs}, \textit{@copyOf}, \textit{@next}, \textit{@prev}, \textit{@exclude}, \textit{@select})) (att.global.analytic (\textit{@ana})) (att.global.facs (\textit{@facs})) (att.global.change (\textit{@change})) (att.global.responsibility (\textit{@cert}, \textit{@resp})) (att.global.source (\textit{@source})) att.typed (\textit{@type}, \textit{@subtype}) att.damaged (\textit{@agent}, \textit{@degree}, \textit{@group})  (att.dimensions (\textit{@unit}, \textit{@quantity}, \textit{@extent}, \textit{@precision}, \textit{@scope}) (att.ranging (\textit{@atLeast}, \textit{@atMost}, \textit{@min}, \textit{@max}, \textit{@confidence})) ) (att.written (\textit{@hand}))
    \item[{Member of}]
  model.linePart model.pPart.transcriptional
    \item[{Contained by}]
  
    \item[analysis: ]
   cl pc phr s w\par 
    \item[core: ]
   abbr add addrLine author bibl biblScope citedRange corr date del distinct editor email emph expan foreign gloss head headItem headLabel hi item l label measure mentioned name note num orig p pubPlace publisher q quote ref reg rs said sic soCalled speaker stage street term textLang time title unclear\par 
    \item[figures: ]
   cell\par 
    \item[header: ]
   change distributor edition extent geoDecl handNote licence scriptNote typeNote\par 
    \item[linking: ]
   ab seg\par 
    \item[msdescription: ]
   accMat acquisition additions catchwords collation colophon condition custEvent decoNote explicit filiation finalRubric foliation heraldry incipit layout material musicNotation objectType origDate origPlace origin provenance rubric secFol signatures source stamp summary support surrogates watermark\par 
    \item[namesdates: ]
   addName affiliation birth bloc country death district education faith floruit forename genName geogFeat geogName nameLink nationality occupation offset orgName persName placeName region residence roleName settlement sex socecStatus surname\par 
    \item[textcrit: ]
   lem rdg wit witDetail\par 
    \item[textstructure: ]
   byline closer dateline docAuthor docDate docEdition docImprint imprimatur opener salute signed titlePart trailer\par 
    \item[transcr: ]
   am damage fw line metamark mod restore retrace secl supplied surplus zone
    \item[{May contain}]
  
    \item[analysis: ]
   c cl interp interpGrp m pc phr s span spanGrp w\par 
    \item[core: ]
   abbr add address bibl biblStruct cb choice cit corr date del desc distinct email emph expan foreign gap gb gloss graphic hi index l label lb lg list listBibl measure measureGrp media mentioned milestone name note num orig pb ptr q quote ref reg rs said sic soCalled stage term time title unclear\par 
    \item[figures: ]
   figure formula notatedMusic table\par 
    \item[gaiji: ]
   g\par 
    \item[header: ]
   biblFull idno\par 
    \item[linking: ]
   alt altGrp anchor join joinGrp link linkGrp seg timeline\par 
    \item[msdescription: ]
   catchwords depth dim dimensions height heraldry locus locusGrp material msDesc objectType origDate origPlace secFol signatures stamp watermark width\par 
    \item[namesdates: ]
   addName affiliation bloc climate country district forename genName geo geogFeat geogName listEvent listNym listOrg listPerson listPlace location nameLink offset orgName persName placeName population region roleName settlement state surname terrain trait\par 
    \item[textcrit: ]
   app listApp listWit witDetail\par 
    \item[textstructure: ]
   floatingText\par 
    \item[transcr: ]
   addSpan am damage damageSpan delSpan ex fw handShift listTranspose metamark mod redo restore retrace secl space subst substJoin supplied surplus undo\par character data
    \item[{Note}]
  \par
Since damage to text witnesses frequently makes them harder to read, the <damage> element will often contain an <unclear> element. If the damaged area is not continuous (e.g. a stain affecting several strings of text), the {\itshape group} attribute may be used to group together several related <damage> elements; alternatively the <join> element may be used to indicate which <damage> and <unclear> elements are part of the same physical phenomenon.\par
The <damage>, <gap>, <del>, <unclear> and <supplied> elements may be closely allied in use. See section \xref{http://www.tei-c.org/release/doc/tei-p5-doc/en/html/PH.html\#PHCOMB}{11.3.3.2. Use of the gap, del, damage, unclear, and supplied Elements in Combination} for discussion of which element is appropriate for which circumstance.
    \item[{Example}]
  \leavevmode\bgroup\exampleFont \begin{shaded}\noindent\mbox{}{<\textbf{l}>}The Moving Finger wri{<\textbf{damage}\hspace*{6pt}{agent}="{water}"\hspace*{6pt}{group}="{1}">}es; and{</\textbf{damage}>} having writ,{</\textbf{l}>}\mbox{}\newline 
{<\textbf{l}>}Moves {<\textbf{damage}\hspace*{6pt}{agent}="{water}"\hspace*{6pt}{group}="{1}">}\mbox{}\newline 
\hspace*{6pt}\hspace*{6pt}{<\textbf{supplied}>}on: nor all your{</\textbf{supplied}>}\mbox{}\newline 
\hspace*{6pt}{</\textbf{damage}>} Piety nor Wit{</\textbf{l}>}\end{shaded}\egroup 


    \item[{Content model}]
  \mbox{}\hfill\\[-10pt]\begin{Verbatim}[fontsize=\small]
<content>
 <macroRef key="macro.paraContent"/>
</content>
    
\end{Verbatim}

    \item[{Schema Declaration}]
  \mbox{}\hfill\\[-10pt]\begin{Verbatim}[fontsize=\small]
element damage
{
   att.global.attributes,
   att.typed.attributes,
   att.damaged.attributes,
   macro.paraContent}
\end{Verbatim}

\end{reflist}  \index{damageSpan=<damageSpan>|oddindex}
\begin{reflist}
\item[]\begin{specHead}{TEI.damageSpan}{<damageSpan> }(damaged span of text) marks the beginning of a longer sequence of text which is damaged in some way but still legible. [\xref{http://www.tei-c.org/release/doc/tei-p5-doc/en/html/PH.html\#PHDA}{11.3.3.1. Damage, Illegibility, and Supplied Text}]\end{specHead} 
    \item[{Module}]
  transcr
    \item[{Attributes}]
  Attributes att.global (\textit{@xml:id}, \textit{@n}, \textit{@xml:lang}, \textit{@xml:base}, \textit{@xml:space})  (att.global.rendition (\textit{@rend}, \textit{@style}, \textit{@rendition})) (att.global.linking (\textit{@corresp}, \textit{@synch}, \textit{@sameAs}, \textit{@copyOf}, \textit{@next}, \textit{@prev}, \textit{@exclude}, \textit{@select})) (att.global.analytic (\textit{@ana})) (att.global.facs (\textit{@facs})) (att.global.change (\textit{@change})) (att.global.responsibility (\textit{@cert}, \textit{@resp})) (att.global.source (\textit{@source})) att.damaged (\textit{@agent}, \textit{@degree}, \textit{@group})  (att.dimensions (\textit{@unit}, \textit{@quantity}, \textit{@extent}, \textit{@precision}, \textit{@scope}) (att.ranging (\textit{@atLeast}, \textit{@atMost}, \textit{@min}, \textit{@max}, \textit{@confidence})) ) (att.written (\textit{@hand})) att.typed (\textit{@type}, \textit{@subtype}) att.spanning (\textit{@spanTo}) 
    \item[{Member of}]
  model.global.edit
    \item[{Contained by}]
  
    \item[analysis: ]
   cl m phr s span w\par 
    \item[core: ]
   abbr add addrLine address author bibl biblScope cit citedRange corr date del distinct editor email emph expan foreign gloss head headItem headLabel hi imprint item l label lg list measure mentioned name note num orig p pubPlace publisher q quote ref reg resp rs said series sic soCalled sp speaker stage street term textLang time title unclear\par 
    \item[figures: ]
   cell figure table\par 
    \item[header: ]
   authority change classCode distributor edition extent funder geoDecl handNote language licence principal scriptNote sponsor typeNote\par 
    \item[linking: ]
   ab seg\par 
    \item[msdescription: ]
   accMat acquisition additions catchwords collation colophon condition custEvent decoNote explicit filiation finalRubric foliation heraldry incipit layout material msItem musicNotation objectType origDate origPlace origin provenance rubric secFol signatures source stamp summary support surrogates watermark\par 
    \item[namesdates: ]
   addName affiliation age birth bloc country death district education faith floruit forename genName geogFeat geogName langKnown nameLink nationality occupation offset orgName persName person personGrp placeName region residence roleName settlement sex socecStatus surname\par 
    \item[textcrit: ]
   lem rdg wit witDetail\par 
    \item[textstructure: ]
   argument back body byline closer dateline div docAuthor docDate docEdition docImprint docTitle epigraph floatingText front group imprimatur opener postscript salute signed text titlePage titlePart trailer\par 
    \item[transcr: ]
   damage fw line metamark mod restore retrace secl sourceDoc supplied surface surfaceGrp surplus zone
    \item[{May contain}]
  Empty element
    \item[{Note}]
  \par
Both the beginning and ending of the damaged sequence must be marked: the beginning by the <damageSpan> element, the ending by the target of the {\itshape spanTo} attribute: if no other element available, the <anchor> element may be used for this purpose.\par
The damaged text must be at least partially legible, in order for the encoder to be able to transcribe it. If it is not legible at all, the <damageSpan> element should not be used. Rather, the <gap> or <unclear> element should be employed, with the value of the {\itshape reason} attribute giving the cause. See further sections \xref{http://www.tei-c.org/release/doc/tei-p5-doc/en/html/PH.html\#PHDA}{11.3.3.1. Damage, Illegibility, and Supplied Text} and \xref{http://www.tei-c.org/release/doc/tei-p5-doc/en/html/PH.html\#PHCOMB}{11.3.3.2. Use of the gap, del, damage, unclear, and supplied Elements in Combination}.
    \item[{Example}]
  \leavevmode\bgroup\exampleFont \begin{shaded}\noindent\mbox{}{<\textbf{p}>}Paragraph partially damaged. This is the undamaged\mbox{}\newline 
 portion {<\textbf{damageSpan}\hspace*{6pt}{spanTo}="{\#a34}"/>}and this the damaged\mbox{}\newline 
 portion of the paragraph.{</\textbf{p}>}\mbox{}\newline 
{<\textbf{p}>}This paragraph is entirely damaged.{</\textbf{p}>}\mbox{}\newline 
{<\textbf{p}>}Paragraph partially damaged; in the middle of this\mbox{}\newline 
 paragraph the damage ends and the anchor point marks\mbox{}\newline 
 the start of the {<\textbf{anchor}\hspace*{6pt}{xml:id}="{a34}"/>} undamaged part of the text. ...{</\textbf{p}>}\end{shaded}\egroup 


    \item[{Schematron}]
   <s:assert test="@spanTo">The @spanTo attribute of <s:name/> is required.</s:assert>
    \item[{Schematron}]
   <s:assert test="@spanTo">L'attribut spanTo est requis.</s:assert>
    \item[{Content model}]
  \fbox{\ttfamily <content>\newline
</content>\newline
    } 
    \item[{Schema Declaration}]
  \mbox{}\hfill\\[-10pt]\begin{Verbatim}[fontsize=\small]
element damageSpan
{
   att.global.attributes,
   att.damaged.attributes,
   att.typed.attributes,
   att.spanning.attributes,
   empty
}
\end{Verbatim}

\end{reflist}  \index{date=<date>|oddindex}
\begin{reflist}
\item[]\begin{specHead}{TEI.date}{<date> }contains a date in any format. [\xref{http://www.tei-c.org/release/doc/tei-p5-doc/en/html/CO.html\#CONADA}{3.5.4. Dates and Times} \xref{http://www.tei-c.org/release/doc/tei-p5-doc/en/html/HD.html\#HD24}{2.2.4. Publication, Distribution, Licensing, etc.} \xref{http://www.tei-c.org/release/doc/tei-p5-doc/en/html/HD.html\#HD6}{2.6. The Revision Description} \xref{http://www.tei-c.org/release/doc/tei-p5-doc/en/html/CO.html\#COBICOI}{3.11.2.4. Imprint, Size of a Document, and Reprint Information} \xref{http://www.tei-c.org/release/doc/tei-p5-doc/en/html/CC.html\#CCAHSE}{15.2.3. The Setting Description} \xref{http://www.tei-c.org/release/doc/tei-p5-doc/en/html/ND.html\#NDDATE}{13.3.6. Dates and Times}]\end{specHead} 
    \item[{Module}]
  core
    \item[{Attributes}]
  Attributes att.global (\textit{@xml:id}, \textit{@n}, \textit{@xml:lang}, \textit{@xml:base}, \textit{@xml:space})  (att.global.rendition (\textit{@rend}, \textit{@style}, \textit{@rendition})) (att.global.linking (\textit{@corresp}, \textit{@synch}, \textit{@sameAs}, \textit{@copyOf}, \textit{@next}, \textit{@prev}, \textit{@exclude}, \textit{@select})) (att.global.analytic (\textit{@ana})) (att.global.facs (\textit{@facs})) (att.global.change (\textit{@change})) (att.global.responsibility (\textit{@cert}, \textit{@resp})) (att.global.source (\textit{@source})) att.datable (\textit{@calendar}, \textit{@period})  (att.datable.w3c (\textit{@when}, \textit{@notBefore}, \textit{@notAfter}, \textit{@from}, \textit{@to})) (att.datable.iso (\textit{@when-iso}, \textit{@notBefore-iso}, \textit{@notAfter-iso}, \textit{@from-iso}, \textit{@to-iso})) (att.datable.custom (\textit{@when-custom}, \textit{@notBefore-custom}, \textit{@notAfter-custom}, \textit{@from-custom}, \textit{@to-custom}, \textit{@datingPoint}, \textit{@datingMethod})) att.editLike (\textit{@evidence}, \textit{@instant})  (att.dimensions (\textit{@unit}, \textit{@quantity}, \textit{@extent}, \textit{@precision}, \textit{@scope}) (att.ranging (\textit{@atLeast}, \textit{@atMost}, \textit{@min}, \textit{@max}, \textit{@confidence})) ) att.typed (\textit{@type}, \textit{@subtype}) 
    \item[{Member of}]
  model.dateLike model.publicationStmtPart.detail
    \item[{Contained by}]
  
    \item[analysis: ]
   cl phr s span\par 
    \item[core: ]
   abbr add addrLine analytic author bibl biblScope citedRange corr date del desc distinct editor email emph expan foreign gloss head headItem headLabel hi imprint item l label measure meeting mentioned name note num orig p pubPlace publisher q quote ref reg resp rs said sic soCalled speaker stage street term textLang time title unclear\par 
    \item[figures: ]
   cell figDesc\par 
    \item[header: ]
   authority catDesc change classCode correspAction creation distributor edition extent funder geoDecl handNote language licence principal publicationStmt rendition scriptNote sponsor tagUsage typeNote\par 
    \item[linking: ]
   ab seg\par 
    \item[msdescription: ]
   accMat acquisition additions catchwords collation colophon condition custEvent decoNote explicit filiation finalRubric foliation heraldry incipit layout material musicNotation objectType origDate origPlace origin provenance rubric secFol signatures source stamp summary support surrogates watermark\par 
    \item[namesdates: ]
   addName affiliation age birth bloc country death district education faith floruit forename genName geogFeat geogName langKnown nameLink nationality occupation offset orgName persName placeName region residence roleName settlement sex socecStatus surname\par 
    \item[textcrit: ]
   lem rdg wit witDetail witness\par 
    \item[textstructure: ]
   byline closer dateline docAuthor docDate docEdition docImprint imprimatur opener salute signed titlePart trailer\par 
    \item[transcr: ]
   damage fw metamark mod restore retrace secl supplied surplus
    \item[{May contain}]
  
    \item[analysis: ]
   c cl interp interpGrp m pc phr s span spanGrp w\par 
    \item[core: ]
   abbr add address cb choice corr date del distinct email emph expan foreign gap gb gloss graphic hi index lb measure measureGrp media mentioned milestone name note num orig pb ptr ref reg rs sic soCalled term time title unclear\par 
    \item[figures: ]
   figure formula notatedMusic\par 
    \item[gaiji: ]
   g\par 
    \item[header: ]
   idno\par 
    \item[linking: ]
   alt altGrp anchor join joinGrp link linkGrp seg timeline\par 
    \item[msdescription: ]
   catchwords depth dim dimensions height heraldry locus locusGrp material objectType origDate origPlace secFol signatures stamp watermark width\par 
    \item[namesdates: ]
   addName affiliation bloc climate country district forename genName geo geogFeat geogName location nameLink offset orgName persName placeName population region roleName settlement state surname terrain trait\par 
    \item[textcrit: ]
   app witDetail\par 
    \item[transcr: ]
   addSpan am damage damageSpan delSpan ex fw handShift listTranspose metamark mod redo restore retrace secl space subst substJoin supplied surplus undo\par character data
    \item[{Example}]
  \leavevmode\bgroup\exampleFont \begin{shaded}\noindent\mbox{}{<\textbf{date}\hspace*{6pt}{when}="{1980-02}">}early February 1980{</\textbf{date}>}\end{shaded}\egroup 


    \item[{Example}]
  \leavevmode\bgroup\exampleFont \begin{shaded}\noindent\mbox{}Given on the {<\textbf{date}\hspace*{6pt}{when}="{1977-06-12}">}Twelfth Day\mbox{}\newline 
 of June in the Year of Our Lord One Thousand Nine Hundred and Seventy-seven of the Republic\mbox{}\newline 
 the Two Hundredth and first and of the University the Eighty-Sixth.{</\textbf{date}>}\end{shaded}\egroup 


    \item[{Example}]
  \leavevmode\bgroup\exampleFont \begin{shaded}\noindent\mbox{}{<\textbf{date}\hspace*{6pt}{when}="{1990-09}">}September 1990{</\textbf{date}>}\end{shaded}\egroup 


    \item[{Content model}]
  \mbox{}\hfill\\[-10pt]\begin{Verbatim}[fontsize=\small]
<content>
 <alternate maxOccurs="unbounded"
  minOccurs="0">
  <textNode/>
  <classRef key="model.gLike"/>
  <classRef key="model.phrase"/>
  <classRef key="model.global"/>
 </alternate>
</content>
    
\end{Verbatim}

    \item[{Schema Declaration}]
  \mbox{}\hfill\\[-10pt]\begin{Verbatim}[fontsize=\small]
element date
{
   att.global.attributes,
   att.datable.attributes,
   att.editLike.attributes,
   att.typed.attributes,
   ( text | model.gLike | model.phrase | model.global )*
}
\end{Verbatim}

\end{reflist}  \index{dateline=<dateline>|oddindex}
\begin{reflist}
\item[]\begin{specHead}{TEI.dateline}{<dateline> }contains a brief description of the place, date, time, etc. of production of a letter, newspaper story, or other work, prefixed or suffixed to it as a kind of heading or trailer. [\xref{http://www.tei-c.org/release/doc/tei-p5-doc/en/html/DS.html\#DSOC}{4.2.2. Openers and Closers}]\end{specHead} 
    \item[{Module}]
  textstructure
    \item[{Attributes}]
  Attributes att.global (\textit{@xml:id}, \textit{@n}, \textit{@xml:lang}, \textit{@xml:base}, \textit{@xml:space})  (att.global.rendition (\textit{@rend}, \textit{@style}, \textit{@rendition})) (att.global.linking (\textit{@corresp}, \textit{@synch}, \textit{@sameAs}, \textit{@copyOf}, \textit{@next}, \textit{@prev}, \textit{@exclude}, \textit{@select})) (att.global.analytic (\textit{@ana})) (att.global.facs (\textit{@facs})) (att.global.change (\textit{@change})) (att.global.responsibility (\textit{@cert}, \textit{@resp})) (att.global.source (\textit{@source}))
    \item[{Member of}]
  model.divWrapper 
    \item[{Contained by}]
  
    \item[core: ]
   lg list\par 
    \item[figures: ]
   figure table\par 
    \item[textstructure: ]
   body closer div front group opener
    \item[{May contain}]
  
    \item[analysis: ]
   c cl interp interpGrp m pc phr s span spanGrp w\par 
    \item[core: ]
   abbr add address cb choice corr date del distinct email emph expan foreign gap gb gloss graphic hi index lb measure measureGrp media mentioned milestone name note num orig pb ptr ref reg rs sic soCalled term time title unclear\par 
    \item[figures: ]
   figure formula notatedMusic\par 
    \item[gaiji: ]
   g\par 
    \item[header: ]
   idno\par 
    \item[linking: ]
   alt altGrp anchor join joinGrp link linkGrp seg timeline\par 
    \item[msdescription: ]
   catchwords depth dim dimensions height heraldry locus locusGrp material objectType origDate origPlace secFol signatures stamp watermark width\par 
    \item[namesdates: ]
   addName affiliation bloc climate country district forename genName geo geogFeat geogName location nameLink offset orgName persName placeName population region roleName settlement state surname terrain trait\par 
    \item[textcrit: ]
   app witDetail\par 
    \item[textstructure: ]
   docDate\par 
    \item[transcr: ]
   addSpan am damage damageSpan delSpan ex fw handShift listTranspose metamark mod redo restore retrace secl space subst substJoin supplied surplus undo\par character data
    \item[{Example}]
  \leavevmode\bgroup\exampleFont \begin{shaded}\noindent\mbox{}{<\textbf{dateline}>}Walden, this 29. of August 1592{</\textbf{dateline}>}\end{shaded}\egroup 


    \item[{Example}]
  \leavevmode\bgroup\exampleFont \begin{shaded}\noindent\mbox{}{<\textbf{div}\hspace*{6pt}{type}="{chapter}">}\mbox{}\newline 
\hspace*{6pt}{<\textbf{p}>}\mbox{}\newline 
\textit{<!-- ... -->} and his heart was going like mad and yes I said yes I will Yes.{</\textbf{p}>}\mbox{}\newline 
\hspace*{6pt}{<\textbf{closer}>}\mbox{}\newline 
\hspace*{6pt}\hspace*{6pt}{<\textbf{dateline}>}\mbox{}\newline 
\hspace*{6pt}\hspace*{6pt}\hspace*{6pt}{<\textbf{name}\hspace*{6pt}{type}="{place}">}Trieste-Zürich-Paris,{</\textbf{name}>}\mbox{}\newline 
\hspace*{6pt}\hspace*{6pt}\hspace*{6pt}{<\textbf{date}>}1914–1921{</\textbf{date}>}\mbox{}\newline 
\hspace*{6pt}\hspace*{6pt}{</\textbf{dateline}>}\mbox{}\newline 
\hspace*{6pt}{</\textbf{closer}>}\mbox{}\newline 
{</\textbf{div}>}\end{shaded}\egroup 


    \item[{Content model}]
  \mbox{}\hfill\\[-10pt]\begin{Verbatim}[fontsize=\small]
<content>
 <alternate maxOccurs="unbounded"
  minOccurs="0">
  <textNode/>
  <classRef key="model.gLike"/>
  <classRef key="model.phrase"/>
  <classRef key="model.global"/>
  <elementRef key="docDate"/>
 </alternate>
</content>
    
\end{Verbatim}

    \item[{Schema Declaration}]
  \mbox{}\hfill\\[-10pt]\begin{Verbatim}[fontsize=\small]
element dateline
{
   att.global.attributes,
   ( text | model.gLike | model.phrase | model.global | docDate )*
}
\end{Verbatim}

\end{reflist}  \index{death=<death>|oddindex}
\begin{reflist}
\item[]\begin{specHead}{TEI.death}{<death> }contains information about a person's death, such as its date and place. [\xref{http://www.tei-c.org/release/doc/tei-p5-doc/en/html/CC.html\#CCAHPA}{15.2.2. The Participant Description}]\end{specHead} 
    \item[{Module}]
  namesdates
    \item[{Attributes}]
  Attributes att.global (\textit{@xml:id}, \textit{@n}, \textit{@xml:lang}, \textit{@xml:base}, \textit{@xml:space})  (att.global.rendition (\textit{@rend}, \textit{@style}, \textit{@rendition})) (att.global.linking (\textit{@corresp}, \textit{@synch}, \textit{@sameAs}, \textit{@copyOf}, \textit{@next}, \textit{@prev}, \textit{@exclude}, \textit{@select})) (att.global.analytic (\textit{@ana})) (att.global.facs (\textit{@facs})) (att.global.change (\textit{@change})) (att.global.responsibility (\textit{@cert}, \textit{@resp})) (att.global.source (\textit{@source})) att.editLike (\textit{@evidence}, \textit{@instant})  (att.dimensions (\textit{@unit}, \textit{@quantity}, \textit{@extent}, \textit{@precision}, \textit{@scope}) (att.ranging (\textit{@atLeast}, \textit{@atMost}, \textit{@min}, \textit{@max}, \textit{@confidence})) ) att.datable (\textit{@calendar}, \textit{@period})  (att.datable.w3c (\textit{@when}, \textit{@notBefore}, \textit{@notAfter}, \textit{@from}, \textit{@to})) (att.datable.iso (\textit{@when-iso}, \textit{@notBefore-iso}, \textit{@notAfter-iso}, \textit{@from-iso}, \textit{@to-iso})) (att.datable.custom (\textit{@when-custom}, \textit{@notBefore-custom}, \textit{@notAfter-custom}, \textit{@from-custom}, \textit{@to-custom}, \textit{@datingPoint}, \textit{@datingMethod})) att.naming (\textit{@role}, \textit{@nymRef})  (att.canonical (\textit{@key}, \textit{@ref}))
    \item[{Member of}]
  model.personPart
    \item[{Contained by}]
  
    \item[namesdates: ]
   person personGrp
    \item[{May contain}]
  
    \item[analysis: ]
   c cl interp interpGrp m pc phr s span spanGrp w\par 
    \item[core: ]
   abbr add address cb choice corr date del distinct email emph expan foreign gap gb gloss graphic hi index lb measure measureGrp media mentioned milestone name note num orig pb ptr ref reg rs sic soCalled term time title unclear\par 
    \item[figures: ]
   figure formula notatedMusic\par 
    \item[gaiji: ]
   g\par 
    \item[header: ]
   idno\par 
    \item[linking: ]
   alt altGrp anchor join joinGrp link linkGrp seg timeline\par 
    \item[msdescription: ]
   catchwords depth dim dimensions height heraldry locus locusGrp material objectType origDate origPlace secFol signatures stamp watermark width\par 
    \item[namesdates: ]
   addName affiliation bloc climate country district forename genName geo geogFeat geogName location nameLink offset orgName persName placeName population region roleName settlement state surname terrain trait\par 
    \item[textcrit: ]
   app witDetail\par 
    \item[transcr: ]
   addSpan am damage damageSpan delSpan ex fw handShift listTranspose metamark mod redo restore retrace secl space subst substJoin supplied surplus undo\par character data
    \item[{Example}]
  \leavevmode\bgroup\exampleFont \begin{shaded}\noindent\mbox{}{<\textbf{death}\hspace*{6pt}{when}="{1902-10-01}"/>}\end{shaded}\egroup 


    \item[{Example}]
  \leavevmode\bgroup\exampleFont \begin{shaded}\noindent\mbox{}{<\textbf{death}\hspace*{6pt}{when}="{1960-12-10}">}Passed away near {<\textbf{name}\hspace*{6pt}{type}="{place}">}Aix-la-Chapelle{</\textbf{name}>}, after suffering from cerebral palsy. {</\textbf{death}>}\end{shaded}\egroup 


    \item[{Content model}]
  \mbox{}\hfill\\[-10pt]\begin{Verbatim}[fontsize=\small]
<content>
 <macroRef key="macro.phraseSeq"/>
</content>
    
\end{Verbatim}

    \item[{Schema Declaration}]
  \mbox{}\hfill\\[-10pt]\begin{Verbatim}[fontsize=\small]
element death
{
   att.global.attributes,
   att.editLike.attributes,
   att.datable.attributes,
   att.naming.attributes,
   macro.phraseSeq}
\end{Verbatim}

\end{reflist}  \index{decoDesc=<decoDesc>|oddindex}
\begin{reflist}
\item[]\begin{specHead}{TEI.decoDesc}{<decoDesc> }(decoration description) contains a description of the decoration of a manuscript, either as a sequence of paragraphs, or as a sequence of topically organized <decoNote> elements. [\xref{http://www.tei-c.org/release/doc/tei-p5-doc/en/html/MS.html\#msph3}{10.7.3. Bindings, Seals, and Additional Material}]\end{specHead} 
    \item[{Module}]
  msdescription
    \item[{Attributes}]
  Attributes att.global (\textit{@xml:id}, \textit{@n}, \textit{@xml:lang}, \textit{@xml:base}, \textit{@xml:space})  (att.global.rendition (\textit{@rend}, \textit{@style}, \textit{@rendition})) (att.global.linking (\textit{@corresp}, \textit{@synch}, \textit{@sameAs}, \textit{@copyOf}, \textit{@next}, \textit{@prev}, \textit{@exclude}, \textit{@select})) (att.global.analytic (\textit{@ana})) (att.global.facs (\textit{@facs})) (att.global.change (\textit{@change})) (att.global.responsibility (\textit{@cert}, \textit{@resp})) (att.global.source (\textit{@source}))
    \item[{Member of}]
  model.physDescPart
    \item[{Contained by}]
  
    \item[msdescription: ]
   physDesc
    \item[{May contain}]
  
    \item[core: ]
   p\par 
    \item[linking: ]
   ab\par 
    \item[msdescription: ]
   decoNote summary
    \item[{Example}]
  \leavevmode\bgroup\exampleFont \begin{shaded}\noindent\mbox{}{<\textbf{decoDesc}>}\mbox{}\newline 
\hspace*{6pt}{<\textbf{p}>}The start of each book of the Bible with a 10-line historiated\mbox{}\newline 
\hspace*{6pt}\hspace*{6pt} illuminated initial; prefaces decorated with 6-line blue initials with red\mbox{}\newline 
\hspace*{6pt}\hspace*{6pt} penwork flourishing; chapters marked by 3-line plain red initials; verses\mbox{}\newline 
\hspace*{6pt}\hspace*{6pt} with 1-line initials, alternately blue or red.{</\textbf{p}>}\mbox{}\newline 
{</\textbf{decoDesc}>}\end{shaded}\egroup 


    \item[{Content model}]
  \mbox{}\hfill\\[-10pt]\begin{Verbatim}[fontsize=\small]
<content>
 <alternate>
  <classRef key="model.pLike"
   maxOccurs="unbounded" minOccurs="1"/>
  <sequence>
   <elementRef key="summary" minOccurs="0"/>
   <elementRef key="decoNote"
    maxOccurs="unbounded" minOccurs="1"/>
  </sequence>
 </alternate>
</content>
    
\end{Verbatim}

    \item[{Schema Declaration}]
  \mbox{}\hfill\\[-10pt]\begin{Verbatim}[fontsize=\small]
element decoDesc
{
   att.global.attributes,
   ( model.pLike+ | ( summary?, decoNote+ ) )
}
\end{Verbatim}

\end{reflist}  \index{decoNote=<decoNote>|oddindex}
\begin{reflist}
\item[]\begin{specHead}{TEI.decoNote}{<decoNote> }(note on decoration) contains a note describing either a decorative component of a manuscript, or a fairly homogenous class of such components. [\xref{http://www.tei-c.org/release/doc/tei-p5-doc/en/html/MS.html\#msph3}{10.7.3. Bindings, Seals, and Additional Material}]\end{specHead} 
    \item[{Module}]
  msdescription
    \item[{Attributes}]
  Attributes att.global (\textit{@xml:id}, \textit{@n}, \textit{@xml:lang}, \textit{@xml:base}, \textit{@xml:space})  (att.global.rendition (\textit{@rend}, \textit{@style}, \textit{@rendition})) (att.global.linking (\textit{@corresp}, \textit{@synch}, \textit{@sameAs}, \textit{@copyOf}, \textit{@next}, \textit{@prev}, \textit{@exclude}, \textit{@select})) (att.global.analytic (\textit{@ana})) (att.global.facs (\textit{@facs})) (att.global.change (\textit{@change})) (att.global.responsibility (\textit{@cert}, \textit{@resp})) (att.global.source (\textit{@source})) att.typed (\textit{@type}, \textit{@subtype}) 
    \item[{Member of}]
  model.msItemPart 
    \item[{Contained by}]
  
    \item[msdescription: ]
   binding bindingDesc decoDesc msItem msItemStruct seal sealDesc
    \item[{May contain}]
  
    \item[analysis: ]
   c cl interp interpGrp m pc phr s span spanGrp w\par 
    \item[core: ]
   abbr add address bibl biblStruct cb choice cit corr date del desc distinct email emph expan foreign gap gb gloss graphic hi index l label lb lg list listBibl measure measureGrp media mentioned milestone name note num orig p pb ptr q quote ref reg rs said sic soCalled sp stage term time title unclear\par 
    \item[figures: ]
   figure formula notatedMusic table\par 
    \item[gaiji: ]
   g\par 
    \item[header: ]
   biblFull idno\par 
    \item[linking: ]
   ab alt altGrp anchor join joinGrp link linkGrp seg timeline\par 
    \item[msdescription: ]
   catchwords depth dim dimensions height heraldry locus locusGrp material msDesc objectType origDate origPlace secFol signatures stamp watermark width\par 
    \item[namesdates: ]
   addName affiliation bloc climate country district forename genName geo geogFeat geogName listEvent listNym listOrg listPerson listPlace location nameLink offset orgName persName placeName population region roleName settlement state surname terrain trait\par 
    \item[textcrit: ]
   app listApp listWit witDetail\par 
    \item[textstructure: ]
   floatingText\par 
    \item[transcr: ]
   addSpan am damage damageSpan delSpan ex fw handShift listTranspose metamark mod redo restore retrace secl space subst substJoin supplied surplus undo\par character data
    \item[{Example}]
  \leavevmode\bgroup\exampleFont \begin{shaded}\noindent\mbox{}{<\textbf{decoDesc}>}\mbox{}\newline 
\hspace*{6pt}{<\textbf{decoNote}\hspace*{6pt}{type}="{initial}">}\mbox{}\newline 
\hspace*{6pt}\hspace*{6pt}{<\textbf{p}>}The start of each book of the Bible with\mbox{}\newline 
\hspace*{6pt}\hspace*{6pt}\hspace*{6pt}\hspace*{6pt} a 10-line historiated illuminated initial;\mbox{}\newline 
\hspace*{6pt}\hspace*{6pt}\hspace*{6pt}\hspace*{6pt} prefaces decorated with 6-line blue initials\mbox{}\newline 
\hspace*{6pt}\hspace*{6pt}\hspace*{6pt}\hspace*{6pt} with red penwork flourishing; chapters marked by\mbox{}\newline 
\hspace*{6pt}\hspace*{6pt}\hspace*{6pt}\hspace*{6pt} 3-line plain red initials; verses with 1-line initials,\mbox{}\newline 
\hspace*{6pt}\hspace*{6pt}\hspace*{6pt}\hspace*{6pt} alternately blue or red.{</\textbf{p}>}\mbox{}\newline 
\hspace*{6pt}{</\textbf{decoNote}>}\mbox{}\newline 
{</\textbf{decoDesc}>}\end{shaded}\egroup 


    \item[{Content model}]
  \mbox{}\hfill\\[-10pt]\begin{Verbatim}[fontsize=\small]
<content>
 <macroRef key="macro.specialPara"/>
</content>
    
\end{Verbatim}

    \item[{Schema Declaration}]
  \mbox{}\hfill\\[-10pt]\begin{Verbatim}[fontsize=\small]
element decoNote
{
   att.global.attributes,
   att.typed.attributes,
   macro.specialPara}
\end{Verbatim}

\end{reflist}  \index{del=<del>|oddindex}
\begin{reflist}
\item[]\begin{specHead}{TEI.del}{<del> }(deletion) contains a letter, word, or passage deleted, marked as deleted, or otherwise indicated as superfluous or spurious in the copy text by an author, scribe, or a previous annotator or corrector. [\xref{http://www.tei-c.org/release/doc/tei-p5-doc/en/html/CO.html\#COEDADD}{3.4.3. Additions, Deletions, and Omissions}]\end{specHead} 
    \item[{Module}]
  core
    \item[{Attributes}]
  Attributes att.global (\textit{@xml:id}, \textit{@n}, \textit{@xml:lang}, \textit{@xml:base}, \textit{@xml:space})  (att.global.rendition (\textit{@rend}, \textit{@style}, \textit{@rendition})) (att.global.linking (\textit{@corresp}, \textit{@synch}, \textit{@sameAs}, \textit{@copyOf}, \textit{@next}, \textit{@prev}, \textit{@exclude}, \textit{@select})) (att.global.analytic (\textit{@ana})) (att.global.facs (\textit{@facs})) (att.global.change (\textit{@change})) (att.global.responsibility (\textit{@cert}, \textit{@resp})) (att.global.source (\textit{@source})) att.transcriptional (\textit{@status}, \textit{@cause}, \textit{@seq})  (att.editLike (\textit{@evidence}, \textit{@instant}) (att.dimensions (\textit{@unit}, \textit{@quantity}, \textit{@extent}, \textit{@precision}, \textit{@scope}) (att.ranging (\textit{@atLeast}, \textit{@atMost}, \textit{@min}, \textit{@max}, \textit{@confidence})) ) ) (att.written (\textit{@hand})) att.typed (\textit{@type}, \textit{@subtype}) 
    \item[{Member of}]
  model.linePart model.pPart.transcriptional 
    \item[{Contained by}]
  
    \item[analysis: ]
   cl pc phr s w\par 
    \item[core: ]
   abbr add addrLine author bibl biblScope citedRange corr date del distinct editor email emph expan foreign gloss head headItem headLabel hi item l label measure mentioned name note num orig p pubPlace publisher q quote ref reg rs said sic soCalled speaker stage street term textLang time title unclear\par 
    \item[figures: ]
   cell\par 
    \item[header: ]
   change distributor edition extent geoDecl handNote licence scriptNote typeNote\par 
    \item[linking: ]
   ab seg\par 
    \item[msdescription: ]
   accMat acquisition additions catchwords collation colophon condition custEvent decoNote explicit filiation finalRubric foliation heraldry incipit layout material musicNotation objectType origDate origPlace origin provenance rubric secFol signatures source stamp summary support surrogates watermark\par 
    \item[namesdates: ]
   addName affiliation birth bloc country death district education faith floruit forename genName geogFeat geogName nameLink nationality occupation offset orgName persName placeName region residence roleName settlement sex socecStatus surname\par 
    \item[textcrit: ]
   lem rdg wit witDetail\par 
    \item[textstructure: ]
   byline closer dateline docAuthor docDate docEdition docImprint imprimatur opener salute signed titlePart trailer\par 
    \item[transcr: ]
   am damage fw line metamark mod restore retrace secl subst supplied surplus zone
    \item[{May contain}]
  
    \item[analysis: ]
   c cl interp interpGrp m pc phr s span spanGrp w\par 
    \item[core: ]
   abbr add address bibl biblStruct cb choice cit corr date del desc distinct email emph expan foreign gap gb gloss graphic hi index l label lb lg list listBibl measure measureGrp media mentioned milestone name note num orig pb ptr q quote ref reg rs said sic soCalled stage term time title unclear\par 
    \item[figures: ]
   figure formula notatedMusic table\par 
    \item[gaiji: ]
   g\par 
    \item[header: ]
   biblFull idno\par 
    \item[linking: ]
   alt altGrp anchor join joinGrp link linkGrp seg timeline\par 
    \item[msdescription: ]
   catchwords depth dim dimensions height heraldry locus locusGrp material msDesc objectType origDate origPlace secFol signatures stamp watermark width\par 
    \item[namesdates: ]
   addName affiliation bloc climate country district forename genName geo geogFeat geogName listEvent listNym listOrg listPerson listPlace location nameLink offset orgName persName placeName population region roleName settlement state surname terrain trait\par 
    \item[textcrit: ]
   app listApp listWit witDetail\par 
    \item[textstructure: ]
   floatingText\par 
    \item[transcr: ]
   addSpan am damage damageSpan delSpan ex fw handShift listTranspose metamark mod redo restore retrace secl space subst substJoin supplied surplus undo\par character data
    \item[{Note}]
  \par
This element should be used for deletion of shorter sequences of text, typically single words or phrases. The <delSpan> element should be used for longer sequences of text, for those containing structural subdivisions, and for those containing overlapping additions and deletions.\par
The text deleted must be at least partially legible in order for the encoder to be able to transcribe it (unless it is restored in a <supplied> tag). Illegible or lost text within a deletion may be marked using the <gap> tag to signal that text is present but has not been transcribed, or is no longer visible. Attributes on the <gap> element may be used to indicate how much text is omitted, the reason for omitting it, etc. If text is not fully legible, the <unclear> element (available when using the additional tagset for transcription of primary sources) should be used to signal the areas of text which cannot be read with confidence in a similar way.\par
Degrees of uncertainty over what can still be read, or whether a deletion was intended may be indicated by use of the \texttt{<certainty>} element (see \xref{http://www.tei-c.org/release/doc/tei-p5-doc/en/html/CE.html\#CE}{21. Certainty, Precision, and Responsibility}).\par
There is a clear distinction in the TEI between <del> and <surplus> on the one hand and <gap> or <unclear> on the other. <del> indicates a deletion present in the source being transcribed, which states the author's or a later scribe's intent to cancel or remove text. <surplus> indicates material present in the source being transcribed which should have been so deleted, but which is not in fact. <gap> or <unclear>, by contrast, signal an editor's or encoder's decision to omit something or their inability to read the source text. See sections \xref{http://www.tei-c.org/release/doc/tei-p5-doc/en/html/PH.html\#PHOM}{11.3.1.7. Text Omitted from or Supplied in the Transcription} and \xref{http://www.tei-c.org/release/doc/tei-p5-doc/en/html/PH.html\#PHCOMB}{11.3.3.2. Use of the gap, del, damage, unclear, and supplied Elements in Combination} for the relationship between these and other related elements used in detailed transcription.
    \item[{Example}]
  \leavevmode\bgroup\exampleFont \begin{shaded}\noindent\mbox{}{<\textbf{l}>}\mbox{}\newline 
\hspace*{6pt}{<\textbf{del}\hspace*{6pt}{rend}="{overtyped}">}Mein{</\textbf{del}>} Frisch {<\textbf{del}\hspace*{6pt}{rend}="{overstrike}"\hspace*{6pt}{type}="{primary}">}schwebt{</\textbf{del}>}\mbox{}\newline 
 weht der Wind\mbox{}\newline 
{</\textbf{l}>}\end{shaded}\egroup 


    \item[{Example}]
  \leavevmode\bgroup\exampleFont \begin{shaded}\noindent\mbox{}{<\textbf{del}\hspace*{6pt}{rend}="{overstrike}">}\mbox{}\newline 
\hspace*{6pt}{<\textbf{gap}\hspace*{6pt}{quantity}="{5}"\hspace*{6pt}{reason}="{illegible}"\mbox{}\newline 
\hspace*{6pt}\hspace*{6pt}{unit}="{character}"/>}\mbox{}\newline 
{</\textbf{del}>}\end{shaded}\egroup 


    \item[{Content model}]
  \mbox{}\hfill\\[-10pt]\begin{Verbatim}[fontsize=\small]
<content>
 <macroRef key="macro.paraContent"/>
</content>
    
\end{Verbatim}

    \item[{Schema Declaration}]
  \mbox{}\hfill\\[-10pt]\begin{Verbatim}[fontsize=\small]
element del
{
   att.global.attributes,
   att.transcriptional.attributes,
   att.typed.attributes,
   macro.paraContent}
\end{Verbatim}

\end{reflist}  \index{delSpan=<delSpan>|oddindex}
\begin{reflist}
\item[]\begin{specHead}{TEI.delSpan}{<delSpan> }(deleted span of text) marks the beginning of a longer sequence of text deleted, marked as deleted, or otherwise signaled as superfluous or spurious by an author, scribe, annotator, or corrector. [\xref{http://www.tei-c.org/release/doc/tei-p5-doc/en/html/PH.html\#PHAD}{11.3.1.4. Additions and Deletions}]\end{specHead} 
    \item[{Module}]
  transcr
    \item[{Attributes}]
  Attributes att.global (\textit{@xml:id}, \textit{@n}, \textit{@xml:lang}, \textit{@xml:base}, \textit{@xml:space})  (att.global.rendition (\textit{@rend}, \textit{@style}, \textit{@rendition})) (att.global.linking (\textit{@corresp}, \textit{@synch}, \textit{@sameAs}, \textit{@copyOf}, \textit{@next}, \textit{@prev}, \textit{@exclude}, \textit{@select})) (att.global.analytic (\textit{@ana})) (att.global.facs (\textit{@facs})) (att.global.change (\textit{@change})) (att.global.responsibility (\textit{@cert}, \textit{@resp})) (att.global.source (\textit{@source})) att.transcriptional (\textit{@status}, \textit{@cause}, \textit{@seq})  (att.editLike (\textit{@evidence}, \textit{@instant}) (att.dimensions (\textit{@unit}, \textit{@quantity}, \textit{@extent}, \textit{@precision}, \textit{@scope}) (att.ranging (\textit{@atLeast}, \textit{@atMost}, \textit{@min}, \textit{@max}, \textit{@confidence})) ) ) (att.written (\textit{@hand})) att.typed (\textit{@type}, \textit{@subtype}) att.spanning (\textit{@spanTo}) 
    \item[{Member of}]
  model.global.edit
    \item[{Contained by}]
  
    \item[analysis: ]
   cl m phr s span w\par 
    \item[core: ]
   abbr add addrLine address author bibl biblScope cit citedRange corr date del distinct editor email emph expan foreign gloss head headItem headLabel hi imprint item l label lg list measure mentioned name note num orig p pubPlace publisher q quote ref reg resp rs said series sic soCalled sp speaker stage street term textLang time title unclear\par 
    \item[figures: ]
   cell figure table\par 
    \item[header: ]
   authority change classCode distributor edition extent funder geoDecl handNote language licence principal scriptNote sponsor typeNote\par 
    \item[linking: ]
   ab seg\par 
    \item[msdescription: ]
   accMat acquisition additions catchwords collation colophon condition custEvent decoNote explicit filiation finalRubric foliation heraldry incipit layout material msItem musicNotation objectType origDate origPlace origin provenance rubric secFol signatures source stamp summary support surrogates watermark\par 
    \item[namesdates: ]
   addName affiliation age birth bloc country death district education faith floruit forename genName geogFeat geogName langKnown nameLink nationality occupation offset orgName persName person personGrp placeName region residence roleName settlement sex socecStatus surname\par 
    \item[textcrit: ]
   lem rdg wit witDetail\par 
    \item[textstructure: ]
   argument back body byline closer dateline div docAuthor docDate docEdition docImprint docTitle epigraph floatingText front group imprimatur opener postscript salute signed text titlePage titlePart trailer\par 
    \item[transcr: ]
   damage fw line metamark mod restore retrace secl sourceDoc supplied surface surfaceGrp surplus zone
    \item[{May contain}]
  Empty element
    \item[{Note}]
  \par
Both the beginning and ending of the deleted sequence must be marked: the beginning by the <delSpan> element, the ending by the target of the {\itshape spanTo} attribute.\par
The text deleted must be at least partially legible, in order for the encoder to be able to transcribe it. If it is not legible at all, the <delSpan> tag should not be used. Rather, the <gap> tag should be employed to signal that text cannot be transcribed, with the value of the {\itshape reason} attribute giving the cause for the omission from the transcription as deletion. If it is not fully legible, the <unclear> element should be used to signal the areas of text which cannot be read with confidence. See further sections \xref{http://www.tei-c.org/release/doc/tei-p5-doc/en/html/PH.html\#PHOM}{11.3.1.7. Text Omitted from or Supplied in the Transcription} and, for the close association of the <delSpan> tag with the <gap>, <damage>, <unclear> and <supplied> elements, \xref{http://www.tei-c.org/release/doc/tei-p5-doc/en/html/PH.html\#PHCOMB}{11.3.3.2. Use of the gap, del, damage, unclear, and supplied Elements in Combination}.\par
The <delSpan> tag should not be used for deletions made by editors or encoders. In these cases, either the <corr> tag or the <gap> tag should be used.
    \item[{Example}]
  \leavevmode\bgroup\exampleFont \begin{shaded}\noindent\mbox{}{<\textbf{p}>}Paragraph partially deleted. This is the undeleted\mbox{}\newline 
 portion {<\textbf{delSpan}\hspace*{6pt}{spanTo}="{\#a23}"/>}and this the deleted\mbox{}\newline 
 portion of the paragraph.{</\textbf{p}>}\mbox{}\newline 
{<\textbf{p}>}Paragraph deleted together with adjacent material.{</\textbf{p}>}\mbox{}\newline 
{<\textbf{p}>}Second fully deleted paragraph.{</\textbf{p}>}\mbox{}\newline 
{<\textbf{p}>}Paragraph partially deleted; in the middle of this\mbox{}\newline 
 paragraph the deletion ends and the anchor point marks\mbox{}\newline 
 the resumption {<\textbf{anchor}\hspace*{6pt}{xml:id}="{a23}"/>} of the text. ...{</\textbf{p}>}\end{shaded}\egroup 


    \item[{Schematron}]
   <s:assert test="@spanTo">The @spanTo attribute of <s:name/> is required.</s:assert>
    \item[{Schematron}]
   <s:assert test="@spanTo">L'attribut spanTo est requis.</s:assert>
    \item[{Content model}]
  \fbox{\ttfamily <content>\newline
</content>\newline
    } 
    \item[{Schema Declaration}]
  \mbox{}\hfill\\[-10pt]\begin{Verbatim}[fontsize=\small]
element delSpan
{
   att.global.attributes,
   att.transcriptional.attributes,
   att.typed.attributes,
   att.spanning.attributes,
   empty
}
\end{Verbatim}

\end{reflist}  \index{depth=<depth>|oddindex}
\begin{reflist}
\item[]\begin{specHead}{TEI.depth}{<depth> }contains a measurement measured across the spine of a book or codex, or (for other text-bearing objects) perpendicular to the measurement given by the ‘width’ element. [\xref{http://www.tei-c.org/release/doc/tei-p5-doc/en/html/MS.html\#msdim}{10.3.4. Dimensions}]\end{specHead} 
    \item[{Module}]
  msdescription
    \item[{Attributes}]
  Attributes att.global (\textit{@xml:id}, \textit{@n}, \textit{@xml:lang}, \textit{@xml:base}, \textit{@xml:space})  (att.global.rendition (\textit{@rend}, \textit{@style}, \textit{@rendition})) (att.global.linking (\textit{@corresp}, \textit{@synch}, \textit{@sameAs}, \textit{@copyOf}, \textit{@next}, \textit{@prev}, \textit{@exclude}, \textit{@select})) (att.global.analytic (\textit{@ana})) (att.global.facs (\textit{@facs})) (att.global.change (\textit{@change})) (att.global.responsibility (\textit{@cert}, \textit{@resp})) (att.global.source (\textit{@source})) att.dimensions (\textit{@unit}, \textit{@quantity}, \textit{@extent}, \textit{@precision}, \textit{@scope})  (att.ranging (\textit{@atLeast}, \textit{@atMost}, \textit{@min}, \textit{@max}, \textit{@confidence}))
    \item[{Member of}]
  model.dimLike model.measureLike
    \item[{Contained by}]
  
    \item[analysis: ]
   cl phr s span\par 
    \item[core: ]
   abbr add addrLine author bibl biblScope citedRange corr date del desc distinct editor email emph expan foreign gloss head headItem headLabel hi item l label measure measureGrp meeting mentioned name note num orig p pubPlace publisher q quote ref reg resp rs said sic soCalled speaker stage street term textLang time title unclear\par 
    \item[figures: ]
   cell figDesc\par 
    \item[header: ]
   authority catDesc change classCode creation distributor edition extent funder geoDecl handNote language licence principal rendition scriptNote sponsor tagUsage typeNote\par 
    \item[linking: ]
   ab seg\par 
    \item[msdescription: ]
   accMat acquisition additions catchwords collation colophon condition custEvent decoNote dimensions explicit filiation finalRubric foliation heraldry incipit layout material musicNotation objectType origDate origPlace origin provenance rubric secFol signatures source stamp summary support surrogates watermark\par 
    \item[namesdates: ]
   addName affiliation age birth bloc country death district education faith floruit forename genName geogFeat geogName langKnown location nameLink nationality occupation offset orgName persName placeName region residence roleName settlement sex socecStatus surname\par 
    \item[textcrit: ]
   lem rdg wit witDetail witness\par 
    \item[textstructure: ]
   byline closer dateline docAuthor docDate docEdition docImprint imprimatur opener salute signed titlePart trailer\par 
    \item[transcr: ]
   damage fw metamark mod restore retrace secl supplied surplus
    \item[{May contain}]
  
    \item[gaiji: ]
   g\par character data
    \item[{Note}]
  \par
If used to specify the width of a non text-bearing portion of some object, for example a monument, this element conventionally refers to the axis facing the observer, and perpendicular to that indicated by the ‘width’ axis.
    \item[{Example}]
  \leavevmode\bgroup\exampleFont \begin{shaded}\noindent\mbox{}{<\textbf{depth}\hspace*{6pt}{quantity}="{4}"\hspace*{6pt}{unit}="{in}"/>}\end{shaded}\egroup 


    \item[{Content model}]
  \fbox{\ttfamily <content>\newline
 <macroRef key="macro.xtext"/>\newline
</content>\newline
    } 
    \item[{Schema Declaration}]
  \mbox{}\hfill\\[-10pt]\begin{Verbatim}[fontsize=\small]
element depth { att.global.attributes, att.dimensions.attributes, macro.xtext }
\end{Verbatim}

\end{reflist}  \index{desc=<desc>|oddindex}
\begin{reflist}
\item[]\begin{specHead}{TEI.desc}{<desc> }(description) contains a brief description of the object documented by its parent element, typically a documentation element or an entity. [\xref{http://www.tei-c.org/release/doc/tei-p5-doc/en/html/TD.html\#TDcrystalsCEdc}{22.4.1. Description of Components}]\end{specHead} 
    \item[{Module}]
  core
    \item[{Attributes}]
  Attributes att.global (\textit{@xml:id}, \textit{@n}, \textit{@xml:lang}, \textit{@xml:base}, \textit{@xml:space})  (att.global.rendition (\textit{@rend}, \textit{@style}, \textit{@rendition})) (att.global.linking (\textit{@corresp}, \textit{@synch}, \textit{@sameAs}, \textit{@copyOf}, \textit{@next}, \textit{@prev}, \textit{@exclude}, \textit{@select})) (att.global.analytic (\textit{@ana})) (att.global.facs (\textit{@facs})) (att.global.change (\textit{@change})) (att.global.responsibility (\textit{@cert}, \textit{@resp})) (att.global.source (\textit{@source})) att.translatable (\textit{@versionDate}) att.typed (\textit{@type}, \textit{@subtype}) 
    \item[{Member of}]
  model.descLike model.labelLike 
    \item[{Contained by}]
  
    \item[analysis: ]
   interp interpGrp\par 
    \item[core: ]
   add corr del desc emph gap graphic head hi item l lg media meeting note orig p q quote ref reg said sic stage title unclear\par 
    \item[figures: ]
   cell figDesc figure notatedMusic\par 
    \item[gaiji: ]
   char charDecl glyph\par 
    \item[header: ]
   application category change handNote licence rendition schemaRef scriptNote tagUsage taxonomy typeNote\par 
    \item[linking: ]
   ab join seg\par 
    \item[msdescription: ]
   accMat acquisition additions collation condition custEvent decoNote filiation foliation layout musicNotation origin provenance signatures source summary support surrogates\par 
    \item[namesdates: ]
   climate event location occupation org place population relation state terrain trait\par 
    \item[textcrit: ]
   lem rdg witness\par 
    \item[textstructure: ]
   argument body div docEdition epigraph imprimatur postscript salute signed titlePart trailer\par 
    \item[transcr: ]
   damage metamark mod restore retrace secl space substJoin supplied surface surplus
    \item[{May contain}]
  
    \item[core: ]
   abbr address bibl biblStruct choice cit date desc distinct email emph expan foreign gloss hi label list listBibl measure measureGrp mentioned name num ptr q quote ref rs said soCalled stage term time title\par 
    \item[figures: ]
   table\par 
    \item[header: ]
   biblFull idno\par 
    \item[msdescription: ]
   catchwords depth dim dimensions height heraldry locus locusGrp material msDesc objectType origDate origPlace secFol signatures stamp watermark width\par 
    \item[namesdates: ]
   addName affiliation bloc climate country district forename genName geo geogFeat geogName listEvent listNym listOrg listPerson listPlace location nameLink offset orgName persName placeName population region roleName settlement state surname terrain trait\par 
    \item[textcrit: ]
   listApp listWit\par 
    \item[textstructure: ]
   floatingText\par 
    \item[transcr: ]
   am ex subst\par character data
    \item[{Note}]
  \par
When used in a specification element such as \texttt{<elementSpec>}, TEI convention requires that this be expressed as a finite clause, begining with an active verb.
    \item[{Example}]
  \leavevmode\bgroup\exampleFont \begin{shaded}\noindent\mbox{}{<\textbf{desc}>}contains a brief description of the purpose and intended use of a documentation element, or a brief characterisation of a parent entity {</\textbf{desc}>}\end{shaded}\egroup 


    \item[{Content model}]
  \mbox{}\hfill\\[-10pt]\begin{Verbatim}[fontsize=\small]
<content>
 <macroRef key="macro.limitedContent"/>
</content>
    
\end{Verbatim}

    \item[{Schema Declaration}]
  \mbox{}\hfill\\[-10pt]\begin{Verbatim}[fontsize=\small]
element desc
{
   att.global.attributes,
   att.translatable.attributes,
   att.typed.attributes,
   macro.limitedContent}
\end{Verbatim}

\end{reflist}  \index{dim=<dim>|oddindex}
\begin{reflist}
\item[]\begin{specHead}{TEI.dim}{<dim> }contains any single measurement forming part of a dimensional specification of some sort. [\xref{http://www.tei-c.org/release/doc/tei-p5-doc/en/html/MS.html\#msdim}{10.3.4. Dimensions}]\end{specHead} 
    \item[{Module}]
  msdescription
    \item[{Attributes}]
  Attributes att.global (\textit{@xml:id}, \textit{@n}, \textit{@xml:lang}, \textit{@xml:base}, \textit{@xml:space})  (att.global.rendition (\textit{@rend}, \textit{@style}, \textit{@rendition})) (att.global.linking (\textit{@corresp}, \textit{@synch}, \textit{@sameAs}, \textit{@copyOf}, \textit{@next}, \textit{@prev}, \textit{@exclude}, \textit{@select})) (att.global.analytic (\textit{@ana})) (att.global.facs (\textit{@facs})) (att.global.change (\textit{@change})) (att.global.responsibility (\textit{@cert}, \textit{@resp})) (att.global.source (\textit{@source})) att.typed (\textit{@type}, \textit{@subtype}) att.dimensions (\textit{@unit}, \textit{@quantity}, \textit{@extent}, \textit{@precision}, \textit{@scope})  (att.ranging (\textit{@atLeast}, \textit{@atMost}, \textit{@min}, \textit{@max}, \textit{@confidence}))
    \item[{Member of}]
  model.measureLike
    \item[{Contained by}]
  
    \item[analysis: ]
   cl phr s span\par 
    \item[core: ]
   abbr add addrLine author bibl biblScope citedRange corr date del desc distinct editor email emph expan foreign gloss head headItem headLabel hi item l label measure measureGrp meeting mentioned name note num orig p pubPlace publisher q quote ref reg resp rs said sic soCalled speaker stage street term textLang time title unclear\par 
    \item[figures: ]
   cell figDesc\par 
    \item[header: ]
   authority catDesc change classCode creation distributor edition extent funder geoDecl handNote language licence principal rendition scriptNote sponsor tagUsage typeNote\par 
    \item[linking: ]
   ab seg\par 
    \item[msdescription: ]
   accMat acquisition additions catchwords collation colophon condition custEvent decoNote dimensions explicit filiation finalRubric foliation heraldry incipit layout material musicNotation objectType origDate origPlace origin provenance rubric secFol signatures source stamp summary support surrogates watermark\par 
    \item[namesdates: ]
   addName affiliation age birth bloc country death district education faith floruit forename genName geogFeat geogName langKnown location nameLink nationality occupation offset orgName persName placeName region residence roleName settlement sex socecStatus surname\par 
    \item[textcrit: ]
   lem rdg wit witDetail witness\par 
    \item[textstructure: ]
   byline closer dateline docAuthor docDate docEdition docImprint imprimatur opener salute signed titlePart trailer\par 
    \item[transcr: ]
   damage fw metamark mod restore retrace secl supplied surplus
    \item[{May contain}]
  
    \item[gaiji: ]
   g\par character data
    \item[{Note}]
  \par
The specific elements <width>, <height>, and <depth> should be used in preference to this generic element wherever appropriate.
    \item[{Example}]
  \leavevmode\bgroup\exampleFont \begin{shaded}\noindent\mbox{}{<\textbf{dim}\hspace*{6pt}{extent}="{4.67 in}"\hspace*{6pt}{type}="{circumference}"/>}\end{shaded}\egroup 


    \item[{Content model}]
  \fbox{\ttfamily <content>\newline
 <macroRef key="macro.xtext"/>\newline
</content>\newline
    } 
    \item[{Schema Declaration}]
  \mbox{}\hfill\\[-10pt]\begin{Verbatim}[fontsize=\small]
element dim
{
   att.global.attributes,
   att.typed.attributes,
   att.dimensions.attributes,
   macro.xtext}
\end{Verbatim}

\end{reflist}  \index{dimensions=<dimensions>|oddindex}\index{type=@type!<dimensions>|oddindex}
\begin{reflist}
\item[]\begin{specHead}{TEI.dimensions}{<dimensions> }contains a dimensional specification. [\xref{http://www.tei-c.org/release/doc/tei-p5-doc/en/html/MS.html\#msdim}{10.3.4. Dimensions}]\end{specHead} 
    \item[{Module}]
  msdescription
    \item[{Attributes}]
  Attributes att.global (\textit{@xml:id}, \textit{@n}, \textit{@xml:lang}, \textit{@xml:base}, \textit{@xml:space})  (att.global.rendition (\textit{@rend}, \textit{@style}, \textit{@rendition})) (att.global.linking (\textit{@corresp}, \textit{@synch}, \textit{@sameAs}, \textit{@copyOf}, \textit{@next}, \textit{@prev}, \textit{@exclude}, \textit{@select})) (att.global.analytic (\textit{@ana})) (att.global.facs (\textit{@facs})) (att.global.change (\textit{@change})) (att.global.responsibility (\textit{@cert}, \textit{@resp})) (att.global.source (\textit{@source})) att.dimensions (\textit{@unit}, \textit{@quantity}, \textit{@extent}, \textit{@precision}, \textit{@scope})  (att.ranging (\textit{@atLeast}, \textit{@atMost}, \textit{@min}, \textit{@max}, \textit{@confidence})) \hfil\\[-10pt]\begin{sansreflist}
    \item[@type]
  indicates which aspect of the object is being measured.
\begin{reflist}
    \item[{Status}]
  Optional
    \item[{Datatype}]
  teidata.enumerated
    \item[{Sample values include:}]
  \begin{description}

\item[{leaves}]dimensions relate to one or more leaves (e.g. a single leaf, a gathering, or a separately bound part)
\item[{ruled}]dimensions relate to the area of a leaf which has been ruled in preparation for writing.
\item[{pricked}]dimensions relate to the area of a leaf which has been pricked out in preparation for ruling (used where this differs significantly from the ruled area, or where the ruling is not measurable).
\item[{written}]dimensions relate to the area of a leaf which has been written, with the height measured from the top of the minims on the top line of writing, to the bottom of the minims on the bottom line of writing.
\item[{miniatures}]dimensions relate to the miniatures within the manuscript
\item[{binding}]dimensions relate to the binding in which the codex or manuscript is contained
\item[{box}]dimensions relate to the box or other container in which the manuscript is stored.
\end{description} 
\end{reflist}  
\end{sansreflist}  
    \item[{Member of}]
  model.pPart.msdesc
    \item[{Contained by}]
  
    \item[analysis: ]
   cl phr s span\par 
    \item[core: ]
   abbr add addrLine author biblScope citedRange corr date del desc distinct editor email emph expan foreign gloss head headItem headLabel hi item l label measure meeting mentioned name note num orig p pubPlace publisher q quote ref reg resp rs said sic soCalled speaker stage street term textLang time title unclear\par 
    \item[figures: ]
   cell figDesc\par 
    \item[header: ]
   authority catDesc change classCode creation distributor edition extent funder geoDecl handNote language licence principal rendition scriptNote sponsor tagUsage typeNote\par 
    \item[linking: ]
   ab seg\par 
    \item[msdescription: ]
   accMat acquisition additions catchwords collation colophon condition custEvent decoNote explicit filiation finalRubric foliation heraldry incipit layout material musicNotation objectType origDate origPlace origin provenance rubric secFol signatures source stamp summary support surrogates watermark\par 
    \item[namesdates: ]
   addName affiliation age birth bloc country death district education faith floruit forename genName geogFeat geogName langKnown nameLink nationality occupation offset orgName persName placeName region residence roleName settlement sex socecStatus surname\par 
    \item[textcrit: ]
   lem rdg wit witDetail witness\par 
    \item[textstructure: ]
   byline closer dateline docAuthor docDate docEdition docImprint imprimatur opener salute signed titlePart trailer\par 
    \item[transcr: ]
   damage fw metamark mod restore retrace secl supplied surplus
    \item[{May contain}]
  
    \item[msdescription: ]
   depth dim height width
    \item[{Note}]
  \par
Contains no more than one of each of the specialized elements used to express a three-dimensional object's height, width, and depth, combined with any number of other kinds of dimensional specification.
    \item[{Example}]
  \leavevmode\bgroup\exampleFont \begin{shaded}\noindent\mbox{}{<\textbf{dimensions}\hspace*{6pt}{type}="{leaves}">}\mbox{}\newline 
\hspace*{6pt}{<\textbf{height}\hspace*{6pt}{scope}="{range}">}157-160{</\textbf{height}>}\mbox{}\newline 
\hspace*{6pt}{<\textbf{width}>}105{</\textbf{width}>}\mbox{}\newline 
{</\textbf{dimensions}>}\mbox{}\newline 
{<\textbf{dimensions}\hspace*{6pt}{type}="{ruled}">}\mbox{}\newline 
\hspace*{6pt}{<\textbf{height}\hspace*{6pt}{scope}="{most}">}90{</\textbf{height}>}\mbox{}\newline 
\hspace*{6pt}{<\textbf{width}\hspace*{6pt}{scope}="{most}">}48{</\textbf{width}>}\mbox{}\newline 
{</\textbf{dimensions}>}\mbox{}\newline 
{<\textbf{dimensions}\hspace*{6pt}{unit}="{in}">}\mbox{}\newline 
\hspace*{6pt}{<\textbf{height}>}12{</\textbf{height}>}\mbox{}\newline 
\hspace*{6pt}{<\textbf{width}>}10{</\textbf{width}>}\mbox{}\newline 
{</\textbf{dimensions}>}\end{shaded}\egroup 


    \item[{Example}]
  This element may be used to record the dimensions of any text-bearing object, not necessarily a codex. For example:\leavevmode\bgroup\exampleFont \begin{shaded}\noindent\mbox{}{<\textbf{dimensions}\hspace*{6pt}{type}="{panels}">}\mbox{}\newline 
\hspace*{6pt}{<\textbf{height}\hspace*{6pt}{scope}="{all}">}7004{</\textbf{height}>}\mbox{}\newline 
\hspace*{6pt}{<\textbf{width}\hspace*{6pt}{scope}="{all}">}1803{</\textbf{width}>}\mbox{}\newline 
\hspace*{6pt}{<\textbf{dim}\hspace*{6pt}{type}="{relief}"\hspace*{6pt}{unit}="{mm}">}345{</\textbf{dim}>}\mbox{}\newline 
{</\textbf{dimensions}>}\end{shaded}\egroup 

This might be used to show that the inscribed panels on some (imaginary) monument are all the same size (7004 by 1803 cm) and stand out from the rest of the monument by 345 mm.
    \item[{Example}]
  When simple numeric quantities are involved, they may be expressed on the {\itshape quantity} attribute of any or all of the child elements, as in the following example:\leavevmode\bgroup\exampleFont \begin{shaded}\noindent\mbox{}{<\textbf{dimensions}\hspace*{6pt}{type}="{leaves}">}\mbox{}\newline 
\hspace*{6pt}{<\textbf{height}\hspace*{6pt}{scope}="{range}">}157-160{</\textbf{height}>}\mbox{}\newline 
\hspace*{6pt}{<\textbf{width}\hspace*{6pt}{quantity}="{105}"/>}\mbox{}\newline 
{</\textbf{dimensions}>}\mbox{}\newline 
{<\textbf{dimensions}\hspace*{6pt}{type}="{ruled}">}\mbox{}\newline 
\hspace*{6pt}{<\textbf{height}\hspace*{6pt}{quantity}="{90}"\hspace*{6pt}{scope}="{most}"\mbox{}\newline 
\hspace*{6pt}\hspace*{6pt}{unit}="{cm}"/>}\mbox{}\newline 
\hspace*{6pt}{<\textbf{width}\hspace*{6pt}{quantity}="{48}"\hspace*{6pt}{scope}="{most}"\hspace*{6pt}{unit}="{cm}"/>}\mbox{}\newline 
{</\textbf{dimensions}>}\mbox{}\newline 
{<\textbf{dimensions}\hspace*{6pt}{unit}="{in}">}\mbox{}\newline 
\hspace*{6pt}{<\textbf{height}\hspace*{6pt}{quantity}="{12}"/>}\mbox{}\newline 
\hspace*{6pt}{<\textbf{width}\hspace*{6pt}{quantity}="{10}"/>}\mbox{}\newline 
{</\textbf{dimensions}>}\end{shaded}\egroup 


    \item[{Schematron}]
   <s:report test="count(tei:width)> 1">The element <s:name/> may appear once only </s:report> <s:report test="count(tei:height)> 1">The element <s:name/> may appear once only </s:report> <s:report test="count(tei:depth)> 1">The element <s:name/> may appear once only </s:report>
    \item[{Content model}]
  \mbox{}\hfill\\[-10pt]\begin{Verbatim}[fontsize=\small]
<content>
 <alternate maxOccurs="unbounded"
  minOccurs="0">
  <elementRef key="dim"/>
  <classRef key="model.dimLike"/>
 </alternate>
</content>
    
\end{Verbatim}

    \item[{Schema Declaration}]
  \mbox{}\hfill\\[-10pt]\begin{Verbatim}[fontsize=\small]
element dimensions
{
   att.global.attributes,
   att.dimensions.attributes,
   attribute type { text }?,
   ( dim | model.dimLike )*
}
\end{Verbatim}

\end{reflist}  \index{distinct=<distinct>|oddindex}\index{type=@type!<distinct>|oddindex}\index{time=@time!<distinct>|oddindex}\index{space=@space!<distinct>|oddindex}\index{social=@social!<distinct>|oddindex}
\begin{reflist}
\item[]\begin{specHead}{TEI.distinct}{<distinct> }identifies any word or phrase which is regarded as linguistically distinct, for example as archaic, technical, dialectal, non-preferred, etc., or as forming part of a sublanguage. [\xref{http://www.tei-c.org/release/doc/tei-p5-doc/en/html/CO.html\#COHQHD}{3.3.2.3. Other Linguistically Distinct Material}]\end{specHead} 
    \item[{Module}]
  core
    \item[{Attributes}]
  Attributes att.global (\textit{@xml:id}, \textit{@n}, \textit{@xml:lang}, \textit{@xml:base}, \textit{@xml:space})  (att.global.rendition (\textit{@rend}, \textit{@style}, \textit{@rendition})) (att.global.linking (\textit{@corresp}, \textit{@synch}, \textit{@sameAs}, \textit{@copyOf}, \textit{@next}, \textit{@prev}, \textit{@exclude}, \textit{@select})) (att.global.analytic (\textit{@ana})) (att.global.facs (\textit{@facs})) (att.global.change (\textit{@change})) (att.global.responsibility (\textit{@cert}, \textit{@resp})) (att.global.source (\textit{@source})) \hfil\\[-10pt]\begin{sansreflist}
    \item[@type]
  specifies the sublanguage or register to which the word or phrase is being assigned
\begin{reflist}
    \item[{Status}]
  Optional
    \item[{Datatype}]
  teidata.enumerated
\end{reflist}  
    \item[@time]
  specifies how the phrase is distinct diachronically
\begin{reflist}
    \item[{Status}]
  Optional
    \item[{Datatype}]
  teidata.text
\end{reflist}  
    \item[@space]
  specifies how the phrase is distinct diatopically
\begin{reflist}
    \item[{Status}]
  Optional
    \item[{Datatype}]
  teidata.text
\end{reflist}  
    \item[@social]
  specifies how the phrase is distinct diastratically
\begin{reflist}
    \item[{Status}]
  Optional
    \item[{Datatype}]
  teidata.text
\end{reflist}  
\end{sansreflist}  
    \item[{Member of}]
  model.emphLike
    \item[{Contained by}]
  
    \item[analysis: ]
   cl phr s span\par 
    \item[core: ]
   abbr add addrLine author bibl biblScope citedRange corr date del desc distinct editor email emph expan foreign gloss head headItem headLabel hi item l label measure meeting mentioned name note num orig p pubPlace publisher q quote ref reg resp rs said sic soCalled speaker stage street term textLang time title unclear\par 
    \item[figures: ]
   cell figDesc\par 
    \item[header: ]
   authority catDesc change classCode creation distributor edition extent funder geoDecl handNote language licence principal rendition scriptNote sponsor tagUsage typeNote\par 
    \item[linking: ]
   ab seg\par 
    \item[msdescription: ]
   accMat acquisition additions catchwords collation colophon condition custEvent decoNote explicit filiation finalRubric foliation heraldry incipit layout material musicNotation objectType origDate origPlace origin provenance rubric secFol signatures source stamp summary support surrogates watermark\par 
    \item[namesdates: ]
   addName affiliation age birth bloc country death district education faith floruit forename genName geogFeat geogName langKnown nameLink nationality occupation offset orgName persName placeName region residence roleName settlement sex socecStatus surname\par 
    \item[textcrit: ]
   lem rdg wit witDetail witness\par 
    \item[textstructure: ]
   byline closer dateline docAuthor docDate docEdition docImprint imprimatur opener salute signed titlePart trailer\par 
    \item[transcr: ]
   damage fw metamark mod restore retrace secl supplied surplus
    \item[{May contain}]
  
    \item[analysis: ]
   c cl interp interpGrp m pc phr s span spanGrp w\par 
    \item[core: ]
   abbr add address cb choice corr date del distinct email emph expan foreign gap gb gloss graphic hi index lb measure measureGrp media mentioned milestone name note num orig pb ptr ref reg rs sic soCalled term time title unclear\par 
    \item[figures: ]
   figure formula notatedMusic\par 
    \item[gaiji: ]
   g\par 
    \item[header: ]
   idno\par 
    \item[linking: ]
   alt altGrp anchor join joinGrp link linkGrp seg timeline\par 
    \item[msdescription: ]
   catchwords depth dim dimensions height heraldry locus locusGrp material objectType origDate origPlace secFol signatures stamp watermark width\par 
    \item[namesdates: ]
   addName affiliation bloc climate country district forename genName geo geogFeat geogName location nameLink offset orgName persName placeName population region roleName settlement state surname terrain trait\par 
    \item[textcrit: ]
   app witDetail\par 
    \item[transcr: ]
   addSpan am damage damageSpan delSpan ex fw handShift listTranspose metamark mod redo restore retrace secl space subst substJoin supplied surplus undo\par character data
    \item[{Example}]
  \leavevmode\bgroup\exampleFont \begin{shaded}\noindent\mbox{}Next morning a boy\mbox{}\newline 
 in that dormitory confided to his bosom friend, a {<\textbf{distinct}\hspace*{6pt}{type}="{ps\textunderscore slang}">}fag{</\textbf{distinct}>} of\mbox{}\newline 
 Macrea's, that there was trouble in their midst which King {<\textbf{distinct}\hspace*{6pt}{type}="{archaic}">}would fain{</\textbf{distinct}>} \mbox{}\newline 
 keep secret.\mbox{}\newline 
\end{shaded}\egroup 


    \item[{Content model}]
  \mbox{}\hfill\\[-10pt]\begin{Verbatim}[fontsize=\small]
<content>
 <macroRef key="macro.phraseSeq"/>
</content>
    
\end{Verbatim}

    \item[{Schema Declaration}]
  \mbox{}\hfill\\[-10pt]\begin{Verbatim}[fontsize=\small]
element distinct
{
   att.global.attributes,
   attribute type { text }?,
   attribute time { text }?,
   attribute space { text }?,
   attribute social { text }?,
   macro.phraseSeq}
\end{Verbatim}

\end{reflist}  \index{distributor=<distributor>|oddindex}
\begin{reflist}
\item[]\begin{specHead}{TEI.distributor}{<distributor> }supplies the name of a person or other agency responsible for the distribution of a text. [\xref{http://www.tei-c.org/release/doc/tei-p5-doc/en/html/HD.html\#HD24}{2.2.4. Publication, Distribution, Licensing, etc.}]\end{specHead} 
    \item[{Module}]
  header
    \item[{Attributes}]
  Attributes att.global (\textit{@xml:id}, \textit{@n}, \textit{@xml:lang}, \textit{@xml:base}, \textit{@xml:space})  (att.global.rendition (\textit{@rend}, \textit{@style}, \textit{@rendition})) (att.global.linking (\textit{@corresp}, \textit{@synch}, \textit{@sameAs}, \textit{@copyOf}, \textit{@next}, \textit{@prev}, \textit{@exclude}, \textit{@select})) (att.global.analytic (\textit{@ana})) (att.global.facs (\textit{@facs})) (att.global.change (\textit{@change})) (att.global.responsibility (\textit{@cert}, \textit{@resp})) (att.global.source (\textit{@source}))
    \item[{Member of}]
  model.imprintPart model.publicationStmtPart.agency
    \item[{Contained by}]
  
    \item[core: ]
   bibl imprint\par 
    \item[header: ]
   publicationStmt
    \item[{May contain}]
  
    \item[analysis: ]
   c cl interp interpGrp m pc phr s span spanGrp w\par 
    \item[core: ]
   abbr add address cb choice corr date del distinct email emph expan foreign gap gb gloss graphic hi index lb measure measureGrp media mentioned milestone name note num orig pb ptr ref reg rs sic soCalled term time title unclear\par 
    \item[figures: ]
   figure formula notatedMusic\par 
    \item[gaiji: ]
   g\par 
    \item[header: ]
   idno\par 
    \item[linking: ]
   alt altGrp anchor join joinGrp link linkGrp seg timeline\par 
    \item[msdescription: ]
   catchwords depth dim dimensions height heraldry locus locusGrp material objectType origDate origPlace secFol signatures stamp watermark width\par 
    \item[namesdates: ]
   addName affiliation bloc climate country district forename genName geo geogFeat geogName location nameLink offset orgName persName placeName population region roleName settlement state surname terrain trait\par 
    \item[textcrit: ]
   app witDetail\par 
    \item[transcr: ]
   addSpan am damage damageSpan delSpan ex fw handShift listTranspose metamark mod redo restore retrace secl space subst substJoin supplied surplus undo\par character data
    \item[{Example}]
  \leavevmode\bgroup\exampleFont \begin{shaded}\noindent\mbox{}{<\textbf{distributor}>}Oxford Text Archive{</\textbf{distributor}>}\mbox{}\newline 
{<\textbf{distributor}>}Redwood and Burn Ltd{</\textbf{distributor}>}\end{shaded}\egroup 


    \item[{Content model}]
  \mbox{}\hfill\\[-10pt]\begin{Verbatim}[fontsize=\small]
<content>
 <macroRef key="macro.phraseSeq"/>
</content>
    
\end{Verbatim}

    \item[{Schema Declaration}]
  \mbox{}\hfill\\[-10pt]\begin{Verbatim}[fontsize=\small]
element distributor { att.global.attributes, macro.phraseSeq }
\end{Verbatim}

\end{reflist}  \index{district=<district>|oddindex}
\begin{reflist}
\item[]\begin{specHead}{TEI.district}{<district> }contains the name of any kind of subdivision of a settlement, such as a parish, ward, or other administrative or geographic unit. [\xref{http://www.tei-c.org/release/doc/tei-p5-doc/en/html/ND.html\#NDPLAC}{13.2.3. Place Names}]\end{specHead} 
    \item[{Module}]
  namesdates
    \item[{Attributes}]
  Attributes att.global (\textit{@xml:id}, \textit{@n}, \textit{@xml:lang}, \textit{@xml:base}, \textit{@xml:space})  (att.global.rendition (\textit{@rend}, \textit{@style}, \textit{@rendition})) (att.global.linking (\textit{@corresp}, \textit{@synch}, \textit{@sameAs}, \textit{@copyOf}, \textit{@next}, \textit{@prev}, \textit{@exclude}, \textit{@select})) (att.global.analytic (\textit{@ana})) (att.global.facs (\textit{@facs})) (att.global.change (\textit{@change})) (att.global.responsibility (\textit{@cert}, \textit{@resp})) (att.global.source (\textit{@source})) att.naming (\textit{@role}, \textit{@nymRef})  (att.canonical (\textit{@key}, \textit{@ref})) att.typed (\textit{@type}, \textit{@subtype}) att.datable (\textit{@calendar}, \textit{@period})  (att.datable.w3c (\textit{@when}, \textit{@notBefore}, \textit{@notAfter}, \textit{@from}, \textit{@to})) (att.datable.iso (\textit{@when-iso}, \textit{@notBefore-iso}, \textit{@notAfter-iso}, \textit{@from-iso}, \textit{@to-iso})) (att.datable.custom (\textit{@when-custom}, \textit{@notBefore-custom}, \textit{@notAfter-custom}, \textit{@from-custom}, \textit{@to-custom}, \textit{@datingPoint}, \textit{@datingMethod}))
    \item[{Member of}]
  model.placeNamePart
    \item[{Contained by}]
  
    \item[analysis: ]
   cl phr s span\par 
    \item[core: ]
   abbr add addrLine address author bibl biblScope citedRange corr date del desc distinct editor email emph expan foreign gloss head headItem headLabel hi item l label measure meeting mentioned name note num orig p pubPlace publisher q quote ref reg resp rs said sic soCalled speaker stage street term textLang time title unclear\par 
    \item[figures: ]
   cell figDesc\par 
    \item[header: ]
   authority catDesc change classCode correspAction creation distributor edition extent funder geoDecl handNote language licence principal rendition scriptNote sponsor tagUsage typeNote\par 
    \item[linking: ]
   ab seg\par 
    \item[msdescription: ]
   accMat acquisition additions altIdentifier catchwords collation colophon condition custEvent decoNote explicit filiation finalRubric foliation heraldry incipit layout material msIdentifier musicNotation objectType origDate origPlace origin provenance rubric secFol signatures source stamp summary support surrogates watermark\par 
    \item[namesdates: ]
   addName affiliation age birth bloc country death district education faith floruit forename genName geogFeat geogName langKnown location nameLink nationality occupation offset org orgName persName place placeName region residence roleName settlement sex socecStatus surname\par 
    \item[textcrit: ]
   lem rdg wit witDetail witness\par 
    \item[textstructure: ]
   byline closer dateline docAuthor docDate docEdition docImprint imprimatur opener salute signed titlePart trailer\par 
    \item[transcr: ]
   damage fw metamark mod restore retrace secl supplied surplus
    \item[{May contain}]
  
    \item[analysis: ]
   c cl interp interpGrp m pc phr s span spanGrp w\par 
    \item[core: ]
   abbr add address cb choice corr date del distinct email emph expan foreign gap gb gloss graphic hi index lb measure measureGrp media mentioned milestone name note num orig pb ptr ref reg rs sic soCalled term time title unclear\par 
    \item[figures: ]
   figure formula notatedMusic\par 
    \item[gaiji: ]
   g\par 
    \item[header: ]
   idno\par 
    \item[linking: ]
   alt altGrp anchor join joinGrp link linkGrp seg timeline\par 
    \item[msdescription: ]
   catchwords depth dim dimensions height heraldry locus locusGrp material objectType origDate origPlace secFol signatures stamp watermark width\par 
    \item[namesdates: ]
   addName affiliation bloc climate country district forename genName geo geogFeat geogName location nameLink offset orgName persName placeName population region roleName settlement state surname terrain trait\par 
    \item[textcrit: ]
   app witDetail\par 
    \item[transcr: ]
   addSpan am damage damageSpan delSpan ex fw handShift listTranspose metamark mod redo restore retrace secl space subst substJoin supplied surplus undo\par character data
    \item[{Example}]
  \leavevmode\bgroup\exampleFont \begin{shaded}\noindent\mbox{}{<\textbf{placeName}>}\mbox{}\newline 
\hspace*{6pt}{<\textbf{district}\hspace*{6pt}{type}="{ward}">}Jericho{</\textbf{district}>}\mbox{}\newline 
\hspace*{6pt}{<\textbf{settlement}>}Oxford{</\textbf{settlement}>}\mbox{}\newline 
{</\textbf{placeName}>}\end{shaded}\egroup 


    \item[{Example}]
  \leavevmode\bgroup\exampleFont \begin{shaded}\noindent\mbox{}{<\textbf{placeName}>}\mbox{}\newline 
\hspace*{6pt}{<\textbf{district}\hspace*{6pt}{type}="{area}">}South Side{</\textbf{district}>}\mbox{}\newline 
\hspace*{6pt}{<\textbf{settlement}>}Chicago{</\textbf{settlement}>}\mbox{}\newline 
{</\textbf{placeName}>}\end{shaded}\egroup 


    \item[{Content model}]
  \mbox{}\hfill\\[-10pt]\begin{Verbatim}[fontsize=\small]
<content>
 <macroRef key="macro.phraseSeq"/>
</content>
    
\end{Verbatim}

    \item[{Schema Declaration}]
  \mbox{}\hfill\\[-10pt]\begin{Verbatim}[fontsize=\small]
element district
{
   att.global.attributes,
   att.naming.attributes,
   att.typed.attributes,
   att.datable.attributes,
   macro.phraseSeq}
\end{Verbatim}

\end{reflist}  \index{div=<div>|oddindex}
\begin{reflist}
\item[]\begin{specHead}{TEI.div}{<div> }(text division) contains a subdivision of the front, body, or back of a text. [\xref{http://www.tei-c.org/release/doc/tei-p5-doc/en/html/DS.html\#DSDIV}{4.1. Divisions of the Body}]\end{specHead} 
    \item[{Module}]
  textstructure
    \item[{Attributes}]
  Attributes att.global (\textit{@xml:id}, \textit{@n}, \textit{@xml:lang}, \textit{@xml:base}, \textit{@xml:space})  (att.global.rendition (\textit{@rend}, \textit{@style}, \textit{@rendition})) (att.global.linking (\textit{@corresp}, \textit{@synch}, \textit{@sameAs}, \textit{@copyOf}, \textit{@next}, \textit{@prev}, \textit{@exclude}, \textit{@select})) (att.global.analytic (\textit{@ana})) (att.global.facs (\textit{@facs})) (att.global.change (\textit{@change})) (att.global.responsibility (\textit{@cert}, \textit{@resp})) (att.global.source (\textit{@source})) att.divLike (\textit{@org}, \textit{@sample})  (att.fragmentable (\textit{@part})) att.typed (\textit{@type}, \textit{@subtype}) att.declaring (\textit{@decls}) att.written (\textit{@hand}) 
    \item[{Member of}]
  model.divLike
    \item[{Contained by}]
  
    \item[textcrit: ]
   lem rdg\par 
    \item[textstructure: ]
   back body div front
    \item[{May contain}]
  
    \item[analysis: ]
   interp interpGrp span spanGrp\par 
    \item[core: ]
   bibl biblStruct cb cit desc divGen gap gb head index l label lb lg list listBibl meeting milestone note p pb q quote said sp stage\par 
    \item[figures: ]
   figure notatedMusic table\par 
    \item[header: ]
   biblFull\par 
    \item[linking: ]
   ab alt altGrp anchor join joinGrp link linkGrp timeline\par 
    \item[msdescription: ]
   msDesc\par 
    \item[namesdates: ]
   listEvent listNym listOrg listPerson listPlace\par 
    \item[textcrit: ]
   app listApp listWit witDetail\par 
    \item[textstructure: ]
   argument byline closer dateline div docAuthor docDate epigraph floatingText opener postscript salute signed trailer\par 
    \item[transcr: ]
   addSpan damageSpan delSpan fw listTranspose metamark space substJoin
    \item[{Example}]
  \leavevmode\bgroup\exampleFont \begin{shaded}\noindent\mbox{}{<\textbf{body}>}\mbox{}\newline 
\hspace*{6pt}{<\textbf{div}\hspace*{6pt}{type}="{part}">}\mbox{}\newline 
\hspace*{6pt}\hspace*{6pt}{<\textbf{head}>}Fallacies of Authority{</\textbf{head}>}\mbox{}\newline 
\hspace*{6pt}\hspace*{6pt}{<\textbf{p}>}The subject of which is Authority in various shapes, and the object, to repress all\mbox{}\newline 
\hspace*{6pt}\hspace*{6pt}\hspace*{6pt}\hspace*{6pt} exercise of the reasoning faculty.{</\textbf{p}>}\mbox{}\newline 
\hspace*{6pt}\hspace*{6pt}{<\textbf{div}\hspace*{6pt}{n}="{1}"\hspace*{6pt}{type}="{chapter}">}\mbox{}\newline 
\hspace*{6pt}\hspace*{6pt}\hspace*{6pt}{<\textbf{head}>}The Nature of Authority{</\textbf{head}>}\mbox{}\newline 
\hspace*{6pt}\hspace*{6pt}\hspace*{6pt}{<\textbf{p}>}With reference to any proposed measures having for their object the greatest\mbox{}\newline 
\hspace*{6pt}\hspace*{6pt}\hspace*{6pt}\hspace*{6pt}\hspace*{6pt}\hspace*{6pt} happiness of the greatest number [...]{</\textbf{p}>}\mbox{}\newline 
\hspace*{6pt}\hspace*{6pt}\hspace*{6pt}{<\textbf{div}\hspace*{6pt}{n}="{1.1}"\hspace*{6pt}{type}="{section}">}\mbox{}\newline 
\hspace*{6pt}\hspace*{6pt}\hspace*{6pt}\hspace*{6pt}{<\textbf{head}>}Analysis of Authority{</\textbf{head}>}\mbox{}\newline 
\hspace*{6pt}\hspace*{6pt}\hspace*{6pt}\hspace*{6pt}{<\textbf{p}>}What on any given occasion is the legitimate weight or influence to be attached to\mbox{}\newline 
\hspace*{6pt}\hspace*{6pt}\hspace*{6pt}\hspace*{6pt}\hspace*{6pt}\hspace*{6pt}\hspace*{6pt}\hspace*{6pt} authority [...] {</\textbf{p}>}\mbox{}\newline 
\hspace*{6pt}\hspace*{6pt}\hspace*{6pt}{</\textbf{div}>}\mbox{}\newline 
\hspace*{6pt}\hspace*{6pt}\hspace*{6pt}{<\textbf{div}\hspace*{6pt}{n}="{1.2}"\hspace*{6pt}{type}="{section}">}\mbox{}\newline 
\hspace*{6pt}\hspace*{6pt}\hspace*{6pt}\hspace*{6pt}{<\textbf{head}>}Appeal to Authority, in What Cases Fallacious.{</\textbf{head}>}\mbox{}\newline 
\hspace*{6pt}\hspace*{6pt}\hspace*{6pt}\hspace*{6pt}{<\textbf{p}>}Reference to authority is open to the charge of fallacy when [...] {</\textbf{p}>}\mbox{}\newline 
\hspace*{6pt}\hspace*{6pt}\hspace*{6pt}{</\textbf{div}>}\mbox{}\newline 
\hspace*{6pt}\hspace*{6pt}{</\textbf{div}>}\mbox{}\newline 
\hspace*{6pt}{</\textbf{div}>}\mbox{}\newline 
{</\textbf{body}>}\end{shaded}\egroup 


    \item[{Schematron}]
   <s:report test="ancestor::tei:l"> Abstract model violation: Lines may not contain higher-level structural elements such as div. </s:report>
    \item[{Schematron}]
   <s:report test="ancestor::tei:p or ancestor::tei:ab and not(ancestor::tei:floatingText)"> Abstract model violation: p and ab may not contain higher-level structural elements such as div. </s:report>
    \item[{Content model}]
  \mbox{}\hfill\\[-10pt]\begin{Verbatim}[fontsize=\small]
<content>
 <sequence>
  <alternate maxOccurs="unbounded"
   minOccurs="0">
   <classRef key="model.divTop"/>
   <classRef key="model.global"/>
  </alternate>
  <sequence minOccurs="0">
   <alternate>
    <sequence maxOccurs="unbounded"
     minOccurs="1">
     <alternate>
      <classRef key="model.divLike"/>
      <classRef key="model.divGenLike"/>
     </alternate>
     <classRef key="model.global"
      maxOccurs="unbounded" minOccurs="0"/>
    </sequence>
    <sequence>
     <sequence maxOccurs="unbounded"
      minOccurs="1">
      <classRef key="model.common"/>
      <classRef key="model.global"
       maxOccurs="unbounded" minOccurs="0"/>
     </sequence>
     <sequence maxOccurs="unbounded"
      minOccurs="0">
      <alternate>
       <classRef key="model.divLike"/>
       <classRef key="model.divGenLike"/>
      </alternate>
      <classRef key="model.global"
       maxOccurs="unbounded" minOccurs="0"/>
     </sequence>
    </sequence>
   </alternate>
   <sequence maxOccurs="unbounded"
    minOccurs="0">
    <classRef key="model.divBottom"/>
    <classRef key="model.global"
     maxOccurs="unbounded" minOccurs="0"/>
   </sequence>
  </sequence>
 </sequence>
</content>
    
\end{Verbatim}

    \item[{Schema Declaration}]
  \mbox{}\hfill\\[-10pt]\begin{Verbatim}[fontsize=\small]
element div
{
   att.global.attributes,
   att.divLike.attributes,
   att.typed.attributes,
   att.declaring.attributes,
   att.written.attributes,
   (
      ( model.divTop | model.global )*,
      (
         (
            ( ( model.divLike | model.divGenLike ), model.global* )+
          | (
               ( model.common, model.global* )+,
               ( ( model.divLike | model.divGenLike ), model.global* )*
            )
         ),
         ( model.divBottom, model.global* )*
      )?
   )
}
\end{Verbatim}

\end{reflist}  \index{divGen=<divGen>|oddindex}\index{type=@type!<divGen>|oddindex}
\begin{reflist}
\item[]\begin{specHead}{TEI.divGen}{<divGen> }(automatically generated text division) indicates the location at which a textual division generated automatically by a text-processing application is to appear. [\xref{http://www.tei-c.org/release/doc/tei-p5-doc/en/html/CO.html\#CONOIX}{3.8.2. Index Entries}]\end{specHead} 
    \item[{Module}]
  core
    \item[{Attributes}]
  Attributes att.global (\textit{@xml:id}, \textit{@n}, \textit{@xml:lang}, \textit{@xml:base}, \textit{@xml:space})  (att.global.rendition (\textit{@rend}, \textit{@style}, \textit{@rendition})) (att.global.linking (\textit{@corresp}, \textit{@synch}, \textit{@sameAs}, \textit{@copyOf}, \textit{@next}, \textit{@prev}, \textit{@exclude}, \textit{@select})) (att.global.analytic (\textit{@ana})) (att.global.facs (\textit{@facs})) (att.global.change (\textit{@change})) (att.global.responsibility (\textit{@cert}, \textit{@resp})) (att.global.source (\textit{@source})) \hfil\\[-10pt]\begin{sansreflist}
    \item[@type]
  specifies what type of generated text division (e.g. index, table of contents, etc.) is to appear.
\begin{reflist}
    \item[{Status}]
  Optional
    \item[{Datatype}]
  teidata.enumerated
    \item[{Sample values include:}]
  \begin{description}

\item[{index}]an index is to be generated and inserted at this point.
\item[{toc}]a table of contents
\item[{figlist}]a list of figures
\item[{tablist}]a list of tables
\end{description} 
    \item[{Note}]
  \par
Valid values are application-dependent; those shown are of obvious utility in document production, but are by no means exhaustive.
\end{reflist}  
\end{sansreflist}  
    \item[{Member of}]
  model.divGenLike model.frontPart
    \item[{Contained by}]
  
    \item[textstructure: ]
   back body div front
    \item[{May contain}]
  
    \item[core: ]
   head
    \item[{Note}]
  \par
This element is intended primarily for use in document production or manipulation, rather than in the transcription of pre-existing materials; it makes it easier to specify the location of indices, tables of contents, etc., to be generated by text preparation or word processing software. 
    \item[{Example}]
  One use for this element is to allow document preparation software to generate an index and insert it in the appropriate place in the output. The example below assumes that the {\itshape indexName} attribute on <index> elements in the text has been used to specify index entries for the two generated indexes, named NAMES and THINGS:\leavevmode\bgroup\exampleFont \begin{shaded}\noindent\mbox{}{<\textbf{back}>}\mbox{}\newline 
\hspace*{6pt}{<\textbf{div1}\hspace*{6pt}{type}="{backmat}">}\mbox{}\newline 
\hspace*{6pt}\hspace*{6pt}{<\textbf{head}>}Bibliography{</\textbf{head}>}\mbox{}\newline 
\textit{<!-- ... -->}\mbox{}\newline 
\hspace*{6pt}{</\textbf{div1}>}\mbox{}\newline 
\hspace*{6pt}{<\textbf{div1}\hspace*{6pt}{type}="{backmat}">}\mbox{}\newline 
\hspace*{6pt}\hspace*{6pt}{<\textbf{head}>}Indices{</\textbf{head}>}\mbox{}\newline 
\hspace*{6pt}\hspace*{6pt}{<\textbf{divGen}\hspace*{6pt}{n}="{Index Nominum}"\hspace*{6pt}{type}="{NAMES}"/>}\mbox{}\newline 
\hspace*{6pt}\hspace*{6pt}{<\textbf{divGen}\hspace*{6pt}{n}="{Index Rerum}"\hspace*{6pt}{type}="{THINGS}"/>}\mbox{}\newline 
\hspace*{6pt}{</\textbf{div1}>}\mbox{}\newline 
{</\textbf{back}>}\end{shaded}\egroup 


    \item[{Example}]
  Another use for <divGen> is to specify the location of an automatically produced table of contents:\leavevmode\bgroup\exampleFont \begin{shaded}\noindent\mbox{}{<\textbf{front}>}\mbox{}\newline 
\textit{<!--<titlePage>...</titlePage>-->}\mbox{}\newline 
\hspace*{6pt}{<\textbf{divGen}\hspace*{6pt}{type}="{toc}"/>}\mbox{}\newline 
\hspace*{6pt}{<\textbf{div}>}\mbox{}\newline 
\hspace*{6pt}\hspace*{6pt}{<\textbf{head}>}Preface{</\textbf{head}>}\mbox{}\newline 
\hspace*{6pt}\hspace*{6pt}{<\textbf{p}>} ... {</\textbf{p}>}\mbox{}\newline 
\hspace*{6pt}{</\textbf{div}>}\mbox{}\newline 
{</\textbf{front}>}\end{shaded}\egroup 


    \item[{Content model}]
  \mbox{}\hfill\\[-10pt]\begin{Verbatim}[fontsize=\small]
<content>
 <classRef key="model.headLike"
  maxOccurs="unbounded" minOccurs="0"/>
</content>
    
\end{Verbatim}

    \item[{Schema Declaration}]
  \mbox{}\hfill\\[-10pt]\begin{Verbatim}[fontsize=\small]
element divGen
{
   att.global.attributes,
   attribute type { text }?,
   model.headLike*
}
\end{Verbatim}

\end{reflist}  \index{docAuthor=<docAuthor>|oddindex}
\begin{reflist}
\item[]\begin{specHead}{TEI.docAuthor}{<docAuthor> }(document author) contains the name of the author of the document, as given on the title page (often but not always contained in a byline). [\xref{http://www.tei-c.org/release/doc/tei-p5-doc/en/html/DS.html\#DSTITL}{4.6. Title Pages}]\end{specHead} 
    \item[{Module}]
  textstructure
    \item[{Attributes}]
  Attributes att.global (\textit{@xml:id}, \textit{@n}, \textit{@xml:lang}, \textit{@xml:base}, \textit{@xml:space})  (att.global.rendition (\textit{@rend}, \textit{@style}, \textit{@rendition})) (att.global.linking (\textit{@corresp}, \textit{@synch}, \textit{@sameAs}, \textit{@copyOf}, \textit{@next}, \textit{@prev}, \textit{@exclude}, \textit{@select})) (att.global.analytic (\textit{@ana})) (att.global.facs (\textit{@facs})) (att.global.change (\textit{@change})) (att.global.responsibility (\textit{@cert}, \textit{@resp})) (att.global.source (\textit{@source})) att.canonical (\textit{@key}, \textit{@ref}) 
    \item[{Member of}]
  model.divWrapper model.pLike.front model.titlepagePart
    \item[{Contained by}]
  
    \item[core: ]
   lg list\par 
    \item[figures: ]
   figure table\par 
    \item[msdescription: ]
   msItem\par 
    \item[textstructure: ]
   back body byline div front group titlePage
    \item[{May contain}]
  
    \item[analysis: ]
   c cl interp interpGrp m pc phr s span spanGrp w\par 
    \item[core: ]
   abbr add address cb choice corr date del distinct email emph expan foreign gap gb gloss graphic hi index lb measure measureGrp media mentioned milestone name note num orig pb ptr ref reg rs sic soCalled term time title unclear\par 
    \item[figures: ]
   figure formula notatedMusic\par 
    \item[gaiji: ]
   g\par 
    \item[header: ]
   idno\par 
    \item[linking: ]
   alt altGrp anchor join joinGrp link linkGrp seg timeline\par 
    \item[msdescription: ]
   catchwords depth dim dimensions height heraldry locus locusGrp material objectType origDate origPlace secFol signatures stamp watermark width\par 
    \item[namesdates: ]
   addName affiliation bloc climate country district forename genName geo geogFeat geogName location nameLink offset orgName persName placeName population region roleName settlement state surname terrain trait\par 
    \item[textcrit: ]
   app witDetail\par 
    \item[transcr: ]
   addSpan am damage damageSpan delSpan ex fw handShift listTranspose metamark mod redo restore retrace secl space subst substJoin supplied surplus undo\par character data
    \item[{Note}]
  \par
The document author's name often occurs within a byline, but the <docAuthor> element may be used whether the <byline> element is used or not. It should be used only for the author(s) of the entire document, not for author(s) of any subset or part of it. (Attributions of authorship of a subset or part of the document, for example of a chapter in a textbook or an article in a newspaper, may be encoded with <byline> without <docAuthor>.)
    \item[{Example}]
  \leavevmode\bgroup\exampleFont \begin{shaded}\noindent\mbox{}{<\textbf{titlePage}>}\mbox{}\newline 
\hspace*{6pt}{<\textbf{docTitle}>}\mbox{}\newline 
\hspace*{6pt}\hspace*{6pt}{<\textbf{titlePart}>}Travels into Several Remote Nations of the World, in Four\mbox{}\newline 
\hspace*{6pt}\hspace*{6pt}\hspace*{6pt}\hspace*{6pt} Parts.{</\textbf{titlePart}>}\mbox{}\newline 
\hspace*{6pt}{</\textbf{docTitle}>}\mbox{}\newline 
\hspace*{6pt}{<\textbf{byline}>} By {<\textbf{docAuthor}>}Lemuel Gulliver{</\textbf{docAuthor}>}, First a Surgeon,\mbox{}\newline 
\hspace*{6pt}\hspace*{6pt} and then a Captain of several Ships{</\textbf{byline}>}\mbox{}\newline 
{</\textbf{titlePage}>}\end{shaded}\egroup 


    \item[{Content model}]
  \mbox{}\hfill\\[-10pt]\begin{Verbatim}[fontsize=\small]
<content>
 <macroRef key="macro.phraseSeq"/>
</content>
    
\end{Verbatim}

    \item[{Schema Declaration}]
  \mbox{}\hfill\\[-10pt]\begin{Verbatim}[fontsize=\small]
element docAuthor
{
   att.global.attributes,
   att.canonical.attributes,
   macro.phraseSeq}
\end{Verbatim}

\end{reflist}  \index{docDate=<docDate>|oddindex}\index{when=@when!<docDate>|oddindex}
\begin{reflist}
\item[]\begin{specHead}{TEI.docDate}{<docDate> }(document date) contains the date of a document, as given on a title page or in a dateline. [\xref{http://www.tei-c.org/release/doc/tei-p5-doc/en/html/DS.html\#DSTITL}{4.6. Title Pages}]\end{specHead} 
    \item[{Module}]
  textstructure
    \item[{Attributes}]
  Attributes att.global (\textit{@xml:id}, \textit{@n}, \textit{@xml:lang}, \textit{@xml:base}, \textit{@xml:space})  (att.global.rendition (\textit{@rend}, \textit{@style}, \textit{@rendition})) (att.global.linking (\textit{@corresp}, \textit{@synch}, \textit{@sameAs}, \textit{@copyOf}, \textit{@next}, \textit{@prev}, \textit{@exclude}, \textit{@select})) (att.global.analytic (\textit{@ana})) (att.global.facs (\textit{@facs})) (att.global.change (\textit{@change})) (att.global.responsibility (\textit{@cert}, \textit{@resp})) (att.global.source (\textit{@source})) \hfil\\[-10pt]\begin{sansreflist}
    \item[@when]
  gives the value of the date in standard form, i.e. YYYY-MM-DD.
\begin{reflist}
    \item[{Status}]
  Optional
    \item[{Datatype}]
  teidata.temporal.w3c
    \item[{Note}]
  \par
For simple dates, the {\itshape when} attribute should give the Gregorian or proleptic Gregorian date in one of the formats specified in \textit{XML Schema Part 2: Datatypes Second Edition}.
\end{reflist}  
\end{sansreflist}  
    \item[{Member of}]
  model.divWrapper model.pLike.front model.titlepagePart
    \item[{Contained by}]
  
    \item[core: ]
   lg list\par 
    \item[figures: ]
   figure table\par 
    \item[msdescription: ]
   msItem\par 
    \item[textstructure: ]
   back body dateline div docImprint front group titlePage
    \item[{May contain}]
  
    \item[analysis: ]
   c cl interp interpGrp m pc phr s span spanGrp w\par 
    \item[core: ]
   abbr add address cb choice corr date del distinct email emph expan foreign gap gb gloss graphic hi index lb measure measureGrp media mentioned milestone name note num orig pb ptr ref reg rs sic soCalled term time title unclear\par 
    \item[figures: ]
   figure formula notatedMusic\par 
    \item[gaiji: ]
   g\par 
    \item[header: ]
   idno\par 
    \item[linking: ]
   alt altGrp anchor join joinGrp link linkGrp seg timeline\par 
    \item[msdescription: ]
   catchwords depth dim dimensions height heraldry locus locusGrp material objectType origDate origPlace secFol signatures stamp watermark width\par 
    \item[namesdates: ]
   addName affiliation bloc climate country district forename genName geo geogFeat geogName location nameLink offset orgName persName placeName population region roleName settlement state surname terrain trait\par 
    \item[textcrit: ]
   app witDetail\par 
    \item[transcr: ]
   addSpan am damage damageSpan delSpan ex fw handShift listTranspose metamark mod redo restore retrace secl space subst substJoin supplied surplus undo\par character data
    \item[{Note}]
  \par
Cf. the general <date> element in the core tag set. This specialized element is provided for convenience in marking and processing the date of the documents, since it is likely to require specialized handling for many applications. It should be used only for the date of the entire document, not for any subset or part of it.
    \item[{Example}]
  \leavevmode\bgroup\exampleFont \begin{shaded}\noindent\mbox{}{<\textbf{docImprint}>}Oxford, Clarendon Press, {<\textbf{docDate}>}1987{</\textbf{docDate}>}\mbox{}\newline 
{</\textbf{docImprint}>}\end{shaded}\egroup 


    \item[{Content model}]
  \mbox{}\hfill\\[-10pt]\begin{Verbatim}[fontsize=\small]
<content>
 <macroRef key="macro.phraseSeq"/>
</content>
    
\end{Verbatim}

    \item[{Schema Declaration}]
  \mbox{}\hfill\\[-10pt]\begin{Verbatim}[fontsize=\small]
element docDate
{
   att.global.attributes,
   attribute when { text }?,
   macro.phraseSeq}
\end{Verbatim}

\end{reflist}  \index{docEdition=<docEdition>|oddindex}
\begin{reflist}
\item[]\begin{specHead}{TEI.docEdition}{<docEdition> }(document edition) contains an edition statement as presented on a title page of a document. [\xref{http://www.tei-c.org/release/doc/tei-p5-doc/en/html/DS.html\#DSTITL}{4.6. Title Pages}]\end{specHead} 
    \item[{Module}]
  textstructure
    \item[{Attributes}]
  Attributes att.global (\textit{@xml:id}, \textit{@n}, \textit{@xml:lang}, \textit{@xml:base}, \textit{@xml:space})  (att.global.rendition (\textit{@rend}, \textit{@style}, \textit{@rendition})) (att.global.linking (\textit{@corresp}, \textit{@synch}, \textit{@sameAs}, \textit{@copyOf}, \textit{@next}, \textit{@prev}, \textit{@exclude}, \textit{@select})) (att.global.analytic (\textit{@ana})) (att.global.facs (\textit{@facs})) (att.global.change (\textit{@change})) (att.global.responsibility (\textit{@cert}, \textit{@resp})) (att.global.source (\textit{@source}))
    \item[{Member of}]
  model.pLike.front model.titlepagePart
    \item[{Contained by}]
  
    \item[msdescription: ]
   msItem\par 
    \item[textstructure: ]
   back front titlePage
    \item[{May contain}]
  
    \item[analysis: ]
   c cl interp interpGrp m pc phr s span spanGrp w\par 
    \item[core: ]
   abbr add address bibl biblStruct cb choice cit corr date del desc distinct email emph expan foreign gap gb gloss graphic hi index l label lb lg list listBibl measure measureGrp media mentioned milestone name note num orig pb ptr q quote ref reg rs said sic soCalled stage term time title unclear\par 
    \item[figures: ]
   figure formula notatedMusic table\par 
    \item[gaiji: ]
   g\par 
    \item[header: ]
   biblFull idno\par 
    \item[linking: ]
   alt altGrp anchor join joinGrp link linkGrp seg timeline\par 
    \item[msdescription: ]
   catchwords depth dim dimensions height heraldry locus locusGrp material msDesc objectType origDate origPlace secFol signatures stamp watermark width\par 
    \item[namesdates: ]
   addName affiliation bloc climate country district forename genName geo geogFeat geogName listEvent listNym listOrg listPerson listPlace location nameLink offset orgName persName placeName population region roleName settlement state surname terrain trait\par 
    \item[textcrit: ]
   app listApp listWit witDetail\par 
    \item[textstructure: ]
   floatingText\par 
    \item[transcr: ]
   addSpan am damage damageSpan delSpan ex fw handShift listTranspose metamark mod redo restore retrace secl space subst substJoin supplied surplus undo\par character data
    \item[{Note}]
  \par
Cf. the <edition> element of bibliographic citation. As usual, the shorter name has been given to the more frequent element.
    \item[{Example}]
  \leavevmode\bgroup\exampleFont \begin{shaded}\noindent\mbox{}{<\textbf{docEdition}>}The Third edition Corrected{</\textbf{docEdition}>}\end{shaded}\egroup 


    \item[{Content model}]
  \mbox{}\hfill\\[-10pt]\begin{Verbatim}[fontsize=\small]
<content>
 <macroRef key="macro.paraContent"/>
</content>
    
\end{Verbatim}

    \item[{Schema Declaration}]
  \mbox{}\hfill\\[-10pt]\begin{Verbatim}[fontsize=\small]
element docEdition { att.global.attributes, macro.paraContent }
\end{Verbatim}

\end{reflist}  \index{docImprint=<docImprint>|oddindex}
\begin{reflist}
\item[]\begin{specHead}{TEI.docImprint}{<docImprint> }(document imprint) contains the imprint statement (place and date of publication, publisher name), as given (usually) at the foot of a title page. [\xref{http://www.tei-c.org/release/doc/tei-p5-doc/en/html/DS.html\#DSTITL}{4.6. Title Pages}]\end{specHead} 
    \item[{Module}]
  textstructure
    \item[{Attributes}]
  Attributes att.global (\textit{@xml:id}, \textit{@n}, \textit{@xml:lang}, \textit{@xml:base}, \textit{@xml:space})  (att.global.rendition (\textit{@rend}, \textit{@style}, \textit{@rendition})) (att.global.linking (\textit{@corresp}, \textit{@synch}, \textit{@sameAs}, \textit{@copyOf}, \textit{@next}, \textit{@prev}, \textit{@exclude}, \textit{@select})) (att.global.analytic (\textit{@ana})) (att.global.facs (\textit{@facs})) (att.global.change (\textit{@change})) (att.global.responsibility (\textit{@cert}, \textit{@resp})) (att.global.source (\textit{@source}))
    \item[{Member of}]
  model.pLike.front model.titlepagePart
    \item[{Contained by}]
  
    \item[msdescription: ]
   msItem\par 
    \item[textstructure: ]
   back front titlePage
    \item[{May contain}]
  
    \item[analysis: ]
   c cl interp interpGrp m pc phr s span spanGrp w\par 
    \item[core: ]
   abbr add address cb choice corr date del distinct email emph expan foreign gap gb gloss graphic hi index lb measure measureGrp media mentioned milestone name note num orig pb ptr pubPlace publisher ref reg rs sic soCalled term time title unclear\par 
    \item[figures: ]
   figure formula notatedMusic\par 
    \item[gaiji: ]
   g\par 
    \item[header: ]
   idno\par 
    \item[linking: ]
   alt altGrp anchor join joinGrp link linkGrp seg timeline\par 
    \item[msdescription: ]
   catchwords depth dim dimensions height heraldry locus locusGrp material objectType origDate origPlace secFol signatures stamp watermark width\par 
    \item[namesdates: ]
   addName affiliation bloc climate country district forename genName geo geogFeat geogName location nameLink offset orgName persName placeName population region roleName settlement state surname terrain trait\par 
    \item[textcrit: ]
   app witDetail\par 
    \item[textstructure: ]
   docDate\par 
    \item[transcr: ]
   addSpan am damage damageSpan delSpan ex fw handShift listTranspose metamark mod redo restore retrace secl space subst substJoin supplied surplus undo\par character data
    \item[{Note}]
  \par
Cf. the <imprint> element of bibliographic citations. As with title, author, and editions, the shorter name is reserved for the element likely to be used more often.
    \item[{Example}]
  \leavevmode\bgroup\exampleFont \begin{shaded}\noindent\mbox{}{<\textbf{docImprint}>}Oxford, Clarendon Press, 1987{</\textbf{docImprint}>}\end{shaded}\egroup 

Imprints may be somewhat more complex: \leavevmode\bgroup\exampleFont \begin{shaded}\noindent\mbox{}{<\textbf{docImprint}>}\mbox{}\newline 
\hspace*{6pt}{<\textbf{pubPlace}>}London{</\textbf{pubPlace}>}\mbox{}\newline 
 Printed for {<\textbf{name}>}E. Nutt{</\textbf{name}>},\mbox{}\newline 
 at\mbox{}\newline 
{<\textbf{pubPlace}>}Royal Exchange{</\textbf{pubPlace}>};\mbox{}\newline 
{<\textbf{name}>}J. Roberts{</\textbf{name}>} in\mbox{}\newline 
{<\textbf{pubPlace}>}wick-Lane{</\textbf{pubPlace}>};\mbox{}\newline 
{<\textbf{name}>}A. Dodd{</\textbf{name}>} without\mbox{}\newline 
{<\textbf{pubPlace}>}Temple-Bar{</\textbf{pubPlace}>};\mbox{}\newline 
 and {<\textbf{name}>}J. Graves{</\textbf{name}>} in\mbox{}\newline 
{<\textbf{pubPlace}>}St. James's-street.{</\textbf{pubPlace}>}\mbox{}\newline 
\hspace*{6pt}{<\textbf{date}>}1722.{</\textbf{date}>}\mbox{}\newline 
{</\textbf{docImprint}>}\end{shaded}\egroup 


    \item[{Content model}]
  \mbox{}\hfill\\[-10pt]\begin{Verbatim}[fontsize=\small]
<content>
 <alternate maxOccurs="unbounded"
  minOccurs="0">
  <textNode/>
  <classRef key="model.gLike"/>
  <classRef key="model.phrase"/>
  <elementRef key="pubPlace"/>
  <elementRef key="docDate"/>
  <elementRef key="publisher"/>
  <classRef key="model.global"/>
 </alternate>
</content>
    
\end{Verbatim}

    \item[{Schema Declaration}]
  \mbox{}\hfill\\[-10pt]\begin{Verbatim}[fontsize=\small]
element docImprint
{
   att.global.attributes,
   (
      text
    | model.gLike    | model.phrase    | pubPlace    | docDate    | publisher    | model.global   )*
}
\end{Verbatim}

\end{reflist}  \index{docTitle=<docTitle>|oddindex}
\begin{reflist}
\item[]\begin{specHead}{TEI.docTitle}{<docTitle> }(document title) contains the title of a document, including all its constituents, as given on a title page. [\xref{http://www.tei-c.org/release/doc/tei-p5-doc/en/html/DS.html\#DSTITL}{4.6. Title Pages}]\end{specHead} 
    \item[{Module}]
  textstructure
    \item[{Attributes}]
  Attributes att.global (\textit{@xml:id}, \textit{@n}, \textit{@xml:lang}, \textit{@xml:base}, \textit{@xml:space})  (att.global.rendition (\textit{@rend}, \textit{@style}, \textit{@rendition})) (att.global.linking (\textit{@corresp}, \textit{@synch}, \textit{@sameAs}, \textit{@copyOf}, \textit{@next}, \textit{@prev}, \textit{@exclude}, \textit{@select})) (att.global.analytic (\textit{@ana})) (att.global.facs (\textit{@facs})) (att.global.change (\textit{@change})) (att.global.responsibility (\textit{@cert}, \textit{@resp})) (att.global.source (\textit{@source})) att.canonical (\textit{@key}, \textit{@ref}) 
    \item[{Member of}]
  model.pLike.front model.titlepagePart
    \item[{Contained by}]
  
    \item[msdescription: ]
   msItem\par 
    \item[textstructure: ]
   back front titlePage
    \item[{May contain}]
  
    \item[analysis: ]
   interp interpGrp span spanGrp\par 
    \item[core: ]
   cb gap gb index lb milestone note pb\par 
    \item[figures: ]
   figure notatedMusic\par 
    \item[linking: ]
   alt altGrp anchor join joinGrp link linkGrp timeline\par 
    \item[textcrit: ]
   app witDetail\par 
    \item[textstructure: ]
   titlePart\par 
    \item[transcr: ]
   addSpan damageSpan delSpan fw listTranspose metamark space substJoin
    \item[{Example}]
  \leavevmode\bgroup\exampleFont \begin{shaded}\noindent\mbox{}{<\textbf{docTitle}>}\mbox{}\newline 
\hspace*{6pt}{<\textbf{titlePart}\hspace*{6pt}{type}="{main}">}The DUNCIAD, VARIOURVM.{</\textbf{titlePart}>}\mbox{}\newline 
\hspace*{6pt}{<\textbf{titlePart}\hspace*{6pt}{type}="{sub}">}WITH THE PROLEGOMENA of SCRIBLERUS.{</\textbf{titlePart}>}\mbox{}\newline 
{</\textbf{docTitle}>}\end{shaded}\egroup 


    \item[{Content model}]
  \mbox{}\hfill\\[-10pt]\begin{Verbatim}[fontsize=\small]
<content>
 <sequence>
  <classRef key="model.global"
   maxOccurs="unbounded" minOccurs="0"/>
  <sequence maxOccurs="unbounded"
   minOccurs="1">
   <elementRef key="titlePart"/>
   <classRef key="model.global"
    maxOccurs="unbounded" minOccurs="0"/>
  </sequence>
 </sequence>
</content>
    
\end{Verbatim}

    \item[{Schema Declaration}]
  \mbox{}\hfill\\[-10pt]\begin{Verbatim}[fontsize=\small]
element docTitle
{
   att.global.attributes,
   att.canonical.attributes,
   ( model.global*, ( titlePart, model.global* )+ )
}
\end{Verbatim}

\end{reflist}  \index{edition=<edition>|oddindex}
\begin{reflist}
\item[]\begin{specHead}{TEI.edition}{<edition> }describes the particularities of one edition of a text. [\xref{http://www.tei-c.org/release/doc/tei-p5-doc/en/html/HD.html\#HD22}{2.2.2. The Edition Statement}]\end{specHead} 
    \item[{Module}]
  header
    \item[{Attributes}]
  Attributes att.global (\textit{@xml:id}, \textit{@n}, \textit{@xml:lang}, \textit{@xml:base}, \textit{@xml:space})  (att.global.rendition (\textit{@rend}, \textit{@style}, \textit{@rendition})) (att.global.linking (\textit{@corresp}, \textit{@synch}, \textit{@sameAs}, \textit{@copyOf}, \textit{@next}, \textit{@prev}, \textit{@exclude}, \textit{@select})) (att.global.analytic (\textit{@ana})) (att.global.facs (\textit{@facs})) (att.global.change (\textit{@change})) (att.global.responsibility (\textit{@cert}, \textit{@resp})) (att.global.source (\textit{@source}))
    \item[{Member of}]
  model.biblPart 
    \item[{Contained by}]
  
    \item[core: ]
   bibl monogr\par 
    \item[header: ]
   editionStmt
    \item[{May contain}]
  
    \item[analysis: ]
   c cl interp interpGrp m pc phr s span spanGrp w\par 
    \item[core: ]
   abbr add address cb choice corr date del distinct email emph expan foreign gap gb gloss graphic hi index lb measure measureGrp media mentioned milestone name note num orig pb ptr ref reg rs sic soCalled term time title unclear\par 
    \item[figures: ]
   figure formula notatedMusic\par 
    \item[gaiji: ]
   g\par 
    \item[header: ]
   idno\par 
    \item[linking: ]
   alt altGrp anchor join joinGrp link linkGrp seg timeline\par 
    \item[msdescription: ]
   catchwords depth dim dimensions height heraldry locus locusGrp material objectType origDate origPlace secFol signatures stamp watermark width\par 
    \item[namesdates: ]
   addName affiliation bloc climate country district forename genName geo geogFeat geogName location nameLink offset orgName persName placeName population region roleName settlement state surname terrain trait\par 
    \item[textcrit: ]
   app witDetail\par 
    \item[transcr: ]
   addSpan am damage damageSpan delSpan ex fw handShift listTranspose metamark mod redo restore retrace secl space subst substJoin supplied surplus undo\par character data
    \item[{Example}]
  \leavevmode\bgroup\exampleFont \begin{shaded}\noindent\mbox{}{<\textbf{edition}>}First edition {<\textbf{date}>}Oct 1990{</\textbf{date}>}\mbox{}\newline 
{</\textbf{edition}>}\mbox{}\newline 
{<\textbf{edition}\hspace*{6pt}{n}="{S2}">}Students' edition{</\textbf{edition}>}\end{shaded}\egroup 


    \item[{Content model}]
  \mbox{}\hfill\\[-10pt]\begin{Verbatim}[fontsize=\small]
<content>
 <macroRef key="macro.phraseSeq"/>
</content>
    
\end{Verbatim}

    \item[{Schema Declaration}]
  \mbox{}\hfill\\[-10pt]\begin{Verbatim}[fontsize=\small]
element edition { att.global.attributes, macro.phraseSeq }
\end{Verbatim}

\end{reflist}  \index{editionStmt=<editionStmt>|oddindex}
\begin{reflist}
\item[]\begin{specHead}{TEI.editionStmt}{<editionStmt> }(edition statement) groups information relating to one edition of a text. [\xref{http://www.tei-c.org/release/doc/tei-p5-doc/en/html/HD.html\#HD22}{2.2.2. The Edition Statement} \xref{http://www.tei-c.org/release/doc/tei-p5-doc/en/html/HD.html\#HD2}{2.2. The File Description}]\end{specHead} 
    \item[{Module}]
  header
    \item[{Attributes}]
  Attributes att.global (\textit{@xml:id}, \textit{@n}, \textit{@xml:lang}, \textit{@xml:base}, \textit{@xml:space})  (att.global.rendition (\textit{@rend}, \textit{@style}, \textit{@rendition})) (att.global.linking (\textit{@corresp}, \textit{@synch}, \textit{@sameAs}, \textit{@copyOf}, \textit{@next}, \textit{@prev}, \textit{@exclude}, \textit{@select})) (att.global.analytic (\textit{@ana})) (att.global.facs (\textit{@facs})) (att.global.change (\textit{@change})) (att.global.responsibility (\textit{@cert}, \textit{@resp})) (att.global.source (\textit{@source}))
    \item[{Contained by}]
  
    \item[header: ]
   biblFull fileDesc
    \item[{May contain}]
  
    \item[core: ]
   author editor meeting p respStmt\par 
    \item[header: ]
   edition funder principal sponsor\par 
    \item[linking: ]
   ab
    \item[{Example}]
  \leavevmode\bgroup\exampleFont \begin{shaded}\noindent\mbox{}{<\textbf{editionStmt}>}\mbox{}\newline 
\hspace*{6pt}{<\textbf{edition}\hspace*{6pt}{n}="{S2}">}Students' edition{</\textbf{edition}>}\mbox{}\newline 
\hspace*{6pt}{<\textbf{respStmt}>}\mbox{}\newline 
\hspace*{6pt}\hspace*{6pt}{<\textbf{resp}>}Adapted by {</\textbf{resp}>}\mbox{}\newline 
\hspace*{6pt}\hspace*{6pt}{<\textbf{name}>}Elizabeth Kirk{</\textbf{name}>}\mbox{}\newline 
\hspace*{6pt}{</\textbf{respStmt}>}\mbox{}\newline 
{</\textbf{editionStmt}>}\end{shaded}\egroup 


    \item[{Example}]
  \leavevmode\bgroup\exampleFont \begin{shaded}\noindent\mbox{}{<\textbf{editionStmt}>}\mbox{}\newline 
\hspace*{6pt}{<\textbf{p}>}First edition, {<\textbf{date}>}Michaelmas Term, 1991.{</\textbf{date}>}\mbox{}\newline 
\hspace*{6pt}{</\textbf{p}>}\mbox{}\newline 
{</\textbf{editionStmt}>}\end{shaded}\egroup 


    \item[{Content model}]
  \mbox{}\hfill\\[-10pt]\begin{Verbatim}[fontsize=\small]
<content>
 <alternate>
  <classRef key="model.pLike"
   maxOccurs="unbounded" minOccurs="1"/>
  <sequence>
   <elementRef key="edition"/>
   <classRef key="model.respLike"
    maxOccurs="unbounded" minOccurs="0"/>
  </sequence>
 </alternate>
</content>
    
\end{Verbatim}

    \item[{Schema Declaration}]
  \mbox{}\hfill\\[-10pt]\begin{Verbatim}[fontsize=\small]
element editionStmt
{
   att.global.attributes,
   ( model.pLike+ | ( edition, model.respLike* ) )
}
\end{Verbatim}

\end{reflist}  \index{editor=<editor>|oddindex}
\begin{reflist}
\item[]\begin{specHead}{TEI.editor}{<editor> }contains a secondary statement of responsibility for a bibliographic item, for example the name of an individual, institution or organization, (or of several such) acting as editor, compiler, translator, etc. [\xref{http://www.tei-c.org/release/doc/tei-p5-doc/en/html/CO.html\#COBICOR}{3.11.2.2. Titles, Authors, and Editors}]\end{specHead} 
    \item[{Module}]
  core
    \item[{Attributes}]
  Attributes att.global (\textit{@xml:id}, \textit{@n}, \textit{@xml:lang}, \textit{@xml:base}, \textit{@xml:space})  (att.global.rendition (\textit{@rend}, \textit{@style}, \textit{@rendition})) (att.global.linking (\textit{@corresp}, \textit{@synch}, \textit{@sameAs}, \textit{@copyOf}, \textit{@next}, \textit{@prev}, \textit{@exclude}, \textit{@select})) (att.global.analytic (\textit{@ana})) (att.global.facs (\textit{@facs})) (att.global.change (\textit{@change})) (att.global.responsibility (\textit{@cert}, \textit{@resp})) (att.global.source (\textit{@source})) att.naming (\textit{@role}, \textit{@nymRef})  (att.canonical (\textit{@key}, \textit{@ref}))
    \item[{Member of}]
  model.respLike 
    \item[{Contained by}]
  
    \item[core: ]
   analytic bibl monogr series\par 
    \item[header: ]
   editionStmt seriesStmt titleStmt\par 
    \item[msdescription: ]
   msItem
    \item[{May contain}]
  
    \item[analysis: ]
   c cl interp interpGrp m pc phr s span spanGrp w\par 
    \item[core: ]
   abbr add address cb choice corr date del distinct email emph expan foreign gap gb gloss graphic hi index lb measure measureGrp media mentioned milestone name note num orig pb ptr ref reg rs sic soCalled term time title unclear\par 
    \item[figures: ]
   figure formula notatedMusic\par 
    \item[gaiji: ]
   g\par 
    \item[header: ]
   idno\par 
    \item[linking: ]
   alt altGrp anchor join joinGrp link linkGrp seg timeline\par 
    \item[msdescription: ]
   catchwords depth dim dimensions height heraldry locus locusGrp material objectType origDate origPlace secFol signatures stamp watermark width\par 
    \item[namesdates: ]
   addName affiliation bloc climate country district forename genName geo geogFeat geogName location nameLink offset orgName persName placeName population region roleName settlement state surname terrain trait\par 
    \item[textcrit: ]
   app witDetail\par 
    \item[transcr: ]
   addSpan am damage damageSpan delSpan ex fw handShift listTranspose metamark mod redo restore retrace secl space subst substJoin supplied surplus undo\par character data
    \item[{Note}]
  \par
A consistent format should be adopted.\par
Particularly where cataloguing is likely to be based on the content of the header, it is advisable to use generally recognized authority lists for the exact form of personal names. 
    \item[{Example}]
  \leavevmode\bgroup\exampleFont \begin{shaded}\noindent\mbox{}{<\textbf{editor}>}Eric Johnson{</\textbf{editor}>}\mbox{}\newline 
{<\textbf{editor}\hspace*{6pt}{role}="{illustrator}">}John Tenniel{</\textbf{editor}>}\end{shaded}\egroup 


    \item[{Content model}]
  \mbox{}\hfill\\[-10pt]\begin{Verbatim}[fontsize=\small]
<content>
 <macroRef key="macro.phraseSeq"/>
</content>
    
\end{Verbatim}

    \item[{Schema Declaration}]
  \mbox{}\hfill\\[-10pt]\begin{Verbatim}[fontsize=\small]
element editor
{
   att.global.attributes,
   att.naming.attributes,
   macro.phraseSeq}
\end{Verbatim}

\end{reflist}  \index{editorialDecl=<editorialDecl>|oddindex}
\begin{reflist}
\item[]\begin{specHead}{TEI.editorialDecl}{<editorialDecl> }(editorial practice declaration) provides details of editorial principles and practices applied during the encoding of a text. [\xref{http://www.tei-c.org/release/doc/tei-p5-doc/en/html/HD.html\#HD53}{2.3.3. The Editorial Practices Declaration} \xref{http://www.tei-c.org/release/doc/tei-p5-doc/en/html/HD.html\#HD5}{2.3. The Encoding Description} \xref{http://www.tei-c.org/release/doc/tei-p5-doc/en/html/CC.html\#CCAS2}{15.3.2. Declarable Elements}]\end{specHead} 
    \item[{Module}]
  header
    \item[{Attributes}]
  Attributes att.global (\textit{@xml:id}, \textit{@n}, \textit{@xml:lang}, \textit{@xml:base}, \textit{@xml:space})  (att.global.rendition (\textit{@rend}, \textit{@style}, \textit{@rendition})) (att.global.linking (\textit{@corresp}, \textit{@synch}, \textit{@sameAs}, \textit{@copyOf}, \textit{@next}, \textit{@prev}, \textit{@exclude}, \textit{@select})) (att.global.analytic (\textit{@ana})) (att.global.facs (\textit{@facs})) (att.global.change (\textit{@change})) (att.global.responsibility (\textit{@cert}, \textit{@resp})) (att.global.source (\textit{@source})) att.declarable (\textit{@default}) 
    \item[{Member of}]
  model.encodingDescPart
    \item[{Contained by}]
  
    \item[header: ]
   encodingDesc
    \item[{May contain}]
  
    \item[core: ]
   p\par 
    \item[header: ]
   correction hyphenation interpretation normalization punctuation quotation segmentation stdVals\par 
    \item[linking: ]
   ab
    \item[{Example}]
  \leavevmode\bgroup\exampleFont \begin{shaded}\noindent\mbox{}{<\textbf{editorialDecl}>}\mbox{}\newline 
\hspace*{6pt}{<\textbf{normalization}>}\mbox{}\newline 
\hspace*{6pt}\hspace*{6pt}{<\textbf{p}>}All words converted to Modern American spelling using\mbox{}\newline 
\hspace*{6pt}\hspace*{6pt}\hspace*{6pt}\hspace*{6pt} Websters 9th Collegiate dictionary\mbox{}\newline 
\hspace*{6pt}\hspace*{6pt}{</\textbf{p}>}\mbox{}\newline 
\hspace*{6pt}{</\textbf{normalization}>}\mbox{}\newline 
\hspace*{6pt}{<\textbf{quotation}\hspace*{6pt}{marks}="{all}">}\mbox{}\newline 
\hspace*{6pt}\hspace*{6pt}{<\textbf{p}>}All opening quotation marks converted to “ all closing\mbox{}\newline 
\hspace*{6pt}\hspace*{6pt}\hspace*{6pt}\hspace*{6pt} quotation marks converted to \&cdq;.{</\textbf{p}>}\mbox{}\newline 
\hspace*{6pt}{</\textbf{quotation}>}\mbox{}\newline 
{</\textbf{editorialDecl}>}\end{shaded}\egroup 


    \item[{Content model}]
  \mbox{}\hfill\\[-10pt]\begin{Verbatim}[fontsize=\small]
<content>
 <alternate maxOccurs="unbounded"
  minOccurs="1">
  <classRef key="model.pLike"/>
  <classRef key="model.editorialDeclPart"/>
 </alternate>
</content>
    
\end{Verbatim}

    \item[{Schema Declaration}]
  \mbox{}\hfill\\[-10pt]\begin{Verbatim}[fontsize=\small]
element editorialDecl
{
   att.global.attributes,
   att.declarable.attributes,
   ( model.pLike | model.editorialDeclPart )+
}
\end{Verbatim}

\end{reflist}  \index{education=<education>|oddindex}
\begin{reflist}
\item[]\begin{specHead}{TEI.education}{<education> }contains a description of the educational experience of a person. [\xref{http://www.tei-c.org/release/doc/tei-p5-doc/en/html/CC.html\#CCAHPA}{15.2.2. The Participant Description}]\end{specHead} 
    \item[{Module}]
  namesdates
    \item[{Attributes}]
  Attributes att.global (\textit{@xml:id}, \textit{@n}, \textit{@xml:lang}, \textit{@xml:base}, \textit{@xml:space})  (att.global.rendition (\textit{@rend}, \textit{@style}, \textit{@rendition})) (att.global.linking (\textit{@corresp}, \textit{@synch}, \textit{@sameAs}, \textit{@copyOf}, \textit{@next}, \textit{@prev}, \textit{@exclude}, \textit{@select})) (att.global.analytic (\textit{@ana})) (att.global.facs (\textit{@facs})) (att.global.change (\textit{@change})) (att.global.responsibility (\textit{@cert}, \textit{@resp})) (att.global.source (\textit{@source})) att.editLike (\textit{@evidence}, \textit{@instant})  (att.dimensions (\textit{@unit}, \textit{@quantity}, \textit{@extent}, \textit{@precision}, \textit{@scope}) (att.ranging (\textit{@atLeast}, \textit{@atMost}, \textit{@min}, \textit{@max}, \textit{@confidence})) ) att.datable (\textit{@calendar}, \textit{@period})  (att.datable.w3c (\textit{@when}, \textit{@notBefore}, \textit{@notAfter}, \textit{@from}, \textit{@to})) (att.datable.iso (\textit{@when-iso}, \textit{@notBefore-iso}, \textit{@notAfter-iso}, \textit{@from-iso}, \textit{@to-iso})) (att.datable.custom (\textit{@when-custom}, \textit{@notBefore-custom}, \textit{@notAfter-custom}, \textit{@from-custom}, \textit{@to-custom}, \textit{@datingPoint}, \textit{@datingMethod})) att.naming (\textit{@role}, \textit{@nymRef})  (att.canonical (\textit{@key}, \textit{@ref}))
    \item[{Member of}]
  model.persStateLike
    \item[{Contained by}]
  
    \item[namesdates: ]
   person personGrp
    \item[{May contain}]
  
    \item[analysis: ]
   c cl interp interpGrp m pc phr s span spanGrp w\par 
    \item[core: ]
   abbr add address cb choice corr date del distinct email emph expan foreign gap gb gloss graphic hi index lb measure measureGrp media mentioned milestone name note num orig pb ptr ref reg rs sic soCalled term time title unclear\par 
    \item[figures: ]
   figure formula notatedMusic\par 
    \item[gaiji: ]
   g\par 
    \item[header: ]
   idno\par 
    \item[linking: ]
   alt altGrp anchor join joinGrp link linkGrp seg timeline\par 
    \item[msdescription: ]
   catchwords depth dim dimensions height heraldry locus locusGrp material objectType origDate origPlace secFol signatures stamp watermark width\par 
    \item[namesdates: ]
   addName affiliation bloc climate country district forename genName geo geogFeat geogName location nameLink offset orgName persName placeName population region roleName settlement state surname terrain trait\par 
    \item[textcrit: ]
   app witDetail\par 
    \item[transcr: ]
   addSpan am damage damageSpan delSpan ex fw handShift listTranspose metamark mod redo restore retrace secl space subst substJoin supplied surplus undo\par character data
    \item[{Example}]
  \leavevmode\bgroup\exampleFont \begin{shaded}\noindent\mbox{}{<\textbf{education}>}Left school at age 16{</\textbf{education}>}\end{shaded}\egroup 


    \item[{Example}]
  \leavevmode\bgroup\exampleFont \begin{shaded}\noindent\mbox{}{<\textbf{education}\hspace*{6pt}{from}="{1986-01-01}"\mbox{}\newline 
\hspace*{6pt}{to}="{1990-06-30}">}Attended {<\textbf{name}>}Cherwell School{</\textbf{name}>}\mbox{}\newline 
{</\textbf{education}>}\end{shaded}\egroup 


    \item[{Example}]
  \leavevmode\bgroup\exampleFont \begin{shaded}\noindent\mbox{}{<\textbf{education}\hspace*{6pt}{notAfter}="{1690-06}"\mbox{}\newline 
\hspace*{6pt}{notBefore}="{1685-07}">}Anthony Hammond\mbox{}\newline 
 smuggled her into the University of Cambridge, where she was\mbox{}\newline 
 disguised as his male cousin, Jack. She remained there for some\mbox{}\newline 
 months learning grammar, logic, rhetoric, and ethics{</\textbf{education}>}\end{shaded}\egroup 


    \item[{Content model}]
  \mbox{}\hfill\\[-10pt]\begin{Verbatim}[fontsize=\small]
<content>
 <macroRef key="macro.phraseSeq"/>
</content>
    
\end{Verbatim}

    \item[{Schema Declaration}]
  \mbox{}\hfill\\[-10pt]\begin{Verbatim}[fontsize=\small]
element education
{
   att.global.attributes,
   att.editLike.attributes,
   att.datable.attributes,
   att.naming.attributes,
   macro.phraseSeq}
\end{Verbatim}

\end{reflist}  \index{email=<email>|oddindex}
\begin{reflist}
\item[]\begin{specHead}{TEI.email}{<email> }(electronic mail address) contains an email address identifying a location to which email messages can be delivered. [\xref{http://www.tei-c.org/release/doc/tei-p5-doc/en/html/CO.html\#CONAAD}{3.5.2. Addresses}]\end{specHead} 
    \item[{Module}]
  core
    \item[{Attributes}]
  Attributes att.global (\textit{@xml:id}, \textit{@n}, \textit{@xml:lang}, \textit{@xml:base}, \textit{@xml:space})  (att.global.rendition (\textit{@rend}, \textit{@style}, \textit{@rendition})) (att.global.linking (\textit{@corresp}, \textit{@synch}, \textit{@sameAs}, \textit{@copyOf}, \textit{@next}, \textit{@prev}, \textit{@exclude}, \textit{@select})) (att.global.analytic (\textit{@ana})) (att.global.facs (\textit{@facs})) (att.global.change (\textit{@change})) (att.global.responsibility (\textit{@cert}, \textit{@resp})) (att.global.source (\textit{@source}))
    \item[{Member of}]
  model.addressLike
    \item[{Contained by}]
  
    \item[analysis: ]
   cl phr s span\par 
    \item[core: ]
   abbr add addrLine author bibl biblScope citedRange corr date del desc distinct editor email emph expan foreign gloss head headItem headLabel hi item l label measure meeting mentioned name note num orig p pubPlace publisher q quote ref reg resp rs said sic soCalled speaker stage street term textLang time title unclear\par 
    \item[figures: ]
   cell figDesc\par 
    \item[header: ]
   authority catDesc change classCode correspAction creation distributor edition extent funder geoDecl handNote language licence principal rendition scriptNote sponsor tagUsage typeNote\par 
    \item[linking: ]
   ab seg\par 
    \item[msdescription: ]
   accMat acquisition additions catchwords collation colophon condition custEvent decoNote explicit filiation finalRubric foliation heraldry incipit layout material musicNotation objectType origDate origPlace origin provenance rubric secFol signatures source stamp summary support surrogates watermark\par 
    \item[namesdates: ]
   addName affiliation age birth bloc country death district education faith floruit forename genName geogFeat geogName langKnown location nameLink nationality occupation offset orgName persName placeName region residence roleName settlement sex socecStatus surname\par 
    \item[textcrit: ]
   lem rdg wit witDetail witness\par 
    \item[textstructure: ]
   byline closer dateline docAuthor docDate docEdition docImprint imprimatur opener salute signed titlePart trailer\par 
    \item[transcr: ]
   damage fw metamark mod restore retrace secl supplied surplus
    \item[{May contain}]
  
    \item[analysis: ]
   c cl interp interpGrp m pc phr s span spanGrp w\par 
    \item[core: ]
   abbr add address cb choice corr date del distinct email emph expan foreign gap gb gloss graphic hi index lb measure measureGrp media mentioned milestone name note num orig pb ptr ref reg rs sic soCalled term time title unclear\par 
    \item[figures: ]
   figure formula notatedMusic\par 
    \item[gaiji: ]
   g\par 
    \item[header: ]
   idno\par 
    \item[linking: ]
   alt altGrp anchor join joinGrp link linkGrp seg timeline\par 
    \item[msdescription: ]
   catchwords depth dim dimensions height heraldry locus locusGrp material objectType origDate origPlace secFol signatures stamp watermark width\par 
    \item[namesdates: ]
   addName affiliation bloc climate country district forename genName geo geogFeat geogName location nameLink offset orgName persName placeName population region roleName settlement state surname terrain trait\par 
    \item[textcrit: ]
   app witDetail\par 
    \item[transcr: ]
   addSpan am damage damageSpan delSpan ex fw handShift listTranspose metamark mod redo restore retrace secl space subst substJoin supplied surplus undo\par character data
    \item[{Note}]
  \par
The format of a modern Internet email address is defined in \xref{https://tools.ietf.org/html/rfc2822}{RFC 2822}
    \item[{Example}]
  \leavevmode\bgroup\exampleFont \begin{shaded}\noindent\mbox{}{<\textbf{email}>}membership@tei-c.org{</\textbf{email}>}\end{shaded}\egroup 


    \item[{Content model}]
  \mbox{}\hfill\\[-10pt]\begin{Verbatim}[fontsize=\small]
<content>
 <macroRef key="macro.phraseSeq"/>
</content>
    
\end{Verbatim}

    \item[{Schema Declaration}]
  \mbox{}\hfill\\[-10pt]\begin{Verbatim}[fontsize=\small]
element email { att.global.attributes, macro.phraseSeq }
\end{Verbatim}

\end{reflist}  \index{emph=<emph>|oddindex}
\begin{reflist}
\item[]\begin{specHead}{TEI.emph}{<emph> }(emphasized) marks words or phrases which are stressed or emphasized for linguistic or rhetorical effect. [\xref{http://www.tei-c.org/release/doc/tei-p5-doc/en/html/CO.html\#COHQHE}{3.3.2.2. Emphatic Words and Phrases} \xref{http://www.tei-c.org/release/doc/tei-p5-doc/en/html/CO.html\#COHQH}{3.3.2. Emphasis, Foreign Words, and Unusual Language}]\end{specHead} 
    \item[{Module}]
  core
    \item[{Attributes}]
  Attributes att.global (\textit{@xml:id}, \textit{@n}, \textit{@xml:lang}, \textit{@xml:base}, \textit{@xml:space})  (att.global.rendition (\textit{@rend}, \textit{@style}, \textit{@rendition})) (att.global.linking (\textit{@corresp}, \textit{@synch}, \textit{@sameAs}, \textit{@copyOf}, \textit{@next}, \textit{@prev}, \textit{@exclude}, \textit{@select})) (att.global.analytic (\textit{@ana})) (att.global.facs (\textit{@facs})) (att.global.change (\textit{@change})) (att.global.responsibility (\textit{@cert}, \textit{@resp})) (att.global.source (\textit{@source}))
    \item[{Member of}]
  model.emphLike
    \item[{Contained by}]
  
    \item[analysis: ]
   cl phr s span\par 
    \item[core: ]
   abbr add addrLine author bibl biblScope citedRange corr date del desc distinct editor email emph expan foreign gloss head headItem headLabel hi item l label measure meeting mentioned name note num orig p pubPlace publisher q quote ref reg resp rs said sic soCalled speaker stage street term textLang time title unclear\par 
    \item[figures: ]
   cell figDesc\par 
    \item[header: ]
   authority catDesc change classCode creation distributor edition extent funder geoDecl handNote language licence principal rendition scriptNote sponsor tagUsage typeNote\par 
    \item[linking: ]
   ab seg\par 
    \item[msdescription: ]
   accMat acquisition additions catchwords collation colophon condition custEvent decoNote explicit filiation finalRubric foliation heraldry incipit layout material musicNotation objectType origDate origPlace origin provenance rubric secFol signatures source stamp summary support surrogates watermark\par 
    \item[namesdates: ]
   addName affiliation age birth bloc country death district education faith floruit forename genName geogFeat geogName langKnown nameLink nationality occupation offset orgName persName placeName region residence roleName settlement sex socecStatus surname\par 
    \item[textcrit: ]
   lem rdg wit witDetail witness\par 
    \item[textstructure: ]
   byline closer dateline docAuthor docDate docEdition docImprint imprimatur opener salute signed titlePart trailer\par 
    \item[transcr: ]
   damage fw metamark mod restore retrace secl supplied surplus
    \item[{May contain}]
  
    \item[analysis: ]
   c cl interp interpGrp m pc phr s span spanGrp w\par 
    \item[core: ]
   abbr add address bibl biblStruct cb choice cit corr date del desc distinct email emph expan foreign gap gb gloss graphic hi index l label lb lg list listBibl measure measureGrp media mentioned milestone name note num orig pb ptr q quote ref reg rs said sic soCalled stage term time title unclear\par 
    \item[figures: ]
   figure formula notatedMusic table\par 
    \item[gaiji: ]
   g\par 
    \item[header: ]
   biblFull idno\par 
    \item[linking: ]
   alt altGrp anchor join joinGrp link linkGrp seg timeline\par 
    \item[msdescription: ]
   catchwords depth dim dimensions height heraldry locus locusGrp material msDesc objectType origDate origPlace secFol signatures stamp watermark width\par 
    \item[namesdates: ]
   addName affiliation bloc climate country district forename genName geo geogFeat geogName listEvent listNym listOrg listPerson listPlace location nameLink offset orgName persName placeName population region roleName settlement state surname terrain trait\par 
    \item[textcrit: ]
   app listApp listWit witDetail\par 
    \item[textstructure: ]
   floatingText\par 
    \item[transcr: ]
   addSpan am damage damageSpan delSpan ex fw handShift listTranspose metamark mod redo restore retrace secl space subst substJoin supplied surplus undo\par character data
    \item[{Example}]
  \leavevmode\bgroup\exampleFont \begin{shaded}\noindent\mbox{}You took the car and did {<\textbf{emph}>}what{</\textbf{emph}>}?!!\end{shaded}\egroup 


    \item[{Example}]
  \leavevmode\bgroup\exampleFont \begin{shaded}\noindent\mbox{}{<\textbf{q}>}What it all comes to is this,{</\textbf{q}>} he said. \mbox{}\newline 
{<\textbf{q}>}\mbox{}\newline 
\hspace*{6pt}{<\textbf{emph}>}What\mbox{}\newline 
\hspace*{6pt}\hspace*{6pt} does Christopher Robin do in the morning nowadays?{</\textbf{emph}>}\mbox{}\newline 
{</\textbf{q}>}\end{shaded}\egroup 


    \item[{Content model}]
  \mbox{}\hfill\\[-10pt]\begin{Verbatim}[fontsize=\small]
<content>
 <macroRef key="macro.paraContent"/>
</content>
    
\end{Verbatim}

    \item[{Schema Declaration}]
  \mbox{}\hfill\\[-10pt]\begin{Verbatim}[fontsize=\small]
element emph { att.global.attributes, macro.paraContent }
\end{Verbatim}

\end{reflist}  \index{encodingDesc=<encodingDesc>|oddindex}
\begin{reflist}
\item[]\begin{specHead}{TEI.encodingDesc}{<encodingDesc> }(encoding description) documents the relationship between an electronic text and the source or sources from which it was derived. [\xref{http://www.tei-c.org/release/doc/tei-p5-doc/en/html/HD.html\#HD5}{2.3. The Encoding Description} \xref{http://www.tei-c.org/release/doc/tei-p5-doc/en/html/HD.html\#HD11}{2.1.1. The TEI Header and Its Components}]\end{specHead} 
    \item[{Module}]
  header
    \item[{Attributes}]
  Attributes att.global (\textit{@xml:id}, \textit{@n}, \textit{@xml:lang}, \textit{@xml:base}, \textit{@xml:space})  (att.global.rendition (\textit{@rend}, \textit{@style}, \textit{@rendition})) (att.global.linking (\textit{@corresp}, \textit{@synch}, \textit{@sameAs}, \textit{@copyOf}, \textit{@next}, \textit{@prev}, \textit{@exclude}, \textit{@select})) (att.global.analytic (\textit{@ana})) (att.global.facs (\textit{@facs})) (att.global.change (\textit{@change})) (att.global.responsibility (\textit{@cert}, \textit{@resp})) (att.global.source (\textit{@source}))
    \item[{Member of}]
  model.teiHeaderPart
    \item[{Contained by}]
  
    \item[header: ]
   teiHeader
    \item[{May contain}]
  
    \item[core: ]
   p\par 
    \item[gaiji: ]
   charDecl\par 
    \item[header: ]
   appInfo classDecl editorialDecl geoDecl listPrefixDef projectDesc refsDecl samplingDecl schemaRef styleDefDecl tagsDecl\par 
    \item[linking: ]
   ab\par 
    \item[textcrit: ]
   variantEncoding
    \item[{Example}]
  \leavevmode\bgroup\exampleFont \begin{shaded}\noindent\mbox{}{<\textbf{encodingDesc}>}\mbox{}\newline 
\hspace*{6pt}{<\textbf{p}>}Basic encoding, capturing lexical information only. All\mbox{}\newline 
\hspace*{6pt}\hspace*{6pt} hyphenation, punctuation, and variant spellings normalized. No\mbox{}\newline 
\hspace*{6pt}\hspace*{6pt} formatting or layout information preserved.{</\textbf{p}>}\mbox{}\newline 
{</\textbf{encodingDesc}>}\end{shaded}\egroup 


    \item[{Content model}]
  \mbox{}\hfill\\[-10pt]\begin{Verbatim}[fontsize=\small]
<content>
 <alternate maxOccurs="unbounded"
  minOccurs="1">
  <classRef key="model.encodingDescPart"/>
  <classRef key="model.pLike"/>
 </alternate>
</content>
    
\end{Verbatim}

    \item[{Schema Declaration}]
  \mbox{}\hfill\\[-10pt]\begin{Verbatim}[fontsize=\small]
element encodingDesc
{
   att.global.attributes,
   ( model.encodingDescPart | model.pLike )+
}
\end{Verbatim}

\end{reflist}  \index{epigraph=<epigraph>|oddindex}
\begin{reflist}
\item[]\begin{specHead}{TEI.epigraph}{<epigraph> }contains a quotation, anonymous or attributed, appearing at the start or end of a section or on a title page. [\xref{http://www.tei-c.org/release/doc/tei-p5-doc/en/html/DS.html\#DSAE}{4.2.3. Arguments, Epigraphs, and Postscripts} \xref{http://www.tei-c.org/release/doc/tei-p5-doc/en/html/DS.html\#DSDTB}{4.2. Elements Common to All Divisions} \xref{http://www.tei-c.org/release/doc/tei-p5-doc/en/html/DS.html\#DSTITL}{4.6. Title Pages}]\end{specHead} 
    \item[{Module}]
  textstructure
    \item[{Attributes}]
  Attributes att.global (\textit{@xml:id}, \textit{@n}, \textit{@xml:lang}, \textit{@xml:base}, \textit{@xml:space})  (att.global.rendition (\textit{@rend}, \textit{@style}, \textit{@rendition})) (att.global.linking (\textit{@corresp}, \textit{@synch}, \textit{@sameAs}, \textit{@copyOf}, \textit{@next}, \textit{@prev}, \textit{@exclude}, \textit{@select})) (att.global.analytic (\textit{@ana})) (att.global.facs (\textit{@facs})) (att.global.change (\textit{@change})) (att.global.responsibility (\textit{@cert}, \textit{@resp})) (att.global.source (\textit{@source}))
    \item[{Member of}]
  model.divWrapper model.pLike.front model.titlepagePart 
    \item[{Contained by}]
  
    \item[core: ]
   lg list\par 
    \item[figures: ]
   figure table\par 
    \item[msdescription: ]
   msItem\par 
    \item[textstructure: ]
   back body div front group opener titlePage
    \item[{May contain}]
  
    \item[analysis: ]
   interp interpGrp span spanGrp\par 
    \item[core: ]
   bibl biblStruct cb cit desc gap gb index l label lb lg list listBibl milestone note p pb q quote said sp stage\par 
    \item[figures: ]
   figure notatedMusic table\par 
    \item[header: ]
   biblFull\par 
    \item[linking: ]
   ab alt altGrp anchor join joinGrp link linkGrp timeline\par 
    \item[msdescription: ]
   msDesc\par 
    \item[namesdates: ]
   listEvent listNym listOrg listPerson listPlace\par 
    \item[textcrit: ]
   app listApp listWit witDetail\par 
    \item[textstructure: ]
   floatingText\par 
    \item[transcr: ]
   addSpan damageSpan delSpan fw listTranspose metamark space substJoin
    \item[{Example}]
  \leavevmode\bgroup\exampleFont \begin{shaded}\noindent\mbox{}{<\textbf{epigraph}\hspace*{6pt}{xml:lang}="{la}">}\mbox{}\newline 
\hspace*{6pt}{<\textbf{cit}>}\mbox{}\newline 
\hspace*{6pt}\hspace*{6pt}{<\textbf{bibl}>}Lucret.{</\textbf{bibl}>}\mbox{}\newline 
\hspace*{6pt}\hspace*{6pt}{<\textbf{quote}>}\mbox{}\newline 
\hspace*{6pt}\hspace*{6pt}\hspace*{6pt}{<\textbf{l}\hspace*{6pt}{part}="{F}">}petere inde coronam,{</\textbf{l}>}\mbox{}\newline 
\hspace*{6pt}\hspace*{6pt}\hspace*{6pt}{<\textbf{l}>}Vnde prius nulli velarint tempora Musae.{</\textbf{l}>}\mbox{}\newline 
\hspace*{6pt}\hspace*{6pt}{</\textbf{quote}>}\mbox{}\newline 
\hspace*{6pt}{</\textbf{cit}>}\mbox{}\newline 
{</\textbf{epigraph}>}\end{shaded}\egroup 


    \item[{Content model}]
  \mbox{}\hfill\\[-10pt]\begin{Verbatim}[fontsize=\small]
<content>
 <alternate maxOccurs="unbounded"
  minOccurs="0">
  <classRef key="model.common"/>
  <classRef key="model.global"/>
 </alternate>
</content>
    
\end{Verbatim}

    \item[{Schema Declaration}]
  \mbox{}\hfill\\[-10pt]\begin{Verbatim}[fontsize=\small]
element epigraph { att.global.attributes, ( model.common | model.global )* }
\end{Verbatim}

\end{reflist}  \index{event=<event>|oddindex}\index{where=@where!<event>|oddindex}
\begin{reflist}
\item[]\begin{specHead}{TEI.event}{<event> }contains data relating to any kind of significant event associated with a person, place, or organization. [\xref{http://www.tei-c.org/release/doc/tei-p5-doc/en/html/ND.html\#NDPERSbp}{13.3.1. Basic Principles}]\end{specHead} 
    \item[{Module}]
  namesdates
    \item[{Attributes}]
  Attributes att.global (\textit{@xml:id}, \textit{@n}, \textit{@xml:lang}, \textit{@xml:base}, \textit{@xml:space})  (att.global.rendition (\textit{@rend}, \textit{@style}, \textit{@rendition})) (att.global.linking (\textit{@corresp}, \textit{@synch}, \textit{@sameAs}, \textit{@copyOf}, \textit{@next}, \textit{@prev}, \textit{@exclude}, \textit{@select})) (att.global.analytic (\textit{@ana})) (att.global.facs (\textit{@facs})) (att.global.change (\textit{@change})) (att.global.responsibility (\textit{@cert}, \textit{@resp})) (att.global.source (\textit{@source})) att.datable (\textit{@calendar}, \textit{@period})  (att.datable.w3c (\textit{@when}, \textit{@notBefore}, \textit{@notAfter}, \textit{@from}, \textit{@to})) (att.datable.iso (\textit{@when-iso}, \textit{@notBefore-iso}, \textit{@notAfter-iso}, \textit{@from-iso}, \textit{@to-iso})) (att.datable.custom (\textit{@when-custom}, \textit{@notBefore-custom}, \textit{@notAfter-custom}, \textit{@from-custom}, \textit{@to-custom}, \textit{@datingPoint}, \textit{@datingMethod})) att.editLike (\textit{@evidence}, \textit{@instant})  (att.dimensions (\textit{@unit}, \textit{@quantity}, \textit{@extent}, \textit{@precision}, \textit{@scope}) (att.ranging (\textit{@atLeast}, \textit{@atMost}, \textit{@min}, \textit{@max}, \textit{@confidence})) ) att.typed (\textit{@type}, \textit{@subtype}) att.naming (\textit{@role}, \textit{@nymRef})  (att.canonical (\textit{@key}, \textit{@ref})) att.sortable (\textit{@sortKey}) \hfil\\[-10pt]\begin{sansreflist}
    \item[@where]
  indicates the location of an event by pointing to a <place> element
\begin{reflist}
    \item[{Status}]
  Optional
    \item[{Datatype}]
  teidata.pointer
\end{reflist}  
\end{sansreflist}  
    \item[{Member of}]
  model.eventLike
    \item[{Contained by}]
  
    \item[namesdates: ]
   event listEvent org person personGrp place
    \item[{May contain}]
  
    \item[core: ]
   bibl biblStruct desc head label listBibl note p\par 
    \item[header: ]
   biblFull\par 
    \item[linking: ]
   ab link linkGrp\par 
    \item[msdescription: ]
   msDesc\par 
    \item[namesdates: ]
   event\par 
    \item[textcrit: ]
   witDetail
    \item[{Example}]
  \leavevmode\bgroup\exampleFont \begin{shaded}\noindent\mbox{}{<\textbf{person}>}\mbox{}\newline 
\hspace*{6pt}{<\textbf{event}\hspace*{6pt}{type}="{mat}"\hspace*{6pt}{when}="{1972-10-12}">}\mbox{}\newline 
\hspace*{6pt}\hspace*{6pt}{<\textbf{label}>}matriculation{</\textbf{label}>}\mbox{}\newline 
\hspace*{6pt}{</\textbf{event}>}\mbox{}\newline 
\hspace*{6pt}{<\textbf{event}\hspace*{6pt}{type}="{grad}"\hspace*{6pt}{when}="{1975-06-23}">}\mbox{}\newline 
\hspace*{6pt}\hspace*{6pt}{<\textbf{label}>}graduation{</\textbf{label}>}\mbox{}\newline 
\hspace*{6pt}{</\textbf{event}>}\mbox{}\newline 
{</\textbf{person}>}\end{shaded}\egroup 


    \item[{Content model}]
  \mbox{}\hfill\\[-10pt]\begin{Verbatim}[fontsize=\small]
<content>
 <sequence>
  <classRef key="model.headLike"
   maxOccurs="unbounded" minOccurs="0"/>
  <alternate>
   <classRef key="model.pLike"
    maxOccurs="unbounded" minOccurs="1"/>
   <classRef key="model.labelLike"
    maxOccurs="unbounded" minOccurs="1"/>
  </alternate>
  <alternate maxOccurs="unbounded"
   minOccurs="0">
   <classRef key="model.noteLike"/>
   <classRef key="model.biblLike"/>
   <elementRef key="linkGrp"/>
   <elementRef key="link"/>
  </alternate>
  <elementRef key="event"
   maxOccurs="unbounded" minOccurs="0"/>
 </sequence>
</content>
    
\end{Verbatim}

    \item[{Schema Declaration}]
  \mbox{}\hfill\\[-10pt]\begin{Verbatim}[fontsize=\small]
element event
{
   att.global.attributes,
   att.datable.attributes,
   att.editLike.attributes,
   att.typed.attributes,
   att.naming.attributes,
   att.sortable.attributes,
   attribute where { text }?,
   (
      model.headLike*,
      ( model.pLike+ | model.labelLike+ ),
      ( model.noteLike | model.biblLike | linkGrp | link )*,
      event*
   )
}
\end{Verbatim}

\end{reflist}  \index{ex=<ex>|oddindex}
\begin{reflist}
\item[]\begin{specHead}{TEI.ex}{<ex> }(editorial expansion) contains a sequence of letters added by an editor or transcriber when expanding an abbreviation. [\xref{http://www.tei-c.org/release/doc/tei-p5-doc/en/html/PH.html\#PHAB}{11.3.1.2. Abbreviation and Expansion}]\end{specHead} 
    \item[{Module}]
  transcr
    \item[{Attributes}]
  Attributes att.global (\textit{@xml:id}, \textit{@n}, \textit{@xml:lang}, \textit{@xml:base}, \textit{@xml:space})  (att.global.rendition (\textit{@rend}, \textit{@style}, \textit{@rendition})) (att.global.linking (\textit{@corresp}, \textit{@synch}, \textit{@sameAs}, \textit{@copyOf}, \textit{@next}, \textit{@prev}, \textit{@exclude}, \textit{@select})) (att.global.analytic (\textit{@ana})) (att.global.facs (\textit{@facs})) (att.global.change (\textit{@change})) (att.global.responsibility (\textit{@cert}, \textit{@resp})) (att.global.source (\textit{@source})) att.editLike (\textit{@evidence}, \textit{@instant})  (att.dimensions (\textit{@unit}, \textit{@quantity}, \textit{@extent}, \textit{@precision}, \textit{@scope}) (att.ranging (\textit{@atLeast}, \textit{@atMost}, \textit{@min}, \textit{@max}, \textit{@confidence})) )
    \item[{Member of}]
  model.choicePart model.pPart.editorial
    \item[{Contained by}]
  
    \item[analysis: ]
   cl pc phr s span w\par 
    \item[core: ]
   abbr add addrLine author bibl biblScope choice citedRange corr date del desc distinct editor email emph expan foreign gloss head headItem headLabel hi item l label measure meeting mentioned name note num orig p pubPlace publisher q quote ref reg resp rs said sic soCalled speaker stage street term textLang time title unclear\par 
    \item[figures: ]
   cell figDesc\par 
    \item[header: ]
   authority catDesc change classCode creation distributor edition extent funder geoDecl handNote language licence principal rendition scriptNote sponsor tagUsage typeNote\par 
    \item[linking: ]
   ab seg\par 
    \item[msdescription: ]
   accMat acquisition additions catchwords collation colophon condition custEvent decoNote explicit filiation finalRubric foliation heraldry incipit layout material musicNotation objectType origDate origPlace origin provenance rubric secFol signatures source stamp summary support surrogates watermark\par 
    \item[namesdates: ]
   addName affiliation age birth bloc country death district education faith floruit forename genName geogFeat geogName langKnown nameLink nationality occupation offset orgName persName placeName region residence roleName settlement sex socecStatus surname\par 
    \item[textcrit: ]
   lem rdg wit witDetail witness\par 
    \item[textstructure: ]
   byline closer dateline docAuthor docDate docEdition docImprint imprimatur opener salute signed titlePart trailer\par 
    \item[transcr: ]
   damage fw metamark mod restore retrace secl supplied surplus
    \item[{May contain}]
  
    \item[gaiji: ]
   g\par character data
    \item[{Example}]
  \leavevmode\bgroup\exampleFont \begin{shaded}\noindent\mbox{}The address is Southmoor {<\textbf{choice}>}\mbox{}\newline 
\hspace*{6pt}{<\textbf{expan}>}R{<\textbf{ex}>}oa{</\textbf{ex}>}d{</\textbf{expan}>}\mbox{}\newline 
\hspace*{6pt}{<\textbf{abbr}>}Rd{</\textbf{abbr}>}\mbox{}\newline 
{</\textbf{choice}>}\end{shaded}\egroup 


    \item[{Content model}]
  \fbox{\ttfamily <content>\newline
 <macroRef key="macro.xtext"/>\newline
</content>\newline
    } 
    \item[{Schema Declaration}]
  \mbox{}\hfill\\[-10pt]\begin{Verbatim}[fontsize=\small]
element ex { att.global.attributes, att.editLike.attributes, macro.xtext }
\end{Verbatim}

\end{reflist}  \index{expan=<expan>|oddindex}
\begin{reflist}
\item[]\begin{specHead}{TEI.expan}{<expan> }(expansion) contains the expansion of an abbreviation. [\xref{http://www.tei-c.org/release/doc/tei-p5-doc/en/html/CO.html\#CONAAB}{3.5.5. Abbreviations and Their Expansions}]\end{specHead} 
    \item[{Module}]
  core
    \item[{Attributes}]
  Attributes att.global (\textit{@xml:id}, \textit{@n}, \textit{@xml:lang}, \textit{@xml:base}, \textit{@xml:space})  (att.global.rendition (\textit{@rend}, \textit{@style}, \textit{@rendition})) (att.global.linking (\textit{@corresp}, \textit{@synch}, \textit{@sameAs}, \textit{@copyOf}, \textit{@next}, \textit{@prev}, \textit{@exclude}, \textit{@select})) (att.global.analytic (\textit{@ana})) (att.global.facs (\textit{@facs})) (att.global.change (\textit{@change})) (att.global.responsibility (\textit{@cert}, \textit{@resp})) (att.global.source (\textit{@source})) att.editLike (\textit{@evidence}, \textit{@instant})  (att.dimensions (\textit{@unit}, \textit{@quantity}, \textit{@extent}, \textit{@precision}, \textit{@scope}) (att.ranging (\textit{@atLeast}, \textit{@atMost}, \textit{@min}, \textit{@max}, \textit{@confidence})) )
    \item[{Member of}]
  model.choicePart model.pPart.editorial
    \item[{Contained by}]
  
    \item[analysis: ]
   cl pc phr s span w\par 
    \item[core: ]
   abbr add addrLine author bibl biblScope choice citedRange corr date del desc distinct editor email emph expan foreign gloss head headItem headLabel hi item l label measure meeting mentioned name note num orig p pubPlace publisher q quote ref reg resp rs said sic soCalled speaker stage street term textLang time title unclear\par 
    \item[figures: ]
   cell figDesc\par 
    \item[header: ]
   authority catDesc change classCode creation distributor edition extent funder geoDecl handNote language licence principal rendition scriptNote sponsor tagUsage typeNote\par 
    \item[linking: ]
   ab seg\par 
    \item[msdescription: ]
   accMat acquisition additions catchwords collation colophon condition custEvent decoNote explicit filiation finalRubric foliation heraldry incipit layout material musicNotation objectType origDate origPlace origin provenance rubric secFol signatures source stamp summary support surrogates watermark\par 
    \item[namesdates: ]
   addName affiliation age birth bloc country death district education faith floruit forename genName geogFeat geogName langKnown nameLink nationality occupation offset orgName persName placeName region residence roleName settlement sex socecStatus surname\par 
    \item[textcrit: ]
   lem rdg wit witDetail witness\par 
    \item[textstructure: ]
   byline closer dateline docAuthor docDate docEdition docImprint imprimatur opener salute signed titlePart trailer\par 
    \item[transcr: ]
   damage fw metamark mod restore retrace secl supplied surplus
    \item[{May contain}]
  
    \item[analysis: ]
   c cl interp interpGrp m pc phr s span spanGrp w\par 
    \item[core: ]
   abbr add address cb choice corr date del distinct email emph expan foreign gap gb gloss graphic hi index lb measure measureGrp media mentioned milestone name note num orig pb ptr ref reg rs sic soCalled term time title unclear\par 
    \item[figures: ]
   figure formula notatedMusic\par 
    \item[gaiji: ]
   g\par 
    \item[header: ]
   idno\par 
    \item[linking: ]
   alt altGrp anchor join joinGrp link linkGrp seg timeline\par 
    \item[msdescription: ]
   catchwords depth dim dimensions height heraldry locus locusGrp material objectType origDate origPlace secFol signatures stamp watermark width\par 
    \item[namesdates: ]
   addName affiliation bloc climate country district forename genName geo geogFeat geogName location nameLink offset orgName persName placeName population region roleName settlement state surname terrain trait\par 
    \item[textcrit: ]
   app witDetail\par 
    \item[transcr: ]
   addSpan am damage damageSpan delSpan ex fw handShift listTranspose metamark mod redo restore retrace secl space subst substJoin supplied surplus undo\par character data
    \item[{Note}]
  \par
The content of this element should be the expanded abbreviation, usually (but not always) a complete word or phrase. The <ex> element provided by the \textsf{transcr} module may be used to mark up sequences of letters supplied within such an expansion.
    \item[{Example}]
  \leavevmode\bgroup\exampleFont \begin{shaded}\noindent\mbox{}The address is Southmoor\mbox{}\newline 
{<\textbf{choice}>}\mbox{}\newline 
\hspace*{6pt}{<\textbf{expan}>}Road{</\textbf{expan}>}\mbox{}\newline 
\hspace*{6pt}{<\textbf{abbr}>}Rd{</\textbf{abbr}>}\mbox{}\newline 
{</\textbf{choice}>}\end{shaded}\egroup 


    \item[{Example}]
  \leavevmode\bgroup\exampleFont \begin{shaded}\noindent\mbox{}{<\textbf{choice}\hspace*{6pt}{xml:lang}="{la}">}\mbox{}\newline 
\hspace*{6pt}{<\textbf{abbr}>}Imp{</\textbf{abbr}>}\mbox{}\newline 
\hspace*{6pt}{<\textbf{expan}>}Imp{<\textbf{ex}>}erator{</\textbf{ex}>}\mbox{}\newline 
\hspace*{6pt}{</\textbf{expan}>}\mbox{}\newline 
{</\textbf{choice}>}\end{shaded}\egroup 


    \item[{Content model}]
  \mbox{}\hfill\\[-10pt]\begin{Verbatim}[fontsize=\small]
<content>
 <macroRef key="macro.phraseSeq"/>
</content>
    
\end{Verbatim}

    \item[{Schema Declaration}]
  \mbox{}\hfill\\[-10pt]\begin{Verbatim}[fontsize=\small]
element expan
{
   att.global.attributes,
   att.editLike.attributes,
   macro.phraseSeq}
\end{Verbatim}

\end{reflist}  \index{explicit=<explicit>|oddindex}
\begin{reflist}
\item[]\begin{specHead}{TEI.explicit}{<explicit> }contains the \textit{explicit} of a manuscript item, that is, the closing words of the text proper, exclusive of any rubric or colophon which might follow it. [\xref{http://www.tei-c.org/release/doc/tei-p5-doc/en/html/MS.html\#mscoit}{10.6.1. The msItem and msItemStruct Elements}]\end{specHead} 
    \item[{Module}]
  msdescription
    \item[{Attributes}]
  Attributes att.global (\textit{@xml:id}, \textit{@n}, \textit{@xml:lang}, \textit{@xml:base}, \textit{@xml:space})  (att.global.rendition (\textit{@rend}, \textit{@style}, \textit{@rendition})) (att.global.linking (\textit{@corresp}, \textit{@synch}, \textit{@sameAs}, \textit{@copyOf}, \textit{@next}, \textit{@prev}, \textit{@exclude}, \textit{@select})) (att.global.analytic (\textit{@ana})) (att.global.facs (\textit{@facs})) (att.global.change (\textit{@change})) (att.global.responsibility (\textit{@cert}, \textit{@resp})) (att.global.source (\textit{@source})) att.typed (\textit{@type}, \textit{@subtype}) att.msExcerpt (\textit{@defective}) 
    \item[{Member of}]
  model.msQuoteLike 
    \item[{Contained by}]
  
    \item[msdescription: ]
   msItem msItemStruct
    \item[{May contain}]
  
    \item[analysis: ]
   c cl interp interpGrp m pc phr s span spanGrp w\par 
    \item[core: ]
   abbr add address cb choice corr date del distinct email emph expan foreign gap gb gloss graphic hi index lb measure measureGrp media mentioned milestone name note num orig pb ptr ref reg rs sic soCalled term time title unclear\par 
    \item[figures: ]
   figure formula notatedMusic\par 
    \item[gaiji: ]
   g\par 
    \item[header: ]
   idno\par 
    \item[linking: ]
   alt altGrp anchor join joinGrp link linkGrp seg timeline\par 
    \item[msdescription: ]
   catchwords depth dim dimensions height heraldry locus locusGrp material objectType origDate origPlace secFol signatures stamp watermark width\par 
    \item[namesdates: ]
   addName affiliation bloc climate country district forename genName geo geogFeat geogName location nameLink offset orgName persName placeName population region roleName settlement state surname terrain trait\par 
    \item[textcrit: ]
   app witDetail\par 
    \item[transcr: ]
   addSpan am damage damageSpan delSpan ex fw handShift listTranspose metamark mod redo restore retrace secl space subst substJoin supplied surplus undo\par character data
    \item[{Example}]
  \leavevmode\bgroup\exampleFont \begin{shaded}\noindent\mbox{}{<\textbf{explicit}>}sed libera nos a malo.{</\textbf{explicit}>}\mbox{}\newline 
{<\textbf{rubric}>}Hic explicit oratio qui dicitur dominica.{</\textbf{rubric}>}\mbox{}\newline 
{<\textbf{explicit}\hspace*{6pt}{type}="{defective}">}ex materia quasi et forma sibi\mbox{}\newline 
 proporti{<\textbf{gap}/>}\mbox{}\newline 
{</\textbf{explicit}>}\mbox{}\newline 
{<\textbf{explicit}\hspace*{6pt}{type}="{reverse}">}saued be shulle that doome of day the at\mbox{}\newline 
{</\textbf{explicit}>}\end{shaded}\egroup 


    \item[{Content model}]
  \mbox{}\hfill\\[-10pt]\begin{Verbatim}[fontsize=\small]
<content>
 <macroRef key="macro.phraseSeq"/>
</content>
    
\end{Verbatim}

    \item[{Schema Declaration}]
  \mbox{}\hfill\\[-10pt]\begin{Verbatim}[fontsize=\small]
element explicit
{
   att.global.attributes,
   att.typed.attributes,
   att.msExcerpt.attributes,
   macro.phraseSeq}
\end{Verbatim}

\end{reflist}  \index{extent=<extent>|oddindex}
\begin{reflist}
\item[]\begin{specHead}{TEI.extent}{<extent> }describes the approximate size of a text stored on some carrier medium or of some other object, digital or non-digital, specified in any convenient units. [\xref{http://www.tei-c.org/release/doc/tei-p5-doc/en/html/HD.html\#HD23}{2.2.3. Type and Extent of File} \xref{http://www.tei-c.org/release/doc/tei-p5-doc/en/html/HD.html\#HD2}{2.2. The File Description} \xref{http://www.tei-c.org/release/doc/tei-p5-doc/en/html/CO.html\#COBICOI}{3.11.2.4. Imprint, Size of a Document, and Reprint Information} \xref{http://www.tei-c.org/release/doc/tei-p5-doc/en/html/MS.html\#msph1}{10.7.1. Object Description}]\end{specHead} 
    \item[{Module}]
  header
    \item[{Attributes}]
  Attributes att.global (\textit{@xml:id}, \textit{@n}, \textit{@xml:lang}, \textit{@xml:base}, \textit{@xml:space})  (att.global.rendition (\textit{@rend}, \textit{@style}, \textit{@rendition})) (att.global.linking (\textit{@corresp}, \textit{@synch}, \textit{@sameAs}, \textit{@copyOf}, \textit{@next}, \textit{@prev}, \textit{@exclude}, \textit{@select})) (att.global.analytic (\textit{@ana})) (att.global.facs (\textit{@facs})) (att.global.change (\textit{@change})) (att.global.responsibility (\textit{@cert}, \textit{@resp})) (att.global.source (\textit{@source}))
    \item[{Member of}]
  model.biblPart 
    \item[{Contained by}]
  
    \item[core: ]
   bibl monogr\par 
    \item[header: ]
   biblFull fileDesc\par 
    \item[msdescription: ]
   supportDesc
    \item[{May contain}]
  
    \item[analysis: ]
   c cl interp interpGrp m pc phr s span spanGrp w\par 
    \item[core: ]
   abbr add address cb choice corr date del distinct email emph expan foreign gap gb gloss graphic hi index lb measure measureGrp media mentioned milestone name note num orig pb ptr ref reg rs sic soCalled term time title unclear\par 
    \item[figures: ]
   figure formula notatedMusic\par 
    \item[gaiji: ]
   g\par 
    \item[header: ]
   idno\par 
    \item[linking: ]
   alt altGrp anchor join joinGrp link linkGrp seg timeline\par 
    \item[msdescription: ]
   catchwords depth dim dimensions height heraldry locus locusGrp material objectType origDate origPlace secFol signatures stamp watermark width\par 
    \item[namesdates: ]
   addName affiliation bloc climate country district forename genName geo geogFeat geogName location nameLink offset orgName persName placeName population region roleName settlement state surname terrain trait\par 
    \item[textcrit: ]
   app witDetail\par 
    \item[transcr: ]
   addSpan am damage damageSpan delSpan ex fw handShift listTranspose metamark mod redo restore retrace secl space subst substJoin supplied surplus undo\par character data
    \item[{Example}]
  \leavevmode\bgroup\exampleFont \begin{shaded}\noindent\mbox{}{<\textbf{extent}>}3200 sentences{</\textbf{extent}>}\mbox{}\newline 
{<\textbf{extent}>}between 10 and 20 Mb{</\textbf{extent}>}\mbox{}\newline 
{<\textbf{extent}>}ten 3.5 inch high density diskettes{</\textbf{extent}>}\end{shaded}\egroup 


    \item[{Example}]
  The <measure> element may be used to supply normalised or machine tractable versions of the size or sizes concerned.\leavevmode\bgroup\exampleFont \begin{shaded}\noindent\mbox{}{<\textbf{extent}>}\mbox{}\newline 
\hspace*{6pt}{<\textbf{measure}\hspace*{6pt}{quantity}="{4.2}"\hspace*{6pt}{unit}="{MiB}">}About four megabytes{</\textbf{measure}>}\mbox{}\newline 
\hspace*{6pt}{<\textbf{measure}\hspace*{6pt}{quantity}="{245}"\hspace*{6pt}{unit}="{pages}">}245 pages of source\mbox{}\newline 
\hspace*{6pt}\hspace*{6pt} material{</\textbf{measure}>}\mbox{}\newline 
{</\textbf{extent}>}\end{shaded}\egroup 


    \item[{Content model}]
  \mbox{}\hfill\\[-10pt]\begin{Verbatim}[fontsize=\small]
<content>
 <macroRef key="macro.phraseSeq"/>
</content>
    
\end{Verbatim}

    \item[{Schema Declaration}]
  \mbox{}\hfill\\[-10pt]\begin{Verbatim}[fontsize=\small]
element extent { att.global.attributes, macro.phraseSeq }
\end{Verbatim}

\end{reflist}  \index{facsimile=<facsimile>|oddindex}
\begin{reflist}
\item[]\begin{specHead}{TEI.facsimile}{<facsimile> }contains a representation of some written source in the form of a set of images rather than as transcribed or encoded text. [\xref{http://www.tei-c.org/release/doc/tei-p5-doc/en/html/PH.html\#PHFAX}{11.1. Digital Facsimiles}]\end{specHead} 
    \item[{Module}]
  transcr
    \item[{Attributes}]
  Attributes att.global (\textit{@xml:id}, \textit{@n}, \textit{@xml:lang}, \textit{@xml:base}, \textit{@xml:space})  (att.global.rendition (\textit{@rend}, \textit{@style}, \textit{@rendition})) (att.global.linking (\textit{@corresp}, \textit{@synch}, \textit{@sameAs}, \textit{@copyOf}, \textit{@next}, \textit{@prev}, \textit{@exclude}, \textit{@select})) (att.global.analytic (\textit{@ana})) (att.global.facs (\textit{@facs})) (att.global.change (\textit{@change})) (att.global.responsibility (\textit{@cert}, \textit{@resp})) (att.global.source (\textit{@source})) att.declaring (\textit{@decls}) 
    \item[{Member of}]
  model.resourceLike
    \item[{Contained by}]
  
    \item[core: ]
   teiCorpus\par 
    \item[textstructure: ]
   TEI
    \item[{May contain}]
  
    \item[core: ]
   graphic media\par 
    \item[figures: ]
   formula\par 
    \item[textstructure: ]
   back front\par 
    \item[transcr: ]
   surface surfaceGrp
    \item[{Example}]
  \leavevmode\bgroup\exampleFont \begin{shaded}\noindent\mbox{}{<\textbf{facsimile}>}\mbox{}\newline 
\hspace*{6pt}{<\textbf{graphic}\hspace*{6pt}{url}="{page1.png}"/>}\mbox{}\newline 
\hspace*{6pt}{<\textbf{surface}>}\mbox{}\newline 
\hspace*{6pt}\hspace*{6pt}{<\textbf{graphic}\hspace*{6pt}{url}="{page2-highRes.png}"/>}\mbox{}\newline 
\hspace*{6pt}\hspace*{6pt}{<\textbf{graphic}\hspace*{6pt}{url}="{page2-lowRes.png}"/>}\mbox{}\newline 
\hspace*{6pt}{</\textbf{surface}>}\mbox{}\newline 
\hspace*{6pt}{<\textbf{graphic}\hspace*{6pt}{url}="{page3.png}"/>}\mbox{}\newline 
\hspace*{6pt}{<\textbf{graphic}\hspace*{6pt}{url}="{page4.png}"/>}\mbox{}\newline 
{</\textbf{facsimile}>}\end{shaded}\egroup 


    \item[{Example}]
  \leavevmode\bgroup\exampleFont \begin{shaded}\noindent\mbox{}{<\textbf{facsimile}>}\mbox{}\newline 
\hspace*{6pt}{<\textbf{surface}\hspace*{6pt}{lrx}="{200}"\hspace*{6pt}{lry}="{300}"\hspace*{6pt}{ulx}="{0}"\hspace*{6pt}{uly}="{0}">}\mbox{}\newline 
\hspace*{6pt}\hspace*{6pt}{<\textbf{graphic}\hspace*{6pt}{url}="{Bovelles-49r.png}"/>}\mbox{}\newline 
\hspace*{6pt}{</\textbf{surface}>}\mbox{}\newline 
{</\textbf{facsimile}>}\end{shaded}\egroup 


    \item[{Content model}]
  \mbox{}\hfill\\[-10pt]\begin{Verbatim}[fontsize=\small]
<content>
 <sequence>
  <elementRef key="front" minOccurs="0"/>
  <alternate maxOccurs="unbounded"
   minOccurs="1">
   <classRef key="model.graphicLike"/>
   <elementRef key="surface"/>
   <elementRef key="surfaceGrp"/>
  </alternate>
  <elementRef key="back" minOccurs="0"/>
 </sequence>
</content>
    
\end{Verbatim}

    \item[{Schema Declaration}]
  \mbox{}\hfill\\[-10pt]\begin{Verbatim}[fontsize=\small]
element facsimile
{
   att.global.attributes,
   att.declaring.attributes,
   ( front?, ( model.graphicLike | surface | surfaceGrp )+, back? )
}
\end{Verbatim}

\end{reflist}  \index{faith=<faith>|oddindex}
\begin{reflist}
\item[]\begin{specHead}{TEI.faith}{<faith> }specifies the faith, religion, or belief set of a person. [\xref{http://www.tei-c.org/release/doc/tei-p5-doc/en/html/ND.html\#NDPERSEpc}{13.3.2.1. Personal Characteristics}]\end{specHead} 
    \item[{Module}]
  namesdates
    \item[{Attributes}]
  Attributes att.global (\textit{@xml:id}, \textit{@n}, \textit{@xml:lang}, \textit{@xml:base}, \textit{@xml:space})  (att.global.rendition (\textit{@rend}, \textit{@style}, \textit{@rendition})) (att.global.linking (\textit{@corresp}, \textit{@synch}, \textit{@sameAs}, \textit{@copyOf}, \textit{@next}, \textit{@prev}, \textit{@exclude}, \textit{@select})) (att.global.analytic (\textit{@ana})) (att.global.facs (\textit{@facs})) (att.global.change (\textit{@change})) (att.global.responsibility (\textit{@cert}, \textit{@resp})) (att.global.source (\textit{@source})) att.editLike (\textit{@evidence}, \textit{@instant})  (att.dimensions (\textit{@unit}, \textit{@quantity}, \textit{@extent}, \textit{@precision}, \textit{@scope}) (att.ranging (\textit{@atLeast}, \textit{@atMost}, \textit{@min}, \textit{@max}, \textit{@confidence})) ) att.datable (\textit{@calendar}, \textit{@period})  (att.datable.w3c (\textit{@when}, \textit{@notBefore}, \textit{@notAfter}, \textit{@from}, \textit{@to})) (att.datable.iso (\textit{@when-iso}, \textit{@notBefore-iso}, \textit{@notAfter-iso}, \textit{@from-iso}, \textit{@to-iso})) (att.datable.custom (\textit{@when-custom}, \textit{@notBefore-custom}, \textit{@notAfter-custom}, \textit{@from-custom}, \textit{@to-custom}, \textit{@datingPoint}, \textit{@datingMethod})) att.canonical (\textit{@key}, \textit{@ref}) 
    \item[{Member of}]
  model.persStateLike
    \item[{Contained by}]
  
    \item[namesdates: ]
   person personGrp
    \item[{May contain}]
  
    \item[analysis: ]
   c cl interp interpGrp m pc phr s span spanGrp w\par 
    \item[core: ]
   abbr add address cb choice corr date del distinct email emph expan foreign gap gb gloss graphic hi index lb measure measureGrp media mentioned milestone name note num orig pb ptr ref reg rs sic soCalled term time title unclear\par 
    \item[figures: ]
   figure formula notatedMusic\par 
    \item[gaiji: ]
   g\par 
    \item[header: ]
   idno\par 
    \item[linking: ]
   alt altGrp anchor join joinGrp link linkGrp seg timeline\par 
    \item[msdescription: ]
   catchwords depth dim dimensions height heraldry locus locusGrp material objectType origDate origPlace secFol signatures stamp watermark width\par 
    \item[namesdates: ]
   addName affiliation bloc climate country district forename genName geo geogFeat geogName location nameLink offset orgName persName placeName population region roleName settlement state surname terrain trait\par 
    \item[textcrit: ]
   app witDetail\par 
    \item[transcr: ]
   addSpan am damage damageSpan delSpan ex fw handShift listTranspose metamark mod redo restore retrace secl space subst substJoin supplied surplus undo\par character data
    \item[{Example}]
  \leavevmode\bgroup\exampleFont \begin{shaded}\noindent\mbox{}{<\textbf{faith}>}protestant{</\textbf{faith}>}\end{shaded}\egroup 


    \item[{Example}]
  \leavevmode\bgroup\exampleFont \begin{shaded}\noindent\mbox{}{<\textbf{faith}\hspace*{6pt}{ref}="{http://dbpedia.org/page/Manichaeism}">}Manichaeism{</\textbf{faith}>}\end{shaded}\egroup 


    \item[{Content model}]
  \mbox{}\hfill\\[-10pt]\begin{Verbatim}[fontsize=\small]
<content>
 <macroRef key="macro.phraseSeq"/>
</content>
    
\end{Verbatim}

    \item[{Schema Declaration}]
  \mbox{}\hfill\\[-10pt]\begin{Verbatim}[fontsize=\small]
element faith
{
   att.global.attributes,
   att.editLike.attributes,
   att.datable.attributes,
   att.canonical.attributes,
   macro.phraseSeq}
\end{Verbatim}

\end{reflist}  \index{figDesc=<figDesc>|oddindex}
\begin{reflist}
\item[]\begin{specHead}{TEI.figDesc}{<figDesc> }(description of figure) contains a brief prose description of the appearance or content of a graphic figure, for use when documenting an image without displaying it. [\xref{http://www.tei-c.org/release/doc/tei-p5-doc/en/html/FT.html\#FTGRA}{14.4. Specific Elements for Graphic Images}]\end{specHead} 
    \item[{Module}]
  figures
    \item[{Attributes}]
  Attributes att.global (\textit{@xml:id}, \textit{@n}, \textit{@xml:lang}, \textit{@xml:base}, \textit{@xml:space})  (att.global.rendition (\textit{@rend}, \textit{@style}, \textit{@rendition})) (att.global.linking (\textit{@corresp}, \textit{@synch}, \textit{@sameAs}, \textit{@copyOf}, \textit{@next}, \textit{@prev}, \textit{@exclude}, \textit{@select})) (att.global.analytic (\textit{@ana})) (att.global.facs (\textit{@facs})) (att.global.change (\textit{@change})) (att.global.responsibility (\textit{@cert}, \textit{@resp})) (att.global.source (\textit{@source}))
    \item[{Contained by}]
  
    \item[figures: ]
   figure
    \item[{May contain}]
  
    \item[core: ]
   abbr address bibl biblStruct choice cit date desc distinct email emph expan foreign gloss hi label list listBibl measure measureGrp mentioned name num ptr q quote ref rs said soCalled stage term time title\par 
    \item[figures: ]
   table\par 
    \item[header: ]
   biblFull idno\par 
    \item[msdescription: ]
   catchwords depth dim dimensions height heraldry locus locusGrp material msDesc objectType origDate origPlace secFol signatures stamp watermark width\par 
    \item[namesdates: ]
   addName affiliation bloc climate country district forename genName geo geogFeat geogName listEvent listNym listOrg listPerson listPlace location nameLink offset orgName persName placeName population region roleName settlement state surname terrain trait\par 
    \item[textcrit: ]
   listApp listWit\par 
    \item[textstructure: ]
   floatingText\par 
    \item[transcr: ]
   am ex subst\par character data
    \item[{Note}]
  \par
This element is intended for use as an alternative to the content of its parent <figure> element ; for example, to display when the image is required but the equipment in use cannot display graphic images. It may also be used for indexing or documentary purposes.
    \item[{Example}]
  \leavevmode\bgroup\exampleFont \begin{shaded}\noindent\mbox{}{<\textbf{figure}>}\mbox{}\newline 
\hspace*{6pt}{<\textbf{graphic}\hspace*{6pt}{url}="{emblem1.png}"/>}\mbox{}\newline 
\hspace*{6pt}{<\textbf{head}>}Emblemi d'Amore{</\textbf{head}>}\mbox{}\newline 
\hspace*{6pt}{<\textbf{figDesc}>}A pair of naked winged cupids, each holding a\mbox{}\newline 
\hspace*{6pt}\hspace*{6pt} flaming torch, in a rural setting.{</\textbf{figDesc}>}\mbox{}\newline 
{</\textbf{figure}>}\end{shaded}\egroup 


    \item[{Content model}]
  \mbox{}\hfill\\[-10pt]\begin{Verbatim}[fontsize=\small]
<content>
 <macroRef key="macro.limitedContent"/>
</content>
    
\end{Verbatim}

    \item[{Schema Declaration}]
  \mbox{}\hfill\\[-10pt]\begin{Verbatim}[fontsize=\small]
element figDesc { att.global.attributes, macro.limitedContent }
\end{Verbatim}

\end{reflist}  \index{figure=<figure>|oddindex}
\begin{reflist}
\item[]\begin{specHead}{TEI.figure}{<figure> }groups elements representing or containing graphic information such as an illustration, formula, or figure. [\xref{http://www.tei-c.org/release/doc/tei-p5-doc/en/html/FT.html\#FTGRA}{14.4. Specific Elements for Graphic Images}]\end{specHead} 
    \item[{Module}]
  figures
    \item[{Attributes}]
  Attributes att.global (\textit{@xml:id}, \textit{@n}, \textit{@xml:lang}, \textit{@xml:base}, \textit{@xml:space})  (att.global.rendition (\textit{@rend}, \textit{@style}, \textit{@rendition})) (att.global.linking (\textit{@corresp}, \textit{@synch}, \textit{@sameAs}, \textit{@copyOf}, \textit{@next}, \textit{@prev}, \textit{@exclude}, \textit{@select})) (att.global.analytic (\textit{@ana})) (att.global.facs (\textit{@facs})) (att.global.change (\textit{@change})) (att.global.responsibility (\textit{@cert}, \textit{@resp})) (att.global.source (\textit{@source})) att.placement (\textit{@place}) att.typed (\textit{@type}, \textit{@subtype}) 
    \item[{Member of}]
  model.global
    \item[{Contained by}]
  
    \item[analysis: ]
   cl m phr s span w\par 
    \item[core: ]
   abbr add addrLine address author bibl biblScope cit citedRange corr date del distinct editor email emph expan foreign gloss head headItem headLabel hi imprint item l label lg list measure mentioned name note num orig p pubPlace publisher q quote ref reg resp rs said series sic soCalled sp speaker stage street term textLang time title unclear\par 
    \item[figures: ]
   cell figure table\par 
    \item[gaiji: ]
   char glyph\par 
    \item[header: ]
   authority change classCode distributor edition extent funder geoDecl handNote language licence principal scriptNote sponsor typeNote\par 
    \item[linking: ]
   ab seg\par 
    \item[msdescription: ]
   accMat acquisition additions catchwords collation colophon condition custEvent decoNote explicit filiation finalRubric foliation heraldry incipit layout material msItem musicNotation objectType origDate origPlace origin provenance rubric secFol signatures source stamp summary support surrogates watermark\par 
    \item[namesdates: ]
   addName affiliation age birth bloc country death district education faith floruit forename genName geogFeat geogName langKnown nameLink nationality occupation offset orgName persName person personGrp placeName region residence roleName settlement sex socecStatus surname\par 
    \item[textcrit: ]
   lem rdg wit witDetail\par 
    \item[textstructure: ]
   argument back body byline closer dateline div docAuthor docDate docEdition docImprint docTitle epigraph floatingText front group imprimatur opener postscript salute signed text titlePage titlePart trailer\par 
    \item[transcr: ]
   damage fw line metamark mod restore retrace secl sourceDoc supplied surface surfaceGrp surplus zone
    \item[{May contain}]
  
    \item[analysis: ]
   interp interpGrp span spanGrp\par 
    \item[core: ]
   bibl biblStruct cb cit desc gap gb graphic head index l label lb lg list listBibl media meeting milestone note p pb q quote said sp stage\par 
    \item[figures: ]
   figDesc figure formula notatedMusic table\par 
    \item[header: ]
   biblFull\par 
    \item[linking: ]
   ab alt altGrp anchor join joinGrp link linkGrp timeline\par 
    \item[msdescription: ]
   msDesc\par 
    \item[namesdates: ]
   listEvent listNym listOrg listPerson listPlace\par 
    \item[textcrit: ]
   app listApp listWit witDetail\par 
    \item[textstructure: ]
   argument byline closer dateline docAuthor docDate epigraph floatingText postscript salute signed trailer\par 
    \item[transcr: ]
   addSpan damageSpan delSpan fw listTranspose metamark space substJoin
    \item[{Example}]
  \leavevmode\bgroup\exampleFont \begin{shaded}\noindent\mbox{}{<\textbf{figure}>}\mbox{}\newline 
\hspace*{6pt}{<\textbf{head}>}The View from the Bridge{</\textbf{head}>}\mbox{}\newline 
\hspace*{6pt}{<\textbf{figDesc}>}A Whistleresque view showing four or five sailing boats in the foreground, and a\mbox{}\newline 
\hspace*{6pt}\hspace*{6pt} series of buoys strung out between them.{</\textbf{figDesc}>}\mbox{}\newline 
\hspace*{6pt}{<\textbf{graphic}\hspace*{6pt}{scale}="{0.5}"\mbox{}\newline 
\hspace*{6pt}\hspace*{6pt}{url}="{http://www.example.org/fig1.png}"/>}\mbox{}\newline 
{</\textbf{figure}>}\end{shaded}\egroup 


    \item[{Content model}]
  \mbox{}\hfill\\[-10pt]\begin{Verbatim}[fontsize=\small]
<content>
 <alternate maxOccurs="unbounded"
  minOccurs="0">
  <classRef key="model.headLike"/>
  <classRef key="model.common"/>
  <elementRef key="figDesc"/>
  <classRef key="model.graphicLike"/>
  <classRef key="model.global"/>
  <classRef key="model.divBottom"/>
 </alternate>
</content>
    
\end{Verbatim}

    \item[{Schema Declaration}]
  \mbox{}\hfill\\[-10pt]\begin{Verbatim}[fontsize=\small]
element figure
{
   att.global.attributes,
   att.placement.attributes,
   att.typed.attributes,
   (
      model.headLike    | model.common    | figDesc    | model.graphicLike    | model.global    | model.divBottom   )*
}
\end{Verbatim}

\end{reflist}  \index{fileDesc=<fileDesc>|oddindex}
\begin{reflist}
\item[]\begin{specHead}{TEI.fileDesc}{<fileDesc> }(file description) contains a full bibliographic description of an electronic file. [\xref{http://www.tei-c.org/release/doc/tei-p5-doc/en/html/HD.html\#HD2}{2.2. The File Description} \xref{http://www.tei-c.org/release/doc/tei-p5-doc/en/html/HD.html\#HD11}{2.1.1. The TEI Header and Its Components}]\end{specHead} 
    \item[{Module}]
  header
    \item[{Attributes}]
  Attributes att.global (\textit{@xml:id}, \textit{@n}, \textit{@xml:lang}, \textit{@xml:base}, \textit{@xml:space})  (att.global.rendition (\textit{@rend}, \textit{@style}, \textit{@rendition})) (att.global.linking (\textit{@corresp}, \textit{@synch}, \textit{@sameAs}, \textit{@copyOf}, \textit{@next}, \textit{@prev}, \textit{@exclude}, \textit{@select})) (att.global.analytic (\textit{@ana})) (att.global.facs (\textit{@facs})) (att.global.change (\textit{@change})) (att.global.responsibility (\textit{@cert}, \textit{@resp})) (att.global.source (\textit{@source}))
    \item[{Contained by}]
  
    \item[header: ]
   biblFull teiHeader
    \item[{May contain}]
  
    \item[header: ]
   editionStmt extent notesStmt publicationStmt seriesStmt sourceDesc titleStmt
    \item[{Note}]
  \par
The major source of information for those seeking to create a catalogue entry or bibliographic citation for an electronic file. As such, it provides a title and statements of responsibility together with details of the publication or distribution of the file, of any series to which it belongs, and detailed bibliographic notes for matters not addressed elsewhere in the header. It also contains a full bibliographic description for the source or sources from which the electronic text was derived.
    \item[{Example}]
  \leavevmode\bgroup\exampleFont \begin{shaded}\noindent\mbox{}{<\textbf{fileDesc}>}\mbox{}\newline 
\hspace*{6pt}{<\textbf{titleStmt}>}\mbox{}\newline 
\hspace*{6pt}\hspace*{6pt}{<\textbf{title}>}The shortest possible TEI document{</\textbf{title}>}\mbox{}\newline 
\hspace*{6pt}{</\textbf{titleStmt}>}\mbox{}\newline 
\hspace*{6pt}{<\textbf{publicationStmt}>}\mbox{}\newline 
\hspace*{6pt}\hspace*{6pt}{<\textbf{p}>}Distributed as part of TEI P5{</\textbf{p}>}\mbox{}\newline 
\hspace*{6pt}{</\textbf{publicationStmt}>}\mbox{}\newline 
\hspace*{6pt}{<\textbf{sourceDesc}>}\mbox{}\newline 
\hspace*{6pt}\hspace*{6pt}{<\textbf{p}>}No print source exists: this is an original digital text{</\textbf{p}>}\mbox{}\newline 
\hspace*{6pt}{</\textbf{sourceDesc}>}\mbox{}\newline 
{</\textbf{fileDesc}>}\end{shaded}\egroup 


    \item[{Content model}]
  \mbox{}\hfill\\[-10pt]\begin{Verbatim}[fontsize=\small]
<content>
 <sequence>
  <sequence>
   <elementRef key="titleStmt"/>
   <elementRef key="editionStmt"
    minOccurs="0"/>
   <elementRef key="extent" minOccurs="0"/>
   <elementRef key="publicationStmt"/>
   <elementRef key="seriesStmt"
    minOccurs="0"/>
   <elementRef key="notesStmt"
    minOccurs="0"/>
  </sequence>
  <elementRef key="sourceDesc"
   maxOccurs="unbounded" minOccurs="1"/>
 </sequence>
</content>
    
\end{Verbatim}

    \item[{Schema Declaration}]
  \mbox{}\hfill\\[-10pt]\begin{Verbatim}[fontsize=\small]
element fileDesc
{
   att.global.attributes,
   (
      (
         titleStmt,
         editionStmt?,
         extent?,
         publicationStmt,
         seriesStmt?,
         notesStmt?
      ),
      sourceDesc+
   )
}
\end{Verbatim}

\end{reflist}  \index{filiation=<filiation>|oddindex}
\begin{reflist}
\item[]\begin{specHead}{TEI.filiation}{<filiation> }contains information concerning the manuscript's \textit{filiation}, i.e. its relationship to other surviving manuscripts of the same text, its \textit{protographs}, \textit{antigraphs} and \textit{apographs}. [\xref{http://www.tei-c.org/release/doc/tei-p5-doc/en/html/MS.html\#mscoit}{10.6.1. The msItem and msItemStruct Elements}]\end{specHead} 
    \item[{Module}]
  msdescription
    \item[{Attributes}]
  Attributes att.global (\textit{@xml:id}, \textit{@n}, \textit{@xml:lang}, \textit{@xml:base}, \textit{@xml:space})  (att.global.rendition (\textit{@rend}, \textit{@style}, \textit{@rendition})) (att.global.linking (\textit{@corresp}, \textit{@synch}, \textit{@sameAs}, \textit{@copyOf}, \textit{@next}, \textit{@prev}, \textit{@exclude}, \textit{@select})) (att.global.analytic (\textit{@ana})) (att.global.facs (\textit{@facs})) (att.global.change (\textit{@change})) (att.global.responsibility (\textit{@cert}, \textit{@resp})) (att.global.source (\textit{@source})) att.typed (\textit{@type}, \textit{@subtype}) 
    \item[{Member of}]
  model.msItemPart 
    \item[{Contained by}]
  
    \item[msdescription: ]
   msItem msItemStruct
    \item[{May contain}]
  
    \item[analysis: ]
   c cl interp interpGrp m pc phr s span spanGrp w\par 
    \item[core: ]
   abbr add address bibl biblStruct cb choice cit corr date del desc distinct email emph expan foreign gap gb gloss graphic hi index l label lb lg list listBibl measure measureGrp media mentioned milestone name note num orig p pb ptr q quote ref reg rs said sic soCalled sp stage term time title unclear\par 
    \item[figures: ]
   figure formula notatedMusic table\par 
    \item[gaiji: ]
   g\par 
    \item[header: ]
   biblFull idno\par 
    \item[linking: ]
   ab alt altGrp anchor join joinGrp link linkGrp seg timeline\par 
    \item[msdescription: ]
   catchwords depth dim dimensions height heraldry locus locusGrp material msDesc objectType origDate origPlace secFol signatures stamp watermark width\par 
    \item[namesdates: ]
   addName affiliation bloc climate country district forename genName geo geogFeat geogName listEvent listNym listOrg listPerson listPlace location nameLink offset orgName persName placeName population region roleName settlement state surname terrain trait\par 
    \item[textcrit: ]
   app listApp listWit witDetail\par 
    \item[textstructure: ]
   floatingText\par 
    \item[transcr: ]
   addSpan am damage damageSpan delSpan ex fw handShift listTranspose metamark mod redo restore retrace secl space subst substJoin supplied surplus undo\par character data
    \item[{Example}]
  \leavevmode\bgroup\exampleFont \begin{shaded}\noindent\mbox{}{<\textbf{msContents}>}\mbox{}\newline 
\hspace*{6pt}{<\textbf{msItem}>}\mbox{}\newline 
\hspace*{6pt}\hspace*{6pt}{<\textbf{title}>}Beljakovski sbornik{</\textbf{title}>}\mbox{}\newline 
\hspace*{6pt}\hspace*{6pt}{<\textbf{filiation}\hspace*{6pt}{type}="{protograph}">}Bulgarian{</\textbf{filiation}>}\mbox{}\newline 
\hspace*{6pt}\hspace*{6pt}{<\textbf{filiation}\hspace*{6pt}{type}="{antigraph}">}Middle Bulgarian{</\textbf{filiation}>}\mbox{}\newline 
\hspace*{6pt}\hspace*{6pt}{<\textbf{filiation}\hspace*{6pt}{type}="{apograph}">}\mbox{}\newline 
\hspace*{6pt}\hspace*{6pt}\hspace*{6pt}{<\textbf{ref}\hspace*{6pt}{target}="{\#DN17}">}Dujchev N 17{</\textbf{ref}>}\mbox{}\newline 
\hspace*{6pt}\hspace*{6pt}{</\textbf{filiation}>}\mbox{}\newline 
\hspace*{6pt}{</\textbf{msItem}>}\mbox{}\newline 
{</\textbf{msContents}>}\mbox{}\newline 
\textit{<!-- ... -->}\mbox{}\newline 
{<\textbf{msDesc}\hspace*{6pt}{xml:id}="{DN17}">}\mbox{}\newline 
\hspace*{6pt}{<\textbf{msIdentifier}>}\mbox{}\newline 
\hspace*{6pt}\hspace*{6pt}{<\textbf{settlement}>}Faraway{</\textbf{settlement}>}\mbox{}\newline 
\hspace*{6pt}{</\textbf{msIdentifier}>}\mbox{}\newline 
\textit{<!-- ... -->}\mbox{}\newline 
{</\textbf{msDesc}>}\end{shaded}\egroup 

In this example, the reference to ‘Dujchev N17’ includes a link to some other manuscript description which has the identifier \texttt{DN17}.
    \item[{Example}]
  \leavevmode\bgroup\exampleFont \begin{shaded}\noindent\mbox{}{<\textbf{msItem}>}\mbox{}\newline 
\hspace*{6pt}{<\textbf{title}>}Guan-ben{</\textbf{title}>}\mbox{}\newline 
\hspace*{6pt}{<\textbf{filiation}>}\mbox{}\newline 
\hspace*{6pt}\hspace*{6pt}{<\textbf{p}>}The "Guan-ben" was widely current among mathematicians in the\mbox{}\newline 
\hspace*{6pt}\hspace*{6pt}\hspace*{6pt}\hspace*{6pt} Qing dynasty, and "Zhao Qimei version" was also read. It is\mbox{}\newline 
\hspace*{6pt}\hspace*{6pt}\hspace*{6pt}\hspace*{6pt} therefore difficult to know the correct filiation path to follow.\mbox{}\newline 
\hspace*{6pt}\hspace*{6pt}\hspace*{6pt}\hspace*{6pt} The study of this era is much indebted to Li Di. We explain the\mbox{}\newline 
\hspace*{6pt}\hspace*{6pt}\hspace*{6pt}\hspace*{6pt} outline of his conclusion here. Kong Guangsen\mbox{}\newline 
\hspace*{6pt}\hspace*{6pt}\hspace*{6pt}\hspace*{6pt} (1752-1786)(17) was from the same town as Dai Zhen, so he obtained\mbox{}\newline 
\hspace*{6pt}\hspace*{6pt}\hspace*{6pt}\hspace*{6pt} "Guan-ben" from him and studied it(18). Li Huang (d. 1811)\mbox{}\newline 
\hspace*{6pt}\hspace*{6pt}\hspace*{6pt}\hspace*{6pt} (19) took part in editing Si Ku Quan Shu, so he must have had\mbox{}\newline 
\hspace*{6pt}\hspace*{6pt}\hspace*{6pt}\hspace*{6pt} "Guan-ben". Then Zhang Dunren (1754-1834) obtained this version,\mbox{}\newline 
\hspace*{6pt}\hspace*{6pt}\hspace*{6pt}\hspace*{6pt} and studied "Da Yan Zong Shu Shu" (The General Dayan\mbox{}\newline 
\hspace*{6pt}\hspace*{6pt}\hspace*{6pt}\hspace*{6pt} Computation). He wrote Jiu Yi Suan Shu (Mathematics\mbox{}\newline 
\hspace*{6pt}\hspace*{6pt}\hspace*{6pt}\hspace*{6pt} Searching for One, 1803) based on this version of Shu Xue Jiu\mbox{}\newline 
\hspace*{6pt}\hspace*{6pt}\hspace*{6pt}\hspace*{6pt} Zhang (20).{</\textbf{p}>}\mbox{}\newline 
\hspace*{6pt}\hspace*{6pt}{<\textbf{p}>}One of the most important persons in restoring our knowledge\mbox{}\newline 
\hspace*{6pt}\hspace*{6pt}\hspace*{6pt}\hspace*{6pt} concerning the filiation of these books was Li Rui (1768(21)\mbox{}\newline 
\hspace*{6pt}\hspace*{6pt}\hspace*{6pt}\hspace*{6pt} -1817)(see his biography). ... only two volumes remain of this\mbox{}\newline 
\hspace*{6pt}\hspace*{6pt}\hspace*{6pt}\hspace*{6pt} manuscript, as far as chapter 6 (chapter 3 part 2) p.13, that is,\mbox{}\newline 
\hspace*{6pt}\hspace*{6pt}\hspace*{6pt}\hspace*{6pt} question 2 of "Huan Tian San Ji" (square of three loops),\mbox{}\newline 
\hspace*{6pt}\hspace*{6pt}\hspace*{6pt}\hspace*{6pt} which later has been lost.{</\textbf{p}>}\mbox{}\newline 
\hspace*{6pt}{</\textbf{filiation}>}\mbox{}\newline 
{</\textbf{msItem}>}\mbox{}\newline 
\textit{<!--http://www2.nkfust.edu.tw/\textasciitilde jochi/ed1.htm-->}\end{shaded}\egroup 


    \item[{Content model}]
  \mbox{}\hfill\\[-10pt]\begin{Verbatim}[fontsize=\small]
<content>
 <macroRef key="macro.specialPara"/>
</content>
    
\end{Verbatim}

    \item[{Schema Declaration}]
  \mbox{}\hfill\\[-10pt]\begin{Verbatim}[fontsize=\small]
element filiation
{
   att.global.attributes,
   att.typed.attributes,
   macro.specialPara}
\end{Verbatim}

\end{reflist}  \index{finalRubric=<finalRubric>|oddindex}
\begin{reflist}
\item[]\begin{specHead}{TEI.finalRubric}{<finalRubric> }contains the string of words that denotes the end of a text division, often with an assertion as to its author and title, usually set off from the text itself by red ink, by a different size or type of script, or by some other such visual device. [\xref{http://www.tei-c.org/release/doc/tei-p5-doc/en/html/MS.html\#mscoit}{10.6.1. The msItem and msItemStruct Elements}]\end{specHead} 
    \item[{Module}]
  msdescription
    \item[{Attributes}]
  Attributes att.global (\textit{@xml:id}, \textit{@n}, \textit{@xml:lang}, \textit{@xml:base}, \textit{@xml:space})  (att.global.rendition (\textit{@rend}, \textit{@style}, \textit{@rendition})) (att.global.linking (\textit{@corresp}, \textit{@synch}, \textit{@sameAs}, \textit{@copyOf}, \textit{@next}, \textit{@prev}, \textit{@exclude}, \textit{@select})) (att.global.analytic (\textit{@ana})) (att.global.facs (\textit{@facs})) (att.global.change (\textit{@change})) (att.global.responsibility (\textit{@cert}, \textit{@resp})) (att.global.source (\textit{@source})) att.typed (\textit{@type}, \textit{@subtype}) 
    \item[{Member of}]
  model.msQuoteLike 
    \item[{Contained by}]
  
    \item[msdescription: ]
   msItem msItemStruct
    \item[{May contain}]
  
    \item[analysis: ]
   c cl interp interpGrp m pc phr s span spanGrp w\par 
    \item[core: ]
   abbr add address cb choice corr date del distinct email emph expan foreign gap gb gloss graphic hi index lb measure measureGrp media mentioned milestone name note num orig pb ptr ref reg rs sic soCalled term time title unclear\par 
    \item[figures: ]
   figure formula notatedMusic\par 
    \item[gaiji: ]
   g\par 
    \item[header: ]
   idno\par 
    \item[linking: ]
   alt altGrp anchor join joinGrp link linkGrp seg timeline\par 
    \item[msdescription: ]
   catchwords depth dim dimensions height heraldry locus locusGrp material objectType origDate origPlace secFol signatures stamp watermark width\par 
    \item[namesdates: ]
   addName affiliation bloc climate country district forename genName geo geogFeat geogName location nameLink offset orgName persName placeName population region roleName settlement state surname terrain trait\par 
    \item[textcrit: ]
   app witDetail\par 
    \item[transcr: ]
   addSpan am damage damageSpan delSpan ex fw handShift listTranspose metamark mod redo restore retrace secl space subst substJoin supplied surplus undo\par character data
    \item[{Example}]
  \leavevmode\bgroup\exampleFont \begin{shaded}\noindent\mbox{}{<\textbf{finalRubric}>}Explicit le romans de la Rose ou l'art\mbox{}\newline 
 d'amours est toute enclose.{</\textbf{finalRubric}>}\mbox{}\newline 
{<\textbf{finalRubric}>}ok lúkv ver þar Brennu-Nials savgv{</\textbf{finalRubric}>}\end{shaded}\egroup 


    \item[{Content model}]
  \mbox{}\hfill\\[-10pt]\begin{Verbatim}[fontsize=\small]
<content>
 <macroRef key="macro.phraseSeq"/>
</content>
    
\end{Verbatim}

    \item[{Schema Declaration}]
  \mbox{}\hfill\\[-10pt]\begin{Verbatim}[fontsize=\small]
element finalRubric
{
   att.global.attributes,
   att.typed.attributes,
   macro.phraseSeq}
\end{Verbatim}

\end{reflist}  \index{floatingText=<floatingText>|oddindex}
\begin{reflist}
\item[]\begin{specHead}{TEI.floatingText}{<floatingText> }contains a single text of any kind, whether unitary or composite, which interrupts the text containing it at any point and after which the surrounding text resumes. [\xref{http://www.tei-c.org/release/doc/tei-p5-doc/en/html/DS.html\#DSFLT}{4.3.2. Floating Texts}]\end{specHead} 
    \item[{Module}]
  textstructure
    \item[{Attributes}]
  Attributes att.global (\textit{@xml:id}, \textit{@n}, \textit{@xml:lang}, \textit{@xml:base}, \textit{@xml:space})  (att.global.rendition (\textit{@rend}, \textit{@style}, \textit{@rendition})) (att.global.linking (\textit{@corresp}, \textit{@synch}, \textit{@sameAs}, \textit{@copyOf}, \textit{@next}, \textit{@prev}, \textit{@exclude}, \textit{@select})) (att.global.analytic (\textit{@ana})) (att.global.facs (\textit{@facs})) (att.global.change (\textit{@change})) (att.global.responsibility (\textit{@cert}, \textit{@resp})) (att.global.source (\textit{@source})) att.declaring (\textit{@decls}) att.typed (\textit{@type}, \textit{@subtype}) 
    \item[{Member of}]
  model.qLike
    \item[{Contained by}]
  
    \item[core: ]
   add cit corr del desc emph head hi item l meeting note orig p q quote ref reg said sic sp stage title unclear\par 
    \item[figures: ]
   cell figDesc figure\par 
    \item[header: ]
   change handNote licence rendition scriptNote tagUsage typeNote\par 
    \item[linking: ]
   ab seg\par 
    \item[msdescription: ]
   accMat acquisition additions collation condition custEvent decoNote filiation foliation layout musicNotation origin provenance signatures source summary support surrogates\par 
    \item[namesdates: ]
   occupation\par 
    \item[textcrit: ]
   lem rdg witness\par 
    \item[textstructure: ]
   argument body div docEdition epigraph imprimatur postscript salute signed titlePart trailer\par 
    \item[transcr: ]
   damage metamark mod restore retrace secl supplied surplus
    \item[{May contain}]
  
    \item[analysis: ]
   interp interpGrp span spanGrp\par 
    \item[core: ]
   cb gap gb index lb milestone note pb\par 
    \item[figures: ]
   figure notatedMusic\par 
    \item[linking: ]
   alt altGrp anchor join joinGrp link linkGrp timeline\par 
    \item[textcrit: ]
   app witDetail\par 
    \item[textstructure: ]
   back body front group\par 
    \item[transcr: ]
   addSpan damageSpan delSpan fw listTranspose metamark space substJoin
    \item[{Note}]
  \par
A floating text has the same content as any other <text> and may thus be interrupted by another floating text, or contain a <group> of tesselated texts.
    \item[{Example}]
  \leavevmode\bgroup\exampleFont \begin{shaded}\noindent\mbox{}{<\textbf{body}>}\mbox{}\newline 
\hspace*{6pt}{<\textbf{div}\hspace*{6pt}{type}="{scene}">}\mbox{}\newline 
\hspace*{6pt}\hspace*{6pt}{<\textbf{sp}>}\mbox{}\newline 
\hspace*{6pt}\hspace*{6pt}\hspace*{6pt}{<\textbf{p}>}Hush, the players begin...{</\textbf{p}>}\mbox{}\newline 
\hspace*{6pt}\hspace*{6pt}{</\textbf{sp}>}\mbox{}\newline 
\hspace*{6pt}\hspace*{6pt}{<\textbf{floatingText}\hspace*{6pt}{type}="{pwp}">}\mbox{}\newline 
\hspace*{6pt}\hspace*{6pt}\hspace*{6pt}{<\textbf{body}>}\mbox{}\newline 
\hspace*{6pt}\hspace*{6pt}\hspace*{6pt}\hspace*{6pt}{<\textbf{div}\hspace*{6pt}{type}="{act}">}\mbox{}\newline 
\hspace*{6pt}\hspace*{6pt}\hspace*{6pt}\hspace*{6pt}\hspace*{6pt}{<\textbf{sp}>}\mbox{}\newline 
\hspace*{6pt}\hspace*{6pt}\hspace*{6pt}\hspace*{6pt}\hspace*{6pt}\hspace*{6pt}{<\textbf{l}>}In Athens our tale takes place [...]{</\textbf{l}>}\mbox{}\newline 
\hspace*{6pt}\hspace*{6pt}\hspace*{6pt}\hspace*{6pt}\hspace*{6pt}{</\textbf{sp}>}\mbox{}\newline 
\textit{<!-- ... rest of nested act here -->}\mbox{}\newline 
\hspace*{6pt}\hspace*{6pt}\hspace*{6pt}\hspace*{6pt}{</\textbf{div}>}\mbox{}\newline 
\hspace*{6pt}\hspace*{6pt}\hspace*{6pt}{</\textbf{body}>}\mbox{}\newline 
\hspace*{6pt}\hspace*{6pt}{</\textbf{floatingText}>}\mbox{}\newline 
\hspace*{6pt}\hspace*{6pt}{<\textbf{sp}>}\mbox{}\newline 
\hspace*{6pt}\hspace*{6pt}\hspace*{6pt}{<\textbf{p}>}Now that the play is finished ...{</\textbf{p}>}\mbox{}\newline 
\hspace*{6pt}\hspace*{6pt}{</\textbf{sp}>}\mbox{}\newline 
\hspace*{6pt}{</\textbf{div}>}\mbox{}\newline 
{</\textbf{body}>}\end{shaded}\egroup 


    \item[{Content model}]
  \mbox{}\hfill\\[-10pt]\begin{Verbatim}[fontsize=\small]
<content>
 <sequence>
  <classRef key="model.global"
   maxOccurs="unbounded" minOccurs="0"/>
  <sequence minOccurs="0">
   <elementRef key="front"/>
   <classRef key="model.global"
    maxOccurs="unbounded" minOccurs="0"/>
  </sequence>
  <alternate>
   <elementRef key="body"/>
   <elementRef key="group"/>
  </alternate>
  <classRef key="model.global"
   maxOccurs="unbounded" minOccurs="0"/>
  <sequence minOccurs="0">
   <elementRef key="back"/>
   <classRef key="model.global"
    maxOccurs="unbounded" minOccurs="0"/>
  </sequence>
 </sequence>
</content>
    
\end{Verbatim}

    \item[{Schema Declaration}]
  \mbox{}\hfill\\[-10pt]\begin{Verbatim}[fontsize=\small]
element floatingText
{
   att.global.attributes,
   att.declaring.attributes,
   att.typed.attributes,
   (
      model.global*,
      ( front, model.global* )?,
      ( body | group ),
      model.global*,
      ( back, model.global* )?
   )
}
\end{Verbatim}

\end{reflist}  \index{floruit=<floruit>|oddindex}
\begin{reflist}
\item[]\begin{specHead}{TEI.floruit}{<floruit> }contains information about a person's period of activity. [\xref{http://www.tei-c.org/release/doc/tei-p5-doc/en/html/ND.html\#NDPERSEpc}{13.3.2.1. Personal Characteristics}]\end{specHead} 
    \item[{Module}]
  namesdates
    \item[{Attributes}]
  Attributes att.global (\textit{@xml:id}, \textit{@n}, \textit{@xml:lang}, \textit{@xml:base}, \textit{@xml:space})  (att.global.rendition (\textit{@rend}, \textit{@style}, \textit{@rendition})) (att.global.linking (\textit{@corresp}, \textit{@synch}, \textit{@sameAs}, \textit{@copyOf}, \textit{@next}, \textit{@prev}, \textit{@exclude}, \textit{@select})) (att.global.analytic (\textit{@ana})) (att.global.facs (\textit{@facs})) (att.global.change (\textit{@change})) (att.global.responsibility (\textit{@cert}, \textit{@resp})) (att.global.source (\textit{@source})) att.datable (\textit{@calendar}, \textit{@period})  (att.datable.w3c (\textit{@when}, \textit{@notBefore}, \textit{@notAfter}, \textit{@from}, \textit{@to})) (att.datable.iso (\textit{@when-iso}, \textit{@notBefore-iso}, \textit{@notAfter-iso}, \textit{@from-iso}, \textit{@to-iso})) (att.datable.custom (\textit{@when-custom}, \textit{@notBefore-custom}, \textit{@notAfter-custom}, \textit{@from-custom}, \textit{@to-custom}, \textit{@datingPoint}, \textit{@datingMethod})) att.editLike (\textit{@evidence}, \textit{@instant})  (att.dimensions (\textit{@unit}, \textit{@quantity}, \textit{@extent}, \textit{@precision}, \textit{@scope}) (att.ranging (\textit{@atLeast}, \textit{@atMost}, \textit{@min}, \textit{@max}, \textit{@confidence})) )
    \item[{Member of}]
  model.persStateLike
    \item[{Contained by}]
  
    \item[namesdates: ]
   person personGrp
    \item[{May contain}]
  
    \item[analysis: ]
   c cl interp interpGrp m pc phr s span spanGrp w\par 
    \item[core: ]
   abbr add address cb choice corr date del distinct email emph expan foreign gap gb gloss graphic hi index lb measure measureGrp media mentioned milestone name note num orig pb ptr ref reg rs sic soCalled term time title unclear\par 
    \item[figures: ]
   figure formula notatedMusic\par 
    \item[gaiji: ]
   g\par 
    \item[header: ]
   idno\par 
    \item[linking: ]
   alt altGrp anchor join joinGrp link linkGrp seg timeline\par 
    \item[msdescription: ]
   catchwords depth dim dimensions height heraldry locus locusGrp material objectType origDate origPlace secFol signatures stamp watermark width\par 
    \item[namesdates: ]
   addName affiliation bloc climate country district forename genName geo geogFeat geogName location nameLink offset orgName persName placeName population region roleName settlement state surname terrain trait\par 
    \item[textcrit: ]
   app witDetail\par 
    \item[transcr: ]
   addSpan am damage damageSpan delSpan ex fw handShift listTranspose metamark mod redo restore retrace secl space subst substJoin supplied surplus undo\par character data
    \item[{Example}]
  \leavevmode\bgroup\exampleFont \begin{shaded}\noindent\mbox{}{<\textbf{floruit}\hspace*{6pt}{notAfter}="{1100}"\hspace*{6pt}{notBefore}="{1066}"/>}\end{shaded}\egroup 


    \item[{Content model}]
  \mbox{}\hfill\\[-10pt]\begin{Verbatim}[fontsize=\small]
<content>
 <macroRef key="macro.phraseSeq"/>
</content>
    
\end{Verbatim}

    \item[{Schema Declaration}]
  \mbox{}\hfill\\[-10pt]\begin{Verbatim}[fontsize=\small]
element floruit
{
   att.global.attributes,
   att.datable.attributes,
   att.editLike.attributes,
   macro.phraseSeq}
\end{Verbatim}

\end{reflist}  \index{foliation=<foliation>|oddindex}
\begin{reflist}
\item[]\begin{specHead}{TEI.foliation}{<foliation> }describes the numbering system or systems used to count the leaves or pages in a codex. [\xref{http://www.tei-c.org/release/doc/tei-p5-doc/en/html/MS.html\#msphfo}{10.7.1.4. Foliation}]\end{specHead} 
    \item[{Module}]
  msdescription
    \item[{Attributes}]
  Attributes att.global (\textit{@xml:id}, \textit{@n}, \textit{@xml:lang}, \textit{@xml:base}, \textit{@xml:space})  (att.global.rendition (\textit{@rend}, \textit{@style}, \textit{@rendition})) (att.global.linking (\textit{@corresp}, \textit{@synch}, \textit{@sameAs}, \textit{@copyOf}, \textit{@next}, \textit{@prev}, \textit{@exclude}, \textit{@select})) (att.global.analytic (\textit{@ana})) (att.global.facs (\textit{@facs})) (att.global.change (\textit{@change})) (att.global.responsibility (\textit{@cert}, \textit{@resp})) (att.global.source (\textit{@source}))
    \item[{Contained by}]
  
    \item[msdescription: ]
   supportDesc
    \item[{May contain}]
  
    \item[analysis: ]
   c cl interp interpGrp m pc phr s span spanGrp w\par 
    \item[core: ]
   abbr add address bibl biblStruct cb choice cit corr date del desc distinct email emph expan foreign gap gb gloss graphic hi index l label lb lg list listBibl measure measureGrp media mentioned milestone name note num orig p pb ptr q quote ref reg rs said sic soCalled sp stage term time title unclear\par 
    \item[figures: ]
   figure formula notatedMusic table\par 
    \item[gaiji: ]
   g\par 
    \item[header: ]
   biblFull idno\par 
    \item[linking: ]
   ab alt altGrp anchor join joinGrp link linkGrp seg timeline\par 
    \item[msdescription: ]
   catchwords depth dim dimensions height heraldry locus locusGrp material msDesc objectType origDate origPlace secFol signatures stamp watermark width\par 
    \item[namesdates: ]
   addName affiliation bloc climate country district forename genName geo geogFeat geogName listEvent listNym listOrg listPerson listPlace location nameLink offset orgName persName placeName population region roleName settlement state surname terrain trait\par 
    \item[textcrit: ]
   app listApp listWit witDetail\par 
    \item[textstructure: ]
   floatingText\par 
    \item[transcr: ]
   addSpan am damage damageSpan delSpan ex fw handShift listTranspose metamark mod redo restore retrace secl space subst substJoin supplied surplus undo\par character data
    \item[{Example}]
  \leavevmode\bgroup\exampleFont \begin{shaded}\noindent\mbox{}{<\textbf{foliation}>}Contemporary foliation in red\mbox{}\newline 
 roman numerals in the centre\mbox{}\newline 
 of the outer margin.{</\textbf{foliation}>}\end{shaded}\egroup 


    \item[{Content model}]
  \mbox{}\hfill\\[-10pt]\begin{Verbatim}[fontsize=\small]
<content>
 <macroRef key="macro.specialPara"/>
</content>
    
\end{Verbatim}

    \item[{Schema Declaration}]
  \mbox{}\hfill\\[-10pt]\begin{Verbatim}[fontsize=\small]
element foliation { att.global.attributes, macro.specialPara }
\end{Verbatim}

\end{reflist}  \index{foreign=<foreign>|oddindex}
\begin{reflist}
\item[]\begin{specHead}{TEI.foreign}{<foreign> }identifies a word or phrase as belonging to some language other than that of the surrounding text. [\xref{http://www.tei-c.org/release/doc/tei-p5-doc/en/html/CO.html\#COHQHF}{3.3.2.1. Foreign Words or Expressions}]\end{specHead} 
    \item[{Module}]
  core
    \item[{Attributes}]
  Attributes att.global (\textit{@xml:id}, \textit{@n}, \textit{@xml:lang}, \textit{@xml:base}, \textit{@xml:space})  (att.global.rendition (\textit{@rend}, \textit{@style}, \textit{@rendition})) (att.global.linking (\textit{@corresp}, \textit{@synch}, \textit{@sameAs}, \textit{@copyOf}, \textit{@next}, \textit{@prev}, \textit{@exclude}, \textit{@select})) (att.global.analytic (\textit{@ana})) (att.global.facs (\textit{@facs})) (att.global.change (\textit{@change})) (att.global.responsibility (\textit{@cert}, \textit{@resp})) (att.global.source (\textit{@source}))
    \item[{Member of}]
  model.emphLike
    \item[{Contained by}]
  
    \item[analysis: ]
   cl phr s span\par 
    \item[core: ]
   abbr add addrLine author bibl biblScope citedRange corr date del desc distinct editor email emph expan foreign gloss head headItem headLabel hi item l label measure meeting mentioned name note num orig p pubPlace publisher q quote ref reg resp rs said sic soCalled speaker stage street term textLang time title unclear\par 
    \item[figures: ]
   cell figDesc\par 
    \item[header: ]
   authority catDesc change classCode creation distributor edition extent funder geoDecl handNote language licence principal rendition scriptNote sponsor tagUsage typeNote\par 
    \item[linking: ]
   ab seg\par 
    \item[msdescription: ]
   accMat acquisition additions catchwords collation colophon condition custEvent decoNote explicit filiation finalRubric foliation heraldry incipit layout material musicNotation objectType origDate origPlace origin provenance rubric secFol signatures source stamp summary support surrogates watermark\par 
    \item[namesdates: ]
   addName affiliation age birth bloc country death district education faith floruit forename genName geogFeat geogName langKnown nameLink nationality occupation offset orgName persName placeName region residence roleName settlement sex socecStatus surname\par 
    \item[textcrit: ]
   lem rdg wit witDetail witness\par 
    \item[textstructure: ]
   byline closer dateline docAuthor docDate docEdition docImprint imprimatur opener salute signed titlePart trailer\par 
    \item[transcr: ]
   damage fw metamark mod restore retrace secl supplied surplus
    \item[{May contain}]
  
    \item[analysis: ]
   c cl interp interpGrp m pc phr s span spanGrp w\par 
    \item[core: ]
   abbr add address cb choice corr date del distinct email emph expan foreign gap gb gloss graphic hi index lb measure measureGrp media mentioned milestone name note num orig pb ptr ref reg rs sic soCalled term time title unclear\par 
    \item[figures: ]
   figure formula notatedMusic\par 
    \item[gaiji: ]
   g\par 
    \item[header: ]
   idno\par 
    \item[linking: ]
   alt altGrp anchor join joinGrp link linkGrp seg timeline\par 
    \item[msdescription: ]
   catchwords depth dim dimensions height heraldry locus locusGrp material objectType origDate origPlace secFol signatures stamp watermark width\par 
    \item[namesdates: ]
   addName affiliation bloc climate country district forename genName geo geogFeat geogName location nameLink offset orgName persName placeName population region roleName settlement state surname terrain trait\par 
    \item[textcrit: ]
   app witDetail\par 
    \item[transcr: ]
   addSpan am damage damageSpan delSpan ex fw handShift listTranspose metamark mod redo restore retrace secl space subst substJoin supplied surplus undo\par character data
    \item[{Note}]
  \par
The global {\itshape xml:lang} attribute should be supplied for this element to identify the language of the word or phrase marked. As elsewhere, its value should be a language tag as defined in \xref{http://www.tei-c.org/release/doc/tei-p5-doc/en/html/CH.html\#CHSH}{6.1. Language Identification}.\par
This element is intended for use only where no other element is available to mark the phrase or words concerned. The global {\itshape xml:lang} attribute should be used in preference to this element where it is intended to mark the language of the whole of some text element.\par
The <distinct> element may be used to identify phrases belonging to sublanguages or registers not generally regarded as true languages.
    \item[{Example}]
  \leavevmode\bgroup\exampleFont \begin{shaded}\noindent\mbox{}This is\mbox{}\newline 
 heathen Greek to you still? Your {<\textbf{foreign}\hspace*{6pt}{xml:lang}="{la}">}lapis\mbox{}\newline 
 philosophicus{</\textbf{foreign}>}?\end{shaded}\egroup 


    \item[{Content model}]
  \mbox{}\hfill\\[-10pt]\begin{Verbatim}[fontsize=\small]
<content>
 <macroRef key="macro.phraseSeq"/>
</content>
    
\end{Verbatim}

    \item[{Schema Declaration}]
  \mbox{}\hfill\\[-10pt]\begin{Verbatim}[fontsize=\small]
element foreign { att.global.attributes, macro.phraseSeq }
\end{Verbatim}

\end{reflist}  \index{forename=<forename>|oddindex}
\begin{reflist}
\item[]\begin{specHead}{TEI.forename}{<forename> }contains a forename, given or baptismal name. [\xref{http://www.tei-c.org/release/doc/tei-p5-doc/en/html/ND.html\#NDPER}{13.2.1. Personal Names}]\end{specHead} 
    \item[{Module}]
  namesdates
    \item[{Attributes}]
  Attributes att.global (\textit{@xml:id}, \textit{@n}, \textit{@xml:lang}, \textit{@xml:base}, \textit{@xml:space})  (att.global.rendition (\textit{@rend}, \textit{@style}, \textit{@rendition})) (att.global.linking (\textit{@corresp}, \textit{@synch}, \textit{@sameAs}, \textit{@copyOf}, \textit{@next}, \textit{@prev}, \textit{@exclude}, \textit{@select})) (att.global.analytic (\textit{@ana})) (att.global.facs (\textit{@facs})) (att.global.change (\textit{@change})) (att.global.responsibility (\textit{@cert}, \textit{@resp})) (att.global.source (\textit{@source})) att.personal (\textit{@full}, \textit{@sort})  (att.naming (\textit{@role}, \textit{@nymRef}) (att.canonical (\textit{@key}, \textit{@ref})) ) att.typed (\textit{@type}, \textit{@subtype}) 
    \item[{Member of}]
  model.persNamePart
    \item[{Contained by}]
  
    \item[analysis: ]
   cl phr s span\par 
    \item[core: ]
   abbr add addrLine address author bibl biblScope citedRange corr date del desc distinct editor email emph expan foreign gloss head headItem headLabel hi item l label measure meeting mentioned name note num orig p pubPlace publisher q quote ref reg resp rs said sic soCalled speaker stage street term textLang time title unclear\par 
    \item[figures: ]
   cell figDesc\par 
    \item[header: ]
   authority catDesc change classCode correspAction creation distributor edition extent funder geoDecl handNote language licence principal rendition scriptNote sponsor tagUsage typeNote\par 
    \item[linking: ]
   ab seg\par 
    \item[msdescription: ]
   accMat acquisition additions catchwords collation colophon condition custEvent decoNote explicit filiation finalRubric foliation heraldry incipit layout material musicNotation objectType origDate origPlace origin provenance rubric secFol signatures source stamp summary support surrogates watermark\par 
    \item[namesdates: ]
   addName affiliation age birth bloc country death district education faith floruit forename genName geogFeat geogName langKnown nameLink nationality occupation offset org orgName persName placeName region residence roleName settlement sex socecStatus surname\par 
    \item[textcrit: ]
   lem rdg wit witDetail witness\par 
    \item[textstructure: ]
   byline closer dateline docAuthor docDate docEdition docImprint imprimatur opener salute signed titlePart trailer\par 
    \item[transcr: ]
   damage fw metamark mod restore retrace secl supplied surplus
    \item[{May contain}]
  
    \item[analysis: ]
   c cl interp interpGrp m pc phr s span spanGrp w\par 
    \item[core: ]
   abbr add address cb choice corr date del distinct email emph expan foreign gap gb gloss graphic hi index lb measure measureGrp media mentioned milestone name note num orig pb ptr ref reg rs sic soCalled term time title unclear\par 
    \item[figures: ]
   figure formula notatedMusic\par 
    \item[gaiji: ]
   g\par 
    \item[header: ]
   idno\par 
    \item[linking: ]
   alt altGrp anchor join joinGrp link linkGrp seg timeline\par 
    \item[msdescription: ]
   catchwords depth dim dimensions height heraldry locus locusGrp material objectType origDate origPlace secFol signatures stamp watermark width\par 
    \item[namesdates: ]
   addName affiliation bloc climate country district forename genName geo geogFeat geogName location nameLink offset orgName persName placeName population region roleName settlement state surname terrain trait\par 
    \item[textcrit: ]
   app witDetail\par 
    \item[transcr: ]
   addSpan am damage damageSpan delSpan ex fw handShift listTranspose metamark mod redo restore retrace secl space subst substJoin supplied surplus undo\par character data
    \item[{Example}]
  \leavevmode\bgroup\exampleFont \begin{shaded}\noindent\mbox{}{<\textbf{persName}>}\mbox{}\newline 
\hspace*{6pt}{<\textbf{roleName}>}Ex-President{</\textbf{roleName}>}\mbox{}\newline 
\hspace*{6pt}{<\textbf{forename}>}George{</\textbf{forename}>}\mbox{}\newline 
\hspace*{6pt}{<\textbf{surname}>}Bush{</\textbf{surname}>}\mbox{}\newline 
{</\textbf{persName}>}\end{shaded}\egroup 


    \item[{Content model}]
  \mbox{}\hfill\\[-10pt]\begin{Verbatim}[fontsize=\small]
<content>
 <macroRef key="macro.phraseSeq"/>
</content>
    
\end{Verbatim}

    \item[{Schema Declaration}]
  \mbox{}\hfill\\[-10pt]\begin{Verbatim}[fontsize=\small]
element forename
{
   att.global.attributes,
   att.personal.attributes,
   att.typed.attributes,
   macro.phraseSeq}
\end{Verbatim}

\end{reflist}  \index{formula=<formula>|oddindex}
\begin{reflist}
\item[]\begin{specHead}{TEI.formula}{<formula> }contains a mathematical or other formula. [\xref{http://www.tei-c.org/release/doc/tei-p5-doc/en/html/FT.html\#FTFOR}{14.2. Formulæ and Mathematical Expressions}]\end{specHead} 
    \item[{Module}]
  figures
    \item[{Attributes}]
  Attributes att.global (\textit{@xml:id}, \textit{@n}, \textit{@xml:lang}, \textit{@xml:base}, \textit{@xml:space})  (att.global.rendition (\textit{@rend}, \textit{@style}, \textit{@rendition})) (att.global.linking (\textit{@corresp}, \textit{@synch}, \textit{@sameAs}, \textit{@copyOf}, \textit{@next}, \textit{@prev}, \textit{@exclude}, \textit{@select})) (att.global.analytic (\textit{@ana})) (att.global.facs (\textit{@facs})) (att.global.change (\textit{@change})) (att.global.responsibility (\textit{@cert}, \textit{@resp})) (att.global.source (\textit{@source})) att.notated (\textit{@notation}) 
    \item[{Member of}]
  model.graphicLike
    \item[{Contained by}]
  
    \item[analysis: ]
   cl phr s\par 
    \item[core: ]
   abbr add addrLine author biblScope citedRange corr date del distinct editor email emph expan foreign gloss head headItem headLabel hi item l label measure mentioned name note num orig p pubPlace publisher q quote ref reg rs said sic soCalled speaker stage street term textLang time title unclear\par 
    \item[figures: ]
   cell figure formula table\par 
    \item[gaiji: ]
   char glyph\par 
    \item[header: ]
   change distributor edition extent geoDecl handNote licence scriptNote typeNote\par 
    \item[linking: ]
   ab seg\par 
    \item[msdescription: ]
   accMat acquisition additions catchwords collation colophon condition custEvent decoNote explicit filiation finalRubric foliation heraldry incipit layout material musicNotation objectType origDate origPlace origin provenance rubric secFol signatures source stamp summary support surrogates watermark\par 
    \item[namesdates: ]
   addName affiliation birth bloc country death district education faith floruit forename genName geogFeat geogName nameLink nationality occupation offset orgName persName placeName region residence roleName settlement sex socecStatus surname\par 
    \item[textcrit: ]
   lem rdg wit witDetail\par 
    \item[textstructure: ]
   byline closer dateline docAuthor docDate docEdition docImprint imprimatur opener salute signed titlePart trailer\par 
    \item[transcr: ]
   damage facsimile fw metamark mod restore retrace secl sourceDoc supplied surface surplus zone
    \item[{May contain}]
  
    \item[core: ]
   graphic hi media\par 
    \item[figures: ]
   formula\par character data
    \item[{Example}]
  \leavevmode\bgroup\exampleFont \begin{shaded}\noindent\mbox{}{<\textbf{formula}\hspace*{6pt}{notation}="{tex}">}\$E=mc\textasciicircum 2\${</\textbf{formula}>}\end{shaded}\egroup 


    \item[{Example}]
  \leavevmode\bgroup\exampleFont \begin{shaded}\noindent\mbox{}{<\textbf{formula}\hspace*{6pt}{notation}="{none}">}E=mc{<\textbf{hi}\hspace*{6pt}{rend}="{sup}">}2{</\textbf{hi}>}\mbox{}\newline 
{</\textbf{formula}>}\end{shaded}\egroup 


    \item[{Example}]
  \leavevmode\bgroup\exampleFont \begin{shaded}\noindent\mbox{}{<\textbf{formula}\hspace*{6pt}{notation}="{mathml}">}\mbox{}\newline 
\hspace*{6pt}{<\textbf{m:math}>}\mbox{}\newline 
\hspace*{6pt}\hspace*{6pt}{<\textbf{m:mi}>}E{</\textbf{m:mi}>}\mbox{}\newline 
\hspace*{6pt}\hspace*{6pt}{<\textbf{m:mo}>}={</\textbf{m:mo}>}\mbox{}\newline 
\hspace*{6pt}\hspace*{6pt}{<\textbf{m:mi}>}m{</\textbf{m:mi}>}\mbox{}\newline 
\hspace*{6pt}\hspace*{6pt}{<\textbf{m:msup}>}\mbox{}\newline 
\hspace*{6pt}\hspace*{6pt}\hspace*{6pt}{<\textbf{m:mrow}>}\mbox{}\newline 
\hspace*{6pt}\hspace*{6pt}\hspace*{6pt}\hspace*{6pt}{<\textbf{m:mi}>}c{</\textbf{m:mi}>}\mbox{}\newline 
\hspace*{6pt}\hspace*{6pt}\hspace*{6pt}{</\textbf{m:mrow}>}\mbox{}\newline 
\hspace*{6pt}\hspace*{6pt}\hspace*{6pt}{<\textbf{m:mrow}>}\mbox{}\newline 
\hspace*{6pt}\hspace*{6pt}\hspace*{6pt}\hspace*{6pt}{<\textbf{m:mn}>}2{</\textbf{m:mn}>}\mbox{}\newline 
\hspace*{6pt}\hspace*{6pt}\hspace*{6pt}{</\textbf{m:mrow}>}\mbox{}\newline 
\hspace*{6pt}\hspace*{6pt}{</\textbf{m:msup}>}\mbox{}\newline 
\hspace*{6pt}{</\textbf{m:math}>}\mbox{}\newline 
{</\textbf{formula}>}\end{shaded}\egroup 


    \item[{Content model}]
  \mbox{}\hfill\\[-10pt]\begin{Verbatim}[fontsize=\small]
<content>
 <alternate maxOccurs="unbounded"
  minOccurs="0">
  <textNode/>
  <classRef key="model.graphicLike"/>
  <classRef key="model.hiLike"/>
 </alternate>
</content>
    
\end{Verbatim}

    \item[{Schema Declaration}]
  \mbox{}\hfill\\[-10pt]\begin{Verbatim}[fontsize=\small]
element formula
{
   att.global.attributes,
   att.notated.attributes,
   ( text | model.graphicLike | model.hiLike )*
}
\end{Verbatim}

\end{reflist}  \index{front=<front>|oddindex}
\begin{reflist}
\item[]\begin{specHead}{TEI.front}{<front> }(front matter) contains any prefatory matter (headers, abstracts, title page, prefaces, dedications, etc.) found at the start of a document, before the main body. [\xref{http://www.tei-c.org/release/doc/tei-p5-doc/en/html/DS.html\#DSTITL}{4.6. Title Pages} \xref{http://www.tei-c.org/release/doc/tei-p5-doc/en/html/DS.html\#DS}{4. Default Text Structure}]\end{specHead} 
    \item[{Module}]
  textstructure
    \item[{Attributes}]
  Attributes att.global (\textit{@xml:id}, \textit{@n}, \textit{@xml:lang}, \textit{@xml:base}, \textit{@xml:space})  (att.global.rendition (\textit{@rend}, \textit{@style}, \textit{@rendition})) (att.global.linking (\textit{@corresp}, \textit{@synch}, \textit{@sameAs}, \textit{@copyOf}, \textit{@next}, \textit{@prev}, \textit{@exclude}, \textit{@select})) (att.global.analytic (\textit{@ana})) (att.global.facs (\textit{@facs})) (att.global.change (\textit{@change})) (att.global.responsibility (\textit{@cert}, \textit{@resp})) (att.global.source (\textit{@source})) att.declaring (\textit{@decls}) 
    \item[{Contained by}]
  
    \item[textstructure: ]
   floatingText text\par 
    \item[transcr: ]
   facsimile
    \item[{May contain}]
  
    \item[analysis: ]
   interp interpGrp span spanGrp\par 
    \item[core: ]
   cb divGen gap gb head index lb listBibl meeting milestone note p pb\par 
    \item[figures: ]
   figure notatedMusic\par 
    \item[linking: ]
   ab alt altGrp anchor join joinGrp link linkGrp timeline\par 
    \item[textcrit: ]
   app witDetail\par 
    \item[textstructure: ]
   argument byline closer dateline div docAuthor docDate docEdition docImprint docTitle epigraph postscript salute signed titlePage titlePart trailer\par 
    \item[transcr: ]
   addSpan damageSpan delSpan fw listTranspose metamark space substJoin
    \item[{Note}]
  \par
Because cultural conventions differ as to which elements are grouped as front matter and which as back matter, the content models for the <front> and <back> elements are identical.
    \item[{Example}]
  \leavevmode\bgroup\exampleFont \begin{shaded}\noindent\mbox{}{<\textbf{front}>}\mbox{}\newline 
\hspace*{6pt}{<\textbf{epigraph}>}\mbox{}\newline 
\hspace*{6pt}\hspace*{6pt}{<\textbf{quote}>}Nam Sibyllam quidem Cumis ego ipse oculis meis vidi in ampulla\mbox{}\newline 
\hspace*{6pt}\hspace*{6pt}\hspace*{6pt}\hspace*{6pt} pendere, et cum illi pueri dicerent: {<\textbf{q}\hspace*{6pt}{xml:lang}="{gr}">}Σίβυλλα τί\mbox{}\newline 
\hspace*{6pt}\hspace*{6pt}\hspace*{6pt}\hspace*{6pt}\hspace*{6pt}\hspace*{6pt} θέλεις{</\textbf{q}>}; respondebat illa: {<\textbf{q}\hspace*{6pt}{xml:lang}="{gr}">}ὰποθανεῖν θέλω.{</\textbf{q}>}\mbox{}\newline 
\hspace*{6pt}\hspace*{6pt}{</\textbf{quote}>}\mbox{}\newline 
\hspace*{6pt}{</\textbf{epigraph}>}\mbox{}\newline 
\hspace*{6pt}{<\textbf{div}\hspace*{6pt}{type}="{dedication}">}\mbox{}\newline 
\hspace*{6pt}\hspace*{6pt}{<\textbf{p}>}For Ezra Pound {<\textbf{q}\hspace*{6pt}{xml:lang}="{it}">}il miglior fabbro.{</\textbf{q}>}\mbox{}\newline 
\hspace*{6pt}\hspace*{6pt}{</\textbf{p}>}\mbox{}\newline 
\hspace*{6pt}{</\textbf{div}>}\mbox{}\newline 
{</\textbf{front}>}\end{shaded}\egroup 


    \item[{Example}]
  \leavevmode\bgroup\exampleFont \begin{shaded}\noindent\mbox{}{<\textbf{front}>}\mbox{}\newline 
\hspace*{6pt}{<\textbf{div}\hspace*{6pt}{type}="{dedication}">}\mbox{}\newline 
\hspace*{6pt}\hspace*{6pt}{<\textbf{p}>}To our three selves{</\textbf{p}>}\mbox{}\newline 
\hspace*{6pt}{</\textbf{div}>}\mbox{}\newline 
\hspace*{6pt}{<\textbf{div}\hspace*{6pt}{type}="{preface}">}\mbox{}\newline 
\hspace*{6pt}\hspace*{6pt}{<\textbf{head}>}Author's Note{</\textbf{head}>}\mbox{}\newline 
\hspace*{6pt}\hspace*{6pt}{<\textbf{p}>}All the characters in this book are purely imaginary, and if the\mbox{}\newline 
\hspace*{6pt}\hspace*{6pt}\hspace*{6pt}\hspace*{6pt} author has used names that may suggest a reference to living persons\mbox{}\newline 
\hspace*{6pt}\hspace*{6pt}\hspace*{6pt}\hspace*{6pt} she has done so inadvertently. ...{</\textbf{p}>}\mbox{}\newline 
\hspace*{6pt}{</\textbf{div}>}\mbox{}\newline 
{</\textbf{front}>}\end{shaded}\egroup 


    \item[{Example}]
  \leavevmode\bgroup\exampleFont \begin{shaded}\noindent\mbox{}{<\textbf{front}>}\mbox{}\newline 
\hspace*{6pt}{<\textbf{div}\hspace*{6pt}{type}="{abstract}">}\mbox{}\newline 
\hspace*{6pt}\hspace*{6pt}{<\textbf{div}>}\mbox{}\newline 
\hspace*{6pt}\hspace*{6pt}\hspace*{6pt}{<\textbf{head}>} BACKGROUND:{</\textbf{head}>}\mbox{}\newline 
\hspace*{6pt}\hspace*{6pt}\hspace*{6pt}{<\textbf{p}>}Food insecurity can put children at greater risk of obesity because\mbox{}\newline 
\hspace*{6pt}\hspace*{6pt}\hspace*{6pt}\hspace*{6pt}\hspace*{6pt}\hspace*{6pt} of altered food choices and nonuniform consumption patterns.{</\textbf{p}>}\mbox{}\newline 
\hspace*{6pt}\hspace*{6pt}{</\textbf{div}>}\mbox{}\newline 
\hspace*{6pt}\hspace*{6pt}{<\textbf{div}>}\mbox{}\newline 
\hspace*{6pt}\hspace*{6pt}\hspace*{6pt}{<\textbf{head}>} OBJECTIVE:{</\textbf{head}>}\mbox{}\newline 
\hspace*{6pt}\hspace*{6pt}\hspace*{6pt}{<\textbf{p}>}We examined the association between obesity and both child-level\mbox{}\newline 
\hspace*{6pt}\hspace*{6pt}\hspace*{6pt}\hspace*{6pt}\hspace*{6pt}\hspace*{6pt} food insecurity and personal food insecurity in US children.{</\textbf{p}>}\mbox{}\newline 
\hspace*{6pt}\hspace*{6pt}{</\textbf{div}>}\mbox{}\newline 
\hspace*{6pt}\hspace*{6pt}{<\textbf{div}>}\mbox{}\newline 
\hspace*{6pt}\hspace*{6pt}\hspace*{6pt}{<\textbf{head}>} DESIGN:{</\textbf{head}>}\mbox{}\newline 
\hspace*{6pt}\hspace*{6pt}\hspace*{6pt}{<\textbf{p}>}Data from 9,701 participants in the National Health and Nutrition\mbox{}\newline 
\hspace*{6pt}\hspace*{6pt}\hspace*{6pt}\hspace*{6pt}\hspace*{6pt}\hspace*{6pt} Examination Survey, 2001-2010, aged 2 to 11 years were analyzed.\mbox{}\newline 
\hspace*{6pt}\hspace*{6pt}\hspace*{6pt}\hspace*{6pt}\hspace*{6pt}\hspace*{6pt} Child-level food insecurity was assessed with the US Department of\mbox{}\newline 
\hspace*{6pt}\hspace*{6pt}\hspace*{6pt}\hspace*{6pt}\hspace*{6pt}\hspace*{6pt} Agriculture's Food Security Survey Module based on eight\mbox{}\newline 
\hspace*{6pt}\hspace*{6pt}\hspace*{6pt}\hspace*{6pt}\hspace*{6pt}\hspace*{6pt} child-specific questions. Personal food insecurity was assessed with\mbox{}\newline 
\hspace*{6pt}\hspace*{6pt}\hspace*{6pt}\hspace*{6pt}\hspace*{6pt}\hspace*{6pt} five additional questions. Obesity was defined, using physical\mbox{}\newline 
\hspace*{6pt}\hspace*{6pt}\hspace*{6pt}\hspace*{6pt}\hspace*{6pt}\hspace*{6pt} measurements, as body mass index (calculated as kg/m2) greater than\mbox{}\newline 
\hspace*{6pt}\hspace*{6pt}\hspace*{6pt}\hspace*{6pt}\hspace*{6pt}\hspace*{6pt} or equal to the age- and sex-specific 95th percentile of the Centers\mbox{}\newline 
\hspace*{6pt}\hspace*{6pt}\hspace*{6pt}\hspace*{6pt}\hspace*{6pt}\hspace*{6pt} for Disease Control and Prevention growth charts. Logistic\mbox{}\newline 
\hspace*{6pt}\hspace*{6pt}\hspace*{6pt}\hspace*{6pt}\hspace*{6pt}\hspace*{6pt} regressions adjusted for sex, race/ethnic group, poverty level, and\mbox{}\newline 
\hspace*{6pt}\hspace*{6pt}\hspace*{6pt}\hspace*{6pt}\hspace*{6pt}\hspace*{6pt} survey year were conducted to describe associations between obesity\mbox{}\newline 
\hspace*{6pt}\hspace*{6pt}\hspace*{6pt}\hspace*{6pt}\hspace*{6pt}\hspace*{6pt} and food insecurity.{</\textbf{p}>}\mbox{}\newline 
\hspace*{6pt}\hspace*{6pt}{</\textbf{div}>}\mbox{}\newline 
\hspace*{6pt}\hspace*{6pt}{<\textbf{div}>}\mbox{}\newline 
\hspace*{6pt}\hspace*{6pt}\hspace*{6pt}{<\textbf{head}>} RESULTS:{</\textbf{head}>}\mbox{}\newline 
\hspace*{6pt}\hspace*{6pt}\hspace*{6pt}{<\textbf{p}>}Obesity was significantly associated with personal food insecurity\mbox{}\newline 
\hspace*{6pt}\hspace*{6pt}\hspace*{6pt}\hspace*{6pt}\hspace*{6pt}\hspace*{6pt} for children aged 6 to 11 years (odds ratio=1.81; 95\% CI 1.33 to\mbox{}\newline 
\hspace*{6pt}\hspace*{6pt}\hspace*{6pt}\hspace*{6pt}\hspace*{6pt}\hspace*{6pt} 2.48), but not in children aged 2 to 5 years (odds ratio=0.88; 95\%\mbox{}\newline 
\hspace*{6pt}\hspace*{6pt}\hspace*{6pt}\hspace*{6pt}\hspace*{6pt}\hspace*{6pt} CI 0.51 to 1.51). Child-level food insecurity was not associated\mbox{}\newline 
\hspace*{6pt}\hspace*{6pt}\hspace*{6pt}\hspace*{6pt}\hspace*{6pt}\hspace*{6pt} with obesity among 2- to 5-year-olds or 6- to 11-year-olds.{</\textbf{p}>}\mbox{}\newline 
\hspace*{6pt}\hspace*{6pt}{</\textbf{div}>}\mbox{}\newline 
\hspace*{6pt}\hspace*{6pt}{<\textbf{div}>}\mbox{}\newline 
\hspace*{6pt}\hspace*{6pt}\hspace*{6pt}{<\textbf{head}>} CONCLUSIONS:{</\textbf{head}>}\mbox{}\newline 
\hspace*{6pt}\hspace*{6pt}\hspace*{6pt}{<\textbf{p}>}Personal food insecurity is associated with an increased risk of\mbox{}\newline 
\hspace*{6pt}\hspace*{6pt}\hspace*{6pt}\hspace*{6pt}\hspace*{6pt}\hspace*{6pt} obesity only in children aged 6 to 11 years. Personal\mbox{}\newline 
\hspace*{6pt}\hspace*{6pt}\hspace*{6pt}\hspace*{6pt}\hspace*{6pt}\hspace*{6pt} food-insecurity measures may give different results than aggregate\mbox{}\newline 
\hspace*{6pt}\hspace*{6pt}\hspace*{6pt}\hspace*{6pt}\hspace*{6pt}\hspace*{6pt} food-insecurity measures in children.{</\textbf{p}>}\mbox{}\newline 
\hspace*{6pt}\hspace*{6pt}{</\textbf{div}>}\mbox{}\newline 
\hspace*{6pt}{</\textbf{div}>}\mbox{}\newline 
{</\textbf{front}>}\end{shaded}\egroup 


    \item[{Content model}]
  \mbox{}\hfill\\[-10pt]\begin{Verbatim}[fontsize=\small]
<content>
 <sequence>
  <alternate maxOccurs="unbounded"
   minOccurs="0">
   <classRef key="model.frontPart"/>
   <classRef key="model.pLike"/>
   <classRef key="model.pLike.front"/>
   <classRef key="model.global"/>
  </alternate>
  <sequence minOccurs="0">
   <alternate>
    <sequence>
     <classRef key="model.div1Like"/>
     <alternate maxOccurs="unbounded"
      minOccurs="0">
      <classRef key="model.div1Like"/>
      <classRef key="model.frontPart"/>
      <classRef key="model.global"/>
     </alternate>
    </sequence>
    <sequence>
     <classRef key="model.divLike"/>
     <alternate maxOccurs="unbounded"
      minOccurs="0">
      <classRef key="model.divLike"/>
      <classRef key="model.frontPart"/>
      <classRef key="model.global"/>
     </alternate>
    </sequence>
   </alternate>
   <sequence minOccurs="0">
    <classRef key="model.divBottom"/>
    <alternate maxOccurs="unbounded"
     minOccurs="0">
     <classRef key="model.divBottom"/>
     <classRef key="model.global"/>
    </alternate>
   </sequence>
  </sequence>
 </sequence>
</content>
    
\end{Verbatim}

    \item[{Schema Declaration}]
  \mbox{}\hfill\\[-10pt]\begin{Verbatim}[fontsize=\small]
element front
{
   att.global.attributes,
   att.declaring.attributes,
   (
      ( model.frontPart | model.pLike | model.pLike.front | model.global )*,
      (
         (
            (
               model.div1Like,
               ( model.div1Like | model.frontPart | model.global )*
            )
          | (
               model.divLike,
               ( model.divLike | model.frontPart | model.global )*
            )
         ),
         ( model.divBottom, ( model.divBottom | model.global )* )?
      )?
   )
}
\end{Verbatim}

\end{reflist}  \index{funder=<funder>|oddindex}
\begin{reflist}
\item[]\begin{specHead}{TEI.funder}{<funder> }(funding body) specifies the name of an individual, institution, or organization responsible for the funding of a project or text. [\xref{http://www.tei-c.org/release/doc/tei-p5-doc/en/html/HD.html\#HD21}{2.2.1. The Title Statement}]\end{specHead} 
    \item[{Module}]
  header
    \item[{Attributes}]
  Attributes att.global (\textit{@xml:id}, \textit{@n}, \textit{@xml:lang}, \textit{@xml:base}, \textit{@xml:space})  (att.global.rendition (\textit{@rend}, \textit{@style}, \textit{@rendition})) (att.global.linking (\textit{@corresp}, \textit{@synch}, \textit{@sameAs}, \textit{@copyOf}, \textit{@next}, \textit{@prev}, \textit{@exclude}, \textit{@select})) (att.global.analytic (\textit{@ana})) (att.global.facs (\textit{@facs})) (att.global.change (\textit{@change})) (att.global.responsibility (\textit{@cert}, \textit{@resp})) (att.global.source (\textit{@source})) att.canonical (\textit{@key}, \textit{@ref}) 
    \item[{Member of}]
  model.respLike 
    \item[{Contained by}]
  
    \item[core: ]
   bibl monogr\par 
    \item[header: ]
   editionStmt titleStmt\par 
    \item[msdescription: ]
   msItem
    \item[{May contain}]
  
    \item[analysis: ]
   interp interpGrp span spanGrp\par 
    \item[core: ]
   abbr address cb choice date distinct email emph expan foreign gap gb gloss hi index lb measure measureGrp mentioned milestone name note num pb ptr ref rs soCalled term time title\par 
    \item[figures: ]
   figure notatedMusic\par 
    \item[header: ]
   idno\par 
    \item[linking: ]
   alt altGrp anchor join joinGrp link linkGrp timeline\par 
    \item[msdescription: ]
   catchwords depth dim dimensions height heraldry locus locusGrp material objectType origDate origPlace secFol signatures stamp watermark width\par 
    \item[namesdates: ]
   addName affiliation bloc climate country district forename genName geo geogFeat geogName location nameLink offset orgName persName placeName population region roleName settlement state surname terrain trait\par 
    \item[textcrit: ]
   app witDetail\par 
    \item[transcr: ]
   addSpan am damageSpan delSpan ex fw listTranspose metamark space subst substJoin\par character data
    \item[{Note}]
  \par
Funders provide financial support for a project; they are distinct from \textit{sponsors} (see element <sponsor>), who provide intellectual support and authority.
    \item[{Example}]
  \leavevmode\bgroup\exampleFont \begin{shaded}\noindent\mbox{}{<\textbf{funder}>}The National Endowment for the Humanities, an independent federal agency{</\textbf{funder}>}\mbox{}\newline 
{<\textbf{funder}>}Directorate General XIII of the Commission of the European Communities{</\textbf{funder}>}\mbox{}\newline 
{<\textbf{funder}>}The Andrew W. Mellon Foundation{</\textbf{funder}>}\mbox{}\newline 
{<\textbf{funder}>}The Social Sciences and Humanities Research Council of Canada{</\textbf{funder}>}\end{shaded}\egroup 


    \item[{Content model}]
  \mbox{}\hfill\\[-10pt]\begin{Verbatim}[fontsize=\small]
<content>
 <macroRef key="macro.phraseSeq.limited"/>
</content>
    
\end{Verbatim}

    \item[{Schema Declaration}]
  \mbox{}\hfill\\[-10pt]\begin{Verbatim}[fontsize=\small]
element funder
{
   att.global.attributes,
   att.canonical.attributes,
   macro.phraseSeq.limited}
\end{Verbatim}

\end{reflist}  \index{fw=<fw>|oddindex}\index{type=@type!<fw>|oddindex}
\begin{reflist}
\item[]\begin{specHead}{TEI.fw}{<fw> }(forme work) contains a running head (e.g. a header, footer), catchword, or similar material appearing on the current page. [\xref{http://www.tei-c.org/release/doc/tei-p5-doc/en/html/PH.html\#PHSK}{11.6. Headers, Footers, and Similar Matter}]\end{specHead} 
    \item[{Module}]
  transcr
    \item[{Attributes}]
  Attributes att.global (\textit{@xml:id}, \textit{@n}, \textit{@xml:lang}, \textit{@xml:base}, \textit{@xml:space})  (att.global.rendition (\textit{@rend}, \textit{@style}, \textit{@rendition})) (att.global.linking (\textit{@corresp}, \textit{@synch}, \textit{@sameAs}, \textit{@copyOf}, \textit{@next}, \textit{@prev}, \textit{@exclude}, \textit{@select})) (att.global.analytic (\textit{@ana})) (att.global.facs (\textit{@facs})) (att.global.change (\textit{@change})) (att.global.responsibility (\textit{@cert}, \textit{@resp})) (att.global.source (\textit{@source})) att.placement (\textit{@place}) att.written (\textit{@hand}) \hfil\\[-10pt]\begin{sansreflist}
    \item[@type]
  classifies the material encoded according to some useful typology.
\begin{reflist}
    \item[{Status}]
  Recommended
    \item[{Datatype}]
  teidata.enumerated
    \item[{Sample values include:}]
  \begin{description}

\item[{header}]a running title at the top of the page
\item[{footer}]a running title at the bottom of the page
\item[{pageNum}](page number) a page number or foliation symbol
\item[{lineNum}](line number) a line number, either of prose or poetry
\item[{sig}](signature) a signature or gathering symbol
\item[{catch}](catchword) a catch-word
\end{description} 
\end{reflist}  
\end{sansreflist}  
    \item[{Member of}]
  model.milestoneLike
    \item[{Contained by}]
  
    \item[analysis: ]
   cl m phr s span w\par 
    \item[core: ]
   abbr add addrLine address author bibl biblScope cit citedRange corr date del distinct editor email emph expan foreign gloss head headItem headLabel hi imprint item l label lg list listBibl measure mentioned name note num orig p pubPlace publisher q quote ref reg resp rs said series sic soCalled sp speaker stage street term textLang time title unclear\par 
    \item[figures: ]
   cell figure table\par 
    \item[header: ]
   authority change classCode distributor edition extent funder geoDecl handNote language licence principal scriptNote sponsor typeNote\par 
    \item[linking: ]
   ab seg\par 
    \item[msdescription: ]
   accMat acquisition additions catchwords collation colophon condition custEvent decoNote explicit filiation finalRubric foliation heraldry incipit layout material msItem musicNotation objectType origDate origPlace origin provenance rubric secFol signatures source stamp summary support surrogates watermark\par 
    \item[namesdates: ]
   addName affiliation age birth bloc country death district education faith floruit forename genName geogFeat geogName langKnown nameLink nationality occupation offset org orgName persName person personGrp placeName region residence roleName settlement sex socecStatus surname\par 
    \item[textcrit: ]
   lem rdg wit witDetail\par 
    \item[textstructure: ]
   argument back body byline closer dateline div docAuthor docDate docEdition docImprint docTitle epigraph floatingText front group imprimatur opener postscript salute signed text titlePage titlePart trailer\par 
    \item[transcr: ]
   damage fw line metamark mod restore retrace secl sourceDoc subst supplied surface surfaceGrp surplus zone
    \item[{May contain}]
  
    \item[analysis: ]
   c cl interp interpGrp m pc phr s span spanGrp w\par 
    \item[core: ]
   abbr add address cb choice corr date del distinct email emph expan foreign gap gb gloss graphic hi index lb measure measureGrp media mentioned milestone name note num orig pb ptr ref reg rs sic soCalled term time title unclear\par 
    \item[figures: ]
   figure formula notatedMusic\par 
    \item[gaiji: ]
   g\par 
    \item[header: ]
   idno\par 
    \item[linking: ]
   alt altGrp anchor join joinGrp link linkGrp seg timeline\par 
    \item[msdescription: ]
   catchwords depth dim dimensions height heraldry locus locusGrp material objectType origDate origPlace secFol signatures stamp watermark width\par 
    \item[namesdates: ]
   addName affiliation bloc climate country district forename genName geo geogFeat geogName location nameLink offset orgName persName placeName population region roleName settlement state surname terrain trait\par 
    \item[textcrit: ]
   app witDetail\par 
    \item[transcr: ]
   addSpan am damage damageSpan delSpan ex fw handShift listTranspose metamark mod redo restore retrace secl space subst substJoin supplied surplus undo\par character data
    \item[{Note}]
  \par
Where running heads are consistent throughout a chapter or section, it is usually more convenient to relate them to the chapter or section, e.g. by use of the {\itshape rend} attribute. The <fw> element is intended for cases where the running head changes from page to page, or where details of page layout and the internal structure of the running heads are of paramount importance.
    \item[{Example}]
  \leavevmode\bgroup\exampleFont \begin{shaded}\noindent\mbox{}{<\textbf{fw}\hspace*{6pt}{place}="{bottom}"\hspace*{6pt}{type}="{sig}">}C3{</\textbf{fw}>}\end{shaded}\egroup 


    \item[{Content model}]
  \mbox{}\hfill\\[-10pt]\begin{Verbatim}[fontsize=\small]
<content>
 <macroRef key="macro.phraseSeq"/>
</content>
    
\end{Verbatim}

    \item[{Schema Declaration}]
  \mbox{}\hfill\\[-10pt]\begin{Verbatim}[fontsize=\small]
element fw
{
   att.global.attributes,
   att.placement.attributes,
   att.written.attributes,
   attribute type { text }?,
   macro.phraseSeq}
\end{Verbatim}

\end{reflist}  \index{g=<g>|oddindex}\index{ref=@ref!<g>|oddindex}
\begin{reflist}
\item[]\begin{specHead}{TEI.g}{<g> }(character or glyph) represents a glyph, or a non-standard character. [\xref{http://www.tei-c.org/release/doc/tei-p5-doc/en/html/WD.html\#WD}{5. Characters, Glyphs, and Writing Modes}]\end{specHead} 
    \item[{Module}]
  gaiji
    \item[{Attributes}]
  Attributes att.global (\textit{@xml:id}, \textit{@n}, \textit{@xml:lang}, \textit{@xml:base}, \textit{@xml:space})  (att.global.rendition (\textit{@rend}, \textit{@style}, \textit{@rendition})) (att.global.linking (\textit{@corresp}, \textit{@synch}, \textit{@sameAs}, \textit{@copyOf}, \textit{@next}, \textit{@prev}, \textit{@exclude}, \textit{@select})) (att.global.analytic (\textit{@ana})) (att.global.facs (\textit{@facs})) (att.global.change (\textit{@change})) (att.global.responsibility (\textit{@cert}, \textit{@resp})) (att.global.source (\textit{@source})) att.typed (\textit{@type}, \textit{@subtype}) \hfil\\[-10pt]\begin{sansreflist}
    \item[@ref]
  points to a description of the character or glyph intended.
\begin{reflist}
    \item[{Status}]
  Optional
    \item[{Datatype}]
  teidata.pointer
\end{reflist}  
\end{sansreflist}  
    \item[{Member of}]
  model.gLike
    \item[{Contained by}]
  
    \item[analysis: ]
   c cl interp m pc phr s w\par 
    \item[core: ]
   abbr add addrLine author bibl biblScope citedRange corr date del distinct editor email emph expan foreign gloss head headItem headLabel hi item l label measure measureGrp mentioned name note num orig p pubPlace publisher q quote ref reg rs said series sic soCalled speaker stage street term textLang time title unclear\par 
    \item[figures: ]
   cell\par 
    \item[gaiji: ]
   mapping value\par 
    \item[header: ]
   change distributor edition extent geoDecl handNote idno licence scriptNote typeNote\par 
    \item[linking: ]
   ab seg\par 
    \item[msdescription: ]
   accMat acquisition additions catchwords collation collection colophon condition custEvent decoNote depth dim explicit filiation finalRubric foliation height heraldry incipit institution layout locus material msName musicNotation objectType origDate origPlace origin provenance repository rubric secFol signatures source stamp summary support surrogates watermark width\par 
    \item[namesdates: ]
   addName affiliation birth bloc country death district education faith floruit forename genName geogFeat geogName nameLink nationality occupation offset orgName persName placeName region residence roleName settlement sex socecStatus surname\par 
    \item[textcrit: ]
   lem rdg wit witDetail\par 
    \item[textstructure: ]
   byline closer dateline docAuthor docDate docEdition docImprint imprimatur opener salute signed titlePart trailer\par 
    \item[transcr: ]
   am damage ex fw line metamark mod restore retrace secl supplied surplus
    \item[{May contain}]
  Character data only
    \item[{Note}]
  \par
The name \textsf{g} is short for \textit{gaiji}, which is the Japanese term for a non-standardized character or glyph.
    \item[{Example}]
  \leavevmode\bgroup\exampleFont \begin{shaded}\noindent\mbox{}{<\textbf{g}\hspace*{6pt}{ref}="{\#ctlig}">}ct{</\textbf{g}>}\end{shaded}\egroup 

This example points to a <glyph> element with the identifier \texttt{ctlig} like the following: \leavevmode\bgroup\exampleFont \begin{shaded}\noindent\mbox{}{<\textbf{glyph}\hspace*{6pt}{xml:id}="{ctlig}">}\mbox{}\newline 
\textit{<!-- here we describe the particular ct-ligature intended -->}\mbox{}\newline 
{</\textbf{glyph}>}\end{shaded}\egroup 


    \item[{Example}]
  \leavevmode\bgroup\exampleFont \begin{shaded}\noindent\mbox{}{<\textbf{g}\hspace*{6pt}{ref}="{\#per-glyph}">}per{</\textbf{g}>}\end{shaded}\egroup 

The medieval brevigraph per could similarly be considered as an individual glyph, defined in a <glyph> element with the identifier \texttt{per} like the following: \leavevmode\bgroup\exampleFont \begin{shaded}\noindent\mbox{}{<\textbf{glyph}\hspace*{6pt}{xml:id}="{per-glyph}">}\mbox{}\newline 
\textit{<!-- ... -->}\mbox{}\newline 
{</\textbf{glyph}>}\end{shaded}\egroup 


    \item[{Content model}]
  \fbox{\ttfamily <content>\newline
 <textNode/>\newline
</content>\newline
    } 
    \item[{Schema Declaration}]
  \mbox{}\hfill\\[-10pt]\begin{Verbatim}[fontsize=\small]
element g
{
   att.global.attributes,
   att.typed.attributes,
   attribute ref { text }?,
   text
}
\end{Verbatim}

\end{reflist}  \index{gap=<gap>|oddindex}\index{reason=@reason!<gap>|oddindex}\index{hand=@hand!<gap>|oddindex}\index{agent=@agent!<gap>|oddindex}
\begin{reflist}
\item[]\begin{specHead}{TEI.gap}{<gap> }indicates a point where material has been omitted in a transcription, whether for editorial reasons described in the TEI header, as part of sampling practice, or because the material is illegible, invisible, or inaudible. [\xref{http://www.tei-c.org/release/doc/tei-p5-doc/en/html/CO.html\#COEDADD}{3.4.3. Additions, Deletions, and Omissions}]\end{specHead} 
    \item[{Module}]
  core
    \item[{Attributes}]
  Attributes att.global (\textit{@xml:id}, \textit{@n}, \textit{@xml:lang}, \textit{@xml:base}, \textit{@xml:space})  (att.global.rendition (\textit{@rend}, \textit{@style}, \textit{@rendition})) (att.global.linking (\textit{@corresp}, \textit{@synch}, \textit{@sameAs}, \textit{@copyOf}, \textit{@next}, \textit{@prev}, \textit{@exclude}, \textit{@select})) (att.global.analytic (\textit{@ana})) (att.global.facs (\textit{@facs})) (att.global.change (\textit{@change})) (att.global.responsibility (\textit{@cert}, \textit{@resp})) (att.global.source (\textit{@source})) att.timed (\textit{@start}, \textit{@end}) att.editLike (\textit{@evidence}, \textit{@instant})  (att.dimensions (\textit{@unit}, \textit{@quantity}, \textit{@extent}, \textit{@precision}, \textit{@scope}) (att.ranging (\textit{@atLeast}, \textit{@atMost}, \textit{@min}, \textit{@max}, \textit{@confidence})) ) \hfil\\[-10pt]\begin{sansreflist}
    \item[@reason]
  gives the reason for omission. Sample values include sampling, inaudible, irrelevant, cancelled.
\begin{reflist}
    \item[{Status}]
  Optional
    \item[{Datatype}]
  1–∞ occurrences of teidata.word separated by whitespace
\end{reflist}  
    \item[@hand]
  in the case of text omitted from the transcription because of deliberate deletion by an identifiable hand, indicates the hand which made the deletion.
\begin{reflist}
    \item[\xref{http://www.tei-c.org/Activities/Council/Working/tcw27.xml}{Deprecated}]
  will be removed on 2017-08-01
    \item[{Status}]
  Optional
    \item[{Datatype}]
  teidata.pointer
\end{reflist}  
    \item[@agent]
  in the case of text omitted because of damage, categorizes the cause of the damage, if it can be identified.
\begin{reflist}
    \item[{Status}]
  Optional
    \item[{Datatype}]
  teidata.enumerated
    \item[{Sample values include:}]
  \begin{description}

\item[{rubbing}]damage results from rubbing of the leaf edges
\item[{mildew}]damage results from mildew on the leaf surface
\item[{smoke}]damage results from smoke
\end{description} 
\end{reflist}  
\end{sansreflist}  
    \item[{Member of}]
  model.global.edit
    \item[{Contained by}]
  
    \item[analysis: ]
   cl m phr s span w\par 
    \item[core: ]
   abbr add addrLine address author bibl biblScope cit citedRange corr date del distinct editor email emph expan foreign gloss head headItem headLabel hi imprint item l label lg list measure mentioned name note num orig p pubPlace publisher q quote ref reg resp rs said series sic soCalled sp speaker stage street term textLang time title unclear\par 
    \item[figures: ]
   cell figure table\par 
    \item[header: ]
   authority change classCode distributor edition extent funder geoDecl handNote language licence principal scriptNote sponsor typeNote\par 
    \item[linking: ]
   ab seg\par 
    \item[msdescription: ]
   accMat acquisition additions catchwords collation colophon condition custEvent decoNote explicit filiation finalRubric foliation heraldry incipit layout material msItem musicNotation objectType origDate origPlace origin provenance rubric secFol signatures source stamp summary support surrogates watermark\par 
    \item[namesdates: ]
   addName affiliation age birth bloc country death district education faith floruit forename genName geogFeat geogName langKnown nameLink nationality occupation offset orgName persName person personGrp placeName region residence roleName settlement sex socecStatus surname\par 
    \item[textcrit: ]
   lem rdg wit witDetail\par 
    \item[textstructure: ]
   argument back body byline closer dateline div docAuthor docDate docEdition docImprint docTitle epigraph floatingText front group imprimatur opener postscript salute signed text titlePage titlePart trailer\par 
    \item[transcr: ]
   damage fw line metamark mod restore retrace secl sourceDoc supplied surface surfaceGrp surplus zone
    \item[{May contain}]
  
    \item[core: ]
   desc
    \item[{Note}]
  \par
The <gap>, <unclear>, and <del> core tag elements may be closely allied in use with the <damage> and <supplied> elements, available when using the additional tagset for transcription of primary sources. See section \xref{http://www.tei-c.org/release/doc/tei-p5-doc/en/html/PH.html\#PHCOMB}{11.3.3.2. Use of the gap, del, damage, unclear, and supplied Elements in Combination} for discussion of which element is appropriate for which circumstance.\par
The <gap> tag simply signals the editors decision to omit or inability to transcribe a span of text. Other information, such as the interpretation that text was deliberately erased or covered, should be indicated using the relevant tags, such as <del> in the case of deliberate deletion.
    \item[{Example}]
  \leavevmode\bgroup\exampleFont \begin{shaded}\noindent\mbox{}{<\textbf{gap}\hspace*{6pt}{quantity}="{4}"\hspace*{6pt}{reason}="{illegible}"\mbox{}\newline 
\hspace*{6pt}{unit}="{chars}"/>}\end{shaded}\egroup 


    \item[{Example}]
  \leavevmode\bgroup\exampleFont \begin{shaded}\noindent\mbox{}{<\textbf{gap}\hspace*{6pt}{quantity}="{1}"\hspace*{6pt}{reason}="{sampling}"\mbox{}\newline 
\hspace*{6pt}{unit}="{essay}"/>}\end{shaded}\egroup 


    \item[{Example}]
  \leavevmode\bgroup\exampleFont \begin{shaded}\noindent\mbox{}{<\textbf{del}>}\mbox{}\newline 
\hspace*{6pt}{<\textbf{gap}\hspace*{6pt}{atLeast}="{4}"\hspace*{6pt}{atMost}="{8}"\mbox{}\newline 
\hspace*{6pt}\hspace*{6pt}{reason}="{illegible}"\hspace*{6pt}{unit}="{chars}"/>}\mbox{}\newline 
{</\textbf{del}>}\end{shaded}\egroup 


    \item[{Example}]
  \leavevmode\bgroup\exampleFont \begin{shaded}\noindent\mbox{}{<\textbf{gap}\hspace*{6pt}{extent}="{several lines}"\hspace*{6pt}{reason}="{lost}"/>}\end{shaded}\egroup 


    \item[{Content model}]
  \mbox{}\hfill\\[-10pt]\begin{Verbatim}[fontsize=\small]
<content>
 <alternate maxOccurs="unbounded"
  minOccurs="0">
  <classRef key="model.descLike"/>
  <classRef key="model.certLike"/>
 </alternate>
</content>
    
\end{Verbatim}

    \item[{Schema Declaration}]
  \mbox{}\hfill\\[-10pt]\begin{Verbatim}[fontsize=\small]
element gap
{
   att.global.attributes,
   att.timed.attributes,
   att.editLike.attributes,
   attribute reason { list { + } }?,
   attribute hand { text }?,
   attribute agent { text }?,
   ( model.descLike | model.certLike )*
}
\end{Verbatim}

\end{reflist}  \index{gb=<gb>|oddindex}
\begin{reflist}
\item[]\begin{specHead}{TEI.gb}{<gb> }(gathering begins) marks the point in a transcribed codex at which a new gathering or quire begins. [\xref{http://www.tei-c.org/release/doc/tei-p5-doc/en/html/CO.html\#CORS5}{3.10.3. Milestone Elements}]\end{specHead} 
    \item[{Module}]
  core
    \item[{Attributes}]
  Attributes att.global (\textit{@xml:id}, \textit{@n}, \textit{@xml:lang}, \textit{@xml:base}, \textit{@xml:space})  (att.global.rendition (\textit{@rend}, \textit{@style}, \textit{@rendition})) (att.global.linking (\textit{@corresp}, \textit{@synch}, \textit{@sameAs}, \textit{@copyOf}, \textit{@next}, \textit{@prev}, \textit{@exclude}, \textit{@select})) (att.global.analytic (\textit{@ana})) (att.global.facs (\textit{@facs})) (att.global.change (\textit{@change})) (att.global.responsibility (\textit{@cert}, \textit{@resp})) (att.global.source (\textit{@source})) att.typed (\textit{@type}, \textit{@subtype}) att.spanning (\textit{@spanTo}) att.breaking (\textit{@break}) 
    \item[{Member of}]
  model.milestoneLike
    \item[{Contained by}]
  
    \item[analysis: ]
   cl m phr s span w\par 
    \item[core: ]
   abbr add addrLine address author bibl biblScope cit citedRange corr date del distinct editor email emph expan foreign gloss head headItem headLabel hi imprint item l label lg list listBibl measure mentioned name note num orig p pubPlace publisher q quote ref reg resp rs said series sic soCalled sp speaker stage street term textLang time title unclear\par 
    \item[figures: ]
   cell figure table\par 
    \item[header: ]
   authority change classCode distributor edition extent funder geoDecl handNote language licence principal scriptNote sponsor typeNote\par 
    \item[linking: ]
   ab seg\par 
    \item[msdescription: ]
   accMat acquisition additions catchwords collation colophon condition custEvent decoNote explicit filiation finalRubric foliation heraldry incipit layout material msItem musicNotation objectType origDate origPlace origin provenance rubric secFol signatures source stamp summary support surrogates watermark\par 
    \item[namesdates: ]
   addName affiliation age birth bloc country death district education faith floruit forename genName geogFeat geogName langKnown nameLink nationality occupation offset org orgName persName person personGrp placeName region residence roleName settlement sex socecStatus surname\par 
    \item[textcrit: ]
   lem rdg wit witDetail\par 
    \item[textstructure: ]
   argument back body byline closer dateline div docAuthor docDate docEdition docImprint docTitle epigraph floatingText front group imprimatur opener postscript salute signed text titlePage titlePart trailer\par 
    \item[transcr: ]
   damage fw line metamark mod restore retrace secl sourceDoc subst supplied surface surfaceGrp surplus zone
    \item[{May contain}]
  Empty element
    \item[{Note}]
  \par
By convention, <gb> elements should appear at the start of the first page in the gathering. The global {\itshape n} attribute indicates the number or other value used to identify this gathering in a collation.\par
The {\itshape type} attribute may be used to further characterize the gathering in any respect.
    \item[{Example}]
  \leavevmode\bgroup\exampleFont \begin{shaded}\noindent\mbox{}{<\textbf{gb}\hspace*{6pt}{n}="{iii}"/>}\mbox{}\newline 
{<\textbf{pb}\hspace*{6pt}{n}="{2r}"/>}\mbox{}\newline 
\textit{<!-- material from page 2 recto of gathering iii here -->}\mbox{}\newline 
{<\textbf{pb}\hspace*{6pt}{n}="{2v}"/>}\mbox{}\newline 
\textit{<!-- material from page 2 verso of gathering iii here -->}\end{shaded}\egroup 


    \item[{Content model}]
  \fbox{\ttfamily <content>\newline
</content>\newline
    } 
    \item[{Schema Declaration}]
  \mbox{}\hfill\\[-10pt]\begin{Verbatim}[fontsize=\small]
element gb
{
   att.global.attributes,
   att.typed.attributes,
   att.spanning.attributes,
   att.breaking.attributes,
   empty
}
\end{Verbatim}

\end{reflist}  \index{genName=<genName>|oddindex}
\begin{reflist}
\item[]\begin{specHead}{TEI.genName}{<genName> }(generational name component) contains a name component used to distinguish otherwise similar names on the basis of the relative ages or generations of the persons named. [\xref{http://www.tei-c.org/release/doc/tei-p5-doc/en/html/ND.html\#NDPER}{13.2.1. Personal Names}]\end{specHead} 
    \item[{Module}]
  namesdates
    \item[{Attributes}]
  Attributes att.global (\textit{@xml:id}, \textit{@n}, \textit{@xml:lang}, \textit{@xml:base}, \textit{@xml:space})  (att.global.rendition (\textit{@rend}, \textit{@style}, \textit{@rendition})) (att.global.linking (\textit{@corresp}, \textit{@synch}, \textit{@sameAs}, \textit{@copyOf}, \textit{@next}, \textit{@prev}, \textit{@exclude}, \textit{@select})) (att.global.analytic (\textit{@ana})) (att.global.facs (\textit{@facs})) (att.global.change (\textit{@change})) (att.global.responsibility (\textit{@cert}, \textit{@resp})) (att.global.source (\textit{@source})) att.personal (\textit{@full}, \textit{@sort})  (att.naming (\textit{@role}, \textit{@nymRef}) (att.canonical (\textit{@key}, \textit{@ref})) ) att.typed (\textit{@type}, \textit{@subtype}) 
    \item[{Member of}]
  model.persNamePart
    \item[{Contained by}]
  
    \item[analysis: ]
   cl phr s span\par 
    \item[core: ]
   abbr add addrLine address author bibl biblScope citedRange corr date del desc distinct editor email emph expan foreign gloss head headItem headLabel hi item l label measure meeting mentioned name note num orig p pubPlace publisher q quote ref reg resp rs said sic soCalled speaker stage street term textLang time title unclear\par 
    \item[figures: ]
   cell figDesc\par 
    \item[header: ]
   authority catDesc change classCode correspAction creation distributor edition extent funder geoDecl handNote language licence principal rendition scriptNote sponsor tagUsage typeNote\par 
    \item[linking: ]
   ab seg\par 
    \item[msdescription: ]
   accMat acquisition additions catchwords collation colophon condition custEvent decoNote explicit filiation finalRubric foliation heraldry incipit layout material musicNotation objectType origDate origPlace origin provenance rubric secFol signatures source stamp summary support surrogates watermark\par 
    \item[namesdates: ]
   addName affiliation age birth bloc country death district education faith floruit forename genName geogFeat geogName langKnown nameLink nationality occupation offset org orgName persName placeName region residence roleName settlement sex socecStatus surname\par 
    \item[textcrit: ]
   lem rdg wit witDetail witness\par 
    \item[textstructure: ]
   byline closer dateline docAuthor docDate docEdition docImprint imprimatur opener salute signed titlePart trailer\par 
    \item[transcr: ]
   damage fw metamark mod restore retrace secl supplied surplus
    \item[{May contain}]
  
    \item[analysis: ]
   c cl interp interpGrp m pc phr s span spanGrp w\par 
    \item[core: ]
   abbr add address cb choice corr date del distinct email emph expan foreign gap gb gloss graphic hi index lb measure measureGrp media mentioned milestone name note num orig pb ptr ref reg rs sic soCalled term time title unclear\par 
    \item[figures: ]
   figure formula notatedMusic\par 
    \item[gaiji: ]
   g\par 
    \item[header: ]
   idno\par 
    \item[linking: ]
   alt altGrp anchor join joinGrp link linkGrp seg timeline\par 
    \item[msdescription: ]
   catchwords depth dim dimensions height heraldry locus locusGrp material objectType origDate origPlace secFol signatures stamp watermark width\par 
    \item[namesdates: ]
   addName affiliation bloc climate country district forename genName geo geogFeat geogName location nameLink offset orgName persName placeName population region roleName settlement state surname terrain trait\par 
    \item[textcrit: ]
   app witDetail\par 
    \item[transcr: ]
   addSpan am damage damageSpan delSpan ex fw handShift listTranspose metamark mod redo restore retrace secl space subst substJoin supplied surplus undo\par character data
    \item[{Example}]
  \leavevmode\bgroup\exampleFont \begin{shaded}\noindent\mbox{}{<\textbf{persName}>}\mbox{}\newline 
\hspace*{6pt}{<\textbf{forename}>}Charles{</\textbf{forename}>}\mbox{}\newline 
\hspace*{6pt}{<\textbf{genName}>}II{</\textbf{genName}>}\mbox{}\newline 
{</\textbf{persName}>}\end{shaded}\egroup 


    \item[{Example}]
  \leavevmode\bgroup\exampleFont \begin{shaded}\noindent\mbox{}{<\textbf{persName}>}\mbox{}\newline 
\hspace*{6pt}{<\textbf{surname}>}Pitt{</\textbf{surname}>}\mbox{}\newline 
\hspace*{6pt}{<\textbf{genName}>}the Younger{</\textbf{genName}>}\mbox{}\newline 
{</\textbf{persName}>}\end{shaded}\egroup 


    \item[{Content model}]
  \mbox{}\hfill\\[-10pt]\begin{Verbatim}[fontsize=\small]
<content>
 <macroRef key="macro.phraseSeq"/>
</content>
    
\end{Verbatim}

    \item[{Schema Declaration}]
  \mbox{}\hfill\\[-10pt]\begin{Verbatim}[fontsize=\small]
element genName
{
   att.global.attributes,
   att.personal.attributes,
   att.typed.attributes,
   macro.phraseSeq}
\end{Verbatim}

\end{reflist}  \index{geo=<geo>|oddindex}
\begin{reflist}
\item[]\begin{specHead}{TEI.geo}{<geo> }(geographical coordinates) contains any expression of a set of geographic coordinates, representing a point, line, or area on the surface of the earth in some notation. [\xref{http://www.tei-c.org/release/doc/tei-p5-doc/en/html/ND.html\#NDGEOGva}{13.3.4.1. Varieties of Location}]\end{specHead} 
    \item[{Module}]
  namesdates
    \item[{Attributes}]
  Attributes att.global (\textit{@xml:id}, \textit{@n}, \textit{@xml:lang}, \textit{@xml:base}, \textit{@xml:space})  (att.global.rendition (\textit{@rend}, \textit{@style}, \textit{@rendition})) (att.global.linking (\textit{@corresp}, \textit{@synch}, \textit{@sameAs}, \textit{@copyOf}, \textit{@next}, \textit{@prev}, \textit{@exclude}, \textit{@select})) (att.global.analytic (\textit{@ana})) (att.global.facs (\textit{@facs})) (att.global.change (\textit{@change})) (att.global.responsibility (\textit{@cert}, \textit{@resp})) (att.global.source (\textit{@source})) att.declaring (\textit{@decls}) 
    \item[{Member of}]
  model.measureLike
    \item[{Contained by}]
  
    \item[analysis: ]
   cl phr s span\par 
    \item[core: ]
   abbr add addrLine author bibl biblScope citedRange corr date del desc distinct editor email emph expan foreign gloss head headItem headLabel hi item l label measure measureGrp meeting mentioned name note num orig p pubPlace publisher q quote ref reg resp rs said sic soCalled speaker stage street term textLang time title unclear\par 
    \item[figures: ]
   cell figDesc\par 
    \item[header: ]
   authority catDesc change classCode creation distributor edition extent funder geoDecl handNote language licence principal rendition scriptNote sponsor tagUsage typeNote\par 
    \item[linking: ]
   ab seg\par 
    \item[msdescription: ]
   accMat acquisition additions catchwords collation colophon condition custEvent decoNote explicit filiation finalRubric foliation heraldry incipit layout material musicNotation objectType origDate origPlace origin provenance rubric secFol signatures source stamp summary support surrogates watermark\par 
    \item[namesdates: ]
   addName affiliation age birth bloc country death district education faith floruit forename genName geogFeat geogName langKnown location nameLink nationality occupation offset orgName persName placeName region residence roleName settlement sex socecStatus surname\par 
    \item[textcrit: ]
   lem rdg wit witDetail witness\par 
    \item[textstructure: ]
   byline closer dateline docAuthor docDate docEdition docImprint imprimatur opener salute signed titlePart trailer\par 
    \item[transcr: ]
   damage fw metamark mod restore retrace secl supplied surplus
    \item[{May contain}]
  Character data only
    \item[{Note}]
  \par
Uses of <geo> can be associated with a coordinate system, defined by a <geoDecl> element supplied in the TEI header, using the {\itshape decls} attribute. If no such link is made, the assumption is that the content of each <geo> element will be a pair of numbers separated by whitespace, to be interpreted as latitude followed by longitude according to the World Geodetic System.
    \item[{Example}]
  \leavevmode\bgroup\exampleFont \begin{shaded}\noindent\mbox{}{<\textbf{geoDecl}\hspace*{6pt}{datum}="{WGS84}"\hspace*{6pt}{xml:id}="{WGS}">}World Geodetic System{</\textbf{geoDecl}>}\mbox{}\newline 
{<\textbf{geoDecl}\hspace*{6pt}{datum}="{OSGB36}"\hspace*{6pt}{xml:id}="{OS}">}Ordnance Survey{</\textbf{geoDecl}>}\mbox{}\newline 
\textit{<!-- ... -->}\mbox{}\newline 
{<\textbf{location}>}\mbox{}\newline 
\hspace*{6pt}{<\textbf{desc}>}A tombstone plus six lines of\mbox{}\newline 
\hspace*{6pt}\hspace*{6pt} Anglo-Saxon text, built into the west tower (on the south side\mbox{}\newline 
\hspace*{6pt}\hspace*{6pt} of the archway, at 8 ft. above the ground) of the\mbox{}\newline 
\hspace*{6pt}\hspace*{6pt} Church of St. Mary-le-Wigford in Lincoln.{</\textbf{desc}>}\mbox{}\newline 
\hspace*{6pt}{<\textbf{geo}\hspace*{6pt}{decls}="{\#WGS}">}53.226658 -0.541254{</\textbf{geo}>}\mbox{}\newline 
\hspace*{6pt}{<\textbf{geo}\hspace*{6pt}{decls}="{\#OS}">}SK 97481 70947{</\textbf{geo}>}\mbox{}\newline 
{</\textbf{location}>}\end{shaded}\egroup 


    \item[{Example}]
  \leavevmode\bgroup\exampleFont \begin{shaded}\noindent\mbox{}{<\textbf{geo}>}41.687142 -74.870109{</\textbf{geo}>}\end{shaded}\egroup 


    \item[{Content model}]
  \fbox{\ttfamily <content>\newline
 <textNode/>\newline
</content>\newline
    } 
    \item[{Schema Declaration}]
  \mbox{}\hfill\\[-10pt]\begin{Verbatim}[fontsize=\small]
element geo { att.global.attributes, att.declaring.attributes, text }
\end{Verbatim}

\end{reflist}  \index{geoDecl=<geoDecl>|oddindex}\index{datum=@datum!<geoDecl>|oddindex}
\begin{reflist}
\item[]\begin{specHead}{TEI.geoDecl}{<geoDecl> }(geographic coordinates declaration) documents the notation and the datum used for geographic coordinates expressed as content of the <geo> element elsewhere within the document. [\xref{http://www.tei-c.org/release/doc/tei-p5-doc/en/html/HD.html\#HDGDECL}{2.3.8. The Geographic Coordinates Declaration}]\end{specHead} 
    \item[{Module}]
  header
    \item[{Attributes}]
  Attributes att.global (\textit{@xml:id}, \textit{@n}, \textit{@xml:lang}, \textit{@xml:base}, \textit{@xml:space})  (att.global.rendition (\textit{@rend}, \textit{@style}, \textit{@rendition})) (att.global.linking (\textit{@corresp}, \textit{@synch}, \textit{@sameAs}, \textit{@copyOf}, \textit{@next}, \textit{@prev}, \textit{@exclude}, \textit{@select})) (att.global.analytic (\textit{@ana})) (att.global.facs (\textit{@facs})) (att.global.change (\textit{@change})) (att.global.responsibility (\textit{@cert}, \textit{@resp})) (att.global.source (\textit{@source})) att.declarable (\textit{@default}) \hfil\\[-10pt]\begin{sansreflist}
    \item[@datum]
  supplies a commonly used code name for the datum employed.
\begin{reflist}
    \item[{Status}]
  Optional
    \item[{Datatype}]
  teidata.enumerated
    \item[{Suggested values include:}]
  \begin{description}

\item[{WGS84}](World Geodetic System) a pair of numbers to be interpreted as latitude followed by longitude according to the World Geodetic System.{[Default] }
\item[{MGRS}](Military Grid Reference System) the values supplied are geospatial entity object codes, based on
\item[{OSGB36}](ordnance survey great britain) the value supplied is to be interpreted as a British National Grid Reference.
\item[{ED50}](European Datum coordinate system) the value supplied is to be interpreted as latitude followed by longitude according to the European Datum coordinate system.
\end{description} 
\end{reflist}  
\end{sansreflist}  
    \item[{Member of}]
  model.encodingDescPart
    \item[{Contained by}]
  
    \item[header: ]
   encodingDesc
    \item[{May contain}]
  
    \item[analysis: ]
   c cl interp interpGrp m pc phr s span spanGrp w\par 
    \item[core: ]
   abbr add address cb choice corr date del distinct email emph expan foreign gap gb gloss graphic hi index lb measure measureGrp media mentioned milestone name note num orig pb ptr ref reg rs sic soCalled term time title unclear\par 
    \item[figures: ]
   figure formula notatedMusic\par 
    \item[gaiji: ]
   g\par 
    \item[header: ]
   idno\par 
    \item[linking: ]
   alt altGrp anchor join joinGrp link linkGrp seg timeline\par 
    \item[msdescription: ]
   catchwords depth dim dimensions height heraldry locus locusGrp material objectType origDate origPlace secFol signatures stamp watermark width\par 
    \item[namesdates: ]
   addName affiliation bloc climate country district forename genName geo geogFeat geogName location nameLink offset orgName persName placeName population region roleName settlement state surname terrain trait\par 
    \item[textcrit: ]
   app witDetail\par 
    \item[transcr: ]
   addSpan am damage damageSpan delSpan ex fw handShift listTranspose metamark mod redo restore retrace secl space subst substJoin supplied surplus undo\par character data
    \item[{Example}]
  \leavevmode\bgroup\exampleFont \begin{shaded}\noindent\mbox{}{<\textbf{geoDecl}\hspace*{6pt}{datum}="{OSGB36}"/>}\end{shaded}\egroup 


    \item[{Content model}]
  \mbox{}\hfill\\[-10pt]\begin{Verbatim}[fontsize=\small]
<content>
 <macroRef key="macro.phraseSeq"/>
</content>
    
\end{Verbatim}

    \item[{Schema Declaration}]
  \mbox{}\hfill\\[-10pt]\begin{Verbatim}[fontsize=\small]
element geoDecl
{
   att.global.attributes,
   att.declarable.attributes,
   attribute datum { "WGS84" | "MGRS" | "OSGB36" | "ED50" }?,
   macro.phraseSeq}
\end{Verbatim}

\end{reflist}  \index{geogFeat=<geogFeat>|oddindex}
\begin{reflist}
\item[]\begin{specHead}{TEI.geogFeat}{<geogFeat> }(geographical feature name) contains a common noun identifying some geographical feature contained within a geographic name, such as valley, mount, etc. [\xref{http://www.tei-c.org/release/doc/tei-p5-doc/en/html/ND.html\#NDPLAC}{13.2.3. Place Names}]\end{specHead} 
    \item[{Module}]
  namesdates
    \item[{Attributes}]
  Attributes att.datable (\textit{@calendar}, \textit{@period})  (att.datable.w3c (\textit{@when}, \textit{@notBefore}, \textit{@notAfter}, \textit{@from}, \textit{@to})) (att.datable.iso (\textit{@when-iso}, \textit{@notBefore-iso}, \textit{@notAfter-iso}, \textit{@from-iso}, \textit{@to-iso})) (att.datable.custom (\textit{@when-custom}, \textit{@notBefore-custom}, \textit{@notAfter-custom}, \textit{@from-custom}, \textit{@to-custom}, \textit{@datingPoint}, \textit{@datingMethod})) att.editLike (\textit{@evidence}, \textit{@instant})  (att.dimensions (\textit{@unit}, \textit{@quantity}, \textit{@extent}, \textit{@precision}, \textit{@scope}) (att.ranging (\textit{@atLeast}, \textit{@atMost}, \textit{@min}, \textit{@max}, \textit{@confidence})) ) att.global (\textit{@xml:id}, \textit{@n}, \textit{@xml:lang}, \textit{@xml:base}, \textit{@xml:space})  (att.global.rendition (\textit{@rend}, \textit{@style}, \textit{@rendition})) (att.global.linking (\textit{@corresp}, \textit{@synch}, \textit{@sameAs}, \textit{@copyOf}, \textit{@next}, \textit{@prev}, \textit{@exclude}, \textit{@select})) (att.global.analytic (\textit{@ana})) (att.global.facs (\textit{@facs})) (att.global.change (\textit{@change})) (att.global.responsibility (\textit{@cert}, \textit{@resp})) (att.global.source (\textit{@source})) att.naming (\textit{@role}, \textit{@nymRef})  (att.canonical (\textit{@key}, \textit{@ref})) att.typed (\textit{@type}, \textit{@subtype}) 
    \item[{Member of}]
  model.offsetLike
    \item[{Contained by}]
  
    \item[analysis: ]
   cl phr s span\par 
    \item[core: ]
   abbr add addrLine address author bibl biblScope citedRange corr date del desc distinct editor email emph expan foreign gloss head headItem headLabel hi item l label measure meeting mentioned name note num orig p pubPlace publisher q quote ref reg resp rs said sic soCalled speaker stage street term textLang time title unclear\par 
    \item[figures: ]
   cell figDesc\par 
    \item[header: ]
   authority catDesc change classCode correspAction creation distributor edition extent funder geoDecl handNote language licence principal rendition scriptNote sponsor tagUsage typeNote\par 
    \item[linking: ]
   ab seg\par 
    \item[msdescription: ]
   accMat acquisition additions catchwords collation colophon condition custEvent decoNote explicit filiation finalRubric foliation heraldry incipit layout material musicNotation objectType origDate origPlace origin provenance rubric secFol signatures source stamp summary support surrogates watermark\par 
    \item[namesdates: ]
   addName affiliation age birth bloc country death district education faith floruit forename genName geogFeat geogName langKnown location nameLink nationality occupation offset org orgName persName placeName region residence roleName settlement sex socecStatus surname\par 
    \item[textcrit: ]
   lem rdg wit witDetail witness\par 
    \item[textstructure: ]
   byline closer dateline docAuthor docDate docEdition docImprint imprimatur opener salute signed titlePart trailer\par 
    \item[transcr: ]
   damage fw metamark mod restore retrace secl supplied surplus
    \item[{May contain}]
  
    \item[analysis: ]
   c cl interp interpGrp m pc phr s span spanGrp w\par 
    \item[core: ]
   abbr add address cb choice corr date del distinct email emph expan foreign gap gb gloss graphic hi index lb measure measureGrp media mentioned milestone name note num orig pb ptr ref reg rs sic soCalled term time title unclear\par 
    \item[figures: ]
   figure formula notatedMusic\par 
    \item[gaiji: ]
   g\par 
    \item[header: ]
   idno\par 
    \item[linking: ]
   alt altGrp anchor join joinGrp link linkGrp seg timeline\par 
    \item[msdescription: ]
   catchwords depth dim dimensions height heraldry locus locusGrp material objectType origDate origPlace secFol signatures stamp watermark width\par 
    \item[namesdates: ]
   addName affiliation bloc climate country district forename genName geo geogFeat geogName location nameLink offset orgName persName placeName population region roleName settlement state surname terrain trait\par 
    \item[textcrit: ]
   app witDetail\par 
    \item[transcr: ]
   addSpan am damage damageSpan delSpan ex fw handShift listTranspose metamark mod redo restore retrace secl space subst substJoin supplied surplus undo\par character data
    \item[{Example}]
  \leavevmode\bgroup\exampleFont \begin{shaded}\noindent\mbox{}{<\textbf{geogName}>} The {<\textbf{geogFeat}>}vale{</\textbf{geogFeat}>} of White Horse{</\textbf{geogName}>}\end{shaded}\egroup 


    \item[{Content model}]
  \mbox{}\hfill\\[-10pt]\begin{Verbatim}[fontsize=\small]
<content>
 <macroRef key="macro.phraseSeq"/>
</content>
    
\end{Verbatim}

    \item[{Schema Declaration}]
  \mbox{}\hfill\\[-10pt]\begin{Verbatim}[fontsize=\small]
element geogFeat
{
   att.datable.attributes,
   att.editLike.attributes,
   att.global.attributes,
   att.naming.attributes,
   att.typed.attributes,
   macro.phraseSeq}
\end{Verbatim}

\end{reflist}  \index{geogName=<geogName>|oddindex}
\begin{reflist}
\item[]\begin{specHead}{TEI.geogName}{<geogName> }(geographical name) identifies a name associated with some geographical feature such as Windrush Valley or Mount Sinai. [\xref{http://www.tei-c.org/release/doc/tei-p5-doc/en/html/ND.html\#NDPLAC}{13.2.3. Place Names}]\end{specHead} 
    \item[{Module}]
  namesdates
    \item[{Attributes}]
  Attributes att.datable (\textit{@calendar}, \textit{@period})  (att.datable.w3c (\textit{@when}, \textit{@notBefore}, \textit{@notAfter}, \textit{@from}, \textit{@to})) (att.datable.iso (\textit{@when-iso}, \textit{@notBefore-iso}, \textit{@notAfter-iso}, \textit{@from-iso}, \textit{@to-iso})) (att.datable.custom (\textit{@when-custom}, \textit{@notBefore-custom}, \textit{@notAfter-custom}, \textit{@from-custom}, \textit{@to-custom}, \textit{@datingPoint}, \textit{@datingMethod})) att.editLike (\textit{@evidence}, \textit{@instant})  (att.dimensions (\textit{@unit}, \textit{@quantity}, \textit{@extent}, \textit{@precision}, \textit{@scope}) (att.ranging (\textit{@atLeast}, \textit{@atMost}, \textit{@min}, \textit{@max}, \textit{@confidence})) ) att.global (\textit{@xml:id}, \textit{@n}, \textit{@xml:lang}, \textit{@xml:base}, \textit{@xml:space})  (att.global.rendition (\textit{@rend}, \textit{@style}, \textit{@rendition})) (att.global.linking (\textit{@corresp}, \textit{@synch}, \textit{@sameAs}, \textit{@copyOf}, \textit{@next}, \textit{@prev}, \textit{@exclude}, \textit{@select})) (att.global.analytic (\textit{@ana})) (att.global.facs (\textit{@facs})) (att.global.change (\textit{@change})) (att.global.responsibility (\textit{@cert}, \textit{@resp})) (att.global.source (\textit{@source})) att.naming (\textit{@role}, \textit{@nymRef})  (att.canonical (\textit{@key}, \textit{@ref})) att.typed (\textit{@type}, \textit{@subtype}) 
    \item[{Member of}]
  model.placeNamePart
    \item[{Contained by}]
  
    \item[analysis: ]
   cl phr s span\par 
    \item[core: ]
   abbr add addrLine address author bibl biblScope citedRange corr date del desc distinct editor email emph expan foreign gloss head headItem headLabel hi item l label measure meeting mentioned name note num orig p pubPlace publisher q quote ref reg resp rs said sic soCalled speaker stage street term textLang time title unclear\par 
    \item[figures: ]
   cell figDesc\par 
    \item[header: ]
   authority catDesc change classCode correspAction creation distributor edition extent funder geoDecl handNote language licence principal rendition scriptNote sponsor tagUsage typeNote\par 
    \item[linking: ]
   ab seg\par 
    \item[msdescription: ]
   accMat acquisition additions altIdentifier catchwords collation colophon condition custEvent decoNote explicit filiation finalRubric foliation heraldry incipit layout material msIdentifier musicNotation objectType origDate origPlace origin provenance rubric secFol signatures source stamp summary support surrogates watermark\par 
    \item[namesdates: ]
   addName affiliation age birth bloc country death district education faith floruit forename genName geogFeat geogName langKnown location nameLink nationality occupation offset org orgName persName place placeName region residence roleName settlement sex socecStatus surname\par 
    \item[textcrit: ]
   lem rdg wit witDetail witness\par 
    \item[textstructure: ]
   byline closer dateline docAuthor docDate docEdition docImprint imprimatur opener salute signed titlePart trailer\par 
    \item[transcr: ]
   damage fw metamark mod restore retrace secl supplied surplus
    \item[{May contain}]
  
    \item[analysis: ]
   c cl interp interpGrp m pc phr s span spanGrp w\par 
    \item[core: ]
   abbr add address cb choice corr date del distinct email emph expan foreign gap gb gloss graphic hi index lb measure measureGrp media mentioned milestone name note num orig pb ptr ref reg rs sic soCalled term time title unclear\par 
    \item[figures: ]
   figure formula notatedMusic\par 
    \item[gaiji: ]
   g\par 
    \item[header: ]
   idno\par 
    \item[linking: ]
   alt altGrp anchor join joinGrp link linkGrp seg timeline\par 
    \item[msdescription: ]
   catchwords depth dim dimensions height heraldry locus locusGrp material objectType origDate origPlace secFol signatures stamp watermark width\par 
    \item[namesdates: ]
   addName affiliation bloc climate country district forename genName geo geogFeat geogName location nameLink offset orgName persName placeName population region roleName settlement state surname terrain trait\par 
    \item[textcrit: ]
   app witDetail\par 
    \item[transcr: ]
   addSpan am damage damageSpan delSpan ex fw handShift listTranspose metamark mod redo restore retrace secl space subst substJoin supplied surplus undo\par character data
    \item[{Example}]
  \leavevmode\bgroup\exampleFont \begin{shaded}\noindent\mbox{}{<\textbf{geogName}>}\mbox{}\newline 
\hspace*{6pt}{<\textbf{geogFeat}>}Mount{</\textbf{geogFeat}>}\mbox{}\newline 
\hspace*{6pt}{<\textbf{name}>}Sinai{</\textbf{name}>}\mbox{}\newline 
{</\textbf{geogName}>}\end{shaded}\egroup 


    \item[{Content model}]
  \mbox{}\hfill\\[-10pt]\begin{Verbatim}[fontsize=\small]
<content>
 <macroRef key="macro.phraseSeq"/>
</content>
    
\end{Verbatim}

    \item[{Schema Declaration}]
  \mbox{}\hfill\\[-10pt]\begin{Verbatim}[fontsize=\small]
element geogName
{
   att.datable.attributes,
   att.editLike.attributes,
   att.global.attributes,
   att.naming.attributes,
   att.typed.attributes,
   macro.phraseSeq}
\end{Verbatim}

\end{reflist}  \index{gloss=<gloss>|oddindex}
\begin{reflist}
\item[]\begin{specHead}{TEI.gloss}{<gloss> }identifies a phrase or word used to provide a gloss or definition for some other word or phrase. [\xref{http://www.tei-c.org/release/doc/tei-p5-doc/en/html/CO.html\#COHQU}{3.3.4. Terms, Glosses, Equivalents, and Descriptions} \xref{http://www.tei-c.org/release/doc/tei-p5-doc/en/html/TD.html\#TDcrystalsCEdc}{22.4.1. Description of Components}]\end{specHead} 
    \item[{Module}]
  core
    \item[{Attributes}]
  Attributes att.global (\textit{@xml:id}, \textit{@n}, \textit{@xml:lang}, \textit{@xml:base}, \textit{@xml:space})  (att.global.rendition (\textit{@rend}, \textit{@style}, \textit{@rendition})) (att.global.linking (\textit{@corresp}, \textit{@synch}, \textit{@sameAs}, \textit{@copyOf}, \textit{@next}, \textit{@prev}, \textit{@exclude}, \textit{@select})) (att.global.analytic (\textit{@ana})) (att.global.facs (\textit{@facs})) (att.global.change (\textit{@change})) (att.global.responsibility (\textit{@cert}, \textit{@resp})) (att.global.source (\textit{@source})) att.declaring (\textit{@decls}) att.translatable (\textit{@versionDate}) att.typed (\textit{@type}, \textit{@subtype}) att.pointing (\textit{@targetLang}, \textit{@target}, \textit{@evaluate}) att.cReferencing (\textit{@cRef}) 
    \item[{Member of}]
  model.emphLike model.glossLike
    \item[{Contained by}]
  
    \item[analysis: ]
   cl phr s span\par 
    \item[core: ]
   abbr add addrLine author bibl biblScope citedRange corr date del desc distinct editor email emph expan foreign gloss head headItem headLabel hi item l label measure meeting mentioned name note num orig p pubPlace publisher q quote ref reg resp rs said sic soCalled speaker stage street term textLang time title unclear\par 
    \item[figures: ]
   cell figDesc\par 
    \item[header: ]
   authority catDesc category change classCode creation distributor edition extent funder geoDecl handNote language licence principal rendition scriptNote sponsor tagUsage taxonomy typeNote\par 
    \item[linking: ]
   ab joinGrp seg\par 
    \item[msdescription: ]
   accMat acquisition additions catchwords collation colophon condition custEvent decoNote explicit filiation finalRubric foliation heraldry incipit layout material musicNotation objectType origDate origPlace origin provenance rubric secFol signatures source stamp summary support surrogates watermark\par 
    \item[namesdates: ]
   addName affiliation age birth bloc country death district education faith floruit forename genName geogFeat geogName langKnown nameLink nationality occupation offset orgName persName placeName region residence roleName settlement sex socecStatus surname\par 
    \item[textcrit: ]
   lem rdg wit witDetail witness\par 
    \item[textstructure: ]
   byline closer dateline docAuthor docDate docEdition docImprint imprimatur opener salute signed titlePart trailer\par 
    \item[transcr: ]
   damage fw metamark mod restore retrace secl supplied surplus
    \item[{May contain}]
  
    \item[analysis: ]
   c cl interp interpGrp m pc phr s span spanGrp w\par 
    \item[core: ]
   abbr add address cb choice corr date del distinct email emph expan foreign gap gb gloss graphic hi index lb measure measureGrp media mentioned milestone name note num orig pb ptr ref reg rs sic soCalled term time title unclear\par 
    \item[figures: ]
   figure formula notatedMusic\par 
    \item[gaiji: ]
   g\par 
    \item[header: ]
   idno\par 
    \item[linking: ]
   alt altGrp anchor join joinGrp link linkGrp seg timeline\par 
    \item[msdescription: ]
   catchwords depth dim dimensions height heraldry locus locusGrp material objectType origDate origPlace secFol signatures stamp watermark width\par 
    \item[namesdates: ]
   addName affiliation bloc climate country district forename genName geo geogFeat geogName location nameLink offset orgName persName placeName population region roleName settlement state surname terrain trait\par 
    \item[textcrit: ]
   app witDetail\par 
    \item[transcr: ]
   addSpan am damage damageSpan delSpan ex fw handShift listTranspose metamark mod redo restore retrace secl space subst substJoin supplied surplus undo\par character data
    \item[{Note}]
  \par
The {\itshape target} and {\itshape cRef} attributes are mutually exclusive.
    \item[{Example}]
  \leavevmode\bgroup\exampleFont \begin{shaded}\noindent\mbox{}We may define {<\textbf{term}\hspace*{6pt}{rend}="{sc}"\hspace*{6pt}{xml:id}="{tdpv}">}discoursal point of view{</\textbf{term}>} as \mbox{}\newline 
{<\textbf{gloss}\hspace*{6pt}{target}="{\#tdpv}">}the relationship, expressed\mbox{}\newline 
 through discourse structure, between the implied author or some other addresser, and the\mbox{}\newline 
 fiction.{</\textbf{gloss}>}\end{shaded}\egroup 


    \item[{Content model}]
  \mbox{}\hfill\\[-10pt]\begin{Verbatim}[fontsize=\small]
<content>
 <macroRef key="macro.phraseSeq"/>
</content>
    
\end{Verbatim}

    \item[{Schema Declaration}]
  \mbox{}\hfill\\[-10pt]\begin{Verbatim}[fontsize=\small]
element gloss
{
   att.global.attributes,
   att.declaring.attributes,
   att.translatable.attributes,
   att.typed.attributes,
   att.pointing.attributes,
   att.cReferencing.attributes,
   macro.phraseSeq}
\end{Verbatim}

\end{reflist}  \index{glyph=<glyph>|oddindex}
\begin{reflist}
\item[]\begin{specHead}{TEI.glyph}{<glyph> }(character glyph) provides descriptive information about a character glyph. [\xref{http://www.tei-c.org/release/doc/tei-p5-doc/en/html/WD.html\#D25-20}{5.2. Markup Constructs for Representation of Characters and Glyphs}]\end{specHead} 
    \item[{Module}]
  gaiji
    \item[{Attributes}]
  Attributes att.global (\textit{@xml:id}, \textit{@n}, \textit{@xml:lang}, \textit{@xml:base}, \textit{@xml:space})  (att.global.rendition (\textit{@rend}, \textit{@style}, \textit{@rendition})) (att.global.linking (\textit{@corresp}, \textit{@synch}, \textit{@sameAs}, \textit{@copyOf}, \textit{@next}, \textit{@prev}, \textit{@exclude}, \textit{@select})) (att.global.analytic (\textit{@ana})) (att.global.facs (\textit{@facs})) (att.global.change (\textit{@change})) (att.global.responsibility (\textit{@cert}, \textit{@resp})) (att.global.source (\textit{@source}))
    \item[{Contained by}]
  
    \item[gaiji: ]
   charDecl
    \item[{May contain}]
  
    \item[core: ]
   desc graphic media note\par 
    \item[figures: ]
   figure formula\par 
    \item[gaiji: ]
   charProp glyphName mapping\par 
    \item[textcrit: ]
   witDetail
    \item[{Example}]
  \leavevmode\bgroup\exampleFont \begin{shaded}\noindent\mbox{}{<\textbf{glyph}\hspace*{6pt}{xml:id}="{rstroke}">}\mbox{}\newline 
\hspace*{6pt}{<\textbf{glyphName}>}LATIN SMALL LETTER R WITH A FUNNY STROKE{</\textbf{glyphName}>}\mbox{}\newline 
\hspace*{6pt}{<\textbf{charProp}>}\mbox{}\newline 
\hspace*{6pt}\hspace*{6pt}{<\textbf{localName}>}entity{</\textbf{localName}>}\mbox{}\newline 
\hspace*{6pt}\hspace*{6pt}{<\textbf{value}>}rstroke{</\textbf{value}>}\mbox{}\newline 
\hspace*{6pt}{</\textbf{charProp}>}\mbox{}\newline 
\hspace*{6pt}{<\textbf{figure}>}\mbox{}\newline 
\hspace*{6pt}\hspace*{6pt}{<\textbf{graphic}\hspace*{6pt}{url}="{glyph-rstroke.png}"/>}\mbox{}\newline 
\hspace*{6pt}{</\textbf{figure}>}\mbox{}\newline 
{</\textbf{glyph}>}\end{shaded}\egroup 


    \item[{Content model}]
  \mbox{}\hfill\\[-10pt]\begin{Verbatim}[fontsize=\small]
<content>
 <sequence>
  <elementRef key="glyphName" minOccurs="0"/>
  <classRef key="model.descLike"
   maxOccurs="unbounded" minOccurs="0"/>
  <elementRef key="charProp"
   maxOccurs="unbounded" minOccurs="0"/>
  <elementRef key="mapping"
   maxOccurs="unbounded" minOccurs="0"/>
  <elementRef key="figure"
   maxOccurs="unbounded" minOccurs="0"/>
  <classRef key="model.graphicLike"
   maxOccurs="unbounded" minOccurs="0"/>
  <classRef key="model.noteLike"
   maxOccurs="unbounded" minOccurs="0"/>
 </sequence>
</content>
    
\end{Verbatim}

    \item[{Schema Declaration}]
  \mbox{}\hfill\\[-10pt]\begin{Verbatim}[fontsize=\small]
element glyph
{
   att.global.attributes,
   (
      glyphName?,
      model.descLike*,
      charProp*,
      mapping*,
      figure*,
      model.graphicLike*,
      model.noteLike*
   )
}
\end{Verbatim}

\end{reflist}  \index{glyphName=<glyphName>|oddindex}
\begin{reflist}
\item[]\begin{specHead}{TEI.glyphName}{<glyphName> }(character glyph name) contains the name of a glyph, expressed following Unicode conventions for character names. [\xref{http://www.tei-c.org/release/doc/tei-p5-doc/en/html/WD.html\#D25-20}{5.2. Markup Constructs for Representation of Characters and Glyphs}]\end{specHead} 
    \item[{Module}]
  gaiji
    \item[{Attributes}]
  Attributes att.global (\textit{@xml:id}, \textit{@n}, \textit{@xml:lang}, \textit{@xml:base}, \textit{@xml:space})  (att.global.rendition (\textit{@rend}, \textit{@style}, \textit{@rendition})) (att.global.linking (\textit{@corresp}, \textit{@synch}, \textit{@sameAs}, \textit{@copyOf}, \textit{@next}, \textit{@prev}, \textit{@exclude}, \textit{@select})) (att.global.analytic (\textit{@ana})) (att.global.facs (\textit{@facs})) (att.global.change (\textit{@change})) (att.global.responsibility (\textit{@cert}, \textit{@resp})) (att.global.source (\textit{@source}))
    \item[{Contained by}]
  
    \item[gaiji: ]
   glyph
    \item[{May contain}]
  Character data only
    \item[{Note}]
  \par
For characters of non-ideographic scripts, a name following the conventions for Unicode names should be chosen. For ideographic scripts, an \textit{Ideographic Description Sequence} (IDS) as described in Chapter 10.1 of the Unicode Standard is recommended where possible. Projects working in similar fields are recommended to coordinate and publish their list of <glyphName>s to facilitate data exchange.
    \item[{Example}]
  \leavevmode\bgroup\exampleFont \begin{shaded}\noindent\mbox{}{<\textbf{glyphName}>}CIRCLED IDEOGRAPH 4EBA{</\textbf{glyphName}>}\end{shaded}\egroup 


    \item[{Content model}]
  \fbox{\ttfamily <content>\newline
 <textNode/>\newline
</content>\newline
    } 
    \item[{Schema Declaration}]
  \fbox{\ttfamily element glyphName ❴ att.global.attributes, text ❵} 
\end{reflist}  \index{graphic=<graphic>|oddindex}
\begin{reflist}
\item[]\begin{specHead}{TEI.graphic}{<graphic> }indicates the location of a graphic or illustration, either forming part of a text, or providing an image of it. [\xref{http://www.tei-c.org/release/doc/tei-p5-doc/en/html/CO.html\#COGR}{3.9. Graphics and Other Non-textual Components} \xref{http://www.tei-c.org/release/doc/tei-p5-doc/en/html/PH.html\#PHFAX}{11.1. Digital Facsimiles}]\end{specHead} 
    \item[{Module}]
  core
    \item[{Attributes}]
  Attributes att.global (\textit{@xml:id}, \textit{@n}, \textit{@xml:lang}, \textit{@xml:base}, \textit{@xml:space})  (att.global.rendition (\textit{@rend}, \textit{@style}, \textit{@rendition})) (att.global.linking (\textit{@corresp}, \textit{@synch}, \textit{@sameAs}, \textit{@copyOf}, \textit{@next}, \textit{@prev}, \textit{@exclude}, \textit{@select})) (att.global.analytic (\textit{@ana})) (att.global.facs (\textit{@facs})) (att.global.change (\textit{@change})) (att.global.responsibility (\textit{@cert}, \textit{@resp})) (att.global.source (\textit{@source})) att.media (\textit{@width}, \textit{@height}, \textit{@scale})  (att.internetMedia (\textit{@mimeType})) att.resourced (\textit{@url}) att.declaring (\textit{@decls}) 
    \item[{Member of}]
  model.graphicLike model.titlepagePart 
    \item[{Contained by}]
  
    \item[analysis: ]
   cl phr s\par 
    \item[core: ]
   abbr add addrLine author biblScope citedRange corr date del distinct editor email emph expan foreign gloss head headItem headLabel hi item l label measure mentioned name note num orig p pubPlace publisher q quote ref reg rs said sic soCalled speaker stage street term textLang time title unclear\par 
    \item[figures: ]
   cell figure formula notatedMusic table\par 
    \item[gaiji: ]
   char glyph\par 
    \item[header: ]
   change distributor edition extent geoDecl handNote licence scriptNote typeNote\par 
    \item[linking: ]
   ab seg\par 
    \item[msdescription: ]
   accMat acquisition additions catchwords collation colophon condition custEvent decoNote explicit filiation finalRubric foliation heraldry incipit layout material msItem musicNotation objectType origDate origPlace origin provenance rubric secFol signatures source stamp summary support surrogates watermark\par 
    \item[namesdates: ]
   addName affiliation birth bloc country death district education faith floruit forename genName geogFeat geogName nameLink nationality occupation offset orgName persName placeName region residence roleName settlement sex socecStatus surname\par 
    \item[textcrit: ]
   lem rdg wit witDetail\par 
    \item[textstructure: ]
   byline closer dateline docAuthor docDate docEdition docImprint imprimatur opener salute signed titlePage titlePart trailer\par 
    \item[transcr: ]
   damage facsimile fw metamark mod restore retrace secl sourceDoc supplied surface surplus zone
    \item[{May contain}]
  
    \item[core: ]
   desc
    \item[{Note}]
  \par
The {\itshape mimeType} attribute should be used to supply the MIME media type of the image specified by the {\itshape url} attribute.\par
Within the body of a text, a <graphic> element indicates the presence of a graphic component in the source itself. Within the context of a <facsimile> or <sourceDoc> element, however, a <graphic> element provides an additional digital representation of some part of the source being encoded.
    \item[{Example}]
  \leavevmode\bgroup\exampleFont \begin{shaded}\noindent\mbox{}{<\textbf{figure}>}\mbox{}\newline 
\hspace*{6pt}{<\textbf{graphic}\hspace*{6pt}{url}="{fig1.png}"/>}\mbox{}\newline 
\hspace*{6pt}{<\textbf{head}>}Figure One: The View from the Bridge{</\textbf{head}>}\mbox{}\newline 
\hspace*{6pt}{<\textbf{figDesc}>}A Whistleresque view showing four or five sailing boats in the foreground, and a\mbox{}\newline 
\hspace*{6pt}\hspace*{6pt} series of buoys strung out between them.{</\textbf{figDesc}>}\mbox{}\newline 
{</\textbf{figure}>}\end{shaded}\egroup 


    \item[{Example}]
  \leavevmode\bgroup\exampleFont \begin{shaded}\noindent\mbox{}{<\textbf{facsimile}>}\mbox{}\newline 
\hspace*{6pt}{<\textbf{surfaceGrp}\hspace*{6pt}{n}="{leaf1}">}\mbox{}\newline 
\hspace*{6pt}\hspace*{6pt}{<\textbf{surface}>}\mbox{}\newline 
\hspace*{6pt}\hspace*{6pt}\hspace*{6pt}{<\textbf{graphic}\hspace*{6pt}{url}="{page1.png}"/>}\mbox{}\newline 
\hspace*{6pt}\hspace*{6pt}{</\textbf{surface}>}\mbox{}\newline 
\hspace*{6pt}\hspace*{6pt}{<\textbf{surface}>}\mbox{}\newline 
\hspace*{6pt}\hspace*{6pt}\hspace*{6pt}{<\textbf{graphic}\hspace*{6pt}{url}="{page2-highRes.png}"/>}\mbox{}\newline 
\hspace*{6pt}\hspace*{6pt}\hspace*{6pt}{<\textbf{graphic}\hspace*{6pt}{url}="{page2-lowRes.png}"/>}\mbox{}\newline 
\hspace*{6pt}\hspace*{6pt}{</\textbf{surface}>}\mbox{}\newline 
\hspace*{6pt}{</\textbf{surfaceGrp}>}\mbox{}\newline 
{</\textbf{facsimile}>}\end{shaded}\egroup 


    \item[{Content model}]
  \mbox{}\hfill\\[-10pt]\begin{Verbatim}[fontsize=\small]
<content>
 <classRef key="model.descLike"
  maxOccurs="unbounded" minOccurs="0"/>
</content>
    
\end{Verbatim}

    \item[{Schema Declaration}]
  \mbox{}\hfill\\[-10pt]\begin{Verbatim}[fontsize=\small]
element graphic
{
   att.global.attributes,
   att.media.attributes,
   att.resourced.attributes,
   att.declaring.attributes,
   model.descLike*
}
\end{Verbatim}

\end{reflist}  \index{group=<group>|oddindex}
\begin{reflist}
\item[]\begin{specHead}{TEI.group}{<group> }contains the body of a composite text, grouping together a sequence of distinct texts (or groups of such texts) which are regarded as a unit for some purpose, for example the collected works of an author, a sequence of prose essays, etc. [\xref{http://www.tei-c.org/release/doc/tei-p5-doc/en/html/DS.html\#DS}{4. Default Text Structure} \xref{http://www.tei-c.org/release/doc/tei-p5-doc/en/html/DS.html\#DSGRP}{4.3.1. Grouped Texts} \xref{http://www.tei-c.org/release/doc/tei-p5-doc/en/html/CC.html\#CCDEF}{15.1. Varieties of Composite Text}]\end{specHead} 
    \item[{Module}]
  textstructure
    \item[{Attributes}]
  Attributes att.global (\textit{@xml:id}, \textit{@n}, \textit{@xml:lang}, \textit{@xml:base}, \textit{@xml:space})  (att.global.rendition (\textit{@rend}, \textit{@style}, \textit{@rendition})) (att.global.linking (\textit{@corresp}, \textit{@synch}, \textit{@sameAs}, \textit{@copyOf}, \textit{@next}, \textit{@prev}, \textit{@exclude}, \textit{@select})) (att.global.analytic (\textit{@ana})) (att.global.facs (\textit{@facs})) (att.global.change (\textit{@change})) (att.global.responsibility (\textit{@cert}, \textit{@resp})) (att.global.source (\textit{@source})) att.declaring (\textit{@decls}) att.typed (\textit{@type}, \textit{@subtype}) 
    \item[{Contained by}]
  
    \item[textstructure: ]
   floatingText group text
    \item[{May contain}]
  
    \item[analysis: ]
   interp interpGrp span spanGrp\par 
    \item[core: ]
   cb gap gb head index lb meeting milestone note pb\par 
    \item[figures: ]
   figure notatedMusic\par 
    \item[linking: ]
   alt altGrp anchor join joinGrp link linkGrp timeline\par 
    \item[textcrit: ]
   app witDetail\par 
    \item[textstructure: ]
   argument byline closer dateline docAuthor docDate epigraph group opener postscript salute signed text trailer\par 
    \item[transcr: ]
   addSpan damageSpan delSpan fw listTranspose metamark space substJoin
    \item[{Example}]
  \leavevmode\bgroup\exampleFont \begin{shaded}\noindent\mbox{}{<\textbf{text}>}\mbox{}\newline 
\textit{<!-- Section on Alexander Pope starts -->}\mbox{}\newline 
\hspace*{6pt}{<\textbf{front}>}\mbox{}\newline 
\textit{<!-- biographical notice by editor -->}\mbox{}\newline 
\hspace*{6pt}{</\textbf{front}>}\mbox{}\newline 
\hspace*{6pt}{<\textbf{group}>}\mbox{}\newline 
\hspace*{6pt}\hspace*{6pt}{<\textbf{text}>}\mbox{}\newline 
\textit{<!-- first poem -->}\mbox{}\newline 
\hspace*{6pt}\hspace*{6pt}{</\textbf{text}>}\mbox{}\newline 
\hspace*{6pt}\hspace*{6pt}{<\textbf{text}>}\mbox{}\newline 
\textit{<!-- second poem -->}\mbox{}\newline 
\hspace*{6pt}\hspace*{6pt}{</\textbf{text}>}\mbox{}\newline 
\hspace*{6pt}{</\textbf{group}>}\mbox{}\newline 
{</\textbf{text}>}\mbox{}\newline 
\textit{<!-- end of Pope section-->}\end{shaded}\egroup 


    \item[{Content model}]
  \mbox{}\hfill\\[-10pt]\begin{Verbatim}[fontsize=\small]
<content>
 <sequence>
  <alternate maxOccurs="unbounded"
   minOccurs="0">
   <classRef key="model.divTop"/>
   <classRef key="model.global"/>
  </alternate>
  <sequence>
   <alternate>
    <elementRef key="text"/>
    <elementRef key="group"/>
   </alternate>
   <alternate maxOccurs="unbounded"
    minOccurs="0">
    <elementRef key="text"/>
    <elementRef key="group"/>
    <classRef key="model.global"/>
   </alternate>
  </sequence>
  <classRef key="model.divBottom"
   maxOccurs="unbounded" minOccurs="0"/>
 </sequence>
</content>
    
\end{Verbatim}

    \item[{Schema Declaration}]
  \mbox{}\hfill\\[-10pt]\begin{Verbatim}[fontsize=\small]
element group
{
   att.global.attributes,
   att.declaring.attributes,
   att.typed.attributes,
   (
      ( model.divTop | model.global )*,
      ( ( text | group ), ( text | group | model.global )* ),
      model.divBottom*
   )
}
\end{Verbatim}

\end{reflist}  \index{handDesc=<handDesc>|oddindex}\index{hands=@hands!<handDesc>|oddindex}
\begin{reflist}
\item[]\begin{specHead}{TEI.handDesc}{<handDesc> }(description of hands) contains a description of all the different kinds of writing used in a manuscript. [\xref{http://www.tei-c.org/release/doc/tei-p5-doc/en/html/MS.html\#msph2}{10.7.2. Writing, Decoration, and Other Notations}]\end{specHead} 
    \item[{Module}]
  msdescription
    \item[{Attributes}]
  Attributes att.global (\textit{@xml:id}, \textit{@n}, \textit{@xml:lang}, \textit{@xml:base}, \textit{@xml:space})  (att.global.rendition (\textit{@rend}, \textit{@style}, \textit{@rendition})) (att.global.linking (\textit{@corresp}, \textit{@synch}, \textit{@sameAs}, \textit{@copyOf}, \textit{@next}, \textit{@prev}, \textit{@exclude}, \textit{@select})) (att.global.analytic (\textit{@ana})) (att.global.facs (\textit{@facs})) (att.global.change (\textit{@change})) (att.global.responsibility (\textit{@cert}, \textit{@resp})) (att.global.source (\textit{@source})) \hfil\\[-10pt]\begin{sansreflist}
    \item[@hands]
  specifies the number of distinct hands identified within the manuscript
\begin{reflist}
    \item[{Status}]
  Optional
    \item[{Datatype}]
  teidata.count
\end{reflist}  
\end{sansreflist}  
    \item[{Member of}]
  model.physDescPart
    \item[{Contained by}]
  
    \item[msdescription: ]
   physDesc
    \item[{May contain}]
  
    \item[core: ]
   p\par 
    \item[header: ]
   handNote\par 
    \item[linking: ]
   ab\par 
    \item[msdescription: ]
   summary
    \item[{Example}]
  \leavevmode\bgroup\exampleFont \begin{shaded}\noindent\mbox{}{<\textbf{handDesc}>}\mbox{}\newline 
\hspace*{6pt}{<\textbf{handNote}\hspace*{6pt}{scope}="{major}">}Written throughout in {<\textbf{term}>}angelicana formata{</\textbf{term}>}.{</\textbf{handNote}>}\mbox{}\newline 
{</\textbf{handDesc}>}\end{shaded}\egroup 


    \item[{Example}]
  \leavevmode\bgroup\exampleFont \begin{shaded}\noindent\mbox{}{<\textbf{handDesc}\hspace*{6pt}{hands}="{2}">}\mbox{}\newline 
\hspace*{6pt}{<\textbf{p}>}The manuscript is written in two contemporary hands, otherwise\mbox{}\newline 
\hspace*{6pt}\hspace*{6pt} unknown, but clearly those of practised scribes. Hand I writes\mbox{}\newline 
\hspace*{6pt}\hspace*{6pt} ff. 1r-22v and hand II ff. 23 and 24. Some scholars, notably\mbox{}\newline 
\hspace*{6pt}\hspace*{6pt} Verner Dahlerup and Hreinn Benediktsson, have argued for a third hand\mbox{}\newline 
\hspace*{6pt}\hspace*{6pt} on f. 24, but the evidence for this is insubstantial.{</\textbf{p}>}\mbox{}\newline 
{</\textbf{handDesc}>}\end{shaded}\egroup 


    \item[{Content model}]
  \mbox{}\hfill\\[-10pt]\begin{Verbatim}[fontsize=\small]
<content>
 <alternate>
  <classRef key="model.pLike"
   maxOccurs="unbounded" minOccurs="1"/>
  <sequence>
   <elementRef key="summary" minOccurs="0"/>
   <elementRef key="handNote"
    maxOccurs="unbounded" minOccurs="1"/>
  </sequence>
 </alternate>
</content>
    
\end{Verbatim}

    \item[{Schema Declaration}]
  \mbox{}\hfill\\[-10pt]\begin{Verbatim}[fontsize=\small]
element handDesc
{
   att.global.attributes,
   attribute hands { text }?,
   ( model.pLike+ | ( summary?, handNote+ ) )
}
\end{Verbatim}

\end{reflist}  \index{handNote=<handNote>|oddindex}
\begin{reflist}
\item[]\begin{specHead}{TEI.handNote}{<handNote> }(note on hand) describes a particular style or hand distinguished within a manuscript. [\xref{http://www.tei-c.org/release/doc/tei-p5-doc/en/html/MS.html\#msph2}{10.7.2. Writing, Decoration, and Other Notations}]\end{specHead} 
    \item[{Module}]
  header
    \item[{Attributes}]
  Attributes att.global (\textit{@xml:id}, \textit{@n}, \textit{@xml:lang}, \textit{@xml:base}, \textit{@xml:space})  (att.global.rendition (\textit{@rend}, \textit{@style}, \textit{@rendition})) (att.global.linking (\textit{@corresp}, \textit{@synch}, \textit{@sameAs}, \textit{@copyOf}, \textit{@next}, \textit{@prev}, \textit{@exclude}, \textit{@select})) (att.global.analytic (\textit{@ana})) (att.global.facs (\textit{@facs})) (att.global.change (\textit{@change})) (att.global.responsibility (\textit{@cert}, \textit{@resp})) (att.global.source (\textit{@source})) att.handFeatures (\textit{@scribe}, \textit{@scribeRef}, \textit{@script}, \textit{@scriptRef}, \textit{@medium}, \textit{@scope}) 
    \item[{Contained by}]
  
    \item[msdescription: ]
   handDesc\par 
    \item[transcr: ]
   handNotes
    \item[{May contain}]
  
    \item[analysis: ]
   c cl interp interpGrp m pc phr s span spanGrp w\par 
    \item[core: ]
   abbr add address bibl biblStruct cb choice cit corr date del desc distinct email emph expan foreign gap gb gloss graphic hi index l label lb lg list listBibl measure measureGrp media mentioned milestone name note num orig p pb ptr q quote ref reg rs said sic soCalled sp stage term time title unclear\par 
    \item[figures: ]
   figure formula notatedMusic table\par 
    \item[gaiji: ]
   g\par 
    \item[header: ]
   biblFull idno\par 
    \item[linking: ]
   ab alt altGrp anchor join joinGrp link linkGrp seg timeline\par 
    \item[msdescription: ]
   catchwords depth dim dimensions height heraldry locus locusGrp material msDesc objectType origDate origPlace secFol signatures stamp watermark width\par 
    \item[namesdates: ]
   addName affiliation bloc climate country district forename genName geo geogFeat geogName listEvent listNym listOrg listPerson listPlace location nameLink offset orgName persName placeName population region roleName settlement state surname terrain trait\par 
    \item[textcrit: ]
   app listApp listWit witDetail\par 
    \item[textstructure: ]
   floatingText\par 
    \item[transcr: ]
   addSpan am damage damageSpan delSpan ex fw handShift listTranspose metamark mod redo restore retrace secl space subst substJoin supplied surplus undo\par character data
    \item[{Example}]
  \leavevmode\bgroup\exampleFont \begin{shaded}\noindent\mbox{}{<\textbf{handNote}\hspace*{6pt}{scope}="{sole}">}\mbox{}\newline 
\hspace*{6pt}{<\textbf{p}>}Written in insular\mbox{}\newline 
\hspace*{6pt}\hspace*{6pt} phase II half-uncial with interlinear Old English gloss in an Anglo-Saxon pointed\mbox{}\newline 
\hspace*{6pt}\hspace*{6pt} minuscule.{</\textbf{p}>}\mbox{}\newline 
{</\textbf{handNote}>}\end{shaded}\egroup 


    \item[{Content model}]
  \mbox{}\hfill\\[-10pt]\begin{Verbatim}[fontsize=\small]
<content>
 <macroRef key="macro.specialPara"/>
</content>
    
\end{Verbatim}

    \item[{Schema Declaration}]
  \mbox{}\hfill\\[-10pt]\begin{Verbatim}[fontsize=\small]
element handNote
{
   att.global.attributes,
   att.handFeatures.attributes,
   macro.specialPara}
\end{Verbatim}

\end{reflist}  \index{handNotes=<handNotes>|oddindex}
\begin{reflist}
\item[]\begin{specHead}{TEI.handNotes}{<handNotes> }contains one or more <handNote> elements documenting the different hands identified within the source texts. [\xref{http://www.tei-c.org/release/doc/tei-p5-doc/en/html/PH.html\#PHDH}{11.3.2.1. Document Hands}]\end{specHead} 
    \item[{Module}]
  transcr
    \item[{Attributes}]
  Attributes att.global (\textit{@xml:id}, \textit{@n}, \textit{@xml:lang}, \textit{@xml:base}, \textit{@xml:space})  (att.global.rendition (\textit{@rend}, \textit{@style}, \textit{@rendition})) (att.global.linking (\textit{@corresp}, \textit{@synch}, \textit{@sameAs}, \textit{@copyOf}, \textit{@next}, \textit{@prev}, \textit{@exclude}, \textit{@select})) (att.global.analytic (\textit{@ana})) (att.global.facs (\textit{@facs})) (att.global.change (\textit{@change})) (att.global.responsibility (\textit{@cert}, \textit{@resp})) (att.global.source (\textit{@source}))
    \item[{Member of}]
  model.profileDescPart
    \item[{Contained by}]
  
    \item[header: ]
   profileDesc
    \item[{May contain}]
  
    \item[header: ]
   handNote
    \item[{Example}]
  \leavevmode\bgroup\exampleFont \begin{shaded}\noindent\mbox{}{<\textbf{handNotes}>}\mbox{}\newline 
\hspace*{6pt}{<\textbf{handNote}\hspace*{6pt}{medium}="{brown-ink}"\mbox{}\newline 
\hspace*{6pt}\hspace*{6pt}{script}="{copperplate}"\hspace*{6pt}{xml:id}="{H1}">}Carefully written with regular descenders{</\textbf{handNote}>}\mbox{}\newline 
\hspace*{6pt}{<\textbf{handNote}\hspace*{6pt}{medium}="{pencil}"\hspace*{6pt}{script}="{print}"\mbox{}\newline 
\hspace*{6pt}\hspace*{6pt}{xml:id}="{H2}">}Unschooled scrawl{</\textbf{handNote}>}\mbox{}\newline 
{</\textbf{handNotes}>}\end{shaded}\egroup 


    \item[{Content model}]
  \mbox{}\hfill\\[-10pt]\begin{Verbatim}[fontsize=\small]
<content>
 <elementRef key="handNote"
  maxOccurs="unbounded" minOccurs="1"/>
</content>
    
\end{Verbatim}

    \item[{Schema Declaration}]
  \mbox{}\hfill\\[-10pt]\begin{Verbatim}[fontsize=\small]
element handNotes { att.global.attributes, handNote+ }
\end{Verbatim}

\end{reflist}  \index{handShift=<handShift>|oddindex}\index{new=@new!<handShift>|oddindex}
\begin{reflist}
\item[]\begin{specHead}{TEI.handShift}{<handShift> }marks the beginning of a sequence of text written in a new hand, or the beginning of a scribal stint. [\xref{http://www.tei-c.org/release/doc/tei-p5-doc/en/html/PH.html\#PHDH}{11.3.2.1. Document Hands}]\end{specHead} 
    \item[{Module}]
  transcr
    \item[{Attributes}]
  Attributes att.global (\textit{@xml:id}, \textit{@n}, \textit{@xml:lang}, \textit{@xml:base}, \textit{@xml:space})  (att.global.rendition (\textit{@rend}, \textit{@style}, \textit{@rendition})) (att.global.linking (\textit{@corresp}, \textit{@synch}, \textit{@sameAs}, \textit{@copyOf}, \textit{@next}, \textit{@prev}, \textit{@exclude}, \textit{@select})) (att.global.analytic (\textit{@ana})) (att.global.facs (\textit{@facs})) (att.global.change (\textit{@change})) (att.global.responsibility (\textit{@cert}, \textit{@resp})) (att.global.source (\textit{@source})) att.handFeatures (\textit{@scribe}, \textit{@scribeRef}, \textit{@script}, \textit{@scriptRef}, \textit{@medium}, \textit{@scope}) \hfil\\[-10pt]\begin{sansreflist}
    \item[@new]
  indicates a <handNote> element describing the hand concerned.
\begin{reflist}
    \item[{Status}]
  Recommended
    \item[{Datatype}]
  teidata.pointer
    \item[{Note}]
  \par
This attribute serves the same function as the {\itshape hand} attribute provided for those elements which are members of the \textsf{att.transcriptional} class. It may be renamed at a subsequent major release.
\end{reflist}  
\end{sansreflist}  
    \item[{Member of}]
  model.linePart model.pPart.transcriptional
    \item[{Contained by}]
  
    \item[analysis: ]
   cl pc phr s w\par 
    \item[core: ]
   abbr add addrLine author bibl biblScope citedRange corr date del distinct editor email emph expan foreign gloss head headItem headLabel hi item l label measure mentioned name note num orig p pubPlace publisher q quote ref reg rs said sic soCalled speaker stage street term textLang time title unclear\par 
    \item[figures: ]
   cell\par 
    \item[header: ]
   change distributor edition extent geoDecl handNote licence scriptNote typeNote\par 
    \item[linking: ]
   ab seg\par 
    \item[msdescription: ]
   accMat acquisition additions catchwords collation colophon condition custEvent decoNote explicit filiation finalRubric foliation heraldry incipit layout material musicNotation objectType origDate origPlace origin provenance rubric secFol signatures source stamp summary support surrogates watermark\par 
    \item[namesdates: ]
   addName affiliation birth bloc country death district education faith floruit forename genName geogFeat geogName nameLink nationality occupation offset orgName persName placeName region residence roleName settlement sex socecStatus surname\par 
    \item[textcrit: ]
   lem rdg wit witDetail\par 
    \item[textstructure: ]
   byline closer dateline docAuthor docDate docEdition docImprint imprimatur opener salute signed titlePart trailer\par 
    \item[transcr: ]
   am damage fw line metamark mod restore retrace secl supplied surplus zone
    \item[{May contain}]
  Empty element
    \item[{Note}]
  \par
The <handShift> element may be used either to denote a shift in the document hand (as from one scribe to another, on one writing style to another). Or, it may indicate a shift within a document hand, as a change of writing style, character or ink. Like other milestone elements, it should appear at the point of transition from some other state to the state which it describes.
    \item[{Example}]
  \leavevmode\bgroup\exampleFont \begin{shaded}\noindent\mbox{}{<\textbf{l}>}When wolde the cat dwelle in his ynne{</\textbf{l}>}\mbox{}\newline 
{<\textbf{handShift}\hspace*{6pt}{medium}="{greenish-ink}"/>}\mbox{}\newline 
{<\textbf{l}>}And if the cattes skynne be slyk {<\textbf{handShift}\hspace*{6pt}{medium}="{black-ink}"/>} and gaye{</\textbf{l}>}\end{shaded}\egroup 


    \item[{Content model}]
  \fbox{\ttfamily <content>\newline
</content>\newline
    } 
    \item[{Schema Declaration}]
  \mbox{}\hfill\\[-10pt]\begin{Verbatim}[fontsize=\small]
element handShift
{
   att.global.attributes,
   att.handFeatures.attributes,
   attribute new { text }?,
   empty
}
\end{Verbatim}

\end{reflist}  \index{head=<head>|oddindex}
\begin{reflist}
\item[]\begin{specHead}{TEI.head}{<head> }(heading) contains any type of heading, for example the title of a section, or the heading of a list, glossary, manuscript description, etc. [\xref{http://www.tei-c.org/release/doc/tei-p5-doc/en/html/DS.html\#DSHD}{4.2.1. Headings and Trailers}]\end{specHead} 
    \item[{Module}]
  core
    \item[{Attributes}]
  Attributes att.global (\textit{@xml:id}, \textit{@n}, \textit{@xml:lang}, \textit{@xml:base}, \textit{@xml:space})  (att.global.rendition (\textit{@rend}, \textit{@style}, \textit{@rendition})) (att.global.linking (\textit{@corresp}, \textit{@synch}, \textit{@sameAs}, \textit{@copyOf}, \textit{@next}, \textit{@prev}, \textit{@exclude}, \textit{@select})) (att.global.analytic (\textit{@ana})) (att.global.facs (\textit{@facs})) (att.global.change (\textit{@change})) (att.global.responsibility (\textit{@cert}, \textit{@resp})) (att.global.source (\textit{@source})) att.typed (\textit{@type}, \textit{@subtype}) att.written (\textit{@hand}) 
    \item[{Member of}]
  model.headLike model.pLike.front
    \item[{Contained by}]
  
    \item[core: ]
   divGen lg list listBibl\par 
    \item[figures: ]
   figure table\par 
    \item[msdescription: ]
   msDesc msFrag msPart\par 
    \item[namesdates: ]
   climate event listEvent listNym listOrg listPerson listPlace listRelation org place population state terrain trait\par 
    \item[textcrit: ]
   listApp listWit\par 
    \item[textstructure: ]
   argument back body div front group postscript
    \item[{May contain}]
  
    \item[analysis: ]
   c cl interp interpGrp m pc phr s span spanGrp w\par 
    \item[core: ]
   abbr add address bibl biblStruct cb choice cit corr date del desc distinct email emph expan foreign gap gb gloss graphic hi index l label lb lg list listBibl measure measureGrp media mentioned milestone name note num orig pb ptr q quote ref reg rs said sic soCalled stage term time title unclear\par 
    \item[figures: ]
   figure formula notatedMusic table\par 
    \item[gaiji: ]
   g\par 
    \item[header: ]
   biblFull idno\par 
    \item[linking: ]
   alt altGrp anchor join joinGrp link linkGrp seg timeline\par 
    \item[msdescription: ]
   catchwords depth dim dimensions height heraldry locus locusGrp material msDesc objectType origDate origPlace secFol signatures stamp watermark width\par 
    \item[namesdates: ]
   addName affiliation bloc climate country district forename genName geo geogFeat geogName listEvent listNym listOrg listPerson listPlace location nameLink offset orgName persName placeName population region roleName settlement state surname terrain trait\par 
    \item[textcrit: ]
   app listApp listWit witDetail\par 
    \item[textstructure: ]
   floatingText\par 
    \item[transcr: ]
   addSpan am damage damageSpan delSpan ex fw handShift listTranspose metamark mod redo restore retrace secl space subst substJoin supplied surplus undo\par character data
    \item[{Note}]
  \par
The <head> element is used for headings at all levels; software which treats (e.g.) chapter headings, section headings, and list titles differently must determine the proper processing of a <head> element based on its structural position. A <head> occurring as the first element of a list is the title of that list; one occurring as the first element of a \texttt{<div1>} is the title of that chapter or section.
    \item[{Example}]
  The most common use for the <head> element is to mark the headings of sections. In older writings, the headings or \textit{incipits} may be rather longer than usual in modern works. If a section has an explicit ending as well as a heading, it should be marked as a <trailer>, as in this example:\leavevmode\bgroup\exampleFont \begin{shaded}\noindent\mbox{}{<\textbf{div1}\hspace*{6pt}{n}="{I}"\hspace*{6pt}{type}="{book}">}\mbox{}\newline 
\hspace*{6pt}{<\textbf{head}>}In the name of Christ here begins the first book of the ecclesiastical history of\mbox{}\newline 
\hspace*{6pt}\hspace*{6pt} Georgius Florentinus, known as Gregory, Bishop of Tours.{</\textbf{head}>}\mbox{}\newline 
\hspace*{6pt}{<\textbf{div2}\hspace*{6pt}{type}="{section}">}\mbox{}\newline 
\hspace*{6pt}\hspace*{6pt}{<\textbf{head}>}In the name of Christ here begins Book I of the history.{</\textbf{head}>}\mbox{}\newline 
\hspace*{6pt}\hspace*{6pt}{<\textbf{p}>}Proposing as I do ...{</\textbf{p}>}\mbox{}\newline 
\hspace*{6pt}\hspace*{6pt}{<\textbf{p}>}From the Passion of our Lord until the death of Saint Martin four hundred and twelve\mbox{}\newline 
\hspace*{6pt}\hspace*{6pt}\hspace*{6pt}\hspace*{6pt} years passed.{</\textbf{p}>}\mbox{}\newline 
\hspace*{6pt}\hspace*{6pt}{<\textbf{trailer}>}Here ends the first Book, which covers five thousand, five hundred and ninety-six\mbox{}\newline 
\hspace*{6pt}\hspace*{6pt}\hspace*{6pt}\hspace*{6pt} years from the beginning of the world down to the death of Saint Martin.{</\textbf{trailer}>}\mbox{}\newline 
\hspace*{6pt}{</\textbf{div2}>}\mbox{}\newline 
{</\textbf{div1}>}\end{shaded}\egroup 


    \item[{Example}]
  The <head> element is also used to mark headings of other units, such as lists:\leavevmode\bgroup\exampleFont \begin{shaded}\noindent\mbox{}With a few exceptions, connectives are equally\mbox{}\newline 
 useful in all kinds of discourse: description, narration, exposition, argument. {<\textbf{list}\hspace*{6pt}{rend}="{bulleted}">}\mbox{}\newline 
\hspace*{6pt}{<\textbf{head}>}Connectives{</\textbf{head}>}\mbox{}\newline 
\hspace*{6pt}{<\textbf{item}>}above{</\textbf{item}>}\mbox{}\newline 
\hspace*{6pt}{<\textbf{item}>}accordingly{</\textbf{item}>}\mbox{}\newline 
\hspace*{6pt}{<\textbf{item}>}across from{</\textbf{item}>}\mbox{}\newline 
\hspace*{6pt}{<\textbf{item}>}adjacent to{</\textbf{item}>}\mbox{}\newline 
\hspace*{6pt}{<\textbf{item}>}again{</\textbf{item}>}\mbox{}\newline 
\hspace*{6pt}{<\textbf{item}>}\mbox{}\newline 
\textit{<!-- ... -->}\mbox{}\newline 
\hspace*{6pt}{</\textbf{item}>}\mbox{}\newline 
{</\textbf{list}>}\end{shaded}\egroup 


    \item[{Content model}]
  \mbox{}\hfill\\[-10pt]\begin{Verbatim}[fontsize=\small]
<content>
 <alternate maxOccurs="unbounded"
  minOccurs="0">
  <textNode/>
  <elementRef key="lg"/>
  <classRef key="model.gLike"/>
  <classRef key="model.phrase"/>
  <classRef key="model.inter"/>
  <classRef key="model.lLike"/>
  <classRef key="model.global"/>
 </alternate>
</content>
    
\end{Verbatim}

    \item[{Schema Declaration}]
  \mbox{}\hfill\\[-10pt]\begin{Verbatim}[fontsize=\small]
element head
{
   att.global.attributes,
   att.typed.attributes,
   att.written.attributes,
   (
      text
    | lg    | model.gLike    | model.phrase    | model.inter    | model.lLike    | model.global   )*
}
\end{Verbatim}

\end{reflist}  \index{headItem=<headItem>|oddindex}
\begin{reflist}
\item[]\begin{specHead}{TEI.headItem}{<headItem> }(heading for list items) contains the heading for the item or gloss column in a glossary list or similar structured list. [\xref{http://www.tei-c.org/release/doc/tei-p5-doc/en/html/CO.html\#COLI}{3.7. Lists}]\end{specHead} 
    \item[{Module}]
  core
    \item[{Attributes}]
  Attributes att.global (\textit{@xml:id}, \textit{@n}, \textit{@xml:lang}, \textit{@xml:base}, \textit{@xml:space})  (att.global.rendition (\textit{@rend}, \textit{@style}, \textit{@rendition})) (att.global.linking (\textit{@corresp}, \textit{@synch}, \textit{@sameAs}, \textit{@copyOf}, \textit{@next}, \textit{@prev}, \textit{@exclude}, \textit{@select})) (att.global.analytic (\textit{@ana})) (att.global.facs (\textit{@facs})) (att.global.change (\textit{@change})) (att.global.responsibility (\textit{@cert}, \textit{@resp})) (att.global.source (\textit{@source}))
    \item[{Contained by}]
  
    \item[core: ]
   list
    \item[{May contain}]
  
    \item[analysis: ]
   c cl interp interpGrp m pc phr s span spanGrp w\par 
    \item[core: ]
   abbr add address cb choice corr date del distinct email emph expan foreign gap gb gloss graphic hi index lb measure measureGrp media mentioned milestone name note num orig pb ptr ref reg rs sic soCalled term time title unclear\par 
    \item[figures: ]
   figure formula notatedMusic\par 
    \item[gaiji: ]
   g\par 
    \item[header: ]
   idno\par 
    \item[linking: ]
   alt altGrp anchor join joinGrp link linkGrp seg timeline\par 
    \item[msdescription: ]
   catchwords depth dim dimensions height heraldry locus locusGrp material objectType origDate origPlace secFol signatures stamp watermark width\par 
    \item[namesdates: ]
   addName affiliation bloc climate country district forename genName geo geogFeat geogName location nameLink offset orgName persName placeName population region roleName settlement state surname terrain trait\par 
    \item[textcrit: ]
   app witDetail\par 
    \item[transcr: ]
   addSpan am damage damageSpan delSpan ex fw handShift listTranspose metamark mod redo restore retrace secl space subst substJoin supplied surplus undo\par character data
    \item[{Note}]
  \par
The <headItem> element may appear only if each item in the list is preceded by a <label>.
    \item[{Example}]
  \leavevmode\bgroup\exampleFont \begin{shaded}\noindent\mbox{}The simple, straightforward statement of an\mbox{}\newline 
 idea is preferable to the use of a worn-out expression. {<\textbf{list}\hspace*{6pt}{type}="{gloss}">}\mbox{}\newline 
\hspace*{6pt}{<\textbf{headLabel}\hspace*{6pt}{rend}="{smallcaps}">}TRITE{</\textbf{headLabel}>}\mbox{}\newline 
\hspace*{6pt}{<\textbf{headItem}\hspace*{6pt}{rend}="{smallcaps}">}SIMPLE, STRAIGHTFORWARD{</\textbf{headItem}>}\mbox{}\newline 
\hspace*{6pt}{<\textbf{label}>}bury the hatchet{</\textbf{label}>}\mbox{}\newline 
\hspace*{6pt}{<\textbf{item}>}stop fighting, make peace{</\textbf{item}>}\mbox{}\newline 
\hspace*{6pt}{<\textbf{label}>}at loose ends{</\textbf{label}>}\mbox{}\newline 
\hspace*{6pt}{<\textbf{item}>}disorganized{</\textbf{item}>}\mbox{}\newline 
\hspace*{6pt}{<\textbf{label}>}on speaking terms{</\textbf{label}>}\mbox{}\newline 
\hspace*{6pt}{<\textbf{item}>}friendly{</\textbf{item}>}\mbox{}\newline 
\hspace*{6pt}{<\textbf{label}>}fair and square{</\textbf{label}>}\mbox{}\newline 
\hspace*{6pt}{<\textbf{item}>}completely honest{</\textbf{item}>}\mbox{}\newline 
\hspace*{6pt}{<\textbf{label}>}at death's door{</\textbf{label}>}\mbox{}\newline 
\hspace*{6pt}{<\textbf{item}>}near death{</\textbf{item}>}\mbox{}\newline 
{</\textbf{list}>}\end{shaded}\egroup 


    \item[{Content model}]
  \mbox{}\hfill\\[-10pt]\begin{Verbatim}[fontsize=\small]
<content>
 <macroRef key="macro.phraseSeq"/>
</content>
    
\end{Verbatim}

    \item[{Schema Declaration}]
  \mbox{}\hfill\\[-10pt]\begin{Verbatim}[fontsize=\small]
element headItem { att.global.attributes, macro.phraseSeq }
\end{Verbatim}

\end{reflist}  \index{headLabel=<headLabel>|oddindex}
\begin{reflist}
\item[]\begin{specHead}{TEI.headLabel}{<headLabel> }(heading for list labels) contains the heading for the label or term column in a glossary list or similar structured list. [\xref{http://www.tei-c.org/release/doc/tei-p5-doc/en/html/CO.html\#COLI}{3.7. Lists}]\end{specHead} 
    \item[{Module}]
  core
    \item[{Attributes}]
  Attributes att.global (\textit{@xml:id}, \textit{@n}, \textit{@xml:lang}, \textit{@xml:base}, \textit{@xml:space})  (att.global.rendition (\textit{@rend}, \textit{@style}, \textit{@rendition})) (att.global.linking (\textit{@corresp}, \textit{@synch}, \textit{@sameAs}, \textit{@copyOf}, \textit{@next}, \textit{@prev}, \textit{@exclude}, \textit{@select})) (att.global.analytic (\textit{@ana})) (att.global.facs (\textit{@facs})) (att.global.change (\textit{@change})) (att.global.responsibility (\textit{@cert}, \textit{@resp})) (att.global.source (\textit{@source}))
    \item[{Contained by}]
  
    \item[core: ]
   list
    \item[{May contain}]
  
    \item[analysis: ]
   c cl interp interpGrp m pc phr s span spanGrp w\par 
    \item[core: ]
   abbr add address cb choice corr date del distinct email emph expan foreign gap gb gloss graphic hi index lb measure measureGrp media mentioned milestone name note num orig pb ptr ref reg rs sic soCalled term time title unclear\par 
    \item[figures: ]
   figure formula notatedMusic\par 
    \item[gaiji: ]
   g\par 
    \item[header: ]
   idno\par 
    \item[linking: ]
   alt altGrp anchor join joinGrp link linkGrp seg timeline\par 
    \item[msdescription: ]
   catchwords depth dim dimensions height heraldry locus locusGrp material objectType origDate origPlace secFol signatures stamp watermark width\par 
    \item[namesdates: ]
   addName affiliation bloc climate country district forename genName geo geogFeat geogName location nameLink offset orgName persName placeName population region roleName settlement state surname terrain trait\par 
    \item[textcrit: ]
   app witDetail\par 
    \item[transcr: ]
   addSpan am damage damageSpan delSpan ex fw handShift listTranspose metamark mod redo restore retrace secl space subst substJoin supplied surplus undo\par character data
    \item[{Note}]
  \par
The <headLabel> element may appear only if each item in the list is preceded by a <label>.
    \item[{Example}]
  \leavevmode\bgroup\exampleFont \begin{shaded}\noindent\mbox{}The simple, straightforward statement of an\mbox{}\newline 
 idea is preferable to the use of a worn-out expression. {<\textbf{list}\hspace*{6pt}{type}="{gloss}">}\mbox{}\newline 
\hspace*{6pt}{<\textbf{headLabel}\hspace*{6pt}{rend}="{smallcaps}">}TRITE{</\textbf{headLabel}>}\mbox{}\newline 
\hspace*{6pt}{<\textbf{headItem}\hspace*{6pt}{rend}="{smallcaps}">}SIMPLE, STRAIGHTFORWARD{</\textbf{headItem}>}\mbox{}\newline 
\hspace*{6pt}{<\textbf{label}>}bury the hatchet{</\textbf{label}>}\mbox{}\newline 
\hspace*{6pt}{<\textbf{item}>}stop fighting, make peace{</\textbf{item}>}\mbox{}\newline 
\hspace*{6pt}{<\textbf{label}>}at loose ends{</\textbf{label}>}\mbox{}\newline 
\hspace*{6pt}{<\textbf{item}>}disorganized{</\textbf{item}>}\mbox{}\newline 
\hspace*{6pt}{<\textbf{label}>}on speaking terms{</\textbf{label}>}\mbox{}\newline 
\hspace*{6pt}{<\textbf{item}>}friendly{</\textbf{item}>}\mbox{}\newline 
\hspace*{6pt}{<\textbf{label}>}fair and square{</\textbf{label}>}\mbox{}\newline 
\hspace*{6pt}{<\textbf{item}>}completely honest{</\textbf{item}>}\mbox{}\newline 
\hspace*{6pt}{<\textbf{label}>}at death's door{</\textbf{label}>}\mbox{}\newline 
\hspace*{6pt}{<\textbf{item}>}near death{</\textbf{item}>}\mbox{}\newline 
{</\textbf{list}>}\end{shaded}\egroup 


    \item[{Content model}]
  \mbox{}\hfill\\[-10pt]\begin{Verbatim}[fontsize=\small]
<content>
 <macroRef key="macro.phraseSeq"/>
</content>
    
\end{Verbatim}

    \item[{Schema Declaration}]
  \mbox{}\hfill\\[-10pt]\begin{Verbatim}[fontsize=\small]
element headLabel { att.global.attributes, macro.phraseSeq }
\end{Verbatim}

\end{reflist}  \index{height=<height>|oddindex}
\begin{reflist}
\item[]\begin{specHead}{TEI.height}{<height> }contains a measurement measured along the axis at right angles to the bottom of the written surface, i.e. parallel to the spine for a codex or book. [\xref{http://www.tei-c.org/release/doc/tei-p5-doc/en/html/MS.html\#msdim}{10.3.4. Dimensions}]\end{specHead} 
    \item[{Module}]
  msdescription
    \item[{Attributes}]
  Attributes att.global (\textit{@xml:id}, \textit{@n}, \textit{@xml:lang}, \textit{@xml:base}, \textit{@xml:space})  (att.global.rendition (\textit{@rend}, \textit{@style}, \textit{@rendition})) (att.global.linking (\textit{@corresp}, \textit{@synch}, \textit{@sameAs}, \textit{@copyOf}, \textit{@next}, \textit{@prev}, \textit{@exclude}, \textit{@select})) (att.global.analytic (\textit{@ana})) (att.global.facs (\textit{@facs})) (att.global.change (\textit{@change})) (att.global.responsibility (\textit{@cert}, \textit{@resp})) (att.global.source (\textit{@source})) att.dimensions (\textit{@unit}, \textit{@quantity}, \textit{@extent}, \textit{@precision}, \textit{@scope})  (att.ranging (\textit{@atLeast}, \textit{@atMost}, \textit{@min}, \textit{@max}, \textit{@confidence}))
    \item[{Member of}]
  model.dimLike model.measureLike
    \item[{Contained by}]
  
    \item[analysis: ]
   cl phr s span\par 
    \item[core: ]
   abbr add addrLine author bibl biblScope citedRange corr date del desc distinct editor email emph expan foreign gloss head headItem headLabel hi item l label measure measureGrp meeting mentioned name note num orig p pubPlace publisher q quote ref reg resp rs said sic soCalled speaker stage street term textLang time title unclear\par 
    \item[figures: ]
   cell figDesc\par 
    \item[header: ]
   authority catDesc change classCode creation distributor edition extent funder geoDecl handNote language licence principal rendition scriptNote sponsor tagUsage typeNote\par 
    \item[linking: ]
   ab seg\par 
    \item[msdescription: ]
   accMat acquisition additions catchwords collation colophon condition custEvent decoNote dimensions explicit filiation finalRubric foliation heraldry incipit layout material musicNotation objectType origDate origPlace origin provenance rubric secFol signatures source stamp summary support surrogates watermark\par 
    \item[namesdates: ]
   addName affiliation age birth bloc country death district education faith floruit forename genName geogFeat geogName langKnown location nameLink nationality occupation offset orgName persName placeName region residence roleName settlement sex socecStatus surname\par 
    \item[textcrit: ]
   lem rdg wit witDetail witness\par 
    \item[textstructure: ]
   byline closer dateline docAuthor docDate docEdition docImprint imprimatur opener salute signed titlePart trailer\par 
    \item[transcr: ]
   damage fw metamark mod restore retrace secl supplied surplus
    \item[{May contain}]
  
    \item[gaiji: ]
   g\par character data
    \item[{Note}]
  \par
If used to specify the height of a non text-bearing portion of some object, for example a monument, this element conventionally refers to the axis perpendicular to the surface of the earth.
    \item[{Example}]
  \leavevmode\bgroup\exampleFont \begin{shaded}\noindent\mbox{}{<\textbf{height}\hspace*{6pt}{quantity}="{7}"\hspace*{6pt}{unit}="{in}"/>}\end{shaded}\egroup 


    \item[{Content model}]
  \fbox{\ttfamily <content>\newline
 <macroRef key="macro.xtext"/>\newline
</content>\newline
    } 
    \item[{Schema Declaration}]
  \mbox{}\hfill\\[-10pt]\begin{Verbatim}[fontsize=\small]
element height
{
   att.global.attributes,
   att.dimensions.attributes,
   macro.xtext}
\end{Verbatim}

\end{reflist}  \index{heraldry=<heraldry>|oddindex}
\begin{reflist}
\item[]\begin{specHead}{TEI.heraldry}{<heraldry> }contains a heraldic formula or phrase, typically found as part of a blazon, coat of arms, etc.  [\xref{http://www.tei-c.org/release/doc/tei-p5-doc/en/html/MS.html\#mshera}{10.3.8. Heraldry}]\end{specHead} 
    \item[{Module}]
  msdescription
    \item[{Attributes}]
  Attributes att.global (\textit{@xml:id}, \textit{@n}, \textit{@xml:lang}, \textit{@xml:base}, \textit{@xml:space})  (att.global.rendition (\textit{@rend}, \textit{@style}, \textit{@rendition})) (att.global.linking (\textit{@corresp}, \textit{@synch}, \textit{@sameAs}, \textit{@copyOf}, \textit{@next}, \textit{@prev}, \textit{@exclude}, \textit{@select})) (att.global.analytic (\textit{@ana})) (att.global.facs (\textit{@facs})) (att.global.change (\textit{@change})) (att.global.responsibility (\textit{@cert}, \textit{@resp})) (att.global.source (\textit{@source}))
    \item[{Member of}]
  model.pPart.msdesc
    \item[{Contained by}]
  
    \item[analysis: ]
   cl phr s span\par 
    \item[core: ]
   abbr add addrLine author biblScope citedRange corr date del desc distinct editor email emph expan foreign gloss head headItem headLabel hi item l label measure meeting mentioned name note num orig p pubPlace publisher q quote ref reg resp rs said sic soCalled speaker stage street term textLang time title unclear\par 
    \item[figures: ]
   cell figDesc\par 
    \item[header: ]
   authority catDesc change classCode creation distributor edition extent funder geoDecl handNote language licence principal rendition scriptNote sponsor tagUsage typeNote\par 
    \item[linking: ]
   ab seg\par 
    \item[msdescription: ]
   accMat acquisition additions catchwords collation colophon condition custEvent decoNote explicit filiation finalRubric foliation heraldry incipit layout material musicNotation objectType origDate origPlace origin provenance rubric secFol signatures source stamp summary support surrogates watermark\par 
    \item[namesdates: ]
   addName affiliation age birth bloc country death district education faith floruit forename genName geogFeat geogName langKnown nameLink nationality occupation offset orgName persName placeName region residence roleName settlement sex socecStatus surname\par 
    \item[textcrit: ]
   lem rdg wit witDetail witness\par 
    \item[textstructure: ]
   byline closer dateline docAuthor docDate docEdition docImprint imprimatur opener salute signed titlePart trailer\par 
    \item[transcr: ]
   damage fw metamark mod restore retrace secl supplied surplus
    \item[{May contain}]
  
    \item[analysis: ]
   c cl interp interpGrp m pc phr s span spanGrp w\par 
    \item[core: ]
   abbr add address cb choice corr date del distinct email emph expan foreign gap gb gloss graphic hi index lb measure measureGrp media mentioned milestone name note num orig pb ptr ref reg rs sic soCalled term time title unclear\par 
    \item[figures: ]
   figure formula notatedMusic\par 
    \item[gaiji: ]
   g\par 
    \item[header: ]
   idno\par 
    \item[linking: ]
   alt altGrp anchor join joinGrp link linkGrp seg timeline\par 
    \item[msdescription: ]
   catchwords depth dim dimensions height heraldry locus locusGrp material objectType origDate origPlace secFol signatures stamp watermark width\par 
    \item[namesdates: ]
   addName affiliation bloc climate country district forename genName geo geogFeat geogName location nameLink offset orgName persName placeName population region roleName settlement state surname terrain trait\par 
    \item[textcrit: ]
   app witDetail\par 
    \item[transcr: ]
   addSpan am damage damageSpan delSpan ex fw handShift listTranspose metamark mod redo restore retrace secl space subst substJoin supplied surplus undo\par character data
    \item[{Example}]
  \leavevmode\bgroup\exampleFont \begin{shaded}\noindent\mbox{}{<\textbf{p}>}Ownership stamp (xvii cent.) on i recto with the arms\mbox{}\newline 
{<\textbf{heraldry}>}A bull passant within a bordure bezanty,\mbox{}\newline 
\hspace*{6pt}\hspace*{6pt} in chief a crescent for difference{</\textbf{heraldry}>} [Cole],\mbox{}\newline 
 crest, and the legend {<\textbf{q}>}Cole Deum{</\textbf{q}>}.{</\textbf{p}>}\end{shaded}\egroup 


    \item[{Content model}]
  \mbox{}\hfill\\[-10pt]\begin{Verbatim}[fontsize=\small]
<content>
 <macroRef key="macro.phraseSeq"/>
</content>
    
\end{Verbatim}

    \item[{Schema Declaration}]
  \mbox{}\hfill\\[-10pt]\begin{Verbatim}[fontsize=\small]
element heraldry { att.global.attributes, macro.phraseSeq }
\end{Verbatim}

\end{reflist}  \index{hi=<hi>|oddindex}
\begin{reflist}
\item[]\begin{specHead}{TEI.hi}{<hi> }(highlighted) marks a word or phrase as graphically distinct from the surrounding text, for reasons concerning which no claim is made. [\xref{http://www.tei-c.org/release/doc/tei-p5-doc/en/html/CO.html\#COHQHE}{3.3.2.2. Emphatic Words and Phrases} \xref{http://www.tei-c.org/release/doc/tei-p5-doc/en/html/CO.html\#COHQH}{3.3.2. Emphasis, Foreign Words, and Unusual Language}]\end{specHead} 
    \item[{Module}]
  core
    \item[{Attributes}]
  Attributes att.global (\textit{@xml:id}, \textit{@n}, \textit{@xml:lang}, \textit{@xml:base}, \textit{@xml:space})  (att.global.rendition (\textit{@rend}, \textit{@style}, \textit{@rendition})) (att.global.linking (\textit{@corresp}, \textit{@synch}, \textit{@sameAs}, \textit{@copyOf}, \textit{@next}, \textit{@prev}, \textit{@exclude}, \textit{@select})) (att.global.analytic (\textit{@ana})) (att.global.facs (\textit{@facs})) (att.global.change (\textit{@change})) (att.global.responsibility (\textit{@cert}, \textit{@resp})) (att.global.source (\textit{@source})) att.written (\textit{@hand}) 
    \item[{Member of}]
  model.hiLike
    \item[{Contained by}]
  
    \item[analysis: ]
   cl m phr s span w\par 
    \item[core: ]
   abbr add addrLine author bibl biblScope citedRange corr date del desc distinct editor email emph expan foreign gloss head headItem headLabel hi item l label measure meeting mentioned name note num orig p pubPlace publisher q quote ref reg resp rs said sic soCalled speaker stage street term textLang time title unclear\par 
    \item[figures: ]
   cell figDesc formula\par 
    \item[header: ]
   authority catDesc change classCode creation distributor edition extent funder geoDecl handNote language licence principal rendition scriptNote sponsor tagUsage typeNote\par 
    \item[linking: ]
   ab seg\par 
    \item[msdescription: ]
   accMat acquisition additions catchwords collation colophon condition custEvent decoNote explicit filiation finalRubric foliation heraldry incipit layout material musicNotation objectType origDate origPlace origin provenance rubric secFol signatures source stamp summary support surrogates watermark\par 
    \item[namesdates: ]
   addName affiliation age birth bloc country death district education faith floruit forename genName geogFeat geogName langKnown nameLink nationality occupation offset orgName persName placeName region residence roleName settlement sex socecStatus surname\par 
    \item[textcrit: ]
   lem rdg wit witDetail witness\par 
    \item[textstructure: ]
   byline closer dateline docAuthor docDate docEdition docImprint imprimatur opener salute signed titlePart trailer\par 
    \item[transcr: ]
   damage fw line metamark mod restore retrace secl supplied surplus zone
    \item[{May contain}]
  
    \item[analysis: ]
   c cl interp interpGrp m pc phr s span spanGrp w\par 
    \item[core: ]
   abbr add address bibl biblStruct cb choice cit corr date del desc distinct email emph expan foreign gap gb gloss graphic hi index l label lb lg list listBibl measure measureGrp media mentioned milestone name note num orig pb ptr q quote ref reg rs said sic soCalled stage term time title unclear\par 
    \item[figures: ]
   figure formula notatedMusic table\par 
    \item[gaiji: ]
   g\par 
    \item[header: ]
   biblFull idno\par 
    \item[linking: ]
   alt altGrp anchor join joinGrp link linkGrp seg timeline\par 
    \item[msdescription: ]
   catchwords depth dim dimensions height heraldry locus locusGrp material msDesc objectType origDate origPlace secFol signatures stamp watermark width\par 
    \item[namesdates: ]
   addName affiliation bloc climate country district forename genName geo geogFeat geogName listEvent listNym listOrg listPerson listPlace location nameLink offset orgName persName placeName population region roleName settlement state surname terrain trait\par 
    \item[textcrit: ]
   app listApp listWit witDetail\par 
    \item[textstructure: ]
   floatingText\par 
    \item[transcr: ]
   addSpan am damage damageSpan delSpan ex fw handShift listTranspose metamark mod redo restore retrace secl space subst substJoin supplied surplus undo\par character data
    \item[{Example}]
  \leavevmode\bgroup\exampleFont \begin{shaded}\noindent\mbox{}{<\textbf{hi}\hspace*{6pt}{rend}="{gothic}">}And this Indenture further witnesseth{</\textbf{hi}>}\mbox{}\newline 
 that the said {<\textbf{hi}\hspace*{6pt}{rend}="{italic}">}Walter Shandy{</\textbf{hi}>}, merchant,\mbox{}\newline 
 in consideration of the said intended marriage ...\end{shaded}\egroup 


    \item[{Content model}]
  \mbox{}\hfill\\[-10pt]\begin{Verbatim}[fontsize=\small]
<content>
 <macroRef key="macro.paraContent"/>
</content>
    
\end{Verbatim}

    \item[{Schema Declaration}]
  \mbox{}\hfill\\[-10pt]\begin{Verbatim}[fontsize=\small]
element hi { att.global.attributes, att.written.attributes, macro.paraContent }
\end{Verbatim}

\end{reflist}  \index{history=<history>|oddindex}
\begin{reflist}
\item[]\begin{specHead}{TEI.history}{<history> }groups elements describing the full history of a manuscript or manuscript part. [\xref{http://www.tei-c.org/release/doc/tei-p5-doc/en/html/MS.html\#mshy}{10.8. History}]\end{specHead} 
    \item[{Module}]
  msdescription
    \item[{Attributes}]
  Attributes att.global (\textit{@xml:id}, \textit{@n}, \textit{@xml:lang}, \textit{@xml:base}, \textit{@xml:space})  (att.global.rendition (\textit{@rend}, \textit{@style}, \textit{@rendition})) (att.global.linking (\textit{@corresp}, \textit{@synch}, \textit{@sameAs}, \textit{@copyOf}, \textit{@next}, \textit{@prev}, \textit{@exclude}, \textit{@select})) (att.global.analytic (\textit{@ana})) (att.global.facs (\textit{@facs})) (att.global.change (\textit{@change})) (att.global.responsibility (\textit{@cert}, \textit{@resp})) (att.global.source (\textit{@source}))
    \item[{Contained by}]
  
    \item[msdescription: ]
   msDesc msFrag msPart
    \item[{May contain}]
  
    \item[core: ]
   p\par 
    \item[linking: ]
   ab\par 
    \item[msdescription: ]
   acquisition origin provenance summary
    \item[{Example}]
  \leavevmode\bgroup\exampleFont \begin{shaded}\noindent\mbox{}{<\textbf{history}>}\mbox{}\newline 
\hspace*{6pt}{<\textbf{origin}>}\mbox{}\newline 
\hspace*{6pt}\hspace*{6pt}{<\textbf{p}>}Written in Durham during the mid twelfth\mbox{}\newline 
\hspace*{6pt}\hspace*{6pt}\hspace*{6pt}\hspace*{6pt} century.{</\textbf{p}>}\mbox{}\newline 
\hspace*{6pt}{</\textbf{origin}>}\mbox{}\newline 
\hspace*{6pt}{<\textbf{provenance}>}\mbox{}\newline 
\hspace*{6pt}\hspace*{6pt}{<\textbf{p}>}Recorded in two medieval\mbox{}\newline 
\hspace*{6pt}\hspace*{6pt}\hspace*{6pt}\hspace*{6pt} catalogues of the books belonging to Durham Priory, made in 1391 and\mbox{}\newline 
\hspace*{6pt}\hspace*{6pt}\hspace*{6pt}\hspace*{6pt} 1405.{</\textbf{p}>}\mbox{}\newline 
\hspace*{6pt}{</\textbf{provenance}>}\mbox{}\newline 
\hspace*{6pt}{<\textbf{provenance}>}\mbox{}\newline 
\hspace*{6pt}\hspace*{6pt}{<\textbf{p}>}Given to W. Olleyf by William Ebchester, Prior (1446-56)\mbox{}\newline 
\hspace*{6pt}\hspace*{6pt}\hspace*{6pt}\hspace*{6pt} and later belonged to Henry Dalton, Prior of Holy Island (Lindisfarne)\mbox{}\newline 
\hspace*{6pt}\hspace*{6pt}\hspace*{6pt}\hspace*{6pt} according to inscriptions on ff. 4v and 5.{</\textbf{p}>}\mbox{}\newline 
\hspace*{6pt}{</\textbf{provenance}>}\mbox{}\newline 
\hspace*{6pt}{<\textbf{acquisition}>}\mbox{}\newline 
\hspace*{6pt}\hspace*{6pt}{<\textbf{p}>}Presented to Trinity College in 1738 by\mbox{}\newline 
\hspace*{6pt}\hspace*{6pt}\hspace*{6pt}\hspace*{6pt} Thomas Gale and his son Roger.{</\textbf{p}>}\mbox{}\newline 
\hspace*{6pt}{</\textbf{acquisition}>}\mbox{}\newline 
{</\textbf{history}>}\end{shaded}\egroup 


    \item[{Content model}]
  \mbox{}\hfill\\[-10pt]\begin{Verbatim}[fontsize=\small]
<content>
 <alternate>
  <classRef key="model.pLike"
   maxOccurs="unbounded" minOccurs="1"/>
  <sequence>
   <elementRef key="summary" minOccurs="0"/>
   <elementRef key="origin" minOccurs="0"/>
   <elementRef key="provenance"
    maxOccurs="unbounded" minOccurs="0"/>
   <elementRef key="acquisition"
    minOccurs="0"/>
  </sequence>
 </alternate>
</content>
    
\end{Verbatim}

    \item[{Schema Declaration}]
  \mbox{}\hfill\\[-10pt]\begin{Verbatim}[fontsize=\small]
element history
{
   att.global.attributes,
   ( model.pLike+ | ( summary?, origin?, provenance*, acquisition? ) )
}
\end{Verbatim}

\end{reflist}  \index{hyphenation=<hyphenation>|oddindex}\index{eol=@eol!<hyphenation>|oddindex}
\begin{reflist}
\item[]\begin{specHead}{TEI.hyphenation}{<hyphenation> }summarizes the way in which hyphenation in a source text has been treated in an encoded version of it. [\xref{http://www.tei-c.org/release/doc/tei-p5-doc/en/html/HD.html\#HD53}{2.3.3. The Editorial Practices Declaration} \xref{http://www.tei-c.org/release/doc/tei-p5-doc/en/html/CC.html\#CCAS2}{15.3.2. Declarable Elements}]\end{specHead} 
    \item[{Module}]
  header
    \item[{Attributes}]
  Attributes att.global (\textit{@xml:id}, \textit{@n}, \textit{@xml:lang}, \textit{@xml:base}, \textit{@xml:space})  (att.global.rendition (\textit{@rend}, \textit{@style}, \textit{@rendition})) (att.global.linking (\textit{@corresp}, \textit{@synch}, \textit{@sameAs}, \textit{@copyOf}, \textit{@next}, \textit{@prev}, \textit{@exclude}, \textit{@select})) (att.global.analytic (\textit{@ana})) (att.global.facs (\textit{@facs})) (att.global.change (\textit{@change})) (att.global.responsibility (\textit{@cert}, \textit{@resp})) (att.global.source (\textit{@source})) att.declarable (\textit{@default}) \hfil\\[-10pt]\begin{sansreflist}
    \item[@eol]
  (end-of-line) indicates whether or not end-of-line hyphenation has been retained in a text.
\begin{reflist}
    \item[{Status}]
  Optional
    \item[{Datatype}]
  teidata.enumerated
    \item[{Legal values are:}]
  \begin{description}

\item[{all}]all end-of-line hyphenation has been retained, even though the lineation of the original may not have been.
\item[{some}]end-of-line hyphenation has been retained in some cases.{[Default] }
\item[{hard}]all soft end-of-line hyphenation has been removed: any remaining end-of-line hyphenation should be retained.
\item[{none}]all end-of-line hyphenation has been removed: any remaining hyphenation occurred within the line.
\end{description} 
\end{reflist}  
\end{sansreflist}  
    \item[{Member of}]
  model.editorialDeclPart
    \item[{Contained by}]
  
    \item[header: ]
   editorialDecl
    \item[{May contain}]
  
    \item[core: ]
   p\par 
    \item[linking: ]
   ab
    \item[{Example}]
  \leavevmode\bgroup\exampleFont \begin{shaded}\noindent\mbox{}{<\textbf{hyphenation}\hspace*{6pt}{eol}="{some}">}\mbox{}\newline 
\hspace*{6pt}{<\textbf{p}>}End-of-line hyphenation silently removed where appropriate{</\textbf{p}>}\mbox{}\newline 
{</\textbf{hyphenation}>}\end{shaded}\egroup 


    \item[{Content model}]
  \mbox{}\hfill\\[-10pt]\begin{Verbatim}[fontsize=\small]
<content>
 <classRef key="model.pLike"
  maxOccurs="unbounded" minOccurs="1"/>
</content>
    
\end{Verbatim}

    \item[{Schema Declaration}]
  \mbox{}\hfill\\[-10pt]\begin{Verbatim}[fontsize=\small]
element hyphenation
{
   att.global.attributes,
   att.declarable.attributes,
   attribute eol { "all" | "some" | "hard" | "none" }?,
   model.pLike+
}
\end{Verbatim}

\end{reflist}  \index{idno=<idno>|oddindex}\index{type=@type!<idno>|oddindex}
\begin{reflist}
\item[]\begin{specHead}{TEI.idno}{<idno> }(identifier) supplies any form of identifier used to identify some object, such as a bibliographic item, a person, a title, an organization, etc. in a standardized way. [\xref{http://www.tei-c.org/release/doc/tei-p5-doc/en/html/HD.html\#HD24}{2.2.4. Publication, Distribution, Licensing, etc.} \xref{http://www.tei-c.org/release/doc/tei-p5-doc/en/html/HD.html\#HD26}{2.2.5. The Series Statement} \xref{http://www.tei-c.org/release/doc/tei-p5-doc/en/html/CO.html\#COBICOI}{3.11.2.4. Imprint, Size of a Document, and Reprint Information}]\end{specHead} 
    \item[{Module}]
  header
    \item[{Attributes}]
  Attributes att.global (\textit{@xml:id}, \textit{@n}, \textit{@xml:lang}, \textit{@xml:base}, \textit{@xml:space})  (att.global.rendition (\textit{@rend}, \textit{@style}, \textit{@rendition})) (att.global.linking (\textit{@corresp}, \textit{@synch}, \textit{@sameAs}, \textit{@copyOf}, \textit{@next}, \textit{@prev}, \textit{@exclude}, \textit{@select})) (att.global.analytic (\textit{@ana})) (att.global.facs (\textit{@facs})) (att.global.change (\textit{@change})) (att.global.responsibility (\textit{@cert}, \textit{@resp})) (att.global.source (\textit{@source})) att.sortable (\textit{@sortKey}) att.datable (\textit{@calendar}, \textit{@period})  (att.datable.w3c (\textit{@when}, \textit{@notBefore}, \textit{@notAfter}, \textit{@from}, \textit{@to})) (att.datable.iso (\textit{@when-iso}, \textit{@notBefore-iso}, \textit{@notAfter-iso}, \textit{@from-iso}, \textit{@to-iso})) (att.datable.custom (\textit{@when-custom}, \textit{@notBefore-custom}, \textit{@notAfter-custom}, \textit{@from-custom}, \textit{@to-custom}, \textit{@datingPoint}, \textit{@datingMethod})) att.typed (\unusedattribute{type}, @subtype) \hfil\\[-10pt]\begin{sansreflist}
    \item[@type]
  categorizes the identifier, for example as an ISBN, Social Security number, etc.
\begin{reflist}
    \item[{Derived from}]
  att.typed
    \item[{Status}]
  Optional
    \item[{Datatype}]
  teidata.enumerated
    \item[{Suggested values include:}]
  \begin{description}

\item[{ISBN}]International Standard Book Number: a 13- or (if assigned prior to 2007) 10-digit identifying number assigned by the publishing industry to a published book or similar item, registered with the \xref{https://www.isbn-international.org}{International ISBN Agency.}
\item[{ISSN}]International Standard Serial Number: an eight-digit number to uniquely identify a serial publication.
\item[{DOI}]Digital Object Identifier: a unique string of letters and numbers assigned to an electronic document.
\item[{URI}]Uniform Resource Identifier: a string of characters to uniquely identify a resource which usually contains indication of the means of accessing that resource, the name of its host, and its filepath.
\item[{VIAF}]A data number in the Virtual Internet Authority File assigned to link different names in catalogs around the world for the same entity.
\item[{ESTC}]English Short-Title Catalogue number: an identifying number assigned to a document in English printed in the British Isles or North America before 1801.
\item[{OCLC}]union catalog number in WorldCat representing a resource held by one or more of the member libraries in the global cooperative Online Computer Library Center.
\end{description} 
\end{reflist}  
\end{sansreflist}  
    \item[{Member of}]
  model.msItemPart model.nameLike model.personPart model.publicationStmtPart.detail 
    \item[{Contained by}]
  
    \item[analysis: ]
   cl phr s span\par 
    \item[core: ]
   abbr add addrLine address analytic author bibl biblScope citedRange corr date del desc distinct editor email emph expan foreign gloss head headItem headLabel hi item l label measure meeting mentioned monogr name note num orig p pubPlace publisher q quote ref reg resp rs said series sic soCalled speaker stage street term textLang time title unclear\par 
    \item[figures: ]
   cell figDesc\par 
    \item[header: ]
   authority catDesc change classCode correspAction creation distributor edition extent funder geoDecl handNote idno language licence principal publicationStmt rendition scriptNote seriesStmt sponsor tagUsage typeNote\par 
    \item[linking: ]
   ab seg\par 
    \item[msdescription: ]
   accMat acquisition additions altIdentifier catchwords collation colophon condition custEvent decoNote explicit filiation finalRubric foliation heraldry incipit layout material msIdentifier msItem musicNotation objectType origDate origPlace origin provenance rubric secFol signatures source stamp summary support surrogates watermark\par 
    \item[namesdates: ]
   addName affiliation age birth bloc country death district education faith floruit forename genName geogFeat geogName langKnown nameLink nationality occupation offset org orgName persName person personGrp place placeName region residence roleName settlement sex socecStatus surname\par 
    \item[textcrit: ]
   lem rdg wit witDetail witness\par 
    \item[textstructure: ]
   byline closer dateline docAuthor docDate docEdition docImprint imprimatur opener salute signed titlePart trailer\par 
    \item[transcr: ]
   damage fw metamark mod restore retrace secl supplied surplus
    \item[{May contain}]
  
    \item[gaiji: ]
   g\par 
    \item[header: ]
   idno\par character data
    \item[{Note}]
  \par
<idno> should be used for labels which identify an object or concept in a formal cataloguing system such as a database or an RDF store, or in a distributed system such as the World Wide Web. Some suggested values for {\itshape type} on <idno> are \texttt{ISBN}, \texttt{ISSN}, \texttt{DOI}, and \texttt{URI}.
    \item[{Example}]
  \leavevmode\bgroup\exampleFont \begin{shaded}\noindent\mbox{}{<\textbf{idno}\hspace*{6pt}{type}="{ISBN}">}978-1-906964-22-1{</\textbf{idno}>}\mbox{}\newline 
{<\textbf{idno}\hspace*{6pt}{type}="{ISSN}">}0143-3385{</\textbf{idno}>}\mbox{}\newline 
{<\textbf{idno}\hspace*{6pt}{type}="{DOI}">}10.1000/123{</\textbf{idno}>}\mbox{}\newline 
{<\textbf{idno}\hspace*{6pt}{type}="{URI}">}http://www.worldcat.org/oclc/185922478{</\textbf{idno}>}\mbox{}\newline 
{<\textbf{idno}\hspace*{6pt}{type}="{URI}">}http://authority.nzetc.org/463/{</\textbf{idno}>}\mbox{}\newline 
{<\textbf{idno}\hspace*{6pt}{type}="{LT}">}Thomason Tract E.537(17){</\textbf{idno}>}\mbox{}\newline 
{<\textbf{idno}\hspace*{6pt}{type}="{Wing}">}C695{</\textbf{idno}>}\mbox{}\newline 
{<\textbf{idno}\hspace*{6pt}{type}="{oldCat}">}\mbox{}\newline 
\hspace*{6pt}{<\textbf{g}\hspace*{6pt}{ref}="{\#sym}"/>}345\mbox{}\newline 
{</\textbf{idno}>}\end{shaded}\egroup 

In the last case, the identifier includes a non-Unicode character which is defined elsewhere by means of a <glyph> or <char> element referenced here as \texttt{\#sym}.
    \item[{Content model}]
  \mbox{}\hfill\\[-10pt]\begin{Verbatim}[fontsize=\small]
<content>
 <alternate maxOccurs="unbounded"
  minOccurs="0">
  <textNode/>
  <classRef key="model.gLike"/>
  <elementRef key="idno"/>
 </alternate>
</content>
    
\end{Verbatim}

    \item[{Schema Declaration}]
  \mbox{}\hfill\\[-10pt]\begin{Verbatim}[fontsize=\small]
element idno
{
   att.global.attributes,
   att.sortable.attributes,
   att.datable.attributes,
   att.typed.attribute.subtype,
   attribute type
   {
      "ISBN" | "ISSN" | "DOI" | "URI" | "VIAF" | "ESTC" | "OCLC"
   }?,
   ( text | model.gLike | idno )*
}
\end{Verbatim}

\end{reflist}  \index{imprimatur=<imprimatur>|oddindex}
\begin{reflist}
\item[]\begin{specHead}{TEI.imprimatur}{<imprimatur> }contains a formal statement authorizing the publication of a work, sometimes required to appear on a title page or its verso. [\xref{http://www.tei-c.org/release/doc/tei-p5-doc/en/html/DS.html\#DSTITL}{4.6. Title Pages}]\end{specHead} 
    \item[{Module}]
  textstructure
    \item[{Attributes}]
  Attributes att.global (\textit{@xml:id}, \textit{@n}, \textit{@xml:lang}, \textit{@xml:base}, \textit{@xml:space})  (att.global.rendition (\textit{@rend}, \textit{@style}, \textit{@rendition})) (att.global.linking (\textit{@corresp}, \textit{@synch}, \textit{@sameAs}, \textit{@copyOf}, \textit{@next}, \textit{@prev}, \textit{@exclude}, \textit{@select})) (att.global.analytic (\textit{@ana})) (att.global.facs (\textit{@facs})) (att.global.change (\textit{@change})) (att.global.responsibility (\textit{@cert}, \textit{@resp})) (att.global.source (\textit{@source}))
    \item[{Member of}]
  model.titlepagePart
    \item[{Contained by}]
  
    \item[msdescription: ]
   msItem\par 
    \item[textstructure: ]
   titlePage
    \item[{May contain}]
  
    \item[analysis: ]
   c cl interp interpGrp m pc phr s span spanGrp w\par 
    \item[core: ]
   abbr add address bibl biblStruct cb choice cit corr date del desc distinct email emph expan foreign gap gb gloss graphic hi index l label lb lg list listBibl measure measureGrp media mentioned milestone name note num orig pb ptr q quote ref reg rs said sic soCalled stage term time title unclear\par 
    \item[figures: ]
   figure formula notatedMusic table\par 
    \item[gaiji: ]
   g\par 
    \item[header: ]
   biblFull idno\par 
    \item[linking: ]
   alt altGrp anchor join joinGrp link linkGrp seg timeline\par 
    \item[msdescription: ]
   catchwords depth dim dimensions height heraldry locus locusGrp material msDesc objectType origDate origPlace secFol signatures stamp watermark width\par 
    \item[namesdates: ]
   addName affiliation bloc climate country district forename genName geo geogFeat geogName listEvent listNym listOrg listPerson listPlace location nameLink offset orgName persName placeName population region roleName settlement state surname terrain trait\par 
    \item[textcrit: ]
   app listApp listWit witDetail\par 
    \item[textstructure: ]
   floatingText\par 
    \item[transcr: ]
   addSpan am damage damageSpan delSpan ex fw handShift listTranspose metamark mod redo restore retrace secl space subst substJoin supplied surplus undo\par character data
    \item[{Example}]
  \leavevmode\bgroup\exampleFont \begin{shaded}\noindent\mbox{}{<\textbf{imprimatur}>}Licensed and entred acording to Order.{</\textbf{imprimatur}>}\end{shaded}\egroup 


    \item[{Content model}]
  \mbox{}\hfill\\[-10pt]\begin{Verbatim}[fontsize=\small]
<content>
 <macroRef key="macro.paraContent"/>
</content>
    
\end{Verbatim}

    \item[{Schema Declaration}]
  \mbox{}\hfill\\[-10pt]\begin{Verbatim}[fontsize=\small]
element imprimatur { att.global.attributes, macro.paraContent }
\end{Verbatim}

\end{reflist}  \index{imprint=<imprint>|oddindex}
\begin{reflist}
\item[]\begin{specHead}{TEI.imprint}{<imprint> }groups information relating to the publication or distribution of a bibliographic item. [\xref{http://www.tei-c.org/release/doc/tei-p5-doc/en/html/CO.html\#COBICOI}{3.11.2.4. Imprint, Size of a Document, and Reprint Information}]\end{specHead} 
    \item[{Module}]
  core
    \item[{Attributes}]
  Attributes att.global (\textit{@xml:id}, \textit{@n}, \textit{@xml:lang}, \textit{@xml:base}, \textit{@xml:space})  (att.global.rendition (\textit{@rend}, \textit{@style}, \textit{@rendition})) (att.global.linking (\textit{@corresp}, \textit{@synch}, \textit{@sameAs}, \textit{@copyOf}, \textit{@next}, \textit{@prev}, \textit{@exclude}, \textit{@select})) (att.global.analytic (\textit{@ana})) (att.global.facs (\textit{@facs})) (att.global.change (\textit{@change})) (att.global.responsibility (\textit{@cert}, \textit{@resp})) (att.global.source (\textit{@source}))
    \item[{Contained by}]
  
    \item[core: ]
   monogr
    \item[{May contain}]
  
    \item[analysis: ]
   interp interpGrp span spanGrp\par 
    \item[core: ]
   biblScope cb date gap gb index lb milestone note pb pubPlace publisher respStmt time\par 
    \item[figures: ]
   figure notatedMusic\par 
    \item[header: ]
   catRef classCode distributor\par 
    \item[linking: ]
   alt altGrp anchor join joinGrp link linkGrp timeline\par 
    \item[textcrit: ]
   app witDetail\par 
    \item[transcr: ]
   addSpan damageSpan delSpan fw listTranspose metamark space substJoin
    \item[{Example}]
  \leavevmode\bgroup\exampleFont \begin{shaded}\noindent\mbox{}{<\textbf{imprint}>}\mbox{}\newline 
\hspace*{6pt}{<\textbf{pubPlace}>}Oxford{</\textbf{pubPlace}>}\mbox{}\newline 
\hspace*{6pt}{<\textbf{publisher}>}Clarendon Press{</\textbf{publisher}>}\mbox{}\newline 
\hspace*{6pt}{<\textbf{date}>}1987{</\textbf{date}>}\mbox{}\newline 
{</\textbf{imprint}>}\end{shaded}\egroup 


    \item[{Content model}]
  \mbox{}\hfill\\[-10pt]\begin{Verbatim}[fontsize=\small]
<content>
 <sequence>
  <alternate maxOccurs="unbounded"
   minOccurs="0">
   <elementRef key="classCode"/>
   <elementRef key="catRef"/>
  </alternate>
  <sequence maxOccurs="unbounded"
   minOccurs="1">
   <alternate>
    <classRef key="model.imprintPart"/>
    <classRef key="model.dateLike"/>
   </alternate>
   <elementRef key="respStmt"
    maxOccurs="unbounded" minOccurs="0"/>
   <classRef key="model.global"
    maxOccurs="unbounded" minOccurs="0"/>
  </sequence>
 </sequence>
</content>
    
\end{Verbatim}

    \item[{Schema Declaration}]
  \mbox{}\hfill\\[-10pt]\begin{Verbatim}[fontsize=\small]
element imprint
{
   att.global.attributes,
   (
      ( classCode | catRef )*,
      ( ( model.imprintPart | model.dateLike ), respStmt*, model.global* )+
   )
}
\end{Verbatim}

\end{reflist}  \index{incipit=<incipit>|oddindex}
\begin{reflist}
\item[]\begin{specHead}{TEI.incipit}{<incipit> }contains the \textit{incipit} of a manuscript item, that is the opening words of the text proper, exclusive of any \textit{rubric} which might precede it, of sufficient length to identify the work uniquely; such incipits were, in former times, frequently used a means of reference to a work, in place of a title. [\xref{http://www.tei-c.org/release/doc/tei-p5-doc/en/html/MS.html\#mscoit}{10.6.1. The msItem and msItemStruct Elements}]\end{specHead} 
    \item[{Module}]
  msdescription
    \item[{Attributes}]
  Attributes att.global (\textit{@xml:id}, \textit{@n}, \textit{@xml:lang}, \textit{@xml:base}, \textit{@xml:space})  (att.global.rendition (\textit{@rend}, \textit{@style}, \textit{@rendition})) (att.global.linking (\textit{@corresp}, \textit{@synch}, \textit{@sameAs}, \textit{@copyOf}, \textit{@next}, \textit{@prev}, \textit{@exclude}, \textit{@select})) (att.global.analytic (\textit{@ana})) (att.global.facs (\textit{@facs})) (att.global.change (\textit{@change})) (att.global.responsibility (\textit{@cert}, \textit{@resp})) (att.global.source (\textit{@source})) att.typed (\textit{@type}, \textit{@subtype}) att.msExcerpt (\textit{@defective}) 
    \item[{Member of}]
  model.msQuoteLike 
    \item[{Contained by}]
  
    \item[msdescription: ]
   msItem msItemStruct
    \item[{May contain}]
  
    \item[analysis: ]
   c cl interp interpGrp m pc phr s span spanGrp w\par 
    \item[core: ]
   abbr add address cb choice corr date del distinct email emph expan foreign gap gb gloss graphic hi index lb measure measureGrp media mentioned milestone name note num orig pb ptr ref reg rs sic soCalled term time title unclear\par 
    \item[figures: ]
   figure formula notatedMusic\par 
    \item[gaiji: ]
   g\par 
    \item[header: ]
   idno\par 
    \item[linking: ]
   alt altGrp anchor join joinGrp link linkGrp seg timeline\par 
    \item[msdescription: ]
   catchwords depth dim dimensions height heraldry locus locusGrp material objectType origDate origPlace secFol signatures stamp watermark width\par 
    \item[namesdates: ]
   addName affiliation bloc climate country district forename genName geo geogFeat geogName location nameLink offset orgName persName placeName population region roleName settlement state surname terrain trait\par 
    \item[textcrit: ]
   app witDetail\par 
    \item[transcr: ]
   addSpan am damage damageSpan delSpan ex fw handShift listTranspose metamark mod redo restore retrace secl space subst substJoin supplied surplus undo\par character data
    \item[{Example}]
  \leavevmode\bgroup\exampleFont \begin{shaded}\noindent\mbox{}{<\textbf{incipit}>}Pater noster qui es in celis{</\textbf{incipit}>}\mbox{}\newline 
{<\textbf{incipit}\hspace*{6pt}{defective}="{true}">}tatem dedit hominibus alleluia.{</\textbf{incipit}>}\mbox{}\newline 
{<\textbf{incipit}\hspace*{6pt}{type}="{biblical}">}Ghif ons huden onse dagelix broet{</\textbf{incipit}>}\mbox{}\newline 
{<\textbf{incipit}>}O ongehoerde gewerdighe christi{</\textbf{incipit}>}\mbox{}\newline 
{<\textbf{incipit}\hspace*{6pt}{type}="{lemma}">}Firmiter{</\textbf{incipit}>}\mbox{}\newline 
{<\textbf{incipit}>}Ideo dicit firmiter quia ordo fidei nostre probari non potest{</\textbf{incipit}>}\end{shaded}\egroup 


    \item[{Content model}]
  \mbox{}\hfill\\[-10pt]\begin{Verbatim}[fontsize=\small]
<content>
 <macroRef key="macro.phraseSeq"/>
</content>
    
\end{Verbatim}

    \item[{Schema Declaration}]
  \mbox{}\hfill\\[-10pt]\begin{Verbatim}[fontsize=\small]
element incipit
{
   att.global.attributes,
   att.typed.attributes,
   att.msExcerpt.attributes,
   macro.phraseSeq}
\end{Verbatim}

\end{reflist}  \index{index=<index>|oddindex}\index{indexName=@indexName!<index>|oddindex}
\begin{reflist}
\item[]\begin{specHead}{TEI.index}{<index> }(index entry) marks a location to be indexed for whatever purpose. [\xref{http://www.tei-c.org/release/doc/tei-p5-doc/en/html/CO.html\#CONOIX}{3.8.2. Index Entries}]\end{specHead} 
    \item[{Module}]
  core
    \item[{Attributes}]
  Attributes att.global (\textit{@xml:id}, \textit{@n}, \textit{@xml:lang}, \textit{@xml:base}, \textit{@xml:space})  (att.global.rendition (\textit{@rend}, \textit{@style}, \textit{@rendition})) (att.global.linking (\textit{@corresp}, \textit{@synch}, \textit{@sameAs}, \textit{@copyOf}, \textit{@next}, \textit{@prev}, \textit{@exclude}, \textit{@select})) (att.global.analytic (\textit{@ana})) (att.global.facs (\textit{@facs})) (att.global.change (\textit{@change})) (att.global.responsibility (\textit{@cert}, \textit{@resp})) (att.global.source (\textit{@source})) att.spanning (\textit{@spanTo}) \hfil\\[-10pt]\begin{sansreflist}
    \item[@indexName]
  a single word which follows the rules defining a legal XML name (see \url{http://www.w3.org/TR/REC-xml/\#dt-name}), supplying a name to specify which index (of several) the index entry belongs to.
\begin{reflist}
    \item[{Status}]
  Optional
    \item[{Datatype}]
  teidata.name
    \item[{Note}]
  \par
This attribute makes it possible to create multiple indexes for a text.
\end{reflist}  
\end{sansreflist}  
    \item[{Member of}]
  model.global.meta
    \item[{Contained by}]
  
    \item[analysis: ]
   cl m phr s span w\par 
    \item[core: ]
   abbr add addrLine address author bibl biblScope cit citedRange corr date del distinct editor email emph expan foreign gloss head headItem headLabel hi imprint index item l label lg list measure mentioned name note num orig p pubPlace publisher q quote ref reg resp rs said series sic soCalled sp speaker stage street term textLang time title unclear\par 
    \item[figures: ]
   cell figure table\par 
    \item[header: ]
   authority change classCode distributor edition extent funder geoDecl handNote language licence principal scriptNote sponsor typeNote\par 
    \item[linking: ]
   ab seg\par 
    \item[msdescription: ]
   accMat acquisition additions catchwords collation colophon condition custEvent decoNote explicit filiation finalRubric foliation heraldry incipit layout material msItem musicNotation objectType origDate origPlace origin provenance rubric secFol signatures source stamp summary support surrogates watermark\par 
    \item[namesdates: ]
   addName affiliation age birth bloc country death district education faith floruit forename genName geogFeat geogName langKnown nameLink nationality occupation offset orgName persName person personGrp placeName region residence roleName settlement sex socecStatus surname\par 
    \item[textcrit: ]
   lem rdg wit witDetail\par 
    \item[textstructure: ]
   argument back body byline closer dateline div docAuthor docDate docEdition docImprint docTitle epigraph floatingText front group imprimatur opener postscript salute signed text titlePage titlePart trailer\par 
    \item[transcr: ]
   damage fw line metamark mod restore retrace secl sourceDoc supplied surface surfaceGrp surplus zone
    \item[{May contain}]
  
    \item[core: ]
   index term
    \item[{Example}]
  \leavevmode\bgroup\exampleFont \begin{shaded}\noindent\mbox{}David's other principal backer, Josiah ha-Kohen\mbox{}\newline 
{<\textbf{index}\hspace*{6pt}{indexName}="{NAMES}">}\mbox{}\newline 
\hspace*{6pt}{<\textbf{term}>}Josiah ha-Kohen b. Azarya{</\textbf{term}>}\mbox{}\newline 
{</\textbf{index}>} b. Azarya, son of one of the last gaons of Sura {<\textbf{index}\hspace*{6pt}{indexName}="{PLACES}">}\mbox{}\newline 
\hspace*{6pt}{<\textbf{term}>}Sura{</\textbf{term}>}\mbox{}\newline 
{</\textbf{index}>} was David's own first cousin.\end{shaded}\egroup 


    \item[{Content model}]
  \mbox{}\hfill\\[-10pt]\begin{Verbatim}[fontsize=\small]
<content>
 <sequence maxOccurs="unbounded"
  minOccurs="0">
  <elementRef key="term"/>
  <elementRef key="index" minOccurs="0"/>
 </sequence>
</content>
    
\end{Verbatim}

    \item[{Schema Declaration}]
  \mbox{}\hfill\\[-10pt]\begin{Verbatim}[fontsize=\small]
element index
{
   att.global.attributes,
   att.spanning.attributes,
   attribute indexName { text }?,
   ( term, index? )*
}
\end{Verbatim}

\end{reflist}  \index{institution=<institution>|oddindex}
\begin{reflist}
\item[]\begin{specHead}{TEI.institution}{<institution> }contains the name of an organization such as a university or library, with which a manuscript is identified, generally its holding institution. [\xref{http://www.tei-c.org/release/doc/tei-p5-doc/en/html/MS.html\#msid}{10.4. The Manuscript Identifier}]\end{specHead} 
    \item[{Module}]
  msdescription
    \item[{Attributes}]
  Attributes att.global (\textit{@xml:id}, \textit{@n}, \textit{@xml:lang}, \textit{@xml:base}, \textit{@xml:space})  (att.global.rendition (\textit{@rend}, \textit{@style}, \textit{@rendition})) (att.global.linking (\textit{@corresp}, \textit{@synch}, \textit{@sameAs}, \textit{@copyOf}, \textit{@next}, \textit{@prev}, \textit{@exclude}, \textit{@select})) (att.global.analytic (\textit{@ana})) (att.global.facs (\textit{@facs})) (att.global.change (\textit{@change})) (att.global.responsibility (\textit{@cert}, \textit{@resp})) (att.global.source (\textit{@source})) att.naming (\textit{@role}, \textit{@nymRef})  (att.canonical (\textit{@key}, \textit{@ref}))
    \item[{Contained by}]
  
    \item[msdescription: ]
   altIdentifier msIdentifier
    \item[{May contain}]
  
    \item[gaiji: ]
   g\par character data
    \item[{Example}]
  \leavevmode\bgroup\exampleFont \begin{shaded}\noindent\mbox{}{<\textbf{msIdentifier}>}\mbox{}\newline 
\hspace*{6pt}{<\textbf{settlement}>}Oxford{</\textbf{settlement}>}\mbox{}\newline 
\hspace*{6pt}{<\textbf{institution}>}University of Oxford{</\textbf{institution}>}\mbox{}\newline 
\hspace*{6pt}{<\textbf{repository}>}Bodleian Library{</\textbf{repository}>}\mbox{}\newline 
\hspace*{6pt}{<\textbf{idno}>}MS. Bodley 406{</\textbf{idno}>}\mbox{}\newline 
{</\textbf{msIdentifier}>}\end{shaded}\egroup 


    \item[{Content model}]
  \fbox{\ttfamily <content>\newline
 <macroRef key="macro.xtext"/>\newline
</content>\newline
    } 
    \item[{Schema Declaration}]
  \mbox{}\hfill\\[-10pt]\begin{Verbatim}[fontsize=\small]
element institution
{
   att.global.attributes,
   att.naming.attributes,
   macro.xtext}
\end{Verbatim}

\end{reflist}  \index{interp=<interp>|oddindex}
\begin{reflist}
\item[]\begin{specHead}{TEI.interp}{<interp> }(interpretation) summarizes a specific interpretative annotation which can be linked to a span of text. [\xref{http://www.tei-c.org/release/doc/tei-p5-doc/en/html/AI.html\#AISP}{17.3. Spans and Interpretations}]\end{specHead} 
    \item[{Module}]
  analysis
    \item[{Attributes}]
  Attributes att.global (\textit{@xml:id}, \textit{@n}, \textit{@xml:lang}, \textit{@xml:base}, \textit{@xml:space})  (att.global.rendition (\textit{@rend}, \textit{@style}, \textit{@rendition})) (att.global.linking (\textit{@corresp}, \textit{@synch}, \textit{@sameAs}, \textit{@copyOf}, \textit{@next}, \textit{@prev}, \textit{@exclude}, \textit{@select})) (att.global.analytic (\textit{@ana})) (att.global.facs (\textit{@facs})) (att.global.change (\textit{@change})) (att.global.responsibility (\textit{@cert}, \textit{@resp})) (att.global.source (\textit{@source})) att.interpLike (\textit{@type}, \textit{@inst}) 
    \item[{Member of}]
  model.global.meta
    \item[{Contained by}]
  
    \item[analysis: ]
   cl interpGrp m phr s span w\par 
    \item[core: ]
   abbr add addrLine address author bibl biblScope cit citedRange corr date del distinct editor email emph expan foreign gloss head headItem headLabel hi imprint item l label lg list measure mentioned name note num orig p pubPlace publisher q quote ref reg resp rs said series sic soCalled sp speaker stage street term textLang time title unclear\par 
    \item[figures: ]
   cell figure table\par 
    \item[header: ]
   authority change classCode distributor edition extent funder geoDecl handNote language licence principal scriptNote sponsor typeNote\par 
    \item[linking: ]
   ab seg\par 
    \item[msdescription: ]
   accMat acquisition additions catchwords collation colophon condition custEvent decoNote explicit filiation finalRubric foliation heraldry incipit layout material msItem musicNotation objectType origDate origPlace origin provenance rubric secFol signatures source stamp summary support surrogates watermark\par 
    \item[namesdates: ]
   addName affiliation age birth bloc country death district education faith floruit forename genName geogFeat geogName langKnown nameLink nationality occupation offset orgName persName person personGrp placeName region residence roleName settlement sex socecStatus surname\par 
    \item[textcrit: ]
   lem rdg wit witDetail\par 
    \item[textstructure: ]
   argument back body byline closer dateline div docAuthor docDate docEdition docImprint docTitle epigraph floatingText front group imprimatur opener postscript salute signed text titlePage titlePart trailer\par 
    \item[transcr: ]
   damage fw line metamark mod restore retrace secl sourceDoc supplied surface surfaceGrp surplus zone
    \item[{May contain}]
  
    \item[core: ]
   desc\par 
    \item[gaiji: ]
   g\par character data
    \item[{Note}]
  \par
Generally, each <interp> element carries an {\itshape xml:id} attribute. This permits the encoder to explicitly associate the interpretation represented by the content of an <interp> with any textual element through its {\itshape ana} attribute.\par
Alternatively (or, in addition) an <interp> may carry an {\itshape inst} attribute which points to one or more textual elements to which the analysis represented by the content of the <interp> applies.
    \item[{Example}]
  \leavevmode\bgroup\exampleFont \begin{shaded}\noindent\mbox{}{<\textbf{interp}\hspace*{6pt}{type}="{structuralunit}"\mbox{}\newline 
\hspace*{6pt}{xml:id}="{ana\textunderscore am}">}aftermath{</\textbf{interp}>}\end{shaded}\egroup 


    \item[{Content model}]
  \mbox{}\hfill\\[-10pt]\begin{Verbatim}[fontsize=\small]
<content>
 <alternate maxOccurs="unbounded"
  minOccurs="0">
  <textNode/>
  <classRef key="model.gLike"/>
  <classRef key="model.descLike"/>
  <classRef key="model.certLike"/>
 </alternate>
</content>
    
\end{Verbatim}

    \item[{Schema Declaration}]
  \mbox{}\hfill\\[-10pt]\begin{Verbatim}[fontsize=\small]
element interp
{
   att.global.attributes,
   att.interpLike.attributes,
   ( text | model.gLike | model.descLike | model.certLike )*
}
\end{Verbatim}

\end{reflist}  \index{interpGrp=<interpGrp>|oddindex}
\begin{reflist}
\item[]\begin{specHead}{TEI.interpGrp}{<interpGrp> }(interpretation group) collects together a set of related interpretations which share responsibility or type. [\xref{http://www.tei-c.org/release/doc/tei-p5-doc/en/html/AI.html\#AISP}{17.3. Spans and Interpretations}]\end{specHead} 
    \item[{Module}]
  analysis
    \item[{Attributes}]
  Attributes att.global (\textit{@xml:id}, \textit{@n}, \textit{@xml:lang}, \textit{@xml:base}, \textit{@xml:space})  (att.global.rendition (\textit{@rend}, \textit{@style}, \textit{@rendition})) (att.global.linking (\textit{@corresp}, \textit{@synch}, \textit{@sameAs}, \textit{@copyOf}, \textit{@next}, \textit{@prev}, \textit{@exclude}, \textit{@select})) (att.global.analytic (\textit{@ana})) (att.global.facs (\textit{@facs})) (att.global.change (\textit{@change})) (att.global.responsibility (\textit{@cert}, \textit{@resp})) (att.global.source (\textit{@source})) att.interpLike (\textit{@type}, \textit{@inst}) 
    \item[{Member of}]
  model.global.meta
    \item[{Contained by}]
  
    \item[analysis: ]
   cl m phr s span w\par 
    \item[core: ]
   abbr add addrLine address author bibl biblScope cit citedRange corr date del distinct editor email emph expan foreign gloss head headItem headLabel hi imprint item l label lg list measure mentioned name note num orig p pubPlace publisher q quote ref reg resp rs said series sic soCalled sp speaker stage street term textLang time title unclear\par 
    \item[figures: ]
   cell figure table\par 
    \item[header: ]
   authority change classCode distributor edition extent funder geoDecl handNote language licence principal scriptNote sponsor typeNote\par 
    \item[linking: ]
   ab seg\par 
    \item[msdescription: ]
   accMat acquisition additions catchwords collation colophon condition custEvent decoNote explicit filiation finalRubric foliation heraldry incipit layout material msItem musicNotation objectType origDate origPlace origin provenance rubric secFol signatures source stamp summary support surrogates watermark\par 
    \item[namesdates: ]
   addName affiliation age birth bloc country death district education faith floruit forename genName geogFeat geogName langKnown nameLink nationality occupation offset orgName persName person personGrp placeName region residence roleName settlement sex socecStatus surname\par 
    \item[textcrit: ]
   lem rdg wit witDetail\par 
    \item[textstructure: ]
   argument back body byline closer dateline div docAuthor docDate docEdition docImprint docTitle epigraph floatingText front group imprimatur opener postscript salute signed text titlePage titlePart trailer\par 
    \item[transcr: ]
   damage fw line metamark mod restore retrace secl sourceDoc supplied surface surfaceGrp surplus zone
    \item[{May contain}]
  
    \item[analysis: ]
   interp\par 
    \item[core: ]
   desc
    \item[{Note}]
  \par
Any number of <interp> elements.
    \item[{Example}]
  \leavevmode\bgroup\exampleFont \begin{shaded}\noindent\mbox{}{<\textbf{interpGrp}\hspace*{6pt}{resp}="{\#TMA}"\mbox{}\newline 
\hspace*{6pt}{type}="{structuralunit}">}\mbox{}\newline 
\hspace*{6pt}{<\textbf{desc}>}basic structural organization{</\textbf{desc}>}\mbox{}\newline 
\hspace*{6pt}{<\textbf{interp}\hspace*{6pt}{xml:id}="{I1}">}introduction{</\textbf{interp}>}\mbox{}\newline 
\hspace*{6pt}{<\textbf{interp}\hspace*{6pt}{xml:id}="{I2}">}conflict{</\textbf{interp}>}\mbox{}\newline 
\hspace*{6pt}{<\textbf{interp}\hspace*{6pt}{xml:id}="{I3}">}climax{</\textbf{interp}>}\mbox{}\newline 
\hspace*{6pt}{<\textbf{interp}\hspace*{6pt}{xml:id}="{I4}">}revenge{</\textbf{interp}>}\mbox{}\newline 
\hspace*{6pt}{<\textbf{interp}\hspace*{6pt}{xml:id}="{I5}">}reconciliation{</\textbf{interp}>}\mbox{}\newline 
\hspace*{6pt}{<\textbf{interp}\hspace*{6pt}{xml:id}="{I6}">}aftermath{</\textbf{interp}>}\mbox{}\newline 
{</\textbf{interpGrp}>}\mbox{}\newline 
{<\textbf{bibl}\hspace*{6pt}{xml:id}="{TMA}">}\mbox{}\newline 
\textit{<!-- bibliographic citation for source of this\newline
interpretive framework -->}\mbox{}\newline 
{</\textbf{bibl}>}\end{shaded}\egroup 


    \item[{Content model}]
  \mbox{}\hfill\\[-10pt]\begin{Verbatim}[fontsize=\small]
<content>
 <sequence>
  <classRef key="model.descLike"
   maxOccurs="unbounded" minOccurs="0"/>
  <elementRef key="interp"
   maxOccurs="unbounded" minOccurs="1"/>
 </sequence>
</content>
    
\end{Verbatim}

    \item[{Schema Declaration}]
  \mbox{}\hfill\\[-10pt]\begin{Verbatim}[fontsize=\small]
element interpGrp
{
   att.global.attributes,
   att.interpLike.attributes,
   ( model.descLike*, interp+ )
}
\end{Verbatim}

\end{reflist}  \index{interpretation=<interpretation>|oddindex}
\begin{reflist}
\item[]\begin{specHead}{TEI.interpretation}{<interpretation> }describes the scope of any analytic or interpretive information added to the text in addition to the transcription. [\xref{http://www.tei-c.org/release/doc/tei-p5-doc/en/html/HD.html\#HD53}{2.3.3. The Editorial Practices Declaration}]\end{specHead} 
    \item[{Module}]
  header
    \item[{Attributes}]
  Attributes att.global (\textit{@xml:id}, \textit{@n}, \textit{@xml:lang}, \textit{@xml:base}, \textit{@xml:space})  (att.global.rendition (\textit{@rend}, \textit{@style}, \textit{@rendition})) (att.global.linking (\textit{@corresp}, \textit{@synch}, \textit{@sameAs}, \textit{@copyOf}, \textit{@next}, \textit{@prev}, \textit{@exclude}, \textit{@select})) (att.global.analytic (\textit{@ana})) (att.global.facs (\textit{@facs})) (att.global.change (\textit{@change})) (att.global.responsibility (\textit{@cert}, \textit{@resp})) (att.global.source (\textit{@source})) att.declarable (\textit{@default}) 
    \item[{Member of}]
  model.editorialDeclPart
    \item[{Contained by}]
  
    \item[header: ]
   editorialDecl
    \item[{May contain}]
  
    \item[core: ]
   p\par 
    \item[linking: ]
   ab
    \item[{Example}]
  \leavevmode\bgroup\exampleFont \begin{shaded}\noindent\mbox{}{<\textbf{interpretation}>}\mbox{}\newline 
\hspace*{6pt}{<\textbf{p}>}The part of speech analysis applied throughout section 4 was added by hand and has not\mbox{}\newline 
\hspace*{6pt}\hspace*{6pt} been checked{</\textbf{p}>}\mbox{}\newline 
{</\textbf{interpretation}>}\end{shaded}\egroup 


    \item[{Content model}]
  \mbox{}\hfill\\[-10pt]\begin{Verbatim}[fontsize=\small]
<content>
 <classRef key="model.pLike"
  maxOccurs="unbounded" minOccurs="1"/>
</content>
    
\end{Verbatim}

    \item[{Schema Declaration}]
  \mbox{}\hfill\\[-10pt]\begin{Verbatim}[fontsize=\small]
element interpretation
{
   att.global.attributes,
   att.declarable.attributes,
   model.pLike+
}
\end{Verbatim}

\end{reflist}  \index{item=<item>|oddindex}
\begin{reflist}
\item[]\begin{specHead}{TEI.item}{<item> }contains one component of a list. [\xref{http://www.tei-c.org/release/doc/tei-p5-doc/en/html/CO.html\#COLI}{3.7. Lists} \xref{http://www.tei-c.org/release/doc/tei-p5-doc/en/html/HD.html\#HD6}{2.6. The Revision Description}]\end{specHead} 
    \item[{Module}]
  core
    \item[{Attributes}]
  Attributes att.global (\textit{@xml:id}, \textit{@n}, \textit{@xml:lang}, \textit{@xml:base}, \textit{@xml:space})  (att.global.rendition (\textit{@rend}, \textit{@style}, \textit{@rendition})) (att.global.linking (\textit{@corresp}, \textit{@synch}, \textit{@sameAs}, \textit{@copyOf}, \textit{@next}, \textit{@prev}, \textit{@exclude}, \textit{@select})) (att.global.analytic (\textit{@ana})) (att.global.facs (\textit{@facs})) (att.global.change (\textit{@change})) (att.global.responsibility (\textit{@cert}, \textit{@resp})) (att.global.source (\textit{@source})) att.sortable (\textit{@sortKey}) 
    \item[{Contained by}]
  
    \item[core: ]
   list
    \item[{May contain}]
  
    \item[analysis: ]
   c cl interp interpGrp m pc phr s span spanGrp w\par 
    \item[core: ]
   abbr add address bibl biblStruct cb choice cit corr date del desc distinct email emph expan foreign gap gb gloss graphic hi index l label lb lg list listBibl measure measureGrp media mentioned milestone name note num orig p pb ptr q quote ref reg rs said sic soCalled sp stage term time title unclear\par 
    \item[figures: ]
   figure formula notatedMusic table\par 
    \item[gaiji: ]
   g\par 
    \item[header: ]
   biblFull idno\par 
    \item[linking: ]
   ab alt altGrp anchor join joinGrp link linkGrp seg timeline\par 
    \item[msdescription: ]
   catchwords depth dim dimensions height heraldry locus locusGrp material msDesc objectType origDate origPlace secFol signatures stamp watermark width\par 
    \item[namesdates: ]
   addName affiliation bloc climate country district forename genName geo geogFeat geogName listEvent listNym listOrg listPerson listPlace location nameLink offset orgName persName placeName population region roleName settlement state surname terrain trait\par 
    \item[textcrit: ]
   app listApp listWit witDetail\par 
    \item[textstructure: ]
   floatingText\par 
    \item[transcr: ]
   addSpan am damage damageSpan delSpan ex fw handShift listTranspose metamark mod redo restore retrace secl space subst substJoin supplied surplus undo\par character data
    \item[{Note}]
  \par
May contain simple prose or a sequence of chunks.\par
Whatever string of characters is used to label a list item in the copy text may be used as the value of the global {\itshape n} attribute, but it is not required that numbering be recorded explicitly. In ordered lists, the {\itshape n} attribute on the <item> element is by definition synonymous with the use of the <label> element to record the enumerator of the list item. In glossary lists, however, the term being defined should be given with the <label> element, not {\itshape n}.
    \item[{Example}]
  \leavevmode\bgroup\exampleFont \begin{shaded}\noindent\mbox{}{<\textbf{list}\hspace*{6pt}{rend}="{numbered}">}\mbox{}\newline 
\hspace*{6pt}{<\textbf{head}>}Here begin the chapter headings of Book IV{</\textbf{head}>}\mbox{}\newline 
\hspace*{6pt}{<\textbf{item}\hspace*{6pt}{n}="{4.1}">}The death of Queen Clotild.{</\textbf{item}>}\mbox{}\newline 
\hspace*{6pt}{<\textbf{item}\hspace*{6pt}{n}="{4.2}">}How King Lothar wanted to appropriate one third of the Church revenues.{</\textbf{item}>}\mbox{}\newline 
\hspace*{6pt}{<\textbf{item}\hspace*{6pt}{n}="{4.3}">}The wives and children of Lothar.{</\textbf{item}>}\mbox{}\newline 
\hspace*{6pt}{<\textbf{item}\hspace*{6pt}{n}="{4.4}">}The Counts of the Bretons.{</\textbf{item}>}\mbox{}\newline 
\hspace*{6pt}{<\textbf{item}\hspace*{6pt}{n}="{4.5}">}Saint Gall the Bishop.{</\textbf{item}>}\mbox{}\newline 
\hspace*{6pt}{<\textbf{item}\hspace*{6pt}{n}="{4.6}">}The priest Cato.{</\textbf{item}>}\mbox{}\newline 
\hspace*{6pt}{<\textbf{item}>} ...{</\textbf{item}>}\mbox{}\newline 
{</\textbf{list}>}\end{shaded}\egroup 


    \item[{Content model}]
  \mbox{}\hfill\\[-10pt]\begin{Verbatim}[fontsize=\small]
<content>
 <macroRef key="macro.specialPara"/>
</content>
    
\end{Verbatim}

    \item[{Schema Declaration}]
  \mbox{}\hfill\\[-10pt]\begin{Verbatim}[fontsize=\small]
element item
{
   att.global.attributes,
   att.sortable.attributes,
   macro.specialPara}
\end{Verbatim}

\end{reflist}  \index{join=<join>|oddindex}\index{result=@result!<join>|oddindex}\index{scope=@scope!<join>|oddindex}
\begin{reflist}
\item[]\begin{specHead}{TEI.join}{<join> }identifies a possibly fragmented segment of text, by pointing at the possibly discontiguous elements which compose it. [\xref{http://www.tei-c.org/release/doc/tei-p5-doc/en/html/SA.html\#SAAG}{16.7. Aggregation}]\end{specHead} 
    \item[{Module}]
  linking
    \item[{Attributes}]
  Attributes att.global (\textit{@xml:id}, \textit{@n}, \textit{@xml:lang}, \textit{@xml:base}, \textit{@xml:space})  (att.global.rendition (\textit{@rend}, \textit{@style}, \textit{@rendition})) (att.global.linking (\textit{@corresp}, \textit{@synch}, \textit{@sameAs}, \textit{@copyOf}, \textit{@next}, \textit{@prev}, \textit{@exclude}, \textit{@select})) (att.global.analytic (\textit{@ana})) (att.global.facs (\textit{@facs})) (att.global.change (\textit{@change})) (att.global.responsibility (\textit{@cert}, \textit{@resp})) (att.global.source (\textit{@source})) att.pointing (\textit{@targetLang}, \textit{@target}, \textit{@evaluate}) att.typed (\textit{@type}, \textit{@subtype}) \hfil\\[-10pt]\begin{sansreflist}
    \item[@result]
  specifies the name of an element which this aggregation may be understood to represent.
\begin{reflist}
    \item[{Status}]
  Optional
    \item[{Datatype}]
  teidata.name
\end{reflist}  
    \item[@scope]
  indicates whether the targets to be joined include the entire element indicated (the entire subtree including its root), or just the children of the target (the branches of the subtree).
\begin{reflist}
    \item[{Status}]
  Recommended
    \item[{Datatype}]
  teidata.enumerated
    \item[{Legal values are:}]
  \begin{description}

\item[{root}]the rooted subtrees indicated by the {\itshape target} attribute are joined, each subtree become a child of the virtual element created by the join{[Default] }
\item[{branches}]the children of the subtrees indicated by the {\itshape target} attribute become the children of the virtual element (i.e. the roots of the subtrees are discarded)
\end{description} 
\end{reflist}  
\end{sansreflist}  
    \item[{Member of}]
  model.global.meta
    \item[{Contained by}]
  
    \item[analysis: ]
   cl m phr s span w\par 
    \item[core: ]
   abbr add addrLine address author bibl biblScope cit citedRange corr date del distinct editor email emph expan foreign gloss head headItem headLabel hi imprint item l label lg list measure mentioned name note num orig p pubPlace publisher q quote ref reg resp rs said series sic soCalled sp speaker stage street term textLang time title unclear\par 
    \item[figures: ]
   cell figure table\par 
    \item[header: ]
   authority change classCode distributor edition extent funder geoDecl handNote language licence principal scriptNote sponsor typeNote\par 
    \item[linking: ]
   ab joinGrp seg\par 
    \item[msdescription: ]
   accMat acquisition additions catchwords collation colophon condition custEvent decoNote explicit filiation finalRubric foliation heraldry incipit layout material msItem musicNotation objectType origDate origPlace origin provenance rubric secFol signatures source stamp summary support surrogates watermark\par 
    \item[namesdates: ]
   addName affiliation age birth bloc country death district education faith floruit forename genName geogFeat geogName langKnown nameLink nationality occupation offset orgName persName person personGrp placeName region residence roleName settlement sex socecStatus surname\par 
    \item[textcrit: ]
   lem rdg wit witDetail\par 
    \item[textstructure: ]
   argument back body byline closer dateline div docAuthor docDate docEdition docImprint docTitle epigraph floatingText front group imprimatur opener postscript salute signed text titlePage titlePart trailer\par 
    \item[transcr: ]
   damage fw line metamark mod restore retrace secl sourceDoc supplied surface surfaceGrp surplus zone
    \item[{May contain}]
  
    \item[core: ]
   desc
    \item[{Example}]
  The following example is discussed in section SAAG:\leavevmode\bgroup\exampleFont \begin{shaded}\noindent\mbox{}{<\textbf{sp}>}\mbox{}\newline 
\hspace*{6pt}{<\textbf{speaker}>}Hughie{</\textbf{speaker}>}\mbox{}\newline 
\hspace*{6pt}{<\textbf{p}>}How does it go? {<\textbf{q}>}\mbox{}\newline 
\hspace*{6pt}\hspace*{6pt}\hspace*{6pt}{<\textbf{l}\hspace*{6pt}{xml:id}="{frog\textunderscore x1}">}da-da-da{</\textbf{l}>}\mbox{}\newline 
\hspace*{6pt}\hspace*{6pt}\hspace*{6pt}{<\textbf{l}\hspace*{6pt}{xml:id}="{frog\textunderscore l2}">}gets a new frog{</\textbf{l}>}\mbox{}\newline 
\hspace*{6pt}\hspace*{6pt}\hspace*{6pt}{<\textbf{l}>}...{</\textbf{l}>}\mbox{}\newline 
\hspace*{6pt}\hspace*{6pt}{</\textbf{q}>}\mbox{}\newline 
\hspace*{6pt}{</\textbf{p}>}\mbox{}\newline 
{</\textbf{sp}>}\mbox{}\newline 
{<\textbf{sp}>}\mbox{}\newline 
\hspace*{6pt}{<\textbf{speaker}>}Louie{</\textbf{speaker}>}\mbox{}\newline 
\hspace*{6pt}{<\textbf{p}>}\mbox{}\newline 
\hspace*{6pt}\hspace*{6pt}{<\textbf{q}>}\mbox{}\newline 
\hspace*{6pt}\hspace*{6pt}\hspace*{6pt}{<\textbf{l}\hspace*{6pt}{xml:id}="{frog\textunderscore l1}">}When the old pond{</\textbf{l}>}\mbox{}\newline 
\hspace*{6pt}\hspace*{6pt}\hspace*{6pt}{<\textbf{l}>}...{</\textbf{l}>}\mbox{}\newline 
\hspace*{6pt}\hspace*{6pt}{</\textbf{q}>}\mbox{}\newline 
\hspace*{6pt}{</\textbf{p}>}\mbox{}\newline 
{</\textbf{sp}>}\mbox{}\newline 
{<\textbf{sp}>}\mbox{}\newline 
\hspace*{6pt}{<\textbf{speaker}>}Dewey{</\textbf{speaker}>}\mbox{}\newline 
\hspace*{6pt}{<\textbf{p}>}\mbox{}\newline 
\hspace*{6pt}\hspace*{6pt}{<\textbf{q}>}... {<\textbf{l}\hspace*{6pt}{xml:id}="{frog\textunderscore l3}">}It's a new pond.{</\textbf{l}>}\mbox{}\newline 
\hspace*{6pt}\hspace*{6pt}{</\textbf{q}>}\mbox{}\newline 
\hspace*{6pt}{</\textbf{p}>}\mbox{}\newline 
\hspace*{6pt}{<\textbf{join}\hspace*{6pt}{result}="{lg}"\hspace*{6pt}{scope}="{root}"\mbox{}\newline 
\hspace*{6pt}\hspace*{6pt}{target}="{\#frog\textunderscore l1 \#frog\textunderscore l2 \#frog\textunderscore l3}"/>}\mbox{}\newline 
{</\textbf{sp}>}\end{shaded}\egroup 

The <join> element here identifies a linegroup (<lg>) comprising the three lines indicated by the {\itshape target} attribute. The value \texttt{root} for the {\itshape scope} attribute indicates that the resulting virtual element contains the three <l> elements linked to at \#frog\textunderscore l1 \#frog\textunderscore l2 \#frog\textunderscore l3, rather than their character data content.
    \item[{Example}]
  In this example, the attribute {\itshape scope} is specified with the value of \texttt{branches} to indicate that the virtual list being constructed is to be made by taking the lists indicated by the {\itshape target} attribute of the <join> element, discarding the <list> tags which enclose them, and combining the items contained within the lists into a single virtual list:\leavevmode\bgroup\exampleFont \begin{shaded}\noindent\mbox{}{<\textbf{p}>}Southern dialect (my own variety, at least) has only {<\textbf{list}\hspace*{6pt}{xml:id}="{LP1}">}\mbox{}\newline 
\hspace*{6pt}\hspace*{6pt}{<\textbf{item}>}\mbox{}\newline 
\hspace*{6pt}\hspace*{6pt}\hspace*{6pt}{<\textbf{s}>}I done gone{</\textbf{s}>}\mbox{}\newline 
\hspace*{6pt}\hspace*{6pt}{</\textbf{item}>}\mbox{}\newline 
\hspace*{6pt}\hspace*{6pt}{<\textbf{item}>}\mbox{}\newline 
\hspace*{6pt}\hspace*{6pt}\hspace*{6pt}{<\textbf{s}>}I done went{</\textbf{s}>}\mbox{}\newline 
\hspace*{6pt}\hspace*{6pt}{</\textbf{item}>}\mbox{}\newline 
\hspace*{6pt}{</\textbf{list}>} whereas Negro Non-Standard basilect has both these and {<\textbf{list}\hspace*{6pt}{xml:id}="{LP2}">}\mbox{}\newline 
\hspace*{6pt}\hspace*{6pt}{<\textbf{item}>}\mbox{}\newline 
\hspace*{6pt}\hspace*{6pt}\hspace*{6pt}{<\textbf{s}>}I done go{</\textbf{s}>}\mbox{}\newline 
\hspace*{6pt}\hspace*{6pt}{</\textbf{item}>}\mbox{}\newline 
\hspace*{6pt}{</\textbf{list}>}.{</\textbf{p}>}\mbox{}\newline 
{<\textbf{p}>}White Southern dialect also has {<\textbf{list}\hspace*{6pt}{xml:id}="{LP3}">}\mbox{}\newline 
\hspace*{6pt}\hspace*{6pt}{<\textbf{item}>}\mbox{}\newline 
\hspace*{6pt}\hspace*{6pt}\hspace*{6pt}{<\textbf{s}>}I've done gone{</\textbf{s}>}\mbox{}\newline 
\hspace*{6pt}\hspace*{6pt}{</\textbf{item}>}\mbox{}\newline 
\hspace*{6pt}\hspace*{6pt}{<\textbf{item}>}\mbox{}\newline 
\hspace*{6pt}\hspace*{6pt}\hspace*{6pt}{<\textbf{s}>}I've done went{</\textbf{s}>}\mbox{}\newline 
\hspace*{6pt}\hspace*{6pt}{</\textbf{item}>}\mbox{}\newline 
\hspace*{6pt}{</\textbf{list}>} which, when they occur in Negro dialect, should probably be considered as borrowings from other varieties of English.{</\textbf{p}>}\mbox{}\newline 
{<\textbf{join}\hspace*{6pt}{result}="{list}"\hspace*{6pt}{scope}="{branches}"\mbox{}\newline 
\hspace*{6pt}{target}="{\#LP1 \#LP2 \#LP3}"\hspace*{6pt}{xml:id}="{LST1}">}\mbox{}\newline 
\hspace*{6pt}{<\textbf{desc}>}Sample sentences in Southern speech{</\textbf{desc}>}\mbox{}\newline 
{</\textbf{join}>}\end{shaded}\egroup 


    \item[{Schematron}]
   <s:assert test="contains(@target,' ')">You must supply at least two values for @target on <s:name/> </s:assert>
    \item[{Content model}]
  \mbox{}\hfill\\[-10pt]\begin{Verbatim}[fontsize=\small]
<content>
 <alternate maxOccurs="unbounded"
  minOccurs="0">
  <classRef key="model.descLike"/>
  <classRef key="model.certLike"/>
 </alternate>
</content>
    
\end{Verbatim}

    \item[{Schema Declaration}]
  \mbox{}\hfill\\[-10pt]\begin{Verbatim}[fontsize=\small]
element join
{
   att.global.attributes,
   att.pointing.attributes,
   att.typed.attributes,
   attribute result { text }?,
   attribute scope { "root" | "branches" }?,
   ( model.descLike | model.certLike )*
}
\end{Verbatim}

\end{reflist}  \index{joinGrp=<joinGrp>|oddindex}\index{result=@result!<joinGrp>|oddindex}
\begin{reflist}
\item[]\begin{specHead}{TEI.joinGrp}{<joinGrp> }(join group) groups a collection of join elements and possibly pointers. [\xref{http://www.tei-c.org/release/doc/tei-p5-doc/en/html/SA.html\#SAAG}{16.7. Aggregation}]\end{specHead} 
    \item[{Module}]
  linking
    \item[{Attributes}]
  Attributes att.global (\textit{@xml:id}, \textit{@n}, \textit{@xml:lang}, \textit{@xml:base}, \textit{@xml:space})  (att.global.rendition (\textit{@rend}, \textit{@style}, \textit{@rendition})) (att.global.linking (\textit{@corresp}, \textit{@synch}, \textit{@sameAs}, \textit{@copyOf}, \textit{@next}, \textit{@prev}, \textit{@exclude}, \textit{@select})) (att.global.analytic (\textit{@ana})) (att.global.facs (\textit{@facs})) (att.global.change (\textit{@change})) (att.global.responsibility (\textit{@cert}, \textit{@resp})) (att.global.source (\textit{@source})) att.pointing.group (\textit{@domains}, \textit{@targFunc})  (att.pointing (\textit{@targetLang}, \textit{@target}, \textit{@evaluate})) (att.typed (\textit{@type}, \textit{@subtype})) \hfil\\[-10pt]\begin{sansreflist}
    \item[@result]
  supplies the default value for the {\itshape result} on each <join> included within the group.
\begin{reflist}
    \item[{Status}]
  Optional
    \item[{Datatype}]
  teidata.name
\end{reflist}  
\end{sansreflist}  
    \item[{Member of}]
  model.global.meta
    \item[{Contained by}]
  
    \item[analysis: ]
   cl m phr s span w\par 
    \item[core: ]
   abbr add addrLine address author bibl biblScope cit citedRange corr date del distinct editor email emph expan foreign gloss head headItem headLabel hi imprint item l label lg list measure mentioned name note num orig p pubPlace publisher q quote ref reg resp rs said series sic soCalled sp speaker stage street term textLang time title unclear\par 
    \item[figures: ]
   cell figure table\par 
    \item[header: ]
   authority change classCode distributor edition extent funder geoDecl handNote language licence principal scriptNote sponsor typeNote\par 
    \item[linking: ]
   ab seg\par 
    \item[msdescription: ]
   accMat acquisition additions catchwords collation colophon condition custEvent decoNote explicit filiation finalRubric foliation heraldry incipit layout material msItem musicNotation objectType origDate origPlace origin provenance rubric secFol signatures source stamp summary support surrogates watermark\par 
    \item[namesdates: ]
   addName affiliation age birth bloc country death district education faith floruit forename genName geogFeat geogName langKnown nameLink nationality occupation offset orgName persName person personGrp placeName region residence roleName settlement sex socecStatus surname\par 
    \item[textcrit: ]
   lem rdg wit witDetail\par 
    \item[textstructure: ]
   argument back body byline closer dateline div docAuthor docDate docEdition docImprint docTitle epigraph floatingText front group imprimatur opener postscript salute signed text titlePage titlePart trailer\par 
    \item[transcr: ]
   damage fw line metamark mod restore retrace secl sourceDoc supplied surface surfaceGrp surplus zone
    \item[{May contain}]
  
    \item[core: ]
   gloss ptr\par 
    \item[linking: ]
   join
    \item[{Note}]
  \par
Any number of <join> or <ptr> elements.
    \item[{Example}]
  \leavevmode\bgroup\exampleFont \begin{shaded}\noindent\mbox{}{<\textbf{joinGrp}\hspace*{6pt}{domains}="{\#zuitxt1 \#zuitxt2 \#zuitxt3}"\mbox{}\newline 
\hspace*{6pt}{result}="{q}">}\mbox{}\newline 
\hspace*{6pt}{<\textbf{join}\hspace*{6pt}{target}="{\#zuiq1 \#zuiq2 \#zuiq6}"/>}\mbox{}\newline 
\hspace*{6pt}{<\textbf{join}\hspace*{6pt}{target}="{\#zuiq3 \#zuiq4 \#zuiq5}"/>}\mbox{}\newline 
{</\textbf{joinGrp}>}\end{shaded}\egroup 


    \item[{Content model}]
  \mbox{}\hfill\\[-10pt]\begin{Verbatim}[fontsize=\small]
<content>
 <sequence>
  <classRef key="model.glossLike"
   maxOccurs="unbounded" minOccurs="0"/>
  <alternate maxOccurs="unbounded"
   minOccurs="1">
   <elementRef key="join"/>
   <elementRef key="ptr"/>
  </alternate>
 </sequence>
</content>
    
\end{Verbatim}

    \item[{Schema Declaration}]
  \mbox{}\hfill\\[-10pt]\begin{Verbatim}[fontsize=\small]
element joinGrp
{
   att.global.attributes,
   att.pointing.group.attributes,
   attribute result { text }?,
   ( model.glossLike*, ( join | ptr )+ )
}
\end{Verbatim}

\end{reflist}  \index{keywords=<keywords>|oddindex}\index{scheme=@scheme!<keywords>|oddindex}
\begin{reflist}
\item[]\begin{specHead}{TEI.keywords}{<keywords> }contains a list of keywords or phrases identifying the topic or nature of a text. [\xref{http://www.tei-c.org/release/doc/tei-p5-doc/en/html/HD.html\#HD43}{2.4.3. The Text Classification}]\end{specHead} 
    \item[{Module}]
  header
    \item[{Attributes}]
  Attributes att.global (\textit{@xml:id}, \textit{@n}, \textit{@xml:lang}, \textit{@xml:base}, \textit{@xml:space})  (att.global.rendition (\textit{@rend}, \textit{@style}, \textit{@rendition})) (att.global.linking (\textit{@corresp}, \textit{@synch}, \textit{@sameAs}, \textit{@copyOf}, \textit{@next}, \textit{@prev}, \textit{@exclude}, \textit{@select})) (att.global.analytic (\textit{@ana})) (att.global.facs (\textit{@facs})) (att.global.change (\textit{@change})) (att.global.responsibility (\textit{@cert}, \textit{@resp})) (att.global.source (\textit{@source})) \hfil\\[-10pt]\begin{sansreflist}
    \item[@scheme]
  identifies the controlled vocabulary within which the set of keywords concerned is defined, for example by a <taxonomy> element, or by some other resource.
\begin{reflist}
    \item[{Status}]
  Optional
    \item[{Datatype}]
  teidata.pointer
\end{reflist}  
\end{sansreflist}  
    \item[{Contained by}]
  
    \item[header: ]
   textClass
    \item[{May contain}]
  
    \item[core: ]
   list term
    \item[{Note}]
  \par
Each individual keyword (including compound subject headings) should be supplied as a <term> element directly within the <keywords> element. An alternative usage, in which each <term> appears within a <item> inside a <list> is permitted for backwards compatibility, but is deprecated.\par
If no control list exists for the keywords used, then no value should be supplied for the {\itshape scheme} attribute.
    \item[{Example}]
  \leavevmode\bgroup\exampleFont \begin{shaded}\noindent\mbox{}{<\textbf{keywords}\hspace*{6pt}{scheme}="{http://classificationweb.net}">}\mbox{}\newline 
\hspace*{6pt}{<\textbf{term}>}Babbage, Charles{</\textbf{term}>}\mbox{}\newline 
\hspace*{6pt}{<\textbf{term}>}Mathematicians - Great Britain - Biography{</\textbf{term}>}\mbox{}\newline 
{</\textbf{keywords}>}\end{shaded}\egroup 


    \item[{Example}]
  \leavevmode\bgroup\exampleFont \begin{shaded}\noindent\mbox{}{<\textbf{keywords}>}\mbox{}\newline 
\hspace*{6pt}{<\textbf{term}>}Fermented beverages{</\textbf{term}>}\mbox{}\newline 
\hspace*{6pt}{<\textbf{term}>}Central Andes{</\textbf{term}>}\mbox{}\newline 
\hspace*{6pt}{<\textbf{term}>}Schinus molle{</\textbf{term}>}\mbox{}\newline 
\hspace*{6pt}{<\textbf{term}>}Molle beer{</\textbf{term}>}\mbox{}\newline 
\hspace*{6pt}{<\textbf{term}>}Indigenous peoples{</\textbf{term}>}\mbox{}\newline 
\hspace*{6pt}{<\textbf{term}>}Ethnography{</\textbf{term}>}\mbox{}\newline 
\hspace*{6pt}{<\textbf{term}>}Archaeology{</\textbf{term}>}\mbox{}\newline 
{</\textbf{keywords}>}\end{shaded}\egroup 


    \item[{Content model}]
  \mbox{}\hfill\\[-10pt]\begin{Verbatim}[fontsize=\small]
<content>
 <alternate>
  <elementRef key="term"
   maxOccurs="unbounded" minOccurs="1"/>
  <elementRef key="list"/>
 </alternate>
</content>
    
\end{Verbatim}

    \item[{Schema Declaration}]
  \mbox{}\hfill\\[-10pt]\begin{Verbatim}[fontsize=\small]
element keywords
{
   att.global.attributes,
   attribute scheme { text }?,
   ( term+ | list )
}
\end{Verbatim}

\end{reflist}  \index{l=<l>|oddindex}
\begin{reflist}
\item[]\begin{specHead}{TEI.l}{<l> }(verse line) contains a single, possibly incomplete, line of verse. [\xref{http://www.tei-c.org/release/doc/tei-p5-doc/en/html/CO.html\#COVE}{3.12.1. Core Tags for Verse} \xref{http://www.tei-c.org/release/doc/tei-p5-doc/en/html/CO.html\#CODV}{3.12. Passages of Verse or Drama} \xref{http://www.tei-c.org/release/doc/tei-p5-doc/en/html/DR.html\#DRPAL}{7.2.5. Speech Contents}]\end{specHead} 
    \item[{Module}]
  core
    \item[{Attributes}]
  Attributes att.global (\textit{@xml:id}, \textit{@n}, \textit{@xml:lang}, \textit{@xml:base}, \textit{@xml:space})  (att.global.rendition (\textit{@rend}, \textit{@style}, \textit{@rendition})) (att.global.linking (\textit{@corresp}, \textit{@synch}, \textit{@sameAs}, \textit{@copyOf}, \textit{@next}, \textit{@prev}, \textit{@exclude}, \textit{@select})) (att.global.analytic (\textit{@ana})) (att.global.facs (\textit{@facs})) (att.global.change (\textit{@change})) (att.global.responsibility (\textit{@cert}, \textit{@resp})) (att.global.source (\textit{@source})) att.fragmentable (\textit{@part}) 
    \item[{Member of}]
  model.lLike
    \item[{Contained by}]
  
    \item[core: ]
   add corr del emph head hi item lg note orig p q quote ref reg said sic sp stage title unclear\par 
    \item[figures: ]
   cell figure\par 
    \item[header: ]
   change handNote licence scriptNote typeNote\par 
    \item[linking: ]
   ab seg\par 
    \item[msdescription: ]
   accMat acquisition additions collation condition custEvent decoNote filiation foliation layout musicNotation origin provenance signatures source summary support surrogates\par 
    \item[namesdates: ]
   occupation\par 
    \item[textcrit: ]
   lem rdg\par 
    \item[textstructure: ]
   argument body div docEdition epigraph imprimatur postscript salute signed titlePart trailer\par 
    \item[transcr: ]
   damage metamark mod restore retrace secl supplied surplus
    \item[{May contain}]
  
    \item[analysis: ]
   c cl interp interpGrp m pc phr s span spanGrp w\par 
    \item[core: ]
   abbr add address bibl biblStruct cb choice cit corr date del desc distinct email emph expan foreign gap gb gloss graphic hi index label lb list listBibl measure measureGrp media mentioned milestone name note num orig pb ptr q quote ref reg rs said sic soCalled stage term time title unclear\par 
    \item[figures: ]
   figure formula notatedMusic table\par 
    \item[gaiji: ]
   g\par 
    \item[header: ]
   biblFull idno\par 
    \item[linking: ]
   alt altGrp anchor join joinGrp link linkGrp seg timeline\par 
    \item[msdescription: ]
   catchwords depth dim dimensions height heraldry locus locusGrp material msDesc objectType origDate origPlace secFol signatures stamp watermark width\par 
    \item[namesdates: ]
   addName affiliation bloc climate country district forename genName geo geogFeat geogName listEvent listNym listOrg listPerson listPlace location nameLink offset orgName persName placeName population region roleName settlement state surname terrain trait\par 
    \item[textcrit: ]
   app listApp listWit witDetail\par 
    \item[textstructure: ]
   floatingText\par 
    \item[transcr: ]
   addSpan am damage damageSpan delSpan ex fw handShift listTranspose metamark mod redo restore retrace secl space subst substJoin supplied surplus undo\par character data
    \item[{Example}]
  \leavevmode\bgroup\exampleFont \begin{shaded}\noindent\mbox{}{<\textbf{l}\hspace*{6pt}{met}="{x/x/x/x/x/}"\hspace*{6pt}{real}="{/xx/x/x/x/}">}Shall I compare thee to a summer's day?{</\textbf{l}>}\end{shaded}\egroup 


    \item[{Schematron}]
   <s:report test="ancestor::tei:l[not(.//tei:note//tei:l[. = current()])]"> Abstract model violation: Lines may not contain lines or lg elements. </s:report>
    \item[{Content model}]
  \mbox{}\hfill\\[-10pt]\begin{Verbatim}[fontsize=\small]
<content>
 <alternate maxOccurs="unbounded"
  minOccurs="0">
  <textNode/>
  <classRef key="model.gLike"/>
  <classRef key="model.phrase"/>
  <classRef key="model.inter"/>
  <classRef key="model.global"/>
 </alternate>
</content>
    
\end{Verbatim}

    \item[{Schema Declaration}]
  \mbox{}\hfill\\[-10pt]\begin{Verbatim}[fontsize=\small]
element l
{
   att.global.attributes,
   att.fragmentable.attributes,
   ( text | model.gLike | model.phrase | model.inter | model.global )*
}
\end{Verbatim}

\end{reflist}  \index{label=<label>|oddindex}
\begin{reflist}
\item[]\begin{specHead}{TEI.label}{<label> }contains any label or heading used to identify part of a text, typically but not exclusively in a list or glossary. [\xref{http://www.tei-c.org/release/doc/tei-p5-doc/en/html/CO.html\#COLI}{3.7. Lists}]\end{specHead} 
    \item[{Module}]
  core
    \item[{Attributes}]
  Attributes att.global (\textit{@xml:id}, \textit{@n}, \textit{@xml:lang}, \textit{@xml:base}, \textit{@xml:space})  (att.global.rendition (\textit{@rend}, \textit{@style}, \textit{@rendition})) (att.global.linking (\textit{@corresp}, \textit{@synch}, \textit{@sameAs}, \textit{@copyOf}, \textit{@next}, \textit{@prev}, \textit{@exclude}, \textit{@select})) (att.global.analytic (\textit{@ana})) (att.global.facs (\textit{@facs})) (att.global.change (\textit{@change})) (att.global.responsibility (\textit{@cert}, \textit{@resp})) (att.global.source (\textit{@source})) att.typed (\textit{@type}, \textit{@subtype}) att.placement (\textit{@place}) att.written (\textit{@hand}) 
    \item[{Member of}]
  model.labelLike
    \item[{Contained by}]
  
    \item[core: ]
   add corr del desc emph head hi item l lg list meeting note orig p q quote ref reg said sic stage title unclear\par 
    \item[figures: ]
   cell figDesc figure notatedMusic\par 
    \item[header: ]
   application change handNote licence rendition scriptNote tagUsage typeNote\par 
    \item[linking: ]
   ab seg\par 
    \item[msdescription: ]
   accMat acquisition additions collation condition custEvent decoNote filiation foliation layout musicNotation origin provenance signatures source summary support surrogates\par 
    \item[namesdates: ]
   climate event location occupation org place population state terrain trait\par 
    \item[textcrit: ]
   lem rdg witness\par 
    \item[textstructure: ]
   argument body div docEdition epigraph imprimatur postscript salute signed titlePart trailer\par 
    \item[transcr: ]
   damage metamark mod restore retrace secl supplied surface surplus
    \item[{May contain}]
  
    \item[analysis: ]
   c cl interp interpGrp m pc phr s span spanGrp w\par 
    \item[core: ]
   abbr add address cb choice corr date del distinct email emph expan foreign gap gb gloss graphic hi index lb measure measureGrp media mentioned milestone name note num orig pb ptr ref reg rs sic soCalled term time title unclear\par 
    \item[figures: ]
   figure formula notatedMusic\par 
    \item[gaiji: ]
   g\par 
    \item[header: ]
   idno\par 
    \item[linking: ]
   alt altGrp anchor join joinGrp link linkGrp seg timeline\par 
    \item[msdescription: ]
   catchwords depth dim dimensions height heraldry locus locusGrp material objectType origDate origPlace secFol signatures stamp watermark width\par 
    \item[namesdates: ]
   addName affiliation bloc climate country district forename genName geo geogFeat geogName location nameLink offset orgName persName placeName population region roleName settlement state surname terrain trait\par 
    \item[textcrit: ]
   app witDetail\par 
    \item[transcr: ]
   addSpan am damage damageSpan delSpan ex fw handShift listTranspose metamark mod redo restore retrace secl space subst substJoin supplied surplus undo\par character data
    \item[{Example}]
  Labels are commonly used for the headwords in glossary lists; note the use of the global {\itshape xml:lang} attribute to set the default language of the glossary list to Middle English, and identify the glosses and headings as modern English or Latin:\leavevmode\bgroup\exampleFont \begin{shaded}\noindent\mbox{}{<\textbf{list}\hspace*{6pt}{type}="{gloss}"\hspace*{6pt}{xml:lang}="{enm}">}\mbox{}\newline 
\hspace*{6pt}{<\textbf{head}\hspace*{6pt}{xml:lang}="{en}">}Vocabulary{</\textbf{head}>}\mbox{}\newline 
\hspace*{6pt}{<\textbf{headLabel}\hspace*{6pt}{xml:lang}="{en}">}Middle English{</\textbf{headLabel}>}\mbox{}\newline 
\hspace*{6pt}{<\textbf{headItem}\hspace*{6pt}{xml:lang}="{en}">}New English{</\textbf{headItem}>}\mbox{}\newline 
\hspace*{6pt}{<\textbf{label}>}nu{</\textbf{label}>}\mbox{}\newline 
\hspace*{6pt}{<\textbf{item}\hspace*{6pt}{xml:lang}="{en}">}now{</\textbf{item}>}\mbox{}\newline 
\hspace*{6pt}{<\textbf{label}>}lhude{</\textbf{label}>}\mbox{}\newline 
\hspace*{6pt}{<\textbf{item}\hspace*{6pt}{xml:lang}="{en}">}loudly{</\textbf{item}>}\mbox{}\newline 
\hspace*{6pt}{<\textbf{label}>}bloweth{</\textbf{label}>}\mbox{}\newline 
\hspace*{6pt}{<\textbf{item}\hspace*{6pt}{xml:lang}="{en}">}blooms{</\textbf{item}>}\mbox{}\newline 
\hspace*{6pt}{<\textbf{label}>}med{</\textbf{label}>}\mbox{}\newline 
\hspace*{6pt}{<\textbf{item}\hspace*{6pt}{xml:lang}="{en}">}meadow{</\textbf{item}>}\mbox{}\newline 
\hspace*{6pt}{<\textbf{label}>}wude{</\textbf{label}>}\mbox{}\newline 
\hspace*{6pt}{<\textbf{item}\hspace*{6pt}{xml:lang}="{en}">}wood{</\textbf{item}>}\mbox{}\newline 
\hspace*{6pt}{<\textbf{label}>}awe{</\textbf{label}>}\mbox{}\newline 
\hspace*{6pt}{<\textbf{item}\hspace*{6pt}{xml:lang}="{en}">}ewe{</\textbf{item}>}\mbox{}\newline 
\hspace*{6pt}{<\textbf{label}>}lhouth{</\textbf{label}>}\mbox{}\newline 
\hspace*{6pt}{<\textbf{item}\hspace*{6pt}{xml:lang}="{en}">}lows{</\textbf{item}>}\mbox{}\newline 
\hspace*{6pt}{<\textbf{label}>}sterteth{</\textbf{label}>}\mbox{}\newline 
\hspace*{6pt}{<\textbf{item}\hspace*{6pt}{xml:lang}="{en}">}bounds, frisks (cf. {<\textbf{cit}>}\mbox{}\newline 
\hspace*{6pt}\hspace*{6pt}\hspace*{6pt}{<\textbf{ref}>}Chaucer, K.T.644{</\textbf{ref}>}\mbox{}\newline 
\hspace*{6pt}\hspace*{6pt}\hspace*{6pt}{<\textbf{quote}>}a courser, {<\textbf{term}>}sterting{</\textbf{term}>}as the fyr{</\textbf{quote}>}\mbox{}\newline 
\hspace*{6pt}\hspace*{6pt}{</\textbf{cit}>}\mbox{}\newline 
\hspace*{6pt}{</\textbf{item}>}\mbox{}\newline 
\hspace*{6pt}{<\textbf{label}>}verteth{</\textbf{label}>}\mbox{}\newline 
\hspace*{6pt}{<\textbf{item}\hspace*{6pt}{xml:lang}="{la}">}pedit{</\textbf{item}>}\mbox{}\newline 
\hspace*{6pt}{<\textbf{label}>}murie{</\textbf{label}>}\mbox{}\newline 
\hspace*{6pt}{<\textbf{item}\hspace*{6pt}{xml:lang}="{en}">}merrily{</\textbf{item}>}\mbox{}\newline 
\hspace*{6pt}{<\textbf{label}>}swik{</\textbf{label}>}\mbox{}\newline 
\hspace*{6pt}{<\textbf{item}\hspace*{6pt}{xml:lang}="{en}">}cease{</\textbf{item}>}\mbox{}\newline 
\hspace*{6pt}{<\textbf{label}>}naver{</\textbf{label}>}\mbox{}\newline 
\hspace*{6pt}{<\textbf{item}\hspace*{6pt}{xml:lang}="{en}">}never{</\textbf{item}>}\mbox{}\newline 
{</\textbf{list}>}\end{shaded}\egroup 


    \item[{Example}]
  Labels may also be used to record explicitly the numbers or letters which mark list items in ordered lists, as in this extract from Gibbon's \textit{Autobiography}. In this usage the <label> element is synonymous with the {\itshape n} attribute on the <item> element:\leavevmode\bgroup\exampleFont \begin{shaded}\noindent\mbox{}I will add two facts, which have seldom occurred\mbox{}\newline 
 in the composition of six, or at least of five quartos. {<\textbf{list}\hspace*{6pt}{rend}="{runon}"\hspace*{6pt}{type}="{ordered}">}\mbox{}\newline 
\hspace*{6pt}{<\textbf{label}>}(1){</\textbf{label}>}\mbox{}\newline 
\hspace*{6pt}{<\textbf{item}>}My first rough manuscript, without any intermediate copy, has been sent to the press.{</\textbf{item}>}\mbox{}\newline 
\hspace*{6pt}{<\textbf{label}>}(2) {</\textbf{label}>}\mbox{}\newline 
\hspace*{6pt}{<\textbf{item}>}Not a sheet has been seen by any human eyes, excepting those of the author and the\mbox{}\newline 
\hspace*{6pt}\hspace*{6pt} printer: the faults and the merits are exclusively my own.{</\textbf{item}>}\mbox{}\newline 
{</\textbf{list}>}\end{shaded}\egroup 


    \item[{Example}]
  Labels may also be used for other structured list items, as in this extract from the journal of Edward Gibbon:\leavevmode\bgroup\exampleFont \begin{shaded}\noindent\mbox{}{<\textbf{list}\hspace*{6pt}{type}="{gloss}">}\mbox{}\newline 
\hspace*{6pt}{<\textbf{label}>}March 1757.{</\textbf{label}>}\mbox{}\newline 
\hspace*{6pt}{<\textbf{item}>}I wrote some critical observations upon Plautus.{</\textbf{item}>}\mbox{}\newline 
\hspace*{6pt}{<\textbf{label}>}March 8th.{</\textbf{label}>}\mbox{}\newline 
\hspace*{6pt}{<\textbf{item}>}I wrote a long dissertation upon some lines of Virgil.{</\textbf{item}>}\mbox{}\newline 
\hspace*{6pt}{<\textbf{label}>}June.{</\textbf{label}>}\mbox{}\newline 
\hspace*{6pt}{<\textbf{item}>}I saw Mademoiselle Curchod — {<\textbf{quote}\hspace*{6pt}{xml:lang}="{la}">}Omnia vincit amor, et nos cedamus\mbox{}\newline 
\hspace*{6pt}\hspace*{6pt}\hspace*{6pt}\hspace*{6pt} amori.{</\textbf{quote}>}\mbox{}\newline 
\hspace*{6pt}{</\textbf{item}>}\mbox{}\newline 
\hspace*{6pt}{<\textbf{label}>}August.{</\textbf{label}>}\mbox{}\newline 
\hspace*{6pt}{<\textbf{item}>}I went to Crassy, and staid two days.{</\textbf{item}>}\mbox{}\newline 
{</\textbf{list}>}\end{shaded}\egroup 

Note that the <label> might also appear within the <item> rather than as its sibling. Though syntactically valid, this usage is not recommended TEI practice.
    \item[{Example}]
  Labels may also be used to represent a label or heading attached to a paragraph or sequence of paragraphs not treated as a structural division, or to a group of verse lines. Note that, in this case, the <label> element appears \textit{within} the <p> or <lg> element, rather than as a preceding sibling of it.\leavevmode\bgroup\exampleFont \begin{shaded}\noindent\mbox{}{<\textbf{p}>}[...]\mbox{}\newline 
{<\textbf{lb}/>}\& n’entrer en mauuais \& mal-heu-\mbox{}\newline 
{<\textbf{lb}/>}ré meſnage. Or des que le conſente-\mbox{}\newline 
{<\textbf{lb}/>}ment des parties y eſt le mariage eſt\mbox{}\newline 
{<\textbf{lb}/>} arreſté, quoy que de faict il ne ſoit\mbox{}\newline 
{<\textbf{label}\hspace*{6pt}{place}="{margin}">}Puiſſance maritale\mbox{}\newline 
\hspace*{6pt}\hspace*{6pt} entre les Romains.{</\textbf{label}>}\mbox{}\newline 
\hspace*{6pt}{<\textbf{lb}/>} conſommé. Depuis la conſomma-\mbox{}\newline 
{<\textbf{lb}/>}tion du mariage la femme eſt ſoubs\mbox{}\newline 
{<\textbf{lb}/>} la puiſſance du mary, s’il n’eſt eſcla-\mbox{}\newline 
{<\textbf{lb}/>}ue ou enfant de famille : car en ce\mbox{}\newline 
{<\textbf{lb}/>} cas, la femme, qui a eſpouſé vn en-\mbox{}\newline 
{<\textbf{lb}/>}fant de famille, eſt ſous la puiſſance\mbox{}\newline 
 [...]{</\textbf{p}>}\end{shaded}\egroup 

In this example the text of the label appears in the right hand margin of the original source, next to the paragraph it describes, but approximately in the middle of it. If so desired the {\itshape type} attribute may be used to distinguish different categories of label.
    \item[{Content model}]
  \mbox{}\hfill\\[-10pt]\begin{Verbatim}[fontsize=\small]
<content>
 <macroRef key="macro.phraseSeq"/>
</content>
    
\end{Verbatim}

    \item[{Schema Declaration}]
  \mbox{}\hfill\\[-10pt]\begin{Verbatim}[fontsize=\small]
element label
{
   att.global.attributes,
   att.typed.attributes,
   att.placement.attributes,
   att.written.attributes,
   macro.phraseSeq}
\end{Verbatim}

\end{reflist}  \index{lacunaEnd=<lacunaEnd>|oddindex}
\begin{reflist}
\item[]\begin{specHead}{TEI.lacunaEnd}{<lacunaEnd> }indicates the end of a lacuna in a mostly complete textual witness. [\xref{http://www.tei-c.org/release/doc/tei-p5-doc/en/html/TC.html\#TCAPMI}{12.1.5. Fragmentary Witnesses}]\end{specHead} 
    \item[{Module}]
  textcrit
    \item[{Attributes}]
  Attributes att.global (\textit{@xml:id}, \textit{@n}, \textit{@xml:lang}, \textit{@xml:base}, \textit{@xml:space})  (att.global.rendition (\textit{@rend}, \textit{@style}, \textit{@rendition})) (att.global.linking (\textit{@corresp}, \textit{@synch}, \textit{@sameAs}, \textit{@copyOf}, \textit{@next}, \textit{@prev}, \textit{@exclude}, \textit{@select})) (att.global.analytic (\textit{@ana})) (att.global.facs (\textit{@facs})) (att.global.change (\textit{@change})) (att.global.responsibility (\textit{@cert}, \textit{@resp})) (att.global.source (\textit{@source})) att.rdgPart (\textit{@wit}) 
    \item[{Member of}]
  model.rdgPart
    \item[{Contained by}]
  
    \item[textcrit: ]
   lem rdg
    \item[{May contain}]
  Empty element
    \item[{Example}]
  \leavevmode\bgroup\exampleFont \begin{shaded}\noindent\mbox{}{<\textbf{rdg}\hspace*{6pt}{wit}="{\#X}">}\mbox{}\newline 
\hspace*{6pt}{<\textbf{lacunaEnd}/>}auctorite\mbox{}\newline 
{</\textbf{rdg}>}\end{shaded}\egroup 


    \item[{Content model}]
  \fbox{\ttfamily <content>\newline
</content>\newline
    } 
    \item[{Schema Declaration}]
  \mbox{}\hfill\\[-10pt]\begin{Verbatim}[fontsize=\small]
element lacunaEnd { att.global.attributes, att.rdgPart.attributes, empty }
\end{Verbatim}

\end{reflist}  \index{lacunaStart=<lacunaStart>|oddindex}
\begin{reflist}
\item[]\begin{specHead}{TEI.lacunaStart}{<lacunaStart> }indicates the beginning of a lacuna in the text of a mostly complete textual witness. [\xref{http://www.tei-c.org/release/doc/tei-p5-doc/en/html/TC.html\#TCAPMI}{12.1.5. Fragmentary Witnesses}]\end{specHead} 
    \item[{Module}]
  textcrit
    \item[{Attributes}]
  Attributes att.global (\textit{@xml:id}, \textit{@n}, \textit{@xml:lang}, \textit{@xml:base}, \textit{@xml:space})  (att.global.rendition (\textit{@rend}, \textit{@style}, \textit{@rendition})) (att.global.linking (\textit{@corresp}, \textit{@synch}, \textit{@sameAs}, \textit{@copyOf}, \textit{@next}, \textit{@prev}, \textit{@exclude}, \textit{@select})) (att.global.analytic (\textit{@ana})) (att.global.facs (\textit{@facs})) (att.global.change (\textit{@change})) (att.global.responsibility (\textit{@cert}, \textit{@resp})) (att.global.source (\textit{@source})) att.rdgPart (\textit{@wit}) 
    \item[{Member of}]
  model.rdgPart
    \item[{Contained by}]
  
    \item[textcrit: ]
   lem rdg
    \item[{May contain}]
  Empty element
    \item[{Example}]
  \leavevmode\bgroup\exampleFont \begin{shaded}\noindent\mbox{}{<\textbf{app}>}\mbox{}\newline 
\hspace*{6pt}{<\textbf{lem}\hspace*{6pt}{wit}="{\#El \#Hg}">}Experience{</\textbf{lem}>}\mbox{}\newline 
\hspace*{6pt}{<\textbf{rdg}\hspace*{6pt}{wit}="{\#Ha4}">}Ex{<\textbf{g}\hspace*{6pt}{ref}="{\#per}"/>}\mbox{}\newline 
\hspace*{6pt}\hspace*{6pt}{<\textbf{lacunaStart}/>}\mbox{}\newline 
\hspace*{6pt}{</\textbf{rdg}>}\mbox{}\newline 
{</\textbf{app}>}\end{shaded}\egroup 


    \item[{Content model}]
  \fbox{\ttfamily <content>\newline
</content>\newline
    } 
    \item[{Schema Declaration}]
  \mbox{}\hfill\\[-10pt]\begin{Verbatim}[fontsize=\small]
element lacunaStart { att.global.attributes, att.rdgPart.attributes, empty }
\end{Verbatim}

\end{reflist}  \index{langKnowledge=<langKnowledge>|oddindex}\index{tags=@tags!<langKnowledge>|oddindex}
\begin{reflist}
\item[]\begin{specHead}{TEI.langKnowledge}{<langKnowledge> }(language knowledge) summarizes the state of a person's linguistic knowledge, either as prose or by a list of <langKnown> elements. [\xref{http://www.tei-c.org/release/doc/tei-p5-doc/en/html/ND.html\#NDPERSEpc}{13.3.2.1. Personal Characteristics}]\end{specHead} 
    \item[{Module}]
  namesdates
    \item[{Attributes}]
  Attributes att.global (\textit{@xml:id}, \textit{@n}, \textit{@xml:lang}, \textit{@xml:base}, \textit{@xml:space})  (att.global.rendition (\textit{@rend}, \textit{@style}, \textit{@rendition})) (att.global.linking (\textit{@corresp}, \textit{@synch}, \textit{@sameAs}, \textit{@copyOf}, \textit{@next}, \textit{@prev}, \textit{@exclude}, \textit{@select})) (att.global.analytic (\textit{@ana})) (att.global.facs (\textit{@facs})) (att.global.change (\textit{@change})) (att.global.responsibility (\textit{@cert}, \textit{@resp})) (att.global.source (\textit{@source})) att.datable (\textit{@calendar}, \textit{@period})  (att.datable.w3c (\textit{@when}, \textit{@notBefore}, \textit{@notAfter}, \textit{@from}, \textit{@to})) (att.datable.iso (\textit{@when-iso}, \textit{@notBefore-iso}, \textit{@notAfter-iso}, \textit{@from-iso}, \textit{@to-iso})) (att.datable.custom (\textit{@when-custom}, \textit{@notBefore-custom}, \textit{@notAfter-custom}, \textit{@from-custom}, \textit{@to-custom}, \textit{@datingPoint}, \textit{@datingMethod})) att.editLike (\textit{@evidence}, \textit{@instant})  (att.dimensions (\textit{@unit}, \textit{@quantity}, \textit{@extent}, \textit{@precision}, \textit{@scope}) (att.ranging (\textit{@atLeast}, \textit{@atMost}, \textit{@min}, \textit{@max}, \textit{@confidence})) ) \hfil\\[-10pt]\begin{sansreflist}
    \item[@tags]
  supplies one or more valid language tags for the languages specified
\begin{reflist}
    \item[{Status}]
  Optional
    \item[{Datatype}]
  1–∞ occurrences of teidata.language separated by whitespace
    \item[{Note}]
  \par
This attribute should be supplied only if the element contains no <langKnown> children. Its values are language ‘tags’ as defined in \xref{http://www.rfc-editor.org/rfc/rfc4646.txt}{RFC 4646} or its successor
\end{reflist}  
\end{sansreflist}  
    \item[{Member of}]
  model.persStateLike
    \item[{Contained by}]
  
    \item[namesdates: ]
   person personGrp
    \item[{May contain}]
  
    \item[core: ]
   p\par 
    \item[linking: ]
   ab\par 
    \item[namesdates: ]
   langKnown
    \item[{Example}]
  \leavevmode\bgroup\exampleFont \begin{shaded}\noindent\mbox{}{<\textbf{langKnowledge}\hspace*{6pt}{tags}="{en-GB fr}">}\mbox{}\newline 
\hspace*{6pt}{<\textbf{p}>}British English and French{</\textbf{p}>}\mbox{}\newline 
{</\textbf{langKnowledge}>}\end{shaded}\egroup 


    \item[{Example}]
  \leavevmode\bgroup\exampleFont \begin{shaded}\noindent\mbox{}{<\textbf{langKnowledge}>}\mbox{}\newline 
\hspace*{6pt}{<\textbf{langKnown}\hspace*{6pt}{level}="{H}"\hspace*{6pt}{tag}="{en-GB}">}British English{</\textbf{langKnown}>}\mbox{}\newline 
\hspace*{6pt}{<\textbf{langKnown}\hspace*{6pt}{level}="{M}"\hspace*{6pt}{tag}="{fr}">}French{</\textbf{langKnown}>}\mbox{}\newline 
{</\textbf{langKnowledge}>}\end{shaded}\egroup 


    \item[{Content model}]
  \mbox{}\hfill\\[-10pt]\begin{Verbatim}[fontsize=\small]
<content>
 <sequence>
  <elementRef key="precision"
   maxOccurs="unbounded" minOccurs="0"/>
  <alternate>
   <classRef key="model.pLike"/>
   <elementRef key="langKnown"
    maxOccurs="unbounded" minOccurs="1"/>
  </alternate>
 </sequence>
</content>
    
\end{Verbatim}

    \item[{Schema Declaration}]
  \mbox{}\hfill\\[-10pt]\begin{Verbatim}[fontsize=\small]
element langKnowledge
{
   att.global.attributes,
   att.datable.attributes,
   att.editLike.attributes,
   attribute tags { list { + } }?,
   ( precision*, ( model.pLike | langKnown+ ) )
}
\end{Verbatim}

\end{reflist}  \index{langKnown=<langKnown>|oddindex}\index{tag=@tag!<langKnown>|oddindex}\index{level=@level!<langKnown>|oddindex}
\begin{reflist}
\item[]\begin{specHead}{TEI.langKnown}{<langKnown> }(language known) summarizes the state of a person's linguistic competence, i.e., knowledge of a single language. [\xref{http://www.tei-c.org/release/doc/tei-p5-doc/en/html/CC.html\#CCAHPA}{15.2.2. The Participant Description}]\end{specHead} 
    \item[{Module}]
  namesdates
    \item[{Attributes}]
  Attributes att.global (\textit{@xml:id}, \textit{@n}, \textit{@xml:lang}, \textit{@xml:base}, \textit{@xml:space})  (att.global.rendition (\textit{@rend}, \textit{@style}, \textit{@rendition})) (att.global.linking (\textit{@corresp}, \textit{@synch}, \textit{@sameAs}, \textit{@copyOf}, \textit{@next}, \textit{@prev}, \textit{@exclude}, \textit{@select})) (att.global.analytic (\textit{@ana})) (att.global.facs (\textit{@facs})) (att.global.change (\textit{@change})) (att.global.responsibility (\textit{@cert}, \textit{@resp})) (att.global.source (\textit{@source})) att.datable (\textit{@calendar}, \textit{@period})  (att.datable.w3c (\textit{@when}, \textit{@notBefore}, \textit{@notAfter}, \textit{@from}, \textit{@to})) (att.datable.iso (\textit{@when-iso}, \textit{@notBefore-iso}, \textit{@notAfter-iso}, \textit{@from-iso}, \textit{@to-iso})) (att.datable.custom (\textit{@when-custom}, \textit{@notBefore-custom}, \textit{@notAfter-custom}, \textit{@from-custom}, \textit{@to-custom}, \textit{@datingPoint}, \textit{@datingMethod})) att.editLike (\textit{@evidence}, \textit{@instant})  (att.dimensions (\textit{@unit}, \textit{@quantity}, \textit{@extent}, \textit{@precision}, \textit{@scope}) (att.ranging (\textit{@atLeast}, \textit{@atMost}, \textit{@min}, \textit{@max}, \textit{@confidence})) ) \hfil\\[-10pt]\begin{sansreflist}
    \item[@tag]
  supplies a valid language tag for the language concerned.
\begin{reflist}
    \item[{Status}]
  Required
    \item[{Datatype}]
  teidata.language
    \item[{Note}]
  \par
The value for this attribute should be a language ‘tag’ as defined in \xref{https://tools.ietf.org/html/bcp47}{BCP 47}.
\end{reflist}  
    \item[@level]
  a code indicating the person's level of knowledge for this language
\begin{reflist}
    \item[{Status}]
  Optional
    \item[{Datatype}]
  teidata.word
\end{reflist}  
\end{sansreflist}  
    \item[{Contained by}]
  
    \item[namesdates: ]
   langKnowledge
    \item[{May contain}]
  
    \item[analysis: ]
   interp interpGrp span spanGrp\par 
    \item[core: ]
   abbr address cb choice date distinct email emph expan foreign gap gb gloss hi index lb measure measureGrp mentioned milestone name note num pb ptr ref rs soCalled term time title\par 
    \item[figures: ]
   figure notatedMusic\par 
    \item[header: ]
   idno\par 
    \item[linking: ]
   alt altGrp anchor join joinGrp link linkGrp timeline\par 
    \item[msdescription: ]
   catchwords depth dim dimensions height heraldry locus locusGrp material objectType origDate origPlace secFol signatures stamp watermark width\par 
    \item[namesdates: ]
   addName affiliation bloc climate country district forename genName geo geogFeat geogName location nameLink offset orgName persName placeName population region roleName settlement state surname terrain trait\par 
    \item[textcrit: ]
   app witDetail\par 
    \item[transcr: ]
   addSpan am damageSpan delSpan ex fw listTranspose metamark space subst substJoin\par character data
    \item[{Example}]
  \leavevmode\bgroup\exampleFont \begin{shaded}\noindent\mbox{}{<\textbf{langKnown}\hspace*{6pt}{level}="{H}"\hspace*{6pt}{tag}="{en-GB}">}British English{</\textbf{langKnown}>}\mbox{}\newline 
{<\textbf{langKnown}\hspace*{6pt}{level}="{M}"\hspace*{6pt}{tag}="{fr}">}French{</\textbf{langKnown}>}\end{shaded}\egroup 


    \item[{Content model}]
  \mbox{}\hfill\\[-10pt]\begin{Verbatim}[fontsize=\small]
<content>
 <macroRef key="macro.phraseSeq.limited"/>
</content>
    
\end{Verbatim}

    \item[{Schema Declaration}]
  \mbox{}\hfill\\[-10pt]\begin{Verbatim}[fontsize=\small]
element langKnown
{
   att.global.attributes,
   att.datable.attributes,
   att.editLike.attributes,
   attribute tag { text },
   attribute level { text }?,
   macro.phraseSeq.limited}
\end{Verbatim}

\end{reflist}  \index{langUsage=<langUsage>|oddindex}
\begin{reflist}
\item[]\begin{specHead}{TEI.langUsage}{<langUsage> }(language usage) describes the languages, sublanguages, registers, dialects, etc. represented within a text. [\xref{http://www.tei-c.org/release/doc/tei-p5-doc/en/html/HD.html\#HD41}{2.4.2. Language Usage} \xref{http://www.tei-c.org/release/doc/tei-p5-doc/en/html/HD.html\#HD4}{2.4. The Profile Description} \xref{http://www.tei-c.org/release/doc/tei-p5-doc/en/html/CC.html\#CCAS2}{15.3.2. Declarable Elements}]\end{specHead} 
    \item[{Module}]
  header
    \item[{Attributes}]
  Attributes att.global (\textit{@xml:id}, \textit{@n}, \textit{@xml:lang}, \textit{@xml:base}, \textit{@xml:space})  (att.global.rendition (\textit{@rend}, \textit{@style}, \textit{@rendition})) (att.global.linking (\textit{@corresp}, \textit{@synch}, \textit{@sameAs}, \textit{@copyOf}, \textit{@next}, \textit{@prev}, \textit{@exclude}, \textit{@select})) (att.global.analytic (\textit{@ana})) (att.global.facs (\textit{@facs})) (att.global.change (\textit{@change})) (att.global.responsibility (\textit{@cert}, \textit{@resp})) (att.global.source (\textit{@source})) att.declarable (\textit{@default}) 
    \item[{Member of}]
  model.profileDescPart
    \item[{Contained by}]
  
    \item[header: ]
   profileDesc
    \item[{May contain}]
  
    \item[core: ]
   p\par 
    \item[header: ]
   language\par 
    \item[linking: ]
   ab
    \item[{Example}]
  \leavevmode\bgroup\exampleFont \begin{shaded}\noindent\mbox{}{<\textbf{langUsage}>}\mbox{}\newline 
\hspace*{6pt}{<\textbf{language}\hspace*{6pt}{ident}="{fr-CA}"\hspace*{6pt}{usage}="{60}">}Québecois{</\textbf{language}>}\mbox{}\newline 
\hspace*{6pt}{<\textbf{language}\hspace*{6pt}{ident}="{en-CA}"\hspace*{6pt}{usage}="{20}">}Canadian business English{</\textbf{language}>}\mbox{}\newline 
\hspace*{6pt}{<\textbf{language}\hspace*{6pt}{ident}="{en-GB}"\hspace*{6pt}{usage}="{20}">}British English{</\textbf{language}>}\mbox{}\newline 
{</\textbf{langUsage}>}\end{shaded}\egroup 


    \item[{Content model}]
  \mbox{}\hfill\\[-10pt]\begin{Verbatim}[fontsize=\small]
<content>
 <alternate>
  <classRef key="model.pLike"
   maxOccurs="unbounded" minOccurs="1"/>
  <elementRef key="language"
   maxOccurs="unbounded" minOccurs="1"/>
 </alternate>
</content>
    
\end{Verbatim}

    \item[{Schema Declaration}]
  \mbox{}\hfill\\[-10pt]\begin{Verbatim}[fontsize=\small]
element langUsage
{
   att.global.attributes,
   att.declarable.attributes,
   ( model.pLike+ | language+ )
}
\end{Verbatim}

\end{reflist}  \index{language=<language>|oddindex}\index{ident=@ident!<language>|oddindex}\index{usage=@usage!<language>|oddindex}
\begin{reflist}
\item[]\begin{specHead}{TEI.language}{<language> }characterizes a single language or sublanguage used within a text. [\xref{http://www.tei-c.org/release/doc/tei-p5-doc/en/html/HD.html\#HD41}{2.4.2. Language Usage}]\end{specHead} 
    \item[{Module}]
  header
    \item[{Attributes}]
  Attributes att.global (\textit{@xml:id}, \textit{@n}, \textit{@xml:lang}, \textit{@xml:base}, \textit{@xml:space})  (att.global.rendition (\textit{@rend}, \textit{@style}, \textit{@rendition})) (att.global.linking (\textit{@corresp}, \textit{@synch}, \textit{@sameAs}, \textit{@copyOf}, \textit{@next}, \textit{@prev}, \textit{@exclude}, \textit{@select})) (att.global.analytic (\textit{@ana})) (att.global.facs (\textit{@facs})) (att.global.change (\textit{@change})) (att.global.responsibility (\textit{@cert}, \textit{@resp})) (att.global.source (\textit{@source})) \hfil\\[-10pt]\begin{sansreflist}
    \item[@ident]
  (identifier) Supplies a language code constructed as defined in \xref{https://tools.ietf.org/html/bcp47}{BCP 47} which is used to identify the language documented by this element, and which is referenced by the global {\itshape xml:lang} attribute.
\begin{reflist}
    \item[{Status}]
  Required
    \item[{Datatype}]
  teidata.language
\end{reflist}  
    \item[@usage]
  specifies the approximate percentage (by volume) of the text which uses this language.
\begin{reflist}
    \item[{Status}]
  Optional
    \item[{Datatype}]
  
\end{reflist}  
\end{sansreflist}  
    \item[{Contained by}]
  
    \item[header: ]
   langUsage
    \item[{May contain}]
  
    \item[analysis: ]
   interp interpGrp span spanGrp\par 
    \item[core: ]
   abbr address cb choice date distinct email emph expan foreign gap gb gloss hi index lb measure measureGrp mentioned milestone name note num pb ptr ref rs soCalled term time title\par 
    \item[figures: ]
   figure notatedMusic\par 
    \item[header: ]
   idno\par 
    \item[linking: ]
   alt altGrp anchor join joinGrp link linkGrp timeline\par 
    \item[msdescription: ]
   catchwords depth dim dimensions height heraldry locus locusGrp material objectType origDate origPlace secFol signatures stamp watermark width\par 
    \item[namesdates: ]
   addName affiliation bloc climate country district forename genName geo geogFeat geogName location nameLink offset orgName persName placeName population region roleName settlement state surname terrain trait\par 
    \item[textcrit: ]
   app witDetail\par 
    \item[transcr: ]
   addSpan am damageSpan delSpan ex fw listTranspose metamark space subst substJoin\par character data
    \item[{Note}]
  \par
Particularly for sublanguages, an informal prose characterization should be supplied as content for the element.
    \item[{Example}]
  \leavevmode\bgroup\exampleFont \begin{shaded}\noindent\mbox{}{<\textbf{langUsage}>}\mbox{}\newline 
\hspace*{6pt}{<\textbf{language}\hspace*{6pt}{ident}="{en-US}"\hspace*{6pt}{usage}="{75}">}modern American English{</\textbf{language}>}\mbox{}\newline 
\hspace*{6pt}{<\textbf{language}\hspace*{6pt}{ident}="{i-az-Arab}"\hspace*{6pt}{usage}="{20}">}Azerbaijani in Arabic script{</\textbf{language}>}\mbox{}\newline 
\hspace*{6pt}{<\textbf{language}\hspace*{6pt}{ident}="{x-lap}"\hspace*{6pt}{usage}="{05}">}Pig Latin{</\textbf{language}>}\mbox{}\newline 
{</\textbf{langUsage}>}\end{shaded}\egroup 


    \item[{Content model}]
  \mbox{}\hfill\\[-10pt]\begin{Verbatim}[fontsize=\small]
<content>
 <macroRef key="macro.phraseSeq.limited"/>
</content>
    
\end{Verbatim}

    \item[{Schema Declaration}]
  \mbox{}\hfill\\[-10pt]\begin{Verbatim}[fontsize=\small]
element language
{
   att.global.attributes,
   attribute ident { text },
   attribute usage { text }?,
   macro.phraseSeq.limited}
\end{Verbatim}

\end{reflist}  \index{layout=<layout>|oddindex}\index{columns=@columns!<layout>|oddindex}\index{ruledLines=@ruledLines!<layout>|oddindex}\index{writtenLines=@writtenLines!<layout>|oddindex}
\begin{reflist}
\item[]\begin{specHead}{TEI.layout}{<layout> }describes how text is laid out on the page, including information about any ruling, pricking, or other evidence of page-preparation techniques. [\xref{http://www.tei-c.org/release/doc/tei-p5-doc/en/html/MS.html\#msph2}{10.7.2. Writing, Decoration, and Other Notations}]\end{specHead} 
    \item[{Module}]
  msdescription
    \item[{Attributes}]
  Attributes att.global (\textit{@xml:id}, \textit{@n}, \textit{@xml:lang}, \textit{@xml:base}, \textit{@xml:space})  (att.global.rendition (\textit{@rend}, \textit{@style}, \textit{@rendition})) (att.global.linking (\textit{@corresp}, \textit{@synch}, \textit{@sameAs}, \textit{@copyOf}, \textit{@next}, \textit{@prev}, \textit{@exclude}, \textit{@select})) (att.global.analytic (\textit{@ana})) (att.global.facs (\textit{@facs})) (att.global.change (\textit{@change})) (att.global.responsibility (\textit{@cert}, \textit{@resp})) (att.global.source (\textit{@source})) \hfil\\[-10pt]\begin{sansreflist}
    \item[@columns]
  specifies the number of columns per page
\begin{reflist}
    \item[{Status}]
  Optional
    \item[{Datatype}]
  1–2 occurrences of teidata.count separated by whitespace
    \item[{Note}]
  \par
If a single number is given, all pages have this number of columns. If two numbers are given, the number of columns per page varies between the values supplied.
\end{reflist}  
    \item[@ruledLines]
  specifies the number of ruled lines per column
\begin{reflist}
    \item[{Status}]
  Optional
    \item[{Datatype}]
  1–2 occurrences of teidata.count separated by whitespace
    \item[{Note}]
  \par
If a single number is given, all columns have this number of ruled lines. If two numbers are given, the number of ruled lines per column varies between the values supplied.
\end{reflist}  
    \item[@writtenLines]
  specifies the number of written lines per column
\begin{reflist}
    \item[{Status}]
  Optional
    \item[{Datatype}]
  1–2 occurrences of teidata.count separated by whitespace
    \item[{Note}]
  \par
If a single number is given, all columns have this number of written lines. If two numbers are given, the number of written lines per column varies between the values supplied.
\end{reflist}  
\end{sansreflist}  
    \item[{Contained by}]
  
    \item[msdescription: ]
   layoutDesc
    \item[{May contain}]
  
    \item[analysis: ]
   c cl interp interpGrp m pc phr s span spanGrp w\par 
    \item[core: ]
   abbr add address bibl biblStruct cb choice cit corr date del desc distinct email emph expan foreign gap gb gloss graphic hi index l label lb lg list listBibl measure measureGrp media mentioned milestone name note num orig p pb ptr q quote ref reg rs said sic soCalled sp stage term time title unclear\par 
    \item[figures: ]
   figure formula notatedMusic table\par 
    \item[gaiji: ]
   g\par 
    \item[header: ]
   biblFull idno\par 
    \item[linking: ]
   ab alt altGrp anchor join joinGrp link linkGrp seg timeline\par 
    \item[msdescription: ]
   catchwords depth dim dimensions height heraldry locus locusGrp material msDesc objectType origDate origPlace secFol signatures stamp watermark width\par 
    \item[namesdates: ]
   addName affiliation bloc climate country district forename genName geo geogFeat geogName listEvent listNym listOrg listPerson listPlace location nameLink offset orgName persName placeName population region roleName settlement state surname terrain trait\par 
    \item[textcrit: ]
   app listApp listWit witDetail\par 
    \item[textstructure: ]
   floatingText\par 
    \item[transcr: ]
   addSpan am damage damageSpan delSpan ex fw handShift listTranspose metamark mod redo restore retrace secl space subst substJoin supplied surplus undo\par character data
    \item[{Example}]
  \leavevmode\bgroup\exampleFont \begin{shaded}\noindent\mbox{}{<\textbf{layout}\hspace*{6pt}{columns}="{1}"\hspace*{6pt}{ruledLines}="{25 32}">}Most pages have between 25 and 32 long lines ruled in lead.{</\textbf{layout}>}\end{shaded}\egroup 


    \item[{Example}]
  \leavevmode\bgroup\exampleFont \begin{shaded}\noindent\mbox{}{<\textbf{layout}\hspace*{6pt}{columns}="{2}"\hspace*{6pt}{ruledLines}="{42}">}\mbox{}\newline 
\hspace*{6pt}{<\textbf{p}>}2 columns of 42 lines ruled in ink, with central rule\mbox{}\newline 
\hspace*{6pt}\hspace*{6pt} between the columns.{</\textbf{p}>}\mbox{}\newline 
{</\textbf{layout}>}\end{shaded}\egroup 


    \item[{Example}]
  \leavevmode\bgroup\exampleFont \begin{shaded}\noindent\mbox{}{<\textbf{layout}\hspace*{6pt}{columns}="{1 2}"\hspace*{6pt}{writtenLines}="{40 50}">}\mbox{}\newline 
\hspace*{6pt}{<\textbf{p}>}Some pages have 2 columns, with central rule\mbox{}\newline 
\hspace*{6pt}\hspace*{6pt} between the columns; each column with between 40 and 50 lines of writing.{</\textbf{p}>}\mbox{}\newline 
{</\textbf{layout}>}\end{shaded}\egroup 


    \item[{Content model}]
  \mbox{}\hfill\\[-10pt]\begin{Verbatim}[fontsize=\small]
<content>
 <macroRef key="macro.specialPara"/>
</content>
    
\end{Verbatim}

    \item[{Schema Declaration}]
  \mbox{}\hfill\\[-10pt]\begin{Verbatim}[fontsize=\small]
element layout
{
   att.global.attributes,
   attribute columns { list { ? } }?,
   attribute ruledLines { list { ? } }?,
   attribute writtenLines { list { ? } }?,
   macro.specialPara}
\end{Verbatim}

\end{reflist}  \index{layoutDesc=<layoutDesc>|oddindex}
\begin{reflist}
\item[]\begin{specHead}{TEI.layoutDesc}{<layoutDesc> }(layout description) collects the set of layout descriptions applicable to a manuscript. [\xref{http://www.tei-c.org/release/doc/tei-p5-doc/en/html/MS.html\#msph2}{10.7.2. Writing, Decoration, and Other Notations}]\end{specHead} 
    \item[{Module}]
  msdescription
    \item[{Attributes}]
  Attributes att.global (\textit{@xml:id}, \textit{@n}, \textit{@xml:lang}, \textit{@xml:base}, \textit{@xml:space})  (att.global.rendition (\textit{@rend}, \textit{@style}, \textit{@rendition})) (att.global.linking (\textit{@corresp}, \textit{@synch}, \textit{@sameAs}, \textit{@copyOf}, \textit{@next}, \textit{@prev}, \textit{@exclude}, \textit{@select})) (att.global.analytic (\textit{@ana})) (att.global.facs (\textit{@facs})) (att.global.change (\textit{@change})) (att.global.responsibility (\textit{@cert}, \textit{@resp})) (att.global.source (\textit{@source}))
    \item[{Contained by}]
  
    \item[msdescription: ]
   objectDesc
    \item[{May contain}]
  
    \item[core: ]
   p\par 
    \item[linking: ]
   ab\par 
    \item[msdescription: ]
   layout summary
    \item[{Example}]
  \leavevmode\bgroup\exampleFont \begin{shaded}\noindent\mbox{}{<\textbf{layoutDesc}>}\mbox{}\newline 
\hspace*{6pt}{<\textbf{p}>}Most pages have between 25 and 32 long lines ruled in lead.{</\textbf{p}>}\mbox{}\newline 
{</\textbf{layoutDesc}>}\end{shaded}\egroup 


    \item[{Example}]
  \leavevmode\bgroup\exampleFont \begin{shaded}\noindent\mbox{}{<\textbf{layoutDesc}>}\mbox{}\newline 
\hspace*{6pt}{<\textbf{layout}\hspace*{6pt}{columns}="{2}"\hspace*{6pt}{ruledLines}="{42}">}\mbox{}\newline 
\hspace*{6pt}\hspace*{6pt}{<\textbf{p}>}\mbox{}\newline 
\hspace*{6pt}\hspace*{6pt}\hspace*{6pt}{<\textbf{locus}\hspace*{6pt}{from}="{f12r}"\hspace*{6pt}{to}="{f15v}"/>}\mbox{}\newline 
\hspace*{6pt}\hspace*{6pt}\hspace*{6pt}\hspace*{6pt} 2 columns of 42 lines pricked and ruled in ink, with\mbox{}\newline 
\hspace*{6pt}\hspace*{6pt}\hspace*{6pt}\hspace*{6pt} central rule between the columns.{</\textbf{p}>}\mbox{}\newline 
\hspace*{6pt}{</\textbf{layout}>}\mbox{}\newline 
\hspace*{6pt}{<\textbf{layout}\hspace*{6pt}{columns}="{3}">}\mbox{}\newline 
\hspace*{6pt}\hspace*{6pt}{<\textbf{p}\hspace*{6pt}{xml:lang}="{zh-TW}">}\mbox{}\newline 
\hspace*{6pt}\hspace*{6pt}\hspace*{6pt}{<\textbf{locus}\hspace*{6pt}{from}="{f16}"/>}{\textChinese 小孔的三欄可見.}{</\textbf{p}>}\mbox{}\newline 
\hspace*{6pt}{</\textbf{layout}>}\mbox{}\newline 
{</\textbf{layoutDesc}>}\end{shaded}\egroup 


    \item[{Content model}]
  \mbox{}\hfill\\[-10pt]\begin{Verbatim}[fontsize=\small]
<content>
 <alternate>
  <classRef key="model.pLike"
   maxOccurs="unbounded" minOccurs="1"/>
  <sequence>
   <elementRef key="summary" minOccurs="0"/>
   <elementRef key="layout"
    maxOccurs="unbounded" minOccurs="1"/>
  </sequence>
 </alternate>
</content>
    
\end{Verbatim}

    \item[{Schema Declaration}]
  \mbox{}\hfill\\[-10pt]\begin{Verbatim}[fontsize=\small]
element layoutDesc
{
   att.global.attributes,
   ( model.pLike+ | ( summary?, layout+ ) )
}
\end{Verbatim}

\end{reflist}  \index{lb=<lb>|oddindex}
\begin{reflist}
\item[]\begin{specHead}{TEI.lb}{<lb> }(line break) marks the start of a new (typographic) line in some edition or version of a text. [\xref{http://www.tei-c.org/release/doc/tei-p5-doc/en/html/CO.html\#CORS5}{3.10.3. Milestone Elements} \xref{http://www.tei-c.org/release/doc/tei-p5-doc/en/html/DR.html\#DRPAL}{7.2.5. Speech Contents}]\end{specHead} 
    \item[{Module}]
  core
    \item[{Attributes}]
  Attributes att.global (\textit{@xml:id}, \textit{@n}, \textit{@xml:lang}, \textit{@xml:base}, \textit{@xml:space})  (att.global.rendition (\textit{@rend}, \textit{@style}, \textit{@rendition})) (att.global.linking (\textit{@corresp}, \textit{@synch}, \textit{@sameAs}, \textit{@copyOf}, \textit{@next}, \textit{@prev}, \textit{@exclude}, \textit{@select})) (att.global.analytic (\textit{@ana})) (att.global.facs (\textit{@facs})) (att.global.change (\textit{@change})) (att.global.responsibility (\textit{@cert}, \textit{@resp})) (att.global.source (\textit{@source})) att.typed (\textit{@type}, \textit{@subtype}) att.edition (\textit{@ed}, \textit{@edRef}) att.spanning (\textit{@spanTo}) att.breaking (\textit{@break}) 
    \item[{Member of}]
  model.milestoneLike
    \item[{Contained by}]
  
    \item[analysis: ]
   cl m phr s span w\par 
    \item[core: ]
   abbr add addrLine address author bibl biblScope cit citedRange corr date del distinct editor email emph expan foreign gloss head headItem headLabel hi imprint item l label lg list listBibl measure mentioned name note num orig p pubPlace publisher q quote ref reg resp rs said series sic soCalled sp speaker stage street term textLang time title unclear\par 
    \item[figures: ]
   cell figure table\par 
    \item[header: ]
   authority change classCode distributor edition extent funder geoDecl handNote language licence principal scriptNote sponsor typeNote\par 
    \item[linking: ]
   ab seg\par 
    \item[msdescription: ]
   accMat acquisition additions catchwords collation colophon condition custEvent decoNote explicit filiation finalRubric foliation heraldry incipit layout material msItem musicNotation objectType origDate origPlace origin provenance rubric secFol signatures source stamp summary support surrogates watermark\par 
    \item[namesdates: ]
   addName affiliation age birth bloc country death district education faith floruit forename genName geogFeat geogName langKnown nameLink nationality occupation offset org orgName persName person personGrp placeName region residence roleName settlement sex socecStatus surname\par 
    \item[textcrit: ]
   lem rdg wit witDetail\par 
    \item[textstructure: ]
   argument back body byline closer dateline div docAuthor docDate docEdition docImprint docTitle epigraph floatingText front group imprimatur opener postscript salute signed text titlePage titlePart trailer\par 
    \item[transcr: ]
   damage fw line metamark mod restore retrace secl sourceDoc subst supplied surface surfaceGrp surplus zone
    \item[{May contain}]
  Empty element
    \item[{Note}]
  \par
By convention, <lb> elements should appear at the point in the text where a new line starts. The {\itshape n} attribute, if used, indicates the number or other value associated with the text between this point and the next <lb> element, typically the sequence number of the line within the page, or other appropriate unit. This element is intended to be used for marking actual line breaks on a manuscript or printed page, at the point where they occur; it should not be used to tag structural units such as lines of verse (for which the <l> element is available) except in circumstances where structural units cannot otherwise be marked.\par
The {\itshape type} attribute may be used to characterize the line break in any respect. The more specialized attributes {\itshape break}, {\itshape ed}, or {\itshape edRef} should be preferred when the intent is to indicate whether or not the line break is word-breaking, or to note the source from which it derives.
    \item[{Example}]
  This example shows typographical line breaks within metrical lines, where they occur at different places in different editions:\leavevmode\bgroup\exampleFont \begin{shaded}\noindent\mbox{}{<\textbf{l}>}Of Mans First Disobedience,{<\textbf{lb}\hspace*{6pt}{ed}="{1674}"/>} and{<\textbf{lb}\hspace*{6pt}{ed}="{1667}"/>} the Fruit{</\textbf{l}>}\mbox{}\newline 
{<\textbf{l}>}Of that Forbidden Tree, whose{<\textbf{lb}\hspace*{6pt}{ed}="{1667 1674}"/>} mortal tast{</\textbf{l}>}\mbox{}\newline 
{<\textbf{l}>}Brought Death into the World,{<\textbf{lb}\hspace*{6pt}{ed}="{1667}"/>} and all{<\textbf{lb}\hspace*{6pt}{ed}="{1674}"/>} our woe,{</\textbf{l}>}\end{shaded}\egroup 


    \item[{Example}]
  This example encodes typographical line breaks as a means of preserving the visual appearance of a title page. The {\itshape break} attribute is used to show that the line break does not (as elsewhere) mark the start of a new word.\leavevmode\bgroup\exampleFont \begin{shaded}\noindent\mbox{}{<\textbf{titlePart}>}\mbox{}\newline 
\hspace*{6pt}{<\textbf{lb}/>}With Additions, ne-{<\textbf{lb}\hspace*{6pt}{break}="{no}"/>}ver before Printed.\mbox{}\newline 
{</\textbf{titlePart}>}\end{shaded}\egroup 


    \item[{Content model}]
  \fbox{\ttfamily <content>\newline
</content>\newline
    } 
    \item[{Schema Declaration}]
  \mbox{}\hfill\\[-10pt]\begin{Verbatim}[fontsize=\small]
element lb
{
   att.global.attributes,
   att.typed.attributes,
   att.edition.attributes,
   att.spanning.attributes,
   att.breaking.attributes,
   empty
}
\end{Verbatim}

\end{reflist}  \index{lem=<lem>|oddindex}
\begin{reflist}
\item[]\begin{specHead}{TEI.lem}{<lem> }(lemma) contains the lemma, or base text, of a textual variation. [\xref{http://www.tei-c.org/release/doc/tei-p5-doc/en/html/TC.html\#TCAPLL}{12.1. The Apparatus Entry, Readings, and Witnesses}]\end{specHead} 
    \item[{Module}]
  textcrit
    \item[{Attributes}]
  Attributes att.global (\textit{@xml:id}, \textit{@n}, \textit{@xml:lang}, \textit{@xml:base}, \textit{@xml:space})  (att.global.rendition (\textit{@rend}, \textit{@style}, \textit{@rendition})) (att.global.linking (\textit{@corresp}, \textit{@synch}, \textit{@sameAs}, \textit{@copyOf}, \textit{@next}, \textit{@prev}, \textit{@exclude}, \textit{@select})) (att.global.analytic (\textit{@ana})) (att.global.facs (\textit{@facs})) (att.global.change (\textit{@change})) (att.global.responsibility (\textit{@cert}, \textit{@resp})) (att.global.source (\textit{@source})) att.textCritical (\textit{@type}, \textit{@cause}, \textit{@varSeq}, \textit{@require})  (att.written (\textit{@hand})) att.witnessed (\textit{@wit}) 
    \item[{Contained by}]
  
    \item[textcrit: ]
   app rdgGrp
    \item[{May contain}]
  
    \item[analysis: ]
   c cl interp interpGrp m pc phr s span spanGrp w\par 
    \item[core: ]
   abbr add address bibl biblStruct cb choice cit corr date del desc distinct email emph expan foreign gap gb gloss graphic hi index l label lb lg list listBibl measure measureGrp media mentioned milestone name note num orig p pb ptr q quote ref reg rs said sic soCalled sp stage term time title unclear\par 
    \item[figures: ]
   figure formula notatedMusic table\par 
    \item[gaiji: ]
   g\par 
    \item[header: ]
   biblFull idno\par 
    \item[linking: ]
   ab alt altGrp anchor join joinGrp link linkGrp seg timeline\par 
    \item[msdescription: ]
   catchwords depth dim dimensions height heraldry locus locusGrp material msDesc objectType origDate origPlace secFol signatures stamp watermark width\par 
    \item[namesdates: ]
   addName affiliation bloc climate country district forename genName geo geogFeat geogName listEvent listNym listOrg listPerson listPlace location nameLink offset orgName persName placeName population region roleName settlement state surname terrain trait\par 
    \item[textcrit: ]
   app lacunaEnd lacunaStart listApp listWit wit witDetail witEnd witStart\par 
    \item[textstructure: ]
   div floatingText\par 
    \item[transcr: ]
   addSpan am damage damageSpan delSpan ex fw handShift listTranspose metamark mod redo restore retrace secl space subst substJoin supplied surplus undo\par character data
    \item[{Note}]
  \par
The term \textit{lemma} is used in text criticism to describe the reading in the text itself (as opposed to those in the apparatus); this usage is distinct from that of mathematics (where a lemma is a major step in a proof) and natural-language processing (where a lemma is the dictionary form associated with an inflected form in the running text).
    \item[{Example}]
  \leavevmode\bgroup\exampleFont \begin{shaded}\noindent\mbox{}{<\textbf{app}>}\mbox{}\newline 
\hspace*{6pt}{<\textbf{lem}\hspace*{6pt}{wit}="{\#El \#Hg}">}Experience{</\textbf{lem}>}\mbox{}\newline 
\hspace*{6pt}{<\textbf{rdg}\hspace*{6pt}{type}="{substantive}"\hspace*{6pt}{wit}="{\#La}">}Experiment{</\textbf{rdg}>}\mbox{}\newline 
\hspace*{6pt}{<\textbf{rdg}\hspace*{6pt}{type}="{substantive}"\hspace*{6pt}{wit}="{\#Ra2}">}Eryment{</\textbf{rdg}>}\mbox{}\newline 
{</\textbf{app}>}\end{shaded}\egroup 


    \item[{Content model}]
  \mbox{}\hfill\\[-10pt]\begin{Verbatim}[fontsize=\small]
<content>
 <alternate maxOccurs="unbounded"
  minOccurs="0">
  <textNode/>
  <classRef key="model.divLike"/>
  <classRef key="model.divPart"/>
  <classRef key="model.gLike"/>
  <classRef key="model.phrase"/>
  <classRef key="model.inter"/>
  <classRef key="model.global"/>
  <classRef key="model.rdgPart"/>
 </alternate>
</content>
    
\end{Verbatim}

    \item[{Schema Declaration}]
  \mbox{}\hfill\\[-10pt]\begin{Verbatim}[fontsize=\small]
element lem
{
   att.global.attributes,
   att.textCritical.attributes,
   att.witnessed.attributes,
   (
      text
    | model.divLike    | model.divPart    | model.gLike    | model.phrase    | model.inter    | model.global    | model.rdgPart   )*
}
\end{Verbatim}

\end{reflist}  \index{lg=<lg>|oddindex}
\begin{reflist}
\item[]\begin{specHead}{TEI.lg}{<lg> }(line group) contains one or more verse lines functioning as a formal unit, e.g. a stanza, refrain, verse paragraph, etc. [\xref{http://www.tei-c.org/release/doc/tei-p5-doc/en/html/CO.html\#COVE}{3.12.1. Core Tags for Verse} \xref{http://www.tei-c.org/release/doc/tei-p5-doc/en/html/CO.html\#CODV}{3.12. Passages of Verse or Drama} \xref{http://www.tei-c.org/release/doc/tei-p5-doc/en/html/DR.html\#DRPAL}{7.2.5. Speech Contents}]\end{specHead} 
    \item[{Module}]
  core
    \item[{Attributes}]
  Attributes att.global (\textit{@xml:id}, \textit{@n}, \textit{@xml:lang}, \textit{@xml:base}, \textit{@xml:space})  (att.global.rendition (\textit{@rend}, \textit{@style}, \textit{@rendition})) (att.global.linking (\textit{@corresp}, \textit{@synch}, \textit{@sameAs}, \textit{@copyOf}, \textit{@next}, \textit{@prev}, \textit{@exclude}, \textit{@select})) (att.global.analytic (\textit{@ana})) (att.global.facs (\textit{@facs})) (att.global.change (\textit{@change})) (att.global.responsibility (\textit{@cert}, \textit{@resp})) (att.global.source (\textit{@source})) att.divLike (\textit{@org}, \textit{@sample})  (att.fragmentable (\textit{@part})) att.typed (\textit{@type}, \textit{@subtype}) att.declaring (\textit{@decls}) 
    \item[{Member of}]
  macro.paraContent model.divPart 
    \item[{Contained by}]
  
    \item[core: ]
   add corr del emph head hi item lg note orig p q quote ref reg said sic sp stage title unclear\par 
    \item[figures: ]
   cell figure\par 
    \item[header: ]
   change handNote licence scriptNote typeNote\par 
    \item[linking: ]
   ab seg\par 
    \item[msdescription: ]
   accMat acquisition additions collation condition custEvent decoNote filiation foliation layout musicNotation origin provenance signatures source summary support surrogates\par 
    \item[namesdates: ]
   occupation\par 
    \item[textcrit: ]
   lem rdg\par 
    \item[textstructure: ]
   argument body div docEdition epigraph imprimatur postscript salute signed titlePart trailer\par 
    \item[transcr: ]
   damage metamark mod restore retrace secl supplied surplus
    \item[{May contain}]
  
    \item[analysis: ]
   interp interpGrp span spanGrp\par 
    \item[core: ]
   cb desc gap gb head index l label lb lg meeting milestone note pb stage\par 
    \item[figures: ]
   figure notatedMusic\par 
    \item[linking: ]
   alt altGrp anchor join joinGrp link linkGrp timeline\par 
    \item[textcrit: ]
   app witDetail\par 
    \item[textstructure: ]
   argument byline closer dateline docAuthor docDate epigraph opener postscript salute signed trailer\par 
    \item[transcr: ]
   addSpan damageSpan delSpan fw listTranspose metamark space substJoin
    \item[{Note}]
  \par
contains verse lines or nested line groups only, possibly prefixed by a heading.
    \item[{Example}]
  \leavevmode\bgroup\exampleFont \begin{shaded}\noindent\mbox{}{<\textbf{lg}\hspace*{6pt}{type}="{free}">}\mbox{}\newline 
\hspace*{6pt}{<\textbf{l}>}Let me be my own fool{</\textbf{l}>}\mbox{}\newline 
\hspace*{6pt}{<\textbf{l}>}of my own making, the sum of it{</\textbf{l}>}\mbox{}\newline 
{</\textbf{lg}>}\mbox{}\newline 
{<\textbf{lg}\hspace*{6pt}{type}="{free}">}\mbox{}\newline 
\hspace*{6pt}{<\textbf{l}>}is equivocal.{</\textbf{l}>}\mbox{}\newline 
\hspace*{6pt}{<\textbf{l}>}One says of the drunken farmer:{</\textbf{l}>}\mbox{}\newline 
{</\textbf{lg}>}\mbox{}\newline 
{<\textbf{lg}\hspace*{6pt}{type}="{free}">}\mbox{}\newline 
\hspace*{6pt}{<\textbf{l}>}leave him lay off it. And this is{</\textbf{l}>}\mbox{}\newline 
\hspace*{6pt}{<\textbf{l}>}the explanation.{</\textbf{l}>}\mbox{}\newline 
{</\textbf{lg}>}\end{shaded}\egroup 


    \item[{Schematron}]
   <sch:assert test="count(descendant::tei:lg|descendant::tei:l|descendant::tei:gap) >   0">An lg element  must contain at least one child l, lg or gap element.</sch:assert>
    \item[{Schematron}]
   <s:report test="ancestor::tei:l[not(.//tei:note//tei:lg[. = current()])]"> Abstract model violation: Lines may not contain line groups. </s:report>
    \item[{Content model}]
  \mbox{}\hfill\\[-10pt]\begin{Verbatim}[fontsize=\small]
<content>
 <sequence>
  <alternate maxOccurs="unbounded"
   minOccurs="0">
   <classRef key="model.divTop"/>
   <classRef key="model.global"/>
  </alternate>
  <alternate>
   <classRef key="model.lLike"/>
   <classRef key="model.stageLike"/>
   <classRef key="model.labelLike"/>
   <elementRef key="lg"/>
  </alternate>
  <alternate maxOccurs="unbounded"
   minOccurs="0">
   <classRef key="model.lLike"/>
   <classRef key="model.stageLike"/>
   <classRef key="model.labelLike"/>
   <classRef key="model.global"/>
   <elementRef key="lg"/>
  </alternate>
  <sequence maxOccurs="unbounded"
   minOccurs="0">
   <classRef key="model.divBottom"/>
   <classRef key="model.global"
    maxOccurs="unbounded" minOccurs="0"/>
  </sequence>
 </sequence>
</content>
    
\end{Verbatim}

    \item[{Schema Declaration}]
  \mbox{}\hfill\\[-10pt]\begin{Verbatim}[fontsize=\small]
element lg
{
   att.global.attributes,
   att.divLike.attributes,
   att.typed.attributes,
   att.declaring.attributes,
   (
      ( model.divTop | model.global )*,
      ( model.lLike | model.stageLike | model.labelLike | lg ),
      ( model.lLike | model.stageLike | model.labelLike | model.global | lg )*,
      ( model.divBottom, model.global* )*
   )
}
\end{Verbatim}

\end{reflist}  \index{licence=<licence>|oddindex}
\begin{reflist}
\item[]\begin{specHead}{TEI.licence}{<licence> }contains information about a licence or other legal agreement applicable to the text. [\xref{http://www.tei-c.org/release/doc/tei-p5-doc/en/html/HD.html\#HD24}{2.2.4. Publication, Distribution, Licensing, etc.}]\end{specHead} 
    \item[{Module}]
  header
    \item[{Attributes}]
  Attributes att.global (\textit{@xml:id}, \textit{@n}, \textit{@xml:lang}, \textit{@xml:base}, \textit{@xml:space})  (att.global.rendition (\textit{@rend}, \textit{@style}, \textit{@rendition})) (att.global.linking (\textit{@corresp}, \textit{@synch}, \textit{@sameAs}, \textit{@copyOf}, \textit{@next}, \textit{@prev}, \textit{@exclude}, \textit{@select})) (att.global.analytic (\textit{@ana})) (att.global.facs (\textit{@facs})) (att.global.change (\textit{@change})) (att.global.responsibility (\textit{@cert}, \textit{@resp})) (att.global.source (\textit{@source})) att.pointing (\textit{@targetLang}, \textit{@target}, \textit{@evaluate}) att.datable (\textit{@calendar}, \textit{@period})  (att.datable.w3c (\textit{@when}, \textit{@notBefore}, \textit{@notAfter}, \textit{@from}, \textit{@to})) (att.datable.iso (\textit{@when-iso}, \textit{@notBefore-iso}, \textit{@notAfter-iso}, \textit{@from-iso}, \textit{@to-iso})) (att.datable.custom (\textit{@when-custom}, \textit{@notBefore-custom}, \textit{@notAfter-custom}, \textit{@from-custom}, \textit{@to-custom}, \textit{@datingPoint}, \textit{@datingMethod}))
    \item[{Member of}]
  model.availabilityPart
    \item[{Contained by}]
  
    \item[header: ]
   availability
    \item[{May contain}]
  
    \item[analysis: ]
   c cl interp interpGrp m pc phr s span spanGrp w\par 
    \item[core: ]
   abbr add address bibl biblStruct cb choice cit corr date del desc distinct email emph expan foreign gap gb gloss graphic hi index l label lb lg list listBibl measure measureGrp media mentioned milestone name note num orig p pb ptr q quote ref reg rs said sic soCalled sp stage term time title unclear\par 
    \item[figures: ]
   figure formula notatedMusic table\par 
    \item[gaiji: ]
   g\par 
    \item[header: ]
   biblFull idno\par 
    \item[linking: ]
   ab alt altGrp anchor join joinGrp link linkGrp seg timeline\par 
    \item[msdescription: ]
   catchwords depth dim dimensions height heraldry locus locusGrp material msDesc objectType origDate origPlace secFol signatures stamp watermark width\par 
    \item[namesdates: ]
   addName affiliation bloc climate country district forename genName geo geogFeat geogName listEvent listNym listOrg listPerson listPlace location nameLink offset orgName persName placeName population region roleName settlement state surname terrain trait\par 
    \item[textcrit: ]
   app listApp listWit witDetail\par 
    \item[textstructure: ]
   floatingText\par 
    \item[transcr: ]
   addSpan am damage damageSpan delSpan ex fw handShift listTranspose metamark mod redo restore retrace secl space subst substJoin supplied surplus undo\par character data
    \item[{Note}]
  \par
A <licence> element should be supplied for each licence agreement applicable to the text in question. The {\itshape target} attribute may be used to reference a full version of the licence. The {\itshape when}, {\itshape notBefore}, {\itshape notAfter}, {\itshape from} or {\itshape to} attributes may be used in combination to indicate the date or dates of applicability of the licence.
    \item[{Example}]
  \leavevmode\bgroup\exampleFont \begin{shaded}\noindent\mbox{}{<\textbf{licence}\hspace*{6pt}{target}="{http://www.nzetc.org/tm/scholarly/tei-NZETC-Help.html\#licensing}">} Licence: Creative Commons Attribution-Share Alike 3.0 New Zealand Licence\mbox{}\newline 
{</\textbf{licence}>}\end{shaded}\egroup 


    \item[{Example}]
  \leavevmode\bgroup\exampleFont \begin{shaded}\noindent\mbox{}{<\textbf{availability}>}\mbox{}\newline 
\hspace*{6pt}{<\textbf{licence}\hspace*{6pt}{notBefore}="{2013-01-01}"\mbox{}\newline 
\hspace*{6pt}\hspace*{6pt}{target}="{http://creativecommons.org/licenses/by/3.0/}">}\mbox{}\newline 
\hspace*{6pt}\hspace*{6pt}{<\textbf{p}>}The Creative Commons Attribution 3.0 Unported (CC BY 3.0) Licence\mbox{}\newline 
\hspace*{6pt}\hspace*{6pt}\hspace*{6pt}\hspace*{6pt} applies to this document.{</\textbf{p}>}\mbox{}\newline 
\hspace*{6pt}\hspace*{6pt}{<\textbf{p}>}The licence was added on January 1, 2013.{</\textbf{p}>}\mbox{}\newline 
\hspace*{6pt}{</\textbf{licence}>}\mbox{}\newline 
{</\textbf{availability}>}\end{shaded}\egroup 


    \item[{Content model}]
  \mbox{}\hfill\\[-10pt]\begin{Verbatim}[fontsize=\small]
<content>
 <macroRef key="macro.specialPara"/>
</content>
    
\end{Verbatim}

    \item[{Schema Declaration}]
  \mbox{}\hfill\\[-10pt]\begin{Verbatim}[fontsize=\small]
element licence
{
   att.global.attributes,
   att.pointing.attributes,
   att.datable.attributes,
   macro.specialPara}
\end{Verbatim}

\end{reflist}  \index{line=<line>|oddindex}
\begin{reflist}
\item[]\begin{specHead}{TEI.line}{<line> }contains the transcription of a topographic line in the source document [\xref{http://www.tei-c.org/release/doc/tei-p5-doc/en/html/PH.html\#PHZLAB}{11.2.2. Embedded Transcription}]\end{specHead} 
    \item[{Module}]
  transcr
    \item[{Attributes}]
  Attributes att.typed (\textit{@type}, \textit{@subtype}) att.global (\textit{@xml:id}, \textit{@n}, \textit{@xml:lang}, \textit{@xml:base}, \textit{@xml:space})  (att.global.rendition (\textit{@rend}, \textit{@style}, \textit{@rendition})) (att.global.linking (\textit{@corresp}, \textit{@synch}, \textit{@sameAs}, \textit{@copyOf}, \textit{@next}, \textit{@prev}, \textit{@exclude}, \textit{@select})) (att.global.analytic (\textit{@ana})) (att.global.facs (\textit{@facs})) (att.global.change (\textit{@change})) (att.global.responsibility (\textit{@cert}, \textit{@resp})) (att.global.source (\textit{@source})) att.coordinated (\textit{@start}, \textit{@ulx}, \textit{@uly}, \textit{@lrx}, \textit{@lry}, \textit{@points}) att.written (\textit{@hand}) 
    \item[{Member of}]
  model.linePart 
    \item[{Contained by}]
  
    \item[transcr: ]
   line surface zone
    \item[{May contain}]
  
    \item[analysis: ]
   c interp interpGrp pc span spanGrp w\par 
    \item[core: ]
   add cb choice del gap gb hi index lb milestone note pb unclear\par 
    \item[figures: ]
   figure notatedMusic\par 
    \item[gaiji: ]
   g\par 
    \item[linking: ]
   alt altGrp anchor join joinGrp link linkGrp seg timeline\par 
    \item[textcrit: ]
   app witDetail\par 
    \item[transcr: ]
   addSpan damage damageSpan delSpan fw handShift line listTranspose metamark mod redo restore retrace space substJoin undo zone\par character data
    \item[{Note}]
  \par
This element should be used only to mark up writing which is topographically organized as a series of lines, horizontal or vertical. It should not be used to mark lines of verse (for which use <l>) nor to mark linebreaks within text which has been encoded using structural elements such as <p> (for which use <lb>).
    \item[{Example}]
  \leavevmode\bgroup\exampleFont \begin{shaded}\noindent\mbox{}{<\textbf{surface}>}\mbox{}\newline 
\hspace*{6pt}{<\textbf{zone}>}\mbox{}\newline 
\hspace*{6pt}\hspace*{6pt}{<\textbf{line}>}Poem{</\textbf{line}>}\mbox{}\newline 
\hspace*{6pt}\hspace*{6pt}{<\textbf{line}>}As in Visions of — at{</\textbf{line}>}\mbox{}\newline 
\hspace*{6pt}\hspace*{6pt}{<\textbf{line}>}night —{</\textbf{line}>}\mbox{}\newline 
\hspace*{6pt}\hspace*{6pt}{<\textbf{line}>}All sorts of fancies running through{</\textbf{line}>}\mbox{}\newline 
\hspace*{6pt}\hspace*{6pt}{<\textbf{line}>}the head{</\textbf{line}>}\mbox{}\newline 
\hspace*{6pt}{</\textbf{zone}>}\mbox{}\newline 
{</\textbf{surface}>}\end{shaded}\egroup 


    \item[{Example}]
  \leavevmode\bgroup\exampleFont \begin{shaded}\noindent\mbox{}{<\textbf{surface}>}\mbox{}\newline 
\hspace*{6pt}{<\textbf{zone}>}\mbox{}\newline 
\hspace*{6pt}\hspace*{6pt}{<\textbf{line}>}Hope you enjoyed{</\textbf{line}>}\mbox{}\newline 
\hspace*{6pt}\hspace*{6pt}{<\textbf{line}>}Wales, as they\mbox{}\newline 
\hspace*{6pt}\hspace*{6pt}\hspace*{6pt}\hspace*{6pt} said{</\textbf{line}>}\mbox{}\newline 
\hspace*{6pt}\hspace*{6pt}{<\textbf{line}>}to Mrs FitzHerbert{</\textbf{line}>}\mbox{}\newline 
\hspace*{6pt}\hspace*{6pt}{<\textbf{line}>}Mama{</\textbf{line}>}\mbox{}\newline 
\hspace*{6pt}{</\textbf{zone}>}\mbox{}\newline 
\hspace*{6pt}{<\textbf{zone}>}\mbox{}\newline 
\hspace*{6pt}\hspace*{6pt}{<\textbf{line}>}Printed in England{</\textbf{line}>}\mbox{}\newline 
\hspace*{6pt}{</\textbf{zone}>}\mbox{}\newline 
{</\textbf{surface}>}\end{shaded}\egroup 


    \item[{Content model}]
  \mbox{}\hfill\\[-10pt]\begin{Verbatim}[fontsize=\small]
<content>
 <alternate maxOccurs="unbounded"
  minOccurs="0">
  <textNode/>
  <classRef key="model.global"/>
  <classRef key="model.gLike"/>
  <classRef key="model.linePart"/>
 </alternate>
</content>
    
\end{Verbatim}

    \item[{Schema Declaration}]
  \mbox{}\hfill\\[-10pt]\begin{Verbatim}[fontsize=\small]
element line
{
   att.typed.attributes,
   att.global.attributes,
   att.coordinated.attributes,
   att.written.attributes,
   ( text | model.global | model.gLike | model.linePart )*
}
\end{Verbatim}

\end{reflist}  \index{link=<link>|oddindex}
\begin{reflist}
\item[]\begin{specHead}{TEI.link}{<link> }defines an association or hypertextual link among elements or passages, of some type not more precisely specifiable by other elements. [\xref{http://www.tei-c.org/release/doc/tei-p5-doc/en/html/SA.html\#SAPT}{16.1. Links}]\end{specHead} 
    \item[{Module}]
  linking
    \item[{Attributes}]
  Attributes att.global (\textit{@xml:id}, \textit{@n}, \textit{@xml:lang}, \textit{@xml:base}, \textit{@xml:space})  (att.global.rendition (\textit{@rend}, \textit{@style}, \textit{@rendition})) (att.global.linking (\textit{@corresp}, \textit{@synch}, \textit{@sameAs}, \textit{@copyOf}, \textit{@next}, \textit{@prev}, \textit{@exclude}, \textit{@select})) (att.global.analytic (\textit{@ana})) (att.global.facs (\textit{@facs})) (att.global.change (\textit{@change})) (att.global.responsibility (\textit{@cert}, \textit{@resp})) (att.global.source (\textit{@source})) att.pointing (\textit{@targetLang}, \textit{@target}, \textit{@evaluate}) att.typed (\textit{@type}, \textit{@subtype}) 
    \item[{Member of}]
  model.global.meta 
    \item[{Contained by}]
  
    \item[analysis: ]
   cl m phr s span w\par 
    \item[core: ]
   abbr add addrLine address author bibl biblScope cit citedRange corr date del distinct editor email emph expan foreign gloss head headItem headLabel hi imprint item l label lg list measure mentioned name note num orig p pubPlace publisher q quote ref reg resp rs said series sic soCalled sp speaker stage street term textLang time title unclear\par 
    \item[figures: ]
   cell figure table\par 
    \item[header: ]
   authority change classCode distributor edition extent funder geoDecl handNote language licence principal scriptNote sponsor typeNote\par 
    \item[linking: ]
   ab linkGrp seg\par 
    \item[msdescription: ]
   accMat acquisition additions catchwords collation colophon condition custEvent decoNote explicit filiation finalRubric foliation heraldry incipit layout material msItem musicNotation objectType origDate origPlace origin provenance rubric secFol signatures source stamp summary support surrogates watermark\par 
    \item[namesdates: ]
   addName affiliation age birth bloc country death district education event faith floruit forename genName geogFeat geogName langKnown nameLink nationality occupation offset org orgName persName person personGrp place placeName region residence roleName settlement sex socecStatus surname\par 
    \item[textcrit: ]
   lem rdg wit witDetail\par 
    \item[textstructure: ]
   argument back body byline closer dateline div docAuthor docDate docEdition docImprint docTitle epigraph floatingText front group imprimatur opener postscript salute signed text titlePage titlePart trailer\par 
    \item[transcr: ]
   damage fw line metamark mod restore retrace secl sourceDoc supplied surface surfaceGrp surplus zone
    \item[{May contain}]
  Empty element
    \item[{Note}]
  \par
This element should only be used to encode associations not otherwise provided for by more specific elements.\par
The location of this element within a document has no significance, unless it is included within a <linkGrp>, in which case it may inherit the value of the {\itshape type} attribute from the value given on the <linkGrp>.
    \item[{Example}]
  \leavevmode\bgroup\exampleFont \begin{shaded}\noindent\mbox{}{<\textbf{s}\hspace*{6pt}{n}="{1}">}The state Supreme Court has refused to release {<\textbf{rs}\hspace*{6pt}{xml:id}="{R1}">}\mbox{}\newline 
\hspace*{6pt}\hspace*{6pt}{<\textbf{rs}\hspace*{6pt}{xml:id}="{R2}">}Rahway State Prison{</\textbf{rs}>} inmate{</\textbf{rs}>}\mbox{}\newline 
\hspace*{6pt}{<\textbf{rs}\hspace*{6pt}{xml:id}="{R3}">}James Scott{</\textbf{rs}>} on bail.{</\textbf{s}>}\mbox{}\newline 
{<\textbf{s}\hspace*{6pt}{n}="{2}">}\mbox{}\newline 
\hspace*{6pt}{<\textbf{rs}\hspace*{6pt}{xml:id}="{R4}">}The fighter{</\textbf{rs}>} is serving 30-40 years\mbox{}\newline 
 for a 1975 armed robbery conviction in {<\textbf{rs}\hspace*{6pt}{xml:id}="{R5}">}the penitentiary{</\textbf{rs}>}.\mbox{}\newline 
{</\textbf{s}>}\mbox{}\newline 
\textit{<!-- ... -->}\mbox{}\newline 
{<\textbf{linkGrp}\hspace*{6pt}{type}="{periphrasis}">}\mbox{}\newline 
\hspace*{6pt}{<\textbf{link}\hspace*{6pt}{target}="{\#R1 \#R3 \#R4}"/>}\mbox{}\newline 
\hspace*{6pt}{<\textbf{link}\hspace*{6pt}{target}="{\#R2 \#R5}"/>}\mbox{}\newline 
{</\textbf{linkGrp}>}\end{shaded}\egroup 


    \item[{Schematron}]
   <sch:assert test="contains(normalize-space(@target),' ')">You must supply at least two values for @target or on <sch:name/> </sch:assert>
    \item[{Content model}]
  \fbox{\ttfamily <content>\newline
</content>\newline
    } 
    \item[{Schema Declaration}]
  \mbox{}\hfill\\[-10pt]\begin{Verbatim}[fontsize=\small]
element link
{
   att.global.attributes,
   att.pointing.attributes,
   att.typed.attributes,
   empty
}
\end{Verbatim}

\end{reflist}  \index{linkGrp=<linkGrp>|oddindex}
\begin{reflist}
\item[]\begin{specHead}{TEI.linkGrp}{<linkGrp> }(link group) defines a collection of associations or hypertextual links. [\xref{http://www.tei-c.org/release/doc/tei-p5-doc/en/html/SA.html\#SAPT}{16.1. Links}]\end{specHead} 
    \item[{Module}]
  linking
    \item[{Attributes}]
  Attributes att.global (\textit{@xml:id}, \textit{@n}, \textit{@xml:lang}, \textit{@xml:base}, \textit{@xml:space})  (att.global.rendition (\textit{@rend}, \textit{@style}, \textit{@rendition})) (att.global.linking (\textit{@corresp}, \textit{@synch}, \textit{@sameAs}, \textit{@copyOf}, \textit{@next}, \textit{@prev}, \textit{@exclude}, \textit{@select})) (att.global.analytic (\textit{@ana})) (att.global.facs (\textit{@facs})) (att.global.change (\textit{@change})) (att.global.responsibility (\textit{@cert}, \textit{@resp})) (att.global.source (\textit{@source})) att.pointing.group (\textit{@domains}, \textit{@targFunc})  (att.pointing (\textit{@targetLang}, \textit{@target}, \textit{@evaluate})) (att.typed (\textit{@type}, \textit{@subtype}))
    \item[{Member of}]
  model.global.meta 
    \item[{Contained by}]
  
    \item[analysis: ]
   cl m phr s span w\par 
    \item[core: ]
   abbr add addrLine address author bibl biblScope cit citedRange corr date del distinct editor email emph expan foreign gloss head headItem headLabel hi imprint item l label lg list measure mentioned name note num orig p pubPlace publisher q quote ref reg resp rs said series sic soCalled sp speaker stage street term textLang time title unclear\par 
    \item[figures: ]
   cell figure table\par 
    \item[header: ]
   authority change classCode distributor edition extent funder geoDecl handNote language licence principal scriptNote sponsor typeNote\par 
    \item[linking: ]
   ab seg\par 
    \item[msdescription: ]
   accMat acquisition additions catchwords collation colophon condition custEvent decoNote explicit filiation finalRubric foliation heraldry incipit layout material msItem musicNotation objectType origDate origPlace origin provenance rubric secFol signatures source stamp summary support surrogates watermark\par 
    \item[namesdates: ]
   addName affiliation age birth bloc country death district education event faith floruit forename genName geogFeat geogName langKnown nameLink nationality occupation offset org orgName persName person personGrp place placeName region residence roleName settlement sex socecStatus surname\par 
    \item[textcrit: ]
   lem rdg wit witDetail\par 
    \item[textstructure: ]
   argument back body byline closer dateline div docAuthor docDate docEdition docImprint docTitle epigraph floatingText front group imprimatur opener postscript salute signed text titlePage titlePart trailer\par 
    \item[transcr: ]
   damage fw line metamark mod restore retrace secl sourceDoc supplied surface surfaceGrp surplus zone
    \item[{May contain}]
  
    \item[core: ]
   ptr\par 
    \item[linking: ]
   link
    \item[{Note}]
  \par
May contain one or more <link> elements only, optionally with interspersed pointer elements.\par
A web or link group is an administrative convenience, which should be used to collect a set of links together for any purpose, not simply to supply a default value for the {\itshape type} attribute.
    \item[{Example}]
  \leavevmode\bgroup\exampleFont \begin{shaded}\noindent\mbox{}{<\textbf{linkGrp}\hspace*{6pt}{type}="{translation}">}\mbox{}\newline 
\hspace*{6pt}{<\textbf{link}\hspace*{6pt}{target}="{\#CCS1 \#SW1}"/>}\mbox{}\newline 
\hspace*{6pt}{<\textbf{link}\hspace*{6pt}{target}="{\#CCS2 \#SW2}"/>}\mbox{}\newline 
\hspace*{6pt}{<\textbf{link}\hspace*{6pt}{target}="{\#CCS \#SW}"/>}\mbox{}\newline 
{</\textbf{linkGrp}>}\mbox{}\newline 
{<\textbf{div}\hspace*{6pt}{type}="{volume}"\hspace*{6pt}{xml:id}="{CCS}"\mbox{}\newline 
\hspace*{6pt}{xml:lang}="{fr}">}\mbox{}\newline 
\hspace*{6pt}{<\textbf{p}>}\mbox{}\newline 
\hspace*{6pt}\hspace*{6pt}{<\textbf{s}\hspace*{6pt}{xml:id}="{CCS1}">}Longtemps, je me suis couché de bonne heure.{</\textbf{s}>}\mbox{}\newline 
\hspace*{6pt}\hspace*{6pt}{<\textbf{s}\hspace*{6pt}{xml:id}="{CCS2}">}Parfois, à peine ma bougie éteinte, mes yeux se fermaient si vite que je n'avais pas le temps de me dire : "Je m'endors."{</\textbf{s}>}\mbox{}\newline 
\hspace*{6pt}{</\textbf{p}>}\mbox{}\newline 
\textit{<!-- ... -->}\mbox{}\newline 
{</\textbf{div}>}\mbox{}\newline 
{<\textbf{div}\hspace*{6pt}{type}="{volume}"\hspace*{6pt}{xml:id}="{SW}"\hspace*{6pt}{xml:lang}="{en}">}\mbox{}\newline 
\hspace*{6pt}{<\textbf{p}>}\mbox{}\newline 
\hspace*{6pt}\hspace*{6pt}{<\textbf{s}\hspace*{6pt}{xml:id}="{SW1}">}For a long time I used to go to bed early.{</\textbf{s}>}\mbox{}\newline 
\hspace*{6pt}\hspace*{6pt}{<\textbf{s}\hspace*{6pt}{xml:id}="{SW2}">}Sometimes, when I had put out my candle, my eyes would close so quickly that I had not even time to say "I'm going to sleep."{</\textbf{s}>}\mbox{}\newline 
\hspace*{6pt}{</\textbf{p}>}\mbox{}\newline 
\textit{<!-- ... -->}\mbox{}\newline 
{</\textbf{div}>}\end{shaded}\egroup 


    \item[{Content model}]
  \mbox{}\hfill\\[-10pt]\begin{Verbatim}[fontsize=\small]
<content>
 <alternate maxOccurs="unbounded"
  minOccurs="1">
  <elementRef key="link"/>
  <elementRef key="ptr"/>
 </alternate>
</content>
    
\end{Verbatim}

    \item[{Schema Declaration}]
  \mbox{}\hfill\\[-10pt]\begin{Verbatim}[fontsize=\small]
element linkGrp
{
   att.global.attributes,
   att.pointing.group.attributes,
   ( link | ptr )+
}
\end{Verbatim}

\end{reflist}  \index{list=<list>|oddindex}\index{type=@type!<list>|oddindex}
\begin{reflist}
\item[]\begin{specHead}{TEI.list}{<list> }contains any sequence of items organized as a list. [\xref{http://www.tei-c.org/release/doc/tei-p5-doc/en/html/CO.html\#COLI}{3.7. Lists}]\end{specHead} 
    \item[{Module}]
  core
    \item[{Attributes}]
  Attributes att.global (\textit{@xml:id}, \textit{@n}, \textit{@xml:lang}, \textit{@xml:base}, \textit{@xml:space})  (att.global.rendition (\textit{@rend}, \textit{@style}, \textit{@rendition})) (att.global.linking (\textit{@corresp}, \textit{@synch}, \textit{@sameAs}, \textit{@copyOf}, \textit{@next}, \textit{@prev}, \textit{@exclude}, \textit{@select})) (att.global.analytic (\textit{@ana})) (att.global.facs (\textit{@facs})) (att.global.change (\textit{@change})) (att.global.responsibility (\textit{@cert}, \textit{@resp})) (att.global.source (\textit{@source})) att.sortable (\textit{@sortKey}) att.typed (\unusedattribute{type}, @subtype) \hfil\\[-10pt]\begin{sansreflist}
    \item[@type]
  describes the nature of the items in the list.
\begin{reflist}
    \item[{Derived from}]
  att.typed
    \item[{Status}]
  Optional
    \item[{Datatype}]
  teidata.enumerated
    \item[{Suggested values include:}]
  \begin{description}

\item[{gloss}]each list item glosses some term or concept, which is given by a <label> element preceding the list item.
\item[{index}]each list item is an entry in an index such as the alphabetical topical index at the back of a print volume.
\item[{instructions}]each list item is a step in a sequence of instructions, as in a recipe.
\item[{litany}]each list item is one of a sequence of petitions, supplications or invocations, typically in a religious ritual.
\item[{syllogism}]each list item is part of an argument consisting of two or more propositions and a final conclusion derived from them.
\end{description} 
    \item[{Note}]
  \par
Previous versions of these Guidelines recommended the use of {\itshape type} on <list> to encode the rendering or appearance of a list (whether it was bulleted, numbered, etc.). The current recommendation is to use the {\itshape rend} or {\itshape style} attributes for these aspects of a list, while using {\itshape type} for the more appropriate task of characterizing the nature of the content of a list.
    \item[{Note}]
  \par
The formal syntax of the element declarations allows <label> tags to be omitted from lists tagged <list type="gloss">; this is however a semantic error.
\end{reflist}  
\end{sansreflist}  
    \item[{Member of}]
  model.listLike 
    \item[{Contained by}]
  
    \item[core: ]
   add corr del desc emph head hi item l meeting note orig p q quote ref reg said sic sp stage title unclear\par 
    \item[figures: ]
   cell figDesc figure\par 
    \item[header: ]
   abstract change handNote keywords licence rendition revisionDesc scriptNote sourceDesc tagUsage typeNote\par 
    \item[linking: ]
   ab seg\par 
    \item[msdescription: ]
   accMat acquisition additions collation condition custEvent decoNote filiation foliation layout musicNotation origin provenance signatures source summary support surrogates\par 
    \item[namesdates: ]
   occupation\par 
    \item[textcrit: ]
   lem rdg witness\par 
    \item[textstructure: ]
   argument back body div docEdition epigraph imprimatur postscript salute signed titlePart trailer\par 
    \item[transcr: ]
   damage metamark mod restore retrace secl supplied surplus
    \item[{May contain}]
  
    \item[analysis: ]
   interp interpGrp span spanGrp\par 
    \item[core: ]
   cb gap gb head headItem headLabel index item label lb meeting milestone note pb\par 
    \item[figures: ]
   figure notatedMusic\par 
    \item[linking: ]
   alt altGrp anchor join joinGrp link linkGrp timeline\par 
    \item[textcrit: ]
   app witDetail\par 
    \item[textstructure: ]
   argument byline closer dateline docAuthor docDate epigraph opener postscript salute signed trailer\par 
    \item[transcr: ]
   addSpan damageSpan delSpan fw listTranspose metamark space substJoin
    \item[{Note}]
  \par
May contain an optional heading followed by a series of items, or a series of label and item pairs, the latter being optionally preceded by one or two specialized headings.
    \item[{Example}]
  \leavevmode\bgroup\exampleFont \begin{shaded}\noindent\mbox{}{<\textbf{list}\hspace*{6pt}{rend}="{numbered}">}\mbox{}\newline 
\hspace*{6pt}{<\textbf{item}>}a butcher{</\textbf{item}>}\mbox{}\newline 
\hspace*{6pt}{<\textbf{item}>}a baker{</\textbf{item}>}\mbox{}\newline 
\hspace*{6pt}{<\textbf{item}>}a candlestick maker, with\mbox{}\newline 
\hspace*{6pt}{<\textbf{list}\hspace*{6pt}{rend}="{bulleted}">}\mbox{}\newline 
\hspace*{6pt}\hspace*{6pt}\hspace*{6pt}{<\textbf{item}>}rings on his fingers{</\textbf{item}>}\mbox{}\newline 
\hspace*{6pt}\hspace*{6pt}\hspace*{6pt}{<\textbf{item}>}bells on his toes{</\textbf{item}>}\mbox{}\newline 
\hspace*{6pt}\hspace*{6pt}{</\textbf{list}>}\mbox{}\newline 
\hspace*{6pt}{</\textbf{item}>}\mbox{}\newline 
{</\textbf{list}>}\end{shaded}\egroup 


    \item[{Example}]
  \leavevmode\bgroup\exampleFont \begin{shaded}\noindent\mbox{}{<\textbf{list}\hspace*{6pt}{rend}="{bulleted}"\hspace*{6pt}{type}="{syllogism}">}\mbox{}\newline 
\hspace*{6pt}{<\textbf{item}>}All Cretans are liars.{</\textbf{item}>}\mbox{}\newline 
\hspace*{6pt}{<\textbf{item}>}Epimenides is a Cretan.{</\textbf{item}>}\mbox{}\newline 
\hspace*{6pt}{<\textbf{item}>}ERGO Epimenides is a liar.{</\textbf{item}>}\mbox{}\newline 
{</\textbf{list}>}\end{shaded}\egroup 


    \item[{Example}]
  \leavevmode\bgroup\exampleFont \begin{shaded}\noindent\mbox{}{<\textbf{list}\hspace*{6pt}{rend}="{simple}"\hspace*{6pt}{type}="{litany}">}\mbox{}\newline 
\hspace*{6pt}{<\textbf{item}>}God save us from drought.{</\textbf{item}>}\mbox{}\newline 
\hspace*{6pt}{<\textbf{item}>}God save us from pestilence.{</\textbf{item}>}\mbox{}\newline 
\hspace*{6pt}{<\textbf{item}>}God save us from wickedness in high places.{</\textbf{item}>}\mbox{}\newline 
\hspace*{6pt}{<\textbf{item}>}Praise be to God.{</\textbf{item}>}\mbox{}\newline 
{</\textbf{list}>}\end{shaded}\egroup 


    \item[{Example}]
  The following example treats the short numbered clauses of Anglo-Saxon legal codes as lists of items. The text is from an ordinance of King Athelstan (924–939):\leavevmode\bgroup\exampleFont \begin{shaded}\noindent\mbox{}{<\textbf{div1}\hspace*{6pt}{type}="{section}">}\mbox{}\newline 
\hspace*{6pt}{<\textbf{head}>}Athelstan's Ordinance{</\textbf{head}>}\mbox{}\newline 
\hspace*{6pt}{<\textbf{list}\hspace*{6pt}{rend}="{numbered}">}\mbox{}\newline 
\hspace*{6pt}\hspace*{6pt}{<\textbf{item}\hspace*{6pt}{n}="{1}">}Concerning thieves. First, that no thief is to be spared who is caught with\mbox{}\newline 
\hspace*{6pt}\hspace*{6pt}\hspace*{6pt}\hspace*{6pt} the stolen goods, [if he is] over twelve years and [if the value of the goods is] over\mbox{}\newline 
\hspace*{6pt}\hspace*{6pt}\hspace*{6pt}\hspace*{6pt} eightpence.\mbox{}\newline 
\hspace*{6pt}\hspace*{6pt}{<\textbf{list}\hspace*{6pt}{rend}="{numbered}">}\mbox{}\newline 
\hspace*{6pt}\hspace*{6pt}\hspace*{6pt}\hspace*{6pt}{<\textbf{item}\hspace*{6pt}{n}="{1.1}">}And if anyone does spare one, he is to pay for the thief with his\mbox{}\newline 
\hspace*{6pt}\hspace*{6pt}\hspace*{6pt}\hspace*{6pt}\hspace*{6pt}\hspace*{6pt}\hspace*{6pt}\hspace*{6pt} wergild — and the thief is to be no nearer a settlement on that account — or to\mbox{}\newline 
\hspace*{6pt}\hspace*{6pt}\hspace*{6pt}\hspace*{6pt}\hspace*{6pt}\hspace*{6pt}\hspace*{6pt}\hspace*{6pt} clear himself by an oath of that amount.{</\textbf{item}>}\mbox{}\newline 
\hspace*{6pt}\hspace*{6pt}\hspace*{6pt}\hspace*{6pt}{<\textbf{item}\hspace*{6pt}{n}="{1.2}">}If, however, he [the thief] wishes to defend himself or to escape, he is\mbox{}\newline 
\hspace*{6pt}\hspace*{6pt}\hspace*{6pt}\hspace*{6pt}\hspace*{6pt}\hspace*{6pt}\hspace*{6pt}\hspace*{6pt} not to be spared [whether younger or older than twelve].{</\textbf{item}>}\mbox{}\newline 
\hspace*{6pt}\hspace*{6pt}\hspace*{6pt}\hspace*{6pt}{<\textbf{item}\hspace*{6pt}{n}="{1.3}">}If a thief is put into prison, he is to be in prison 40 days, and he may\mbox{}\newline 
\hspace*{6pt}\hspace*{6pt}\hspace*{6pt}\hspace*{6pt}\hspace*{6pt}\hspace*{6pt}\hspace*{6pt}\hspace*{6pt} then be redeemed with 120 shillings; and the kindred are to stand surety for him\mbox{}\newline 
\hspace*{6pt}\hspace*{6pt}\hspace*{6pt}\hspace*{6pt}\hspace*{6pt}\hspace*{6pt}\hspace*{6pt}\hspace*{6pt} that he will desist for ever.{</\textbf{item}>}\mbox{}\newline 
\hspace*{6pt}\hspace*{6pt}\hspace*{6pt}\hspace*{6pt}{<\textbf{item}\hspace*{6pt}{n}="{1.4}">}And if he steals after that, they are to pay for him with his wergild,\mbox{}\newline 
\hspace*{6pt}\hspace*{6pt}\hspace*{6pt}\hspace*{6pt}\hspace*{6pt}\hspace*{6pt}\hspace*{6pt}\hspace*{6pt} or to bring him back there.{</\textbf{item}>}\mbox{}\newline 
\hspace*{6pt}\hspace*{6pt}\hspace*{6pt}\hspace*{6pt}{<\textbf{item}\hspace*{6pt}{n}="{1.5}">}And if he steals after that, they are to pay for him with his wergild,\mbox{}\newline 
\hspace*{6pt}\hspace*{6pt}\hspace*{6pt}\hspace*{6pt}\hspace*{6pt}\hspace*{6pt}\hspace*{6pt}\hspace*{6pt} whether to the king or to him to whom it rightly belongs; and everyone of those who\mbox{}\newline 
\hspace*{6pt}\hspace*{6pt}\hspace*{6pt}\hspace*{6pt}\hspace*{6pt}\hspace*{6pt}\hspace*{6pt}\hspace*{6pt} supported him is to pay 120 shillings to the king as a fine.{</\textbf{item}>}\mbox{}\newline 
\hspace*{6pt}\hspace*{6pt}\hspace*{6pt}{</\textbf{list}>}\mbox{}\newline 
\hspace*{6pt}\hspace*{6pt}{</\textbf{item}>}\mbox{}\newline 
\hspace*{6pt}\hspace*{6pt}{<\textbf{item}\hspace*{6pt}{n}="{2}">}Concerning lordless men. And we pronounced about these lordless men, from whom\mbox{}\newline 
\hspace*{6pt}\hspace*{6pt}\hspace*{6pt}\hspace*{6pt} no justice can be obtained, that one should order their kindred to fetch back such a\mbox{}\newline 
\hspace*{6pt}\hspace*{6pt}\hspace*{6pt}\hspace*{6pt} person to justice and to find him a lord in public meeting.\mbox{}\newline 
\hspace*{6pt}\hspace*{6pt}{<\textbf{list}\hspace*{6pt}{rend}="{numbered}">}\mbox{}\newline 
\hspace*{6pt}\hspace*{6pt}\hspace*{6pt}\hspace*{6pt}{<\textbf{item}\hspace*{6pt}{n}="{2.1}">}And if they then will not, or cannot, produce him on that appointed day,\mbox{}\newline 
\hspace*{6pt}\hspace*{6pt}\hspace*{6pt}\hspace*{6pt}\hspace*{6pt}\hspace*{6pt}\hspace*{6pt}\hspace*{6pt} he is then to be a fugitive afterwards, and he who encounters him is to strike him\mbox{}\newline 
\hspace*{6pt}\hspace*{6pt}\hspace*{6pt}\hspace*{6pt}\hspace*{6pt}\hspace*{6pt}\hspace*{6pt}\hspace*{6pt} down as a thief.{</\textbf{item}>}\mbox{}\newline 
\hspace*{6pt}\hspace*{6pt}\hspace*{6pt}\hspace*{6pt}{<\textbf{item}\hspace*{6pt}{n}="{2.2}">}And he who harbours him after that, is to pay for him with his wergild\mbox{}\newline 
\hspace*{6pt}\hspace*{6pt}\hspace*{6pt}\hspace*{6pt}\hspace*{6pt}\hspace*{6pt}\hspace*{6pt}\hspace*{6pt} or to clear himself by an oath of that amount.{</\textbf{item}>}\mbox{}\newline 
\hspace*{6pt}\hspace*{6pt}\hspace*{6pt}{</\textbf{list}>}\mbox{}\newline 
\hspace*{6pt}\hspace*{6pt}{</\textbf{item}>}\mbox{}\newline 
\hspace*{6pt}\hspace*{6pt}{<\textbf{item}\hspace*{6pt}{n}="{3}">}Concerning the refusal of justice. The lord who refuses justice and upholds\mbox{}\newline 
\hspace*{6pt}\hspace*{6pt}\hspace*{6pt}\hspace*{6pt} his guilty man, so that the king is appealed to, is to repay the value of the goods and\mbox{}\newline 
\hspace*{6pt}\hspace*{6pt}\hspace*{6pt}\hspace*{6pt} 120 shillings to the king; and he who appeals to the king before he demands justice as\mbox{}\newline 
\hspace*{6pt}\hspace*{6pt}\hspace*{6pt}\hspace*{6pt} often as he ought, is to pay the same fine as the other would have done, if he had\mbox{}\newline 
\hspace*{6pt}\hspace*{6pt}\hspace*{6pt}\hspace*{6pt} refused him justice.\mbox{}\newline 
\hspace*{6pt}\hspace*{6pt}{<\textbf{list}\hspace*{6pt}{rend}="{numbered}">}\mbox{}\newline 
\hspace*{6pt}\hspace*{6pt}\hspace*{6pt}\hspace*{6pt}{<\textbf{item}\hspace*{6pt}{n}="{3.1}">}And the lord who is an accessory to a theft by his slave, and it becomes\mbox{}\newline 
\hspace*{6pt}\hspace*{6pt}\hspace*{6pt}\hspace*{6pt}\hspace*{6pt}\hspace*{6pt}\hspace*{6pt}\hspace*{6pt} known about him, is to forfeit the slave and be liable to his wergild on the first\mbox{}\newline 
\hspace*{6pt}\hspace*{6pt}\hspace*{6pt}\hspace*{6pt}\hspace*{6pt}\hspace*{6pt}\hspace*{6pt}\hspace*{6pt} occasionp if he does it more often, he is to be liable to pay all that he owns.{</\textbf{item}>}\mbox{}\newline 
\hspace*{6pt}\hspace*{6pt}\hspace*{6pt}\hspace*{6pt}{<\textbf{item}\hspace*{6pt}{n}="{3.2}">}And likewise any of the king's treasurers or of our reeves, who has been\mbox{}\newline 
\hspace*{6pt}\hspace*{6pt}\hspace*{6pt}\hspace*{6pt}\hspace*{6pt}\hspace*{6pt}\hspace*{6pt}\hspace*{6pt} an accessory of thieves who have committed theft, is to liable to the same.{</\textbf{item}>}\mbox{}\newline 
\hspace*{6pt}\hspace*{6pt}\hspace*{6pt}{</\textbf{list}>}\mbox{}\newline 
\hspace*{6pt}\hspace*{6pt}{</\textbf{item}>}\mbox{}\newline 
\hspace*{6pt}\hspace*{6pt}{<\textbf{item}\hspace*{6pt}{n}="{4}">}Concerning treachery to a lord. And we have pronounced concerning treachery to\mbox{}\newline 
\hspace*{6pt}\hspace*{6pt}\hspace*{6pt}\hspace*{6pt} a lord, that he [who is accused] is to forfeit his life if he cannot deny it or is\mbox{}\newline 
\hspace*{6pt}\hspace*{6pt}\hspace*{6pt}\hspace*{6pt} afterwards convicted at the three-fold ordeal.{</\textbf{item}>}\mbox{}\newline 
\hspace*{6pt}{</\textbf{list}>}\mbox{}\newline 
{</\textbf{div1}>}\end{shaded}\egroup 

Note that nested lists have been used so the tagging mirrors the structure indicated by the two-level numbering of the clauses. The clauses could have been treated as a one-level list with irregular numbering, if desired.
    \item[{Example}]
  \leavevmode\bgroup\exampleFont \begin{shaded}\noindent\mbox{}{<\textbf{p}>}These decrees, most blessed Pope Hadrian, we propounded in the public council ... and they\mbox{}\newline 
 confirmed them in our hand in your stead with the sign of the Holy Cross, and afterwards\mbox{}\newline 
 inscribed with a careful pen on the paper of this page, affixing thus the sign of the Holy\mbox{}\newline 
 Cross.\mbox{}\newline 
{<\textbf{list}\hspace*{6pt}{rend}="{simple}">}\mbox{}\newline 
\hspace*{6pt}\hspace*{6pt}{<\textbf{item}>}I, Eanbald, by the grace of God archbishop of the holy church of York, have\mbox{}\newline 
\hspace*{6pt}\hspace*{6pt}\hspace*{6pt}\hspace*{6pt} subscribed to the pious and catholic validity of this document with the sign of the Holy\mbox{}\newline 
\hspace*{6pt}\hspace*{6pt}\hspace*{6pt}\hspace*{6pt} Cross.{</\textbf{item}>}\mbox{}\newline 
\hspace*{6pt}\hspace*{6pt}{<\textbf{item}>}I, Ælfwold, king of the people across the Humber, consenting have subscribed with\mbox{}\newline 
\hspace*{6pt}\hspace*{6pt}\hspace*{6pt}\hspace*{6pt} the sign of the Holy Cross.{</\textbf{item}>}\mbox{}\newline 
\hspace*{6pt}\hspace*{6pt}{<\textbf{item}>}I, Tilberht, prelate of the church of Hexham, rejoicing have subscribed with the\mbox{}\newline 
\hspace*{6pt}\hspace*{6pt}\hspace*{6pt}\hspace*{6pt} sign of the Holy Cross.{</\textbf{item}>}\mbox{}\newline 
\hspace*{6pt}\hspace*{6pt}{<\textbf{item}>}I, Higbald, bishop of the church of Lindisfarne, obeying have subscribed with the\mbox{}\newline 
\hspace*{6pt}\hspace*{6pt}\hspace*{6pt}\hspace*{6pt} sign of the Holy Cross.{</\textbf{item}>}\mbox{}\newline 
\hspace*{6pt}\hspace*{6pt}{<\textbf{item}>}I, Ethelbert, bishop of Candida Casa, suppliant, have subscribed with thef sign of\mbox{}\newline 
\hspace*{6pt}\hspace*{6pt}\hspace*{6pt}\hspace*{6pt} the Holy Cross.{</\textbf{item}>}\mbox{}\newline 
\hspace*{6pt}\hspace*{6pt}{<\textbf{item}>}I, Ealdwulf, bishop of the church of Mayo, have subscribed with devout will.{</\textbf{item}>}\mbox{}\newline 
\hspace*{6pt}\hspace*{6pt}{<\textbf{item}>}I, Æthelwine, bishop, have subscribed through delegates.{</\textbf{item}>}\mbox{}\newline 
\hspace*{6pt}\hspace*{6pt}{<\textbf{item}>}I, Sicga, patrician, have subscribed with serene mind with the sign of the Holy\mbox{}\newline 
\hspace*{6pt}\hspace*{6pt}\hspace*{6pt}\hspace*{6pt} Cross.{</\textbf{item}>}\mbox{}\newline 
\hspace*{6pt}{</\textbf{list}>}\mbox{}\newline 
{</\textbf{p}>}\end{shaded}\egroup 


    \item[{Schematron}]
   <sch:rule context="tei:list[@type='gloss']"> <sch:assert test="tei:label">The content of a "gloss" list should include a sequence of one or more pairs of a label element followed by an item element</sch:assert> </sch:rule>
    \item[{Content model}]
  \mbox{}\hfill\\[-10pt]\begin{Verbatim}[fontsize=\small]
<content>
 <sequence>
  <alternate maxOccurs="unbounded"
   minOccurs="0">
   <classRef key="model.divTop"/>
   <classRef key="model.global"/>
  </alternate>
  <alternate>
   <sequence maxOccurs="unbounded"
    minOccurs="1">
    <elementRef key="item"/>
    <classRef key="model.global"
     maxOccurs="unbounded" minOccurs="0"/>
   </sequence>
   <sequence>
    <elementRef key="headLabel"
     minOccurs="0"/>
    <elementRef key="headItem"
     minOccurs="0"/>
    <sequence maxOccurs="unbounded"
     minOccurs="1">
     <elementRef key="label"/>
     <classRef key="model.global"
      maxOccurs="unbounded" minOccurs="0"/>
     <elementRef key="item"/>
     <classRef key="model.global"
      maxOccurs="unbounded" minOccurs="0"/>
    </sequence>
   </sequence>
  </alternate>
  <sequence maxOccurs="unbounded"
   minOccurs="0">
   <classRef key="model.divBottom"/>
   <classRef key="model.global"
    maxOccurs="unbounded" minOccurs="0"/>
  </sequence>
 </sequence>
</content>
    
\end{Verbatim}

    \item[{Schema Declaration}]
  \mbox{}\hfill\\[-10pt]\begin{Verbatim}[fontsize=\small]
element list
{
   att.global.attributes,
   att.sortable.attributes,
   att.typed.attribute.subtype,
   attribute type
   {
      "gloss" | "index" | "instructions" | "litany" | "syllogism"
   }?,
   (
      ( model.divTop | model.global )*,
      (
         ( item, model.global* )+
       | (
            headLabel?,
            headItem?,
            ( label, model.global*, item, model.global* )+
         )
      ),
      ( model.divBottom, model.global* )*
   )
}
\end{Verbatim}

\end{reflist}  \index{listApp=<listApp>|oddindex}
\begin{reflist}
\item[]\begin{specHead}{TEI.listApp}{<listApp> }(list of apparatus entries) contains a list of apparatus entries. [\xref{http://www.tei-c.org/release/doc/tei-p5-doc/en/html/TC.html\#TCAPLK}{12.2. Linking the Apparatus to the Text}]\end{specHead} 
    \item[{Module}]
  textcrit
    \item[{Attributes}]
  Attributes att.global (\textit{@xml:id}, \textit{@n}, \textit{@xml:lang}, \textit{@xml:base}, \textit{@xml:space})  (att.global.rendition (\textit{@rend}, \textit{@style}, \textit{@rendition})) (att.global.linking (\textit{@corresp}, \textit{@synch}, \textit{@sameAs}, \textit{@copyOf}, \textit{@next}, \textit{@prev}, \textit{@exclude}, \textit{@select})) (att.global.analytic (\textit{@ana})) (att.global.facs (\textit{@facs})) (att.global.change (\textit{@change})) (att.global.responsibility (\textit{@cert}, \textit{@resp})) (att.global.source (\textit{@source})) att.sortable (\textit{@sortKey}) att.declarable (\textit{@default}) att.typed (\textit{@type}, \textit{@subtype}) 
    \item[{Member of}]
  model.listLike
    \item[{Contained by}]
  
    \item[core: ]
   add corr del desc emph head hi item l meeting note orig p q quote ref reg said sic sp stage title unclear\par 
    \item[figures: ]
   cell figDesc figure\par 
    \item[header: ]
   abstract change handNote licence rendition scriptNote sourceDesc tagUsage typeNote\par 
    \item[linking: ]
   ab seg\par 
    \item[msdescription: ]
   accMat acquisition additions collation condition custEvent decoNote filiation foliation layout musicNotation origin provenance signatures source summary support surrogates\par 
    \item[namesdates: ]
   occupation\par 
    \item[textcrit: ]
   lem listApp rdg witness\par 
    \item[textstructure: ]
   argument back body div docEdition epigraph imprimatur postscript salute signed titlePart trailer\par 
    \item[transcr: ]
   damage metamark mod restore retrace secl supplied surplus
    \item[{May contain}]
  
    \item[core: ]
   head\par 
    \item[textcrit: ]
   app listApp
    \item[{Note}]
  \par
<listApp> elements would normally be located in the <back> part of a document, but they may appear elsewhere.
    \item[{Example}]
  In the following example from the exegetical Yasna, the base text is encoded in the <body> of the document, and two separate <listApp> elements are used in the <back>, containing variant readings written in different scripts.\leavevmode\bgroup\exampleFont \begin{shaded}\noindent\mbox{}{<\textbf{body}>}\mbox{}\newline 
\hspace*{6pt}{<\textbf{div}>}\mbox{}\newline 
\hspace*{6pt}\hspace*{6pt}{<\textbf{lg}\hspace*{6pt}{rend}="{italic}"\hspace*{6pt}{type}="{stanza}"\mbox{}\newline 
\hspace*{6pt}\hspace*{6pt}\hspace*{6pt}{xml:id}="{Y-36.01}"\hspace*{6pt}{xml:lang}="{pal-Avst}">}\mbox{}\newline 
\hspace*{6pt}\hspace*{6pt}\hspace*{6pt}{<\textbf{l}\hspace*{6pt}{xml:id}="{Y-36.01\textunderscore L-1}">}\mbox{}\newline 
\hspace*{6pt}\hspace*{6pt}\hspace*{6pt}\hspace*{6pt}{<\textbf{w}\hspace*{6pt}{xml:id}="{Y-36.01\textunderscore L1\textunderscore W-01}">}ahiiā{</\textbf{w}>}\mbox{}\newline 
\hspace*{6pt}\hspace*{6pt}\hspace*{6pt}\hspace*{6pt}{<\textbf{w}\hspace*{6pt}{xml:id}="{Y-36.01\textunderscore L1\textunderscore W-02}">}ϑβā{</\textbf{w}>}\mbox{}\newline 
\hspace*{6pt}\hspace*{6pt}\hspace*{6pt}\hspace*{6pt}{<\textbf{w}\hspace*{6pt}{xml:id}="{Y-36.01\textunderscore L1\textunderscore W-03}">}āϑrō{</\textbf{w}>}\mbox{}\newline 
\hspace*{6pt}\hspace*{6pt}\hspace*{6pt}\hspace*{6pt}{<\textbf{w}\hspace*{6pt}{xml:id}="{Y-36.01\textunderscore L1\textunderscore W-04}">}vərəzə̄nā{</\textbf{w}>}\mbox{}\newline 
\hspace*{6pt}\hspace*{6pt}\hspace*{6pt}\hspace*{6pt}{<\textbf{w}\hspace*{6pt}{xml:id}="{Y-36.01\textunderscore L1\textunderscore W-05}">}paouruiiē{</\textbf{w}>}\mbox{}\newline 
\hspace*{6pt}\hspace*{6pt}\hspace*{6pt}\hspace*{6pt}{<\textbf{w}\hspace*{6pt}{xml:id}="{Y-36.01\textunderscore L1\textunderscore W-06}">}pairijasāmaiδē{</\textbf{w}>}\mbox{}\newline 
\hspace*{6pt}\hspace*{6pt}\hspace*{6pt}\hspace*{6pt}{<\textbf{w}\hspace*{6pt}{xml:id}="{Y-36.01\textunderscore L1\textunderscore W-07}">}mazdā{</\textbf{w}>}\mbox{}\newline 
\hspace*{6pt}\hspace*{6pt}\hspace*{6pt}\hspace*{6pt}{<\textbf{w}\hspace*{6pt}{xml:id}="{Y-36.01\textunderscore L1\textunderscore W-08}">}ahurā{</\textbf{w}>}\mbox{}\newline 
\hspace*{6pt}\hspace*{6pt}\hspace*{6pt}{</\textbf{l}>}\mbox{}\newline 
\textit{<!-- ... -->}\mbox{}\newline 
\hspace*{6pt}\hspace*{6pt}{</\textbf{lg}>}\mbox{}\newline 
\hspace*{6pt}{</\textbf{div}>}\mbox{}\newline 
{</\textbf{body}>}\mbox{}\newline 
\textit{<!-- ... -->}\mbox{}\newline 
{<\textbf{back}>}\mbox{}\newline 
\hspace*{6pt}{<\textbf{div}>}\mbox{}\newline 
\hspace*{6pt}\hspace*{6pt}{<\textbf{listApp}\hspace*{6pt}{xml:id}="{CA\textunderscore Y-36}"\mbox{}\newline 
\hspace*{6pt}\hspace*{6pt}\hspace*{6pt}{xml:lang}="{pal-Avst}">}\mbox{}\newline 
\hspace*{6pt}\hspace*{6pt}\hspace*{6pt}{<\textbf{head}>}Variants from witnesses in Avestan script{</\textbf{head}>}\mbox{}\newline 
\hspace*{6pt}\hspace*{6pt}\hspace*{6pt}{<\textbf{app}\hspace*{6pt}{from}="{\#Y-36.01\textunderscore L1\textunderscore W-01}">}\mbox{}\newline 
\hspace*{6pt}\hspace*{6pt}\hspace*{6pt}\hspace*{6pt}{<\textbf{rdg}\hspace*{6pt}{wit}="{\#Pt4 \#F2 \#J2 \#M1}">}ahiiā{</\textbf{rdg}>}\mbox{}\newline 
\hspace*{6pt}\hspace*{6pt}\hspace*{6pt}{</\textbf{app}>}\mbox{}\newline 
\hspace*{6pt}\hspace*{6pt}\hspace*{6pt}{<\textbf{app}\hspace*{6pt}{from}="{\#Y-36.01\textunderscore L1\textunderscore W-02}">}\mbox{}\newline 
\hspace*{6pt}\hspace*{6pt}\hspace*{6pt}\hspace*{6pt}{<\textbf{rdg}\hspace*{6pt}{wit}="{\#Pt4 \#F2 \#J2 \#M1}">}ϑβā{</\textbf{rdg}>}\mbox{}\newline 
\hspace*{6pt}\hspace*{6pt}\hspace*{6pt}{</\textbf{app}>}\mbox{}\newline 
\hspace*{6pt}\hspace*{6pt}\hspace*{6pt}{<\textbf{app}\hspace*{6pt}{from}="{\#Y-36.01\textunderscore L1\textunderscore W-03}">}\mbox{}\newline 
\hspace*{6pt}\hspace*{6pt}\hspace*{6pt}\hspace*{6pt}{<\textbf{rdg}\hspace*{6pt}{wit}="{\#Pt4 \#J2 \#M1}">}āϑrō{</\textbf{rdg}>}\mbox{}\newline 
\hspace*{6pt}\hspace*{6pt}\hspace*{6pt}\hspace*{6pt}{<\textbf{rdg}\hspace*{6pt}{wit}="{\#F2}">}āϑrōi{</\textbf{rdg}>}\mbox{}\newline 
\hspace*{6pt}\hspace*{6pt}\hspace*{6pt}{</\textbf{app}>}\mbox{}\newline 
\textit{<!-- ... -->}\mbox{}\newline 
\hspace*{6pt}\hspace*{6pt}{</\textbf{listApp}>}\mbox{}\newline 
\hspace*{6pt}\hspace*{6pt}{<\textbf{listApp}\hspace*{6pt}{xml:id}="{CA\textunderscore PY-36}"\mbox{}\newline 
\hspace*{6pt}\hspace*{6pt}\hspace*{6pt}{xml:lang}="{pal-Phlv}">}\mbox{}\newline 
\hspace*{6pt}\hspace*{6pt}\hspace*{6pt}{<\textbf{head}>}Variants from witnesses written in Pahlavi script{</\textbf{head}>}\mbox{}\newline 
\hspace*{6pt}\hspace*{6pt}\hspace*{6pt}{<\textbf{app}\hspace*{6pt}{from}="{\#PY-36.01\textunderscore L1\textunderscore W-01}">}\mbox{}\newline 
\hspace*{6pt}\hspace*{6pt}\hspace*{6pt}\hspace*{6pt}{<\textbf{rdg}\hspace*{6pt}{wit}="{\#Pt4 \#F2 \#J2 \#M1}">}ʾytwnˈ{</\textbf{rdg}>}\mbox{}\newline 
\hspace*{6pt}\hspace*{6pt}\hspace*{6pt}{</\textbf{app}>}\mbox{}\newline 
\hspace*{6pt}\hspace*{6pt}\hspace*{6pt}{<\textbf{app}\hspace*{6pt}{from}="{\#PY-36.01\textunderscore L1\textunderscore W-02}">}\mbox{}\newline 
\hspace*{6pt}\hspace*{6pt}\hspace*{6pt}\hspace*{6pt}{<\textbf{rdg}\hspace*{6pt}{wit}="{\#Pt4 \#F2 \#J2 \#M1}">}ʾwˈ{</\textbf{rdg}>}\mbox{}\newline 
\hspace*{6pt}\hspace*{6pt}\hspace*{6pt}{</\textbf{app}>}\mbox{}\newline 
\hspace*{6pt}\hspace*{6pt}\hspace*{6pt}{<\textbf{app}\hspace*{6pt}{from}="{\#PY-36.01\textunderscore L1\textunderscore W-03}">}\mbox{}\newline 
\hspace*{6pt}\hspace*{6pt}\hspace*{6pt}\hspace*{6pt}{<\textbf{rdg}\hspace*{6pt}{wit}="{\#Pt4 \#F2 \#J2 \#M1}">}ḤNʾ{</\textbf{rdg}>}\mbox{}\newline 
\hspace*{6pt}\hspace*{6pt}\hspace*{6pt}{</\textbf{app}>}\mbox{}\newline 
\textit{<!-- ... -->}\mbox{}\newline 
\hspace*{6pt}\hspace*{6pt}{</\textbf{listApp}>}\mbox{}\newline 
\hspace*{6pt}{</\textbf{div}>}\mbox{}\newline 
{</\textbf{back}>}\end{shaded}\egroup 


    \item[{Content model}]
  \mbox{}\hfill\\[-10pt]\begin{Verbatim}[fontsize=\small]
<content>
 <sequence>
  <classRef key="model.headLike"
   maxOccurs="unbounded" minOccurs="0"/>
  <alternate maxOccurs="unbounded"
   minOccurs="1">
   <elementRef key="app"/>
   <elementRef key="listApp"/>
  </alternate>
 </sequence>
</content>
    
\end{Verbatim}

    \item[{Schema Declaration}]
  \mbox{}\hfill\\[-10pt]\begin{Verbatim}[fontsize=\small]
element listApp
{
   att.global.attributes,
   att.sortable.attributes,
   att.declarable.attributes,
   att.typed.attributes,
   ( model.headLike*, ( app | listApp )+ )
}
\end{Verbatim}

\end{reflist}  \index{listBibl=<listBibl>|oddindex}
\begin{reflist}
\item[]\begin{specHead}{TEI.listBibl}{<listBibl> }(citation list) contains a list of bibliographic citations of any kind. [\xref{http://www.tei-c.org/release/doc/tei-p5-doc/en/html/CO.html\#COBITY}{3.11.1. Methods of Encoding Bibliographic References and Lists of References} \xref{http://www.tei-c.org/release/doc/tei-p5-doc/en/html/HD.html\#HD3}{2.2.7. The Source Description} \xref{http://www.tei-c.org/release/doc/tei-p5-doc/en/html/CC.html\#CCAS2}{15.3.2. Declarable Elements}]\end{specHead} 
    \item[{Module}]
  core
    \item[{Attributes}]
  Attributes att.global (\textit{@xml:id}, \textit{@n}, \textit{@xml:lang}, \textit{@xml:base}, \textit{@xml:space})  (att.global.rendition (\textit{@rend}, \textit{@style}, \textit{@rendition})) (att.global.linking (\textit{@corresp}, \textit{@synch}, \textit{@sameAs}, \textit{@copyOf}, \textit{@next}, \textit{@prev}, \textit{@exclude}, \textit{@select})) (att.global.analytic (\textit{@ana})) (att.global.facs (\textit{@facs})) (att.global.change (\textit{@change})) (att.global.responsibility (\textit{@cert}, \textit{@resp})) (att.global.source (\textit{@source})) att.sortable (\textit{@sortKey}) att.declarable (\textit{@default}) att.typed (\textit{@type}, \textit{@subtype}) 
    \item[{Member of}]
  model.biblLike model.frontPart 
    \item[{Contained by}]
  
    \item[core: ]
   add cit corr del desc emph head hi item l listBibl meeting note orig p q quote ref reg relatedItem said sic stage title unclear\par 
    \item[figures: ]
   cell figDesc figure\par 
    \item[header: ]
   change handNote licence rendition scriptNote sourceDesc tagUsage taxonomy typeNote\par 
    \item[linking: ]
   ab seg\par 
    \item[msdescription: ]
   accMat acquisition additional additions collation condition custEvent decoNote filiation foliation layout msItem msItemStruct musicNotation origin provenance signatures source summary support surrogates\par 
    \item[namesdates: ]
   climate event location occupation org person personGrp place population state terrain trait\par 
    \item[textcrit: ]
   lem rdg witness\par 
    \item[textstructure: ]
   argument back body div docEdition epigraph front imprimatur postscript salute signed titlePart trailer\par 
    \item[transcr: ]
   damage metamark mod restore retrace secl supplied surplus
    \item[{May contain}]
  
    \item[core: ]
   bibl biblStruct cb gb head lb listBibl milestone pb\par 
    \item[header: ]
   biblFull\par 
    \item[linking: ]
   anchor\par 
    \item[msdescription: ]
   msDesc\par 
    \item[namesdates: ]
   listRelation relation\par 
    \item[transcr: ]
   fw
    \item[{Example}]
  \leavevmode\bgroup\exampleFont \begin{shaded}\noindent\mbox{}{<\textbf{listBibl}>}\mbox{}\newline 
\hspace*{6pt}{<\textbf{head}>}Works consulted{</\textbf{head}>}\mbox{}\newline 
\hspace*{6pt}{<\textbf{bibl}>}Blain, Clements and Grundy: Feminist Companion to\mbox{}\newline 
\hspace*{6pt}\hspace*{6pt} Literature in English (Yale, 1990)\mbox{}\newline 
\hspace*{6pt}{</\textbf{bibl}>}\mbox{}\newline 
\hspace*{6pt}{<\textbf{biblStruct}>}\mbox{}\newline 
\hspace*{6pt}\hspace*{6pt}{<\textbf{analytic}>}\mbox{}\newline 
\hspace*{6pt}\hspace*{6pt}\hspace*{6pt}{<\textbf{title}>}The Interesting story of the Children in the Wood{</\textbf{title}>}\mbox{}\newline 
\hspace*{6pt}\hspace*{6pt}{</\textbf{analytic}>}\mbox{}\newline 
\hspace*{6pt}\hspace*{6pt}{<\textbf{monogr}>}\mbox{}\newline 
\hspace*{6pt}\hspace*{6pt}\hspace*{6pt}{<\textbf{title}>}The Penny Histories{</\textbf{title}>}\mbox{}\newline 
\hspace*{6pt}\hspace*{6pt}\hspace*{6pt}{<\textbf{author}>}Victor E Neuberg{</\textbf{author}>}\mbox{}\newline 
\hspace*{6pt}\hspace*{6pt}\hspace*{6pt}{<\textbf{imprint}>}\mbox{}\newline 
\hspace*{6pt}\hspace*{6pt}\hspace*{6pt}\hspace*{6pt}{<\textbf{publisher}>}OUP{</\textbf{publisher}>}\mbox{}\newline 
\hspace*{6pt}\hspace*{6pt}\hspace*{6pt}\hspace*{6pt}{<\textbf{date}>}1968{</\textbf{date}>}\mbox{}\newline 
\hspace*{6pt}\hspace*{6pt}\hspace*{6pt}{</\textbf{imprint}>}\mbox{}\newline 
\hspace*{6pt}\hspace*{6pt}{</\textbf{monogr}>}\mbox{}\newline 
\hspace*{6pt}{</\textbf{biblStruct}>}\mbox{}\newline 
{</\textbf{listBibl}>}\end{shaded}\egroup 


    \item[{Content model}]
  \mbox{}\hfill\\[-10pt]\begin{Verbatim}[fontsize=\small]
<content>
 <sequence>
  <classRef key="model.headLike"
   maxOccurs="unbounded" minOccurs="0"/>
  <alternate maxOccurs="unbounded"
   minOccurs="1">
   <classRef key="model.biblLike"/>
   <classRef key="model.milestoneLike"/>
  </alternate>
  <alternate maxOccurs="unbounded"
   minOccurs="0">
   <elementRef key="relation"/>
   <elementRef key="listRelation"/>
  </alternate>
 </sequence>
</content>
    
\end{Verbatim}

    \item[{Schema Declaration}]
  \mbox{}\hfill\\[-10pt]\begin{Verbatim}[fontsize=\small]
element listBibl
{
   att.global.attributes,
   att.sortable.attributes,
   att.declarable.attributes,
   att.typed.attributes,
   (
      model.headLike*,
      ( model.biblLike | model.milestoneLike )+,
      ( relation | listRelation )*
   )
}
\end{Verbatim}

\end{reflist}  \index{listChange=<listChange>|oddindex}\index{ordered=@ordered!<listChange>|oddindex}
\begin{reflist}
\item[]\begin{specHead}{TEI.listChange}{<listChange> }groups a number of change descriptions associated with either the creation of a source text or the revision of an encoded text. [\xref{http://www.tei-c.org/release/doc/tei-p5-doc/en/html/HD.html\#HD6}{2.6. The Revision Description} \xref{http://www.tei-c.org/release/doc/tei-p5-doc/en/html/PH.html\#PH-changes}{11.7. Identifying Changes and Revisions}]\end{specHead} 
    \item[{Module}]
  header
    \item[{Attributes}]
  Attributes att.global (\textit{@xml:id}, \textit{@n}, \textit{@xml:lang}, \textit{@xml:base}, \textit{@xml:space})  (att.global.rendition (\textit{@rend}, \textit{@style}, \textit{@rendition})) (att.global.linking (\textit{@corresp}, \textit{@synch}, \textit{@sameAs}, \textit{@copyOf}, \textit{@next}, \textit{@prev}, \textit{@exclude}, \textit{@select})) (att.global.analytic (\textit{@ana})) (att.global.facs (\textit{@facs})) (att.global.change (\textit{@change})) (att.global.responsibility (\textit{@cert}, \textit{@resp})) (att.global.source (\textit{@source})) att.sortable (\textit{@sortKey}) att.typed (\textit{@type}, \textit{@subtype}) \hfil\\[-10pt]\begin{sansreflist}
    \item[@ordered]
  indicates whether the ordering of its child <change> elements is to be considered significant or not
\begin{reflist}
    \item[{Status}]
  Optional
    \item[{Datatype}]
  teidata.truthValue
    \item[{Default}]
  true
\end{reflist}  
\end{sansreflist}  
    \item[{Contained by}]
  
    \item[header: ]
   creation listChange revisionDesc
    \item[{May contain}]
  
    \item[header: ]
   change listChange
    \item[{Note}]
  \par
When this element appears within the <creation> element it documents the set of revision campaigns or stages identified during the evolution of the original text. When it appears within the <revisionDesc> element, it documents only changes made during the evolution of the encoded representation of that text.
    \item[{Example}]
  \leavevmode\bgroup\exampleFont \begin{shaded}\noindent\mbox{}{<\textbf{revisionDesc}>}\mbox{}\newline 
\hspace*{6pt}{<\textbf{listChange}>}\mbox{}\newline 
\hspace*{6pt}\hspace*{6pt}{<\textbf{change}\hspace*{6pt}{when}="{1991-11-11}"\hspace*{6pt}{who}="{\#LB}">} deleted chapter 10 {</\textbf{change}>}\mbox{}\newline 
\hspace*{6pt}\hspace*{6pt}{<\textbf{change}\hspace*{6pt}{when}="{1991-11-02}"\hspace*{6pt}{who}="{\#MSM}">} completed first draft {</\textbf{change}>}\mbox{}\newline 
\hspace*{6pt}{</\textbf{listChange}>}\mbox{}\newline 
{</\textbf{revisionDesc}>}\end{shaded}\egroup 


    \item[{Example}]
  \leavevmode\bgroup\exampleFont \begin{shaded}\noindent\mbox{}{<\textbf{profileDesc}>}\mbox{}\newline 
\hspace*{6pt}{<\textbf{creation}>}\mbox{}\newline 
\hspace*{6pt}\hspace*{6pt}{<\textbf{listChange}\hspace*{6pt}{ordered}="{true}">}\mbox{}\newline 
\hspace*{6pt}\hspace*{6pt}\hspace*{6pt}{<\textbf{change}\hspace*{6pt}{xml:id}="{CHG-1}">}First stage, written in ink by a writer{</\textbf{change}>}\mbox{}\newline 
\hspace*{6pt}\hspace*{6pt}\hspace*{6pt}{<\textbf{change}\hspace*{6pt}{xml:id}="{CHG-2}">}Second stage, written in Goethe's hand using pencil{</\textbf{change}>}\mbox{}\newline 
\hspace*{6pt}\hspace*{6pt}\hspace*{6pt}{<\textbf{change}\hspace*{6pt}{xml:id}="{CHG-3}">}Fixation of the revised passages and further revisions by\mbox{}\newline 
\hspace*{6pt}\hspace*{6pt}\hspace*{6pt}\hspace*{6pt}\hspace*{6pt}\hspace*{6pt} Goethe using ink{</\textbf{change}>}\mbox{}\newline 
\hspace*{6pt}\hspace*{6pt}\hspace*{6pt}{<\textbf{change}\hspace*{6pt}{xml:id}="{CHG-4}">}Addition of another stanza in a different hand,\mbox{}\newline 
\hspace*{6pt}\hspace*{6pt}\hspace*{6pt}\hspace*{6pt}\hspace*{6pt}\hspace*{6pt} probably at a later stage{</\textbf{change}>}\mbox{}\newline 
\hspace*{6pt}\hspace*{6pt}{</\textbf{listChange}>}\mbox{}\newline 
\hspace*{6pt}{</\textbf{creation}>}\mbox{}\newline 
{</\textbf{profileDesc}>}\end{shaded}\egroup 


    \item[{Content model}]
  \mbox{}\hfill\\[-10pt]\begin{Verbatim}[fontsize=\small]
<content>
 <alternate maxOccurs="unbounded"
  minOccurs="1">
  <elementRef key="listChange"/>
  <elementRef key="change"/>
 </alternate>
</content>
    
\end{Verbatim}

    \item[{Schema Declaration}]
  \mbox{}\hfill\\[-10pt]\begin{Verbatim}[fontsize=\small]
element listChange
{
   att.global.attributes,
   att.sortable.attributes,
   att.typed.attributes,
   attribute ordered { text }?,
   ( listChange | change )+
}
\end{Verbatim}

\end{reflist}  \index{listEvent=<listEvent>|oddindex}
\begin{reflist}
\item[]\begin{specHead}{TEI.listEvent}{<listEvent> }(list of events) contains a list of descriptions, each of which provides information about an identifiable event. [\xref{http://www.tei-c.org/release/doc/tei-p5-doc/en/html/ND.html\#NDPERSbp}{13.3.1. Basic Principles}]\end{specHead} 
    \item[{Module}]
  namesdates
    \item[{Attributes}]
  Attributes att.global (\textit{@xml:id}, \textit{@n}, \textit{@xml:lang}, \textit{@xml:base}, \textit{@xml:space})  (att.global.rendition (\textit{@rend}, \textit{@style}, \textit{@rendition})) (att.global.linking (\textit{@corresp}, \textit{@synch}, \textit{@sameAs}, \textit{@copyOf}, \textit{@next}, \textit{@prev}, \textit{@exclude}, \textit{@select})) (att.global.analytic (\textit{@ana})) (att.global.facs (\textit{@facs})) (att.global.change (\textit{@change})) (att.global.responsibility (\textit{@cert}, \textit{@resp})) (att.global.source (\textit{@source})) att.typed (\textit{@type}, \textit{@subtype}) att.declarable (\textit{@default}) att.sortable (\textit{@sortKey}) 
    \item[{Member of}]
  model.eventLike model.listLike
    \item[{Contained by}]
  
    \item[core: ]
   add corr del desc emph head hi item l meeting note orig p q quote ref reg said sic sp stage title unclear\par 
    \item[figures: ]
   cell figDesc figure\par 
    \item[header: ]
   abstract change handNote licence rendition scriptNote sourceDesc tagUsage typeNote\par 
    \item[linking: ]
   ab seg\par 
    \item[msdescription: ]
   accMat acquisition additions collation condition custEvent decoNote filiation foliation layout musicNotation origin provenance signatures source summary support surrogates\par 
    \item[namesdates: ]
   listEvent occupation org person personGrp place\par 
    \item[textcrit: ]
   lem rdg witness\par 
    \item[textstructure: ]
   argument back body div docEdition epigraph imprimatur postscript salute signed titlePart trailer\par 
    \item[transcr: ]
   damage metamark mod restore retrace secl supplied surplus
    \item[{May contain}]
  
    \item[core: ]
   head\par 
    \item[namesdates: ]
   event listEvent listRelation relation
    \item[{Example}]
  \leavevmode\bgroup\exampleFont \begin{shaded}\noindent\mbox{}{<\textbf{listEvent}>}\mbox{}\newline 
\hspace*{6pt}{<\textbf{head}>}Battles of the American Civil War: Kentucky{</\textbf{head}>}\mbox{}\newline 
\hspace*{6pt}{<\textbf{event}\hspace*{6pt}{when}="{1861-09-19}"\hspace*{6pt}{xml:id}="{event01}">}\mbox{}\newline 
\hspace*{6pt}\hspace*{6pt}{<\textbf{label}>}Barbourville{</\textbf{label}>}\mbox{}\newline 
\hspace*{6pt}\hspace*{6pt}{<\textbf{desc}>}The Battle of Barbourville was one of the early engagements of\mbox{}\newline 
\hspace*{6pt}\hspace*{6pt}\hspace*{6pt}\hspace*{6pt} the American Civil War. It occurred September 19, 1861, in Knox\mbox{}\newline 
\hspace*{6pt}\hspace*{6pt}\hspace*{6pt}\hspace*{6pt} County, Kentucky during the campaign known as the Kentucky Confederate\mbox{}\newline 
\hspace*{6pt}\hspace*{6pt}\hspace*{6pt}\hspace*{6pt} Offensive. The battle is considered the first Confederate victory in\mbox{}\newline 
\hspace*{6pt}\hspace*{6pt}\hspace*{6pt}\hspace*{6pt} the commonwealth, and threw a scare into Federal commanders, who\mbox{}\newline 
\hspace*{6pt}\hspace*{6pt}\hspace*{6pt}\hspace*{6pt} rushed troops to central Kentucky in an effort to repel the invasion,\mbox{}\newline 
\hspace*{6pt}\hspace*{6pt}\hspace*{6pt}\hspace*{6pt} which was finally thwarted at the {<\textbf{ref}\hspace*{6pt}{target}="{\#event02}">}Battle of\mbox{}\newline 
\hspace*{6pt}\hspace*{6pt}\hspace*{6pt}\hspace*{6pt}\hspace*{6pt}\hspace*{6pt} Camp Wildcat{</\textbf{ref}>} in October.{</\textbf{desc}>}\mbox{}\newline 
\hspace*{6pt}{</\textbf{event}>}\mbox{}\newline 
\hspace*{6pt}{<\textbf{event}\hspace*{6pt}{when}="{1861-10-21}"\hspace*{6pt}{xml:id}="{event02}">}\mbox{}\newline 
\hspace*{6pt}\hspace*{6pt}{<\textbf{label}>}Camp Wild Cat{</\textbf{label}>}\mbox{}\newline 
\hspace*{6pt}\hspace*{6pt}{<\textbf{desc}>}The Battle of Camp Wildcat (also known as Wildcat Mountain and Camp\mbox{}\newline 
\hspace*{6pt}\hspace*{6pt}\hspace*{6pt}\hspace*{6pt} Wild Cat) was one of the early engagements of the American Civil\mbox{}\newline 
\hspace*{6pt}\hspace*{6pt}\hspace*{6pt}\hspace*{6pt} War. It occurred October 21, 1861, in northern Laurel County, Kentucky\mbox{}\newline 
\hspace*{6pt}\hspace*{6pt}\hspace*{6pt}\hspace*{6pt} during the campaign known as the Kentucky Confederate Offensive. The\mbox{}\newline 
\hspace*{6pt}\hspace*{6pt}\hspace*{6pt}\hspace*{6pt} battle is considered one of the very first Union victories, and marked\mbox{}\newline 
\hspace*{6pt}\hspace*{6pt}\hspace*{6pt}\hspace*{6pt} the first engagement of troops in the commonwealth of Kentucky.{</\textbf{desc}>}\mbox{}\newline 
\hspace*{6pt}{</\textbf{event}>}\mbox{}\newline 
\hspace*{6pt}{<\textbf{event}\hspace*{6pt}{from}="{1864-06-11}"\hspace*{6pt}{to}="{1864-06-12}"\mbox{}\newline 
\hspace*{6pt}\hspace*{6pt}{xml:id}="{event03}">}\mbox{}\newline 
\hspace*{6pt}\hspace*{6pt}{<\textbf{label}>}Cynthiana{</\textbf{label}>}\mbox{}\newline 
\hspace*{6pt}\hspace*{6pt}{<\textbf{desc}>}The Battle of Cynthiana (or Kellar’s Bridge) was an engagement\mbox{}\newline 
\hspace*{6pt}\hspace*{6pt}\hspace*{6pt}\hspace*{6pt} during the American Civil War that was fought on June 11 and 12, 1864,\mbox{}\newline 
\hspace*{6pt}\hspace*{6pt}\hspace*{6pt}\hspace*{6pt} in Harrison County, Kentucky, near the town of Cynthiana. A part of\mbox{}\newline 
\hspace*{6pt}\hspace*{6pt}\hspace*{6pt}\hspace*{6pt} Confederate Brigadier General John Hunt Morgan's 1864 Raid into\mbox{}\newline 
\hspace*{6pt}\hspace*{6pt}\hspace*{6pt}\hspace*{6pt} Kentucky, the battle resulted in a victory by Union forces over the\mbox{}\newline 
\hspace*{6pt}\hspace*{6pt}\hspace*{6pt}\hspace*{6pt} raiders and saved the town from capture.{</\textbf{desc}>}\mbox{}\newline 
\hspace*{6pt}{</\textbf{event}>}\mbox{}\newline 
{</\textbf{listEvent}>}\end{shaded}\egroup 


    \item[{Content model}]
  \mbox{}\hfill\\[-10pt]\begin{Verbatim}[fontsize=\small]
<content>
 <sequence>
  <classRef key="model.headLike"
   maxOccurs="unbounded" minOccurs="0"/>
  <alternate maxOccurs="unbounded"
   minOccurs="1">
   <elementRef key="event"/>
   <elementRef key="listEvent"/>
  </alternate>
  <alternate maxOccurs="unbounded"
   minOccurs="0">
   <elementRef key="relation"/>
   <elementRef key="listRelation"/>
  </alternate>
 </sequence>
</content>
    
\end{Verbatim}

    \item[{Schema Declaration}]
  \mbox{}\hfill\\[-10pt]\begin{Verbatim}[fontsize=\small]
element listEvent
{
   att.global.attributes,
   att.typed.attributes,
   att.declarable.attributes,
   att.sortable.attributes,
   ( model.headLike*, ( event | listEvent )+, ( relation | listRelation )* )
}
\end{Verbatim}

\end{reflist}  \index{listNym=<listNym>|oddindex}
\begin{reflist}
\item[]\begin{specHead}{TEI.listNym}{<listNym> }(list of canonical names) contains a list of nyms, that is, standardized names for any thing. [\xref{http://www.tei-c.org/release/doc/tei-p5-doc/en/html/ND.html\#NDNYM}{13.3.5. Names and Nyms}]\end{specHead} 
    \item[{Module}]
  namesdates
    \item[{Attributes}]
  Attributes att.global (\textit{@xml:id}, \textit{@n}, \textit{@xml:lang}, \textit{@xml:base}, \textit{@xml:space})  (att.global.rendition (\textit{@rend}, \textit{@style}, \textit{@rendition})) (att.global.linking (\textit{@corresp}, \textit{@synch}, \textit{@sameAs}, \textit{@copyOf}, \textit{@next}, \textit{@prev}, \textit{@exclude}, \textit{@select})) (att.global.analytic (\textit{@ana})) (att.global.facs (\textit{@facs})) (att.global.change (\textit{@change})) (att.global.responsibility (\textit{@cert}, \textit{@resp})) (att.global.source (\textit{@source})) att.typed (\textit{@type}, \textit{@subtype}) att.declarable (\textit{@default}) att.sortable (\textit{@sortKey}) 
    \item[{Member of}]
  model.listLike
    \item[{Contained by}]
  
    \item[core: ]
   add corr del desc emph head hi item l meeting note orig p q quote ref reg said sic sp stage title unclear\par 
    \item[figures: ]
   cell figDesc figure\par 
    \item[header: ]
   abstract change handNote licence rendition scriptNote sourceDesc tagUsage typeNote\par 
    \item[linking: ]
   ab seg\par 
    \item[msdescription: ]
   accMat acquisition additions collation condition custEvent decoNote filiation foliation layout musicNotation origin provenance signatures source summary support surrogates\par 
    \item[namesdates: ]
   listNym occupation\par 
    \item[textcrit: ]
   lem rdg witness\par 
    \item[textstructure: ]
   argument back body div docEdition epigraph imprimatur postscript salute signed titlePart trailer\par 
    \item[transcr: ]
   damage metamark mod restore retrace secl supplied surplus
    \item[{May contain}]
  
    \item[core: ]
   head\par 
    \item[namesdates: ]
   listNym listRelation nym relation
    \item[{Note}]
  \par
The type attribute may be used to distinguish lists of names of a particular type if convenient.
    \item[{Example}]
  \leavevmode\bgroup\exampleFont \begin{shaded}\noindent\mbox{}{<\textbf{listNym}\hspace*{6pt}{type}="{floral}">}\mbox{}\newline 
\hspace*{6pt}{<\textbf{nym}\hspace*{6pt}{xml:id}="{ROSE}">}\mbox{}\newline 
\hspace*{6pt}\hspace*{6pt}{<\textbf{form}>}Rose{</\textbf{form}>}\mbox{}\newline 
\hspace*{6pt}{</\textbf{nym}>}\mbox{}\newline 
\hspace*{6pt}{<\textbf{nym}\hspace*{6pt}{xml:id}="{DAISY}">}\mbox{}\newline 
\hspace*{6pt}\hspace*{6pt}{<\textbf{form}>}Daisy{</\textbf{form}>}\mbox{}\newline 
\hspace*{6pt}\hspace*{6pt}{<\textbf{etym}>}Contraction of {<\textbf{mentioned}>}day's eye{</\textbf{mentioned}>}\mbox{}\newline 
\hspace*{6pt}\hspace*{6pt}{</\textbf{etym}>}\mbox{}\newline 
\hspace*{6pt}{</\textbf{nym}>}\mbox{}\newline 
\hspace*{6pt}{<\textbf{nym}\hspace*{6pt}{xml:id}="{HTHR}">}\mbox{}\newline 
\hspace*{6pt}\hspace*{6pt}{<\textbf{form}>}Heather{</\textbf{form}>}\mbox{}\newline 
\hspace*{6pt}{</\textbf{nym}>}\mbox{}\newline 
{</\textbf{listNym}>}\end{shaded}\egroup 


    \item[{Content model}]
  \mbox{}\hfill\\[-10pt]\begin{Verbatim}[fontsize=\small]
<content>
 <sequence>
  <classRef key="model.headLike"
   maxOccurs="unbounded" minOccurs="0"/>
  <alternate maxOccurs="unbounded"
   minOccurs="1">
   <elementRef key="nym"/>
   <elementRef key="listNym"/>
  </alternate>
  <alternate maxOccurs="unbounded"
   minOccurs="0">
   <elementRef key="relation"/>
   <elementRef key="listRelation"/>
  </alternate>
 </sequence>
</content>
    
\end{Verbatim}

    \item[{Schema Declaration}]
  \mbox{}\hfill\\[-10pt]\begin{Verbatim}[fontsize=\small]
element listNym
{
   att.global.attributes,
   att.typed.attributes,
   att.declarable.attributes,
   att.sortable.attributes,
   ( model.headLike*, ( nym | listNym )+, ( relation | listRelation )* )
}
\end{Verbatim}

\end{reflist}  \index{listOrg=<listOrg>|oddindex}
\begin{reflist}
\item[]\begin{specHead}{TEI.listOrg}{<listOrg> }(list of organizations) contains a list of elements, each of which provides information about an identifiable organization. [\xref{http://www.tei-c.org/release/doc/tei-p5-doc/en/html/ND.html\#NDORG}{13.2.2. Organizational Names}]\end{specHead} 
    \item[{Module}]
  namesdates
    \item[{Attributes}]
  Attributes att.global (\textit{@xml:id}, \textit{@n}, \textit{@xml:lang}, \textit{@xml:base}, \textit{@xml:space})  (att.global.rendition (\textit{@rend}, \textit{@style}, \textit{@rendition})) (att.global.linking (\textit{@corresp}, \textit{@synch}, \textit{@sameAs}, \textit{@copyOf}, \textit{@next}, \textit{@prev}, \textit{@exclude}, \textit{@select})) (att.global.analytic (\textit{@ana})) (att.global.facs (\textit{@facs})) (att.global.change (\textit{@change})) (att.global.responsibility (\textit{@cert}, \textit{@resp})) (att.global.source (\textit{@source})) att.typed (\textit{@type}, \textit{@subtype}) att.declarable (\textit{@default}) att.sortable (\textit{@sortKey}) 
    \item[{Member of}]
  model.listLike model.orgPart
    \item[{Contained by}]
  
    \item[core: ]
   add corr del desc emph head hi item l meeting note orig p q quote ref reg said sic sp stage title unclear\par 
    \item[figures: ]
   cell figDesc figure\par 
    \item[header: ]
   abstract change handNote licence rendition scriptNote sourceDesc tagUsage typeNote\par 
    \item[linking: ]
   ab seg\par 
    \item[msdescription: ]
   accMat acquisition additions collation condition custEvent decoNote filiation foliation layout musicNotation origin provenance signatures source summary support surrogates\par 
    \item[namesdates: ]
   listOrg occupation org\par 
    \item[textcrit: ]
   lem rdg witness\par 
    \item[textstructure: ]
   argument back body div docEdition epigraph imprimatur postscript salute signed titlePart trailer\par 
    \item[transcr: ]
   damage metamark mod restore retrace secl supplied surplus
    \item[{May contain}]
  
    \item[core: ]
   head\par 
    \item[namesdates: ]
   listOrg listRelation org relation
    \item[{Note}]
  \par
The type attribute may be used to distinguish lists of organizations of a particular type if convenient.
    \item[{Example}]
  \leavevmode\bgroup\exampleFont \begin{shaded}\noindent\mbox{}{<\textbf{listOrg}>}\mbox{}\newline 
\hspace*{6pt}{<\textbf{head}>}Libyans{</\textbf{head}>}\mbox{}\newline 
\hspace*{6pt}{<\textbf{org}>}\mbox{}\newline 
\hspace*{6pt}\hspace*{6pt}{<\textbf{orgName}>}Adyrmachidae{</\textbf{orgName}>}\mbox{}\newline 
\hspace*{6pt}\hspace*{6pt}{<\textbf{desc}>}These people have, in most points, the same customs as the Egyptians, but\mbox{}\newline 
\hspace*{6pt}\hspace*{6pt}\hspace*{6pt}\hspace*{6pt} use the costume of the Libyans. Their women wear on each leg a ring made of\mbox{}\newline 
\hspace*{6pt}\hspace*{6pt}\hspace*{6pt}\hspace*{6pt} bronze [...]{</\textbf{desc}>}\mbox{}\newline 
\hspace*{6pt}{</\textbf{org}>}\mbox{}\newline 
\hspace*{6pt}{<\textbf{org}>}\mbox{}\newline 
\hspace*{6pt}\hspace*{6pt}{<\textbf{orgName}>}Nasamonians{</\textbf{orgName}>}\mbox{}\newline 
\hspace*{6pt}\hspace*{6pt}{<\textbf{desc}>}In summer they leave their flocks and herds upon the sea-shore, and go up\mbox{}\newline 
\hspace*{6pt}\hspace*{6pt}\hspace*{6pt}\hspace*{6pt} the country to a place called Augila, where they gather the dates from the\mbox{}\newline 
\hspace*{6pt}\hspace*{6pt}\hspace*{6pt}\hspace*{6pt} palms [...]{</\textbf{desc}>}\mbox{}\newline 
\hspace*{6pt}{</\textbf{org}>}\mbox{}\newline 
\hspace*{6pt}{<\textbf{org}>}\mbox{}\newline 
\hspace*{6pt}\hspace*{6pt}{<\textbf{orgName}>}Garamantians{</\textbf{orgName}>}\mbox{}\newline 
\hspace*{6pt}\hspace*{6pt}{<\textbf{desc}>}[...] avoid all society or intercourse with their fellow-men, have no\mbox{}\newline 
\hspace*{6pt}\hspace*{6pt}\hspace*{6pt}\hspace*{6pt} weapon of war, and do not know how to defend themselves. [...]{</\textbf{desc}>}\mbox{}\newline 
\textit{<!-- ... -->}\mbox{}\newline 
\hspace*{6pt}{</\textbf{org}>}\mbox{}\newline 
{</\textbf{listOrg}>}\end{shaded}\egroup 


    \item[{Content model}]
  \mbox{}\hfill\\[-10pt]\begin{Verbatim}[fontsize=\small]
<content>
 <sequence>
  <classRef key="model.headLike"
   maxOccurs="unbounded" minOccurs="0"/>
  <alternate maxOccurs="unbounded"
   minOccurs="1">
   <elementRef key="org"/>
   <elementRef key="listOrg"/>
  </alternate>
  <alternate maxOccurs="unbounded"
   minOccurs="0">
   <elementRef key="relation"/>
   <elementRef key="listRelation"/>
  </alternate>
 </sequence>
</content>
    
\end{Verbatim}

    \item[{Schema Declaration}]
  \mbox{}\hfill\\[-10pt]\begin{Verbatim}[fontsize=\small]
element listOrg
{
   att.global.attributes,
   att.typed.attributes,
   att.declarable.attributes,
   att.sortable.attributes,
   ( model.headLike*, ( org | listOrg )+, ( relation | listRelation )* )
}
\end{Verbatim}

\end{reflist}  \index{listPerson=<listPerson>|oddindex}
\begin{reflist}
\item[]\begin{specHead}{TEI.listPerson}{<listPerson> }(list of persons) contains a list of descriptions, each of which provides information about an identifiable person or a group of people, for example the participants in a language interaction, or the people referred to in a historical source. [\xref{http://www.tei-c.org/release/doc/tei-p5-doc/en/html/ND.html\#NDPERSE}{13.3.2. The Person Element} \xref{http://www.tei-c.org/release/doc/tei-p5-doc/en/html/CC.html\#CCAH}{15.2. Contextual Information} \xref{http://www.tei-c.org/release/doc/tei-p5-doc/en/html/HD.html\#HD4}{2.4. The Profile Description} \xref{http://www.tei-c.org/release/doc/tei-p5-doc/en/html/CC.html\#CCAS2}{15.3.2. Declarable Elements}]\end{specHead} 
    \item[{Module}]
  namesdates
    \item[{Attributes}]
  Attributes att.global (\textit{@xml:id}, \textit{@n}, \textit{@xml:lang}, \textit{@xml:base}, \textit{@xml:space})  (att.global.rendition (\textit{@rend}, \textit{@style}, \textit{@rendition})) (att.global.linking (\textit{@corresp}, \textit{@synch}, \textit{@sameAs}, \textit{@copyOf}, \textit{@next}, \textit{@prev}, \textit{@exclude}, \textit{@select})) (att.global.analytic (\textit{@ana})) (att.global.facs (\textit{@facs})) (att.global.change (\textit{@change})) (att.global.responsibility (\textit{@cert}, \textit{@resp})) (att.global.source (\textit{@source})) att.typed (\textit{@type}, \textit{@subtype}) att.declarable (\textit{@default}) att.sortable (\textit{@sortKey}) 
    \item[{Member of}]
  model.listLike model.orgPart
    \item[{Contained by}]
  
    \item[core: ]
   add corr del desc emph head hi item l meeting note orig p q quote ref reg said sic sp stage title unclear\par 
    \item[figures: ]
   cell figDesc figure\par 
    \item[header: ]
   abstract change handNote licence rendition scriptNote sourceDesc tagUsage typeNote\par 
    \item[linking: ]
   ab seg\par 
    \item[msdescription: ]
   accMat acquisition additions collation condition custEvent decoNote filiation foliation layout musicNotation origin provenance signatures source summary support surrogates\par 
    \item[namesdates: ]
   listPerson occupation org\par 
    \item[textcrit: ]
   lem rdg witness\par 
    \item[textstructure: ]
   argument back body div docEdition epigraph imprimatur postscript salute signed titlePart trailer\par 
    \item[transcr: ]
   damage metamark mod restore retrace secl supplied surplus
    \item[{May contain}]
  
    \item[core: ]
   head\par 
    \item[namesdates: ]
   listPerson listRelation org person personGrp relation
    \item[{Note}]
  \par
The type attribute may be used to distinguish lists of people of a particular type if convenient.
    \item[{Example}]
  \leavevmode\bgroup\exampleFont \begin{shaded}\noindent\mbox{}{<\textbf{listPerson}\hspace*{6pt}{type}="{respondents}">}\mbox{}\newline 
\hspace*{6pt}{<\textbf{personGrp}\hspace*{6pt}{xml:id}="{PXXX}"/>}\mbox{}\newline 
\hspace*{6pt}{<\textbf{person}\hspace*{6pt}{age}="{mid}"\hspace*{6pt}{sex}="{2}"\hspace*{6pt}{xml:id}="{P1234}"/>}\mbox{}\newline 
\hspace*{6pt}{<\textbf{person}\hspace*{6pt}{age}="{mid}"\hspace*{6pt}{sex}="{1}"\hspace*{6pt}{xml:id}="{P4332}"/>}\mbox{}\newline 
\hspace*{6pt}{<\textbf{listRelation}>}\mbox{}\newline 
\hspace*{6pt}\hspace*{6pt}{<\textbf{relation}\hspace*{6pt}{mutual}="{\#P1234 \#P4332}"\mbox{}\newline 
\hspace*{6pt}\hspace*{6pt}\hspace*{6pt}{name}="{spouse}"\hspace*{6pt}{type}="{personal}"/>}\mbox{}\newline 
\hspace*{6pt}{</\textbf{listRelation}>}\mbox{}\newline 
{</\textbf{listPerson}>}\end{shaded}\egroup 


    \item[{Content model}]
  \mbox{}\hfill\\[-10pt]\begin{Verbatim}[fontsize=\small]
<content>
 <sequence>
  <classRef key="model.headLike"
   maxOccurs="unbounded" minOccurs="0"/>
  <alternate maxOccurs="unbounded"
   minOccurs="1">
   <classRef key="model.personLike"/>
   <elementRef key="listPerson"/>
  </alternate>
  <alternate maxOccurs="unbounded"
   minOccurs="0">
   <elementRef key="relation"/>
   <elementRef key="listRelation"/>
  </alternate>
 </sequence>
</content>
    
\end{Verbatim}

    \item[{Schema Declaration}]
  \mbox{}\hfill\\[-10pt]\begin{Verbatim}[fontsize=\small]
element listPerson
{
   att.global.attributes,
   att.typed.attributes,
   att.declarable.attributes,
   att.sortable.attributes,
   (
      model.headLike*,
      ( model.personLike | listPerson )+,
      ( relation | listRelation )*
   )
}
\end{Verbatim}

\end{reflist}  \index{listPlace=<listPlace>|oddindex}
\begin{reflist}
\item[]\begin{specHead}{TEI.listPlace}{<listPlace> }(list of places) contains a list of places, optionally followed by a list of relationships (other than containment) defined amongst them. [\xref{http://www.tei-c.org/release/doc/tei-p5-doc/en/html/HD.html\#HD3}{2.2.7. The Source Description} \xref{http://www.tei-c.org/release/doc/tei-p5-doc/en/html/ND.html\#NDGEOG}{13.3.4. Places}]\end{specHead} 
    \item[{Module}]
  namesdates
    \item[{Attributes}]
  Attributes att.global (\textit{@xml:id}, \textit{@n}, \textit{@xml:lang}, \textit{@xml:base}, \textit{@xml:space})  (att.global.rendition (\textit{@rend}, \textit{@style}, \textit{@rendition})) (att.global.linking (\textit{@corresp}, \textit{@synch}, \textit{@sameAs}, \textit{@copyOf}, \textit{@next}, \textit{@prev}, \textit{@exclude}, \textit{@select})) (att.global.analytic (\textit{@ana})) (att.global.facs (\textit{@facs})) (att.global.change (\textit{@change})) (att.global.responsibility (\textit{@cert}, \textit{@resp})) (att.global.source (\textit{@source})) att.typed (\textit{@type}, \textit{@subtype}) att.declarable (\textit{@default}) att.sortable (\textit{@sortKey}) 
    \item[{Member of}]
  model.listLike model.orgPart 
    \item[{Contained by}]
  
    \item[core: ]
   add corr del desc emph head hi item l meeting note orig p q quote ref reg said sic sp stage title unclear\par 
    \item[figures: ]
   cell figDesc figure\par 
    \item[header: ]
   abstract change handNote licence rendition scriptNote sourceDesc tagUsage typeNote\par 
    \item[linking: ]
   ab seg\par 
    \item[msdescription: ]
   accMat acquisition additions collation condition custEvent decoNote filiation foliation layout musicNotation origin provenance signatures source summary support surrogates\par 
    \item[namesdates: ]
   listPlace occupation org place\par 
    \item[textcrit: ]
   lem rdg witness\par 
    \item[textstructure: ]
   argument back body div docEdition epigraph imprimatur postscript salute signed titlePart trailer\par 
    \item[transcr: ]
   damage metamark mod restore retrace secl supplied surplus
    \item[{May contain}]
  
    \item[core: ]
   head\par 
    \item[namesdates: ]
   listPlace listRelation place relation
    \item[{Example}]
  \leavevmode\bgroup\exampleFont \begin{shaded}\noindent\mbox{}{<\textbf{listPlace}\hspace*{6pt}{type}="{offshoreIslands}">}\mbox{}\newline 
\hspace*{6pt}{<\textbf{place}>}\mbox{}\newline 
\hspace*{6pt}\hspace*{6pt}{<\textbf{placeName}>}La roche qui pleure{</\textbf{placeName}>}\mbox{}\newline 
\hspace*{6pt}{</\textbf{place}>}\mbox{}\newline 
\hspace*{6pt}{<\textbf{place}>}\mbox{}\newline 
\hspace*{6pt}\hspace*{6pt}{<\textbf{placeName}>}Ile aux cerfs{</\textbf{placeName}>}\mbox{}\newline 
\hspace*{6pt}{</\textbf{place}>}\mbox{}\newline 
{</\textbf{listPlace}>}\end{shaded}\egroup 


    \item[{Content model}]
  \mbox{}\hfill\\[-10pt]\begin{Verbatim}[fontsize=\small]
<content>
 <sequence>
  <classRef key="model.headLike"
   maxOccurs="unbounded" minOccurs="0"/>
  <alternate maxOccurs="unbounded"
   minOccurs="1">
   <classRef key="model.placeLike"/>
   <elementRef key="listPlace"/>
  </alternate>
  <alternate maxOccurs="unbounded"
   minOccurs="0">
   <elementRef key="relation"/>
   <elementRef key="listRelation"/>
  </alternate>
 </sequence>
</content>
    
\end{Verbatim}

    \item[{Schema Declaration}]
  \mbox{}\hfill\\[-10pt]\begin{Verbatim}[fontsize=\small]
element listPlace
{
   att.global.attributes,
   att.typed.attributes,
   att.declarable.attributes,
   att.sortable.attributes,
   (
      model.headLike*,
      ( model.placeLike | listPlace )+,
      ( relation | listRelation )*
   )
}
\end{Verbatim}

\end{reflist}  \index{listPrefixDef=<listPrefixDef>|oddindex}
\begin{reflist}
\item[]\begin{specHead}{TEI.listPrefixDef}{<listPrefixDef> }(list of prefix definitions) contains a list of definitions of prefixing schemes used in \textsf{data.pointer} values, showing how abbreviated URIs using each scheme may be expanded into full URIs. [\xref{http://www.tei-c.org/release/doc/tei-p5-doc/en/html/SA.html\#SAPU}{16.2.3. Using Abbreviated Pointers}]\end{specHead} 
    \item[{Module}]
  header
    \item[{Attributes}]
  Attributes att.global (\textit{@xml:id}, \textit{@n}, \textit{@xml:lang}, \textit{@xml:base}, \textit{@xml:space})  (att.global.rendition (\textit{@rend}, \textit{@style}, \textit{@rendition})) (att.global.linking (\textit{@corresp}, \textit{@synch}, \textit{@sameAs}, \textit{@copyOf}, \textit{@next}, \textit{@prev}, \textit{@exclude}, \textit{@select})) (att.global.analytic (\textit{@ana})) (att.global.facs (\textit{@facs})) (att.global.change (\textit{@change})) (att.global.responsibility (\textit{@cert}, \textit{@resp})) (att.global.source (\textit{@source}))
    \item[{Member of}]
  model.encodingDescPart
    \item[{Contained by}]
  
    \item[header: ]
   encodingDesc listPrefixDef
    \item[{May contain}]
  
    \item[header: ]
   listPrefixDef prefixDef
    \item[{Example}]
  In this example, two private URI scheme prefixes are defined and patterns are provided for dereferencing them. Each prefix is also supplied with a human-readable explanation in a <p> element.\leavevmode\bgroup\exampleFont \begin{shaded}\noindent\mbox{}{<\textbf{listPrefixDef}>}\mbox{}\newline 
\hspace*{6pt}{<\textbf{prefixDef}\hspace*{6pt}{ident}="{psn}"\mbox{}\newline 
\hspace*{6pt}\hspace*{6pt}{matchPattern}="{([A-Z]+)}"\mbox{}\newline 
\hspace*{6pt}\hspace*{6pt}{replacementPattern}="{personography.xml\#\$1}">}\mbox{}\newline 
\hspace*{6pt}\hspace*{6pt}{<\textbf{p}>} Private URIs using the {<\textbf{code}>}psn{</\textbf{code}>} \mbox{}\newline 
\hspace*{6pt}\hspace*{6pt}\hspace*{6pt}\hspace*{6pt} prefix are pointers to {<\textbf{gi}>}person{</\textbf{gi}>} \mbox{}\newline 
\hspace*{6pt}\hspace*{6pt}\hspace*{6pt}\hspace*{6pt} elements in the personography.xml file.\mbox{}\newline 
\hspace*{6pt}\hspace*{6pt}\hspace*{6pt}\hspace*{6pt} For example, {<\textbf{code}>}psn:MDH{</\textbf{code}>} \mbox{}\newline 
\hspace*{6pt}\hspace*{6pt}\hspace*{6pt}\hspace*{6pt} dereferences to {<\textbf{code}>}personography.xml\#MDH{</\textbf{code}>}.\mbox{}\newline 
\hspace*{6pt}\hspace*{6pt}{</\textbf{p}>}\mbox{}\newline 
\hspace*{6pt}{</\textbf{prefixDef}>}\mbox{}\newline 
\hspace*{6pt}{<\textbf{prefixDef}\hspace*{6pt}{ident}="{bibl}"\mbox{}\newline 
\hspace*{6pt}\hspace*{6pt}{matchPattern}="{([a-z]+[a-z0-9]*)}"\mbox{}\newline 
\hspace*{6pt}\hspace*{6pt}{replacementPattern}="{http://www.example.com/getBibl.xql?id=\$1}">}\mbox{}\newline 
\hspace*{6pt}\hspace*{6pt}{<\textbf{p}>} Private URIs using the {<\textbf{code}>}bibl{</\textbf{code}>} prefix can be\mbox{}\newline 
\hspace*{6pt}\hspace*{6pt}\hspace*{6pt}\hspace*{6pt} expanded to form URIs which retrieve the relevant\mbox{}\newline 
\hspace*{6pt}\hspace*{6pt}\hspace*{6pt}\hspace*{6pt} bibliographical reference from www.example.com.\mbox{}\newline 
\hspace*{6pt}\hspace*{6pt}{</\textbf{p}>}\mbox{}\newline 
\hspace*{6pt}{</\textbf{prefixDef}>}\mbox{}\newline 
{</\textbf{listPrefixDef}>}\end{shaded}\egroup 


    \item[{Content model}]
  \mbox{}\hfill\\[-10pt]\begin{Verbatim}[fontsize=\small]
<content>
 <alternate maxOccurs="unbounded"
  minOccurs="1">
  <elementRef key="prefixDef"/>
  <elementRef key="listPrefixDef"/>
 </alternate>
</content>
    
\end{Verbatim}

    \item[{Schema Declaration}]
  \mbox{}\hfill\\[-10pt]\begin{Verbatim}[fontsize=\small]
element listPrefixDef { att.global.attributes, ( prefixDef | listPrefixDef )+ }
\end{Verbatim}

\end{reflist}  \index{listRelation=<listRelation>|oddindex}
\begin{reflist}
\item[]\begin{specHead}{TEI.listRelation}{<listRelation> }provides information about relationships identified amongst people, places, and organizations, either informally as prose or as formally expressed relation links. [\xref{http://www.tei-c.org/release/doc/tei-p5-doc/en/html/ND.html\#NDPERSREL}{13.3.2.3. Personal Relationships}]\end{specHead} 
    \item[{Module}]
  namesdates
    \item[{Attributes}]
  Attributes att.global (\textit{@xml:id}, \textit{@n}, \textit{@xml:lang}, \textit{@xml:base}, \textit{@xml:space})  (att.global.rendition (\textit{@rend}, \textit{@style}, \textit{@rendition})) (att.global.linking (\textit{@corresp}, \textit{@synch}, \textit{@sameAs}, \textit{@copyOf}, \textit{@next}, \textit{@prev}, \textit{@exclude}, \textit{@select})) (att.global.analytic (\textit{@ana})) (att.global.facs (\textit{@facs})) (att.global.change (\textit{@change})) (att.global.responsibility (\textit{@cert}, \textit{@resp})) (att.global.source (\textit{@source})) att.typed (\textit{@type}, \textit{@subtype}) att.sortable (\textit{@sortKey}) 
    \item[{Member of}]
  model.biblPart
    \item[{Contained by}]
  
    \item[core: ]
   bibl listBibl\par 
    \item[namesdates: ]
   listEvent listNym listOrg listPerson listPlace listRelation
    \item[{May contain}]
  
    \item[core: ]
   head p\par 
    \item[linking: ]
   ab\par 
    \item[namesdates: ]
   listRelation relation
    \item[{Note}]
  \par
May contain a prose description organized as paragraphs, or a sequence of <relation> elements.
    \item[{Example}]
  \leavevmode\bgroup\exampleFont \begin{shaded}\noindent\mbox{}{<\textbf{listPerson}>}\mbox{}\newline 
\hspace*{6pt}{<\textbf{person}\hspace*{6pt}{xml:id}="{pp1}">}\mbox{}\newline 
\textit{<!-- data about person pp1 -->}\mbox{}\newline 
\hspace*{6pt}{</\textbf{person}>}\mbox{}\newline 
\hspace*{6pt}{<\textbf{person}\hspace*{6pt}{xml:id}="{pp2}">}\mbox{}\newline 
\textit{<!-- data about person pp1 -->}\mbox{}\newline 
\hspace*{6pt}{</\textbf{person}>}\mbox{}\newline 
\textit{<!-- more person (pp3, pp4) elements here -->}\mbox{}\newline 
{</\textbf{listPerson}>}\mbox{}\newline 
{<\textbf{listRelation}\hspace*{6pt}{type}="{personal}">}\mbox{}\newline 
\hspace*{6pt}{<\textbf{relation}\hspace*{6pt}{active}="{\#pp1 \#pp2}"\hspace*{6pt}{name}="{parent}"\mbox{}\newline 
\hspace*{6pt}\hspace*{6pt}{passive}="{\#pp3 \#pp4}"/>}\mbox{}\newline 
\hspace*{6pt}{<\textbf{relation}\hspace*{6pt}{mutual}="{\#pp1 \#pp2}"\hspace*{6pt}{name}="{spouse}"/>}\mbox{}\newline 
{</\textbf{listRelation}>}\mbox{}\newline 
{<\textbf{listRelation}\hspace*{6pt}{type}="{social}">}\mbox{}\newline 
\hspace*{6pt}{<\textbf{relation}\hspace*{6pt}{active}="{\#pp1}"\hspace*{6pt}{name}="{employer}"\mbox{}\newline 
\hspace*{6pt}\hspace*{6pt}{passive}="{\#pp3 \#pp5 \#pp6 \#pp7}"/>}\mbox{}\newline 
{</\textbf{listRelation}>}\end{shaded}\egroup 

The persons with identifiers pp1 and p2 are the parents of pp3 and pp4; they are also married to each other; pp1 is the employer of pp3, pp5, pp6, and pp7.
    \item[{Example}]
  \leavevmode\bgroup\exampleFont \begin{shaded}\noindent\mbox{}{<\textbf{listRelation}>}\mbox{}\newline 
\hspace*{6pt}{<\textbf{p}>}All speakers are members of the Ceruli family, born in Naples.{</\textbf{p}>}\mbox{}\newline 
{</\textbf{listRelation}>}\end{shaded}\egroup 


    \item[{Content model}]
  \mbox{}\hfill\\[-10pt]\begin{Verbatim}[fontsize=\small]
<content>
 <sequence>
  <classRef key="model.headLike"
   maxOccurs="unbounded" minOccurs="0"/>
  <alternate>
   <classRef key="model.pLike"/>
   <alternate maxOccurs="unbounded"
    minOccurs="1">
    <elementRef key="relation"/>
    <elementRef key="listRelation"/>
   </alternate>
  </alternate>
 </sequence>
</content>
    
\end{Verbatim}

    \item[{Schema Declaration}]
  \mbox{}\hfill\\[-10pt]\begin{Verbatim}[fontsize=\small]
element listRelation
{
   att.global.attributes,
   att.typed.attributes,
   att.sortable.attributes,
   ( model.headLike*, ( model.pLike | ( relation | listRelation )+ ) )
}
\end{Verbatim}

\end{reflist}  \index{listTranspose=<listTranspose>|oddindex}
\begin{reflist}
\item[]\begin{specHead}{TEI.listTranspose}{<listTranspose> }supplies a list of transpositions, each of which is indicated at some point in a document typically by means of metamarks. [\xref{http://www.tei-c.org/release/doc/tei-p5-doc/en/html/PH.html\#transpo}{11.3.4.5. Transpositions}]\end{specHead} 
    \item[{Module}]
  transcr
    \item[{Attributes}]
  Attributes att.global (\textit{@xml:id}, \textit{@n}, \textit{@xml:lang}, \textit{@xml:base}, \textit{@xml:space})  (att.global.rendition (\textit{@rend}, \textit{@style}, \textit{@rendition})) (att.global.linking (\textit{@corresp}, \textit{@synch}, \textit{@sameAs}, \textit{@copyOf}, \textit{@next}, \textit{@prev}, \textit{@exclude}, \textit{@select})) (att.global.analytic (\textit{@ana})) (att.global.facs (\textit{@facs})) (att.global.change (\textit{@change})) (att.global.responsibility (\textit{@cert}, \textit{@resp})) (att.global.source (\textit{@source}))
    \item[{Member of}]
  model.global.meta model.profileDescPart
    \item[{Contained by}]
  
    \item[analysis: ]
   cl m phr s span w\par 
    \item[core: ]
   abbr add addrLine address author bibl biblScope cit citedRange corr date del distinct editor email emph expan foreign gloss head headItem headLabel hi imprint item l label lg list measure mentioned name note num orig p pubPlace publisher q quote ref reg resp rs said series sic soCalled sp speaker stage street term textLang time title unclear\par 
    \item[figures: ]
   cell figure table\par 
    \item[header: ]
   authority change classCode distributor edition extent funder geoDecl handNote language licence principal profileDesc scriptNote sponsor typeNote\par 
    \item[linking: ]
   ab seg\par 
    \item[msdescription: ]
   accMat acquisition additions catchwords collation colophon condition custEvent decoNote explicit filiation finalRubric foliation heraldry incipit layout material msItem musicNotation objectType origDate origPlace origin provenance rubric secFol signatures source stamp summary support surrogates watermark\par 
    \item[namesdates: ]
   addName affiliation age birth bloc country death district education faith floruit forename genName geogFeat geogName langKnown nameLink nationality occupation offset orgName persName person personGrp placeName region residence roleName settlement sex socecStatus surname\par 
    \item[textcrit: ]
   lem rdg wit witDetail\par 
    \item[textstructure: ]
   argument back body byline closer dateline div docAuthor docDate docEdition docImprint docTitle epigraph floatingText front group imprimatur opener postscript salute signed text titlePage titlePart trailer\par 
    \item[transcr: ]
   damage fw line metamark mod restore retrace secl sourceDoc supplied surface surfaceGrp surplus zone
    \item[{May contain}]
  
    \item[transcr: ]
   transpose
    \item[{Example}]
  \leavevmode\bgroup\exampleFont \begin{shaded}\noindent\mbox{}{<\textbf{listTranspose}>}\mbox{}\newline 
\hspace*{6pt}{<\textbf{transpose}>}\mbox{}\newline 
\hspace*{6pt}\hspace*{6pt}{<\textbf{ptr}\hspace*{6pt}{target}="{\#ib02}"/>}\mbox{}\newline 
\hspace*{6pt}\hspace*{6pt}{<\textbf{ptr}\hspace*{6pt}{target}="{\#ib01}"/>}\mbox{}\newline 
\hspace*{6pt}{</\textbf{transpose}>}\mbox{}\newline 
{</\textbf{listTranspose}>}\end{shaded}\egroup 

This example might be used for a source document which indicates in some way that the elements identified by \texttt{ib02} and code \texttt{ib01} should be read in that order (ib02 followed by ib01), rather than in the reading order in which they are presented in the source.
    \item[{Content model}]
  \mbox{}\hfill\\[-10pt]\begin{Verbatim}[fontsize=\small]
<content>
 <elementRef key="transpose"
  maxOccurs="unbounded" minOccurs="1"/>
</content>
    
\end{Verbatim}

    \item[{Schema Declaration}]
  \mbox{}\hfill\\[-10pt]\begin{Verbatim}[fontsize=\small]
element listTranspose { att.global.attributes, transpose+ }
\end{Verbatim}

\end{reflist}  \index{listWit=<listWit>|oddindex}
\begin{reflist}
\item[]\begin{specHead}{TEI.listWit}{<listWit> }(witness list) lists definitions for all the witnesses referred to by a critical apparatus, optionally grouped hierarchically. [\xref{http://www.tei-c.org/release/doc/tei-p5-doc/en/html/TC.html\#TCAPLL}{12.1. The Apparatus Entry, Readings, and Witnesses}]\end{specHead} 
    \item[{Module}]
  textcrit
    \item[{Attributes}]
  Attributes att.global (\textit{@xml:id}, \textit{@n}, \textit{@xml:lang}, \textit{@xml:base}, \textit{@xml:space})  (att.global.rendition (\textit{@rend}, \textit{@style}, \textit{@rendition})) (att.global.linking (\textit{@corresp}, \textit{@synch}, \textit{@sameAs}, \textit{@copyOf}, \textit{@next}, \textit{@prev}, \textit{@exclude}, \textit{@select})) (att.global.analytic (\textit{@ana})) (att.global.facs (\textit{@facs})) (att.global.change (\textit{@change})) (att.global.responsibility (\textit{@cert}, \textit{@resp})) (att.global.source (\textit{@source})) att.sortable (\textit{@sortKey}) 
    \item[{Member of}]
  model.listLike
    \item[{Contained by}]
  
    \item[core: ]
   add corr del desc emph head hi item l meeting note orig p q quote ref reg said sic sp stage title unclear\par 
    \item[figures: ]
   cell figDesc figure\par 
    \item[header: ]
   abstract change handNote licence rendition scriptNote sourceDesc tagUsage typeNote\par 
    \item[linking: ]
   ab seg\par 
    \item[msdescription: ]
   accMat acquisition additions collation condition custEvent decoNote filiation foliation layout musicNotation origin provenance signatures source summary support surrogates\par 
    \item[namesdates: ]
   occupation\par 
    \item[textcrit: ]
   lem listWit rdg witness\par 
    \item[textstructure: ]
   argument back body div docEdition epigraph imprimatur postscript salute signed titlePart trailer\par 
    \item[transcr: ]
   damage metamark mod restore retrace secl supplied surplus
    \item[{May contain}]
  
    \item[core: ]
   head\par 
    \item[textcrit: ]
   listWit witness
    \item[{Note}]
  \par
May contain a series of <witness> or <listWit> elements.\par
The provision of a <listWit> element simplifies the automatic processing of the apparatus, e.g. the reconstruction of the readings for all witnesses from an exhaustive apparatus.\par
Situations commonly arise where there are many more or less fragmentary witnesses, such that there may be quite distinct groups of witnesses for different parts of a text or collection of texts. Such groups may be given separately, or nested within a single <listWit> element at the beginning of the file listing all the witnesses, partial and complete, for the text, with the attestation of fragmentary witnesses indicated within the apparatus by use of the <witStart> and <witEnd> elements described in section \xref{http://www.tei-c.org/release/doc/tei-p5-doc/en/html/TC.html\#TCAPMI}{12.1.5. Fragmentary Witnesses}.\par
Note however that a given witness can only be defined once, and can therefore only appear within a single <listWit> element.
    \item[{Example}]
  \leavevmode\bgroup\exampleFont \begin{shaded}\noindent\mbox{}{<\textbf{listWit}>}\mbox{}\newline 
\hspace*{6pt}{<\textbf{witness}\hspace*{6pt}{xml:id}="{HL26}">}Ellesmere, Huntingdon Library 26.C.9{</\textbf{witness}>}\mbox{}\newline 
\hspace*{6pt}{<\textbf{witness}\hspace*{6pt}{xml:id}="{PN392}">}Hengwrt, National Library of Wales,\mbox{}\newline 
\hspace*{6pt}\hspace*{6pt} Aberystwyth, Peniarth 392D{</\textbf{witness}>}\mbox{}\newline 
\hspace*{6pt}{<\textbf{witness}\hspace*{6pt}{xml:id}="{RP149}">}Bodleian Library Rawlinson Poetic 149\mbox{}\newline 
\hspace*{6pt}\hspace*{6pt} (see further {<\textbf{ptr}\hspace*{6pt}{target}="{\#MSRP149}"/>}){</\textbf{witness}>}\mbox{}\newline 
{</\textbf{listWit}>}\end{shaded}\egroup 


    \item[{Content model}]
  \mbox{}\hfill\\[-10pt]\begin{Verbatim}[fontsize=\small]
<content>
 <sequence>
  <classRef key="model.headLike"
   minOccurs="0"/>
  <alternate maxOccurs="unbounded"
   minOccurs="1">
   <elementRef key="witness"/>
   <elementRef key="listWit"/>
  </alternate>
 </sequence>
</content>
    
\end{Verbatim}

    \item[{Schema Declaration}]
  \mbox{}\hfill\\[-10pt]\begin{Verbatim}[fontsize=\small]
element listWit
{
   att.global.attributes,
   att.sortable.attributes,
   ( model.headLike?, ( witness | listWit )+ )
}
\end{Verbatim}

\end{reflist}  \index{localName=<localName>|oddindex}
\begin{reflist}
\item[]\begin{specHead}{TEI.localName}{<localName> }(locally-defined property name) contains a locally defined name for some property. [\xref{http://www.tei-c.org/release/doc/tei-p5-doc/en/html/WD.html\#ucsprops}{5.2.1. Character Properties}]\end{specHead} 
    \item[{Module}]
  gaiji
    \item[{Attributes}]
  Attributes att.global (\textit{@xml:id}, \textit{@n}, \textit{@xml:lang}, \textit{@xml:base}, \textit{@xml:space})  (att.global.rendition (\textit{@rend}, \textit{@style}, \textit{@rendition})) (att.global.linking (\textit{@corresp}, \textit{@synch}, \textit{@sameAs}, \textit{@copyOf}, \textit{@next}, \textit{@prev}, \textit{@exclude}, \textit{@select})) (att.global.analytic (\textit{@ana})) (att.global.facs (\textit{@facs})) (att.global.change (\textit{@change})) (att.global.responsibility (\textit{@cert}, \textit{@resp})) (att.global.source (\textit{@source}))
    \item[{Contained by}]
  
    \item[gaiji: ]
   charProp
    \item[{May contain}]
  Character data only
    \item[{Note}]
  \par
No definitive list of local names is proposed. However, the name \textsf{entity} is recommended as a means of naming the property identifying the recommended character entity name for this character or glyph.
    \item[{Example}]
  \leavevmode\bgroup\exampleFont \begin{shaded}\noindent\mbox{}{<\textbf{localName}>}daikanwa{</\textbf{localName}>}\mbox{}\newline 
{<\textbf{localName}>}entity{</\textbf{localName}>}\end{shaded}\egroup 


    \item[{Content model}]
  \fbox{\ttfamily <content>\newline
 <textNode/>\newline
</content>\newline
    } 
    \item[{Schema Declaration}]
  \fbox{\ttfamily element localName ❴ att.global.attributes, text ❵} 
\end{reflist}  \index{location=<location>|oddindex}
\begin{reflist}
\item[]\begin{specHead}{TEI.location}{<location> }defines the location of a place as a set of geographical coordinates, in terms of other named geo-political entities, or as an address. [\xref{http://www.tei-c.org/release/doc/tei-p5-doc/en/html/ND.html\#NDGEOG}{13.3.4. Places}]\end{specHead} 
    \item[{Module}]
  namesdates
    \item[{Attributes}]
  Attributes att.global (\textit{@xml:id}, \textit{@n}, \textit{@xml:lang}, \textit{@xml:base}, \textit{@xml:space})  (att.global.rendition (\textit{@rend}, \textit{@style}, \textit{@rendition})) (att.global.linking (\textit{@corresp}, \textit{@synch}, \textit{@sameAs}, \textit{@copyOf}, \textit{@next}, \textit{@prev}, \textit{@exclude}, \textit{@select})) (att.global.analytic (\textit{@ana})) (att.global.facs (\textit{@facs})) (att.global.change (\textit{@change})) (att.global.responsibility (\textit{@cert}, \textit{@resp})) (att.global.source (\textit{@source})) att.typed (\textit{@type}, \textit{@subtype}) att.datable (\textit{@calendar}, \textit{@period})  (att.datable.w3c (\textit{@when}, \textit{@notBefore}, \textit{@notAfter}, \textit{@from}, \textit{@to})) (att.datable.iso (\textit{@when-iso}, \textit{@notBefore-iso}, \textit{@notAfter-iso}, \textit{@from-iso}, \textit{@to-iso})) (att.datable.custom (\textit{@when-custom}, \textit{@notBefore-custom}, \textit{@notAfter-custom}, \textit{@from-custom}, \textit{@to-custom}, \textit{@datingPoint}, \textit{@datingMethod})) att.editLike (\textit{@evidence}, \textit{@instant})  (att.dimensions (\textit{@unit}, \textit{@quantity}, \textit{@extent}, \textit{@precision}, \textit{@scope}) (att.ranging (\textit{@atLeast}, \textit{@atMost}, \textit{@min}, \textit{@max}, \textit{@confidence})) )
    \item[{Member of}]
  model.placeStateLike
    \item[{Contained by}]
  
    \item[analysis: ]
   cl phr s span\par 
    \item[core: ]
   abbr add addrLine address author bibl biblScope citedRange corr date del desc distinct editor email emph expan foreign gloss head headItem headLabel hi item l label measure meeting mentioned name note num orig p pubPlace publisher q quote ref reg resp rs said sic soCalled speaker stage street term textLang time title unclear\par 
    \item[figures: ]
   cell figDesc\par 
    \item[header: ]
   authority catDesc change classCode correspAction creation distributor edition extent funder geoDecl handNote language licence principal rendition scriptNote sponsor tagUsage typeNote\par 
    \item[linking: ]
   ab seg\par 
    \item[msdescription: ]
   accMat acquisition additions catchwords collation colophon condition custEvent decoNote explicit filiation finalRubric foliation heraldry incipit layout material musicNotation objectType origDate origPlace origin provenance rubric secFol signatures source stamp summary support surrogates watermark\par 
    \item[namesdates: ]
   addName affiliation age birth bloc country death district education faith floruit forename genName geogFeat geogName langKnown nameLink nationality occupation offset org orgName persName place placeName region residence roleName settlement sex socecStatus surname\par 
    \item[textcrit: ]
   lem rdg wit witDetail witness\par 
    \item[textstructure: ]
   byline closer dateline docAuthor docDate docEdition docImprint imprimatur opener salute signed titlePart trailer\par 
    \item[transcr: ]
   damage fw metamark mod restore retrace secl supplied surplus
    \item[{May contain}]
  
    \item[core: ]
   address bibl biblStruct desc email label listBibl measure measureGrp note num\par 
    \item[header: ]
   biblFull\par 
    \item[msdescription: ]
   depth dim height msDesc width\par 
    \item[namesdates: ]
   affiliation bloc country district geo geogFeat geogName offset placeName region settlement\par 
    \item[textcrit: ]
   witDetail
    \item[{Example}]
  \leavevmode\bgroup\exampleFont \begin{shaded}\noindent\mbox{}{<\textbf{place}>}\mbox{}\newline 
\hspace*{6pt}{<\textbf{placeName}>}Abbey Dore{</\textbf{placeName}>}\mbox{}\newline 
\hspace*{6pt}{<\textbf{location}>}\mbox{}\newline 
\hspace*{6pt}\hspace*{6pt}{<\textbf{geo}>}51.969604 -2.893146{</\textbf{geo}>}\mbox{}\newline 
\hspace*{6pt}{</\textbf{location}>}\mbox{}\newline 
{</\textbf{place}>}\end{shaded}\egroup 


    \item[{Example}]
  \leavevmode\bgroup\exampleFont \begin{shaded}\noindent\mbox{}{<\textbf{place}\hspace*{6pt}{type}="{building}"\hspace*{6pt}{xml:id}="{BGbuilding}">}\mbox{}\newline 
\hspace*{6pt}{<\textbf{placeName}>}Brasserie Georges{</\textbf{placeName}>}\mbox{}\newline 
\hspace*{6pt}{<\textbf{location}>}\mbox{}\newline 
\hspace*{6pt}\hspace*{6pt}{<\textbf{country}\hspace*{6pt}{key}="{FR}"/>}\mbox{}\newline 
\hspace*{6pt}\hspace*{6pt}{<\textbf{settlement}\hspace*{6pt}{type}="{city}">}Lyon{</\textbf{settlement}>}\mbox{}\newline 
\hspace*{6pt}\hspace*{6pt}{<\textbf{district}\hspace*{6pt}{type}="{arrondissement}">}IIème{</\textbf{district}>}\mbox{}\newline 
\hspace*{6pt}\hspace*{6pt}{<\textbf{district}\hspace*{6pt}{type}="{quartier}">}Perrache{</\textbf{district}>}\mbox{}\newline 
\hspace*{6pt}\hspace*{6pt}{<\textbf{placeName}\hspace*{6pt}{type}="{street}">}\mbox{}\newline 
\hspace*{6pt}\hspace*{6pt}\hspace*{6pt}{<\textbf{num}>}30{</\textbf{num}>}, Cours de Verdun{</\textbf{placeName}>}\mbox{}\newline 
\hspace*{6pt}{</\textbf{location}>}\mbox{}\newline 
{</\textbf{place}>}\end{shaded}\egroup 


    \item[{Example}]
  \leavevmode\bgroup\exampleFont \begin{shaded}\noindent\mbox{}{<\textbf{place}\hspace*{6pt}{type}="{imaginary}">}\mbox{}\newline 
\hspace*{6pt}{<\textbf{placeName}>}Atlantis{</\textbf{placeName}>}\mbox{}\newline 
\hspace*{6pt}{<\textbf{location}>}\mbox{}\newline 
\hspace*{6pt}\hspace*{6pt}{<\textbf{offset}>}beyond{</\textbf{offset}>}\mbox{}\newline 
\hspace*{6pt}\hspace*{6pt}{<\textbf{placeName}>}The Pillars of {<\textbf{persName}>}Hercules{</\textbf{persName}>}\mbox{}\newline 
\hspace*{6pt}\hspace*{6pt}{</\textbf{placeName}>}\mbox{}\newline 
\hspace*{6pt}{</\textbf{location}>}\mbox{}\newline 
{</\textbf{place}>}\end{shaded}\egroup 


    \item[{Content model}]
  \mbox{}\hfill\\[-10pt]\begin{Verbatim}[fontsize=\small]
<content>
 <alternate maxOccurs="unbounded"
  minOccurs="0">
  <elementRef key="precision"/>
  <classRef key="model.labelLike"/>
  <classRef key="model.placeNamePart"/>
  <classRef key="model.offsetLike"/>
  <classRef key="model.measureLike"/>
  <classRef key="model.addressLike"/>
  <classRef key="model.noteLike"/>
  <classRef key="model.biblLike"/>
 </alternate>
</content>
    
\end{Verbatim}

    \item[{Schema Declaration}]
  \mbox{}\hfill\\[-10pt]\begin{Verbatim}[fontsize=\small]
element location
{
   att.global.attributes,
   att.typed.attributes,
   att.datable.attributes,
   att.editLike.attributes,
   (
      precision    | model.labelLike    | model.placeNamePart    | model.offsetLike    | model.measureLike    | model.addressLike    | model.noteLike    | model.biblLike   )*
}
\end{Verbatim}

\end{reflist}  \index{locus=<locus>|oddindex}\index{scheme=@scheme!<locus>|oddindex}\index{from=@from!<locus>|oddindex}\index{to=@to!<locus>|oddindex}
\begin{reflist}
\item[]\begin{specHead}{TEI.locus}{<locus> }defines a location within a manuscript or manuscript part, usually as a (possibly discontinuous) sequence of folio references. [\xref{http://www.tei-c.org/release/doc/tei-p5-doc/en/html/MS.html\#msloc}{10.3.5. References to Locations within a Manuscript}]\end{specHead} 
    \item[{Module}]
  msdescription
    \item[{Attributes}]
  Attributes att.global (\textit{@xml:id}, \textit{@n}, \textit{@xml:lang}, \textit{@xml:base}, \textit{@xml:space})  (att.global.rendition (\textit{@rend}, \textit{@style}, \textit{@rendition})) (att.global.linking (\textit{@corresp}, \textit{@synch}, \textit{@sameAs}, \textit{@copyOf}, \textit{@next}, \textit{@prev}, \textit{@exclude}, \textit{@select})) (att.global.analytic (\textit{@ana})) (att.global.facs (\textit{@facs})) (att.global.change (\textit{@change})) (att.global.responsibility (\textit{@cert}, \textit{@resp})) (att.global.source (\textit{@source})) att.pointing (\textit{@targetLang}, \textit{@target}, \textit{@evaluate}) \hfil\\[-10pt]\begin{sansreflist}
    \item[@scheme]
  identifies the foliation scheme in terms of which the location is being specified by pointing to some <foliation> element defining it, or to some other equivalent resource.
\begin{reflist}
    \item[{Status}]
  Optional
    \item[{Datatype}]
  teidata.pointer
\end{reflist}  
    \item[@from]
  specifies the starting point of the location in a normalized form, typically a page number.
\begin{reflist}
    \item[{Status}]
  Optional
    \item[{Datatype}]
  teidata.word
\end{reflist}  
    \item[@to]
  specifies the end-point of the location in a normalized form, typically as a page number.
\begin{reflist}
    \item[{Status}]
  Optional
    \item[{Datatype}]
  teidata.word
\end{reflist}  
\end{sansreflist}  
    \item[{Member of}]
  model.pPart.msdesc 
    \item[{Contained by}]
  
    \item[analysis: ]
   cl phr s span\par 
    \item[core: ]
   abbr add addrLine author biblScope citedRange corr date del desc distinct editor email emph expan foreign gloss head headItem headLabel hi item l label measure meeting mentioned name note num orig p pubPlace publisher q quote ref reg resp rs said sic soCalled speaker stage street term textLang time title unclear\par 
    \item[figures: ]
   cell figDesc\par 
    \item[header: ]
   authority catDesc change classCode creation distributor edition extent funder geoDecl handNote language licence principal rendition scriptNote sponsor tagUsage typeNote\par 
    \item[linking: ]
   ab seg\par 
    \item[msdescription: ]
   accMat acquisition additions catchwords collation colophon condition custEvent decoNote explicit filiation finalRubric foliation heraldry incipit layout locusGrp material msItem msItemStruct musicNotation objectType origDate origPlace origin provenance rubric secFol signatures source stamp summary support surrogates watermark\par 
    \item[namesdates: ]
   addName affiliation age birth bloc country death district education faith floruit forename genName geogFeat geogName langKnown nameLink nationality occupation offset orgName persName placeName region residence roleName settlement sex socecStatus surname\par 
    \item[textcrit: ]
   lem rdg wit witDetail witness\par 
    \item[textstructure: ]
   byline closer dateline docAuthor docDate docEdition docImprint imprimatur opener salute signed titlePart trailer\par 
    \item[transcr: ]
   damage fw metamark mod restore retrace secl supplied surplus
    \item[{May contain}]
  
    \item[gaiji: ]
   g\par character data
    \item[{Note}]
  \par
The {\itshape target} attribute should only be used to point to elements that contain or indicate a transcription of the locus being described, as in the first example above. To associate a <locus> element with a page image or other comparable representation, the global {\itshape facs} attribute should be used instead, as shown in the second example. Use of the {\itshape target} attribute to indicate an image is strongly deprecated. The {\itshape facs} attribute may be used to indicate one or more image files, as above, or alternatively it may point to one or more appropriate XML elements, such as the <surface>, <zone> element, <graphic>, or \texttt{<binaryObject>} elements.
    \item[{Note}]
  \par
When the location being defined consists of a single page, use the {\itshape from} and {\itshape to} attributes with an identical value. When no clear endpoint is given the {\itshape from} attribute should be used without {\itshape to}. For example, if the manuscript description being transcribed has ‘p. 3ff’ as the locus.
    \item[{Example}]
  \leavevmode\bgroup\exampleFont \begin{shaded}\noindent\mbox{}\mbox{}\newline 
\textit{<!-- within ms description -->}{<\textbf{msItem}\hspace*{6pt}{n}="{1}">}\mbox{}\newline 
\hspace*{6pt}{<\textbf{locus}\hspace*{6pt}{from}="{1r}"\hspace*{6pt}{target}="{\#F1r \#F1v \#F2r}"\mbox{}\newline 
\hspace*{6pt}\hspace*{6pt}{to}="{2r}">}ff. 1r-2r{</\textbf{locus}>}\mbox{}\newline 
\hspace*{6pt}{<\textbf{author}>}Ben Jonson{</\textbf{author}>}\mbox{}\newline 
\hspace*{6pt}{<\textbf{title}>}Ode to himself{</\textbf{title}>}\mbox{}\newline 
\hspace*{6pt}{<\textbf{rubric}\hspace*{6pt}{rend}="{italics}">} An Ode{<\textbf{lb}/>} to him selfe.{</\textbf{rubric}>}\mbox{}\newline 
\hspace*{6pt}{<\textbf{incipit}>}Com leaue the loathed stage{</\textbf{incipit}>}\mbox{}\newline 
\hspace*{6pt}{<\textbf{explicit}>}And see his chariot triumph ore his wayne.{</\textbf{explicit}>}\mbox{}\newline 
\hspace*{6pt}{<\textbf{bibl}>}\mbox{}\newline 
\hspace*{6pt}\hspace*{6pt}{<\textbf{name}>}Beal{</\textbf{name}>}, {<\textbf{title}>}Index 1450-1625{</\textbf{title}>}, JnB 380{</\textbf{bibl}>}\mbox{}\newline 
{</\textbf{msItem}>}\mbox{}\newline 
\textit{<!-- within transcription ... -->}\mbox{}\newline 
{<\textbf{pb}\hspace*{6pt}{xml:id}="{F1r}"/>}\mbox{}\newline 
\textit{<!-- ... -->}\mbox{}\newline 
{<\textbf{pb}\hspace*{6pt}{xml:id}="{F1v}"/>}\mbox{}\newline 
\textit{<!-- ... -->}\mbox{}\newline 
{<\textbf{pb}\hspace*{6pt}{xml:id}="{F2r}"/>}\mbox{}\newline 
\textit{<!-- ... -->}\end{shaded}\egroup 


    \item[{Example}]
  The {\itshape facs} attribute is available globally when the \textsf{transcr} module is included in a schema. It may be used to point directly to an image file, as in the following example:\leavevmode\bgroup\exampleFont \begin{shaded}\noindent\mbox{}{<\textbf{msItem}>}\mbox{}\newline 
\hspace*{6pt}{<\textbf{locus}\hspace*{6pt}{facs}="{images/08v.jpg images/09r.jpg images/09v.jpg images/10r.jpg images/10v.jpg}">}fols. 8v-10v{</\textbf{locus}>}\mbox{}\newline 
\hspace*{6pt}{<\textbf{title}>}Birds Praise of Love{</\textbf{title}>}\mbox{}\newline 
\hspace*{6pt}{<\textbf{bibl}>}\mbox{}\newline 
\hspace*{6pt}\hspace*{6pt}{<\textbf{title}>}IMEV{</\textbf{title}>}\mbox{}\newline 
\hspace*{6pt}\hspace*{6pt}{<\textbf{biblScope}>}1506{</\textbf{biblScope}>}\mbox{}\newline 
\hspace*{6pt}{</\textbf{bibl}>}\mbox{}\newline 
{</\textbf{msItem}>}\end{shaded}\egroup 


    \item[{Content model}]
  \fbox{\ttfamily <content>\newline
 <macroRef key="macro.xtext"/>\newline
</content>\newline
    } 
    \item[{Schema Declaration}]
  \mbox{}\hfill\\[-10pt]\begin{Verbatim}[fontsize=\small]
element locus
{
   att.global.attributes,
   att.pointing.attributes,
   attribute scheme { text }?,
   attribute from { text }?,
   attribute to { text }?,
   macro.xtext}
\end{Verbatim}

\end{reflist}  \index{locusGrp=<locusGrp>|oddindex}\index{scheme=@scheme!<locusGrp>|oddindex}
\begin{reflist}
\item[]\begin{specHead}{TEI.locusGrp}{<locusGrp> }groups a number of locations which together form a distinct but discontinuous item within a manuscript or manuscript part, according to a specific foliation. [\xref{http://www.tei-c.org/release/doc/tei-p5-doc/en/html/MS.html\#msloc}{10.3.5. References to Locations within a Manuscript}]\end{specHead} 
    \item[{Module}]
  msdescription
    \item[{Attributes}]
  Attributes att.global (\textit{@xml:id}, \textit{@n}, \textit{@xml:lang}, \textit{@xml:base}, \textit{@xml:space})  (att.global.rendition (\textit{@rend}, \textit{@style}, \textit{@rendition})) (att.global.linking (\textit{@corresp}, \textit{@synch}, \textit{@sameAs}, \textit{@copyOf}, \textit{@next}, \textit{@prev}, \textit{@exclude}, \textit{@select})) (att.global.analytic (\textit{@ana})) (att.global.facs (\textit{@facs})) (att.global.change (\textit{@change})) (att.global.responsibility (\textit{@cert}, \textit{@resp})) (att.global.source (\textit{@source})) \hfil\\[-10pt]\begin{sansreflist}
    \item[@scheme]
  identifies the foliation scheme in terms of which all the locations contained by the group are specified by pointing to some <foliation> element defining it, or to some other equivalent resource.
\begin{reflist}
    \item[{Status}]
  Optional
    \item[{Datatype}]
  teidata.pointer
\end{reflist}  
\end{sansreflist}  
    \item[{Member of}]
  model.pPart.msdesc 
    \item[{Contained by}]
  
    \item[analysis: ]
   cl phr s span\par 
    \item[core: ]
   abbr add addrLine author biblScope citedRange corr date del desc distinct editor email emph expan foreign gloss head headItem headLabel hi item l label measure meeting mentioned name note num orig p pubPlace publisher q quote ref reg resp rs said sic soCalled speaker stage street term textLang time title unclear\par 
    \item[figures: ]
   cell figDesc\par 
    \item[header: ]
   authority catDesc change classCode creation distributor edition extent funder geoDecl handNote language licence principal rendition scriptNote sponsor tagUsage typeNote\par 
    \item[linking: ]
   ab seg\par 
    \item[msdescription: ]
   accMat acquisition additions catchwords collation colophon condition custEvent decoNote explicit filiation finalRubric foliation heraldry incipit layout material msItem msItemStruct musicNotation objectType origDate origPlace origin provenance rubric secFol signatures source stamp summary support surrogates watermark\par 
    \item[namesdates: ]
   addName affiliation age birth bloc country death district education faith floruit forename genName geogFeat geogName langKnown nameLink nationality occupation offset orgName persName placeName region residence roleName settlement sex socecStatus surname\par 
    \item[textcrit: ]
   lem rdg wit witDetail witness\par 
    \item[textstructure: ]
   byline closer dateline docAuthor docDate docEdition docImprint imprimatur opener salute signed titlePart trailer\par 
    \item[transcr: ]
   damage fw metamark mod restore retrace secl supplied surplus
    \item[{May contain}]
  
    \item[msdescription: ]
   locus
    \item[{Example}]
  \leavevmode\bgroup\exampleFont \begin{shaded}\noindent\mbox{}{<\textbf{msItem}>}\mbox{}\newline 
\hspace*{6pt}{<\textbf{locusGrp}>}\mbox{}\newline 
\hspace*{6pt}\hspace*{6pt}{<\textbf{locus}\hspace*{6pt}{from}="{13}"\hspace*{6pt}{to}="{26}">}Bl. 13--26{</\textbf{locus}>}\mbox{}\newline 
\hspace*{6pt}\hspace*{6pt}{<\textbf{locus}\hspace*{6pt}{from}="{37}"\hspace*{6pt}{to}="{58}">}37--58{</\textbf{locus}>}\mbox{}\newline 
\hspace*{6pt}\hspace*{6pt}{<\textbf{locus}\hspace*{6pt}{from}="{82}"\hspace*{6pt}{to}="{96}">}82--96{</\textbf{locus}>}\mbox{}\newline 
\hspace*{6pt}{</\textbf{locusGrp}>}\mbox{}\newline 
\hspace*{6pt}{<\textbf{note}>}Stücke von Daniel Ecklin’s Reise ins h. Land{</\textbf{note}>}\mbox{}\newline 
{</\textbf{msItem}>}\end{shaded}\egroup 


    \item[{Content model}]
  \mbox{}\hfill\\[-10pt]\begin{Verbatim}[fontsize=\small]
<content>
 <elementRef key="locus"
  maxOccurs="unbounded" minOccurs="1"/>
</content>
    
\end{Verbatim}

    \item[{Schema Declaration}]
  \mbox{}\hfill\\[-10pt]\begin{Verbatim}[fontsize=\small]
element locusGrp { att.global.attributes, attribute scheme { text }?, locus+ }
\end{Verbatim}

\end{reflist}  \index{m=<m>|oddindex}\index{baseForm=@baseForm!<m>|oddindex}
\begin{reflist}
\item[]\begin{specHead}{TEI.m}{<m> }(morpheme) represents a grammatical morpheme. [\xref{http://www.tei-c.org/release/doc/tei-p5-doc/en/html/AI.html\#AILC}{17.1. Linguistic Segment Categories}]\end{specHead} 
    \item[{Module}]
  analysis
    \item[{Attributes}]
  Attributes att.global (\textit{@xml:id}, \textit{@n}, \textit{@xml:lang}, \textit{@xml:base}, \textit{@xml:space})  (att.global.rendition (\textit{@rend}, \textit{@style}, \textit{@rendition})) (att.global.linking (\textit{@corresp}, \textit{@synch}, \textit{@sameAs}, \textit{@copyOf}, \textit{@next}, \textit{@prev}, \textit{@exclude}, \textit{@select})) (att.global.analytic (\textit{@ana})) (att.global.facs (\textit{@facs})) (att.global.change (\textit{@change})) (att.global.responsibility (\textit{@cert}, \textit{@resp})) (att.global.source (\textit{@source})) att.segLike (\textit{@function})  (att.datcat (\textit{@datcat}, \textit{@valueDatcat})) (att.fragmentable (\textit{@part})) att.typed (\textit{@type}, \textit{@subtype}) \hfil\\[-10pt]\begin{sansreflist}
    \item[@baseForm]
  supplies the morpheme's base form.
\begin{reflist}
    \item[{Status}]
  Optional
    \item[{Datatype}]
  teidata.word
\end{reflist}  
\end{sansreflist}  
    \item[{Member of}]
  model.segLike 
    \item[{Contained by}]
  
    \item[analysis: ]
   cl m phr s w\par 
    \item[core: ]
   abbr add addrLine author bibl biblScope citedRange corr date del distinct editor email emph expan foreign gloss head headItem headLabel hi item l label measure mentioned name note num orig p pubPlace publisher q quote ref reg rs said sic soCalled speaker stage street term textLang time title unclear\par 
    \item[figures: ]
   cell\par 
    \item[header: ]
   change distributor edition extent geoDecl handNote licence scriptNote typeNote\par 
    \item[linking: ]
   ab seg\par 
    \item[msdescription: ]
   accMat acquisition additions catchwords collation colophon condition custEvent decoNote explicit filiation finalRubric foliation heraldry incipit layout material musicNotation objectType origDate origPlace origin provenance rubric secFol signatures source stamp summary support surrogates watermark\par 
    \item[namesdates: ]
   addName affiliation birth bloc country death district education faith floruit forename genName geogFeat geogName nameLink nationality occupation offset orgName persName placeName region residence roleName settlement sex socecStatus surname\par 
    \item[textcrit: ]
   lem rdg wit witDetail\par 
    \item[textstructure: ]
   byline closer dateline docAuthor docDate docEdition docImprint imprimatur opener salute signed titlePart trailer\par 
    \item[transcr: ]
   damage fw metamark mod restore retrace secl supplied surplus
    \item[{May contain}]
  
    \item[analysis: ]
   c interp interpGrp m span spanGrp\par 
    \item[core: ]
   cb gap gb hi index lb milestone note pb\par 
    \item[figures: ]
   figure notatedMusic\par 
    \item[gaiji: ]
   g\par 
    \item[linking: ]
   alt altGrp anchor join joinGrp link linkGrp seg timeline\par 
    \item[textcrit: ]
   app witDetail\par 
    \item[transcr: ]
   addSpan damageSpan delSpan fw listTranspose metamark space substJoin\par character data
    \item[{Note}]
  \par
The {\itshape type} attribute may be used to indicate the type of morpheme, taking values such as clitic, prefix, stem, etc. as appropriate.
    \item[{Example}]
  \leavevmode\bgroup\exampleFont \begin{shaded}\noindent\mbox{}{<\textbf{w}\hspace*{6pt}{type}="{adjective}">}\mbox{}\newline 
\hspace*{6pt}{<\textbf{w}\hspace*{6pt}{type}="{noun}">}\mbox{}\newline 
\hspace*{6pt}\hspace*{6pt}{<\textbf{m}\hspace*{6pt}{baseForm}="{con}"\hspace*{6pt}{type}="{prefix}">}com{</\textbf{m}>}\mbox{}\newline 
\hspace*{6pt}\hspace*{6pt}{<\textbf{m}\hspace*{6pt}{type}="{root}">}fort{</\textbf{m}>}\mbox{}\newline 
\hspace*{6pt}{</\textbf{w}>}\mbox{}\newline 
\hspace*{6pt}{<\textbf{m}\hspace*{6pt}{type}="{suffix}">}able{</\textbf{m}>}\mbox{}\newline 
{</\textbf{w}>}\end{shaded}\egroup 


    \item[{Content model}]
  \mbox{}\hfill\\[-10pt]\begin{Verbatim}[fontsize=\small]
<content>
 <alternate maxOccurs="unbounded"
  minOccurs="0">
  <textNode/>
  <classRef key="model.gLike"/>
  <classRef key="model.hiLike"/>
  <elementRef key="seg"/>
  <elementRef key="m"/>
  <elementRef key="c"/>
  <classRef key="model.global"/>
 </alternate>
</content>
    
\end{Verbatim}

    \item[{Schema Declaration}]
  \mbox{}\hfill\\[-10pt]\begin{Verbatim}[fontsize=\small]
element m
{
   att.global.attributes,
   att.segLike.attributes,
   att.typed.attributes,
   attribute baseForm { text }?,
   ( text | model.gLike | model.hiLike | seg | m | c | model.global )*
}
\end{Verbatim}

\end{reflist}  \index{mapping=<mapping>|oddindex}
\begin{reflist}
\item[]\begin{specHead}{TEI.mapping}{<mapping> }(character mapping) contains one or more characters which are related to the parent character or glyph in some respect, as specified by the {\itshape type} attribute. [\xref{http://www.tei-c.org/release/doc/tei-p5-doc/en/html/WD.html\#D25-20}{5.2. Markup Constructs for Representation of Characters and Glyphs}]\end{specHead} 
    \item[{Module}]
  gaiji
    \item[{Attributes}]
  Attributes att.global (\textit{@xml:id}, \textit{@n}, \textit{@xml:lang}, \textit{@xml:base}, \textit{@xml:space})  (att.global.rendition (\textit{@rend}, \textit{@style}, \textit{@rendition})) (att.global.linking (\textit{@corresp}, \textit{@synch}, \textit{@sameAs}, \textit{@copyOf}, \textit{@next}, \textit{@prev}, \textit{@exclude}, \textit{@select})) (att.global.analytic (\textit{@ana})) (att.global.facs (\textit{@facs})) (att.global.change (\textit{@change})) (att.global.responsibility (\textit{@cert}, \textit{@resp})) (att.global.source (\textit{@source})) att.typed (\textit{@type}, \textit{@subtype}) 
    \item[{Contained by}]
  
    \item[gaiji: ]
   char glyph
    \item[{May contain}]
  
    \item[gaiji: ]
   g\par character data
    \item[{Note}]
  \par
Suggested values for the {\itshape type} attribute include \texttt{exact} for exact equivalences, \texttt{uppercase} for uppercase equivalences, \texttt{lowercase} for lowercase equivalences, and \texttt{simplified} for simplified characters. The <g> elements contained by this element can point to either another <char> or <glyph>element or contain a character that is intended to be the target of this mapping.
    \item[{Example}]
  \leavevmode\bgroup\exampleFont \begin{shaded}\noindent\mbox{}{<\textbf{mapping}\hspace*{6pt}{type}="{modern}">}r{</\textbf{mapping}>}\mbox{}\newline 
{<\textbf{mapping}\hspace*{6pt}{type}="{standard}">}人{</\textbf{mapping}>}\end{shaded}\egroup 


    \item[{Content model}]
  \fbox{\ttfamily <content>\newline
 <macroRef key="macro.xtext"/>\newline
</content>\newline
    } 
    \item[{Schema Declaration}]
  \mbox{}\hfill\\[-10pt]\begin{Verbatim}[fontsize=\small]
element mapping { att.global.attributes, att.typed.attributes, macro.xtext }
\end{Verbatim}

\end{reflist}  \index{material=<material>|oddindex}
\begin{reflist}
\item[]\begin{specHead}{TEI.material}{<material> }contains a word or phrase describing the material of which the object being described is composed. [\xref{http://www.tei-c.org/release/doc/tei-p5-doc/en/html/MS.html\#msmat}{10.3.2. Material and Object Type}]\end{specHead} 
    \item[{Module}]
  msdescription
    \item[{Attributes}]
  Attributes att.global (\textit{@xml:id}, \textit{@n}, \textit{@xml:lang}, \textit{@xml:base}, \textit{@xml:space})  (att.global.rendition (\textit{@rend}, \textit{@style}, \textit{@rendition})) (att.global.linking (\textit{@corresp}, \textit{@synch}, \textit{@sameAs}, \textit{@copyOf}, \textit{@next}, \textit{@prev}, \textit{@exclude}, \textit{@select})) (att.global.analytic (\textit{@ana})) (att.global.facs (\textit{@facs})) (att.global.change (\textit{@change})) (att.global.responsibility (\textit{@cert}, \textit{@resp})) (att.global.source (\textit{@source})) att.canonical (\textit{@key}, \textit{@ref}) 
    \item[{Member of}]
  model.pPart.msdesc
    \item[{Contained by}]
  
    \item[analysis: ]
   cl phr s span\par 
    \item[core: ]
   abbr add addrLine author biblScope citedRange corr date del desc distinct editor email emph expan foreign gloss head headItem headLabel hi item l label measure meeting mentioned name note num orig p pubPlace publisher q quote ref reg resp rs said sic soCalled speaker stage street term textLang time title unclear\par 
    \item[figures: ]
   cell figDesc\par 
    \item[header: ]
   authority catDesc change classCode creation distributor edition extent funder geoDecl handNote language licence principal rendition scriptNote sponsor tagUsage typeNote\par 
    \item[linking: ]
   ab seg\par 
    \item[msdescription: ]
   accMat acquisition additions catchwords collation colophon condition custEvent decoNote explicit filiation finalRubric foliation heraldry incipit layout material musicNotation objectType origDate origPlace origin provenance rubric secFol signatures source stamp summary support surrogates watermark\par 
    \item[namesdates: ]
   addName affiliation age birth bloc country death district education faith floruit forename genName geogFeat geogName langKnown nameLink nationality occupation offset orgName persName placeName region residence roleName settlement sex socecStatus surname\par 
    \item[textcrit: ]
   lem rdg wit witDetail witness\par 
    \item[textstructure: ]
   byline closer dateline docAuthor docDate docEdition docImprint imprimatur opener salute signed titlePart trailer\par 
    \item[transcr: ]
   damage fw metamark mod restore retrace secl supplied surplus
    \item[{May contain}]
  
    \item[analysis: ]
   c cl interp interpGrp m pc phr s span spanGrp w\par 
    \item[core: ]
   abbr add address cb choice corr date del distinct email emph expan foreign gap gb gloss graphic hi index lb measure measureGrp media mentioned milestone name note num orig pb ptr ref reg rs sic soCalled term time title unclear\par 
    \item[figures: ]
   figure formula notatedMusic\par 
    \item[gaiji: ]
   g\par 
    \item[header: ]
   idno\par 
    \item[linking: ]
   alt altGrp anchor join joinGrp link linkGrp seg timeline\par 
    \item[msdescription: ]
   catchwords depth dim dimensions height heraldry locus locusGrp material objectType origDate origPlace secFol signatures stamp watermark width\par 
    \item[namesdates: ]
   addName affiliation bloc climate country district forename genName geo geogFeat geogName location nameLink offset orgName persName placeName population region roleName settlement state surname terrain trait\par 
    \item[textcrit: ]
   app witDetail\par 
    \item[transcr: ]
   addSpan am damage damageSpan delSpan ex fw handShift listTranspose metamark mod redo restore retrace secl space subst substJoin supplied surplus undo\par character data
    \item[{Note}]
  \par
The {\itshape ref} attribute may be used to point to one or more items within a taxonomy of types of material, defined either internally or externally.
    \item[{Example}]
  \leavevmode\bgroup\exampleFont \begin{shaded}\noindent\mbox{}{<\textbf{physDesc}>}\mbox{}\newline 
\hspace*{6pt}{<\textbf{p}>}\mbox{}\newline 
\hspace*{6pt}\hspace*{6pt}{<\textbf{material}>}Parchment{</\textbf{material}>} leaves with a\mbox{}\newline 
\hspace*{6pt}{<\textbf{material}>}sharkskin{</\textbf{material}>} binding.{</\textbf{p}>}\mbox{}\newline 
{</\textbf{physDesc}>}\end{shaded}\egroup 


    \item[{Content model}]
  \mbox{}\hfill\\[-10pt]\begin{Verbatim}[fontsize=\small]
<content>
 <macroRef key="macro.phraseSeq"/>
</content>
    
\end{Verbatim}

    \item[{Schema Declaration}]
  \mbox{}\hfill\\[-10pt]\begin{Verbatim}[fontsize=\small]
element material
{
   att.global.attributes,
   att.canonical.attributes,
   macro.phraseSeq}
\end{Verbatim}

\end{reflist}  \index{measure=<measure>|oddindex}\index{type=@type!<measure>|oddindex}
\begin{reflist}
\item[]\begin{specHead}{TEI.measure}{<measure> }contains a word or phrase referring to some quantity of an object or commodity, usually comprising a number, a unit, and a commodity name. [\xref{http://www.tei-c.org/release/doc/tei-p5-doc/en/html/CO.html\#CONANU}{3.5.3. Numbers and Measures}]\end{specHead} 
    \item[{Module}]
  core
    \item[{Attributes}]
  Attributes att.global (\textit{@xml:id}, \textit{@n}, \textit{@xml:lang}, \textit{@xml:base}, \textit{@xml:space})  (att.global.rendition (\textit{@rend}, \textit{@style}, \textit{@rendition})) (att.global.linking (\textit{@corresp}, \textit{@synch}, \textit{@sameAs}, \textit{@copyOf}, \textit{@next}, \textit{@prev}, \textit{@exclude}, \textit{@select})) (att.global.analytic (\textit{@ana})) (att.global.facs (\textit{@facs})) (att.global.change (\textit{@change})) (att.global.responsibility (\textit{@cert}, \textit{@resp})) (att.global.source (\textit{@source})) att.measurement (\textit{@unit}, \textit{@quantity}, \textit{@commodity}) \hfil\\[-10pt]\begin{sansreflist}
    \item[@type]
  specifies the type of measurement in any convenient typology.
\begin{reflist}
    \item[{Status}]
  Optional
    \item[{Datatype}]
  teidata.enumerated
\end{reflist}  
\end{sansreflist}  
    \item[{Member of}]
  model.measureLike
    \item[{Contained by}]
  
    \item[analysis: ]
   cl phr s span\par 
    \item[core: ]
   abbr add addrLine author bibl biblScope citedRange corr date del desc distinct editor email emph expan foreign gloss head headItem headLabel hi item l label measure measureGrp meeting mentioned name note num orig p pubPlace publisher q quote ref reg resp rs said sic soCalled speaker stage street term textLang time title unclear\par 
    \item[figures: ]
   cell figDesc\par 
    \item[header: ]
   authority catDesc change classCode creation distributor edition extent funder geoDecl handNote language licence principal rendition scriptNote sponsor tagUsage typeNote\par 
    \item[linking: ]
   ab seg\par 
    \item[msdescription: ]
   accMat acquisition additions catchwords collation colophon condition custEvent decoNote explicit filiation finalRubric foliation heraldry incipit layout material musicNotation objectType origDate origPlace origin provenance rubric secFol signatures source stamp summary support surrogates watermark\par 
    \item[namesdates: ]
   addName affiliation age birth bloc country death district education faith floruit forename genName geogFeat geogName langKnown location nameLink nationality occupation offset orgName persName placeName region residence roleName settlement sex socecStatus surname\par 
    \item[textcrit: ]
   lem rdg wit witDetail witness\par 
    \item[textstructure: ]
   byline closer dateline docAuthor docDate docEdition docImprint imprimatur opener salute signed titlePart trailer\par 
    \item[transcr: ]
   damage fw metamark mod restore retrace secl supplied surplus
    \item[{May contain}]
  
    \item[analysis: ]
   c cl interp interpGrp m pc phr s span spanGrp w\par 
    \item[core: ]
   abbr add address cb choice corr date del distinct email emph expan foreign gap gb gloss graphic hi index lb measure measureGrp media mentioned milestone name note num orig pb ptr ref reg rs sic soCalled term time title unclear\par 
    \item[figures: ]
   figure formula notatedMusic\par 
    \item[gaiji: ]
   g\par 
    \item[header: ]
   idno\par 
    \item[linking: ]
   alt altGrp anchor join joinGrp link linkGrp seg timeline\par 
    \item[msdescription: ]
   catchwords depth dim dimensions height heraldry locus locusGrp material objectType origDate origPlace secFol signatures stamp watermark width\par 
    \item[namesdates: ]
   addName affiliation bloc climate country district forename genName geo geogFeat geogName location nameLink offset orgName persName placeName population region roleName settlement state surname terrain trait\par 
    \item[textcrit: ]
   app witDetail\par 
    \item[transcr: ]
   addSpan am damage damageSpan delSpan ex fw handShift listTranspose metamark mod redo restore retrace secl space subst substJoin supplied surplus undo\par character data
    \item[{Example}]
  \leavevmode\bgroup\exampleFont \begin{shaded}\noindent\mbox{}{<\textbf{measure}\hspace*{6pt}{type}="{weight}">}\mbox{}\newline 
\hspace*{6pt}{<\textbf{num}>}2{</\textbf{num}>} pounds of flesh\mbox{}\newline 
{</\textbf{measure}>}\mbox{}\newline 
{<\textbf{measure}\hspace*{6pt}{type}="{currency}">}£10-11-6d{</\textbf{measure}>}\mbox{}\newline 
{<\textbf{measure}\hspace*{6pt}{type}="{area}">}2 merks of old extent{</\textbf{measure}>}\end{shaded}\egroup 


    \item[{Example}]
  \leavevmode\bgroup\exampleFont \begin{shaded}\noindent\mbox{}{<\textbf{measure}\hspace*{6pt}{commodity}="{rum}"\hspace*{6pt}{quantity}="{40}"\mbox{}\newline 
\hspace*{6pt}{unit}="{hogshead}">}2 score hh rum{</\textbf{measure}>}\mbox{}\newline 
{<\textbf{measure}\hspace*{6pt}{commodity}="{roses}"\hspace*{6pt}{quantity}="{12}"\mbox{}\newline 
\hspace*{6pt}{unit}="{count}">}1 doz. roses{</\textbf{measure}>}\mbox{}\newline 
{<\textbf{measure}\hspace*{6pt}{commodity}="{tulips}"\hspace*{6pt}{quantity}="{1}"\mbox{}\newline 
\hspace*{6pt}{unit}="{count}">}a yellow tulip{</\textbf{measure}>}\end{shaded}\egroup 


    \item[{Content model}]
  \mbox{}\hfill\\[-10pt]\begin{Verbatim}[fontsize=\small]
<content>
 <macroRef key="macro.phraseSeq"/>
</content>
    
\end{Verbatim}

    \item[{Schema Declaration}]
  \mbox{}\hfill\\[-10pt]\begin{Verbatim}[fontsize=\small]
element measure
{
   att.global.attributes,
   att.measurement.attributes,
   attribute type { text }?,
   macro.phraseSeq}
\end{Verbatim}

\end{reflist}  \index{measureGrp=<measureGrp>|oddindex}
\begin{reflist}
\item[]\begin{specHead}{TEI.measureGrp}{<measureGrp> }(measure group) contains a group of dimensional specifications which relate to the same object, for example the height and width of a manuscript page. [\xref{http://www.tei-c.org/release/doc/tei-p5-doc/en/html/MS.html\#msdim}{10.3.4. Dimensions}]\end{specHead} 
    \item[{Module}]
  core
    \item[{Attributes}]
  Attributes att.global (\textit{@xml:id}, \textit{@n}, \textit{@xml:lang}, \textit{@xml:base}, \textit{@xml:space})  (att.global.rendition (\textit{@rend}, \textit{@style}, \textit{@rendition})) (att.global.linking (\textit{@corresp}, \textit{@synch}, \textit{@sameAs}, \textit{@copyOf}, \textit{@next}, \textit{@prev}, \textit{@exclude}, \textit{@select})) (att.global.analytic (\textit{@ana})) (att.global.facs (\textit{@facs})) (att.global.change (\textit{@change})) (att.global.responsibility (\textit{@cert}, \textit{@resp})) (att.global.source (\textit{@source})) att.measurement (\textit{@unit}, \textit{@quantity}, \textit{@commodity}) att.typed (\textit{@type}, \textit{@subtype}) 
    \item[{Member of}]
  model.measureLike
    \item[{Contained by}]
  
    \item[analysis: ]
   cl phr s span\par 
    \item[core: ]
   abbr add addrLine author bibl biblScope citedRange corr date del desc distinct editor email emph expan foreign gloss head headItem headLabel hi item l label measure measureGrp meeting mentioned name note num orig p pubPlace publisher q quote ref reg resp rs said sic soCalled speaker stage street term textLang time title unclear\par 
    \item[figures: ]
   cell figDesc\par 
    \item[header: ]
   authority catDesc change classCode creation distributor edition extent funder geoDecl handNote language licence principal rendition scriptNote sponsor tagUsage typeNote\par 
    \item[linking: ]
   ab seg\par 
    \item[msdescription: ]
   accMat acquisition additions catchwords collation colophon condition custEvent decoNote explicit filiation finalRubric foliation heraldry incipit layout material musicNotation objectType origDate origPlace origin provenance rubric secFol signatures source stamp summary support surrogates watermark\par 
    \item[namesdates: ]
   addName affiliation age birth bloc country death district education faith floruit forename genName geogFeat geogName langKnown location nameLink nationality occupation offset orgName persName placeName region residence roleName settlement sex socecStatus surname\par 
    \item[textcrit: ]
   lem rdg wit witDetail witness\par 
    \item[textstructure: ]
   byline closer dateline docAuthor docDate docEdition docImprint imprimatur opener salute signed titlePart trailer\par 
    \item[transcr: ]
   damage fw metamark mod restore retrace secl supplied surplus
    \item[{May contain}]
  
    \item[core: ]
   measure measureGrp num\par 
    \item[gaiji: ]
   g\par 
    \item[msdescription: ]
   depth dim height width\par 
    \item[namesdates: ]
   geo\par character data
    \item[{Example}]
  \leavevmode\bgroup\exampleFont \begin{shaded}\noindent\mbox{}{<\textbf{measureGrp}\hspace*{6pt}{type}="{leaves}"\hspace*{6pt}{unit}="{mm}">}\mbox{}\newline 
\hspace*{6pt}{<\textbf{height}\hspace*{6pt}{scope}="{range}">}157-160{</\textbf{height}>}\mbox{}\newline 
\hspace*{6pt}{<\textbf{width}\hspace*{6pt}{quantity}="{105}"/>}\mbox{}\newline 
{</\textbf{measureGrp}>}\mbox{}\newline 
{<\textbf{measureGrp}\hspace*{6pt}{type}="{ruledArea}"\hspace*{6pt}{unit}="{mm}">}\mbox{}\newline 
\hspace*{6pt}{<\textbf{height}\hspace*{6pt}{quantity}="{90}"\hspace*{6pt}{scope}="{most}"/>}\mbox{}\newline 
\hspace*{6pt}{<\textbf{width}\hspace*{6pt}{quantity}="{48}"\hspace*{6pt}{scope}="{most}"/>}\mbox{}\newline 
{</\textbf{measureGrp}>}\mbox{}\newline 
{<\textbf{measureGrp}\hspace*{6pt}{type}="{box}"\hspace*{6pt}{unit}="{in}">}\mbox{}\newline 
\hspace*{6pt}{<\textbf{height}\hspace*{6pt}{quantity}="{12}"/>}\mbox{}\newline 
\hspace*{6pt}{<\textbf{width}\hspace*{6pt}{quantity}="{10}"/>}\mbox{}\newline 
\hspace*{6pt}{<\textbf{depth}\hspace*{6pt}{quantity}="{6}"/>}\mbox{}\newline 
{</\textbf{measureGrp}>}\end{shaded}\egroup 


    \item[{Content model}]
  \mbox{}\hfill\\[-10pt]\begin{Verbatim}[fontsize=\small]
<content>
 <alternate maxOccurs="unbounded"
  minOccurs="0">
  <textNode/>
  <classRef key="model.gLike"/>
  <classRef key="model.measureLike"/>
 </alternate>
</content>
    
\end{Verbatim}

    \item[{Schema Declaration}]
  \mbox{}\hfill\\[-10pt]\begin{Verbatim}[fontsize=\small]
element measureGrp
{
   att.global.attributes,
   att.measurement.attributes,
   att.typed.attributes,
   ( text | model.gLike | model.measureLike )*
}
\end{Verbatim}

\end{reflist}  \index{media=<media>|oddindex}\index{mimeType=@mimeType!<media>|oddindex}
\begin{reflist}
\item[]\begin{specHead}{TEI.media}{<media> }indicates the location of any form of external media such as an audio or video clip etc. [\xref{http://www.tei-c.org/release/doc/tei-p5-doc/en/html/CO.html\#COGR}{3.9. Graphics and Other Non-textual Components}]\end{specHead} 
    \item[{Module}]
  core
    \item[{Attributes}]
  Attributes att.typed (\textit{@type}, \textit{@subtype}) att.global (\textit{@xml:id}, \textit{@n}, \textit{@xml:lang}, \textit{@xml:base}, \textit{@xml:space})  (att.global.rendition (\textit{@rend}, \textit{@style}, \textit{@rendition})) (att.global.linking (\textit{@corresp}, \textit{@synch}, \textit{@sameAs}, \textit{@copyOf}, \textit{@next}, \textit{@prev}, \textit{@exclude}, \textit{@select})) (att.global.analytic (\textit{@ana})) (att.global.facs (\textit{@facs})) (att.global.change (\textit{@change})) (att.global.responsibility (\textit{@cert}, \textit{@resp})) (att.global.source (\textit{@source})) att.resourced (\textit{@url}) att.declaring (\textit{@decls}) att.timed (\textit{@start}, \textit{@end}) att.media (@width, @height, @scale) \hfil\\[-10pt]\begin{sansreflist}
    \item[@mimeType]
  (MIME media type) specifies the applicable multimedia internet mail extension (MIME) media type
\begin{reflist}
    \item[{Derived from}]
  att.internetMedia
    \item[{Status}]
  Required
    \item[{Datatype}]
  1–∞ occurrences of teidata.word separated by whitespace
\end{reflist}  
\end{sansreflist}  
    \item[{Member of}]
  model.graphicLike
    \item[{Contained by}]
  
    \item[analysis: ]
   cl phr s\par 
    \item[core: ]
   abbr add addrLine author biblScope citedRange corr date del distinct editor email emph expan foreign gloss head headItem headLabel hi item l label measure mentioned name note num orig p pubPlace publisher q quote ref reg rs said sic soCalled speaker stage street term textLang time title unclear\par 
    \item[figures: ]
   cell figure formula table\par 
    \item[gaiji: ]
   char glyph\par 
    \item[header: ]
   change distributor edition extent geoDecl handNote licence scriptNote typeNote\par 
    \item[linking: ]
   ab seg\par 
    \item[msdescription: ]
   accMat acquisition additions catchwords collation colophon condition custEvent decoNote explicit filiation finalRubric foliation heraldry incipit layout material musicNotation objectType origDate origPlace origin provenance rubric secFol signatures source stamp summary support surrogates watermark\par 
    \item[namesdates: ]
   addName affiliation birth bloc country death district education faith floruit forename genName geogFeat geogName nameLink nationality occupation offset orgName persName placeName region residence roleName settlement sex socecStatus surname\par 
    \item[textcrit: ]
   lem rdg wit witDetail\par 
    \item[textstructure: ]
   byline closer dateline docAuthor docDate docEdition docImprint imprimatur opener salute signed titlePart trailer\par 
    \item[transcr: ]
   damage facsimile fw metamark mod restore retrace secl sourceDoc supplied surface surplus zone
    \item[{May contain}]
  
    \item[core: ]
   desc
    \item[{Note}]
  \par
The attributes available for this element are not appropriate in all cases. For example, it makes no sense to specify the temporal duration of a graphic. Such errors are not currently detected.\par
The {\itshape mimeType} attribute must be used to specify the MIME media type of the resource specified by the {\itshape url} attribute.
    \item[{Example}]
  \leavevmode\bgroup\exampleFont \begin{shaded}\noindent\mbox{}{<\textbf{figure}>}\mbox{}\newline 
\hspace*{6pt}{<\textbf{media}\hspace*{6pt}{mimeType}="{image/png}"\hspace*{6pt}{url}="{fig1.png}"/>}\mbox{}\newline 
\hspace*{6pt}{<\textbf{head}>}Figure One: The View from the Bridge{</\textbf{head}>}\mbox{}\newline 
\hspace*{6pt}{<\textbf{figDesc}>}A Whistleresque view showing four or five sailing boats in the foreground, and a\mbox{}\newline 
\hspace*{6pt}\hspace*{6pt} series of buoys strung out between them.{</\textbf{figDesc}>}\mbox{}\newline 
{</\textbf{figure}>}\end{shaded}\egroup 


    \item[{Example}]
  \leavevmode\bgroup\exampleFont \begin{shaded}\noindent\mbox{}{<\textbf{media}\hspace*{6pt}{dur}="{PT10S}"\hspace*{6pt}{mimeType}="{audio/wav}"\mbox{}\newline 
\hspace*{6pt}{url}="{dingDong.wav}">}\mbox{}\newline 
\hspace*{6pt}{<\textbf{desc}>}Ten seconds of bellringing sound{</\textbf{desc}>}\mbox{}\newline 
{</\textbf{media}>}\end{shaded}\egroup 


    \item[{Example}]
  \leavevmode\bgroup\exampleFont \begin{shaded}\noindent\mbox{}{<\textbf{media}\hspace*{6pt}{dur}="{PT45M}"\hspace*{6pt}{mimeType}="{video/mp4}"\mbox{}\newline 
\hspace*{6pt}{url}="{clip45.mp4}"\hspace*{6pt}{width}="{500px}">}\mbox{}\newline 
\hspace*{6pt}{<\textbf{desc}>}A 45 minute video clip to be displayed in a window 500\mbox{}\newline 
\hspace*{6pt}\hspace*{6pt} px wide{</\textbf{desc}>}\mbox{}\newline 
{</\textbf{media}>}\end{shaded}\egroup 


    \item[{Content model}]
  \mbox{}\hfill\\[-10pt]\begin{Verbatim}[fontsize=\small]
<content>
 <classRef key="model.descLike"
  maxOccurs="unbounded" minOccurs="0"/>
</content>
    
\end{Verbatim}

    \item[{Schema Declaration}]
  \mbox{}\hfill\\[-10pt]\begin{Verbatim}[fontsize=\small]
element media
{
   att.typed.attributes,
   att.global.attributes,
   att.media.attribute.width,
   att.media.attribute.height,
   att.media.attribute.scale,
   att.resourced.attributes,
   att.declaring.attributes,
   att.timed.attributes,
   attribute mimeType { list { + } },
   model.descLike*
}
\end{Verbatim}

\end{reflist}  \index{meeting=<meeting>|oddindex}
\begin{reflist}
\item[]\begin{specHead}{TEI.meeting}{<meeting> }contains the formalized descriptive title for a meeting or conference, for use in a bibliographic description for an item derived from such a meeting, or as a heading or preamble to publications emanating from it. [\xref{http://www.tei-c.org/release/doc/tei-p5-doc/en/html/CO.html\#COBICOR}{3.11.2.2. Titles, Authors, and Editors}]\end{specHead} 
    \item[{Module}]
  core
    \item[{Attributes}]
  Attributes att.global (\textit{@xml:id}, \textit{@n}, \textit{@xml:lang}, \textit{@xml:base}, \textit{@xml:space})  (att.global.rendition (\textit{@rend}, \textit{@style}, \textit{@rendition})) (att.global.linking (\textit{@corresp}, \textit{@synch}, \textit{@sameAs}, \textit{@copyOf}, \textit{@next}, \textit{@prev}, \textit{@exclude}, \textit{@select})) (att.global.analytic (\textit{@ana})) (att.global.facs (\textit{@facs})) (att.global.change (\textit{@change})) (att.global.responsibility (\textit{@cert}, \textit{@resp})) (att.global.source (\textit{@source})) att.canonical (\textit{@key}, \textit{@ref}) 
    \item[{Member of}]
  model.divWrapper model.respLike 
    \item[{Contained by}]
  
    \item[core: ]
   bibl lg list monogr\par 
    \item[figures: ]
   figure table\par 
    \item[header: ]
   editionStmt titleStmt\par 
    \item[msdescription: ]
   msItem\par 
    \item[textstructure: ]
   body div front group
    \item[{May contain}]
  
    \item[core: ]
   abbr address bibl biblStruct choice cit date desc distinct email emph expan foreign gloss hi label list listBibl measure measureGrp mentioned name num ptr q quote ref rs said soCalled stage term time title\par 
    \item[figures: ]
   table\par 
    \item[header: ]
   biblFull idno\par 
    \item[msdescription: ]
   catchwords depth dim dimensions height heraldry locus locusGrp material msDesc objectType origDate origPlace secFol signatures stamp watermark width\par 
    \item[namesdates: ]
   addName affiliation bloc climate country district forename genName geo geogFeat geogName listEvent listNym listOrg listPerson listPlace location nameLink offset orgName persName placeName population region roleName settlement state surname terrain trait\par 
    \item[textcrit: ]
   listApp listWit\par 
    \item[textstructure: ]
   floatingText\par 
    \item[transcr: ]
   am ex subst\par character data
    \item[{Example}]
  \leavevmode\bgroup\exampleFont \begin{shaded}\noindent\mbox{}{<\textbf{div}>}\mbox{}\newline 
\hspace*{6pt}{<\textbf{meeting}>}Ninth International Conference on Middle High German Textual Criticism, Aachen,\mbox{}\newline 
\hspace*{6pt}\hspace*{6pt} June 1998.{</\textbf{meeting}>}\mbox{}\newline 
\hspace*{6pt}{<\textbf{list}\hspace*{6pt}{type}="{attendance}">}\mbox{}\newline 
\hspace*{6pt}\hspace*{6pt}{<\textbf{head}>}List of Participants{</\textbf{head}>}\mbox{}\newline 
\hspace*{6pt}\hspace*{6pt}{<\textbf{item}>}\mbox{}\newline 
\hspace*{6pt}\hspace*{6pt}\hspace*{6pt}{<\textbf{persName}>}...{</\textbf{persName}>}\mbox{}\newline 
\hspace*{6pt}\hspace*{6pt}{</\textbf{item}>}\mbox{}\newline 
\hspace*{6pt}\hspace*{6pt}{<\textbf{item}>}\mbox{}\newline 
\hspace*{6pt}\hspace*{6pt}\hspace*{6pt}{<\textbf{persName}>}...{</\textbf{persName}>}\mbox{}\newline 
\hspace*{6pt}\hspace*{6pt}{</\textbf{item}>}\mbox{}\newline 
\textit{<!--...-->}\mbox{}\newline 
\hspace*{6pt}{</\textbf{list}>}\mbox{}\newline 
\hspace*{6pt}{<\textbf{p}>}...{</\textbf{p}>}\mbox{}\newline 
{</\textbf{div}>}\end{shaded}\egroup 


    \item[{Content model}]
  \mbox{}\hfill\\[-10pt]\begin{Verbatim}[fontsize=\small]
<content>
 <macroRef key="macro.limitedContent"/>
</content>
    
\end{Verbatim}

    \item[{Schema Declaration}]
  \mbox{}\hfill\\[-10pt]\begin{Verbatim}[fontsize=\small]
element meeting
{
   att.global.attributes,
   att.canonical.attributes,
   macro.limitedContent}
\end{Verbatim}

\end{reflist}  \index{mentioned=<mentioned>|oddindex}
\begin{reflist}
\item[]\begin{specHead}{TEI.mentioned}{<mentioned> }marks words or phrases mentioned, not used. [\xref{http://www.tei-c.org/release/doc/tei-p5-doc/en/html/CO.html\#COHQQ}{3.3.3. Quotation}]\end{specHead} 
    \item[{Module}]
  core
    \item[{Attributes}]
  Attributes att.global (\textit{@xml:id}, \textit{@n}, \textit{@xml:lang}, \textit{@xml:base}, \textit{@xml:space})  (att.global.rendition (\textit{@rend}, \textit{@style}, \textit{@rendition})) (att.global.linking (\textit{@corresp}, \textit{@synch}, \textit{@sameAs}, \textit{@copyOf}, \textit{@next}, \textit{@prev}, \textit{@exclude}, \textit{@select})) (att.global.analytic (\textit{@ana})) (att.global.facs (\textit{@facs})) (att.global.change (\textit{@change})) (att.global.responsibility (\textit{@cert}, \textit{@resp})) (att.global.source (\textit{@source}))
    \item[{Member of}]
  model.emphLike
    \item[{Contained by}]
  
    \item[analysis: ]
   cl phr s span\par 
    \item[core: ]
   abbr add addrLine author bibl biblScope citedRange corr date del desc distinct editor email emph expan foreign gloss head headItem headLabel hi item l label measure meeting mentioned name note num orig p pubPlace publisher q quote ref reg resp rs said sic soCalled speaker stage street term textLang time title unclear\par 
    \item[figures: ]
   cell figDesc\par 
    \item[header: ]
   authority catDesc change classCode creation distributor edition extent funder geoDecl handNote language licence principal rendition scriptNote sponsor tagUsage typeNote\par 
    \item[linking: ]
   ab seg\par 
    \item[msdescription: ]
   accMat acquisition additions catchwords collation colophon condition custEvent decoNote explicit filiation finalRubric foliation heraldry incipit layout material musicNotation objectType origDate origPlace origin provenance rubric secFol signatures source stamp summary support surrogates watermark\par 
    \item[namesdates: ]
   addName affiliation age birth bloc country death district education faith floruit forename genName geogFeat geogName langKnown nameLink nationality occupation offset orgName persName placeName region residence roleName settlement sex socecStatus surname\par 
    \item[textcrit: ]
   lem rdg wit witDetail witness\par 
    \item[textstructure: ]
   byline closer dateline docAuthor docDate docEdition docImprint imprimatur opener salute signed titlePart trailer\par 
    \item[transcr: ]
   damage fw metamark mod restore retrace secl supplied surplus
    \item[{May contain}]
  
    \item[analysis: ]
   c cl interp interpGrp m pc phr s span spanGrp w\par 
    \item[core: ]
   abbr add address cb choice corr date del distinct email emph expan foreign gap gb gloss graphic hi index lb measure measureGrp media mentioned milestone name note num orig pb ptr ref reg rs sic soCalled term time title unclear\par 
    \item[figures: ]
   figure formula notatedMusic\par 
    \item[gaiji: ]
   g\par 
    \item[header: ]
   idno\par 
    \item[linking: ]
   alt altGrp anchor join joinGrp link linkGrp seg timeline\par 
    \item[msdescription: ]
   catchwords depth dim dimensions height heraldry locus locusGrp material objectType origDate origPlace secFol signatures stamp watermark width\par 
    \item[namesdates: ]
   addName affiliation bloc climate country district forename genName geo geogFeat geogName location nameLink offset orgName persName placeName population region roleName settlement state surname terrain trait\par 
    \item[textcrit: ]
   app witDetail\par 
    \item[transcr: ]
   addSpan am damage damageSpan delSpan ex fw handShift listTranspose metamark mod redo restore retrace secl space subst substJoin supplied surplus undo\par character data
    \item[{Example}]
  \leavevmode\bgroup\exampleFont \begin{shaded}\noindent\mbox{}There is thus a\mbox{}\newline 
 striking accentual difference between a verbal form like {<\textbf{mentioned}\hspace*{6pt}{xml:id}="{X234}"\hspace*{6pt}{xml:lang}="{el}">}eluthemen{</\textbf{mentioned}>}\mbox{}\newline 
{<\textbf{gloss}\hspace*{6pt}{target}="{\#X234}">}we were released,{</\textbf{gloss}>} accented on the second syllable of the\mbox{}\newline 
 word, and its participial derivative \mbox{}\newline 
{<\textbf{mentioned}\hspace*{6pt}{xml:id}="{X235}"\hspace*{6pt}{xml:lang}="{el}">}lutheis{</\textbf{mentioned}>}\mbox{}\newline 
{<\textbf{gloss}\hspace*{6pt}{target}="{\#X235}">}released,{</\textbf{gloss}>} accented on the last.\end{shaded}\egroup 


    \item[{Content model}]
  \mbox{}\hfill\\[-10pt]\begin{Verbatim}[fontsize=\small]
<content>
 <macroRef key="macro.phraseSeq"/>
</content>
    
\end{Verbatim}

    \item[{Schema Declaration}]
  \mbox{}\hfill\\[-10pt]\begin{Verbatim}[fontsize=\small]
element mentioned { att.global.attributes, macro.phraseSeq }
\end{Verbatim}

\end{reflist}  \index{metamark=<metamark>|oddindex}\index{function=@function!<metamark>|oddindex}\index{target=@target!<metamark>|oddindex}
\begin{reflist}
\item[]\begin{specHead}{TEI.metamark}{<metamark> }contains or describes any kind of graphic or written signal within a document the function of which is to determine how it should be read rather than forming part of the actual content of the document. [\xref{http://www.tei-c.org/release/doc/tei-p5-doc/en/html/PH.html\#PH-meta}{11.3.4.2. Metamarks}]\end{specHead} 
    \item[{Module}]
  transcr
    \item[{Attributes}]
  Attributes att.spanning (\textit{@spanTo}) att.placement (\textit{@place}) att.global (\textit{@xml:id}, \textit{@n}, \textit{@xml:lang}, \textit{@xml:base}, \textit{@xml:space})  (att.global.rendition (\textit{@rend}, \textit{@style}, \textit{@rendition})) (att.global.linking (\textit{@corresp}, \textit{@synch}, \textit{@sameAs}, \textit{@copyOf}, \textit{@next}, \textit{@prev}, \textit{@exclude}, \textit{@select})) (att.global.analytic (\textit{@ana})) (att.global.facs (\textit{@facs})) (att.global.change (\textit{@change})) (att.global.responsibility (\textit{@cert}, \textit{@resp})) (att.global.source (\textit{@source})) \hfil\\[-10pt]\begin{sansreflist}
    \item[@function]
  describes the function (for example status, insertion, deletion, transposition) of the metamark.
\begin{reflist}
    \item[{Status}]
  Optional
    \item[{Datatype}]
  teidata.word
\end{reflist}  
    \item[@target]
  identifies one or more elements to which the metamark applies.
\begin{reflist}
    \item[{Status}]
  Optional
    \item[{Datatype}]
  1–∞ occurrences of teidata.pointer separated by whitespace
\end{reflist}  
\end{sansreflist}  
    \item[{Member of}]
  model.global
    \item[{Contained by}]
  
    \item[analysis: ]
   cl m phr s span w\par 
    \item[core: ]
   abbr add addrLine address author bibl biblScope cit citedRange corr date del distinct editor email emph expan foreign gloss head headItem headLabel hi imprint item l label lg list measure mentioned name note num orig p pubPlace publisher q quote ref reg resp rs said series sic soCalled sp speaker stage street term textLang time title unclear\par 
    \item[figures: ]
   cell figure table\par 
    \item[header: ]
   authority change classCode distributor edition extent funder geoDecl handNote language licence principal scriptNote sponsor typeNote\par 
    \item[linking: ]
   ab seg\par 
    \item[msdescription: ]
   accMat acquisition additions catchwords collation colophon condition custEvent decoNote explicit filiation finalRubric foliation heraldry incipit layout material msItem musicNotation objectType origDate origPlace origin provenance rubric secFol signatures source stamp summary support surrogates watermark\par 
    \item[namesdates: ]
   addName affiliation age birth bloc country death district education faith floruit forename genName geogFeat geogName langKnown nameLink nationality occupation offset orgName persName person personGrp placeName region residence roleName settlement sex socecStatus surname\par 
    \item[textcrit: ]
   lem rdg wit witDetail\par 
    \item[textstructure: ]
   argument back body byline closer dateline div docAuthor docDate docEdition docImprint docTitle epigraph floatingText front group imprimatur opener postscript salute signed text titlePage titlePart trailer\par 
    \item[transcr: ]
   damage fw line metamark mod restore retrace secl sourceDoc supplied surface surfaceGrp surplus zone
    \item[{May contain}]
  
    \item[analysis: ]
   c cl interp interpGrp m pc phr s span spanGrp w\par 
    \item[core: ]
   abbr add address bibl biblStruct cb choice cit corr date del desc distinct email emph expan foreign gap gb gloss graphic hi index l label lb lg list listBibl measure measureGrp media mentioned milestone name note num orig p pb ptr q quote ref reg rs said sic soCalled sp stage term time title unclear\par 
    \item[figures: ]
   figure formula notatedMusic table\par 
    \item[gaiji: ]
   g\par 
    \item[header: ]
   biblFull idno\par 
    \item[linking: ]
   ab alt altGrp anchor join joinGrp link linkGrp seg timeline\par 
    \item[msdescription: ]
   catchwords depth dim dimensions height heraldry locus locusGrp material msDesc objectType origDate origPlace secFol signatures stamp watermark width\par 
    \item[namesdates: ]
   addName affiliation bloc climate country district forename genName geo geogFeat geogName listEvent listNym listOrg listPerson listPlace location nameLink offset orgName persName placeName population region roleName settlement state surname terrain trait\par 
    \item[textcrit: ]
   app listApp listWit witDetail\par 
    \item[textstructure: ]
   floatingText\par 
    \item[transcr: ]
   addSpan am damage damageSpan delSpan ex fw handShift listTranspose metamark mod redo restore retrace secl space subst substJoin supplied surplus undo\par character data
    \item[{Content model}]
  \mbox{}\hfill\\[-10pt]\begin{Verbatim}[fontsize=\small]
<content>
 <macroRef key="macro.specialPara"/>
</content>
    
\end{Verbatim}

    \item[{Schema Declaration}]
  \mbox{}\hfill\\[-10pt]\begin{Verbatim}[fontsize=\small]
element metamark
{
   att.spanning.attributes,
   att.placement.attributes,
   att.global.attributes,
   attribute function { text }?,
   attribute target { list { + } }?,
   macro.specialPara}
\end{Verbatim}

\end{reflist}  \index{milestone=<milestone>|oddindex}
\begin{reflist}
\item[]\begin{specHead}{TEI.milestone}{<milestone> }marks a boundary point separating any kind of section of a text, typically but not necessarily indicating a point at which some part of a standard reference system changes, where the change is not represented by a structural element. [\xref{http://www.tei-c.org/release/doc/tei-p5-doc/en/html/CO.html\#CORS5}{3.10.3. Milestone Elements}]\end{specHead} 
    \item[{Module}]
  core
    \item[{Attributes}]
  Attributes att.global (\textit{@xml:id}, \textit{@n}, \textit{@xml:lang}, \textit{@xml:base}, \textit{@xml:space})  (att.global.rendition (\textit{@rend}, \textit{@style}, \textit{@rendition})) (att.global.linking (\textit{@corresp}, \textit{@synch}, \textit{@sameAs}, \textit{@copyOf}, \textit{@next}, \textit{@prev}, \textit{@exclude}, \textit{@select})) (att.global.analytic (\textit{@ana})) (att.global.facs (\textit{@facs})) (att.global.change (\textit{@change})) (att.global.responsibility (\textit{@cert}, \textit{@resp})) (att.global.source (\textit{@source})) att.milestoneUnit (\textit{@unit}) att.typed (\textit{@type}, \textit{@subtype}) att.edition (\textit{@ed}, \textit{@edRef}) att.spanning (\textit{@spanTo}) att.breaking (\textit{@break}) 
    \item[{Member of}]
  model.milestoneLike
    \item[{Contained by}]
  
    \item[analysis: ]
   cl m phr s span w\par 
    \item[core: ]
   abbr add addrLine address author bibl biblScope cit citedRange corr date del distinct editor email emph expan foreign gloss head headItem headLabel hi imprint item l label lg list listBibl measure mentioned name note num orig p pubPlace publisher q quote ref reg resp rs said series sic soCalled sp speaker stage street term textLang time title unclear\par 
    \item[figures: ]
   cell figure table\par 
    \item[header: ]
   authority change classCode distributor edition extent funder geoDecl handNote language licence principal scriptNote sponsor typeNote\par 
    \item[linking: ]
   ab seg\par 
    \item[msdescription: ]
   accMat acquisition additions catchwords collation colophon condition custEvent decoNote explicit filiation finalRubric foliation heraldry incipit layout material msItem musicNotation objectType origDate origPlace origin provenance rubric secFol signatures source stamp summary support surrogates watermark\par 
    \item[namesdates: ]
   addName affiliation age birth bloc country death district education faith floruit forename genName geogFeat geogName langKnown nameLink nationality occupation offset org orgName persName person personGrp placeName region residence roleName settlement sex socecStatus surname\par 
    \item[textcrit: ]
   lem rdg wit witDetail\par 
    \item[textstructure: ]
   argument back body byline closer dateline div docAuthor docDate docEdition docImprint docTitle epigraph floatingText front group imprimatur opener postscript salute signed text titlePage titlePart trailer\par 
    \item[transcr: ]
   damage fw line metamark mod restore retrace secl sourceDoc subst supplied surface surfaceGrp surplus zone
    \item[{May contain}]
  Empty element
    \item[{Note}]
  \par
For this element, the global {\itshape n} attribute indicates the new number or other value for the unit which changes at this milestone. The special value \textit{unnumbered} should be used in passages which fall outside the normal numbering scheme, such as chapter or other headings, poem numbers or titles, etc.\par
The order in which <milestone> elements are given at a given point is not normally significant.
    \item[{Example}]
  \leavevmode\bgroup\exampleFont \begin{shaded}\noindent\mbox{}{<\textbf{milestone}\hspace*{6pt}{ed}="{La}"\hspace*{6pt}{n}="{23}"\hspace*{6pt}{unit}="{Dreissiger}"/>}\mbox{}\newline 
 ... {<\textbf{milestone}\hspace*{6pt}{ed}="{AV}"\hspace*{6pt}{n}="{24}"\hspace*{6pt}{unit}="{verse}"/>} ...\end{shaded}\egroup 


    \item[{Content model}]
  \fbox{\ttfamily <content>\newline
</content>\newline
    } 
    \item[{Schema Declaration}]
  \mbox{}\hfill\\[-10pt]\begin{Verbatim}[fontsize=\small]
element milestone
{
   att.global.attributes,
   att.milestoneUnit.attributes,
   att.typed.attributes,
   att.edition.attributes,
   att.spanning.attributes,
   att.breaking.attributes,
   empty
}
\end{Verbatim}

\end{reflist}  \index{mod=<mod>|oddindex}
\begin{reflist}
\item[]\begin{specHead}{TEI.mod}{<mod> }represents any kind of modification identified within a single document. [\xref{http://www.tei-c.org/release/doc/tei-p5-doc/en/html/PH.html\#PH-mod}{11.3.4.1. Generic Modification}]\end{specHead} 
    \item[{Module}]
  transcr
    \item[{Attributes}]
  Attributes att.global (\textit{@xml:id}, \textit{@n}, \textit{@xml:lang}, \textit{@xml:base}, \textit{@xml:space})  (att.global.rendition (\textit{@rend}, \textit{@style}, \textit{@rendition})) (att.global.linking (\textit{@corresp}, \textit{@synch}, \textit{@sameAs}, \textit{@copyOf}, \textit{@next}, \textit{@prev}, \textit{@exclude}, \textit{@select})) (att.global.analytic (\textit{@ana})) (att.global.facs (\textit{@facs})) (att.global.change (\textit{@change})) (att.global.responsibility (\textit{@cert}, \textit{@resp})) (att.global.source (\textit{@source})) att.transcriptional (\textit{@status}, \textit{@cause}, \textit{@seq})  (att.editLike (\textit{@evidence}, \textit{@instant}) (att.dimensions (\textit{@unit}, \textit{@quantity}, \textit{@extent}, \textit{@precision}, \textit{@scope}) (att.ranging (\textit{@atLeast}, \textit{@atMost}, \textit{@min}, \textit{@max}, \textit{@confidence})) ) ) (att.written (\textit{@hand})) att.typed (\textit{@type}, \textit{@subtype}) att.spanning (\textit{@spanTo}) 
    \item[{Member of}]
  model.linePart model.pPart.transcriptional
    \item[{Contained by}]
  
    \item[analysis: ]
   cl pc phr s w\par 
    \item[core: ]
   abbr add addrLine author bibl biblScope citedRange corr date del distinct editor email emph expan foreign gloss head headItem headLabel hi item l label measure mentioned name note num orig p pubPlace publisher q quote ref reg rs said sic soCalled speaker stage street term textLang time title unclear\par 
    \item[figures: ]
   cell\par 
    \item[header: ]
   change distributor edition extent geoDecl handNote licence scriptNote typeNote\par 
    \item[linking: ]
   ab seg\par 
    \item[msdescription: ]
   accMat acquisition additions catchwords collation colophon condition custEvent decoNote explicit filiation finalRubric foliation heraldry incipit layout material musicNotation objectType origDate origPlace origin provenance rubric secFol signatures source stamp summary support surrogates watermark\par 
    \item[namesdates: ]
   addName affiliation birth bloc country death district education faith floruit forename genName geogFeat geogName nameLink nationality occupation offset orgName persName placeName region residence roleName settlement sex socecStatus surname\par 
    \item[textcrit: ]
   lem rdg wit witDetail\par 
    \item[textstructure: ]
   byline closer dateline docAuthor docDate docEdition docImprint imprimatur opener salute signed titlePart trailer\par 
    \item[transcr: ]
   am damage fw line metamark mod restore retrace secl supplied surplus zone
    \item[{May contain}]
  
    \item[analysis: ]
   c cl interp interpGrp m pc phr s span spanGrp w\par 
    \item[core: ]
   abbr add address bibl biblStruct cb choice cit corr date del desc distinct email emph expan foreign gap gb gloss graphic hi index l label lb lg list listBibl measure measureGrp media mentioned milestone name note num orig pb ptr q quote ref reg rs said sic soCalled stage term time title unclear\par 
    \item[figures: ]
   figure formula notatedMusic table\par 
    \item[gaiji: ]
   g\par 
    \item[header: ]
   biblFull idno\par 
    \item[linking: ]
   alt altGrp anchor join joinGrp link linkGrp seg timeline\par 
    \item[msdescription: ]
   catchwords depth dim dimensions height heraldry locus locusGrp material msDesc objectType origDate origPlace secFol signatures stamp watermark width\par 
    \item[namesdates: ]
   addName affiliation bloc climate country district forename genName geo geogFeat geogName listEvent listNym listOrg listPerson listPlace location nameLink offset orgName persName placeName population region roleName settlement state surname terrain trait\par 
    \item[textcrit: ]
   app listApp listWit witDetail\par 
    \item[textstructure: ]
   floatingText\par 
    \item[transcr: ]
   addSpan am damage damageSpan delSpan ex fw handShift listTranspose metamark mod redo restore retrace secl space subst substJoin supplied surplus undo\par character data
    \item[{Example}]
  \leavevmode\bgroup\exampleFont \begin{shaded}\noindent\mbox{}{<\textbf{mod}\hspace*{6pt}{type}="{subst}">}\mbox{}\newline 
\hspace*{6pt}{<\textbf{add}>}pleasing{</\textbf{add}>}\mbox{}\newline 
\hspace*{6pt}{<\textbf{del}>}agreable{</\textbf{del}>}\mbox{}\newline 
{</\textbf{mod}>}\end{shaded}\egroup 


    \item[{Content model}]
  \mbox{}\hfill\\[-10pt]\begin{Verbatim}[fontsize=\small]
<content>
 <macroRef key="macro.paraContent"/>
</content>
    
\end{Verbatim}

    \item[{Schema Declaration}]
  \mbox{}\hfill\\[-10pt]\begin{Verbatim}[fontsize=\small]
element mod
{
   att.global.attributes,
   att.transcriptional.attributes,
   att.typed.attributes,
   att.spanning.attributes,
   macro.paraContent}
\end{Verbatim}

\end{reflist}  \index{monogr=<monogr>|oddindex}
\begin{reflist}
\item[]\begin{specHead}{TEI.monogr}{<monogr> }(monographic level) contains bibliographic elements describing an item (e.g. a book or journal) published as an independent item (i.e. as a separate physical object). [\xref{http://www.tei-c.org/release/doc/tei-p5-doc/en/html/CO.html\#COBICOL}{3.11.2.1. Analytic, Monographic, and Series Levels}]\end{specHead} 
    \item[{Module}]
  core
    \item[{Attributes}]
  Attributes att.global (\textit{@xml:id}, \textit{@n}, \textit{@xml:lang}, \textit{@xml:base}, \textit{@xml:space})  (att.global.rendition (\textit{@rend}, \textit{@style}, \textit{@rendition})) (att.global.linking (\textit{@corresp}, \textit{@synch}, \textit{@sameAs}, \textit{@copyOf}, \textit{@next}, \textit{@prev}, \textit{@exclude}, \textit{@select})) (att.global.analytic (\textit{@ana})) (att.global.facs (\textit{@facs})) (att.global.change (\textit{@change})) (att.global.responsibility (\textit{@cert}, \textit{@resp})) (att.global.source (\textit{@source}))
    \item[{Contained by}]
  
    \item[core: ]
   biblStruct
    \item[{May contain}]
  
    \item[core: ]
   author biblScope editor imprint meeting note ptr ref respStmt textLang title\par 
    \item[header: ]
   authority availability edition extent funder idno sponsor\par 
    \item[textcrit: ]
   witDetail
    \item[{Note}]
  \par
May contain specialized bibliographic elements, in a prescribed order.\par
The <monogr> element may only occur only within a <biblStruct>, where its use is mandatory for the description of a monographic-level bibliographic item.
    \item[{Example}]
  \leavevmode\bgroup\exampleFont \begin{shaded}\noindent\mbox{}{<\textbf{biblStruct}>}\mbox{}\newline 
\hspace*{6pt}{<\textbf{analytic}>}\mbox{}\newline 
\hspace*{6pt}\hspace*{6pt}{<\textbf{author}>}Chesnutt, David{</\textbf{author}>}\mbox{}\newline 
\hspace*{6pt}\hspace*{6pt}{<\textbf{title}>}Historical Editions in the States{</\textbf{title}>}\mbox{}\newline 
\hspace*{6pt}{</\textbf{analytic}>}\mbox{}\newline 
\hspace*{6pt}{<\textbf{monogr}>}\mbox{}\newline 
\hspace*{6pt}\hspace*{6pt}{<\textbf{title}\hspace*{6pt}{level}="{j}">}Computers and the Humanities{</\textbf{title}>}\mbox{}\newline 
\hspace*{6pt}\hspace*{6pt}{<\textbf{imprint}>}\mbox{}\newline 
\hspace*{6pt}\hspace*{6pt}\hspace*{6pt}{<\textbf{date}\hspace*{6pt}{when}="{1991-12}">}(December, 1991):{</\textbf{date}>}\mbox{}\newline 
\hspace*{6pt}\hspace*{6pt}{</\textbf{imprint}>}\mbox{}\newline 
\hspace*{6pt}\hspace*{6pt}{<\textbf{biblScope}>}25.6{</\textbf{biblScope}>}\mbox{}\newline 
\hspace*{6pt}\hspace*{6pt}{<\textbf{biblScope}\hspace*{6pt}{from}="{377}"\hspace*{6pt}{to}="{380}"\hspace*{6pt}{unit}="{page}">}377–380{</\textbf{biblScope}>}\mbox{}\newline 
\hspace*{6pt}{</\textbf{monogr}>}\mbox{}\newline 
{</\textbf{biblStruct}>}\end{shaded}\egroup 


    \item[{Example}]
  \leavevmode\bgroup\exampleFont \begin{shaded}\noindent\mbox{}{<\textbf{biblStruct}\hspace*{6pt}{type}="{book}">}\mbox{}\newline 
\hspace*{6pt}{<\textbf{monogr}>}\mbox{}\newline 
\hspace*{6pt}\hspace*{6pt}{<\textbf{author}>}\mbox{}\newline 
\hspace*{6pt}\hspace*{6pt}\hspace*{6pt}{<\textbf{persName}>}\mbox{}\newline 
\hspace*{6pt}\hspace*{6pt}\hspace*{6pt}\hspace*{6pt}{<\textbf{forename}>}Leo Joachim{</\textbf{forename}>}\mbox{}\newline 
\hspace*{6pt}\hspace*{6pt}\hspace*{6pt}\hspace*{6pt}{<\textbf{surname}>}Frachtenberg{</\textbf{surname}>}\mbox{}\newline 
\hspace*{6pt}\hspace*{6pt}\hspace*{6pt}{</\textbf{persName}>}\mbox{}\newline 
\hspace*{6pt}\hspace*{6pt}{</\textbf{author}>}\mbox{}\newline 
\hspace*{6pt}\hspace*{6pt}{<\textbf{title}\hspace*{6pt}{level}="{m}"\hspace*{6pt}{type}="{main}">}Lower Umpqua Texts{</\textbf{title}>}\mbox{}\newline 
\hspace*{6pt}\hspace*{6pt}{<\textbf{imprint}>}\mbox{}\newline 
\hspace*{6pt}\hspace*{6pt}\hspace*{6pt}{<\textbf{pubPlace}>}New York{</\textbf{pubPlace}>}\mbox{}\newline 
\hspace*{6pt}\hspace*{6pt}\hspace*{6pt}{<\textbf{publisher}>}Columbia University Press{</\textbf{publisher}>}\mbox{}\newline 
\hspace*{6pt}\hspace*{6pt}\hspace*{6pt}{<\textbf{date}>}1914{</\textbf{date}>}\mbox{}\newline 
\hspace*{6pt}\hspace*{6pt}{</\textbf{imprint}>}\mbox{}\newline 
\hspace*{6pt}{</\textbf{monogr}>}\mbox{}\newline 
\hspace*{6pt}{<\textbf{series}>}\mbox{}\newline 
\hspace*{6pt}\hspace*{6pt}{<\textbf{title}\hspace*{6pt}{level}="{s}"\hspace*{6pt}{type}="{main}">}Columbia University Contributions to\mbox{}\newline 
\hspace*{6pt}\hspace*{6pt}\hspace*{6pt}\hspace*{6pt} Anthropology{</\textbf{title}>}\mbox{}\newline 
\hspace*{6pt}\hspace*{6pt}{<\textbf{biblScope}\hspace*{6pt}{unit}="{volume}">}4{</\textbf{biblScope}>}\mbox{}\newline 
\hspace*{6pt}{</\textbf{series}>}\mbox{}\newline 
{</\textbf{biblStruct}>}\end{shaded}\egroup 


    \item[{Content model}]
  \mbox{}\hfill\\[-10pt]\begin{Verbatim}[fontsize=\small]
<content>
 <sequence>
  <alternate minOccurs="0">
   <sequence>
    <alternate>
     <elementRef key="author"/>
     <elementRef key="editor"/>
     <elementRef key="meeting"/>
     <elementRef key="respStmt"/>
    </alternate>
    <alternate maxOccurs="unbounded"
     minOccurs="0">
     <elementRef key="author"/>
     <elementRef key="editor"/>
     <elementRef key="meeting"/>
     <elementRef key="respStmt"/>
    </alternate>
    <elementRef key="title"
     maxOccurs="unbounded" minOccurs="1"/>
    <alternate maxOccurs="unbounded"
     minOccurs="0">
     <classRef key="model.ptrLike"/>
     <elementRef key="idno"/>
     <elementRef key="textLang"/>
     <elementRef key="editor"/>
     <elementRef key="respStmt"/>
    </alternate>
   </sequence>
   <sequence>
    <alternate maxOccurs="unbounded"
     minOccurs="1">
     <elementRef key="title"/>
     <classRef key="model.ptrLike"/>
     <elementRef key="idno"/>
    </alternate>
    <alternate maxOccurs="unbounded"
     minOccurs="0">
     <elementRef key="textLang"/>
     <elementRef key="author"/>
     <elementRef key="editor"/>
     <elementRef key="meeting"/>
     <elementRef key="respStmt"/>
    </alternate>
   </sequence>
   <sequence>
    <elementRef key="authority"/>
    <elementRef key="idno"/>
   </sequence>
  </alternate>
  <elementRef key="availability"
   maxOccurs="unbounded" minOccurs="0"/>
  <classRef key="model.noteLike"
   maxOccurs="unbounded" minOccurs="0"/>
  <sequence maxOccurs="unbounded"
   minOccurs="0">
   <elementRef key="edition"/>
   <alternate maxOccurs="unbounded"
    minOccurs="0">
    <elementRef key="idno"/>
    <classRef key="model.ptrLike"/>
    <elementRef key="editor"/>
    <elementRef key="sponsor"/>
    <elementRef key="funder"/>
    <elementRef key="respStmt"/>
   </alternate>
  </sequence>
  <elementRef key="imprint"/>
  <alternate maxOccurs="unbounded"
   minOccurs="0">
   <elementRef key="imprint"/>
   <elementRef key="extent"/>
   <elementRef key="biblScope"/>
  </alternate>
 </sequence>
</content>
    
\end{Verbatim}

    \item[{Schema Declaration}]
  \mbox{}\hfill\\[-10pt]\begin{Verbatim}[fontsize=\small]
element monogr
{
   att.global.attributes,
   (
      (
         (
            ( author | editor | meeting | respStmt ),
            ( author | editor | meeting | respStmt )*,
            title+,
            ( model.ptrLike | idno | textLang | editor | respStmt )*
         )
       | (
            ( title | model.ptrLike | idno )+,
            ( textLang | author | editor | meeting | respStmt )*
         )
       | ( authority, idno )
      )?,
      availability*,
      model.noteLike*,
      (
         edition,
         ( idno | model.ptrLike | editor | sponsor | funder | respStmt )*
      )*,
      imprint,
      ( imprint | extent | biblScope )*
   )
}
\end{Verbatim}

\end{reflist}  \index{msContents=<msContents>|oddindex}\index{class=@class!<msContents>|oddindex}
\begin{reflist}
\item[]\begin{specHead}{TEI.msContents}{<msContents> }(manuscript contents) describes the intellectual content of a manuscript or manuscript part, either as a series of paragraphs or as a series of structured manuscript items. [\xref{http://www.tei-c.org/release/doc/tei-p5-doc/en/html/MS.html\#msco}{10.6. Intellectual Content}]\end{specHead} 
    \item[{Module}]
  msdescription
    \item[{Attributes}]
  Attributes att.global (\textit{@xml:id}, \textit{@n}, \textit{@xml:lang}, \textit{@xml:base}, \textit{@xml:space})  (att.global.rendition (\textit{@rend}, \textit{@style}, \textit{@rendition})) (att.global.linking (\textit{@corresp}, \textit{@synch}, \textit{@sameAs}, \textit{@copyOf}, \textit{@next}, \textit{@prev}, \textit{@exclude}, \textit{@select})) (att.global.analytic (\textit{@ana})) (att.global.facs (\textit{@facs})) (att.global.change (\textit{@change})) (att.global.responsibility (\textit{@cert}, \textit{@resp})) (att.global.source (\textit{@source})) att.msExcerpt (\textit{@defective}) \hfil\\[-10pt]\begin{sansreflist}
    \item[@class]
  identifies the text types or classifications applicable to this object by pointing to other elements or resources defining the classification concerned. 
\begin{reflist}
    \item[{Status}]
  Optional
    \item[{Datatype}]
  1–∞ occurrences of teidata.pointer separated by whitespace
\end{reflist}  
\end{sansreflist}  
    \item[{Contained by}]
  
    \item[msdescription: ]
   msDesc msFrag msPart
    \item[{May contain}]
  
    \item[core: ]
   p textLang\par 
    \item[linking: ]
   ab\par 
    \item[msdescription: ]
   msItem msItemStruct summary\par 
    \item[textstructure: ]
   titlePage
    \item[{Note}]
  \par
Unless it contains a simple prose description, this element should contain at least one of the elements <summary>, <msItem>, or <msItemStruct>. This constraint is not currently enforced by the schema.
    \item[{Example}]
  \leavevmode\bgroup\exampleFont \begin{shaded}\noindent\mbox{}{<\textbf{msContents}\hspace*{6pt}{class}="{\#sermons}">}\mbox{}\newline 
\hspace*{6pt}{<\textbf{p}>}A collection of Lollard sermons{</\textbf{p}>}\mbox{}\newline 
{</\textbf{msContents}>}\end{shaded}\egroup 


    \item[{Example}]
  \leavevmode\bgroup\exampleFont \begin{shaded}\noindent\mbox{}{<\textbf{msContents}>}\mbox{}\newline 
\hspace*{6pt}{<\textbf{msItem}\hspace*{6pt}{n}="{1}">}\mbox{}\newline 
\hspace*{6pt}\hspace*{6pt}{<\textbf{locus}>}fols. 5r-7v{</\textbf{locus}>}\mbox{}\newline 
\hspace*{6pt}\hspace*{6pt}{<\textbf{title}>}An ABC{</\textbf{title}>}\mbox{}\newline 
\hspace*{6pt}\hspace*{6pt}{<\textbf{bibl}>}\mbox{}\newline 
\hspace*{6pt}\hspace*{6pt}\hspace*{6pt}{<\textbf{title}>}IMEV{</\textbf{title}>}\mbox{}\newline 
\hspace*{6pt}\hspace*{6pt}\hspace*{6pt}{<\textbf{biblScope}>}239{</\textbf{biblScope}>}\mbox{}\newline 
\hspace*{6pt}\hspace*{6pt}{</\textbf{bibl}>}\mbox{}\newline 
\hspace*{6pt}{</\textbf{msItem}>}\mbox{}\newline 
\hspace*{6pt}{<\textbf{msItem}\hspace*{6pt}{n}="{2}">}\mbox{}\newline 
\hspace*{6pt}\hspace*{6pt}{<\textbf{locus}>}fols. 7v-8v{</\textbf{locus}>}\mbox{}\newline 
\hspace*{6pt}\hspace*{6pt}{<\textbf{title}\hspace*{6pt}{xml:lang}="{frm}">}Lenvoy de Chaucer a Scogan{</\textbf{title}>}\mbox{}\newline 
\hspace*{6pt}\hspace*{6pt}{<\textbf{bibl}>}\mbox{}\newline 
\hspace*{6pt}\hspace*{6pt}\hspace*{6pt}{<\textbf{title}>}IMEV{</\textbf{title}>}\mbox{}\newline 
\hspace*{6pt}\hspace*{6pt}\hspace*{6pt}{<\textbf{biblScope}>}3747{</\textbf{biblScope}>}\mbox{}\newline 
\hspace*{6pt}\hspace*{6pt}{</\textbf{bibl}>}\mbox{}\newline 
\hspace*{6pt}{</\textbf{msItem}>}\mbox{}\newline 
\hspace*{6pt}{<\textbf{msItem}\hspace*{6pt}{n}="{3}">}\mbox{}\newline 
\hspace*{6pt}\hspace*{6pt}{<\textbf{locus}>}fol. 8v{</\textbf{locus}>}\mbox{}\newline 
\hspace*{6pt}\hspace*{6pt}{<\textbf{title}>}Truth{</\textbf{title}>}\mbox{}\newline 
\hspace*{6pt}\hspace*{6pt}{<\textbf{bibl}>}\mbox{}\newline 
\hspace*{6pt}\hspace*{6pt}\hspace*{6pt}{<\textbf{title}>}IMEV{</\textbf{title}>}\mbox{}\newline 
\hspace*{6pt}\hspace*{6pt}\hspace*{6pt}{<\textbf{biblScope}>}809{</\textbf{biblScope}>}\mbox{}\newline 
\hspace*{6pt}\hspace*{6pt}{</\textbf{bibl}>}\mbox{}\newline 
\hspace*{6pt}{</\textbf{msItem}>}\mbox{}\newline 
\hspace*{6pt}{<\textbf{msItem}\hspace*{6pt}{n}="{4}">}\mbox{}\newline 
\hspace*{6pt}\hspace*{6pt}{<\textbf{locus}>}fols. 8v-10v{</\textbf{locus}>}\mbox{}\newline 
\hspace*{6pt}\hspace*{6pt}{<\textbf{title}>}Birds Praise of Love{</\textbf{title}>}\mbox{}\newline 
\hspace*{6pt}\hspace*{6pt}{<\textbf{bibl}>}\mbox{}\newline 
\hspace*{6pt}\hspace*{6pt}\hspace*{6pt}{<\textbf{title}>}IMEV{</\textbf{title}>}\mbox{}\newline 
\hspace*{6pt}\hspace*{6pt}\hspace*{6pt}{<\textbf{biblScope}>}1506{</\textbf{biblScope}>}\mbox{}\newline 
\hspace*{6pt}\hspace*{6pt}{</\textbf{bibl}>}\mbox{}\newline 
\hspace*{6pt}{</\textbf{msItem}>}\mbox{}\newline 
\hspace*{6pt}{<\textbf{msItem}\hspace*{6pt}{n}="{5}">}\mbox{}\newline 
\hspace*{6pt}\hspace*{6pt}{<\textbf{locus}>}fols. 10v-11v{</\textbf{locus}>}\mbox{}\newline 
\hspace*{6pt}\hspace*{6pt}{<\textbf{title}\hspace*{6pt}{xml:lang}="{la}">}De amico ad amicam{</\textbf{title}>}\mbox{}\newline 
\hspace*{6pt}\hspace*{6pt}{<\textbf{title}\hspace*{6pt}{xml:lang}="{la}">}Responcio{</\textbf{title}>}\mbox{}\newline 
\hspace*{6pt}\hspace*{6pt}{<\textbf{bibl}>}\mbox{}\newline 
\hspace*{6pt}\hspace*{6pt}\hspace*{6pt}{<\textbf{title}>}IMEV{</\textbf{title}>}\mbox{}\newline 
\hspace*{6pt}\hspace*{6pt}\hspace*{6pt}{<\textbf{biblScope}>}16 \& 19{</\textbf{biblScope}>}\mbox{}\newline 
\hspace*{6pt}\hspace*{6pt}{</\textbf{bibl}>}\mbox{}\newline 
\hspace*{6pt}{</\textbf{msItem}>}\mbox{}\newline 
\hspace*{6pt}{<\textbf{msItem}\hspace*{6pt}{n}="{6}">}\mbox{}\newline 
\hspace*{6pt}\hspace*{6pt}{<\textbf{locus}>}fols. 14r-126v{</\textbf{locus}>}\mbox{}\newline 
\hspace*{6pt}\hspace*{6pt}{<\textbf{title}>}Troilus and Criseyde{</\textbf{title}>}\mbox{}\newline 
\hspace*{6pt}\hspace*{6pt}{<\textbf{note}>}Bk. 1:71-Bk. 5:1701, with additional losses due to mutilation throughout{</\textbf{note}>}\mbox{}\newline 
\hspace*{6pt}{</\textbf{msItem}>}\mbox{}\newline 
{</\textbf{msContents}>}\end{shaded}\egroup 


    \item[{Content model}]
  \mbox{}\hfill\\[-10pt]\begin{Verbatim}[fontsize=\small]
<content>
 <alternate>
  <classRef key="model.pLike"
   maxOccurs="unbounded" minOccurs="1"/>
  <sequence>
   <elementRef key="summary" minOccurs="0"/>
   <elementRef key="textLang" minOccurs="0"/>
   <elementRef key="titlePage"
    minOccurs="0"/>
   <alternate maxOccurs="unbounded"
    minOccurs="0">
    <elementRef key="msItem"/>
    <elementRef key="msItemStruct"/>
   </alternate>
  </sequence>
 </alternate>
</content>
    
\end{Verbatim}

    \item[{Schema Declaration}]
  \mbox{}\hfill\\[-10pt]\begin{Verbatim}[fontsize=\small]
element msContents
{
   att.global.attributes,
   att.msExcerpt.attributes,
   attribute class { list { + } }?,
   (
      model.pLike+
    | ( summary?, textLang?, titlePage?, ( msItem | msItemStruct )* )
   )
}
\end{Verbatim}

\end{reflist}  \index{msDesc=<msDesc>|oddindex}
\begin{reflist}
\item[]\begin{specHead}{TEI.msDesc}{<msDesc> }(manuscript description) contains a description of a single identifiable manuscript or other text-bearing object. [\xref{http://www.tei-c.org/release/doc/tei-p5-doc/en/html/MS.html\#msov}{10.1. Overview}]\end{specHead} 
    \item[{Module}]
  msdescription
    \item[{Attributes}]
  Attributes att.global (\textit{@xml:id}, \textit{@n}, \textit{@xml:lang}, \textit{@xml:base}, \textit{@xml:space})  (att.global.rendition (\textit{@rend}, \textit{@style}, \textit{@rendition})) (att.global.linking (\textit{@corresp}, \textit{@synch}, \textit{@sameAs}, \textit{@copyOf}, \textit{@next}, \textit{@prev}, \textit{@exclude}, \textit{@select})) (att.global.analytic (\textit{@ana})) (att.global.facs (\textit{@facs})) (att.global.change (\textit{@change})) (att.global.responsibility (\textit{@cert}, \textit{@resp})) (att.global.source (\textit{@source})) att.sortable (\textit{@sortKey}) att.typed (\textit{@type}, \textit{@subtype}) att.declaring (\textit{@decls}) 
    \item[{Member of}]
  model.biblLike
    \item[{Contained by}]
  
    \item[core: ]
   add cit corr del desc emph head hi item l listBibl meeting note orig p q quote ref reg relatedItem said sic stage title unclear\par 
    \item[figures: ]
   cell figDesc figure\par 
    \item[header: ]
   change handNote licence rendition scriptNote sourceDesc tagUsage taxonomy typeNote\par 
    \item[linking: ]
   ab seg\par 
    \item[msdescription: ]
   accMat acquisition additions collation condition custEvent decoNote filiation foliation layout msItem musicNotation origin provenance signatures source summary support surrogates\par 
    \item[namesdates: ]
   climate event location occupation org person personGrp place population state terrain trait\par 
    \item[textcrit: ]
   lem rdg witness\par 
    \item[textstructure: ]
   argument body div docEdition epigraph imprimatur postscript salute signed titlePart trailer\par 
    \item[transcr: ]
   damage metamark mod restore retrace secl supplied surplus
    \item[{May contain}]
  
    \item[core: ]
   head p\par 
    \item[linking: ]
   ab\par 
    \item[msdescription: ]
   additional history msContents msFrag msIdentifier msPart physDesc
    \item[{Example}]
  \leavevmode\bgroup\exampleFont \begin{shaded}\noindent\mbox{}{<\textbf{msDesc}>}\mbox{}\newline 
\hspace*{6pt}{<\textbf{msIdentifier}>}\mbox{}\newline 
\hspace*{6pt}\hspace*{6pt}{<\textbf{settlement}>}Oxford{</\textbf{settlement}>}\mbox{}\newline 
\hspace*{6pt}\hspace*{6pt}{<\textbf{repository}>}Bodleian Library{</\textbf{repository}>}\mbox{}\newline 
\hspace*{6pt}\hspace*{6pt}{<\textbf{idno}\hspace*{6pt}{type}="{Bod}">}MS Poet. Rawl. D. 169.{</\textbf{idno}>}\mbox{}\newline 
\hspace*{6pt}{</\textbf{msIdentifier}>}\mbox{}\newline 
\hspace*{6pt}{<\textbf{msContents}>}\mbox{}\newline 
\hspace*{6pt}\hspace*{6pt}{<\textbf{msItem}>}\mbox{}\newline 
\hspace*{6pt}\hspace*{6pt}\hspace*{6pt}{<\textbf{author}>}Geoffrey Chaucer{</\textbf{author}>}\mbox{}\newline 
\hspace*{6pt}\hspace*{6pt}\hspace*{6pt}{<\textbf{title}>}The Canterbury Tales{</\textbf{title}>}\mbox{}\newline 
\hspace*{6pt}\hspace*{6pt}{</\textbf{msItem}>}\mbox{}\newline 
\hspace*{6pt}{</\textbf{msContents}>}\mbox{}\newline 
\hspace*{6pt}{<\textbf{physDesc}>}\mbox{}\newline 
\hspace*{6pt}\hspace*{6pt}{<\textbf{objectDesc}>}\mbox{}\newline 
\hspace*{6pt}\hspace*{6pt}\hspace*{6pt}{<\textbf{p}>}A parchment codex of 136 folios, measuring approx\mbox{}\newline 
\hspace*{6pt}\hspace*{6pt}\hspace*{6pt}\hspace*{6pt}\hspace*{6pt}\hspace*{6pt} 28 by 19 inches, and containing 24 quires.{</\textbf{p}>}\mbox{}\newline 
\hspace*{6pt}\hspace*{6pt}\hspace*{6pt}{<\textbf{p}>}The pages are margined and ruled throughout.{</\textbf{p}>}\mbox{}\newline 
\hspace*{6pt}\hspace*{6pt}\hspace*{6pt}{<\textbf{p}>}Four hands have been identified in the manuscript: the first 44\mbox{}\newline 
\hspace*{6pt}\hspace*{6pt}\hspace*{6pt}\hspace*{6pt}\hspace*{6pt}\hspace*{6pt} folios being written in two cursive anglicana scripts, while the\mbox{}\newline 
\hspace*{6pt}\hspace*{6pt}\hspace*{6pt}\hspace*{6pt}\hspace*{6pt}\hspace*{6pt} remainder is for the most part in a mixed secretary hand.{</\textbf{p}>}\mbox{}\newline 
\hspace*{6pt}\hspace*{6pt}{</\textbf{objectDesc}>}\mbox{}\newline 
\hspace*{6pt}{</\textbf{physDesc}>}\mbox{}\newline 
{</\textbf{msDesc}>}\end{shaded}\egroup 


    \item[{Content model}]
  \mbox{}\hfill\\[-10pt]\begin{Verbatim}[fontsize=\small]
<content>
 <sequence>
  <elementRef key="msIdentifier"/>
  <classRef key="model.headLike"
   maxOccurs="unbounded" minOccurs="0"/>
  <alternate>
   <classRef key="model.pLike"
    maxOccurs="unbounded" minOccurs="1"/>
   <sequence>
    <elementRef key="msContents"
     minOccurs="0"/>
    <elementRef key="physDesc"
     minOccurs="0"/>
    <elementRef key="history" minOccurs="0"/>
    <elementRef key="additional"
     minOccurs="0"/>
    <alternate>
     <elementRef key="msPart"
      maxOccurs="unbounded" minOccurs="0"/>
     <elementRef key="msFrag"
      maxOccurs="unbounded" minOccurs="0"/>
    </alternate>
   </sequence>
  </alternate>
 </sequence>
</content>
    
\end{Verbatim}

    \item[{Schema Declaration}]
  \mbox{}\hfill\\[-10pt]\begin{Verbatim}[fontsize=\small]
element msDesc
{
   att.global.attributes,
   att.sortable.attributes,
   att.typed.attributes,
   att.declaring.attributes,
   (
      msIdentifier,
      model.headLike*,
      (
         model.pLike+
       | (
            msContents?,
            physDesc?,
            history?,
            additional?,
            ( msPart* | msFrag* )
         )
      )
   )
}
\end{Verbatim}

\end{reflist}  \index{msFrag=<msFrag>|oddindex}
\begin{reflist}
\item[]\begin{specHead}{TEI.msFrag}{<msFrag> }(manuscript fragment) contains information about a fragment of a scattered manuscript now held as a single unit or bound into a larger manuscript. [\xref{http://www.tei-c.org/release/doc/tei-p5-doc/en/html/MS.html\#msfg}{10.11. Manuscript Fragments}]\end{specHead} 
    \item[{Module}]
  msdescription
    \item[{Attributes}]
  Attributes att.global (\textit{@xml:id}, \textit{@n}, \textit{@xml:lang}, \textit{@xml:base}, \textit{@xml:space})  (att.global.rendition (\textit{@rend}, \textit{@style}, \textit{@rendition})) (att.global.linking (\textit{@corresp}, \textit{@synch}, \textit{@sameAs}, \textit{@copyOf}, \textit{@next}, \textit{@prev}, \textit{@exclude}, \textit{@select})) (att.global.analytic (\textit{@ana})) (att.global.facs (\textit{@facs})) (att.global.change (\textit{@change})) (att.global.responsibility (\textit{@cert}, \textit{@resp})) (att.global.source (\textit{@source})) att.typed (\textit{@type}, \textit{@subtype}) 
    \item[{Contained by}]
  
    \item[msdescription: ]
   msDesc
    \item[{May contain}]
  
    \item[core: ]
   head p\par 
    \item[linking: ]
   ab\par 
    \item[msdescription: ]
   additional altIdentifier history msContents msIdentifier physDesc
    \item[{Example}]
  \leavevmode\bgroup\exampleFont \begin{shaded}\noindent\mbox{}{<\textbf{msDesc}>}\mbox{}\newline 
\hspace*{6pt}{<\textbf{msIdentifier}>}\mbox{}\newline 
\hspace*{6pt}\hspace*{6pt}{<\textbf{msName}\hspace*{6pt}{xml:lang}="{la}">}Codex Suprasliensis{</\textbf{msName}>}\mbox{}\newline 
\hspace*{6pt}{</\textbf{msIdentifier}>}\mbox{}\newline 
\hspace*{6pt}{<\textbf{msFrag}>}\mbox{}\newline 
\hspace*{6pt}\hspace*{6pt}{<\textbf{msIdentifier}>}\mbox{}\newline 
\hspace*{6pt}\hspace*{6pt}\hspace*{6pt}{<\textbf{settlement}>}Ljubljana{</\textbf{settlement}>}\mbox{}\newline 
\hspace*{6pt}\hspace*{6pt}\hspace*{6pt}{<\textbf{repository}>}Narodna in univerzitetna knjiznica{</\textbf{repository}>}\mbox{}\newline 
\hspace*{6pt}\hspace*{6pt}\hspace*{6pt}{<\textbf{idno}>}MS Kopitar 2{</\textbf{idno}>}\mbox{}\newline 
\hspace*{6pt}\hspace*{6pt}{</\textbf{msIdentifier}>}\mbox{}\newline 
\hspace*{6pt}\hspace*{6pt}{<\textbf{msContents}>}\mbox{}\newline 
\hspace*{6pt}\hspace*{6pt}\hspace*{6pt}{<\textbf{summary}>}Contains ff. 10 to 42 only{</\textbf{summary}>}\mbox{}\newline 
\hspace*{6pt}\hspace*{6pt}{</\textbf{msContents}>}\mbox{}\newline 
\hspace*{6pt}{</\textbf{msFrag}>}\mbox{}\newline 
\hspace*{6pt}{<\textbf{msFrag}>}\mbox{}\newline 
\hspace*{6pt}\hspace*{6pt}{<\textbf{msIdentifier}>}\mbox{}\newline 
\hspace*{6pt}\hspace*{6pt}\hspace*{6pt}{<\textbf{settlement}>}Warszawa{</\textbf{settlement}>}\mbox{}\newline 
\hspace*{6pt}\hspace*{6pt}\hspace*{6pt}{<\textbf{repository}>}Biblioteka Narodowa{</\textbf{repository}>}\mbox{}\newline 
\hspace*{6pt}\hspace*{6pt}\hspace*{6pt}{<\textbf{idno}>}BO 3.201{</\textbf{idno}>}\mbox{}\newline 
\hspace*{6pt}\hspace*{6pt}{</\textbf{msIdentifier}>}\mbox{}\newline 
\hspace*{6pt}{</\textbf{msFrag}>}\mbox{}\newline 
\hspace*{6pt}{<\textbf{msFrag}>}\mbox{}\newline 
\hspace*{6pt}\hspace*{6pt}{<\textbf{msIdentifier}>}\mbox{}\newline 
\hspace*{6pt}\hspace*{6pt}\hspace*{6pt}{<\textbf{settlement}>}Sankt-Peterburg{</\textbf{settlement}>}\mbox{}\newline 
\hspace*{6pt}\hspace*{6pt}\hspace*{6pt}{<\textbf{repository}>}Rossiiskaia natsional'naia biblioteka{</\textbf{repository}>}\mbox{}\newline 
\hspace*{6pt}\hspace*{6pt}\hspace*{6pt}{<\textbf{idno}>}Q.p.I.72{</\textbf{idno}>}\mbox{}\newline 
\hspace*{6pt}\hspace*{6pt}{</\textbf{msIdentifier}>}\mbox{}\newline 
\hspace*{6pt}{</\textbf{msFrag}>}\mbox{}\newline 
{</\textbf{msDesc}>}\end{shaded}\egroup 


    \item[{Content model}]
  \mbox{}\hfill\\[-10pt]\begin{Verbatim}[fontsize=\small]
<content>
 <sequence>
  <alternate>
   <elementRef key="altIdentifier"/>
   <elementRef key="msIdentifier"/>
  </alternate>
  <classRef key="model.headLike"
   maxOccurs="unbounded" minOccurs="0"/>
  <alternate>
   <classRef key="model.pLike"
    maxOccurs="unbounded" minOccurs="1"/>
   <sequence>
    <elementRef key="msContents"
     minOccurs="0"/>
    <elementRef key="physDesc"
     minOccurs="0"/>
    <elementRef key="history" minOccurs="0"/>
    <elementRef key="additional"
     minOccurs="0"/>
   </sequence>
  </alternate>
 </sequence>
</content>
    
\end{Verbatim}

    \item[{Schema Declaration}]
  \mbox{}\hfill\\[-10pt]\begin{Verbatim}[fontsize=\small]
element msFrag
{
   att.global.attributes,
   att.typed.attributes,
   (
      ( altIdentifier | msIdentifier ),
      model.headLike*,
      ( model.pLike+ | ( msContents?, physDesc?, history?, additional? ) )
   )
}
\end{Verbatim}

\end{reflist}  \index{msIdentifier=<msIdentifier>|oddindex}
\begin{reflist}
\item[]\begin{specHead}{TEI.msIdentifier}{<msIdentifier> }(manuscript identifier) contains the information required to identify the manuscript being described. [\xref{http://www.tei-c.org/release/doc/tei-p5-doc/en/html/MS.html\#msid}{10.4. The Manuscript Identifier}]\end{specHead} 
    \item[{Module}]
  msdescription
    \item[{Attributes}]
  Attributes att.global (\textit{@xml:id}, \textit{@n}, \textit{@xml:lang}, \textit{@xml:base}, \textit{@xml:space})  (att.global.rendition (\textit{@rend}, \textit{@style}, \textit{@rendition})) (att.global.linking (\textit{@corresp}, \textit{@synch}, \textit{@sameAs}, \textit{@copyOf}, \textit{@next}, \textit{@prev}, \textit{@exclude}, \textit{@select})) (att.global.analytic (\textit{@ana})) (att.global.facs (\textit{@facs})) (att.global.change (\textit{@change})) (att.global.responsibility (\textit{@cert}, \textit{@resp})) (att.global.source (\textit{@source}))
    \item[{Member of}]
  model.biblPart 
    \item[{Contained by}]
  
    \item[core: ]
   bibl\par 
    \item[msdescription: ]
   msDesc msFrag msPart
    \item[{May contain}]
  
    \item[header: ]
   idno\par 
    \item[msdescription: ]
   altIdentifier collection institution msName repository\par 
    \item[namesdates: ]
   bloc country district geogName placeName region settlement
    \item[{Example}]
  \leavevmode\bgroup\exampleFont \begin{shaded}\noindent\mbox{}{<\textbf{msIdentifier}>}\mbox{}\newline 
\hspace*{6pt}{<\textbf{settlement}>}San Marino{</\textbf{settlement}>}\mbox{}\newline 
\hspace*{6pt}{<\textbf{repository}>}Huntington Library{</\textbf{repository}>}\mbox{}\newline 
\hspace*{6pt}{<\textbf{idno}>}MS.El.26.C.9{</\textbf{idno}>}\mbox{}\newline 
{</\textbf{msIdentifier}>}\end{shaded}\egroup 


    \item[{Schematron}]
   <s:report test="not(parent::tei:msPart) and (local-name(*[1])='idno' or local-name(*[1])='altIdentifier'   or normalize-space(.)='')">An msIdentifier must contain either a repository or location of some type, or a manuscript name</s:report>
    \item[{Content model}]
  \mbox{}\hfill\\[-10pt]\begin{Verbatim}[fontsize=\small]
<content>
 <sequence>
  <sequence>
   <classRef expand="sequenceOptional"
    key="model.placeNamePart"/>
   <elementRef key="institution"
    minOccurs="0"/>
   <elementRef key="repository"
    minOccurs="0"/>
   <elementRef key="collection"
    maxOccurs="unbounded" minOccurs="0"/>
   <elementRef key="idno" minOccurs="0"/>
  </sequence>
  <alternate maxOccurs="unbounded"
   minOccurs="0">
   <elementRef key="msName"/>
   <elementRef key="altIdentifier"/>
  </alternate>
 </sequence>
</content>
    
\end{Verbatim}

    \item[{Schema Declaration}]
  \mbox{}\hfill\\[-10pt]\begin{Verbatim}[fontsize=\small]
element msIdentifier
{
   att.global.attributes,
   (
      (
         placeName?,
         bloc?,
         country?,
         region?,
         district?,
         settlement?,
         geogName?,
         institution?,
         repository?,
         collection*,
         idno?
      ),
      ( msName | altIdentifier )*
   )
}
\end{Verbatim}

\end{reflist}  \index{msItem=<msItem>|oddindex}\index{class=@class!<msItem>|oddindex}
\begin{reflist}
\item[]\begin{specHead}{TEI.msItem}{<msItem> }(manuscript item) describes an individual work or item within the intellectual content of a manuscript or manuscript part. [\xref{http://www.tei-c.org/release/doc/tei-p5-doc/en/html/MS.html\#mscoit}{10.6.1. The msItem and msItemStruct Elements}]\end{specHead} 
    \item[{Module}]
  msdescription
    \item[{Attributes}]
  Attributes att.global (\textit{@xml:id}, \textit{@n}, \textit{@xml:lang}, \textit{@xml:base}, \textit{@xml:space})  (att.global.rendition (\textit{@rend}, \textit{@style}, \textit{@rendition})) (att.global.linking (\textit{@corresp}, \textit{@synch}, \textit{@sameAs}, \textit{@copyOf}, \textit{@next}, \textit{@prev}, \textit{@exclude}, \textit{@select})) (att.global.analytic (\textit{@ana})) (att.global.facs (\textit{@facs})) (att.global.change (\textit{@change})) (att.global.responsibility (\textit{@cert}, \textit{@resp})) (att.global.source (\textit{@source})) att.msExcerpt (\textit{@defective}) \hfil\\[-10pt]\begin{sansreflist}
    \item[@class]
  identifies the text types or classifications applicable to this item by pointing to other elements or resources defining the classification concerned. 
\begin{reflist}
    \item[{Status}]
  Optional
    \item[{Datatype}]
  1–∞ occurrences of teidata.pointer separated by whitespace
\end{reflist}  
\end{sansreflist}  
    \item[{Member of}]
  model.msItemPart 
    \item[{Contained by}]
  
    \item[msdescription: ]
   msContents msItem
    \item[{May contain}]
  
    \item[analysis: ]
   interp interpGrp span spanGrp\par 
    \item[core: ]
   author bibl biblStruct cb cit editor gap gb graphic index lb listBibl meeting milestone note p pb quote respStmt textLang title\par 
    \item[figures: ]
   figure notatedMusic\par 
    \item[header: ]
   biblFull funder idno principal sponsor\par 
    \item[linking: ]
   ab alt altGrp anchor join joinGrp link linkGrp timeline\par 
    \item[msdescription: ]
   colophon decoNote explicit filiation finalRubric incipit locus locusGrp msDesc msItem msItemStruct rubric\par 
    \item[textcrit: ]
   app witDetail\par 
    \item[textstructure: ]
   argument byline docAuthor docDate docEdition docImprint docTitle epigraph imprimatur titlePart\par 
    \item[transcr: ]
   addSpan damageSpan delSpan fw listTranspose metamark space substJoin
    \item[{Example}]
  \leavevmode\bgroup\exampleFont \begin{shaded}\noindent\mbox{}{<\textbf{msItem}\hspace*{6pt}{class}="{\#saga}">}\mbox{}\newline 
\hspace*{6pt}{<\textbf{locus}>}ff. 1r-24v{</\textbf{locus}>}\mbox{}\newline 
\hspace*{6pt}{<\textbf{title}>}Agrip af Noregs konunga sögum{</\textbf{title}>}\mbox{}\newline 
\hspace*{6pt}{<\textbf{incipit}>}regi oc h{<\textbf{ex}>}ann{</\textbf{ex}>} setiho\mbox{}\newline 
\hspace*{6pt}{<\textbf{gap}\hspace*{6pt}{extent}="{7}"\hspace*{6pt}{reason}="{illegible}"/>}sc\mbox{}\newline 
\hspace*{6pt}\hspace*{6pt} heim se{<\textbf{ex}>}m{</\textbf{ex}>} þio{</\textbf{incipit}>}\mbox{}\newline 
\hspace*{6pt}{<\textbf{explicit}>}h{<\textbf{ex}>}on{</\textbf{ex}>} hev{<\textbf{ex}>}er{</\textbf{ex}>}\mbox{}\newline 
\hspace*{6pt}\hspace*{6pt}{<\textbf{ex}>}oc{</\textbf{ex}>}þa buit hesta .ij. aNan viþ\mbox{}\newline 
\hspace*{6pt}\hspace*{6pt} fé enh{<\textbf{ex}>}on{</\textbf{ex}>}o{<\textbf{ex}>}m{</\textbf{ex}>} aNan til\mbox{}\newline 
\hspace*{6pt}\hspace*{6pt} reiþ{<\textbf{ex}>}ar{</\textbf{ex}>}\mbox{}\newline 
\hspace*{6pt}{</\textbf{explicit}>}\mbox{}\newline 
\hspace*{6pt}{<\textbf{textLang}\hspace*{6pt}{mainLang}="{non}">}Old Norse/Icelandic{</\textbf{textLang}>}\mbox{}\newline 
{</\textbf{msItem}>}\end{shaded}\egroup 


    \item[{Content model}]
  \mbox{}\hfill\\[-10pt]\begin{Verbatim}[fontsize=\small]
<content>
 <sequence>
  <alternate maxOccurs="unbounded"
   minOccurs="0">
   <elementRef key="locus"/>
   <elementRef key="locusGrp"/>
  </alternate>
  <alternate>
   <classRef key="model.pLike"
    maxOccurs="unbounded" minOccurs="1"/>
   <alternate maxOccurs="unbounded"
    minOccurs="1">
    <classRef key="model.titlepagePart"/>
    <classRef key="model.msItemPart"/>
    <classRef key="model.global"/>
   </alternate>
  </alternate>
 </sequence>
</content>
    
\end{Verbatim}

    \item[{Schema Declaration}]
  \mbox{}\hfill\\[-10pt]\begin{Verbatim}[fontsize=\small]
element msItem
{
   att.global.attributes,
   att.msExcerpt.attributes,
   attribute class { list { + } }?,
   (
      ( locus | locusGrp )*,
      (
         model.pLike+
       | ( model.titlepagePart | model.msItemPart | model.global )+
      )
   )
}
\end{Verbatim}

\end{reflist}  \index{msItemStruct=<msItemStruct>|oddindex}\index{class=@class!<msItemStruct>|oddindex}
\begin{reflist}
\item[]\begin{specHead}{TEI.msItemStruct}{<msItemStruct> }(structured manuscript item) contains a structured description for an individual work or item within the intellectual content of a manuscript or manuscript part. [\xref{http://www.tei-c.org/release/doc/tei-p5-doc/en/html/MS.html\#mscoit}{10.6.1. The msItem and msItemStruct Elements}]\end{specHead} 
    \item[{Module}]
  msdescription
    \item[{Attributes}]
  Attributes att.global (\textit{@xml:id}, \textit{@n}, \textit{@xml:lang}, \textit{@xml:base}, \textit{@xml:space})  (att.global.rendition (\textit{@rend}, \textit{@style}, \textit{@rendition})) (att.global.linking (\textit{@corresp}, \textit{@synch}, \textit{@sameAs}, \textit{@copyOf}, \textit{@next}, \textit{@prev}, \textit{@exclude}, \textit{@select})) (att.global.analytic (\textit{@ana})) (att.global.facs (\textit{@facs})) (att.global.change (\textit{@change})) (att.global.responsibility (\textit{@cert}, \textit{@resp})) (att.global.source (\textit{@source})) att.msExcerpt (\textit{@defective}) \hfil\\[-10pt]\begin{sansreflist}
    \item[@class]
  identifies the text types or classifications applicable to this item by pointing to other elements or resources defining the classification concerned.
\begin{reflist}
    \item[{Status}]
  Optional
    \item[{Datatype}]
  1–∞ occurrences of teidata.pointer separated by whitespace
\end{reflist}  
\end{sansreflist}  
    \item[{Member of}]
  model.msItemPart 
    \item[{Contained by}]
  
    \item[msdescription: ]
   msContents msItem msItemStruct
    \item[{May contain}]
  
    \item[core: ]
   author bibl biblStruct listBibl note p respStmt textLang title\par 
    \item[linking: ]
   ab\par 
    \item[msdescription: ]
   colophon decoNote explicit filiation finalRubric incipit locus locusGrp msItemStruct rubric\par 
    \item[textcrit: ]
   witDetail
    \item[{Example}]
  \leavevmode\bgroup\exampleFont \begin{shaded}\noindent\mbox{}{<\textbf{msItemStruct}\hspace*{6pt}{class}="{\#biblComm}"\mbox{}\newline 
\hspace*{6pt}{defective}="{false}"\hspace*{6pt}{n}="{2}">}\mbox{}\newline 
\hspace*{6pt}{<\textbf{locus}\hspace*{6pt}{from}="{24v}"\hspace*{6pt}{to}="{97v}">}24v-97v{</\textbf{locus}>}\mbox{}\newline 
\hspace*{6pt}{<\textbf{author}>}Apringius de Beja{</\textbf{author}>}\mbox{}\newline 
\hspace*{6pt}{<\textbf{title}\hspace*{6pt}{type}="{uniform}"\hspace*{6pt}{xml:lang}="{la}">}Tractatus in Apocalypsin{</\textbf{title}>}\mbox{}\newline 
\hspace*{6pt}{<\textbf{rubric}>}Incipit Trac{<\textbf{supplied}\hspace*{6pt}{reason}="{omitted}">}ta{</\textbf{supplied}>}tus\mbox{}\newline 
\hspace*{6pt}\hspace*{6pt} in apoka{<\textbf{lb}/>}lipsin eruditissimi uiri {<\textbf{lb}/>} Apringi ep{<\textbf{ex}>}iscop{</\textbf{ex}>}i\mbox{}\newline 
\hspace*{6pt}\hspace*{6pt} Pacensis eccl{<\textbf{ex}>}esi{</\textbf{ex}>}e{</\textbf{rubric}>}\mbox{}\newline 
\hspace*{6pt}{<\textbf{finalRubric}>}EXPLIC{<\textbf{ex}>}IT{</\textbf{ex}>} EXPO{<\textbf{lb}/>}SITIO APOCALIPSIS\mbox{}\newline 
\hspace*{6pt}\hspace*{6pt} QVA{<\textbf{ex}>}M{</\textbf{ex}>} EXPOSVIT DOM{<\textbf{lb}/>}NVS APRINGIUS EP{<\textbf{ex}>}ISCOPU{</\textbf{ex}>}S.\mbox{}\newline 
\hspace*{6pt}\hspace*{6pt} DEO GR{<\textbf{ex}>}ACI{</\textbf{ex}>}AS AGO. FI{<\textbf{lb}/>}NITO LABORE ISTO.{</\textbf{finalRubric}>}\mbox{}\newline 
\hspace*{6pt}{<\textbf{bibl}>}\mbox{}\newline 
\hspace*{6pt}\hspace*{6pt}{<\textbf{ref}\hspace*{6pt}{target}="{http://amiBibl.xml\#Apringius1900}">}Apringius{</\textbf{ref}>}, ed. Férotin{</\textbf{bibl}>}\mbox{}\newline 
\hspace*{6pt}{<\textbf{textLang}\hspace*{6pt}{mainLang}="{la}">}Latin{</\textbf{textLang}>}\mbox{}\newline 
{</\textbf{msItemStruct}>}\end{shaded}\egroup 


    \item[{Content model}]
  \mbox{}\hfill\\[-10pt]\begin{Verbatim}[fontsize=\small]
<content>
 <sequence>
  <alternate minOccurs="0">
   <elementRef key="locus"/>
   <elementRef key="locusGrp"/>
  </alternate>
  <alternate>
   <classRef key="model.pLike"
    maxOccurs="unbounded" minOccurs="1"/>
   <sequence>
    <elementRef key="author"
     maxOccurs="unbounded" minOccurs="0"/>
    <elementRef key="respStmt"
     maxOccurs="unbounded" minOccurs="0"/>
    <elementRef key="title"
     maxOccurs="unbounded" minOccurs="0"/>
    <elementRef key="rubric" minOccurs="0"/>
    <elementRef key="incipit" minOccurs="0"/>
    <elementRef key="msItemStruct"
     maxOccurs="unbounded" minOccurs="0"/>
    <elementRef key="explicit"
     minOccurs="0"/>
    <elementRef key="finalRubric"
     minOccurs="0"/>
    <elementRef key="colophon"
     maxOccurs="unbounded" minOccurs="0"/>
    <elementRef key="decoNote"
     maxOccurs="unbounded" minOccurs="0"/>
    <elementRef key="listBibl"
     maxOccurs="unbounded" minOccurs="0"/>
    <alternate maxOccurs="unbounded"
     minOccurs="0">
     <elementRef key="bibl"/>
     <elementRef key="biblStruct"/>
    </alternate>
    <elementRef key="filiation"
     minOccurs="0"/>
    <classRef key="model.noteLike"
     maxOccurs="unbounded" minOccurs="0"/>
    <elementRef key="textLang"
     minOccurs="0"/>
   </sequence>
  </alternate>
 </sequence>
</content>
    
\end{Verbatim}

    \item[{Schema Declaration}]
  \mbox{}\hfill\\[-10pt]\begin{Verbatim}[fontsize=\small]
element msItemStruct
{
   att.global.attributes,
   att.msExcerpt.attributes,
   attribute class { list { + } }?,
   (
      ( locus | locusGrp )?,
      (
         model.pLike+
       | (
            author*,
            respStmt*,
            title*,
            rubric?,
            incipit?,
            msItemStruct*,
            explicit?,
            finalRubric?,
            colophon*,
            decoNote*,
            listBibl*,
            ( bibl | biblStruct )*,
            filiation?,
            model.noteLike*,
            textLang?
         )
      )
   )
}
\end{Verbatim}

\end{reflist}  \index{msName=<msName>|oddindex}
\begin{reflist}
\item[]\begin{specHead}{TEI.msName}{<msName> }(alternative name) contains any form of unstructured alternative name used for a manuscript, such as an ‘ocellus nominum’, or nickname. [\xref{http://www.tei-c.org/release/doc/tei-p5-doc/en/html/MS.html\#msid}{10.4. The Manuscript Identifier}]\end{specHead} 
    \item[{Module}]
  msdescription
    \item[{Attributes}]
  Attributes att.global (\textit{@xml:id}, \textit{@n}, \textit{@xml:lang}, \textit{@xml:base}, \textit{@xml:space})  (att.global.rendition (\textit{@rend}, \textit{@style}, \textit{@rendition})) (att.global.linking (\textit{@corresp}, \textit{@synch}, \textit{@sameAs}, \textit{@copyOf}, \textit{@next}, \textit{@prev}, \textit{@exclude}, \textit{@select})) (att.global.analytic (\textit{@ana})) (att.global.facs (\textit{@facs})) (att.global.change (\textit{@change})) (att.global.responsibility (\textit{@cert}, \textit{@resp})) (att.global.source (\textit{@source})) att.typed (\textit{@type}, \textit{@subtype}) 
    \item[{Contained by}]
  
    \item[msdescription: ]
   msIdentifier
    \item[{May contain}]
  
    \item[gaiji: ]
   g\par character data
    \item[{Example}]
  \leavevmode\bgroup\exampleFont \begin{shaded}\noindent\mbox{}{<\textbf{msName}>}The Vercelli Book{</\textbf{msName}>}\end{shaded}\egroup 


    \item[{Content model}]
  \fbox{\ttfamily <content>\newline
 <macroRef key="macro.xtext"/>\newline
</content>\newline
    } 
    \item[{Schema Declaration}]
  \mbox{}\hfill\\[-10pt]\begin{Verbatim}[fontsize=\small]
element msName { att.global.attributes, att.typed.attributes, macro.xtext }
\end{Verbatim}

\end{reflist}  \index{msPart=<msPart>|oddindex}
\begin{reflist}
\item[]\begin{specHead}{TEI.msPart}{<msPart> }(manuscript part) contains information about an originally distinct manuscript or part of a manuscript, which is now part of a composite manuscript. [\xref{http://www.tei-c.org/release/doc/tei-p5-doc/en/html/MS.html\#mspt}{10.10. Manuscript Parts}]\end{specHead} 
    \item[{Module}]
  msdescription
    \item[{Attributes}]
  Attributes att.global (\textit{@xml:id}, \textit{@n}, \textit{@xml:lang}, \textit{@xml:base}, \textit{@xml:space})  (att.global.rendition (\textit{@rend}, \textit{@style}, \textit{@rendition})) (att.global.linking (\textit{@corresp}, \textit{@synch}, \textit{@sameAs}, \textit{@copyOf}, \textit{@next}, \textit{@prev}, \textit{@exclude}, \textit{@select})) (att.global.analytic (\textit{@ana})) (att.global.facs (\textit{@facs})) (att.global.change (\textit{@change})) (att.global.responsibility (\textit{@cert}, \textit{@resp})) (att.global.source (\textit{@source})) att.typed (\textit{@type}, \textit{@subtype}) 
    \item[{Contained by}]
  
    \item[msdescription: ]
   msDesc msPart
    \item[{May contain}]
  
    \item[core: ]
   head p\par 
    \item[linking: ]
   ab\par 
    \item[msdescription: ]
   additional history msContents msIdentifier msPart physDesc
    \item[{Note}]
  \par
As this last example shows, for compatibility reasons the identifier of a manuscript part may be supplied as a simple <altIdentifier> rather than using the more structured <msIdentifier> element. This usage is however deprecated.
    \item[{Example}]
  \leavevmode\bgroup\exampleFont \begin{shaded}\noindent\mbox{}{<\textbf{msPart}>}\mbox{}\newline 
\hspace*{6pt}{<\textbf{msIdentifier}>}\mbox{}\newline 
\hspace*{6pt}\hspace*{6pt}{<\textbf{idno}>}A{</\textbf{idno}>}\mbox{}\newline 
\hspace*{6pt}\hspace*{6pt}{<\textbf{altIdentifier}\hspace*{6pt}{type}="{catalog}">}\mbox{}\newline 
\hspace*{6pt}\hspace*{6pt}\hspace*{6pt}{<\textbf{collection}>}Becker{</\textbf{collection}>}\mbox{}\newline 
\hspace*{6pt}\hspace*{6pt}\hspace*{6pt}{<\textbf{idno}>}48, Nr. 145{</\textbf{idno}>}\mbox{}\newline 
\hspace*{6pt}\hspace*{6pt}{</\textbf{altIdentifier}>}\mbox{}\newline 
\hspace*{6pt}\hspace*{6pt}{<\textbf{altIdentifier}\hspace*{6pt}{type}="{catalog}">}\mbox{}\newline 
\hspace*{6pt}\hspace*{6pt}\hspace*{6pt}{<\textbf{collection}>}Wiener Liste{</\textbf{collection}>}\mbox{}\newline 
\hspace*{6pt}\hspace*{6pt}\hspace*{6pt}{<\textbf{idno}>}4°5{</\textbf{idno}>}\mbox{}\newline 
\hspace*{6pt}\hspace*{6pt}{</\textbf{altIdentifier}>}\mbox{}\newline 
\hspace*{6pt}{</\textbf{msIdentifier}>}\mbox{}\newline 
\hspace*{6pt}{<\textbf{head}>}\mbox{}\newline 
\hspace*{6pt}\hspace*{6pt}{<\textbf{title}\hspace*{6pt}{xml:lang}="{la}">}Gregorius: Homiliae in Ezechielem{</\textbf{title}>}\mbox{}\newline 
\hspace*{6pt}\hspace*{6pt}{<\textbf{origPlace}\hspace*{6pt}{key}="{tgn\textunderscore 7008085}">}Weissenburg (?){</\textbf{origPlace}>}\mbox{}\newline 
\hspace*{6pt}\hspace*{6pt}{<\textbf{origDate}\hspace*{6pt}{notAfter}="{0815}"\mbox{}\newline 
\hspace*{6pt}\hspace*{6pt}\hspace*{6pt}{notBefore}="{0801}">}IX. Jh., Anfang{</\textbf{origDate}>}\mbox{}\newline 
\hspace*{6pt}{</\textbf{head}>}\mbox{}\newline 
{</\textbf{msPart}>}\end{shaded}\egroup 


    \item[{Example}]
  \leavevmode\bgroup\exampleFont \begin{shaded}\noindent\mbox{}{<\textbf{msDesc}>}\mbox{}\newline 
\hspace*{6pt}{<\textbf{msIdentifier}>}\mbox{}\newline 
\hspace*{6pt}\hspace*{6pt}{<\textbf{settlement}>}Amiens{</\textbf{settlement}>}\mbox{}\newline 
\hspace*{6pt}\hspace*{6pt}{<\textbf{repository}>}Bibliothèque Municipale{</\textbf{repository}>}\mbox{}\newline 
\hspace*{6pt}\hspace*{6pt}{<\textbf{idno}>}MS 3{</\textbf{idno}>}\mbox{}\newline 
\hspace*{6pt}\hspace*{6pt}{<\textbf{msName}>}Maurdramnus Bible{</\textbf{msName}>}\mbox{}\newline 
\hspace*{6pt}{</\textbf{msIdentifier}>}\mbox{}\newline 
\hspace*{6pt}{<\textbf{msContents}>}\mbox{}\newline 
\hspace*{6pt}\hspace*{6pt}{<\textbf{summary}\hspace*{6pt}{xml:lang}="{lat}">}Miscellany of various texts; Prudentius, Psychomachia; Physiologus de natura animantium{</\textbf{summary}>}\mbox{}\newline 
\hspace*{6pt}\hspace*{6pt}{<\textbf{textLang}\hspace*{6pt}{mainLang}="{lat}">}Latin{</\textbf{textLang}>}\mbox{}\newline 
\hspace*{6pt}{</\textbf{msContents}>}\mbox{}\newline 
\hspace*{6pt}{<\textbf{physDesc}>}\mbox{}\newline 
\hspace*{6pt}\hspace*{6pt}{<\textbf{objectDesc}\hspace*{6pt}{form}="{composite\textunderscore manuscript}"/>}\mbox{}\newline 
\hspace*{6pt}{</\textbf{physDesc}>}\mbox{}\newline 
\hspace*{6pt}{<\textbf{msPart}>}\mbox{}\newline 
\hspace*{6pt}\hspace*{6pt}{<\textbf{msIdentifier}>}\mbox{}\newline 
\hspace*{6pt}\hspace*{6pt}\hspace*{6pt}{<\textbf{idno}>}ms. 10066-77 ff. 140r-156v{</\textbf{idno}>}\mbox{}\newline 
\hspace*{6pt}\hspace*{6pt}{</\textbf{msIdentifier}>}\mbox{}\newline 
\hspace*{6pt}\hspace*{6pt}{<\textbf{msContents}>}\mbox{}\newline 
\hspace*{6pt}\hspace*{6pt}\hspace*{6pt}{<\textbf{summary}\hspace*{6pt}{xml:lang}="{lat}">}Physiologus{</\textbf{summary}>}\mbox{}\newline 
\hspace*{6pt}\hspace*{6pt}\hspace*{6pt}{<\textbf{textLang}\hspace*{6pt}{mainLang}="{lat}">}Latin{</\textbf{textLang}>}\mbox{}\newline 
\hspace*{6pt}\hspace*{6pt}{</\textbf{msContents}>}\mbox{}\newline 
\hspace*{6pt}{</\textbf{msPart}>}\mbox{}\newline 
\hspace*{6pt}{<\textbf{msPart}>}\mbox{}\newline 
\hspace*{6pt}\hspace*{6pt}{<\textbf{msIdentifier}>}\mbox{}\newline 
\hspace*{6pt}\hspace*{6pt}\hspace*{6pt}{<\textbf{altIdentifier}>}\mbox{}\newline 
\hspace*{6pt}\hspace*{6pt}\hspace*{6pt}\hspace*{6pt}{<\textbf{idno}>}MS 6{</\textbf{idno}>}\mbox{}\newline 
\hspace*{6pt}\hspace*{6pt}\hspace*{6pt}{</\textbf{altIdentifier}>}\mbox{}\newline 
\hspace*{6pt}\hspace*{6pt}{</\textbf{msIdentifier}>}\mbox{}\newline 
\textit{<!-- other information specific to this part here -->}\mbox{}\newline 
\hspace*{6pt}{</\textbf{msPart}>}\mbox{}\newline 
\textit{<!-- more parts here -->}\mbox{}\newline 
{</\textbf{msDesc}>}\end{shaded}\egroup 


    \item[{Content model}]
  \mbox{}\hfill\\[-10pt]\begin{Verbatim}[fontsize=\small]
<content>
 <sequence>
  <elementRef key="msIdentifier"/>
  <classRef key="model.headLike"
   maxOccurs="unbounded" minOccurs="0"/>
  <alternate>
   <classRef key="model.pLike"
    maxOccurs="unbounded" minOccurs="1"/>
   <sequence>
    <elementRef key="msContents"
     minOccurs="0"/>
    <elementRef key="physDesc"
     minOccurs="0"/>
    <elementRef key="history" minOccurs="0"/>
    <elementRef key="additional"
     minOccurs="0"/>
    <elementRef key="msPart"
     maxOccurs="unbounded" minOccurs="0"/>
   </sequence>
  </alternate>
 </sequence>
</content>
    
\end{Verbatim}

    \item[{Schema Declaration}]
  \mbox{}\hfill\\[-10pt]\begin{Verbatim}[fontsize=\small]
element msPart
{
   att.global.attributes,
   att.typed.attributes,
   (
      msIdentifier,
      model.headLike*,
      (
         model.pLike+
       | ( msContents?, physDesc?, history?, additional?, msPart* )
      )
   )
}
\end{Verbatim}

\end{reflist}  \index{musicNotation=<musicNotation>|oddindex}
\begin{reflist}
\item[]\begin{specHead}{TEI.musicNotation}{<musicNotation> }contains description of type of musical notation. [\xref{http://www.tei-c.org/release/doc/tei-p5-doc/en/html/MS.html\#msph2}{10.7.2. Writing, Decoration, and Other Notations}]\end{specHead} 
    \item[{Module}]
  msdescription
    \item[{Attributes}]
  Attributes att.global (\textit{@xml:id}, \textit{@n}, \textit{@xml:lang}, \textit{@xml:base}, \textit{@xml:space})  (att.global.rendition (\textit{@rend}, \textit{@style}, \textit{@rendition})) (att.global.linking (\textit{@corresp}, \textit{@synch}, \textit{@sameAs}, \textit{@copyOf}, \textit{@next}, \textit{@prev}, \textit{@exclude}, \textit{@select})) (att.global.analytic (\textit{@ana})) (att.global.facs (\textit{@facs})) (att.global.change (\textit{@change})) (att.global.responsibility (\textit{@cert}, \textit{@resp})) (att.global.source (\textit{@source}))
    \item[{Member of}]
  model.physDescPart
    \item[{Contained by}]
  
    \item[msdescription: ]
   physDesc
    \item[{May contain}]
  
    \item[analysis: ]
   c cl interp interpGrp m pc phr s span spanGrp w\par 
    \item[core: ]
   abbr add address bibl biblStruct cb choice cit corr date del desc distinct email emph expan foreign gap gb gloss graphic hi index l label lb lg list listBibl measure measureGrp media mentioned milestone name note num orig p pb ptr q quote ref reg rs said sic soCalled sp stage term time title unclear\par 
    \item[figures: ]
   figure formula notatedMusic table\par 
    \item[gaiji: ]
   g\par 
    \item[header: ]
   biblFull idno\par 
    \item[linking: ]
   ab alt altGrp anchor join joinGrp link linkGrp seg timeline\par 
    \item[msdescription: ]
   catchwords depth dim dimensions height heraldry locus locusGrp material msDesc objectType origDate origPlace secFol signatures stamp watermark width\par 
    \item[namesdates: ]
   addName affiliation bloc climate country district forename genName geo geogFeat geogName listEvent listNym listOrg listPerson listPlace location nameLink offset orgName persName placeName population region roleName settlement state surname terrain trait\par 
    \item[textcrit: ]
   app listApp listWit witDetail\par 
    \item[textstructure: ]
   floatingText\par 
    \item[transcr: ]
   addSpan am damage damageSpan delSpan ex fw handShift listTranspose metamark mod redo restore retrace secl space subst substJoin supplied surplus undo\par character data
    \item[{Example}]
  \leavevmode\bgroup\exampleFont \begin{shaded}\noindent\mbox{}{<\textbf{musicNotation}>}\mbox{}\newline 
\hspace*{6pt}{<\textbf{p}>}Square notation of 4-line red staves.{</\textbf{p}>}\mbox{}\newline 
{</\textbf{musicNotation}>}\end{shaded}\egroup 


    \item[{Example}]
  \leavevmode\bgroup\exampleFont \begin{shaded}\noindent\mbox{}{<\textbf{musicNotation}>}Neumes in {<\textbf{term}>}campo aperto{</\textbf{term}>} of the St. Gall type.\mbox{}\newline 
{</\textbf{musicNotation}>}\end{shaded}\egroup 


    \item[{Content model}]
  \mbox{}\hfill\\[-10pt]\begin{Verbatim}[fontsize=\small]
<content>
 <macroRef key="macro.specialPara"/>
</content>
    
\end{Verbatim}

    \item[{Schema Declaration}]
  \mbox{}\hfill\\[-10pt]\begin{Verbatim}[fontsize=\small]
element musicNotation { att.global.attributes, macro.specialPara }
\end{Verbatim}

\end{reflist}  \index{name=<name>|oddindex}
\begin{reflist}
\item[]\begin{specHead}{TEI.name}{<name> }(name, proper noun) contains a proper noun or noun phrase. [\xref{http://www.tei-c.org/release/doc/tei-p5-doc/en/html/CO.html\#CONARS}{3.5.1. Referring Strings}]\end{specHead} 
    \item[{Module}]
  core
    \item[{Attributes}]
  Attributes att.global (\textit{@xml:id}, \textit{@n}, \textit{@xml:lang}, \textit{@xml:base}, \textit{@xml:space})  (att.global.rendition (\textit{@rend}, \textit{@style}, \textit{@rendition})) (att.global.linking (\textit{@corresp}, \textit{@synch}, \textit{@sameAs}, \textit{@copyOf}, \textit{@next}, \textit{@prev}, \textit{@exclude}, \textit{@select})) (att.global.analytic (\textit{@ana})) (att.global.facs (\textit{@facs})) (att.global.change (\textit{@change})) (att.global.responsibility (\textit{@cert}, \textit{@resp})) (att.global.source (\textit{@source})) att.personal (\textit{@full}, \textit{@sort})  (att.naming (\textit{@role}, \textit{@nymRef}) (att.canonical (\textit{@key}, \textit{@ref})) ) att.datable (\textit{@calendar}, \textit{@period})  (att.datable.w3c (\textit{@when}, \textit{@notBefore}, \textit{@notAfter}, \textit{@from}, \textit{@to})) (att.datable.iso (\textit{@when-iso}, \textit{@notBefore-iso}, \textit{@notAfter-iso}, \textit{@from-iso}, \textit{@to-iso})) (att.datable.custom (\textit{@when-custom}, \textit{@notBefore-custom}, \textit{@notAfter-custom}, \textit{@from-custom}, \textit{@to-custom}, \textit{@datingPoint}, \textit{@datingMethod})) att.editLike (\textit{@evidence}, \textit{@instant})  (att.dimensions (\textit{@unit}, \textit{@quantity}, \textit{@extent}, \textit{@precision}, \textit{@scope}) (att.ranging (\textit{@atLeast}, \textit{@atMost}, \textit{@min}, \textit{@max}, \textit{@confidence})) ) att.typed (\textit{@type}, \textit{@subtype}) 
    \item[{Member of}]
  model.nameLike.agent
    \item[{Contained by}]
  
    \item[analysis: ]
   cl phr s span\par 
    \item[core: ]
   abbr add addrLine address author bibl biblScope citedRange corr date del desc distinct editor email emph expan foreign gloss head headItem headLabel hi item l label measure meeting mentioned name note num orig p pubPlace publisher q quote ref reg resp respStmt rs said sic soCalled speaker stage street term textLang time title unclear\par 
    \item[figures: ]
   cell figDesc\par 
    \item[header: ]
   authority catDesc change classCode correspAction creation distributor edition extent funder geoDecl handNote language licence principal rendition scriptNote sponsor tagUsage typeNote\par 
    \item[linking: ]
   ab seg\par 
    \item[msdescription: ]
   accMat acquisition additions catchwords collation colophon condition custEvent decoNote explicit filiation finalRubric foliation heraldry incipit layout material musicNotation objectType origDate origPlace origin provenance rubric secFol signatures source stamp summary support surrogates watermark\par 
    \item[namesdates: ]
   addName affiliation age birth bloc country death district education faith floruit forename genName geogFeat geogName langKnown nameLink nationality occupation offset org orgName persName placeName region residence roleName settlement sex socecStatus surname\par 
    \item[textcrit: ]
   lem rdg wit witDetail witness\par 
    \item[textstructure: ]
   byline closer dateline docAuthor docDate docEdition docImprint imprimatur opener salute signed titlePart trailer\par 
    \item[transcr: ]
   damage fw metamark mod restore retrace secl supplied surplus
    \item[{May contain}]
  
    \item[analysis: ]
   c cl interp interpGrp m pc phr s span spanGrp w\par 
    \item[core: ]
   abbr add address cb choice corr date del distinct email emph expan foreign gap gb gloss graphic hi index lb measure measureGrp media mentioned milestone name note num orig pb ptr ref reg rs sic soCalled term time title unclear\par 
    \item[figures: ]
   figure formula notatedMusic\par 
    \item[gaiji: ]
   g\par 
    \item[header: ]
   idno\par 
    \item[linking: ]
   alt altGrp anchor join joinGrp link linkGrp seg timeline\par 
    \item[msdescription: ]
   catchwords depth dim dimensions height heraldry locus locusGrp material objectType origDate origPlace secFol signatures stamp watermark width\par 
    \item[namesdates: ]
   addName affiliation bloc climate country district forename genName geo geogFeat geogName location nameLink offset orgName persName placeName population region roleName settlement state surname terrain trait\par 
    \item[textcrit: ]
   app witDetail\par 
    \item[transcr: ]
   addSpan am damage damageSpan delSpan ex fw handShift listTranspose metamark mod redo restore retrace secl space subst substJoin supplied surplus undo\par character data
    \item[{Note}]
  \par
Proper nouns referring to people, places, and organizations may be tagged instead with <persName>, <placeName>, or <orgName>, when the TEI module for names and dates is included.
    \item[{Example}]
  \leavevmode\bgroup\exampleFont \begin{shaded}\noindent\mbox{}{<\textbf{name}\hspace*{6pt}{type}="{person}">}Thomas Hoccleve{</\textbf{name}>}\mbox{}\newline 
{<\textbf{name}\hspace*{6pt}{type}="{place}">}Villingaholt{</\textbf{name}>}\mbox{}\newline 
{<\textbf{name}\hspace*{6pt}{type}="{org}">}Vetus Latina Institut{</\textbf{name}>}\mbox{}\newline 
{<\textbf{name}\hspace*{6pt}{ref}="{\#HOC001}"\hspace*{6pt}{type}="{person}">}Occleve{</\textbf{name}>}\end{shaded}\egroup 


    \item[{Content model}]
  \mbox{}\hfill\\[-10pt]\begin{Verbatim}[fontsize=\small]
<content>
 <macroRef key="macro.phraseSeq"/>
</content>
    
\end{Verbatim}

    \item[{Schema Declaration}]
  \mbox{}\hfill\\[-10pt]\begin{Verbatim}[fontsize=\small]
element name
{
   att.global.attributes,
   att.personal.attributes,
   att.datable.attributes,
   att.editLike.attributes,
   att.typed.attributes,
   macro.phraseSeq}
\end{Verbatim}

\end{reflist}  \index{nameLink=<nameLink>|oddindex}
\begin{reflist}
\item[]\begin{specHead}{TEI.nameLink}{<nameLink> }(name link) contains a connecting phrase or link used within a name but not regarded as part of it, such as \textit{van der} or \textit{of}. [\xref{http://www.tei-c.org/release/doc/tei-p5-doc/en/html/ND.html\#NDPER}{13.2.1. Personal Names}]\end{specHead} 
    \item[{Module}]
  namesdates
    \item[{Attributes}]
  Attributes att.global (\textit{@xml:id}, \textit{@n}, \textit{@xml:lang}, \textit{@xml:base}, \textit{@xml:space})  (att.global.rendition (\textit{@rend}, \textit{@style}, \textit{@rendition})) (att.global.linking (\textit{@corresp}, \textit{@synch}, \textit{@sameAs}, \textit{@copyOf}, \textit{@next}, \textit{@prev}, \textit{@exclude}, \textit{@select})) (att.global.analytic (\textit{@ana})) (att.global.facs (\textit{@facs})) (att.global.change (\textit{@change})) (att.global.responsibility (\textit{@cert}, \textit{@resp})) (att.global.source (\textit{@source})) att.typed (\textit{@type}, \textit{@subtype}) 
    \item[{Member of}]
  model.persNamePart
    \item[{Contained by}]
  
    \item[analysis: ]
   cl phr s span\par 
    \item[core: ]
   abbr add addrLine address author bibl biblScope citedRange corr date del desc distinct editor email emph expan foreign gloss head headItem headLabel hi item l label measure meeting mentioned name note num orig p pubPlace publisher q quote ref reg resp rs said sic soCalled speaker stage street term textLang time title unclear\par 
    \item[figures: ]
   cell figDesc\par 
    \item[header: ]
   authority catDesc change classCode correspAction creation distributor edition extent funder geoDecl handNote language licence principal rendition scriptNote sponsor tagUsage typeNote\par 
    \item[linking: ]
   ab seg\par 
    \item[msdescription: ]
   accMat acquisition additions catchwords collation colophon condition custEvent decoNote explicit filiation finalRubric foliation heraldry incipit layout material musicNotation objectType origDate origPlace origin provenance rubric secFol signatures source stamp summary support surrogates watermark\par 
    \item[namesdates: ]
   addName affiliation age birth bloc country death district education faith floruit forename genName geogFeat geogName langKnown nameLink nationality occupation offset org orgName persName placeName region residence roleName settlement sex socecStatus surname\par 
    \item[textcrit: ]
   lem rdg wit witDetail witness\par 
    \item[textstructure: ]
   byline closer dateline docAuthor docDate docEdition docImprint imprimatur opener salute signed titlePart trailer\par 
    \item[transcr: ]
   damage fw metamark mod restore retrace secl supplied surplus
    \item[{May contain}]
  
    \item[analysis: ]
   c cl interp interpGrp m pc phr s span spanGrp w\par 
    \item[core: ]
   abbr add address cb choice corr date del distinct email emph expan foreign gap gb gloss graphic hi index lb measure measureGrp media mentioned milestone name note num orig pb ptr ref reg rs sic soCalled term time title unclear\par 
    \item[figures: ]
   figure formula notatedMusic\par 
    \item[gaiji: ]
   g\par 
    \item[header: ]
   idno\par 
    \item[linking: ]
   alt altGrp anchor join joinGrp link linkGrp seg timeline\par 
    \item[msdescription: ]
   catchwords depth dim dimensions height heraldry locus locusGrp material objectType origDate origPlace secFol signatures stamp watermark width\par 
    \item[namesdates: ]
   addName affiliation bloc climate country district forename genName geo geogFeat geogName location nameLink offset orgName persName placeName population region roleName settlement state surname terrain trait\par 
    \item[textcrit: ]
   app witDetail\par 
    \item[transcr: ]
   addSpan am damage damageSpan delSpan ex fw handShift listTranspose metamark mod redo restore retrace secl space subst substJoin supplied surplus undo\par character data
    \item[{Example}]
  \leavevmode\bgroup\exampleFont \begin{shaded}\noindent\mbox{}{<\textbf{persName}>}\mbox{}\newline 
\hspace*{6pt}{<\textbf{forename}>}Frederick{</\textbf{forename}>}\mbox{}\newline 
\hspace*{6pt}{<\textbf{nameLink}>}van der{</\textbf{nameLink}>}\mbox{}\newline 
\hspace*{6pt}{<\textbf{surname}>}Tronck{</\textbf{surname}>}\mbox{}\newline 
{</\textbf{persName}>}\end{shaded}\egroup 


    \item[{Example}]
  \leavevmode\bgroup\exampleFont \begin{shaded}\noindent\mbox{}{<\textbf{persName}>}\mbox{}\newline 
\hspace*{6pt}{<\textbf{forename}>}Alfred{</\textbf{forename}>}\mbox{}\newline 
\hspace*{6pt}{<\textbf{nameLink}>}de{</\textbf{nameLink}>}\mbox{}\newline 
\hspace*{6pt}{<\textbf{surname}>}Musset{</\textbf{surname}>}\mbox{}\newline 
{</\textbf{persName}>}\end{shaded}\egroup 


    \item[{Content model}]
  \mbox{}\hfill\\[-10pt]\begin{Verbatim}[fontsize=\small]
<content>
 <macroRef key="macro.phraseSeq"/>
</content>
    
\end{Verbatim}

    \item[{Schema Declaration}]
  \mbox{}\hfill\\[-10pt]\begin{Verbatim}[fontsize=\small]
element nameLink
{
   att.global.attributes,
   att.typed.attributes,
   macro.phraseSeq}
\end{Verbatim}

\end{reflist}  \index{namespace=<namespace>|oddindex}\index{name=@name!<namespace>|oddindex}
\begin{reflist}
\item[]\begin{specHead}{TEI.namespace}{<namespace> }supplies the formal name of the namespace to which the elements documented by its children belong. [\xref{http://www.tei-c.org/release/doc/tei-p5-doc/en/html/HD.html\#HD57}{2.3.4. The Tagging Declaration}]\end{specHead} 
    \item[{Module}]
  header
    \item[{Attributes}]
  Attributes att.global (\textit{@xml:id}, \textit{@n}, \textit{@xml:lang}, \textit{@xml:base}, \textit{@xml:space})  (att.global.rendition (\textit{@rend}, \textit{@style}, \textit{@rendition})) (att.global.linking (\textit{@corresp}, \textit{@synch}, \textit{@sameAs}, \textit{@copyOf}, \textit{@next}, \textit{@prev}, \textit{@exclude}, \textit{@select})) (att.global.analytic (\textit{@ana})) (att.global.facs (\textit{@facs})) (att.global.change (\textit{@change})) (att.global.responsibility (\textit{@cert}, \textit{@resp})) (att.global.source (\textit{@source})) \hfil\\[-10pt]\begin{sansreflist}
    \item[@name]
  specifies the full formal name of the namespace concerned.
\begin{reflist}
    \item[{Status}]
  Required
    \item[{Datatype}]
  teidata.namespace
\end{reflist}  
\end{sansreflist}  
    \item[{Contained by}]
  
    \item[header: ]
   tagsDecl
    \item[{May contain}]
  
    \item[header: ]
   tagUsage
    \item[{Example}]
  \leavevmode\bgroup\exampleFont \begin{shaded}\noindent\mbox{}{<\textbf{namespace}\hspace*{6pt}{name}="{http://www.tei-c.org/ns/1.0}">}\mbox{}\newline 
\hspace*{6pt}{<\textbf{tagUsage}\hspace*{6pt}{gi}="{hi}"\hspace*{6pt}{occurs}="{28}"\hspace*{6pt}{withId}="{2}">} Used only to mark English words\mbox{}\newline 
\hspace*{6pt}\hspace*{6pt} italicized in the copy text {</\textbf{tagUsage}>}\mbox{}\newline 
{</\textbf{namespace}>}\end{shaded}\egroup 


    \item[{Content model}]
  \mbox{}\hfill\\[-10pt]\begin{Verbatim}[fontsize=\small]
<content>
 <elementRef key="tagUsage"
  maxOccurs="unbounded" minOccurs="1"/>
</content>
    
\end{Verbatim}

    \item[{Schema Declaration}]
  \mbox{}\hfill\\[-10pt]\begin{Verbatim}[fontsize=\small]
element namespace { att.global.attributes, attribute name { text }, tagUsage+ }
\end{Verbatim}

\end{reflist}  \index{nationality=<nationality>|oddindex}
\begin{reflist}
\item[]\begin{specHead}{TEI.nationality}{<nationality> }contains an informal description of a person's present or past nationality or citizenship. [\xref{http://www.tei-c.org/release/doc/tei-p5-doc/en/html/CC.html\#CCAHPA}{15.2.2. The Participant Description}]\end{specHead} 
    \item[{Module}]
  namesdates
    \item[{Attributes}]
  Attributes att.global (\textit{@xml:id}, \textit{@n}, \textit{@xml:lang}, \textit{@xml:base}, \textit{@xml:space})  (att.global.rendition (\textit{@rend}, \textit{@style}, \textit{@rendition})) (att.global.linking (\textit{@corresp}, \textit{@synch}, \textit{@sameAs}, \textit{@copyOf}, \textit{@next}, \textit{@prev}, \textit{@exclude}, \textit{@select})) (att.global.analytic (\textit{@ana})) (att.global.facs (\textit{@facs})) (att.global.change (\textit{@change})) (att.global.responsibility (\textit{@cert}, \textit{@resp})) (att.global.source (\textit{@source})) att.datable (\textit{@calendar}, \textit{@period})  (att.datable.w3c (\textit{@when}, \textit{@notBefore}, \textit{@notAfter}, \textit{@from}, \textit{@to})) (att.datable.iso (\textit{@when-iso}, \textit{@notBefore-iso}, \textit{@notAfter-iso}, \textit{@from-iso}, \textit{@to-iso})) (att.datable.custom (\textit{@when-custom}, \textit{@notBefore-custom}, \textit{@notAfter-custom}, \textit{@from-custom}, \textit{@to-custom}, \textit{@datingPoint}, \textit{@datingMethod})) att.editLike (\textit{@evidence}, \textit{@instant})  (att.dimensions (\textit{@unit}, \textit{@quantity}, \textit{@extent}, \textit{@precision}, \textit{@scope}) (att.ranging (\textit{@atLeast}, \textit{@atMost}, \textit{@min}, \textit{@max}, \textit{@confidence})) ) att.naming (\textit{@role}, \textit{@nymRef})  (att.canonical (\textit{@key}, \textit{@ref}))
    \item[{Member of}]
  model.persStateLike
    \item[{Contained by}]
  
    \item[namesdates: ]
   person personGrp
    \item[{May contain}]
  
    \item[analysis: ]
   c cl interp interpGrp m pc phr s span spanGrp w\par 
    \item[core: ]
   abbr add address cb choice corr date del distinct email emph expan foreign gap gb gloss graphic hi index lb measure measureGrp media mentioned milestone name note num orig pb ptr ref reg rs sic soCalled term time title unclear\par 
    \item[figures: ]
   figure formula notatedMusic\par 
    \item[gaiji: ]
   g\par 
    \item[header: ]
   idno\par 
    \item[linking: ]
   alt altGrp anchor join joinGrp link linkGrp seg timeline\par 
    \item[msdescription: ]
   catchwords depth dim dimensions height heraldry locus locusGrp material objectType origDate origPlace secFol signatures stamp watermark width\par 
    \item[namesdates: ]
   addName affiliation bloc climate country district forename genName geo geogFeat geogName location nameLink offset orgName persName placeName population region roleName settlement state surname terrain trait\par 
    \item[textcrit: ]
   app witDetail\par 
    \item[transcr: ]
   addSpan am damage damageSpan delSpan ex fw handShift listTranspose metamark mod redo restore retrace secl space subst substJoin supplied surplus undo\par character data
    \item[{Example}]
  \leavevmode\bgroup\exampleFont \begin{shaded}\noindent\mbox{}{<\textbf{nationality}\hspace*{6pt}{key}="{US}"\hspace*{6pt}{notBefore}="{1966}">} Obtained US Citizenship in 1966{</\textbf{nationality}>}\end{shaded}\egroup 


    \item[{Content model}]
  \mbox{}\hfill\\[-10pt]\begin{Verbatim}[fontsize=\small]
<content>
 <macroRef key="macro.phraseSeq"/>
</content>
    
\end{Verbatim}

    \item[{Schema Declaration}]
  \mbox{}\hfill\\[-10pt]\begin{Verbatim}[fontsize=\small]
element nationality
{
   att.global.attributes,
   att.datable.attributes,
   att.editLike.attributes,
   att.naming.attributes,
   macro.phraseSeq}
\end{Verbatim}

\end{reflist}  \index{normalization=<normalization>|oddindex}\index{method=@method!<normalization>|oddindex}
\begin{reflist}
\item[]\begin{specHead}{TEI.normalization}{<normalization> }indicates the extent of normalization or regularization of the original source carried out in converting it to electronic form. [\xref{http://www.tei-c.org/release/doc/tei-p5-doc/en/html/HD.html\#HD53}{2.3.3. The Editorial Practices Declaration} \xref{http://www.tei-c.org/release/doc/tei-p5-doc/en/html/CC.html\#CCAS2}{15.3.2. Declarable Elements}]\end{specHead} 
    \item[{Module}]
  header
    \item[{Attributes}]
  Attributes att.global (\textit{@xml:id}, \textit{@n}, \textit{@xml:lang}, \textit{@xml:base}, \textit{@xml:space})  (att.global.rendition (\textit{@rend}, \textit{@style}, \textit{@rendition})) (att.global.linking (\textit{@corresp}, \textit{@synch}, \textit{@sameAs}, \textit{@copyOf}, \textit{@next}, \textit{@prev}, \textit{@exclude}, \textit{@select})) (att.global.analytic (\textit{@ana})) (att.global.facs (\textit{@facs})) (att.global.change (\textit{@change})) (att.global.responsibility (\textit{@cert}, \textit{@resp})) (att.global.source (\textit{@source})) att.declarable (\textit{@default}) \hfil\\[-10pt]\begin{sansreflist}
    \item[@method]
  indicates the method adopted to indicate normalizations within the text.
\begin{reflist}
    \item[{Status}]
  Optional
    \item[{Datatype}]
  teidata.enumerated
    \item[{Legal values are:}]
  \begin{description}

\item[{silent}]normalization made silently{[Default] }
\item[{markup}]normalization represented using markup
\end{description} 
\end{reflist}  
\end{sansreflist}  
    \item[{Member of}]
  model.editorialDeclPart
    \item[{Contained by}]
  
    \item[header: ]
   editorialDecl
    \item[{May contain}]
  
    \item[core: ]
   p\par 
    \item[linking: ]
   ab
    \item[{Example}]
  \leavevmode\bgroup\exampleFont \begin{shaded}\noindent\mbox{}{<\textbf{editorialDecl}>}\mbox{}\newline 
\hspace*{6pt}{<\textbf{normalization}\hspace*{6pt}{method}="{markup}">}\mbox{}\newline 
\hspace*{6pt}\hspace*{6pt}{<\textbf{p}>}Where both upper- and lower-case i, j, u, v, and vv have been normalized, to modern\mbox{}\newline 
\hspace*{6pt}\hspace*{6pt}\hspace*{6pt}\hspace*{6pt} 20th century typographical practice, the {<\textbf{gi}>}choice{</\textbf{gi}>} element has been used to\mbox{}\newline 
\hspace*{6pt}\hspace*{6pt}\hspace*{6pt}\hspace*{6pt} enclose {<\textbf{gi}>}orig{</\textbf{gi}>} and {<\textbf{gi}>}reg{</\textbf{gi}>} elements giving the original and new values\mbox{}\newline 
\hspace*{6pt}\hspace*{6pt}\hspace*{6pt}\hspace*{6pt} respectively. ... {</\textbf{p}>}\mbox{}\newline 
\hspace*{6pt}{</\textbf{normalization}>}\mbox{}\newline 
\hspace*{6pt}{<\textbf{normalization}\hspace*{6pt}{method}="{silent}">}\mbox{}\newline 
\hspace*{6pt}\hspace*{6pt}{<\textbf{p}>}Spacing between words and following punctuation has been regularized to zero spaces;\mbox{}\newline 
\hspace*{6pt}\hspace*{6pt}\hspace*{6pt}\hspace*{6pt} spacing between words has been regularized to one space.{</\textbf{p}>}\mbox{}\newline 
\hspace*{6pt}{</\textbf{normalization}>}\mbox{}\newline 
\hspace*{6pt}{<\textbf{normalization}\hspace*{6pt}{source}="{http://www.dict.sztaki.hu/webster}">}\mbox{}\newline 
\hspace*{6pt}\hspace*{6pt}{<\textbf{p}>}Spelling converted throughout to Modern American usage, based on Websters 9th\mbox{}\newline 
\hspace*{6pt}\hspace*{6pt}\hspace*{6pt}\hspace*{6pt} Collegiate dictionary.{</\textbf{p}>}\mbox{}\newline 
\hspace*{6pt}{</\textbf{normalization}>}\mbox{}\newline 
{</\textbf{editorialDecl}>}\end{shaded}\egroup 


    \item[{Content model}]
  \mbox{}\hfill\\[-10pt]\begin{Verbatim}[fontsize=\small]
<content>
 <classRef key="model.pLike"
  maxOccurs="unbounded" minOccurs="1"/>
</content>
    
\end{Verbatim}

    \item[{Schema Declaration}]
  \mbox{}\hfill\\[-10pt]\begin{Verbatim}[fontsize=\small]
element normalization
{
   att.global.attributes,
   att.declarable.attributes,
   attribute method { "silent" | "markup" }?,
   model.pLike+
}
\end{Verbatim}

\end{reflist}  \index{notatedMusic=<notatedMusic>|oddindex}
\begin{reflist}
\item[]\begin{specHead}{TEI.notatedMusic}{<notatedMusic> }encodes the presence of music notation in a text [\xref{http://www.tei-c.org/release/doc/tei-p5-doc/en/html/FT.html\#FTNM}{14.3. Notated Music in Written Text}]\end{specHead} 
    \item[{Module}]
  figures
    \item[{Attributes}]
  Attributes att.global (\textit{@xml:id}, \textit{@n}, \textit{@xml:lang}, \textit{@xml:base}, \textit{@xml:space})  (att.global.rendition (\textit{@rend}, \textit{@style}, \textit{@rendition})) (att.global.linking (\textit{@corresp}, \textit{@synch}, \textit{@sameAs}, \textit{@copyOf}, \textit{@next}, \textit{@prev}, \textit{@exclude}, \textit{@select})) (att.global.analytic (\textit{@ana})) (att.global.facs (\textit{@facs})) (att.global.change (\textit{@change})) (att.global.responsibility (\textit{@cert}, \textit{@resp})) (att.global.source (\textit{@source})) att.placement (\textit{@place}) att.typed (\textit{@type}, \textit{@subtype}) 
    \item[{Member of}]
  model.global
    \item[{Contained by}]
  
    \item[analysis: ]
   cl m phr s span w\par 
    \item[core: ]
   abbr add addrLine address author bibl biblScope cit citedRange corr date del distinct editor email emph expan foreign gloss head headItem headLabel hi imprint item l label lg list measure mentioned name note num orig p pubPlace publisher q quote ref reg resp rs said series sic soCalled sp speaker stage street term textLang time title unclear\par 
    \item[figures: ]
   cell figure table\par 
    \item[header: ]
   authority change classCode distributor edition extent funder geoDecl handNote language licence principal scriptNote sponsor typeNote\par 
    \item[linking: ]
   ab seg\par 
    \item[msdescription: ]
   accMat acquisition additions catchwords collation colophon condition custEvent decoNote explicit filiation finalRubric foliation heraldry incipit layout material msItem musicNotation objectType origDate origPlace origin provenance rubric secFol signatures source stamp summary support surrogates watermark\par 
    \item[namesdates: ]
   addName affiliation age birth bloc country death district education faith floruit forename genName geogFeat geogName langKnown nameLink nationality occupation offset orgName persName person personGrp placeName region residence roleName settlement sex socecStatus surname\par 
    \item[textcrit: ]
   lem rdg wit witDetail\par 
    \item[textstructure: ]
   argument back body byline closer dateline div docAuthor docDate docEdition docImprint docTitle epigraph floatingText front group imprimatur opener postscript salute signed text titlePage titlePart trailer\par 
    \item[transcr: ]
   damage fw line metamark mod restore retrace secl sourceDoc supplied surface surfaceGrp surplus zone
    \item[{May contain}]
  
    \item[core: ]
   desc graphic label ptr ref\par 
    \item[linking: ]
   seg
    \item[{Note}]
  \par
It is possible to describe the content of the notation using elements from the \textsf{model.labelLike} class and it is possible to point to an external representation using elements from \textsf{model.ptrLike}. It is possible to specify the location of digital objects representing the notated music in other media such as images or audio-visual files. The encoder's interpretation of the correspondence between the notated music and these digital objects is not encoded explicitly. We recommend the use of graphic and binaryObject mainly as a fallback mechanism when the notated music format is not displayable by the application using the encoding. The alignment of encoded notated music, images carrying the notation, and audio files is a complex matter for which we refer the encoder to other formats and specifications such as MPEG-SMR.
    \item[{Example}]
  \leavevmode\bgroup\exampleFont \begin{shaded}\noindent\mbox{}{<\textbf{notatedMusic}>}\mbox{}\newline 
\hspace*{6pt}{<\textbf{ptr}\hspace*{6pt}{target}="{bar1.xml}"/>}\mbox{}\newline 
\hspace*{6pt}{<\textbf{graphic}\hspace*{6pt}{url}="{bar1.jpg}"/>}\mbox{}\newline 
\hspace*{6pt}{<\textbf{desc}>}First bar of Chopin's Scherzo No.3 Op.39{</\textbf{desc}>}\mbox{}\newline 
{</\textbf{notatedMusic}>}\end{shaded}\egroup 


    \item[{Content model}]
  \mbox{}\hfill\\[-10pt]\begin{Verbatim}[fontsize=\small]
<content>
 <alternate maxOccurs="unbounded"
  minOccurs="0">
  <classRef key="model.labelLike"/>
  <classRef key="model.ptrLike"/>
  <elementRef key="graphic"/>
  <elementRef key="binaryObject"/>
  <elementRef key="seg"/>
 </alternate>
</content>
    
\end{Verbatim}

    \item[{Schema Declaration}]
  \mbox{}\hfill\\[-10pt]\begin{Verbatim}[fontsize=\small]
element notatedMusic
{
   att.global.attributes,
   att.placement.attributes,
   att.typed.attributes,
   ( model.labelLike | model.ptrLike | graphic | binaryObject | seg )*
}
\end{Verbatim}

\end{reflist}  \index{note=<note>|oddindex}\index{anchored=@anchored!<note>|oddindex}\index{targetEnd=@targetEnd!<note>|oddindex}
\begin{reflist}
\item[]\begin{specHead}{TEI.note}{<note> }contains a note or annotation. [\xref{http://www.tei-c.org/release/doc/tei-p5-doc/en/html/CO.html\#CONONO}{3.8.1. Notes and Simple Annotation} \xref{http://www.tei-c.org/release/doc/tei-p5-doc/en/html/HD.html\#HD27}{2.2.6. The Notes Statement} \xref{http://www.tei-c.org/release/doc/tei-p5-doc/en/html/CO.html\#COBICON}{3.11.2.8. Notes and Statement of Language} \xref{http://www.tei-c.org/release/doc/tei-p5-doc/en/html/DI.html\#DITPNO}{9.3.5.4. Notes within Entries}]\end{specHead} 
    \item[{Module}]
  core
    \item[{Attributes}]
  Attributes att.global (\textit{@xml:id}, \textit{@n}, \textit{@xml:lang}, \textit{@xml:base}, \textit{@xml:space})  (att.global.rendition (\textit{@rend}, \textit{@style}, \textit{@rendition})) (att.global.linking (\textit{@corresp}, \textit{@synch}, \textit{@sameAs}, \textit{@copyOf}, \textit{@next}, \textit{@prev}, \textit{@exclude}, \textit{@select})) (att.global.analytic (\textit{@ana})) (att.global.facs (\textit{@facs})) (att.global.change (\textit{@change})) (att.global.responsibility (\textit{@cert}, \textit{@resp})) (att.global.source (\textit{@source})) att.placement (\textit{@place}) att.pointing (\textit{@targetLang}, \textit{@target}, \textit{@evaluate}) att.typed (\textit{@type}, \textit{@subtype}) att.written (\textit{@hand}) \hfil\\[-10pt]\begin{sansreflist}
    \item[@anchored]
  indicates whether the copy text shows the exact place of reference for the note.
\begin{reflist}
    \item[{Status}]
  Optional
    \item[{Datatype}]
  teidata.truthValue
    \item[{Default}]
  true
    \item[{Note}]
  \par
In modern texts, notes are usually anchored by means of explicit footnote or endnote symbols. An explicit indication of the phrase or line annotated may however be used instead (e.g. ‘page 218, lines 3–4’). The {\itshape anchored} attribute indicates whether any explicit location is given, whether by symbol or by prose cross-reference. The value true indicates that such an explicit location is indicated in the copy text; the value false indicates that the copy text does not indicate a specific place of attachment for the note. If the specific symbols used in the copy text at the location the note is anchored are to be recorded, use the {\itshape n} attribute.
\end{reflist}  
    \item[@targetEnd]
  points to the end of the span to which the note is attached, if the note is not embedded in the text at that point.
\begin{reflist}
    \item[{Status}]
  Optional
    \item[{Datatype}]
  1–∞ occurrences of teidata.pointer separated by whitespace
    \item[{Note}]
  \par
This attribute is retained for backwards compatibility; it may be removed at a subsequent release of the Guidelines. The recommended way of pointing to a span of elements is by means of the \textsf{range} function of XPointer, as further described in \xref{http://www.tei-c.org/release/doc/tei-p5-doc/en/html/SA.html\#SATSRN}{16.2.4.6. range()}.
\end{reflist}  
\end{sansreflist}  
    \item[{Member of}]
  model.correspActionPart model.correspContextPart model.correspDescPart model.noteLike
    \item[{Contained by}]
  
    \item[analysis: ]
   cl m phr s span w\par 
    \item[core: ]
   abbr add addrLine address author bibl biblScope biblStruct cit citedRange corr date del distinct editor email emph expan foreign gloss head headItem headLabel hi imprint item l label lg list measure mentioned monogr name note num orig p pubPlace publisher q quote ref reg resp rs said series sic soCalled sp speaker stage street term textLang time title unclear\par 
    \item[figures: ]
   cell figure table\par 
    \item[gaiji: ]
   char glyph\par 
    \item[header: ]
   authority change classCode correspAction correspContext correspDesc distributor edition extent funder geoDecl handNote language licence notesStmt principal scriptNote sponsor typeNote\par 
    \item[linking: ]
   ab seg\par 
    \item[msdescription: ]
   accMat acquisition additions adminInfo altIdentifier catchwords collation colophon condition custEvent decoNote explicit filiation finalRubric foliation heraldry incipit layout material msItem msItemStruct musicNotation objectType origDate origPlace origin provenance rubric secFol signatures source stamp summary support surrogates watermark\par 
    \item[namesdates: ]
   addName affiliation age birth bloc climate country death district education event faith floruit forename genName geogFeat geogName langKnown location nameLink nationality occupation offset org orgName persName person personGrp place placeName population region residence roleName settlement sex socecStatus state surname terrain trait\par 
    \item[textcrit: ]
   app lem rdg rdgGrp wit witDetail\par 
    \item[textstructure: ]
   argument back body byline closer dateline div docAuthor docDate docEdition docImprint docTitle epigraph floatingText front group imprimatur opener postscript salute signed text titlePage titlePart trailer\par 
    \item[transcr: ]
   damage fw line metamark mod restore retrace secl sourceDoc supplied surface surfaceGrp surplus zone
    \item[{May contain}]
  
    \item[analysis: ]
   c cl interp interpGrp m pc phr s span spanGrp w\par 
    \item[core: ]
   abbr add address bibl biblStruct cb choice cit corr date del desc distinct email emph expan foreign gap gb gloss graphic hi index l label lb lg list listBibl measure measureGrp media mentioned milestone name note num orig p pb ptr q quote ref reg rs said sic soCalled sp stage term time title unclear\par 
    \item[figures: ]
   figure formula notatedMusic table\par 
    \item[gaiji: ]
   g\par 
    \item[header: ]
   biblFull idno\par 
    \item[linking: ]
   ab alt altGrp anchor join joinGrp link linkGrp seg timeline\par 
    \item[msdescription: ]
   catchwords depth dim dimensions height heraldry locus locusGrp material msDesc objectType origDate origPlace secFol signatures stamp watermark width\par 
    \item[namesdates: ]
   addName affiliation bloc climate country district forename genName geo geogFeat geogName listEvent listNym listOrg listPerson listPlace location nameLink offset orgName persName placeName population region roleName settlement state surname terrain trait\par 
    \item[textcrit: ]
   app listApp listWit witDetail\par 
    \item[textstructure: ]
   floatingText\par 
    \item[transcr: ]
   addSpan am damage damageSpan delSpan ex fw handShift listTranspose metamark mod redo restore retrace secl space subst substJoin supplied surplus undo\par character data
    \item[{Example}]
  In the following example, the translator has supplied a footnote containing an explanation of the term translated as "painterly": \leavevmode\bgroup\exampleFont \begin{shaded}\noindent\mbox{}And yet it is not only\mbox{}\newline 
 in the great line of Italian renaissance art, but even in the\mbox{}\newline 
 painterly {<\textbf{note}\hspace*{6pt}{place}="{bottom}"\hspace*{6pt}{resp}="{\#MDMH}"\mbox{}\newline 
\hspace*{6pt}{type}="{gloss}">}\mbox{}\newline 
\hspace*{6pt}{<\textbf{term}\hspace*{6pt}{xml:lang}="{de}">}Malerisch{</\textbf{term}>}. This word has, in the German, two\mbox{}\newline 
 distinct meanings, one objective, a quality residing in the object,\mbox{}\newline 
 the other subjective, a mode of apprehension and creation. To avoid\mbox{}\newline 
 confusion, they have been distinguished in English as\mbox{}\newline 
{<\textbf{mentioned}>}picturesque{</\textbf{mentioned}>} and\mbox{}\newline 
{<\textbf{mentioned}>}painterly{</\textbf{mentioned}>} respectively.\mbox{}\newline 
{</\textbf{note}>} style of the\mbox{}\newline 
 Dutch genre painters of the seventeenth century that drapery has this\mbox{}\newline 
 psychological significance. \end{shaded}\egroup 

For this example to be valid, the code \textsf{MDMH} must be defined elsewhere, for example by means of a responsibility statement in the associated TEI header:\leavevmode\bgroup\exampleFont \begin{shaded}\noindent\mbox{}{<\textbf{respStmt}\hspace*{6pt}{xml:id}="{MDMH}">}\mbox{}\newline 
\hspace*{6pt}{<\textbf{resp}>}translation from German to English{</\textbf{resp}>}\mbox{}\newline 
\hspace*{6pt}{<\textbf{name}>}Hottinger, Marie Donald Mackie{</\textbf{name}>}\mbox{}\newline 
{</\textbf{respStmt}>}\end{shaded}\egroup 


    \item[{Example}]
  The global {\itshape n} attribute may be used to supply the symbol or number used to mark the note's point of attachment in the source text, as in the following example:\leavevmode\bgroup\exampleFont \begin{shaded}\noindent\mbox{}Mevorakh b. Saadya's mother, the matriarch of the\mbox{}\newline 
 family during the second half of the eleventh century, {<\textbf{note}\hspace*{6pt}{anchored}="{true}"\hspace*{6pt}{n}="{126}">} The\mbox{}\newline 
 alleged mention of Judah Nagid's mother in a letter from 1071 is, in fact, a reference to\mbox{}\newline 
 Judah's children; cf. above, nn. 111 and 54. {</\textbf{note}>} is well known from Geniza documents\mbox{}\newline 
 published by Jacob Mann.\end{shaded}\egroup 

However, if notes are numbered in sequence and their numbering can be reconstructed automatically by processing software, it may well be considered unnecessary to record the note numbers.
    \item[{Content model}]
  \mbox{}\hfill\\[-10pt]\begin{Verbatim}[fontsize=\small]
<content>
 <macroRef key="macro.specialPara"/>
</content>
    
\end{Verbatim}

    \item[{Schema Declaration}]
  \mbox{}\hfill\\[-10pt]\begin{Verbatim}[fontsize=\small]
element note
{
   att.global.attributes,
   att.placement.attributes,
   att.pointing.attributes,
   att.typed.attributes,
   att.written.attributes,
   attribute anchored { text }?,
   attribute targetEnd { list { + } }?,
   macro.specialPara}
\end{Verbatim}

\end{reflist}  \index{notesStmt=<notesStmt>|oddindex}
\begin{reflist}
\item[]\begin{specHead}{TEI.notesStmt}{<notesStmt> }(notes statement) collects together any notes providing information about a text additional to that recorded in other parts of the bibliographic description. [\xref{http://www.tei-c.org/release/doc/tei-p5-doc/en/html/HD.html\#HD27}{2.2.6. The Notes Statement} \xref{http://www.tei-c.org/release/doc/tei-p5-doc/en/html/HD.html\#HD2}{2.2. The File Description}]\end{specHead} 
    \item[{Module}]
  header
    \item[{Attributes}]
  Attributes att.global (\textit{@xml:id}, \textit{@n}, \textit{@xml:lang}, \textit{@xml:base}, \textit{@xml:space})  (att.global.rendition (\textit{@rend}, \textit{@style}, \textit{@rendition})) (att.global.linking (\textit{@corresp}, \textit{@synch}, \textit{@sameAs}, \textit{@copyOf}, \textit{@next}, \textit{@prev}, \textit{@exclude}, \textit{@select})) (att.global.analytic (\textit{@ana})) (att.global.facs (\textit{@facs})) (att.global.change (\textit{@change})) (att.global.responsibility (\textit{@cert}, \textit{@resp})) (att.global.source (\textit{@source}))
    \item[{Contained by}]
  
    \item[header: ]
   biblFull fileDesc
    \item[{May contain}]
  
    \item[core: ]
   note relatedItem\par 
    \item[textcrit: ]
   witDetail
    \item[{Note}]
  \par
Information of different kinds should not be grouped together into the same note.
    \item[{Example}]
  \leavevmode\bgroup\exampleFont \begin{shaded}\noindent\mbox{}{<\textbf{notesStmt}>}\mbox{}\newline 
\hspace*{6pt}{<\textbf{note}>}Historical commentary provided by Mark Cohen{</\textbf{note}>}\mbox{}\newline 
\hspace*{6pt}{<\textbf{note}>}OCR scanning done at University of Toronto{</\textbf{note}>}\mbox{}\newline 
{</\textbf{notesStmt}>}\end{shaded}\egroup 


    \item[{Content model}]
  \mbox{}\hfill\\[-10pt]\begin{Verbatim}[fontsize=\small]
<content>
 <alternate maxOccurs="unbounded"
  minOccurs="1">
  <classRef key="model.noteLike"/>
  <elementRef key="relatedItem"/>
 </alternate>
</content>
    
\end{Verbatim}

    \item[{Schema Declaration}]
  \mbox{}\hfill\\[-10pt]\begin{Verbatim}[fontsize=\small]
element notesStmt { att.global.attributes, ( model.noteLike | relatedItem )+ }
\end{Verbatim}

\end{reflist}  \index{num=<num>|oddindex}\index{type=@type!<num>|oddindex}\index{value=@value!<num>|oddindex}
\begin{reflist}
\item[]\begin{specHead}{TEI.num}{<num> }(number) contains a number, written in any form. [\xref{http://www.tei-c.org/release/doc/tei-p5-doc/en/html/CO.html\#CONANU}{3.5.3. Numbers and Measures}]\end{specHead} 
    \item[{Module}]
  core
    \item[{Attributes}]
  Attributes att.global (\textit{@xml:id}, \textit{@n}, \textit{@xml:lang}, \textit{@xml:base}, \textit{@xml:space})  (att.global.rendition (\textit{@rend}, \textit{@style}, \textit{@rendition})) (att.global.linking (\textit{@corresp}, \textit{@synch}, \textit{@sameAs}, \textit{@copyOf}, \textit{@next}, \textit{@prev}, \textit{@exclude}, \textit{@select})) (att.global.analytic (\textit{@ana})) (att.global.facs (\textit{@facs})) (att.global.change (\textit{@change})) (att.global.responsibility (\textit{@cert}, \textit{@resp})) (att.global.source (\textit{@source})) att.ranging (\textit{@atLeast}, \textit{@atMost}, \textit{@min}, \textit{@max}, \textit{@confidence}) \hfil\\[-10pt]\begin{sansreflist}
    \item[@type]
  indicates the type of numeric value.
\begin{reflist}
    \item[{Status}]
  Optional
    \item[{Datatype}]
  teidata.enumerated
    \item[{Suggested values include:}]
  \begin{description}

\item[{cardinal}]absolute number, e.g. 21, 21.5
\item[{ordinal}]ordinal number, e.g. 21st
\item[{fraction}]fraction, e.g. one half or three-quarters
\item[{percentage}]a percentage
\end{description} 
    \item[{Note}]
  \par
If a different typology is desired, other values can be used for this attribute.
\end{reflist}  
    \item[@value]
  supplies the value of the number in standard form.
\begin{reflist}
    \item[{Status}]
  Optional
    \item[{Datatype}]
  teidata.numeric
    \item[{Values}]
  a numeric value.
    \item[{Note}]
  \par
The standard form used is defined by the TEI datatype data.numeric.
\end{reflist}  
\end{sansreflist}  
    \item[{Member of}]
  model.measureLike
    \item[{Contained by}]
  
    \item[analysis: ]
   cl phr s span\par 
    \item[core: ]
   abbr add addrLine author bibl biblScope citedRange corr date del desc distinct editor email emph expan foreign gloss head headItem headLabel hi item l label measure measureGrp meeting mentioned name note num orig p pubPlace publisher q quote ref reg resp rs said sic soCalled speaker stage street term textLang time title unclear\par 
    \item[figures: ]
   cell figDesc\par 
    \item[header: ]
   authority catDesc change classCode creation distributor edition extent funder geoDecl handNote language licence principal rendition scriptNote sponsor tagUsage typeNote\par 
    \item[linking: ]
   ab seg\par 
    \item[msdescription: ]
   accMat acquisition additions catchwords collation colophon condition custEvent decoNote explicit filiation finalRubric foliation heraldry incipit layout material musicNotation objectType origDate origPlace origin provenance rubric secFol signatures source stamp summary support surrogates watermark\par 
    \item[namesdates: ]
   addName affiliation age birth bloc country death district education faith floruit forename genName geogFeat geogName langKnown location nameLink nationality occupation offset orgName persName placeName region residence roleName settlement sex socecStatus surname\par 
    \item[textcrit: ]
   lem rdg wit witDetail witness\par 
    \item[textstructure: ]
   byline closer dateline docAuthor docDate docEdition docImprint imprimatur opener salute signed titlePart trailer\par 
    \item[transcr: ]
   damage fw metamark mod restore retrace secl supplied surplus
    \item[{May contain}]
  
    \item[analysis: ]
   c cl interp interpGrp m pc phr s span spanGrp w\par 
    \item[core: ]
   abbr add address cb choice corr date del distinct email emph expan foreign gap gb gloss graphic hi index lb measure measureGrp media mentioned milestone name note num orig pb ptr ref reg rs sic soCalled term time title unclear\par 
    \item[figures: ]
   figure formula notatedMusic\par 
    \item[gaiji: ]
   g\par 
    \item[header: ]
   idno\par 
    \item[linking: ]
   alt altGrp anchor join joinGrp link linkGrp seg timeline\par 
    \item[msdescription: ]
   catchwords depth dim dimensions height heraldry locus locusGrp material objectType origDate origPlace secFol signatures stamp watermark width\par 
    \item[namesdates: ]
   addName affiliation bloc climate country district forename genName geo geogFeat geogName location nameLink offset orgName persName placeName population region roleName settlement state surname terrain trait\par 
    \item[textcrit: ]
   app witDetail\par 
    \item[transcr: ]
   addSpan am damage damageSpan delSpan ex fw handShift listTranspose metamark mod redo restore retrace secl space subst substJoin supplied surplus undo\par character data
    \item[{Note}]
  \par
Detailed analyses of quantities and units of measure in historical documents may also use the feature structure mechanism described in chapter \xref{http://www.tei-c.org/release/doc/tei-p5-doc/en/html/FS.html\#FS}{18. Feature Structures}. The <num> element is intended for use in simple applications.
    \item[{Example}]
  \leavevmode\bgroup\exampleFont \begin{shaded}\noindent\mbox{}{<\textbf{p}>}I reached {<\textbf{num}\hspace*{6pt}{type}="{cardinal}"\hspace*{6pt}{value}="{21}">}twenty-one{</\textbf{num}>} on\mbox{}\newline 
 my {<\textbf{num}\hspace*{6pt}{type}="{ordinal}"\hspace*{6pt}{value}="{21}">}twenty-first{</\textbf{num}>} birthday{</\textbf{p}>}\mbox{}\newline 
{<\textbf{p}>}Light travels at {<\textbf{num}\hspace*{6pt}{value}="{3E10}">}3×10{<\textbf{hi}\hspace*{6pt}{rend}="{sup}">}10{</\textbf{hi}>}\mbox{}\newline 
\hspace*{6pt}{</\textbf{num}>} cm per second.{</\textbf{p}>}\end{shaded}\egroup 


    \item[{Content model}]
  \mbox{}\hfill\\[-10pt]\begin{Verbatim}[fontsize=\small]
<content>
 <macroRef key="macro.phraseSeq"/>
</content>
    
\end{Verbatim}

    \item[{Schema Declaration}]
  \mbox{}\hfill\\[-10pt]\begin{Verbatim}[fontsize=\small]
element num
{
   att.global.attributes,
   att.ranging.attributes,
   attribute type { "cardinal" | "ordinal" | "fraction" | "percentage" }?,
   attribute value { text }?,
   macro.phraseSeq}
\end{Verbatim}

\end{reflist}  \index{nym=<nym>|oddindex}\index{parts=@parts!<nym>|oddindex}
\begin{reflist}
\item[]\begin{specHead}{TEI.nym}{<nym> }(canonical name) contains the definition for a canonical name or name component of any kind. [\xref{http://www.tei-c.org/release/doc/tei-p5-doc/en/html/ND.html\#NDNYM}{13.3.5. Names and Nyms}]\end{specHead} 
    \item[{Module}]
  namesdates
    \item[{Attributes}]
  Attributes att.global (\textit{@xml:id}, \textit{@n}, \textit{@xml:lang}, \textit{@xml:base}, \textit{@xml:space})  (att.global.rendition (\textit{@rend}, \textit{@style}, \textit{@rendition})) (att.global.linking (\textit{@corresp}, \textit{@synch}, \textit{@sameAs}, \textit{@copyOf}, \textit{@next}, \textit{@prev}, \textit{@exclude}, \textit{@select})) (att.global.analytic (\textit{@ana})) (att.global.facs (\textit{@facs})) (att.global.change (\textit{@change})) (att.global.responsibility (\textit{@cert}, \textit{@resp})) (att.global.source (\textit{@source})) att.typed (\textit{@type}, \textit{@subtype}) att.sortable (\textit{@sortKey}) \hfil\\[-10pt]\begin{sansreflist}
    \item[@parts]
  points to constituent nyms
\begin{reflist}
    \item[{Status}]
  Optional
    \item[{Datatype}]
  1–∞ occurrences of teidata.pointer separated by whitespace
\end{reflist}  
\end{sansreflist}  
    \item[{Contained by}]
  
    \item[namesdates: ]
   listNym nym
    \item[{May contain}]
  
    \item[core: ]
   p\par 
    \item[linking: ]
   ab\par 
    \item[namesdates: ]
   nym
    \item[{Example}]
  \leavevmode\bgroup\exampleFont \begin{shaded}\noindent\mbox{}{<\textbf{nym}\hspace*{6pt}{xml:id}="{J452}">}\mbox{}\newline 
\hspace*{6pt}{<\textbf{form}>}\mbox{}\newline 
\hspace*{6pt}\hspace*{6pt}{<\textbf{orth}\hspace*{6pt}{xml:lang}="{en-US}">}Ian{</\textbf{orth}>}\mbox{}\newline 
\hspace*{6pt}\hspace*{6pt}{<\textbf{orth}\hspace*{6pt}{xml:lang}="{en-x-Scots}">}Iain{</\textbf{orth}>}\mbox{}\newline 
\hspace*{6pt}{</\textbf{form}>}\mbox{}\newline 
{</\textbf{nym}>}\end{shaded}\egroup 


    \item[{Content model}]
  \mbox{}\hfill\\[-10pt]\begin{Verbatim}[fontsize=\small]
<content>
 <sequence>
  <classRef key="model.entryPart"
   maxOccurs="unbounded" minOccurs="0"/>
  <classRef key="model.pLike"
   maxOccurs="unbounded" minOccurs="0"/>
  <elementRef key="nym"
   maxOccurs="unbounded" minOccurs="0"/>
 </sequence>
</content>
    
\end{Verbatim}

    \item[{Schema Declaration}]
  \mbox{}\hfill\\[-10pt]\begin{Verbatim}[fontsize=\small]
element nym
{
   att.global.attributes,
   att.typed.attributes,
   att.sortable.attributes,
   attribute parts { list { + } }?,
   ( model.entryPart*, model.pLike*, nym* )
}
\end{Verbatim}

\end{reflist}  \index{objectDesc=<objectDesc>|oddindex}\index{form=@form!<objectDesc>|oddindex}
\begin{reflist}
\item[]\begin{specHead}{TEI.objectDesc}{<objectDesc> }contains a description of the physical components making up the object which is being described. [\xref{http://www.tei-c.org/release/doc/tei-p5-doc/en/html/MS.html\#msph1}{10.7.1. Object Description}]\end{specHead} 
    \item[{Module}]
  msdescription
    \item[{Attributes}]
  Attributes att.global (\textit{@xml:id}, \textit{@n}, \textit{@xml:lang}, \textit{@xml:base}, \textit{@xml:space})  (att.global.rendition (\textit{@rend}, \textit{@style}, \textit{@rendition})) (att.global.linking (\textit{@corresp}, \textit{@synch}, \textit{@sameAs}, \textit{@copyOf}, \textit{@next}, \textit{@prev}, \textit{@exclude}, \textit{@select})) (att.global.analytic (\textit{@ana})) (att.global.facs (\textit{@facs})) (att.global.change (\textit{@change})) (att.global.responsibility (\textit{@cert}, \textit{@resp})) (att.global.source (\textit{@source})) \hfil\\[-10pt]\begin{sansreflist}
    \item[@form]
  a short project-specific name identifying the physical form of the carrier, for example as a codex, roll, fragment, partial leaf, cutting etc.
\begin{reflist}
    \item[{Status}]
  Optional
    \item[{Datatype}]
  teidata.enumerated
    \item[{Note}]
  \par
Definitions for the terms used may typically be provided by a \texttt{<valList>} element in the project schema specification.
\end{reflist}  
\end{sansreflist}  
    \item[{Member of}]
  model.physDescPart
    \item[{Contained by}]
  
    \item[msdescription: ]
   physDesc
    \item[{May contain}]
  
    \item[core: ]
   p\par 
    \item[linking: ]
   ab\par 
    \item[msdescription: ]
   layoutDesc supportDesc
    \item[{Example}]
  \leavevmode\bgroup\exampleFont \begin{shaded}\noindent\mbox{}{<\textbf{objectDesc}\hspace*{6pt}{form}="{codex}">}\mbox{}\newline 
\hspace*{6pt}{<\textbf{supportDesc}\hspace*{6pt}{material}="{mixed}">}\mbox{}\newline 
\hspace*{6pt}\hspace*{6pt}{<\textbf{p}>}Early modern\mbox{}\newline 
\hspace*{6pt}\hspace*{6pt}{<\textbf{material}>}parchment{</\textbf{material}>} and\mbox{}\newline 
\hspace*{6pt}\hspace*{6pt}{<\textbf{material}>}paper{</\textbf{material}>}.{</\textbf{p}>}\mbox{}\newline 
\hspace*{6pt}{</\textbf{supportDesc}>}\mbox{}\newline 
\hspace*{6pt}{<\textbf{layoutDesc}>}\mbox{}\newline 
\hspace*{6pt}\hspace*{6pt}{<\textbf{layout}\hspace*{6pt}{ruledLines}="{25 32}"/>}\mbox{}\newline 
\hspace*{6pt}{</\textbf{layoutDesc}>}\mbox{}\newline 
{</\textbf{objectDesc}>}\end{shaded}\egroup 


    \item[{Content model}]
  \mbox{}\hfill\\[-10pt]\begin{Verbatim}[fontsize=\small]
<content>
 <alternate>
  <classRef key="model.pLike"
   maxOccurs="unbounded" minOccurs="1"/>
  <sequence>
   <elementRef key="supportDesc"
    minOccurs="0"/>
   <elementRef key="layoutDesc"
    minOccurs="0"/>
  </sequence>
 </alternate>
</content>
    
\end{Verbatim}

    \item[{Schema Declaration}]
  \mbox{}\hfill\\[-10pt]\begin{Verbatim}[fontsize=\small]
element objectDesc
{
   att.global.attributes,
   attribute form { text }?,
   ( model.pLike+ | ( supportDesc?, layoutDesc? ) )
}
\end{Verbatim}

\end{reflist}  \index{objectType=<objectType>|oddindex}
\begin{reflist}
\item[]\begin{specHead}{TEI.objectType}{<objectType> }contains a word or phrase describing the type of object being referred to. [\xref{http://www.tei-c.org/release/doc/tei-p5-doc/en/html/MS.html\#msmat}{10.3.2. Material and Object Type}]\end{specHead} 
    \item[{Module}]
  msdescription
    \item[{Attributes}]
  Attributes att.global (\textit{@xml:id}, \textit{@n}, \textit{@xml:lang}, \textit{@xml:base}, \textit{@xml:space})  (att.global.rendition (\textit{@rend}, \textit{@style}, \textit{@rendition})) (att.global.linking (\textit{@corresp}, \textit{@synch}, \textit{@sameAs}, \textit{@copyOf}, \textit{@next}, \textit{@prev}, \textit{@exclude}, \textit{@select})) (att.global.analytic (\textit{@ana})) (att.global.facs (\textit{@facs})) (att.global.change (\textit{@change})) (att.global.responsibility (\textit{@cert}, \textit{@resp})) (att.global.source (\textit{@source})) att.canonical (\textit{@key}, \textit{@ref}) 
    \item[{Member of}]
  model.pPart.msdesc
    \item[{Contained by}]
  
    \item[analysis: ]
   cl phr s span\par 
    \item[core: ]
   abbr add addrLine author biblScope citedRange corr date del desc distinct editor email emph expan foreign gloss head headItem headLabel hi item l label measure meeting mentioned name note num orig p pubPlace publisher q quote ref reg resp rs said sic soCalled speaker stage street term textLang time title unclear\par 
    \item[figures: ]
   cell figDesc\par 
    \item[header: ]
   authority catDesc change classCode creation distributor edition extent funder geoDecl handNote language licence principal rendition scriptNote sponsor tagUsage typeNote\par 
    \item[linking: ]
   ab seg\par 
    \item[msdescription: ]
   accMat acquisition additions catchwords collation colophon condition custEvent decoNote explicit filiation finalRubric foliation heraldry incipit layout material musicNotation objectType origDate origPlace origin provenance rubric secFol signatures source stamp summary support surrogates watermark\par 
    \item[namesdates: ]
   addName affiliation age birth bloc country death district education faith floruit forename genName geogFeat geogName langKnown nameLink nationality occupation offset orgName persName placeName region residence roleName settlement sex socecStatus surname\par 
    \item[textcrit: ]
   lem rdg wit witDetail witness\par 
    \item[textstructure: ]
   byline closer dateline docAuthor docDate docEdition docImprint imprimatur opener salute signed titlePart trailer\par 
    \item[transcr: ]
   damage fw metamark mod restore retrace secl supplied surplus
    \item[{May contain}]
  
    \item[analysis: ]
   c cl interp interpGrp m pc phr s span spanGrp w\par 
    \item[core: ]
   abbr add address cb choice corr date del distinct email emph expan foreign gap gb gloss graphic hi index lb measure measureGrp media mentioned milestone name note num orig pb ptr ref reg rs sic soCalled term time title unclear\par 
    \item[figures: ]
   figure formula notatedMusic\par 
    \item[gaiji: ]
   g\par 
    \item[header: ]
   idno\par 
    \item[linking: ]
   alt altGrp anchor join joinGrp link linkGrp seg timeline\par 
    \item[msdescription: ]
   catchwords depth dim dimensions height heraldry locus locusGrp material objectType origDate origPlace secFol signatures stamp watermark width\par 
    \item[namesdates: ]
   addName affiliation bloc climate country district forename genName geo geogFeat geogName location nameLink offset orgName persName placeName population region roleName settlement state surname terrain trait\par 
    \item[textcrit: ]
   app witDetail\par 
    \item[transcr: ]
   addSpan am damage damageSpan delSpan ex fw handShift listTranspose metamark mod redo restore retrace secl space subst substJoin supplied surplus undo\par character data
    \item[{Note}]
  \par
The {\itshape ref} attribute may be used to point to one or more items within a taxonomy of types of object, defined either internally or externally.
    \item[{Example}]
  \leavevmode\bgroup\exampleFont \begin{shaded}\noindent\mbox{}{<\textbf{physDesc}>}\mbox{}\newline 
\hspace*{6pt}{<\textbf{p}>} Paper and vellum {<\textbf{objectType}>}codex{</\textbf{objectType}>} in modern cloth binding.{</\textbf{p}>}\mbox{}\newline 
{</\textbf{physDesc}>}\end{shaded}\egroup 


    \item[{Example}]
  \leavevmode\bgroup\exampleFont \begin{shaded}\noindent\mbox{}{<\textbf{physDesc}>}\mbox{}\newline 
\hspace*{6pt}{<\textbf{p}>}Fragment of a re-used marble {<\textbf{objectType}>}funerary stele{</\textbf{objectType}>}.\mbox{}\newline 
\hspace*{6pt}{</\textbf{p}>}\mbox{}\newline 
{</\textbf{physDesc}>}\end{shaded}\egroup 


    \item[{Content model}]
  \mbox{}\hfill\\[-10pt]\begin{Verbatim}[fontsize=\small]
<content>
 <macroRef key="macro.phraseSeq"/>
</content>
    
\end{Verbatim}

    \item[{Schema Declaration}]
  \mbox{}\hfill\\[-10pt]\begin{Verbatim}[fontsize=\small]
element objectType
{
   att.global.attributes,
   att.canonical.attributes,
   macro.phraseSeq}
\end{Verbatim}

\end{reflist}  \index{occupation=<occupation>|oddindex}\index{scheme=@scheme!<occupation>|oddindex}\index{code=@code!<occupation>|oddindex}
\begin{reflist}
\item[]\begin{specHead}{TEI.occupation}{<occupation> }contains an informal description of a person's trade, profession or occupation. [\xref{http://www.tei-c.org/release/doc/tei-p5-doc/en/html/CC.html\#CCAHPA}{15.2.2. The Participant Description}]\end{specHead} 
    \item[{Module}]
  namesdates
    \item[{Attributes}]
  Attributes att.global (\textit{@xml:id}, \textit{@n}, \textit{@xml:lang}, \textit{@xml:base}, \textit{@xml:space})  (att.global.rendition (\textit{@rend}, \textit{@style}, \textit{@rendition})) (att.global.linking (\textit{@corresp}, \textit{@synch}, \textit{@sameAs}, \textit{@copyOf}, \textit{@next}, \textit{@prev}, \textit{@exclude}, \textit{@select})) (att.global.analytic (\textit{@ana})) (att.global.facs (\textit{@facs})) (att.global.change (\textit{@change})) (att.global.responsibility (\textit{@cert}, \textit{@resp})) (att.global.source (\textit{@source})) att.datable (\textit{@calendar}, \textit{@period})  (att.datable.w3c (\textit{@when}, \textit{@notBefore}, \textit{@notAfter}, \textit{@from}, \textit{@to})) (att.datable.iso (\textit{@when-iso}, \textit{@notBefore-iso}, \textit{@notAfter-iso}, \textit{@from-iso}, \textit{@to-iso})) (att.datable.custom (\textit{@when-custom}, \textit{@notBefore-custom}, \textit{@notAfter-custom}, \textit{@from-custom}, \textit{@to-custom}, \textit{@datingPoint}, \textit{@datingMethod})) att.editLike (\textit{@evidence}, \textit{@instant})  (att.dimensions (\textit{@unit}, \textit{@quantity}, \textit{@extent}, \textit{@precision}, \textit{@scope}) (att.ranging (\textit{@atLeast}, \textit{@atMost}, \textit{@min}, \textit{@max}, \textit{@confidence})) ) att.naming (\textit{@role}, \textit{@nymRef})  (att.canonical (\textit{@key}, \textit{@ref})) \hfil\\[-10pt]\begin{sansreflist}
    \item[@scheme]
  indicates the classification system or taxonomy in use, for example by supplying the identifier of a <taxonomy> element, or pointing to some other resource.
\begin{reflist}
    \item[{Status}]
  Optional
    \item[{Datatype}]
  teidata.pointer
\end{reflist}  
    \item[@code]
  identifies an occupation code defined within the classification system or taxonomy defined by the {\itshape scheme} attribute.
\begin{reflist}
    \item[{Status}]
  Optional
    \item[{Datatype}]
  teidata.pointer
\end{reflist}  
\end{sansreflist}  
    \item[{Member of}]
  model.persStateLike
    \item[{Contained by}]
  
    \item[namesdates: ]
   person personGrp
    \item[{May contain}]
  
    \item[analysis: ]
   c cl interp interpGrp m pc phr s span spanGrp w\par 
    \item[core: ]
   abbr add address bibl biblStruct cb choice cit corr date del desc distinct email emph expan foreign gap gb gloss graphic hi index l label lb lg list listBibl measure measureGrp media mentioned milestone name note num orig p pb ptr q quote ref reg rs said sic soCalled sp stage term time title unclear\par 
    \item[figures: ]
   figure formula notatedMusic table\par 
    \item[gaiji: ]
   g\par 
    \item[header: ]
   biblFull idno\par 
    \item[linking: ]
   ab alt altGrp anchor join joinGrp link linkGrp seg timeline\par 
    \item[msdescription: ]
   catchwords depth dim dimensions height heraldry locus locusGrp material msDesc objectType origDate origPlace secFol signatures stamp watermark width\par 
    \item[namesdates: ]
   addName affiliation bloc climate country district forename genName geo geogFeat geogName listEvent listNym listOrg listPerson listPlace location nameLink offset orgName persName placeName population region roleName settlement state surname terrain trait\par 
    \item[textcrit: ]
   app listApp listWit witDetail\par 
    \item[textstructure: ]
   floatingText\par 
    \item[transcr: ]
   addSpan am damage damageSpan delSpan ex fw handShift listTranspose metamark mod redo restore retrace secl space subst substJoin supplied surplus undo\par character data
    \item[{Note}]
  \par
The content of this element may be used as an alternative to the more formal specification made possible by its attributes; it may also be used to supplement the formal specification with commentary or clarification.
    \item[{Example}]
  \leavevmode\bgroup\exampleFont \begin{shaded}\noindent\mbox{}{<\textbf{occupation}>}accountant{</\textbf{occupation}>}\end{shaded}\egroup 


    \item[{Example}]
  \leavevmode\bgroup\exampleFont \begin{shaded}\noindent\mbox{}{<\textbf{occupation}\hspace*{6pt}{code}="{\#acc}"\mbox{}\newline 
\hspace*{6pt}{scheme}="{\#occupationtaxonomy}">}accountant{</\textbf{occupation}>}\end{shaded}\egroup 


    \item[{Content model}]
  \mbox{}\hfill\\[-10pt]\begin{Verbatim}[fontsize=\small]
<content>
 <macroRef key="macro.specialPara"/>
</content>
    
\end{Verbatim}

    \item[{Schema Declaration}]
  \mbox{}\hfill\\[-10pt]\begin{Verbatim}[fontsize=\small]
element occupation
{
   att.global.attributes,
   att.datable.attributes,
   att.editLike.attributes,
   att.naming.attributes,
   attribute scheme { text }?,
   attribute code { text }?,
   macro.specialPara}
\end{Verbatim}

\end{reflist}  \index{offset=<offset>|oddindex}
\begin{reflist}
\item[]\begin{specHead}{TEI.offset}{<offset> }marks that part of a relative temporal or spatial expression which indicates the direction of the offset between the two place names, dates, or times involved in the expression. [\xref{http://www.tei-c.org/release/doc/tei-p5-doc/en/html/ND.html\#NDPLAC}{13.2.3. Place Names}]\end{specHead} 
    \item[{Module}]
  namesdates
    \item[{Attributes}]
  Attributes att.datable (\textit{@calendar}, \textit{@period})  (att.datable.w3c (\textit{@when}, \textit{@notBefore}, \textit{@notAfter}, \textit{@from}, \textit{@to})) (att.datable.iso (\textit{@when-iso}, \textit{@notBefore-iso}, \textit{@notAfter-iso}, \textit{@from-iso}, \textit{@to-iso})) (att.datable.custom (\textit{@when-custom}, \textit{@notBefore-custom}, \textit{@notAfter-custom}, \textit{@from-custom}, \textit{@to-custom}, \textit{@datingPoint}, \textit{@datingMethod})) att.editLike (\textit{@evidence}, \textit{@instant})  (att.dimensions (\textit{@unit}, \textit{@quantity}, \textit{@extent}, \textit{@precision}, \textit{@scope}) (att.ranging (\textit{@atLeast}, \textit{@atMost}, \textit{@min}, \textit{@max}, \textit{@confidence})) ) att.global (\textit{@xml:id}, \textit{@n}, \textit{@xml:lang}, \textit{@xml:base}, \textit{@xml:space})  (att.global.rendition (\textit{@rend}, \textit{@style}, \textit{@rendition})) (att.global.linking (\textit{@corresp}, \textit{@synch}, \textit{@sameAs}, \textit{@copyOf}, \textit{@next}, \textit{@prev}, \textit{@exclude}, \textit{@select})) (att.global.analytic (\textit{@ana})) (att.global.facs (\textit{@facs})) (att.global.change (\textit{@change})) (att.global.responsibility (\textit{@cert}, \textit{@resp})) (att.global.source (\textit{@source})) att.naming (\textit{@role}, \textit{@nymRef})  (att.canonical (\textit{@key}, \textit{@ref})) att.typed (\textit{@type}, \textit{@subtype}) 
    \item[{Member of}]
  model.offsetLike
    \item[{Contained by}]
  
    \item[analysis: ]
   cl phr s span\par 
    \item[core: ]
   abbr add addrLine address author bibl biblScope citedRange corr date del desc distinct editor email emph expan foreign gloss head headItem headLabel hi item l label measure meeting mentioned name note num orig p pubPlace publisher q quote ref reg resp rs said sic soCalled speaker stage street term textLang time title unclear\par 
    \item[figures: ]
   cell figDesc\par 
    \item[header: ]
   authority catDesc change classCode correspAction creation distributor edition extent funder geoDecl handNote language licence principal rendition scriptNote sponsor tagUsage typeNote\par 
    \item[linking: ]
   ab seg\par 
    \item[msdescription: ]
   accMat acquisition additions catchwords collation colophon condition custEvent decoNote explicit filiation finalRubric foliation heraldry incipit layout material musicNotation objectType origDate origPlace origin provenance rubric secFol signatures source stamp summary support surrogates watermark\par 
    \item[namesdates: ]
   addName affiliation age birth bloc country death district education faith floruit forename genName geogFeat geogName langKnown location nameLink nationality occupation offset org orgName persName placeName region residence roleName settlement sex socecStatus surname\par 
    \item[textcrit: ]
   lem rdg wit witDetail witness\par 
    \item[textstructure: ]
   byline closer dateline docAuthor docDate docEdition docImprint imprimatur opener salute signed titlePart trailer\par 
    \item[transcr: ]
   damage fw metamark mod restore retrace secl supplied surplus
    \item[{May contain}]
  
    \item[analysis: ]
   c cl interp interpGrp m pc phr s span spanGrp w\par 
    \item[core: ]
   abbr add address cb choice corr date del distinct email emph expan foreign gap gb gloss graphic hi index lb measure measureGrp media mentioned milestone name note num orig pb ptr ref reg rs sic soCalled term time title unclear\par 
    \item[figures: ]
   figure formula notatedMusic\par 
    \item[gaiji: ]
   g\par 
    \item[header: ]
   idno\par 
    \item[linking: ]
   alt altGrp anchor join joinGrp link linkGrp seg timeline\par 
    \item[msdescription: ]
   catchwords depth dim dimensions height heraldry locus locusGrp material objectType origDate origPlace secFol signatures stamp watermark width\par 
    \item[namesdates: ]
   addName affiliation bloc climate country district forename genName geo geogFeat geogName location nameLink offset orgName persName placeName population region roleName settlement state surname terrain trait\par 
    \item[textcrit: ]
   app witDetail\par 
    \item[transcr: ]
   addSpan am damage damageSpan delSpan ex fw handShift listTranspose metamark mod redo restore retrace secl space subst substJoin supplied surplus undo\par character data
    \item[{Example}]
  \leavevmode\bgroup\exampleFont \begin{shaded}\noindent\mbox{}{<\textbf{placeName}\hspace*{6pt}{key}="{NRPA1}">}\mbox{}\newline 
\hspace*{6pt}{<\textbf{offset}>}50 metres below the summit of{</\textbf{offset}>}\mbox{}\newline 
\hspace*{6pt}{<\textbf{geogName}>}\mbox{}\newline 
\hspace*{6pt}\hspace*{6pt}{<\textbf{geogFeat}>}Mount{</\textbf{geogFeat}>}\mbox{}\newline 
\hspace*{6pt}\hspace*{6pt}{<\textbf{name}>}Sinai{</\textbf{name}>}\mbox{}\newline 
\hspace*{6pt}{</\textbf{geogName}>}\mbox{}\newline 
{</\textbf{placeName}>}\end{shaded}\egroup 


    \item[{Content model}]
  \mbox{}\hfill\\[-10pt]\begin{Verbatim}[fontsize=\small]
<content>
 <macroRef key="macro.phraseSeq"/>
</content>
    
\end{Verbatim}

    \item[{Schema Declaration}]
  \mbox{}\hfill\\[-10pt]\begin{Verbatim}[fontsize=\small]
element offset
{
   att.datable.attributes,
   att.editLike.attributes,
   att.global.attributes,
   att.naming.attributes,
   att.typed.attributes,
   macro.phraseSeq}
\end{Verbatim}

\end{reflist}  \index{opener=<opener>|oddindex}
\begin{reflist}
\item[]\begin{specHead}{TEI.opener}{<opener> }groups together dateline, byline, salutation, and similar phrases appearing as a preliminary group at the start of a division, especially of a letter. [\xref{http://www.tei-c.org/release/doc/tei-p5-doc/en/html/DS.html\#DSDTB}{4.2. Elements Common to All Divisions}]\end{specHead} 
    \item[{Module}]
  textstructure
    \item[{Attributes}]
  Attributes att.global (\textit{@xml:id}, \textit{@n}, \textit{@xml:lang}, \textit{@xml:base}, \textit{@xml:space})  (att.global.rendition (\textit{@rend}, \textit{@style}, \textit{@rendition})) (att.global.linking (\textit{@corresp}, \textit{@synch}, \textit{@sameAs}, \textit{@copyOf}, \textit{@next}, \textit{@prev}, \textit{@exclude}, \textit{@select})) (att.global.analytic (\textit{@ana})) (att.global.facs (\textit{@facs})) (att.global.change (\textit{@change})) (att.global.responsibility (\textit{@cert}, \textit{@resp})) (att.global.source (\textit{@source})) att.written (\textit{@hand}) 
    \item[{Member of}]
  model.divTopPart
    \item[{Contained by}]
  
    \item[core: ]
   lg list\par 
    \item[textstructure: ]
   body div group postscript
    \item[{May contain}]
  
    \item[analysis: ]
   c cl interp interpGrp m pc phr s span spanGrp w\par 
    \item[core: ]
   abbr add address cb choice corr date del distinct email emph expan foreign gap gb gloss graphic hi index lb measure measureGrp media mentioned milestone name note num orig pb ptr ref reg rs sic soCalled term time title unclear\par 
    \item[figures: ]
   figure formula notatedMusic\par 
    \item[gaiji: ]
   g\par 
    \item[header: ]
   idno\par 
    \item[linking: ]
   alt altGrp anchor join joinGrp link linkGrp seg timeline\par 
    \item[msdescription: ]
   catchwords depth dim dimensions height heraldry locus locusGrp material objectType origDate origPlace secFol signatures stamp watermark width\par 
    \item[namesdates: ]
   addName affiliation bloc climate country district forename genName geo geogFeat geogName location nameLink offset orgName persName placeName population region roleName settlement state surname terrain trait\par 
    \item[textcrit: ]
   app witDetail\par 
    \item[textstructure: ]
   argument byline dateline epigraph salute signed\par 
    \item[transcr: ]
   addSpan am damage damageSpan delSpan ex fw handShift listTranspose metamark mod redo restore retrace secl space subst substJoin supplied surplus undo\par character data
    \item[{Example}]
  \leavevmode\bgroup\exampleFont \begin{shaded}\noindent\mbox{}{<\textbf{opener}>}\mbox{}\newline 
\hspace*{6pt}{<\textbf{dateline}>}Walden, this 29. of August 1592{</\textbf{dateline}>}\mbox{}\newline 
{</\textbf{opener}>}\end{shaded}\egroup 


    \item[{Example}]
  \leavevmode\bgroup\exampleFont \begin{shaded}\noindent\mbox{}{<\textbf{opener}>}\mbox{}\newline 
\hspace*{6pt}{<\textbf{dateline}>}\mbox{}\newline 
\hspace*{6pt}\hspace*{6pt}{<\textbf{name}\hspace*{6pt}{type}="{place}">}Great Marlborough Street{</\textbf{name}>}\mbox{}\newline 
\hspace*{6pt}\hspace*{6pt}{<\textbf{date}>}November 11, 1848{</\textbf{date}>}\mbox{}\newline 
\hspace*{6pt}{</\textbf{dateline}>}\mbox{}\newline 
\hspace*{6pt}{<\textbf{salute}>}My dear Sir,{</\textbf{salute}>}\mbox{}\newline 
{</\textbf{opener}>}\mbox{}\newline 
{<\textbf{p}>}I am sorry to say that absence from town and other circumstances have prevented me from\mbox{}\newline 
 earlier enquiring...{</\textbf{p}>}\end{shaded}\egroup 


    \item[{Content model}]
  \mbox{}\hfill\\[-10pt]\begin{Verbatim}[fontsize=\small]
<content>
 <alternate maxOccurs="unbounded"
  minOccurs="0">
  <textNode/>
  <classRef key="model.gLike"/>
  <classRef key="model.phrase"/>
  <elementRef key="argument"/>
  <elementRef key="byline"/>
  <elementRef key="dateline"/>
  <elementRef key="epigraph"/>
  <elementRef key="salute"/>
  <elementRef key="signed"/>
  <classRef key="model.global"/>
 </alternate>
</content>
    
\end{Verbatim}

    \item[{Schema Declaration}]
  \mbox{}\hfill\\[-10pt]\begin{Verbatim}[fontsize=\small]
element opener
{
   att.global.attributes,
   att.written.attributes,
   (
      text
    | model.gLike    | model.phrase    | argument    | byline    | dateline    | epigraph    | salute    | signed    | model.global   )*
}
\end{Verbatim}

\end{reflist}  \index{org=<org>|oddindex}\index{role=@role!<org>|oddindex}
\begin{reflist}
\item[]\begin{specHead}{TEI.org}{<org> }(organization) provides information about an identifiable organization such as a business, a tribe, or any other grouping of people. [\xref{http://www.tei-c.org/release/doc/tei-p5-doc/en/html/ND.html\#NDORG}{13.2.2. Organizational Names}]\end{specHead} 
    \item[{Module}]
  namesdates
    \item[{Attributes}]
  Attributes att.global (\textit{@xml:id}, \textit{@n}, \textit{@xml:lang}, \textit{@xml:base}, \textit{@xml:space})  (att.global.rendition (\textit{@rend}, \textit{@style}, \textit{@rendition})) (att.global.linking (\textit{@corresp}, \textit{@synch}, \textit{@sameAs}, \textit{@copyOf}, \textit{@next}, \textit{@prev}, \textit{@exclude}, \textit{@select})) (att.global.analytic (\textit{@ana})) (att.global.facs (\textit{@facs})) (att.global.change (\textit{@change})) (att.global.responsibility (\textit{@cert}, \textit{@resp})) (att.global.source (\textit{@source})) att.typed (\textit{@type}, \textit{@subtype}) att.editLike (\textit{@evidence}, \textit{@instant})  (att.dimensions (\textit{@unit}, \textit{@quantity}, \textit{@extent}, \textit{@precision}, \textit{@scope}) (att.ranging (\textit{@atLeast}, \textit{@atMost}, \textit{@min}, \textit{@max}, \textit{@confidence})) ) att.sortable (\textit{@sortKey}) \hfil\\[-10pt]\begin{sansreflist}
    \item[@role]
  specifies a primary role or classification for the organization.
\begin{reflist}
    \item[{Status}]
  Optional
    \item[{Datatype}]
  1–∞ occurrences of teidata.word separated by whitespace
    \item[{Note}]
  \par
Values for this attribute may be locally defined by a project, using arbitrary keywords such as artist, employer, family group, or political party, each of which should be associated with a definition. Such local definitions will typically be provided by a \texttt{<valList>} element in the project schema specification.
\end{reflist}  
\end{sansreflist}  
    \item[{Member of}]
  model.personLike
    \item[{Contained by}]
  
    \item[namesdates: ]
   listOrg listPerson org
    \item[{May contain}]
  
    \item[core: ]
   bibl biblStruct cb desc gb head label lb listBibl milestone name note p pb rs\par 
    \item[header: ]
   biblFull idno\par 
    \item[linking: ]
   ab anchor link linkGrp\par 
    \item[msdescription: ]
   msDesc\par 
    \item[namesdates: ]
   addName bloc climate country district event forename genName geogFeat geogName listEvent listOrg listPerson listPlace location nameLink offset org orgName persName person personGrp place placeName population region roleName settlement state surname terrain trait\par 
    \item[textcrit: ]
   witDetail\par 
    \item[transcr: ]
   fw
    \item[{Example}]
  \leavevmode\bgroup\exampleFont \begin{shaded}\noindent\mbox{}{<\textbf{org}\hspace*{6pt}{xml:id}="{JAMs}">}\mbox{}\newline 
\hspace*{6pt}{<\textbf{orgName}>}Justified Ancients of Mummu{</\textbf{orgName}>}\mbox{}\newline 
\hspace*{6pt}{<\textbf{desc}>}An underground anarchist collective spearheaded by {<\textbf{persName}>}Hagbard\mbox{}\newline 
\hspace*{6pt}\hspace*{6pt}\hspace*{6pt}\hspace*{6pt} Celine{</\textbf{persName}>}, who fight the Illuminati from a golden submarine, the\mbox{}\newline 
\hspace*{6pt}{<\textbf{name}>}Leif Ericson{</\textbf{name}>}\mbox{}\newline 
\hspace*{6pt}{</\textbf{desc}>}\mbox{}\newline 
\hspace*{6pt}{<\textbf{bibl}>}\mbox{}\newline 
\hspace*{6pt}\hspace*{6pt}{<\textbf{author}>}Robert Shea{</\textbf{author}>}\mbox{}\newline 
\hspace*{6pt}\hspace*{6pt}{<\textbf{author}>}Robert Anton Wilson{</\textbf{author}>}\mbox{}\newline 
\hspace*{6pt}\hspace*{6pt}{<\textbf{title}>}The Illuminatus! Trilogy{</\textbf{title}>}\mbox{}\newline 
\hspace*{6pt}{</\textbf{bibl}>}\mbox{}\newline 
{</\textbf{org}>}\end{shaded}\egroup 


    \item[{Content model}]
  \mbox{}\hfill\\[-10pt]\begin{Verbatim}[fontsize=\small]
<content>
 <sequence>
  <classRef key="model.headLike"
   maxOccurs="unbounded" minOccurs="0"/>
  <alternate>
   <classRef key="model.pLike"
    maxOccurs="unbounded" minOccurs="0"/>
   <alternate maxOccurs="unbounded"
    minOccurs="0">
    <classRef key="model.labelLike"/>
    <classRef key="model.nameLike"/>
    <classRef key="model.placeLike"/>
    <classRef key="model.orgPart"/>
    <classRef key="model.milestoneLike"/>
   </alternate>
  </alternate>
  <alternate maxOccurs="unbounded"
   minOccurs="0">
   <classRef key="model.noteLike"/>
   <classRef key="model.biblLike"/>
   <elementRef key="linkGrp"/>
   <elementRef key="link"/>
  </alternate>
  <classRef key="model.personLike"
   maxOccurs="unbounded" minOccurs="0"/>
 </sequence>
</content>
    
\end{Verbatim}

    \item[{Schema Declaration}]
  \mbox{}\hfill\\[-10pt]\begin{Verbatim}[fontsize=\small]
element org
{
   att.global.attributes,
   att.typed.attributes,
   att.editLike.attributes,
   att.sortable.attributes,
   attribute role { list { + } }?,
   (
      model.headLike*,
      (
         model.pLike*
       | (
            model.labelLike          | model.nameLike          | model.placeLike          | model.orgPart          | model.milestoneLike         )*
      ),
      ( model.noteLike | model.biblLike | linkGrp | link )*,
      model.personLike*
   )
}
\end{Verbatim}

\end{reflist}  \index{orgName=<orgName>|oddindex}
\begin{reflist}
\item[]\begin{specHead}{TEI.orgName}{<orgName> }(organization name) contains an organizational name. [\xref{http://www.tei-c.org/release/doc/tei-p5-doc/en/html/ND.html\#NDORG}{13.2.2. Organizational Names}]\end{specHead} 
    \item[{Module}]
  namesdates
    \item[{Attributes}]
  Attributes att.global (\textit{@xml:id}, \textit{@n}, \textit{@xml:lang}, \textit{@xml:base}, \textit{@xml:space})  (att.global.rendition (\textit{@rend}, \textit{@style}, \textit{@rendition})) (att.global.linking (\textit{@corresp}, \textit{@synch}, \textit{@sameAs}, \textit{@copyOf}, \textit{@next}, \textit{@prev}, \textit{@exclude}, \textit{@select})) (att.global.analytic (\textit{@ana})) (att.global.facs (\textit{@facs})) (att.global.change (\textit{@change})) (att.global.responsibility (\textit{@cert}, \textit{@resp})) (att.global.source (\textit{@source})) att.datable (\textit{@calendar}, \textit{@period})  (att.datable.w3c (\textit{@when}, \textit{@notBefore}, \textit{@notAfter}, \textit{@from}, \textit{@to})) (att.datable.iso (\textit{@when-iso}, \textit{@notBefore-iso}, \textit{@notAfter-iso}, \textit{@from-iso}, \textit{@to-iso})) (att.datable.custom (\textit{@when-custom}, \textit{@notBefore-custom}, \textit{@notAfter-custom}, \textit{@from-custom}, \textit{@to-custom}, \textit{@datingPoint}, \textit{@datingMethod})) att.editLike (\textit{@evidence}, \textit{@instant})  (att.dimensions (\textit{@unit}, \textit{@quantity}, \textit{@extent}, \textit{@precision}, \textit{@scope}) (att.ranging (\textit{@atLeast}, \textit{@atMost}, \textit{@min}, \textit{@max}, \textit{@confidence})) ) att.personal (\textit{@full}, \textit{@sort})  (att.naming (\textit{@role}, \textit{@nymRef}) (att.canonical (\textit{@key}, \textit{@ref})) ) att.typed (\textit{@type}, \textit{@subtype}) 
    \item[{Member of}]
  model.nameLike.agent
    \item[{Contained by}]
  
    \item[analysis: ]
   cl phr s span\par 
    \item[core: ]
   abbr add addrLine address author bibl biblScope citedRange corr date del desc distinct editor email emph expan foreign gloss head headItem headLabel hi item l label measure meeting mentioned name note num orig p pubPlace publisher q quote ref reg resp respStmt rs said sic soCalled speaker stage street term textLang time title unclear\par 
    \item[figures: ]
   cell figDesc\par 
    \item[header: ]
   authority catDesc change classCode correspAction creation distributor edition extent funder geoDecl handNote language licence principal rendition scriptNote sponsor tagUsage typeNote\par 
    \item[linking: ]
   ab seg\par 
    \item[msdescription: ]
   accMat acquisition additions catchwords collation colophon condition custEvent decoNote explicit filiation finalRubric foliation heraldry incipit layout material musicNotation objectType origDate origPlace origin provenance rubric secFol signatures source stamp summary support surrogates watermark\par 
    \item[namesdates: ]
   addName affiliation age birth bloc country death district education faith floruit forename genName geogFeat geogName langKnown nameLink nationality occupation offset org orgName persName placeName region residence roleName settlement sex socecStatus surname\par 
    \item[textcrit: ]
   lem rdg wit witDetail witness\par 
    \item[textstructure: ]
   byline closer dateline docAuthor docDate docEdition docImprint imprimatur opener salute signed titlePart trailer\par 
    \item[transcr: ]
   damage fw metamark mod restore retrace secl supplied surplus
    \item[{May contain}]
  
    \item[analysis: ]
   c cl interp interpGrp m pc phr s span spanGrp w\par 
    \item[core: ]
   abbr add address cb choice corr date del distinct email emph expan foreign gap gb gloss graphic hi index lb measure measureGrp media mentioned milestone name note num orig pb ptr ref reg rs sic soCalled term time title unclear\par 
    \item[figures: ]
   figure formula notatedMusic\par 
    \item[gaiji: ]
   g\par 
    \item[header: ]
   idno\par 
    \item[linking: ]
   alt altGrp anchor join joinGrp link linkGrp seg timeline\par 
    \item[msdescription: ]
   catchwords depth dim dimensions height heraldry locus locusGrp material objectType origDate origPlace secFol signatures stamp watermark width\par 
    \item[namesdates: ]
   addName affiliation bloc climate country district forename genName geo geogFeat geogName location nameLink offset orgName persName placeName population region roleName settlement state surname terrain trait\par 
    \item[textcrit: ]
   app witDetail\par 
    \item[transcr: ]
   addSpan am damage damageSpan delSpan ex fw handShift listTranspose metamark mod redo restore retrace secl space subst substJoin supplied surplus undo\par character data
    \item[{Example}]
  \leavevmode\bgroup\exampleFont \begin{shaded}\noindent\mbox{}About a year back, a question of considerable interest was agitated in the {<\textbf{orgName}\hspace*{6pt}{key}="{PAS1}"\hspace*{6pt}{type}="{voluntary}">}\mbox{}\newline 
\hspace*{6pt}{<\textbf{placeName}\hspace*{6pt}{key}="{PEN}">}Pennsyla.{</\textbf{placeName}>} Abolition Society\mbox{}\newline 
{</\textbf{orgName}>} [...]\end{shaded}\egroup 


    \item[{Content model}]
  \mbox{}\hfill\\[-10pt]\begin{Verbatim}[fontsize=\small]
<content>
 <macroRef key="macro.phraseSeq"/>
</content>
    
\end{Verbatim}

    \item[{Schema Declaration}]
  \mbox{}\hfill\\[-10pt]\begin{Verbatim}[fontsize=\small]
element orgName
{
   att.global.attributes,
   att.datable.attributes,
   att.editLike.attributes,
   att.personal.attributes,
   att.typed.attributes,
   macro.phraseSeq}
\end{Verbatim}

\end{reflist}  \index{orig=<orig>|oddindex}
\begin{reflist}
\item[]\begin{specHead}{TEI.orig}{<orig> }(original form) contains a reading which is marked as following the original, rather than being normalized or corrected. [\xref{http://www.tei-c.org/release/doc/tei-p5-doc/en/html/CO.html\#COEDREG}{3.4.2. Regularization and Normalization} \xref{http://www.tei-c.org/release/doc/tei-p5-doc/en/html/TC.html\#TC}{12. Critical Apparatus}]\end{specHead} 
    \item[{Module}]
  core
    \item[{Attributes}]
  Attributes att.global (\textit{@xml:id}, \textit{@n}, \textit{@xml:lang}, \textit{@xml:base}, \textit{@xml:space})  (att.global.rendition (\textit{@rend}, \textit{@style}, \textit{@rendition})) (att.global.linking (\textit{@corresp}, \textit{@synch}, \textit{@sameAs}, \textit{@copyOf}, \textit{@next}, \textit{@prev}, \textit{@exclude}, \textit{@select})) (att.global.analytic (\textit{@ana})) (att.global.facs (\textit{@facs})) (att.global.change (\textit{@change})) (att.global.responsibility (\textit{@cert}, \textit{@resp})) (att.global.source (\textit{@source}))
    \item[{Member of}]
  model.choicePart model.pPart.transcriptional
    \item[{Contained by}]
  
    \item[analysis: ]
   cl pc phr s w\par 
    \item[core: ]
   abbr add addrLine author bibl biblScope choice citedRange corr date del distinct editor email emph expan foreign gloss head headItem headLabel hi item l label measure mentioned name note num orig p pubPlace publisher q quote ref reg rs said sic soCalled speaker stage street term textLang time title unclear\par 
    \item[figures: ]
   cell\par 
    \item[header: ]
   change distributor edition extent geoDecl handNote licence scriptNote typeNote\par 
    \item[linking: ]
   ab seg\par 
    \item[msdescription: ]
   accMat acquisition additions catchwords collation colophon condition custEvent decoNote explicit filiation finalRubric foliation heraldry incipit layout material musicNotation objectType origDate origPlace origin provenance rubric secFol signatures source stamp summary support surrogates watermark\par 
    \item[namesdates: ]
   addName affiliation birth bloc country death district education faith floruit forename genName geogFeat geogName nameLink nationality occupation offset orgName persName placeName region residence roleName settlement sex socecStatus surname\par 
    \item[textcrit: ]
   lem rdg wit witDetail\par 
    \item[textstructure: ]
   byline closer dateline docAuthor docDate docEdition docImprint imprimatur opener salute signed titlePart trailer\par 
    \item[transcr: ]
   am damage fw metamark mod restore retrace secl supplied surplus
    \item[{May contain}]
  
    \item[analysis: ]
   c cl interp interpGrp m pc phr s span spanGrp w\par 
    \item[core: ]
   abbr add address bibl biblStruct cb choice cit corr date del desc distinct email emph expan foreign gap gb gloss graphic hi index l label lb lg list listBibl measure measureGrp media mentioned milestone name note num orig pb ptr q quote ref reg rs said sic soCalled stage term time title unclear\par 
    \item[figures: ]
   figure formula notatedMusic table\par 
    \item[gaiji: ]
   g\par 
    \item[header: ]
   biblFull idno\par 
    \item[linking: ]
   alt altGrp anchor join joinGrp link linkGrp seg timeline\par 
    \item[msdescription: ]
   catchwords depth dim dimensions height heraldry locus locusGrp material msDesc objectType origDate origPlace secFol signatures stamp watermark width\par 
    \item[namesdates: ]
   addName affiliation bloc climate country district forename genName geo geogFeat geogName listEvent listNym listOrg listPerson listPlace location nameLink offset orgName persName placeName population region roleName settlement state surname terrain trait\par 
    \item[textcrit: ]
   app listApp listWit witDetail\par 
    \item[textstructure: ]
   floatingText\par 
    \item[transcr: ]
   addSpan am damage damageSpan delSpan ex fw handShift listTranspose metamark mod redo restore retrace secl space subst substJoin supplied surplus undo\par character data
    \item[{Example}]
  If all that is desired is to call attention to the original version in the copy text, <orig> may be used alone:\leavevmode\bgroup\exampleFont \begin{shaded}\noindent\mbox{}{<\textbf{l}>}But this will be a {<\textbf{orig}>}meere{</\textbf{orig}>} confusion{</\textbf{l}>}\mbox{}\newline 
{<\textbf{l}>}And hardly shall we all be {<\textbf{orig}>}vnderstoode{</\textbf{orig}>}\mbox{}\newline 
{</\textbf{l}>}\end{shaded}\egroup 


    \item[{Example}]
  More usually, an <orig> will be combined with a regularized form within a <choice> element:\leavevmode\bgroup\exampleFont \begin{shaded}\noindent\mbox{}{<\textbf{l}>}But this will be a {<\textbf{choice}>}\mbox{}\newline 
\hspace*{6pt}\hspace*{6pt}{<\textbf{orig}>}meere{</\textbf{orig}>}\mbox{}\newline 
\hspace*{6pt}\hspace*{6pt}{<\textbf{reg}>}mere{</\textbf{reg}>}\mbox{}\newline 
\hspace*{6pt}{</\textbf{choice}>} confusion{</\textbf{l}>}\mbox{}\newline 
{<\textbf{l}>}And hardly shall we all be {<\textbf{choice}>}\mbox{}\newline 
\hspace*{6pt}\hspace*{6pt}{<\textbf{orig}>}vnderstoode{</\textbf{orig}>}\mbox{}\newline 
\hspace*{6pt}\hspace*{6pt}{<\textbf{reg}>}understood{</\textbf{reg}>}\mbox{}\newline 
\hspace*{6pt}{</\textbf{choice}>}\mbox{}\newline 
{</\textbf{l}>}\end{shaded}\egroup 


    \item[{Content model}]
  \mbox{}\hfill\\[-10pt]\begin{Verbatim}[fontsize=\small]
<content>
 <macroRef key="macro.paraContent"/>
</content>
    
\end{Verbatim}

    \item[{Schema Declaration}]
  \mbox{}\hfill\\[-10pt]\begin{Verbatim}[fontsize=\small]
element orig { att.global.attributes, macro.paraContent }
\end{Verbatim}

\end{reflist}  \index{origDate=<origDate>|oddindex}
\begin{reflist}
\item[]\begin{specHead}{TEI.origDate}{<origDate> }(origin date) contains any form of date, used to identify the date of origin for a manuscript or manuscript part. [\xref{http://www.tei-c.org/release/doc/tei-p5-doc/en/html/MS.html\#msdates}{10.3.1. Origination}]\end{specHead} 
    \item[{Module}]
  msdescription
    \item[{Attributes}]
  Attributes att.global (\textit{@xml:id}, \textit{@n}, \textit{@xml:lang}, \textit{@xml:base}, \textit{@xml:space})  (att.global.rendition (\textit{@rend}, \textit{@style}, \textit{@rendition})) (att.global.linking (\textit{@corresp}, \textit{@synch}, \textit{@sameAs}, \textit{@copyOf}, \textit{@next}, \textit{@prev}, \textit{@exclude}, \textit{@select})) (att.global.analytic (\textit{@ana})) (att.global.facs (\textit{@facs})) (att.global.change (\textit{@change})) (att.global.responsibility (\textit{@cert}, \textit{@resp})) (att.global.source (\textit{@source})) att.datable (\textit{@calendar}, \textit{@period})  (att.datable.w3c (\textit{@when}, \textit{@notBefore}, \textit{@notAfter}, \textit{@from}, \textit{@to})) (att.datable.iso (\textit{@when-iso}, \textit{@notBefore-iso}, \textit{@notAfter-iso}, \textit{@from-iso}, \textit{@to-iso})) (att.datable.custom (\textit{@when-custom}, \textit{@notBefore-custom}, \textit{@notAfter-custom}, \textit{@from-custom}, \textit{@to-custom}, \textit{@datingPoint}, \textit{@datingMethod})) att.editLike (\textit{@evidence}, \textit{@instant})  (att.dimensions (\textit{@unit}, \textit{@quantity}, \textit{@extent}, \textit{@precision}, \textit{@scope}) (att.ranging (\textit{@atLeast}, \textit{@atMost}, \textit{@min}, \textit{@max}, \textit{@confidence})) ) att.typed (\textit{@type}, \textit{@subtype}) 
    \item[{Member of}]
  model.pPart.msdesc
    \item[{Contained by}]
  
    \item[analysis: ]
   cl phr s span\par 
    \item[core: ]
   abbr add addrLine author biblScope citedRange corr date del desc distinct editor email emph expan foreign gloss head headItem headLabel hi item l label measure meeting mentioned name note num orig p pubPlace publisher q quote ref reg resp rs said sic soCalled speaker stage street term textLang time title unclear\par 
    \item[figures: ]
   cell figDesc\par 
    \item[header: ]
   authority catDesc change classCode creation distributor edition extent funder geoDecl handNote language licence principal rendition scriptNote sponsor tagUsage typeNote\par 
    \item[linking: ]
   ab seg\par 
    \item[msdescription: ]
   accMat acquisition additions catchwords collation colophon condition custEvent decoNote explicit filiation finalRubric foliation heraldry incipit layout material musicNotation objectType origDate origPlace origin provenance rubric secFol signatures source stamp summary support surrogates watermark\par 
    \item[namesdates: ]
   addName affiliation age birth bloc country death district education faith floruit forename genName geogFeat geogName langKnown nameLink nationality occupation offset orgName persName placeName region residence roleName settlement sex socecStatus surname\par 
    \item[textcrit: ]
   lem rdg wit witDetail witness\par 
    \item[textstructure: ]
   byline closer dateline docAuthor docDate docEdition docImprint imprimatur opener salute signed titlePart trailer\par 
    \item[transcr: ]
   damage fw metamark mod restore retrace secl supplied surplus
    \item[{May contain}]
  
    \item[analysis: ]
   c cl interp interpGrp m pc phr s span spanGrp w\par 
    \item[core: ]
   abbr add address cb choice corr date del distinct email emph expan foreign gap gb gloss graphic hi index lb measure measureGrp media mentioned milestone name note num orig pb ptr ref reg rs sic soCalled term time title unclear\par 
    \item[figures: ]
   figure formula notatedMusic\par 
    \item[gaiji: ]
   g\par 
    \item[header: ]
   idno\par 
    \item[linking: ]
   alt altGrp anchor join joinGrp link linkGrp seg timeline\par 
    \item[msdescription: ]
   catchwords depth dim dimensions height heraldry locus locusGrp material objectType origDate origPlace secFol signatures stamp watermark width\par 
    \item[namesdates: ]
   addName affiliation bloc climate country district forename genName geo geogFeat geogName location nameLink offset orgName persName placeName population region roleName settlement state surname terrain trait\par 
    \item[textcrit: ]
   app witDetail\par 
    \item[transcr: ]
   addSpan am damage damageSpan delSpan ex fw handShift listTranspose metamark mod redo restore retrace secl space subst substJoin supplied surplus undo\par character data
    \item[{Example}]
  \leavevmode\bgroup\exampleFont \begin{shaded}\noindent\mbox{}{<\textbf{origDate}\hspace*{6pt}{notAfter}="{-0200}"\mbox{}\newline 
\hspace*{6pt}{notBefore}="{-0300}">}3rd century BCE{</\textbf{origDate}>}\end{shaded}\egroup 


    \item[{Content model}]
  \mbox{}\hfill\\[-10pt]\begin{Verbatim}[fontsize=\small]
<content>
 <alternate maxOccurs="unbounded"
  minOccurs="0">
  <textNode/>
  <classRef key="model.gLike"/>
  <classRef key="model.phrase"/>
  <classRef key="model.global"/>
 </alternate>
</content>
    
\end{Verbatim}

    \item[{Schema Declaration}]
  \mbox{}\hfill\\[-10pt]\begin{Verbatim}[fontsize=\small]
element origDate
{
   att.global.attributes,
   att.datable.attributes,
   att.editLike.attributes,
   att.typed.attributes,
   ( text | model.gLike | model.phrase | model.global )*
}
\end{Verbatim}

\end{reflist}  \index{origPlace=<origPlace>|oddindex}
\begin{reflist}
\item[]\begin{specHead}{TEI.origPlace}{<origPlace> }(origin place) contains any form of place name, used to identify the place of origin for a manuscript or manuscript part. [\xref{http://www.tei-c.org/release/doc/tei-p5-doc/en/html/MS.html\#msdates}{10.3.1. Origination}]\end{specHead} 
    \item[{Module}]
  msdescription
    \item[{Attributes}]
  Attributes att.global (\textit{@xml:id}, \textit{@n}, \textit{@xml:lang}, \textit{@xml:base}, \textit{@xml:space})  (att.global.rendition (\textit{@rend}, \textit{@style}, \textit{@rendition})) (att.global.linking (\textit{@corresp}, \textit{@synch}, \textit{@sameAs}, \textit{@copyOf}, \textit{@next}, \textit{@prev}, \textit{@exclude}, \textit{@select})) (att.global.analytic (\textit{@ana})) (att.global.facs (\textit{@facs})) (att.global.change (\textit{@change})) (att.global.responsibility (\textit{@cert}, \textit{@resp})) (att.global.source (\textit{@source})) att.naming (\textit{@role}, \textit{@nymRef})  (att.canonical (\textit{@key}, \textit{@ref})) att.datable (\textit{@calendar}, \textit{@period})  (att.datable.w3c (\textit{@when}, \textit{@notBefore}, \textit{@notAfter}, \textit{@from}, \textit{@to})) (att.datable.iso (\textit{@when-iso}, \textit{@notBefore-iso}, \textit{@notAfter-iso}, \textit{@from-iso}, \textit{@to-iso})) (att.datable.custom (\textit{@when-custom}, \textit{@notBefore-custom}, \textit{@notAfter-custom}, \textit{@from-custom}, \textit{@to-custom}, \textit{@datingPoint}, \textit{@datingMethod})) att.editLike (\textit{@evidence}, \textit{@instant})  (att.dimensions (\textit{@unit}, \textit{@quantity}, \textit{@extent}, \textit{@precision}, \textit{@scope}) (att.ranging (\textit{@atLeast}, \textit{@atMost}, \textit{@min}, \textit{@max}, \textit{@confidence})) ) att.typed (\textit{@type}, \textit{@subtype}) 
    \item[{Member of}]
  model.pPart.msdesc
    \item[{Contained by}]
  
    \item[analysis: ]
   cl phr s span\par 
    \item[core: ]
   abbr add addrLine author biblScope citedRange corr date del desc distinct editor email emph expan foreign gloss head headItem headLabel hi item l label measure meeting mentioned name note num orig p pubPlace publisher q quote ref reg resp rs said sic soCalled speaker stage street term textLang time title unclear\par 
    \item[figures: ]
   cell figDesc\par 
    \item[header: ]
   authority catDesc change classCode creation distributor edition extent funder geoDecl handNote language licence principal rendition scriptNote sponsor tagUsage typeNote\par 
    \item[linking: ]
   ab seg\par 
    \item[msdescription: ]
   accMat acquisition additions catchwords collation colophon condition custEvent decoNote explicit filiation finalRubric foliation heraldry incipit layout material musicNotation objectType origDate origPlace origin provenance rubric secFol signatures source stamp summary support surrogates watermark\par 
    \item[namesdates: ]
   addName affiliation age birth bloc country death district education faith floruit forename genName geogFeat geogName langKnown nameLink nationality occupation offset orgName persName placeName region residence roleName settlement sex socecStatus surname\par 
    \item[textcrit: ]
   lem rdg wit witDetail witness\par 
    \item[textstructure: ]
   byline closer dateline docAuthor docDate docEdition docImprint imprimatur opener salute signed titlePart trailer\par 
    \item[transcr: ]
   damage fw metamark mod restore retrace secl supplied surplus
    \item[{May contain}]
  
    \item[analysis: ]
   c cl interp interpGrp m pc phr s span spanGrp w\par 
    \item[core: ]
   abbr add address cb choice corr date del distinct email emph expan foreign gap gb gloss graphic hi index lb measure measureGrp media mentioned milestone name note num orig pb ptr ref reg rs sic soCalled term time title unclear\par 
    \item[figures: ]
   figure formula notatedMusic\par 
    \item[gaiji: ]
   g\par 
    \item[header: ]
   idno\par 
    \item[linking: ]
   alt altGrp anchor join joinGrp link linkGrp seg timeline\par 
    \item[msdescription: ]
   catchwords depth dim dimensions height heraldry locus locusGrp material objectType origDate origPlace secFol signatures stamp watermark width\par 
    \item[namesdates: ]
   addName affiliation bloc climate country district forename genName geo geogFeat geogName location nameLink offset orgName persName placeName population region roleName settlement state surname terrain trait\par 
    \item[textcrit: ]
   app witDetail\par 
    \item[transcr: ]
   addSpan am damage damageSpan delSpan ex fw handShift listTranspose metamark mod redo restore retrace secl space subst substJoin supplied surplus undo\par character data
    \item[{Note}]
  \par
The {\itshape type} attribute may be used to distinguish different kinds of ‘origin’, for example original place of publication, as opposed to original place of printing.
    \item[{Example}]
  \leavevmode\bgroup\exampleFont \begin{shaded}\noindent\mbox{}{<\textbf{origPlace}>}Birmingham{</\textbf{origPlace}>}\end{shaded}\egroup 


    \item[{Content model}]
  \mbox{}\hfill\\[-10pt]\begin{Verbatim}[fontsize=\small]
<content>
 <macroRef key="macro.phraseSeq"/>
</content>
    
\end{Verbatim}

    \item[{Schema Declaration}]
  \mbox{}\hfill\\[-10pt]\begin{Verbatim}[fontsize=\small]
element origPlace
{
   att.global.attributes,
   att.naming.attributes,
   att.datable.attributes,
   att.editLike.attributes,
   att.typed.attributes,
   macro.phraseSeq}
\end{Verbatim}

\end{reflist}  \index{origin=<origin>|oddindex}
\begin{reflist}
\item[]\begin{specHead}{TEI.origin}{<origin> }contains any descriptive or other information concerning the origin of a manuscript or manuscript part. [\xref{http://www.tei-c.org/release/doc/tei-p5-doc/en/html/MS.html\#mshy}{10.8. History}]\end{specHead} 
    \item[{Module}]
  msdescription
    \item[{Attributes}]
  Attributes att.global (\textit{@xml:id}, \textit{@n}, \textit{@xml:lang}, \textit{@xml:base}, \textit{@xml:space})  (att.global.rendition (\textit{@rend}, \textit{@style}, \textit{@rendition})) (att.global.linking (\textit{@corresp}, \textit{@synch}, \textit{@sameAs}, \textit{@copyOf}, \textit{@next}, \textit{@prev}, \textit{@exclude}, \textit{@select})) (att.global.analytic (\textit{@ana})) (att.global.facs (\textit{@facs})) (att.global.change (\textit{@change})) (att.global.responsibility (\textit{@cert}, \textit{@resp})) (att.global.source (\textit{@source})) att.editLike (\textit{@evidence}, \textit{@instant})  (att.dimensions (\textit{@unit}, \textit{@quantity}, \textit{@extent}, \textit{@precision}, \textit{@scope}) (att.ranging (\textit{@atLeast}, \textit{@atMost}, \textit{@min}, \textit{@max}, \textit{@confidence})) ) att.datable (\textit{@calendar}, \textit{@period})  (att.datable.w3c (\textit{@when}, \textit{@notBefore}, \textit{@notAfter}, \textit{@from}, \textit{@to})) (att.datable.iso (\textit{@when-iso}, \textit{@notBefore-iso}, \textit{@notAfter-iso}, \textit{@from-iso}, \textit{@to-iso})) (att.datable.custom (\textit{@when-custom}, \textit{@notBefore-custom}, \textit{@notAfter-custom}, \textit{@from-custom}, \textit{@to-custom}, \textit{@datingPoint}, \textit{@datingMethod}))
    \item[{Contained by}]
  
    \item[msdescription: ]
   history
    \item[{May contain}]
  
    \item[analysis: ]
   c cl interp interpGrp m pc phr s span spanGrp w\par 
    \item[core: ]
   abbr add address bibl biblStruct cb choice cit corr date del desc distinct email emph expan foreign gap gb gloss graphic hi index l label lb lg list listBibl measure measureGrp media mentioned milestone name note num orig p pb ptr q quote ref reg rs said sic soCalled sp stage term time title unclear\par 
    \item[figures: ]
   figure formula notatedMusic table\par 
    \item[gaiji: ]
   g\par 
    \item[header: ]
   biblFull idno\par 
    \item[linking: ]
   ab alt altGrp anchor join joinGrp link linkGrp seg timeline\par 
    \item[msdescription: ]
   catchwords depth dim dimensions height heraldry locus locusGrp material msDesc objectType origDate origPlace secFol signatures stamp watermark width\par 
    \item[namesdates: ]
   addName affiliation bloc climate country district forename genName geo geogFeat geogName listEvent listNym listOrg listPerson listPlace location nameLink offset orgName persName placeName population region roleName settlement state surname terrain trait\par 
    \item[textcrit: ]
   app listApp listWit witDetail\par 
    \item[textstructure: ]
   floatingText\par 
    \item[transcr: ]
   addSpan am damage damageSpan delSpan ex fw handShift listTranspose metamark mod redo restore retrace secl space subst substJoin supplied surplus undo\par character data
    \item[{Example}]
  \leavevmode\bgroup\exampleFont \begin{shaded}\noindent\mbox{}{<\textbf{origin}\hspace*{6pt}{evidence}="{internal}"\hspace*{6pt}{notAfter}="{1845}"\mbox{}\newline 
\hspace*{6pt}{notBefore}="{1802}"\hspace*{6pt}{resp}="{\#AMH}">}Copied in {<\textbf{name}\hspace*{6pt}{type}="{origPlace}">}Derby{</\textbf{name}>}, probably from an\mbox{}\newline 
 old Flemish original, between 1802 and 1845, according to {<\textbf{persName}\hspace*{6pt}{xml:id}="{AMH}">}Anne-Mette Hansen{</\textbf{persName}>}.\mbox{}\newline 
{</\textbf{origin}>}\end{shaded}\egroup 


    \item[{Content model}]
  \mbox{}\hfill\\[-10pt]\begin{Verbatim}[fontsize=\small]
<content>
 <macroRef key="macro.specialPara"/>
</content>
    
\end{Verbatim}

    \item[{Schema Declaration}]
  \mbox{}\hfill\\[-10pt]\begin{Verbatim}[fontsize=\small]
element origin
{
   att.global.attributes,
   att.editLike.attributes,
   att.datable.attributes,
   macro.specialPara}
\end{Verbatim}

\end{reflist}  \index{p=<p>|oddindex}
\begin{reflist}
\item[]\begin{specHead}{TEI.p}{<p> }(paragraph) marks paragraphs in prose. [\xref{http://www.tei-c.org/release/doc/tei-p5-doc/en/html/CO.html\#COPA}{3.1. Paragraphs} \xref{http://www.tei-c.org/release/doc/tei-p5-doc/en/html/DR.html\#DRPAL}{7.2.5. Speech Contents}]\end{specHead} 
    \item[{Module}]
  core
    \item[{Attributes}]
  Attributes att.global (\textit{@xml:id}, \textit{@n}, \textit{@xml:lang}, \textit{@xml:base}, \textit{@xml:space})  (att.global.rendition (\textit{@rend}, \textit{@style}, \textit{@rendition})) (att.global.linking (\textit{@corresp}, \textit{@synch}, \textit{@sameAs}, \textit{@copyOf}, \textit{@next}, \textit{@prev}, \textit{@exclude}, \textit{@select})) (att.global.analytic (\textit{@ana})) (att.global.facs (\textit{@facs})) (att.global.change (\textit{@change})) (att.global.responsibility (\textit{@cert}, \textit{@resp})) (att.global.source (\textit{@source})) att.declaring (\textit{@decls}) att.fragmentable (\textit{@part}) att.written (\textit{@hand}) 
    \item[{Member of}]
  model.pLike
    \item[{Contained by}]
  
    \item[core: ]
   item note q quote said sp stage\par 
    \item[figures: ]
   cell figure\par 
    \item[header: ]
   abstract application availability cRefPattern calendar change correction correspAction correspContext correspDesc editionStmt editorialDecl encodingDesc handNote hyphenation interpretation langUsage licence normalization prefixDef projectDesc publicationStmt punctuation quotation refsDecl samplingDecl scriptNote segmentation seriesStmt sourceDesc stdVals styleDefDecl typeNote\par 
    \item[msdescription: ]
   accMat acquisition additions binding bindingDesc collation condition custEvent custodialHist decoDesc decoNote filiation foliation handDesc history layout layoutDesc msContents msDesc msFrag msItem msItemStruct msPart musicNotation objectDesc origin physDesc provenance recordHist scriptDesc seal sealDesc signatures source summary support supportDesc surrogates typeDesc\par 
    \item[namesdates: ]
   climate event langKnowledge listRelation nym occupation org person personGrp place population state terrain trait\par 
    \item[textcrit: ]
   lem rdg\par 
    \item[textstructure: ]
   argument back body div epigraph front postscript\par 
    \item[transcr: ]
   metamark
    \item[{May contain}]
  
    \item[analysis: ]
   c cl interp interpGrp m pc phr s span spanGrp w\par 
    \item[core: ]
   abbr add address bibl biblStruct cb choice cit corr date del desc distinct email emph expan foreign gap gb gloss graphic hi index l label lb lg list listBibl measure measureGrp media mentioned milestone name note num orig pb ptr q quote ref reg rs said sic soCalled stage term time title unclear\par 
    \item[figures: ]
   figure formula notatedMusic table\par 
    \item[gaiji: ]
   g\par 
    \item[header: ]
   biblFull idno\par 
    \item[linking: ]
   alt altGrp anchor join joinGrp link linkGrp seg timeline\par 
    \item[msdescription: ]
   catchwords depth dim dimensions height heraldry locus locusGrp material msDesc objectType origDate origPlace secFol signatures stamp watermark width\par 
    \item[namesdates: ]
   addName affiliation bloc climate country district forename genName geo geogFeat geogName listEvent listNym listOrg listPerson listPlace location nameLink offset orgName persName placeName population region roleName settlement state surname terrain trait\par 
    \item[textcrit: ]
   app listApp listWit witDetail\par 
    \item[textstructure: ]
   floatingText\par 
    \item[transcr: ]
   addSpan am damage damageSpan delSpan ex fw handShift listTranspose metamark mod redo restore retrace secl space subst substJoin supplied surplus undo\par character data
    \item[{Example}]
  \leavevmode\bgroup\exampleFont \begin{shaded}\noindent\mbox{}{<\textbf{p}>}Hallgerd was outside. {<\textbf{q}>}There is blood on your axe,{</\textbf{q}>} she said. {<\textbf{q}>}What have you\mbox{}\newline 
\hspace*{6pt}\hspace*{6pt} done?{</\textbf{q}>}\mbox{}\newline 
{</\textbf{p}>}\mbox{}\newline 
{<\textbf{p}>}\mbox{}\newline 
\hspace*{6pt}{<\textbf{q}>}I have now arranged that you can be married a second time,{</\textbf{q}>} replied Thjostolf.\mbox{}\newline 
{</\textbf{p}>}\mbox{}\newline 
{<\textbf{p}>}\mbox{}\newline 
\hspace*{6pt}{<\textbf{q}>}Then you must mean that Thorvald is dead,{</\textbf{q}>} she said.\mbox{}\newline 
{</\textbf{p}>}\mbox{}\newline 
{<\textbf{p}>}\mbox{}\newline 
\hspace*{6pt}{<\textbf{q}>}Yes,{</\textbf{q}>} said Thjostolf. {<\textbf{q}>}And now you must think up some plan for me.{</\textbf{q}>}\mbox{}\newline 
{</\textbf{p}>}\end{shaded}\egroup 


    \item[{Schematron}]
   <s:report test="(ancestor::tei:p or ancestor::tei:ab) and not(parent::tei:exemplum   |parent::tei:item |parent::tei:note |parent::tei:q |parent::tei:quote   |parent::tei:remarks |parent::tei:said |parent::tei:sp |parent::tei:stage   |parent::tei:cell |parent::tei:figure)"> Abstract model violation: Paragraphs may not contain other paragraphs or ab elements. </s:report>
    \item[{Schematron}]
   <s:report test="ancestor::tei:l[not(.//tei:note//tei:p[. = current()])]"> Abstract model violation: Lines may not contain higher-level structural elements such as div, p, or ab. </s:report>
    \item[{Content model}]
  \mbox{}\hfill\\[-10pt]\begin{Verbatim}[fontsize=\small]
<content>
 <macroRef key="macro.paraContent"/>
</content>
    
\end{Verbatim}

    \item[{Schema Declaration}]
  \mbox{}\hfill\\[-10pt]\begin{Verbatim}[fontsize=\small]
element p
{
   att.global.attributes,
   att.declaring.attributes,
   att.fragmentable.attributes,
   att.written.attributes,
   macro.paraContent}
\end{Verbatim}

\end{reflist}  \index{pb=<pb>|oddindex}
\begin{reflist}
\item[]\begin{specHead}{TEI.pb}{<pb> }(page break) marks the start of a new page in a paginated document. [\xref{http://www.tei-c.org/release/doc/tei-p5-doc/en/html/CO.html\#CORS5}{3.10.3. Milestone Elements}]\end{specHead} 
    \item[{Module}]
  core
    \item[{Attributes}]
  Attributes att.global (\textit{@xml:id}, \textit{@n}, \textit{@xml:lang}, \textit{@xml:base}, \textit{@xml:space})  (att.global.rendition (\textit{@rend}, \textit{@style}, \textit{@rendition})) (att.global.linking (\textit{@corresp}, \textit{@synch}, \textit{@sameAs}, \textit{@copyOf}, \textit{@next}, \textit{@prev}, \textit{@exclude}, \textit{@select})) (att.global.analytic (\textit{@ana})) (att.global.facs (\textit{@facs})) (att.global.change (\textit{@change})) (att.global.responsibility (\textit{@cert}, \textit{@resp})) (att.global.source (\textit{@source})) att.typed (\textit{@type}, \textit{@subtype}) att.edition (\textit{@ed}, \textit{@edRef}) att.spanning (\textit{@spanTo}) att.breaking (\textit{@break}) 
    \item[{Member of}]
  model.milestoneLike
    \item[{Contained by}]
  
    \item[analysis: ]
   cl m phr s span w\par 
    \item[core: ]
   abbr add addrLine address author bibl biblScope cit citedRange corr date del distinct editor email emph expan foreign gloss head headItem headLabel hi imprint item l label lg list listBibl measure mentioned name note num orig p pubPlace publisher q quote ref reg resp rs said series sic soCalled sp speaker stage street term textLang time title unclear\par 
    \item[figures: ]
   cell figure table\par 
    \item[header: ]
   authority change classCode distributor edition extent funder geoDecl handNote language licence principal scriptNote sponsor typeNote\par 
    \item[linking: ]
   ab seg\par 
    \item[msdescription: ]
   accMat acquisition additions catchwords collation colophon condition custEvent decoNote explicit filiation finalRubric foliation heraldry incipit layout material msItem musicNotation objectType origDate origPlace origin provenance rubric secFol signatures source stamp summary support surrogates watermark\par 
    \item[namesdates: ]
   addName affiliation age birth bloc country death district education faith floruit forename genName geogFeat geogName langKnown nameLink nationality occupation offset org orgName persName person personGrp placeName region residence roleName settlement sex socecStatus surname\par 
    \item[textcrit: ]
   lem rdg wit witDetail\par 
    \item[textstructure: ]
   argument back body byline closer dateline div docAuthor docDate docEdition docImprint docTitle epigraph floatingText front group imprimatur opener postscript salute signed text titlePage titlePart trailer\par 
    \item[transcr: ]
   damage fw line metamark mod restore retrace secl sourceDoc subst supplied surface surfaceGrp surplus zone
    \item[{May contain}]
  Empty element
    \item[{Note}]
  \par
A <pb> element should appear at the start of the page which it identifies. The global {\itshape n} attribute indicates the number or other value associated with this page. This will normally be the page number or signature printed on it, since the physical sequence number is implicit in the presence of the <pb> element itself.\par
The {\itshape type} attribute may be used to characterize the page break in any respect. The more specialized attributes {\itshape break}, {\itshape ed}, or {\itshape edRef} should be preferred when the intent is to indicate whether or not the page break is word-breaking, or to note the source from which it derives.
    \item[{Example}]
  Page numbers may vary in different editions of a text.\leavevmode\bgroup\exampleFont \begin{shaded}\noindent\mbox{}{<\textbf{p}>} ... {<\textbf{pb}\hspace*{6pt}{ed}="{ed2}"\hspace*{6pt}{n}="{145}"/>}\mbox{}\newline 
\textit{<!-- Page 145 in edition "ed2" starts here -->} ... {<\textbf{pb}\hspace*{6pt}{ed}="{ed1}"\hspace*{6pt}{n}="{283}"/>}\mbox{}\newline 
\textit{<!-- Page 283 in edition "ed1" starts here-->} ... {</\textbf{p}>}\end{shaded}\egroup 


    \item[{Example}]
  A page break may be associated with a facsimile image of the page it introduces by means of the {\itshape facs} attribute\leavevmode\bgroup\exampleFont \begin{shaded}\noindent\mbox{}{<\textbf{body}>}\mbox{}\newline 
\hspace*{6pt}{<\textbf{pb}\hspace*{6pt}{facs}="{page1.png}"\hspace*{6pt}{n}="{1}"/>}\mbox{}\newline 
\textit{<!-- page1.png contains an image of the page;\newline
                        the text it contains is encoded here -->}\mbox{}\newline 
\hspace*{6pt}{<\textbf{p}>}\mbox{}\newline 
\textit{<!-- ... -->}\mbox{}\newline 
\hspace*{6pt}{</\textbf{p}>}\mbox{}\newline 
\hspace*{6pt}{<\textbf{pb}\hspace*{6pt}{facs}="{page2.png}"\hspace*{6pt}{n}="{2}"/>}\mbox{}\newline 
\textit{<!-- similarly, for page 2 -->}\mbox{}\newline 
\hspace*{6pt}{<\textbf{p}>}\mbox{}\newline 
\textit{<!-- ... -->}\mbox{}\newline 
\hspace*{6pt}{</\textbf{p}>}\mbox{}\newline 
{</\textbf{body}>}\end{shaded}\egroup 


    \item[{Content model}]
  \fbox{\ttfamily <content>\newline
</content>\newline
    } 
    \item[{Schema Declaration}]
  \mbox{}\hfill\\[-10pt]\begin{Verbatim}[fontsize=\small]
element pb
{
   att.global.attributes,
   att.typed.attributes,
   att.edition.attributes,
   att.spanning.attributes,
   att.breaking.attributes,
   empty
}
\end{Verbatim}

\end{reflist}  \index{pc=<pc>|oddindex}\index{force=@force!<pc>|oddindex}\index{unit=@unit!<pc>|oddindex}\index{pre=@pre!<pc>|oddindex}
\begin{reflist}
\item[]\begin{specHead}{TEI.pc}{<pc> }(punctuation character) contains a character or string of characters regarded as constituting a single punctuation mark. [\xref{http://www.tei-c.org/release/doc/tei-p5-doc/en/html/AI.html\#AILC}{17.1. Linguistic Segment Categories}]\end{specHead} 
    \item[{Module}]
  analysis
    \item[{Attributes}]
  Attributes att.global (\textit{@xml:id}, \textit{@n}, \textit{@xml:lang}, \textit{@xml:base}, \textit{@xml:space})  (att.global.rendition (\textit{@rend}, \textit{@style}, \textit{@rendition})) (att.global.linking (\textit{@corresp}, \textit{@synch}, \textit{@sameAs}, \textit{@copyOf}, \textit{@next}, \textit{@prev}, \textit{@exclude}, \textit{@select})) (att.global.analytic (\textit{@ana})) (att.global.facs (\textit{@facs})) (att.global.change (\textit{@change})) (att.global.responsibility (\textit{@cert}, \textit{@resp})) (att.global.source (\textit{@source})) att.segLike (\textit{@function})  (att.datcat (\textit{@datcat}, \textit{@valueDatcat})) (att.fragmentable (\textit{@part})) att.typed (\textit{@type}, \textit{@subtype}) \hfil\\[-10pt]\begin{sansreflist}
    \item[@force]
  indicates the extent to which this punctuation mark conventionally separates words or phrases
\begin{reflist}
    \item[{Status}]
  Optional
    \item[{Datatype}]
  teidata.enumerated
    \item[{Legal values are:}]
  \begin{description}

\item[{strong}]the punctuation mark is a word separator
\item[{weak}]the punctuation mark is not a word separator
\item[{inter}]the punctuation mark may or may not be a word separator
\end{description} 
\end{reflist}  
    \item[@unit]
  provides a name for the kind of unit delimited by this punctuation mark.
\begin{reflist}
    \item[{Status}]
  Optional
    \item[{Datatype}]
  teidata.enumerated
\end{reflist}  
    \item[@pre]
  indicates whether this punctuation mark precedes or follows the unit it delimits.
\begin{reflist}
    \item[{Status}]
  Optional
    \item[{Datatype}]
  teidata.truthValue
\end{reflist}  
\end{sansreflist}  
    \item[{Member of}]
  model.linePart model.segLike 
    \item[{Contained by}]
  
    \item[analysis: ]
   cl phr s w\par 
    \item[core: ]
   abbr add addrLine author bibl biblScope citedRange corr date del distinct editor email emph expan foreign gloss head headItem headLabel hi item l label measure mentioned name note num orig p pubPlace publisher q quote ref reg rs said sic soCalled speaker stage street term textLang time title unclear\par 
    \item[figures: ]
   cell\par 
    \item[header: ]
   change distributor edition extent geoDecl handNote licence scriptNote typeNote\par 
    \item[linking: ]
   ab seg\par 
    \item[msdescription: ]
   accMat acquisition additions catchwords collation colophon condition custEvent decoNote explicit filiation finalRubric foliation heraldry incipit layout material musicNotation objectType origDate origPlace origin provenance rubric secFol signatures source stamp summary support surrogates watermark\par 
    \item[namesdates: ]
   addName affiliation birth bloc country death district education faith floruit forename genName geogFeat geogName nameLink nationality occupation offset orgName persName placeName region residence roleName settlement sex socecStatus surname\par 
    \item[textcrit: ]
   lem rdg wit witDetail\par 
    \item[textstructure: ]
   byline closer dateline docAuthor docDate docEdition docImprint imprimatur opener salute signed titlePart trailer\par 
    \item[transcr: ]
   damage fw line metamark mod restore retrace secl supplied surplus zone
    \item[{May contain}]
  
    \item[analysis: ]
   c\par 
    \item[core: ]
   abbr add choice corr del expan orig reg sic unclear\par 
    \item[gaiji: ]
   g\par 
    \item[transcr: ]
   am damage ex handShift mod redo restore retrace secl subst supplied surplus undo\par character data
    \item[{Example}]
  \leavevmode\bgroup\exampleFont \begin{shaded}\noindent\mbox{}{<\textbf{phr}>}\mbox{}\newline 
\hspace*{6pt}{<\textbf{w}>}do{</\textbf{w}>}\mbox{}\newline 
\hspace*{6pt}{<\textbf{w}>}you{</\textbf{w}>}\mbox{}\newline 
\hspace*{6pt}{<\textbf{w}>}understand{</\textbf{w}>}\mbox{}\newline 
\hspace*{6pt}{<\textbf{pc}\hspace*{6pt}{type}="{interrogative}">}?{</\textbf{pc}>}\mbox{}\newline 
{</\textbf{phr}>}\end{shaded}\egroup 


    \item[{Content model}]
  \mbox{}\hfill\\[-10pt]\begin{Verbatim}[fontsize=\small]
<content>
 <alternate maxOccurs="unbounded"
  minOccurs="0">
  <textNode/>
  <classRef key="model.gLike"/>
  <elementRef key="c"/>
  <classRef key="model.pPart.edit"/>
 </alternate>
</content>
    
\end{Verbatim}

    \item[{Schema Declaration}]
  \mbox{}\hfill\\[-10pt]\begin{Verbatim}[fontsize=\small]
element pc
{
   att.global.attributes,
   att.segLike.attributes,
   att.typed.attributes,
   attribute force { "strong" | "weak" | "inter" }?,
   attribute unit { text }?,
   attribute pre { text }?,
   ( text | model.gLike | c | model.pPart.edit )*
}
\end{Verbatim}

\end{reflist}  \index{persName=<persName>|oddindex}
\begin{reflist}
\item[]\begin{specHead}{TEI.persName}{<persName> }(personal name) contains a proper noun or proper-noun phrase referring to a person, possibly including one or more of the person's forenames, surnames, honorifics, added names, etc. [\xref{http://www.tei-c.org/release/doc/tei-p5-doc/en/html/ND.html\#NDPER}{13.2.1. Personal Names}]\end{specHead} 
    \item[{Module}]
  namesdates
    \item[{Attributes}]
  Attributes att.global (\textit{@xml:id}, \textit{@n}, \textit{@xml:lang}, \textit{@xml:base}, \textit{@xml:space})  (att.global.rendition (\textit{@rend}, \textit{@style}, \textit{@rendition})) (att.global.linking (\textit{@corresp}, \textit{@synch}, \textit{@sameAs}, \textit{@copyOf}, \textit{@next}, \textit{@prev}, \textit{@exclude}, \textit{@select})) (att.global.analytic (\textit{@ana})) (att.global.facs (\textit{@facs})) (att.global.change (\textit{@change})) (att.global.responsibility (\textit{@cert}, \textit{@resp})) (att.global.source (\textit{@source})) att.datable (\textit{@calendar}, \textit{@period})  (att.datable.w3c (\textit{@when}, \textit{@notBefore}, \textit{@notAfter}, \textit{@from}, \textit{@to})) (att.datable.iso (\textit{@when-iso}, \textit{@notBefore-iso}, \textit{@notAfter-iso}, \textit{@from-iso}, \textit{@to-iso})) (att.datable.custom (\textit{@when-custom}, \textit{@notBefore-custom}, \textit{@notAfter-custom}, \textit{@from-custom}, \textit{@to-custom}, \textit{@datingPoint}, \textit{@datingMethod})) att.editLike (\textit{@evidence}, \textit{@instant})  (att.dimensions (\textit{@unit}, \textit{@quantity}, \textit{@extent}, \textit{@precision}, \textit{@scope}) (att.ranging (\textit{@atLeast}, \textit{@atMost}, \textit{@min}, \textit{@max}, \textit{@confidence})) ) att.personal (\textit{@full}, \textit{@sort})  (att.naming (\textit{@role}, \textit{@nymRef}) (att.canonical (\textit{@key}, \textit{@ref})) ) att.typed (\textit{@type}, \textit{@subtype}) 
    \item[{Member of}]
  model.nameLike.agent model.persStateLike
    \item[{Contained by}]
  
    \item[analysis: ]
   cl phr s span\par 
    \item[core: ]
   abbr add addrLine address author bibl biblScope citedRange corr date del desc distinct editor email emph expan foreign gloss head headItem headLabel hi item l label measure meeting mentioned name note num orig p pubPlace publisher q quote ref reg resp respStmt rs said sic soCalled speaker stage street term textLang time title unclear\par 
    \item[figures: ]
   cell figDesc\par 
    \item[header: ]
   authority catDesc change classCode correspAction creation distributor edition extent funder geoDecl handNote language licence principal rendition scriptNote sponsor tagUsage typeNote\par 
    \item[linking: ]
   ab seg\par 
    \item[msdescription: ]
   accMat acquisition additions catchwords collation colophon condition custEvent decoNote explicit filiation finalRubric foliation heraldry incipit layout material musicNotation objectType origDate origPlace origin provenance rubric secFol signatures source stamp summary support surrogates watermark\par 
    \item[namesdates: ]
   addName affiliation age birth bloc country death district education faith floruit forename genName geogFeat geogName langKnown nameLink nationality occupation offset org orgName persName person personGrp placeName region residence roleName settlement sex socecStatus surname\par 
    \item[textcrit: ]
   lem rdg wit witDetail witness\par 
    \item[textstructure: ]
   byline closer dateline docAuthor docDate docEdition docImprint imprimatur opener salute signed titlePart trailer\par 
    \item[transcr: ]
   damage fw metamark mod restore retrace secl supplied surplus
    \item[{May contain}]
  
    \item[analysis: ]
   c cl interp interpGrp m pc phr s span spanGrp w\par 
    \item[core: ]
   abbr add address cb choice corr date del distinct email emph expan foreign gap gb gloss graphic hi index lb measure measureGrp media mentioned milestone name note num orig pb ptr ref reg rs sic soCalled term time title unclear\par 
    \item[figures: ]
   figure formula notatedMusic\par 
    \item[gaiji: ]
   g\par 
    \item[header: ]
   idno\par 
    \item[linking: ]
   alt altGrp anchor join joinGrp link linkGrp seg timeline\par 
    \item[msdescription: ]
   catchwords depth dim dimensions height heraldry locus locusGrp material objectType origDate origPlace secFol signatures stamp watermark width\par 
    \item[namesdates: ]
   addName affiliation bloc climate country district forename genName geo geogFeat geogName location nameLink offset orgName persName placeName population region roleName settlement state surname terrain trait\par 
    \item[textcrit: ]
   app witDetail\par 
    \item[transcr: ]
   addSpan am damage damageSpan delSpan ex fw handShift listTranspose metamark mod redo restore retrace secl space subst substJoin supplied surplus undo\par character data
    \item[{Example}]
  \leavevmode\bgroup\exampleFont \begin{shaded}\noindent\mbox{}{<\textbf{persName}>}\mbox{}\newline 
\hspace*{6pt}{<\textbf{forename}>}Edward{</\textbf{forename}>}\mbox{}\newline 
\hspace*{6pt}{<\textbf{forename}>}George{</\textbf{forename}>}\mbox{}\newline 
\hspace*{6pt}{<\textbf{surname}\hspace*{6pt}{type}="{linked}">}Bulwer-Lytton{</\textbf{surname}>}, {<\textbf{roleName}>}Baron Lytton of\mbox{}\newline 
\hspace*{6pt}{<\textbf{placeName}>}Knebworth{</\textbf{placeName}>}\mbox{}\newline 
\hspace*{6pt}{</\textbf{roleName}>}\mbox{}\newline 
{</\textbf{persName}>}\end{shaded}\egroup 


    \item[{Content model}]
  \mbox{}\hfill\\[-10pt]\begin{Verbatim}[fontsize=\small]
<content>
 <macroRef key="macro.phraseSeq"/>
</content>
    
\end{Verbatim}

    \item[{Schema Declaration}]
  \mbox{}\hfill\\[-10pt]\begin{Verbatim}[fontsize=\small]
element persName
{
   att.global.attributes,
   att.datable.attributes,
   att.editLike.attributes,
   att.personal.attributes,
   att.typed.attributes,
   macro.phraseSeq}
\end{Verbatim}

\end{reflist}  \index{person=<person>|oddindex}\index{role=@role!<person>|oddindex}\index{sex=@sex!<person>|oddindex}\index{age=@age!<person>|oddindex}
\begin{reflist}
\item[]\begin{specHead}{TEI.person}{<person> }provides information about an identifiable individual, for example a participant in a language interaction, or a person referred to in a historical source. [\xref{http://www.tei-c.org/release/doc/tei-p5-doc/en/html/ND.html\#NDPERSE}{13.3.2. The Person Element} \xref{http://www.tei-c.org/release/doc/tei-p5-doc/en/html/CC.html\#CCAHPA}{15.2.2. The Participant Description}]\end{specHead} 
    \item[{Module}]
  namesdates
    \item[{Attributes}]
  Attributes att.global (\textit{@xml:id}, \textit{@n}, \textit{@xml:lang}, \textit{@xml:base}, \textit{@xml:space})  (att.global.rendition (\textit{@rend}, \textit{@style}, \textit{@rendition})) (att.global.linking (\textit{@corresp}, \textit{@synch}, \textit{@sameAs}, \textit{@copyOf}, \textit{@next}, \textit{@prev}, \textit{@exclude}, \textit{@select})) (att.global.analytic (\textit{@ana})) (att.global.facs (\textit{@facs})) (att.global.change (\textit{@change})) (att.global.responsibility (\textit{@cert}, \textit{@resp})) (att.global.source (\textit{@source})) att.editLike (\textit{@evidence}, \textit{@instant})  (att.dimensions (\textit{@unit}, \textit{@quantity}, \textit{@extent}, \textit{@precision}, \textit{@scope}) (att.ranging (\textit{@atLeast}, \textit{@atMost}, \textit{@min}, \textit{@max}, \textit{@confidence})) ) att.sortable (\textit{@sortKey}) \hfil\\[-10pt]\begin{sansreflist}
    \item[@role]
  specifies a primary role or classification for the person.
\begin{reflist}
    \item[{Status}]
  Optional
    \item[{Datatype}]
  1–∞ occurrences of teidata.enumerated separated by whitespace
    \item[{Note}]
  \par
Values for this attribute may be locally defined by a project, using arbitrary keywords such as artist, employer, author, relative, or servant, each of which should be associated with a definition. Such local definitions will typically be provided by a \texttt{<valList>} element in the project schema specification.
\end{reflist}  
    \item[@sex]
  specifies the sex of the person.
\begin{reflist}
    \item[{Status}]
  Optional
    \item[{Datatype}]
  1–∞ occurrences of teidata.sex separated by whitespace
    \item[{Note}]
  \par
Values for this attribute may be locally defined by a project, or may refer to an external standard, such as vCard's sex property \url{http://microformats.org/wiki/gender-formats} (in which M indicates male, F female, O other, N none or not applicable, U unknown), or the often used ISO 5218:2004 \textit{Representation of Human Sexes} \url{http://standards.iso.org/ittf/PubliclyAvailableStandards/c036266\textunderscore ISO\textunderscore IEC\textunderscore 5218\textunderscore 2004(E\textunderscore F).zip} (in which 0 indicates unknown; 1 male; 2 female; and 9 not applicable, although the ISO standard is widely considered inadequate); cf. CETH's \textit{Recommendations for Inclusive Data Collection of Trans People} \url{http://transhealth.ucsf.edu/trans?page=lib-data-collection}.
\end{reflist}  
    \item[@age]
  specifies an age group for the person.
\begin{reflist}
    \item[{Status}]
  Optional
    \item[{Datatype}]
  teidata.enumerated
    \item[{Note}]
  \par
Values for this attribute may be locally defined by a project, using arbitrary keywords such as infant, child, teen, adult, or senior, each of which should be associated with a definition. Such local definitions will typically be provided by a \texttt{<valList>} element in the project schema specification.
\end{reflist}  
\end{sansreflist}  
    \item[{Member of}]
  model.personLike
    \item[{Contained by}]
  
    \item[namesdates: ]
   listPerson org
    \item[{May contain}]
  
    \item[analysis: ]
   interp interpGrp span spanGrp\par 
    \item[core: ]
   bibl biblStruct cb gap gb index lb listBibl milestone note p pb\par 
    \item[figures: ]
   figure notatedMusic\par 
    \item[header: ]
   biblFull idno\par 
    \item[linking: ]
   ab alt altGrp anchor join joinGrp link linkGrp timeline\par 
    \item[msdescription: ]
   msDesc\par 
    \item[namesdates: ]
   affiliation age birth death education event faith floruit langKnowledge listEvent nationality occupation persName residence sex socecStatus state trait\par 
    \item[textcrit: ]
   app witDetail\par 
    \item[transcr: ]
   addSpan damageSpan delSpan fw listTranspose metamark space substJoin
    \item[{Note}]
  \par
May contain either a prose description organized as paragraphs, or a sequence of more specific demographic elements drawn from the \textsf{model.personPart} class.
    \item[{Example}]
  \leavevmode\bgroup\exampleFont \begin{shaded}\noindent\mbox{}{<\textbf{person}\hspace*{6pt}{age}="{adult}"\hspace*{6pt}{sex}="{F}">}\mbox{}\newline 
\hspace*{6pt}{<\textbf{p}>}Female respondent, well-educated, born in Shropshire UK, 12 Jan 1950, of unknown occupation. Speaks French fluently. Socio-Economic\mbox{}\newline 
\hspace*{6pt}\hspace*{6pt} status B2.{</\textbf{p}>}\mbox{}\newline 
{</\textbf{person}>}\end{shaded}\egroup 


    \item[{Example}]
  \leavevmode\bgroup\exampleFont \begin{shaded}\noindent\mbox{}{<\textbf{person}\hspace*{6pt}{age}="{immortal}"\hspace*{6pt}{role}="{god}"\mbox{}\newline 
\hspace*{6pt}{sex}="{intersex}">}\mbox{}\newline 
\hspace*{6pt}{<\textbf{persName}>}Hermaphroditos{</\textbf{persName}>}\mbox{}\newline 
\hspace*{6pt}{<\textbf{persName}\hspace*{6pt}{xml:lang}="{grc}">}Ἑρμαφρόδιτος{</\textbf{persName}>}\mbox{}\newline 
{</\textbf{person}>}\end{shaded}\egroup 


    \item[{Example}]
  \leavevmode\bgroup\exampleFont \begin{shaded}\noindent\mbox{}{<\textbf{person}\hspace*{6pt}{role}="{poet}"\hspace*{6pt}{sex}="{1}"\hspace*{6pt}{xml:id}="{Ovi01}">}\mbox{}\newline 
\hspace*{6pt}{<\textbf{persName}\hspace*{6pt}{xml:lang}="{en}">}Ovid{</\textbf{persName}>}\mbox{}\newline 
\hspace*{6pt}{<\textbf{persName}\hspace*{6pt}{xml:lang}="{la}">}Publius Ovidius Naso{</\textbf{persName}>}\mbox{}\newline 
\hspace*{6pt}{<\textbf{birth}\hspace*{6pt}{when}="{-0044-03-20}">} 20 March 43 BC {<\textbf{placeName}>}\mbox{}\newline 
\hspace*{6pt}\hspace*{6pt}\hspace*{6pt}{<\textbf{settlement}\hspace*{6pt}{type}="{city}">}Sulmona{</\textbf{settlement}>}\mbox{}\newline 
\hspace*{6pt}\hspace*{6pt}\hspace*{6pt}{<\textbf{country}\hspace*{6pt}{key}="{IT}">}Italy{</\textbf{country}>}\mbox{}\newline 
\hspace*{6pt}\hspace*{6pt}{</\textbf{placeName}>}\mbox{}\newline 
\hspace*{6pt}{</\textbf{birth}>}\mbox{}\newline 
\hspace*{6pt}{<\textbf{death}\hspace*{6pt}{notAfter}="{0018}"\hspace*{6pt}{notBefore}="{0017}">}17 or 18 AD {<\textbf{placeName}>}\mbox{}\newline 
\hspace*{6pt}\hspace*{6pt}\hspace*{6pt}{<\textbf{settlement}\hspace*{6pt}{type}="{city}">}Tomis (Constanta){</\textbf{settlement}>}\mbox{}\newline 
\hspace*{6pt}\hspace*{6pt}\hspace*{6pt}{<\textbf{country}\hspace*{6pt}{key}="{RO}">}Romania{</\textbf{country}>}\mbox{}\newline 
\hspace*{6pt}\hspace*{6pt}{</\textbf{placeName}>}\mbox{}\newline 
\hspace*{6pt}{</\textbf{death}>}\mbox{}\newline 
{</\textbf{person}>}\end{shaded}\egroup 


    \item[{Content model}]
  \mbox{}\hfill\\[-10pt]\begin{Verbatim}[fontsize=\small]
<content>
 <alternate>
  <classRef key="model.pLike"
   maxOccurs="unbounded" minOccurs="1"/>
  <alternate maxOccurs="unbounded"
   minOccurs="0">
   <classRef key="model.personPart"/>
   <classRef key="model.global"/>
  </alternate>
 </alternate>
</content>
    
\end{Verbatim}

    \item[{Schema Declaration}]
  \mbox{}\hfill\\[-10pt]\begin{Verbatim}[fontsize=\small]
element person
{
   att.global.attributes,
   att.editLike.attributes,
   att.sortable.attributes,
   attribute role { list { + } }?,
   attribute sex { list { + } }?,
   attribute age { text }?,
   ( model.pLike+ | ( model.personPart | model.global )* )
}
\end{Verbatim}

\end{reflist}  \index{personGrp=<personGrp>|oddindex}\index{role=@role!<personGrp>|oddindex}\index{sex=@sex!<personGrp>|oddindex}\index{age=@age!<personGrp>|oddindex}\index{size=@size!<personGrp>|oddindex}
\begin{reflist}
\item[]\begin{specHead}{TEI.personGrp}{<personGrp> }(personal group) describes a group of individuals treated as a single person for analytic purposes. [\xref{http://www.tei-c.org/release/doc/tei-p5-doc/en/html/CC.html\#CCAHPA}{15.2.2. The Participant Description}]\end{specHead} 
    \item[{Module}]
  namesdates
    \item[{Attributes}]
  Attributes att.global (\textit{@xml:id}, \textit{@n}, \textit{@xml:lang}, \textit{@xml:base}, \textit{@xml:space})  (att.global.rendition (\textit{@rend}, \textit{@style}, \textit{@rendition})) (att.global.linking (\textit{@corresp}, \textit{@synch}, \textit{@sameAs}, \textit{@copyOf}, \textit{@next}, \textit{@prev}, \textit{@exclude}, \textit{@select})) (att.global.analytic (\textit{@ana})) (att.global.facs (\textit{@facs})) (att.global.change (\textit{@change})) (att.global.responsibility (\textit{@cert}, \textit{@resp})) (att.global.source (\textit{@source})) att.sortable (\textit{@sortKey}) \hfil\\[-10pt]\begin{sansreflist}
    \item[@role]
  specifies the role of this group of participants in the interaction.
\begin{reflist}
    \item[{Status}]
  Optional
    \item[{Datatype}]
  teidata.enumerated
    \item[{Note}]
  \par
Values for this attribute may be locally defined by a project, using arbitrary keywords such as movement, employers, relatives, or servants, each of which should be associated with a definition. Such local definitions will typically be provided by a \texttt{<valList>} element in the project schema specification.
\end{reflist}  
    \item[@sex]
  specifies the sex of the participant group.
\begin{reflist}
    \item[{Status}]
  Optional
    \item[{Datatype}]
  1–∞ occurrences of teidata.sex separated by whitespace
    \item[{Note}]
  \par
Values for this attribute may be locally defined by a project, or may refer to an external standard, such as vCard's sex property \url{http://microformats.org/wiki/gender-formats} (in which M indicates male, F female, O other, N none or not applicable, U unknown), or the often used ISO 5218:2004 \textit{Representation of Human Sexes} \url{http://standards.iso.org/ittf/PubliclyAvailableStandards/c036266\textunderscore ISO\textunderscore IEC\textunderscore 5218\textunderscore 2004(E\textunderscore F).zip} (in which 0 indicates unknown; 1 male; 2 female; and 9 not applicable, although the ISO standard is widely considered inadequate); cf. CETH's \textit{Recommendations for Inclusive Data Collection of Trans People} \url{http://transhealth.ucsf.edu/trans?page=lib-data-collection}. For a mixed group, a value such as "mixed" may also be supplied.
\end{reflist}  
    \item[@age]
  specifies the age group of the participants.
\begin{reflist}
    \item[{Status}]
  Optional
    \item[{Datatype}]
  teidata.enumerated
    \item[{Note}]
  \par
Values for this attribute may be locally defined by a project, using arbitrary keywords such as infant, child, teen, adult, or senior, each of which should be associated with a definition. Such local definitions will typically be provided by a \texttt{<valList>} element in the project schema specification.
\end{reflist}  
    \item[@size]
  describes informally the size or approximate size of the group for example by means of a number and an indication of accuracy e.g. approx 200.
\begin{reflist}
    \item[{Status}]
  Optional
    \item[{Datatype}]
  1–∞ occurrences of teidata.word separated by whitespace
\end{reflist}  
\end{sansreflist}  
    \item[{Member of}]
  model.personLike
    \item[{Contained by}]
  
    \item[namesdates: ]
   listPerson org
    \item[{May contain}]
  
    \item[analysis: ]
   interp interpGrp span spanGrp\par 
    \item[core: ]
   bibl biblStruct cb gap gb index lb listBibl milestone note p pb\par 
    \item[figures: ]
   figure notatedMusic\par 
    \item[header: ]
   biblFull idno\par 
    \item[linking: ]
   ab alt altGrp anchor join joinGrp link linkGrp timeline\par 
    \item[msdescription: ]
   msDesc\par 
    \item[namesdates: ]
   affiliation age birth death education event faith floruit langKnowledge listEvent nationality occupation persName residence sex socecStatus state trait\par 
    \item[textcrit: ]
   app witDetail\par 
    \item[transcr: ]
   addSpan damageSpan delSpan fw listTranspose metamark space substJoin
    \item[{Note}]
  \par
May contain a prose description organized as paragraphs, or any sequence of demographic elements in any combination.\par
The global {\itshape xml:id} attribute should be used to identify each speaking participant in a spoken text if the {\itshape who} attribute is specified on individual utterances.
    \item[{Example}]
  \leavevmode\bgroup\exampleFont \begin{shaded}\noindent\mbox{}{<\textbf{personGrp}\hspace*{6pt}{role}="{audience}"\hspace*{6pt}{sex}="{mixed}"\mbox{}\newline 
\hspace*{6pt}{size}="{approx 50}"\hspace*{6pt}{xml:id}="{pg1}"/>}\end{shaded}\egroup 


    \item[{Content model}]
  \mbox{}\hfill\\[-10pt]\begin{Verbatim}[fontsize=\small]
<content>
 <alternate>
  <classRef key="model.pLike"
   maxOccurs="unbounded" minOccurs="1"/>
  <alternate maxOccurs="unbounded"
   minOccurs="0">
   <classRef key="model.personPart"/>
   <classRef key="model.global"/>
  </alternate>
 </alternate>
</content>
    
\end{Verbatim}

    \item[{Schema Declaration}]
  \mbox{}\hfill\\[-10pt]\begin{Verbatim}[fontsize=\small]
element personGrp
{
   att.global.attributes,
   att.sortable.attributes,
   attribute role { text }?,
   attribute sex { list { + } }?,
   attribute age { text }?,
   attribute size { list { + } }?,
   ( model.pLike+ | ( model.personPart | model.global )* )
}
\end{Verbatim}

\end{reflist}  \index{phr=<phr>|oddindex}
\begin{reflist}
\item[]\begin{specHead}{TEI.phr}{<phr> }(phrase) represents a grammatical phrase. [\xref{http://www.tei-c.org/release/doc/tei-p5-doc/en/html/AI.html\#AILC}{17.1. Linguistic Segment Categories}]\end{specHead} 
    \item[{Module}]
  analysis
    \item[{Attributes}]
  Attributes att.global (\textit{@xml:id}, \textit{@n}, \textit{@xml:lang}, \textit{@xml:base}, \textit{@xml:space})  (att.global.rendition (\textit{@rend}, \textit{@style}, \textit{@rendition})) (att.global.linking (\textit{@corresp}, \textit{@synch}, \textit{@sameAs}, \textit{@copyOf}, \textit{@next}, \textit{@prev}, \textit{@exclude}, \textit{@select})) (att.global.analytic (\textit{@ana})) (att.global.facs (\textit{@facs})) (att.global.change (\textit{@change})) (att.global.responsibility (\textit{@cert}, \textit{@resp})) (att.global.source (\textit{@source})) att.segLike (\textit{@function})  (att.datcat (\textit{@datcat}, \textit{@valueDatcat})) (att.fragmentable (\textit{@part})) att.typed (\textit{@type}, \textit{@subtype}) 
    \item[{Member of}]
  model.segLike
    \item[{Contained by}]
  
    \item[analysis: ]
   cl phr s\par 
    \item[core: ]
   abbr add addrLine author bibl biblScope citedRange corr date del distinct editor email emph expan foreign gloss head headItem headLabel hi item l label measure mentioned name note num orig p pubPlace publisher q quote ref reg rs said sic soCalled speaker stage street term textLang time title unclear\par 
    \item[figures: ]
   cell\par 
    \item[header: ]
   change distributor edition extent geoDecl handNote licence scriptNote typeNote\par 
    \item[linking: ]
   ab seg\par 
    \item[msdescription: ]
   accMat acquisition additions catchwords collation colophon condition custEvent decoNote explicit filiation finalRubric foliation heraldry incipit layout material musicNotation objectType origDate origPlace origin provenance rubric secFol signatures source stamp summary support surrogates watermark\par 
    \item[namesdates: ]
   addName affiliation birth bloc country death district education faith floruit forename genName geogFeat geogName nameLink nationality occupation offset orgName persName placeName region residence roleName settlement sex socecStatus surname\par 
    \item[textcrit: ]
   lem rdg wit witDetail\par 
    \item[textstructure: ]
   byline closer dateline docAuthor docDate docEdition docImprint imprimatur opener salute signed titlePart trailer\par 
    \item[transcr: ]
   damage fw metamark mod restore retrace secl supplied surplus
    \item[{May contain}]
  
    \item[analysis: ]
   c cl interp interpGrp m pc phr s span spanGrp w\par 
    \item[core: ]
   abbr add address cb choice corr date del distinct email emph expan foreign gap gb gloss graphic hi index lb measure measureGrp media mentioned milestone name note num orig pb ptr ref reg rs sic soCalled term time title unclear\par 
    \item[figures: ]
   figure formula notatedMusic\par 
    \item[gaiji: ]
   g\par 
    \item[header: ]
   idno\par 
    \item[linking: ]
   alt altGrp anchor join joinGrp link linkGrp seg timeline\par 
    \item[msdescription: ]
   catchwords depth dim dimensions height heraldry locus locusGrp material objectType origDate origPlace secFol signatures stamp watermark width\par 
    \item[namesdates: ]
   addName affiliation bloc climate country district forename genName geo geogFeat geogName location nameLink offset orgName persName placeName population region roleName settlement state surname terrain trait\par 
    \item[textcrit: ]
   app witDetail\par 
    \item[transcr: ]
   addSpan am damage damageSpan delSpan ex fw handShift listTranspose metamark mod redo restore retrace secl space subst substJoin supplied surplus undo\par character data
    \item[{Note}]
  \par
The {\itshape type} attribute may be used to indicate the type of phrase, taking values such as noun, verb, preposition, etc. as appropriate.
    \item[{Example}]
  \leavevmode\bgroup\exampleFont \begin{shaded}\noindent\mbox{}{<\textbf{phr}\hspace*{6pt}{function}="{extraposted\textunderscore modifier}"\mbox{}\newline 
\hspace*{6pt}{type}="{verb}">}To talk\mbox{}\newline 
{<\textbf{phr}\hspace*{6pt}{function}="{complement}"\mbox{}\newline 
\hspace*{6pt}\hspace*{6pt}{type}="{preposition}">}of\mbox{}\newline 
\hspace*{6pt}{<\textbf{phr}\hspace*{6pt}{function}="{object}"\hspace*{6pt}{type}="{noun}">}many things{</\textbf{phr}>}\mbox{}\newline 
\hspace*{6pt}{</\textbf{phr}>}\mbox{}\newline 
{</\textbf{phr}>}\end{shaded}\egroup 


    \item[{Content model}]
  \mbox{}\hfill\\[-10pt]\begin{Verbatim}[fontsize=\small]
<content>
 <macroRef key="macro.phraseSeq"/>
</content>
    
\end{Verbatim}

    \item[{Schema Declaration}]
  \mbox{}\hfill\\[-10pt]\begin{Verbatim}[fontsize=\small]
element phr
{
   att.global.attributes,
   att.segLike.attributes,
   att.typed.attributes,
   macro.phraseSeq}
\end{Verbatim}

\end{reflist}  \index{physDesc=<physDesc>|oddindex}
\begin{reflist}
\item[]\begin{specHead}{TEI.physDesc}{<physDesc> }(physical description) contains a full physical description of a manuscript or manuscript part, optionally subdivided using more specialized elements from the \textsf{model.physDescPart} class. [\xref{http://www.tei-c.org/release/doc/tei-p5-doc/en/html/MS.html\#msph}{10.7. Physical Description}]\end{specHead} 
    \item[{Module}]
  msdescription
    \item[{Attributes}]
  Attributes att.global (\textit{@xml:id}, \textit{@n}, \textit{@xml:lang}, \textit{@xml:base}, \textit{@xml:space})  (att.global.rendition (\textit{@rend}, \textit{@style}, \textit{@rendition})) (att.global.linking (\textit{@corresp}, \textit{@synch}, \textit{@sameAs}, \textit{@copyOf}, \textit{@next}, \textit{@prev}, \textit{@exclude}, \textit{@select})) (att.global.analytic (\textit{@ana})) (att.global.facs (\textit{@facs})) (att.global.change (\textit{@change})) (att.global.responsibility (\textit{@cert}, \textit{@resp})) (att.global.source (\textit{@source}))
    \item[{Contained by}]
  
    \item[msdescription: ]
   msDesc msFrag msPart
    \item[{May contain}]
  
    \item[core: ]
   p\par 
    \item[linking: ]
   ab\par 
    \item[msdescription: ]
   accMat additions bindingDesc decoDesc handDesc musicNotation objectDesc scriptDesc sealDesc typeDesc
    \item[{Example}]
  \leavevmode\bgroup\exampleFont \begin{shaded}\noindent\mbox{}{<\textbf{physDesc}>}\mbox{}\newline 
\hspace*{6pt}{<\textbf{objectDesc}\hspace*{6pt}{form}="{codex}">}\mbox{}\newline 
\hspace*{6pt}\hspace*{6pt}{<\textbf{supportDesc}\hspace*{6pt}{material}="{perg}">}\mbox{}\newline 
\hspace*{6pt}\hspace*{6pt}\hspace*{6pt}{<\textbf{support}>}Parchment.{</\textbf{support}>}\mbox{}\newline 
\hspace*{6pt}\hspace*{6pt}\hspace*{6pt}{<\textbf{extent}>}i + 55 leaves\mbox{}\newline 
\hspace*{6pt}\hspace*{6pt}\hspace*{6pt}{<\textbf{dimensions}\hspace*{6pt}{scope}="{all}"\hspace*{6pt}{type}="{leaf}"\mbox{}\newline 
\hspace*{6pt}\hspace*{6pt}\hspace*{6pt}\hspace*{6pt}\hspace*{6pt}{unit}="{inch}">}\mbox{}\newline 
\hspace*{6pt}\hspace*{6pt}\hspace*{6pt}\hspace*{6pt}\hspace*{6pt}{<\textbf{height}>}7¼{</\textbf{height}>}\mbox{}\newline 
\hspace*{6pt}\hspace*{6pt}\hspace*{6pt}\hspace*{6pt}\hspace*{6pt}{<\textbf{width}>}5⅜{</\textbf{width}>}\mbox{}\newline 
\hspace*{6pt}\hspace*{6pt}\hspace*{6pt}\hspace*{6pt}{</\textbf{dimensions}>}\mbox{}\newline 
\hspace*{6pt}\hspace*{6pt}\hspace*{6pt}{</\textbf{extent}>}\mbox{}\newline 
\hspace*{6pt}\hspace*{6pt}{</\textbf{supportDesc}>}\mbox{}\newline 
\hspace*{6pt}\hspace*{6pt}{<\textbf{layoutDesc}>}\mbox{}\newline 
\hspace*{6pt}\hspace*{6pt}\hspace*{6pt}{<\textbf{layout}\hspace*{6pt}{columns}="{2}">}In double columns.{</\textbf{layout}>}\mbox{}\newline 
\hspace*{6pt}\hspace*{6pt}{</\textbf{layoutDesc}>}\mbox{}\newline 
\hspace*{6pt}{</\textbf{objectDesc}>}\mbox{}\newline 
\hspace*{6pt}{<\textbf{handDesc}>}\mbox{}\newline 
\hspace*{6pt}\hspace*{6pt}{<\textbf{p}>}Written in more than one hand.{</\textbf{p}>}\mbox{}\newline 
\hspace*{6pt}{</\textbf{handDesc}>}\mbox{}\newline 
\hspace*{6pt}{<\textbf{decoDesc}>}\mbox{}\newline 
\hspace*{6pt}\hspace*{6pt}{<\textbf{p}>}With a few coloured capitals.{</\textbf{p}>}\mbox{}\newline 
\hspace*{6pt}{</\textbf{decoDesc}>}\mbox{}\newline 
{</\textbf{physDesc}>}\end{shaded}\egroup 


    \item[{Content model}]
  \mbox{}\hfill\\[-10pt]\begin{Verbatim}[fontsize=\small]
<content>
 <sequence>
  <classRef key="model.pLike"
   maxOccurs="unbounded" minOccurs="0"/>
  <classRef expand="sequenceOptional"
   key="model.physDescPart"/>
 </sequence>
</content>
    
\end{Verbatim}

    \item[{Schema Declaration}]
  \mbox{}\hfill\\[-10pt]\begin{Verbatim}[fontsize=\small]
element physDesc
{
   att.global.attributes,
   (
      model.pLike*,
      objectDesc?,
      handDesc?,
      typeDesc?,
      scriptDesc?,
      musicNotation?,
      decoDesc?,
      additions?,
      bindingDesc?,
      sealDesc?,
      accMat?
   )
}
\end{Verbatim}

\end{reflist}  \index{place=<place>|oddindex}
\begin{reflist}
\item[]\begin{specHead}{TEI.place}{<place> }contains data about a geographic location [\xref{http://www.tei-c.org/release/doc/tei-p5-doc/en/html/ND.html\#NDGEOG}{13.3.4. Places}]\end{specHead} 
    \item[{Module}]
  namesdates
    \item[{Attributes}]
  Attributes att.global (\textit{@xml:id}, \textit{@n}, \textit{@xml:lang}, \textit{@xml:base}, \textit{@xml:space})  (att.global.rendition (\textit{@rend}, \textit{@style}, \textit{@rendition})) (att.global.linking (\textit{@corresp}, \textit{@synch}, \textit{@sameAs}, \textit{@copyOf}, \textit{@next}, \textit{@prev}, \textit{@exclude}, \textit{@select})) (att.global.analytic (\textit{@ana})) (att.global.facs (\textit{@facs})) (att.global.change (\textit{@change})) (att.global.responsibility (\textit{@cert}, \textit{@resp})) (att.global.source (\textit{@source})) att.typed (\textit{@type}, \textit{@subtype}) att.editLike (\textit{@evidence}, \textit{@instant})  (att.dimensions (\textit{@unit}, \textit{@quantity}, \textit{@extent}, \textit{@precision}, \textit{@scope}) (att.ranging (\textit{@atLeast}, \textit{@atMost}, \textit{@min}, \textit{@max}, \textit{@confidence})) ) att.sortable (\textit{@sortKey}) 
    \item[{Member of}]
  model.placeLike
    \item[{Contained by}]
  
    \item[namesdates: ]
   listPlace org place
    \item[{May contain}]
  
    \item[core: ]
   bibl biblStruct desc head label listBibl note p\par 
    \item[header: ]
   biblFull idno\par 
    \item[linking: ]
   ab link linkGrp\par 
    \item[msdescription: ]
   msDesc\par 
    \item[namesdates: ]
   bloc climate country district event geogName listEvent listPlace location place placeName population region settlement state terrain trait\par 
    \item[textcrit: ]
   witDetail
    \item[{Example}]
  \leavevmode\bgroup\exampleFont \begin{shaded}\noindent\mbox{}{<\textbf{place}>}\mbox{}\newline 
\hspace*{6pt}{<\textbf{country}>}Lithuania{</\textbf{country}>}\mbox{}\newline 
\hspace*{6pt}{<\textbf{country}\hspace*{6pt}{xml:lang}="{lt}">}Lietuva{</\textbf{country}>}\mbox{}\newline 
\hspace*{6pt}{<\textbf{place}>}\mbox{}\newline 
\hspace*{6pt}\hspace*{6pt}{<\textbf{settlement}>}Vilnius{</\textbf{settlement}>}\mbox{}\newline 
\hspace*{6pt}{</\textbf{place}>}\mbox{}\newline 
\hspace*{6pt}{<\textbf{place}>}\mbox{}\newline 
\hspace*{6pt}\hspace*{6pt}{<\textbf{settlement}>}Kaunas{</\textbf{settlement}>}\mbox{}\newline 
\hspace*{6pt}{</\textbf{place}>}\mbox{}\newline 
{</\textbf{place}>}\end{shaded}\egroup 


    \item[{Content model}]
  \mbox{}\hfill\\[-10pt]\begin{Verbatim}[fontsize=\small]
<content>
 <sequence>
  <classRef key="model.headLike"
   maxOccurs="unbounded" minOccurs="0"/>
  <alternate>
   <classRef key="model.pLike"
    maxOccurs="unbounded" minOccurs="0"/>
   <alternate maxOccurs="unbounded"
    minOccurs="0">
    <classRef key="model.labelLike"/>
    <classRef key="model.placeStateLike"/>
    <classRef key="model.eventLike"/>
   </alternate>
  </alternate>
  <alternate maxOccurs="unbounded"
   minOccurs="0">
   <classRef key="model.noteLike"/>
   <classRef key="model.biblLike"/>
   <elementRef key="idno"/>
   <elementRef key="linkGrp"/>
   <elementRef key="link"/>
  </alternate>
  <alternate maxOccurs="unbounded"
   minOccurs="0">
   <classRef key="model.placeLike"/>
   <elementRef key="listPlace"/>
  </alternate>
 </sequence>
</content>
    
\end{Verbatim}

    \item[{Schema Declaration}]
  \mbox{}\hfill\\[-10pt]\begin{Verbatim}[fontsize=\small]
element place
{
   att.global.attributes,
   att.typed.attributes,
   att.editLike.attributes,
   att.sortable.attributes,
   (
      model.headLike*,
      (
         model.pLike*
       | ( model.labelLike | model.placeStateLike | model.eventLike )*
      ),
      ( model.noteLike | model.biblLike | idno | linkGrp | link )*,
      ( model.placeLike | listPlace )*
   )
}
\end{Verbatim}

\end{reflist}  \index{placeName=<placeName>|oddindex}
\begin{reflist}
\item[]\begin{specHead}{TEI.placeName}{<placeName> }contains an absolute or relative place name. [\xref{http://www.tei-c.org/release/doc/tei-p5-doc/en/html/ND.html\#NDPLAC}{13.2.3. Place Names}]\end{specHead} 
    \item[{Module}]
  namesdates
    \item[{Attributes}]
  Attributes att.datable (\textit{@calendar}, \textit{@period})  (att.datable.w3c (\textit{@when}, \textit{@notBefore}, \textit{@notAfter}, \textit{@from}, \textit{@to})) (att.datable.iso (\textit{@when-iso}, \textit{@notBefore-iso}, \textit{@notAfter-iso}, \textit{@from-iso}, \textit{@to-iso})) (att.datable.custom (\textit{@when-custom}, \textit{@notBefore-custom}, \textit{@notAfter-custom}, \textit{@from-custom}, \textit{@to-custom}, \textit{@datingPoint}, \textit{@datingMethod})) att.editLike (\textit{@evidence}, \textit{@instant})  (att.dimensions (\textit{@unit}, \textit{@quantity}, \textit{@extent}, \textit{@precision}, \textit{@scope}) (att.ranging (\textit{@atLeast}, \textit{@atMost}, \textit{@min}, \textit{@max}, \textit{@confidence})) ) att.global (\textit{@xml:id}, \textit{@n}, \textit{@xml:lang}, \textit{@xml:base}, \textit{@xml:space})  (att.global.rendition (\textit{@rend}, \textit{@style}, \textit{@rendition})) (att.global.linking (\textit{@corresp}, \textit{@synch}, \textit{@sameAs}, \textit{@copyOf}, \textit{@next}, \textit{@prev}, \textit{@exclude}, \textit{@select})) (att.global.analytic (\textit{@ana})) (att.global.facs (\textit{@facs})) (att.global.change (\textit{@change})) (att.global.responsibility (\textit{@cert}, \textit{@resp})) (att.global.source (\textit{@source})) att.personal (\textit{@full}, \textit{@sort})  (att.naming (\textit{@role}, \textit{@nymRef}) (att.canonical (\textit{@key}, \textit{@ref})) ) att.typed (\textit{@type}, \textit{@subtype}) 
    \item[{Member of}]
  model.placeNamePart
    \item[{Contained by}]
  
    \item[analysis: ]
   cl phr s span\par 
    \item[core: ]
   abbr add addrLine address author bibl biblScope citedRange corr date del desc distinct editor email emph expan foreign gloss head headItem headLabel hi item l label measure meeting mentioned name note num orig p pubPlace publisher q quote ref reg resp rs said sic soCalled speaker stage street term textLang time title unclear\par 
    \item[figures: ]
   cell figDesc\par 
    \item[header: ]
   authority catDesc change classCode correspAction creation distributor edition extent funder geoDecl handNote language licence principal rendition scriptNote sponsor tagUsage typeNote\par 
    \item[linking: ]
   ab seg\par 
    \item[msdescription: ]
   accMat acquisition additions altIdentifier catchwords collation colophon condition custEvent decoNote explicit filiation finalRubric foliation heraldry incipit layout material msIdentifier musicNotation objectType origDate origPlace origin provenance rubric secFol signatures source stamp summary support surrogates watermark\par 
    \item[namesdates: ]
   addName affiliation age birth bloc country death district education faith floruit forename genName geogFeat geogName langKnown location nameLink nationality occupation offset org orgName persName place placeName region residence roleName settlement sex socecStatus surname\par 
    \item[textcrit: ]
   lem rdg wit witDetail witness\par 
    \item[textstructure: ]
   byline closer dateline docAuthor docDate docEdition docImprint imprimatur opener salute signed titlePart trailer\par 
    \item[transcr: ]
   damage fw metamark mod restore retrace secl supplied surplus
    \item[{May contain}]
  
    \item[analysis: ]
   c cl interp interpGrp m pc phr s span spanGrp w\par 
    \item[core: ]
   abbr add address cb choice corr date del distinct email emph expan foreign gap gb gloss graphic hi index lb measure measureGrp media mentioned milestone name note num orig pb ptr ref reg rs sic soCalled term time title unclear\par 
    \item[figures: ]
   figure formula notatedMusic\par 
    \item[gaiji: ]
   g\par 
    \item[header: ]
   idno\par 
    \item[linking: ]
   alt altGrp anchor join joinGrp link linkGrp seg timeline\par 
    \item[msdescription: ]
   catchwords depth dim dimensions height heraldry locus locusGrp material objectType origDate origPlace secFol signatures stamp watermark width\par 
    \item[namesdates: ]
   addName affiliation bloc climate country district forename genName geo geogFeat geogName location nameLink offset orgName persName placeName population region roleName settlement state surname terrain trait\par 
    \item[textcrit: ]
   app witDetail\par 
    \item[transcr: ]
   addSpan am damage damageSpan delSpan ex fw handShift listTranspose metamark mod redo restore retrace secl space subst substJoin supplied surplus undo\par character data
    \item[{Example}]
  \leavevmode\bgroup\exampleFont \begin{shaded}\noindent\mbox{}{<\textbf{placeName}>}\mbox{}\newline 
\hspace*{6pt}{<\textbf{settlement}>}Rochester{</\textbf{settlement}>}\mbox{}\newline 
\hspace*{6pt}{<\textbf{region}>}New York{</\textbf{region}>}\mbox{}\newline 
{</\textbf{placeName}>}\end{shaded}\egroup 


    \item[{Example}]
  \leavevmode\bgroup\exampleFont \begin{shaded}\noindent\mbox{}{<\textbf{placeName}>}\mbox{}\newline 
\hspace*{6pt}{<\textbf{geogName}>}Arrochar Alps{</\textbf{geogName}>}\mbox{}\newline 
\hspace*{6pt}{<\textbf{region}>}Argylshire{</\textbf{region}>}\mbox{}\newline 
{</\textbf{placeName}>}\end{shaded}\egroup 


    \item[{Example}]
  \leavevmode\bgroup\exampleFont \begin{shaded}\noindent\mbox{}{<\textbf{placeName}>}\mbox{}\newline 
\hspace*{6pt}{<\textbf{measure}>}10 miles{</\textbf{measure}>}\mbox{}\newline 
\hspace*{6pt}{<\textbf{offset}>}Northeast of{</\textbf{offset}>}\mbox{}\newline 
\hspace*{6pt}{<\textbf{settlement}>}Attica{</\textbf{settlement}>}\mbox{}\newline 
{</\textbf{placeName}>}\end{shaded}\egroup 


    \item[{Content model}]
  \mbox{}\hfill\\[-10pt]\begin{Verbatim}[fontsize=\small]
<content>
 <macroRef key="macro.phraseSeq"/>
</content>
    
\end{Verbatim}

    \item[{Schema Declaration}]
  \mbox{}\hfill\\[-10pt]\begin{Verbatim}[fontsize=\small]
element placeName
{
   att.datable.attributes,
   att.editLike.attributes,
   att.global.attributes,
   att.personal.attributes,
   att.typed.attributes,
   macro.phraseSeq}
\end{Verbatim}

\end{reflist}  \index{population=<population>|oddindex}
\begin{reflist}
\item[]\begin{specHead}{TEI.population}{<population> }contains information about the population of a place. [\xref{http://www.tei-c.org/release/doc/tei-p5-doc/en/html/ND.html\#NDGEOGste}{13.3.4.3. States, Traits, and Events}]\end{specHead} 
    \item[{Module}]
  namesdates
    \item[{Attributes}]
  Attributes att.global (\textit{@xml:id}, \textit{@n}, \textit{@xml:lang}, \textit{@xml:base}, \textit{@xml:space})  (att.global.rendition (\textit{@rend}, \textit{@style}, \textit{@rendition})) (att.global.linking (\textit{@corresp}, \textit{@synch}, \textit{@sameAs}, \textit{@copyOf}, \textit{@next}, \textit{@prev}, \textit{@exclude}, \textit{@select})) (att.global.analytic (\textit{@ana})) (att.global.facs (\textit{@facs})) (att.global.change (\textit{@change})) (att.global.responsibility (\textit{@cert}, \textit{@resp})) (att.global.source (\textit{@source})) att.datable (\textit{@calendar}, \textit{@period})  (att.datable.w3c (\textit{@when}, \textit{@notBefore}, \textit{@notAfter}, \textit{@from}, \textit{@to})) (att.datable.iso (\textit{@when-iso}, \textit{@notBefore-iso}, \textit{@notAfter-iso}, \textit{@from-iso}, \textit{@to-iso})) (att.datable.custom (\textit{@when-custom}, \textit{@notBefore-custom}, \textit{@notAfter-custom}, \textit{@from-custom}, \textit{@to-custom}, \textit{@datingPoint}, \textit{@datingMethod})) att.editLike (\textit{@evidence}, \textit{@instant})  (att.dimensions (\textit{@unit}, \textit{@quantity}, \textit{@extent}, \textit{@precision}, \textit{@scope}) (att.ranging (\textit{@atLeast}, \textit{@atMost}, \textit{@min}, \textit{@max}, \textit{@confidence})) ) att.naming (\textit{@role}, \textit{@nymRef})  (att.canonical (\textit{@key}, \textit{@ref})) att.typed (\textit{@type}, \textit{@subtype}) 
    \item[{Member of}]
  model.placeStateLike 
    \item[{Contained by}]
  
    \item[analysis: ]
   cl phr s span\par 
    \item[core: ]
   abbr add addrLine address author bibl biblScope citedRange corr date del desc distinct editor email emph expan foreign gloss head headItem headLabel hi item l label measure meeting mentioned name note num orig p pubPlace publisher q quote ref reg resp rs said sic soCalled speaker stage street term textLang time title unclear\par 
    \item[figures: ]
   cell figDesc\par 
    \item[header: ]
   authority catDesc change classCode correspAction creation distributor edition extent funder geoDecl handNote language licence principal rendition scriptNote sponsor tagUsage typeNote\par 
    \item[linking: ]
   ab seg\par 
    \item[msdescription: ]
   accMat acquisition additions catchwords collation colophon condition custEvent decoNote explicit filiation finalRubric foliation heraldry incipit layout material musicNotation objectType origDate origPlace origin provenance rubric secFol signatures source stamp summary support surrogates watermark\par 
    \item[namesdates: ]
   addName affiliation age birth bloc country death district education faith floruit forename genName geogFeat geogName langKnown nameLink nationality occupation offset org orgName persName place placeName population region residence roleName settlement sex socecStatus surname\par 
    \item[textcrit: ]
   lem rdg wit witDetail witness\par 
    \item[textstructure: ]
   byline closer dateline docAuthor docDate docEdition docImprint imprimatur opener salute signed titlePart trailer\par 
    \item[transcr: ]
   damage fw metamark mod restore retrace secl supplied surplus
    \item[{May contain}]
  
    \item[core: ]
   bibl biblStruct desc head label listBibl note p\par 
    \item[header: ]
   biblFull\par 
    \item[linking: ]
   ab\par 
    \item[msdescription: ]
   msDesc\par 
    \item[namesdates: ]
   population\par 
    \item[textcrit: ]
   witDetail
    \item[{Example}]
  \leavevmode\bgroup\exampleFont \begin{shaded}\noindent\mbox{}{<\textbf{population}\hspace*{6pt}{resp}="{\#UKCensus}"\hspace*{6pt}{when}="{2001-04}">}\mbox{}\newline 
\hspace*{6pt}{<\textbf{population}\hspace*{6pt}{type}="{white}">}\mbox{}\newline 
\hspace*{6pt}\hspace*{6pt}{<\textbf{desc}>}54153898{</\textbf{desc}>}\mbox{}\newline 
\hspace*{6pt}{</\textbf{population}>}\mbox{}\newline 
\hspace*{6pt}{<\textbf{population}\hspace*{6pt}{type}="{asian}">}\mbox{}\newline 
\hspace*{6pt}\hspace*{6pt}{<\textbf{desc}>}11811423{</\textbf{desc}>}\mbox{}\newline 
\hspace*{6pt}{</\textbf{population}>}\mbox{}\newline 
\hspace*{6pt}{<\textbf{population}\hspace*{6pt}{type}="{black}">}\mbox{}\newline 
\hspace*{6pt}\hspace*{6pt}{<\textbf{desc}>}1148738{</\textbf{desc}>}\mbox{}\newline 
\hspace*{6pt}{</\textbf{population}>}\mbox{}\newline 
\hspace*{6pt}{<\textbf{population}\hspace*{6pt}{type}="{mixed}">}\mbox{}\newline 
\hspace*{6pt}\hspace*{6pt}{<\textbf{desc}>}677117{</\textbf{desc}>}\mbox{}\newline 
\hspace*{6pt}{</\textbf{population}>}\mbox{}\newline 
\hspace*{6pt}{<\textbf{population}\hspace*{6pt}{type}="{chinese}">}\mbox{}\newline 
\hspace*{6pt}\hspace*{6pt}{<\textbf{desc}>}247403{</\textbf{desc}>}\mbox{}\newline 
\hspace*{6pt}{</\textbf{population}>}\mbox{}\newline 
\hspace*{6pt}{<\textbf{population}\hspace*{6pt}{type}="{other}">}\mbox{}\newline 
\hspace*{6pt}\hspace*{6pt}{<\textbf{desc}>}230615{</\textbf{desc}>}\mbox{}\newline 
\hspace*{6pt}{</\textbf{population}>}\mbox{}\newline 
{</\textbf{population}>}\end{shaded}\egroup 


    \item[{Content model}]
  \mbox{}\hfill\\[-10pt]\begin{Verbatim}[fontsize=\small]
<content>
 <sequence>
  <elementRef key="precision"
   maxOccurs="unbounded" minOccurs="0"/>
  <classRef key="model.headLike"
   maxOccurs="unbounded" minOccurs="0"/>
  <sequence minOccurs="0">
   <alternate>
    <classRef key="model.pLike"
     maxOccurs="unbounded" minOccurs="1"/>
    <classRef key="model.labelLike"
     maxOccurs="unbounded" minOccurs="1"/>
   </alternate>
   <alternate maxOccurs="unbounded"
    minOccurs="0">
    <classRef key="model.noteLike"/>
    <classRef key="model.biblLike"/>
   </alternate>
  </sequence>
  <elementRef key="population"
   maxOccurs="unbounded" minOccurs="0"/>
 </sequence>
</content>
    
\end{Verbatim}

    \item[{Schema Declaration}]
  \mbox{}\hfill\\[-10pt]\begin{Verbatim}[fontsize=\small]
element population
{
   att.global.attributes,
   att.datable.attributes,
   att.editLike.attributes,
   att.naming.attributes,
   att.typed.attributes,
   (
      precision*,
      model.headLike*,
      (
         ( model.pLike+ | model.labelLike+ ),
         ( model.noteLike | model.biblLike )*
      )?,
      population*
   )
}
\end{Verbatim}

\end{reflist}  \index{postscript=<postscript>|oddindex}
\begin{reflist}
\item[]\begin{specHead}{TEI.postscript}{<postscript> }contains a postscript, e.g. to a letter. [\xref{http://www.tei-c.org/release/doc/tei-p5-doc/en/html/DS.html\#DSDTB}{4.2. Elements Common to All Divisions}]\end{specHead} 
    \item[{Module}]
  textstructure
    \item[{Attributes}]
  Attributes att.global (\textit{@xml:id}, \textit{@n}, \textit{@xml:lang}, \textit{@xml:base}, \textit{@xml:space})  (att.global.rendition (\textit{@rend}, \textit{@style}, \textit{@rendition})) (att.global.linking (\textit{@corresp}, \textit{@synch}, \textit{@sameAs}, \textit{@copyOf}, \textit{@next}, \textit{@prev}, \textit{@exclude}, \textit{@select})) (att.global.analytic (\textit{@ana})) (att.global.facs (\textit{@facs})) (att.global.change (\textit{@change})) (att.global.responsibility (\textit{@cert}, \textit{@resp})) (att.global.source (\textit{@source}))
    \item[{Member of}]
  model.divBottomPart
    \item[{Contained by}]
  
    \item[core: ]
   lg list\par 
    \item[figures: ]
   figure table\par 
    \item[textstructure: ]
   back body div front group postscript
    \item[{May contain}]
  
    \item[analysis: ]
   interp interpGrp span spanGrp\par 
    \item[core: ]
   bibl biblStruct cb cit desc gap gb head index l label lb lg list listBibl milestone note p pb q quote said sp stage\par 
    \item[figures: ]
   figure notatedMusic table\par 
    \item[header: ]
   biblFull\par 
    \item[linking: ]
   ab alt altGrp anchor join joinGrp link linkGrp timeline\par 
    \item[msdescription: ]
   msDesc\par 
    \item[namesdates: ]
   listEvent listNym listOrg listPerson listPlace\par 
    \item[textcrit: ]
   app listApp listWit witDetail\par 
    \item[textstructure: ]
   closer floatingText opener postscript signed trailer\par 
    \item[transcr: ]
   addSpan damageSpan delSpan fw listTranspose metamark space substJoin
    \item[{Example}]
  \leavevmode\bgroup\exampleFont \begin{shaded}\noindent\mbox{}{<\textbf{div}\hspace*{6pt}{type}="{letter}">}\mbox{}\newline 
\hspace*{6pt}{<\textbf{opener}>}\mbox{}\newline 
\hspace*{6pt}\hspace*{6pt}{<\textbf{dateline}>}\mbox{}\newline 
\hspace*{6pt}\hspace*{6pt}\hspace*{6pt}{<\textbf{placeName}>}Rimaone{</\textbf{placeName}>}\mbox{}\newline 
\hspace*{6pt}\hspace*{6pt}\hspace*{6pt}{<\textbf{date}\hspace*{6pt}{when}="{2006-11-21}">}21 Nov 06{</\textbf{date}>}\mbox{}\newline 
\hspace*{6pt}\hspace*{6pt}{</\textbf{dateline}>}\mbox{}\newline 
\hspace*{6pt}\hspace*{6pt}{<\textbf{salute}>}Dear Susan,{</\textbf{salute}>}\mbox{}\newline 
\hspace*{6pt}{</\textbf{opener}>}\mbox{}\newline 
\hspace*{6pt}{<\textbf{p}>}Thank you very much for the assistance splitting those\mbox{}\newline 
\hspace*{6pt}\hspace*{6pt} logs. I'm sorry about the misunderstanding as to the size of\mbox{}\newline 
\hspace*{6pt}\hspace*{6pt} the task. I really was not asking for help, only to borrow the\mbox{}\newline 
\hspace*{6pt}\hspace*{6pt} axe. Hope you had fun in any case.{</\textbf{p}>}\mbox{}\newline 
\hspace*{6pt}{<\textbf{closer}>}\mbox{}\newline 
\hspace*{6pt}\hspace*{6pt}{<\textbf{salute}>}Sincerely yours,{</\textbf{salute}>}\mbox{}\newline 
\hspace*{6pt}\hspace*{6pt}{<\textbf{signed}>}Seymour{</\textbf{signed}>}\mbox{}\newline 
\hspace*{6pt}{</\textbf{closer}>}\mbox{}\newline 
\hspace*{6pt}{<\textbf{postscript}>}\mbox{}\newline 
\hspace*{6pt}\hspace*{6pt}{<\textbf{label}>}P.S.{</\textbf{label}>}\mbox{}\newline 
\hspace*{6pt}\hspace*{6pt}{<\textbf{p}>}The collision occured on {<\textbf{date}\hspace*{6pt}{when}="{2001-07-06}">}06 Jul 01{</\textbf{date}>}.{</\textbf{p}>}\mbox{}\newline 
\hspace*{6pt}{</\textbf{postscript}>}\mbox{}\newline 
{</\textbf{div}>}\end{shaded}\egroup 


    \item[{Content model}]
  \mbox{}\hfill\\[-10pt]\begin{Verbatim}[fontsize=\small]
<content>
 <sequence>
  <alternate maxOccurs="unbounded"
   minOccurs="0">
   <classRef key="model.global"/>
   <classRef key="model.divTopPart"/>
  </alternate>
  <classRef key="model.common"/>
  <alternate maxOccurs="unbounded"
   minOccurs="0">
   <classRef key="model.global"/>
   <classRef key="model.common"/>
  </alternate>
  <sequence maxOccurs="unbounded"
   minOccurs="0">
   <classRef key="model.divBottomPart"/>
   <classRef key="model.global"
    maxOccurs="unbounded" minOccurs="0"/>
  </sequence>
 </sequence>
</content>
    
\end{Verbatim}

    \item[{Schema Declaration}]
  \mbox{}\hfill\\[-10pt]\begin{Verbatim}[fontsize=\small]
element postscript
{
   att.global.attributes,
   (
      ( model.global | model.divTopPart )*,
      model.common,
      ( model.global | model.common )*,
      ( model.divBottomPart, model.global* )*
   )
}
\end{Verbatim}

\end{reflist}  \index{prefixDef=<prefixDef>|oddindex}\index{ident=@ident!<prefixDef>|oddindex}
\begin{reflist}
\item[]\begin{specHead}{TEI.prefixDef}{<prefixDef> }(prefix definition) defines a prefixing scheme used in \textsf{data.pointer} values, showing how abbreviated URIs using the scheme may be expanded into full URIs. [\xref{http://www.tei-c.org/release/doc/tei-p5-doc/en/html/SA.html\#SAPU}{16.2.3. Using Abbreviated Pointers}]\end{specHead} 
    \item[{Module}]
  header
    \item[{Attributes}]
  Attributes att.global (\textit{@xml:id}, \textit{@n}, \textit{@xml:lang}, \textit{@xml:base}, \textit{@xml:space})  (att.global.rendition (\textit{@rend}, \textit{@style}, \textit{@rendition})) (att.global.linking (\textit{@corresp}, \textit{@synch}, \textit{@sameAs}, \textit{@copyOf}, \textit{@next}, \textit{@prev}, \textit{@exclude}, \textit{@select})) (att.global.analytic (\textit{@ana})) (att.global.facs (\textit{@facs})) (att.global.change (\textit{@change})) (att.global.responsibility (\textit{@cert}, \textit{@resp})) (att.global.source (\textit{@source})) att.patternReplacement (\textit{@matchPattern}, \textit{@replacementPattern}) \hfil\\[-10pt]\begin{sansreflist}
    \item[@ident]
  supplies a name which functions as the prefix for an abbreviated pointing scheme such as a private URI scheme. The prefix constitutes the text preceding the first colon.
\begin{reflist}
    \item[{Status}]
  Required
    \item[{Datatype}]
  teidata.prefix
\end{reflist}  
\end{sansreflist}  
    \item[{Contained by}]
  
    \item[header: ]
   listPrefixDef
    \item[{May contain}]
  
    \item[core: ]
   p\par 
    \item[linking: ]
   ab
    \item[{Note}]
  \par
The abbreviated pointer may be dereferenced to produce either an absolute or a relative URI reference. In the latter case it is combined with the value of {\itshape xml:base} in force at the place where the pointing attribute occurs to form an absolute URI in the usual manner as prescribed by \xref{http://www.w3.org/TR/xmlbase/}{XML Base}.
    \item[{Note}]
  \par
So that the value of this attribute may be mapped directly to a URI prefix, the value is limited to \textsf{teidata.prefix}.
    \item[{Example}]
  \leavevmode\bgroup\exampleFont \begin{shaded}\noindent\mbox{}{<\textbf{prefixDef}\hspace*{6pt}{ident}="{ref}"\mbox{}\newline 
\hspace*{6pt}{matchPattern}="{([a-z]+)}"\mbox{}\newline 
\hspace*{6pt}{replacementPattern}="{../../references/references.xml\#\$1}">}\mbox{}\newline 
\hspace*{6pt}{<\textbf{p}>} In the context of this project, private URIs with\mbox{}\newline 
\hspace*{6pt}\hspace*{6pt} the prefix "ref" point to {<\textbf{gi}>}div{</\textbf{gi}>} elements in\mbox{}\newline 
\hspace*{6pt}\hspace*{6pt} the project's global references.xml file.\mbox{}\newline 
\hspace*{6pt}{</\textbf{p}>}\mbox{}\newline 
{</\textbf{prefixDef}>}\end{shaded}\egroup 


    \item[{Content model}]
  \mbox{}\hfill\\[-10pt]\begin{Verbatim}[fontsize=\small]
<content>
 <classRef key="model.pLike"
  maxOccurs="unbounded" minOccurs="0"/>
</content>
    
\end{Verbatim}

    \item[{Schema Declaration}]
  \mbox{}\hfill\\[-10pt]\begin{Verbatim}[fontsize=\small]
element prefixDef
{
   att.global.attributes,
   att.patternReplacement.attributes,
   attribute ident { text },
   model.pLike*
}
\end{Verbatim}

\end{reflist}  \index{principal=<principal>|oddindex}
\begin{reflist}
\item[]\begin{specHead}{TEI.principal}{<principal> }(principal researcher) supplies the name of the principal researcher responsible for the creation of an electronic text. [\xref{http://www.tei-c.org/release/doc/tei-p5-doc/en/html/HD.html\#HD21}{2.2.1. The Title Statement}]\end{specHead} 
    \item[{Module}]
  header
    \item[{Attributes}]
  Attributes att.global (\textit{@xml:id}, \textit{@n}, \textit{@xml:lang}, \textit{@xml:base}, \textit{@xml:space})  (att.global.rendition (\textit{@rend}, \textit{@style}, \textit{@rendition})) (att.global.linking (\textit{@corresp}, \textit{@synch}, \textit{@sameAs}, \textit{@copyOf}, \textit{@next}, \textit{@prev}, \textit{@exclude}, \textit{@select})) (att.global.analytic (\textit{@ana})) (att.global.facs (\textit{@facs})) (att.global.change (\textit{@change})) (att.global.responsibility (\textit{@cert}, \textit{@resp})) (att.global.source (\textit{@source})) att.canonical (\textit{@key}, \textit{@ref}) 
    \item[{Member of}]
  model.respLike
    \item[{Contained by}]
  
    \item[core: ]
   bibl\par 
    \item[header: ]
   editionStmt titleStmt\par 
    \item[msdescription: ]
   msItem
    \item[{May contain}]
  
    \item[analysis: ]
   interp interpGrp span spanGrp\par 
    \item[core: ]
   abbr address cb choice date distinct email emph expan foreign gap gb gloss hi index lb measure measureGrp mentioned milestone name note num pb ptr ref rs soCalled term time title\par 
    \item[figures: ]
   figure notatedMusic\par 
    \item[header: ]
   idno\par 
    \item[linking: ]
   alt altGrp anchor join joinGrp link linkGrp timeline\par 
    \item[msdescription: ]
   catchwords depth dim dimensions height heraldry locus locusGrp material objectType origDate origPlace secFol signatures stamp watermark width\par 
    \item[namesdates: ]
   addName affiliation bloc climate country district forename genName geo geogFeat geogName location nameLink offset orgName persName placeName population region roleName settlement state surname terrain trait\par 
    \item[textcrit: ]
   app witDetail\par 
    \item[transcr: ]
   addSpan am damageSpan delSpan ex fw listTranspose metamark space subst substJoin\par character data
    \item[{Example}]
  \leavevmode\bgroup\exampleFont \begin{shaded}\noindent\mbox{}{<\textbf{principal}\hspace*{6pt}{ref}="{http://viaf.org/viaf/105517912}">}Gary Taylor{</\textbf{principal}>}\end{shaded}\egroup 


    \item[{Content model}]
  \mbox{}\hfill\\[-10pt]\begin{Verbatim}[fontsize=\small]
<content>
 <macroRef key="macro.phraseSeq.limited"/>
</content>
    
\end{Verbatim}

    \item[{Schema Declaration}]
  \mbox{}\hfill\\[-10pt]\begin{Verbatim}[fontsize=\small]
element principal
{
   att.global.attributes,
   att.canonical.attributes,
   macro.phraseSeq.limited}
\end{Verbatim}

\end{reflist}  \index{profileDesc=<profileDesc>|oddindex}
\begin{reflist}
\item[]\begin{specHead}{TEI.profileDesc}{<profileDesc> }(text-profile description) provides a detailed description of non-bibliographic aspects of a text, specifically the languages and sublanguages used, the situation in which it was produced, the participants and their setting. [\xref{http://www.tei-c.org/release/doc/tei-p5-doc/en/html/HD.html\#HD4}{2.4. The Profile Description} \xref{http://www.tei-c.org/release/doc/tei-p5-doc/en/html/HD.html\#HD11}{2.1.1. The TEI Header and Its Components}]\end{specHead} 
    \item[{Module}]
  header
    \item[{Attributes}]
  Attributes att.global (\textit{@xml:id}, \textit{@n}, \textit{@xml:lang}, \textit{@xml:base}, \textit{@xml:space})  (att.global.rendition (\textit{@rend}, \textit{@style}, \textit{@rendition})) (att.global.linking (\textit{@corresp}, \textit{@synch}, \textit{@sameAs}, \textit{@copyOf}, \textit{@next}, \textit{@prev}, \textit{@exclude}, \textit{@select})) (att.global.analytic (\textit{@ana})) (att.global.facs (\textit{@facs})) (att.global.change (\textit{@change})) (att.global.responsibility (\textit{@cert}, \textit{@resp})) (att.global.source (\textit{@source}))
    \item[{Member of}]
  model.teiHeaderPart
    \item[{Contained by}]
  
    \item[header: ]
   biblFull teiHeader
    \item[{May contain}]
  
    \item[header: ]
   abstract calendarDesc correspDesc creation langUsage textClass\par 
    \item[transcr: ]
   handNotes listTranspose
    \item[{Note}]
  \par
Although the content model permits it, it is rarely meaningful to supply multiple occurrences for any of the child elements of <profileDesc> unless these are documenting multiple texts.\par
In earlier versions of these Guidelines, it was required that the <creation> element appear first.
    \item[{Example}]
  \leavevmode\bgroup\exampleFont \begin{shaded}\noindent\mbox{}{<\textbf{profileDesc}>}\mbox{}\newline 
\hspace*{6pt}{<\textbf{langUsage}>}\mbox{}\newline 
\hspace*{6pt}\hspace*{6pt}{<\textbf{language}\hspace*{6pt}{ident}="{fr}">}French{</\textbf{language}>}\mbox{}\newline 
\hspace*{6pt}{</\textbf{langUsage}>}\mbox{}\newline 
\hspace*{6pt}{<\textbf{textDesc}\hspace*{6pt}{n}="{novel}">}\mbox{}\newline 
\hspace*{6pt}\hspace*{6pt}{<\textbf{channel}\hspace*{6pt}{mode}="{w}">}print; part issues{</\textbf{channel}>}\mbox{}\newline 
\hspace*{6pt}\hspace*{6pt}{<\textbf{constitution}\hspace*{6pt}{type}="{single}"/>}\mbox{}\newline 
\hspace*{6pt}\hspace*{6pt}{<\textbf{derivation}\hspace*{6pt}{type}="{original}"/>}\mbox{}\newline 
\hspace*{6pt}\hspace*{6pt}{<\textbf{domain}\hspace*{6pt}{type}="{art}"/>}\mbox{}\newline 
\hspace*{6pt}\hspace*{6pt}{<\textbf{factuality}\hspace*{6pt}{type}="{fiction}"/>}\mbox{}\newline 
\hspace*{6pt}\hspace*{6pt}{<\textbf{interaction}\hspace*{6pt}{type}="{none}"/>}\mbox{}\newline 
\hspace*{6pt}\hspace*{6pt}{<\textbf{preparedness}\hspace*{6pt}{type}="{prepared}"/>}\mbox{}\newline 
\hspace*{6pt}\hspace*{6pt}{<\textbf{purpose}\hspace*{6pt}{degree}="{high}"\hspace*{6pt}{type}="{entertain}"/>}\mbox{}\newline 
\hspace*{6pt}\hspace*{6pt}{<\textbf{purpose}\hspace*{6pt}{degree}="{medium}"\hspace*{6pt}{type}="{inform}"/>}\mbox{}\newline 
\hspace*{6pt}{</\textbf{textDesc}>}\mbox{}\newline 
\hspace*{6pt}{<\textbf{settingDesc}>}\mbox{}\newline 
\hspace*{6pt}\hspace*{6pt}{<\textbf{setting}>}\mbox{}\newline 
\hspace*{6pt}\hspace*{6pt}\hspace*{6pt}{<\textbf{name}>}Paris, France{</\textbf{name}>}\mbox{}\newline 
\hspace*{6pt}\hspace*{6pt}\hspace*{6pt}{<\textbf{time}>}Late 19th century{</\textbf{time}>}\mbox{}\newline 
\hspace*{6pt}\hspace*{6pt}{</\textbf{setting}>}\mbox{}\newline 
\hspace*{6pt}{</\textbf{settingDesc}>}\mbox{}\newline 
{</\textbf{profileDesc}>}\end{shaded}\egroup 


    \item[{Content model}]
  \mbox{}\hfill\\[-10pt]\begin{Verbatim}[fontsize=\small]
<content>
 <classRef key="model.profileDescPart"
  maxOccurs="unbounded" minOccurs="0"/>
</content>
    
\end{Verbatim}

    \item[{Schema Declaration}]
  \mbox{}\hfill\\[-10pt]\begin{Verbatim}[fontsize=\small]
element profileDesc { att.global.attributes, model.profileDescPart* }
\end{Verbatim}

\end{reflist}  \index{projectDesc=<projectDesc>|oddindex}
\begin{reflist}
\item[]\begin{specHead}{TEI.projectDesc}{<projectDesc> }(project description) describes in detail the aim or purpose for which an electronic file was encoded, together with any other relevant information concerning the process by which it was assembled or collected. [\xref{http://www.tei-c.org/release/doc/tei-p5-doc/en/html/HD.html\#HD51}{2.3.1. The Project Description} \xref{http://www.tei-c.org/release/doc/tei-p5-doc/en/html/HD.html\#HD5}{2.3. The Encoding Description} \xref{http://www.tei-c.org/release/doc/tei-p5-doc/en/html/CC.html\#CCAS2}{15.3.2. Declarable Elements}]\end{specHead} 
    \item[{Module}]
  header
    \item[{Attributes}]
  Attributes att.global (\textit{@xml:id}, \textit{@n}, \textit{@xml:lang}, \textit{@xml:base}, \textit{@xml:space})  (att.global.rendition (\textit{@rend}, \textit{@style}, \textit{@rendition})) (att.global.linking (\textit{@corresp}, \textit{@synch}, \textit{@sameAs}, \textit{@copyOf}, \textit{@next}, \textit{@prev}, \textit{@exclude}, \textit{@select})) (att.global.analytic (\textit{@ana})) (att.global.facs (\textit{@facs})) (att.global.change (\textit{@change})) (att.global.responsibility (\textit{@cert}, \textit{@resp})) (att.global.source (\textit{@source})) att.declarable (\textit{@default}) 
    \item[{Member of}]
  model.encodingDescPart
    \item[{Contained by}]
  
    \item[header: ]
   encodingDesc
    \item[{May contain}]
  
    \item[core: ]
   p\par 
    \item[linking: ]
   ab
    \item[{Example}]
  \leavevmode\bgroup\exampleFont \begin{shaded}\noindent\mbox{}{<\textbf{projectDesc}>}\mbox{}\newline 
\hspace*{6pt}{<\textbf{p}>}Texts collected for use in the Claremont Shakespeare Clinic, June 1990{</\textbf{p}>}\mbox{}\newline 
{</\textbf{projectDesc}>}\end{shaded}\egroup 


    \item[{Content model}]
  \mbox{}\hfill\\[-10pt]\begin{Verbatim}[fontsize=\small]
<content>
 <classRef key="model.pLike"
  maxOccurs="unbounded" minOccurs="1"/>
</content>
    
\end{Verbatim}

    \item[{Schema Declaration}]
  \mbox{}\hfill\\[-10pt]\begin{Verbatim}[fontsize=\small]
element projectDesc
{
   att.global.attributes,
   att.declarable.attributes,
   model.pLike+
}
\end{Verbatim}

\end{reflist}  \index{provenance=<provenance>|oddindex}
\begin{reflist}
\item[]\begin{specHead}{TEI.provenance}{<provenance> }contains any descriptive or other information concerning a single identifiable episode during the history of a manuscript or manuscript part, after its creation but before its acquisition. [\xref{http://www.tei-c.org/release/doc/tei-p5-doc/en/html/MS.html\#mshy}{10.8. History}]\end{specHead} 
    \item[{Module}]
  msdescription
    \item[{Attributes}]
  Attributes att.global (\textit{@xml:id}, \textit{@n}, \textit{@xml:lang}, \textit{@xml:base}, \textit{@xml:space})  (att.global.rendition (\textit{@rend}, \textit{@style}, \textit{@rendition})) (att.global.linking (\textit{@corresp}, \textit{@synch}, \textit{@sameAs}, \textit{@copyOf}, \textit{@next}, \textit{@prev}, \textit{@exclude}, \textit{@select})) (att.global.analytic (\textit{@ana})) (att.global.facs (\textit{@facs})) (att.global.change (\textit{@change})) (att.global.responsibility (\textit{@cert}, \textit{@resp})) (att.global.source (\textit{@source})) att.datable (\textit{@calendar}, \textit{@period})  (att.datable.w3c (\textit{@when}, \textit{@notBefore}, \textit{@notAfter}, \textit{@from}, \textit{@to})) (att.datable.iso (\textit{@when-iso}, \textit{@notBefore-iso}, \textit{@notAfter-iso}, \textit{@from-iso}, \textit{@to-iso})) (att.datable.custom (\textit{@when-custom}, \textit{@notBefore-custom}, \textit{@notAfter-custom}, \textit{@from-custom}, \textit{@to-custom}, \textit{@datingPoint}, \textit{@datingMethod})) att.typed (\textit{@type}, \textit{@subtype}) 
    \item[{Contained by}]
  
    \item[msdescription: ]
   history
    \item[{May contain}]
  
    \item[analysis: ]
   c cl interp interpGrp m pc phr s span spanGrp w\par 
    \item[core: ]
   abbr add address bibl biblStruct cb choice cit corr date del desc distinct email emph expan foreign gap gb gloss graphic hi index l label lb lg list listBibl measure measureGrp media mentioned milestone name note num orig p pb ptr q quote ref reg rs said sic soCalled sp stage term time title unclear\par 
    \item[figures: ]
   figure formula notatedMusic table\par 
    \item[gaiji: ]
   g\par 
    \item[header: ]
   biblFull idno\par 
    \item[linking: ]
   ab alt altGrp anchor join joinGrp link linkGrp seg timeline\par 
    \item[msdescription: ]
   catchwords depth dim dimensions height heraldry locus locusGrp material msDesc objectType origDate origPlace secFol signatures stamp watermark width\par 
    \item[namesdates: ]
   addName affiliation bloc climate country district forename genName geo geogFeat geogName listEvent listNym listOrg listPerson listPlace location nameLink offset orgName persName placeName population region roleName settlement state surname terrain trait\par 
    \item[textcrit: ]
   app listApp listWit witDetail\par 
    \item[textstructure: ]
   floatingText\par 
    \item[transcr: ]
   addSpan am damage damageSpan delSpan ex fw handShift listTranspose metamark mod redo restore retrace secl space subst substJoin supplied surplus undo\par character data
    \item[{Example}]
  \leavevmode\bgroup\exampleFont \begin{shaded}\noindent\mbox{}{<\textbf{provenance}>}Listed as the property of Lawrence Sterne in 1788.{</\textbf{provenance}>}\mbox{}\newline 
{<\textbf{provenance}>}Sold at Sothebys in 1899.{</\textbf{provenance}>}\end{shaded}\egroup 


    \item[{Content model}]
  \mbox{}\hfill\\[-10pt]\begin{Verbatim}[fontsize=\small]
<content>
 <macroRef key="macro.specialPara"/>
</content>
    
\end{Verbatim}

    \item[{Schema Declaration}]
  \mbox{}\hfill\\[-10pt]\begin{Verbatim}[fontsize=\small]
element provenance
{
   att.global.attributes,
   att.datable.attributes,
   att.typed.attributes,
   macro.specialPara}
\end{Verbatim}

\end{reflist}  \index{ptr=<ptr>|oddindex}
\begin{reflist}
\item[]\begin{specHead}{TEI.ptr}{<ptr> }(pointer) defines a pointer to another location. [\xref{http://www.tei-c.org/release/doc/tei-p5-doc/en/html/CO.html\#COXR}{3.6. Simple Links and Cross-References} \xref{http://www.tei-c.org/release/doc/tei-p5-doc/en/html/SA.html\#SAPT}{16.1. Links}]\end{specHead} 
    \item[{Module}]
  core
    \item[{Attributes}]
  Attributes att.global (\textit{@xml:id}, \textit{@n}, \textit{@xml:lang}, \textit{@xml:base}, \textit{@xml:space})  (att.global.rendition (\textit{@rend}, \textit{@style}, \textit{@rendition})) (att.global.linking (\textit{@corresp}, \textit{@synch}, \textit{@sameAs}, \textit{@copyOf}, \textit{@next}, \textit{@prev}, \textit{@exclude}, \textit{@select})) (att.global.analytic (\textit{@ana})) (att.global.facs (\textit{@facs})) (att.global.change (\textit{@change})) (att.global.responsibility (\textit{@cert}, \textit{@resp})) (att.global.source (\textit{@source})) att.pointing (\textit{@targetLang}, \textit{@target}, \textit{@evaluate}) att.internetMedia (\textit{@mimeType}) att.typed (\textit{@type}, \textit{@subtype}) att.declaring (\textit{@decls}) att.cReferencing (\textit{@cRef}) 
    \item[{Member of}]
  model.ptrLike 
    \item[{Contained by}]
  
    \item[analysis: ]
   cl phr s span\par 
    \item[core: ]
   abbr add addrLine analytic author bibl biblScope biblStruct cit citedRange corr date del desc distinct editor email emph expan foreign gloss head headItem headLabel hi item l label measure meeting mentioned monogr name note num orig p pubPlace publisher q quote ref reg relatedItem resp rs said series sic soCalled speaker stage street term textLang time title unclear\par 
    \item[figures: ]
   cell figDesc notatedMusic\par 
    \item[header: ]
   application authority catDesc change classCode correspContext creation distributor edition extent funder geoDecl handNote language licence principal rendition scriptNote sponsor tagUsage typeNote\par 
    \item[linking: ]
   ab altGrp joinGrp linkGrp seg\par 
    \item[msdescription: ]
   accMat acquisition additions catchwords collation colophon condition custEvent decoNote explicit filiation finalRubric foliation heraldry incipit layout material musicNotation objectType origDate origPlace origin provenance rubric secFol signatures source stamp summary support surrogates watermark\par 
    \item[namesdates: ]
   addName affiliation age birth bloc country death district education faith floruit forename genName geogFeat geogName langKnown nameLink nationality occupation offset orgName persName placeName region residence roleName settlement sex socecStatus surname\par 
    \item[textcrit: ]
   lem rdg wit witDetail witness\par 
    \item[textstructure: ]
   byline closer dateline docAuthor docDate docEdition docImprint imprimatur opener salute signed titlePart trailer\par 
    \item[transcr: ]
   damage fw metamark mod restore retrace secl supplied surplus transpose
    \item[{May contain}]
  Empty element
    \item[{Example}]
  \leavevmode\bgroup\exampleFont \begin{shaded}\noindent\mbox{}{<\textbf{ptr}\hspace*{6pt}{target}="{\#p143 \#p144}"/>}\mbox{}\newline 
{<\textbf{ptr}\hspace*{6pt}{target}="{http://www.tei-c.org}"/>}\mbox{}\newline 
{<\textbf{ptr}\hspace*{6pt}{cRef}="{1.3.4}"/>}\end{shaded}\egroup 


    \item[{Schematron}]
   <s:report test="@target and @cRef">Only one of the  attributes @target and @cRef may be supplied on <s:name/>.</s:report>
    \item[{Content model}]
  \fbox{\ttfamily <content>\newline
</content>\newline
    } 
    \item[{Schema Declaration}]
  \mbox{}\hfill\\[-10pt]\begin{Verbatim}[fontsize=\small]
element ptr
{
   att.global.attributes,
   att.pointing.attributes,
   att.internetMedia.attributes,
   att.typed.attributes,
   att.declaring.attributes,
   att.cReferencing.attributes,
   empty
}
\end{Verbatim}

\end{reflist}  \index{pubPlace=<pubPlace>|oddindex}
\begin{reflist}
\item[]\begin{specHead}{TEI.pubPlace}{<pubPlace> }(publication place) contains the name of the place where a bibliographic item was published. [\xref{http://www.tei-c.org/release/doc/tei-p5-doc/en/html/CO.html\#COBICOI}{3.11.2.4. Imprint, Size of a Document, and Reprint Information}]\end{specHead} 
    \item[{Module}]
  core
    \item[{Attributes}]
  Attributes att.global (\textit{@xml:id}, \textit{@n}, \textit{@xml:lang}, \textit{@xml:base}, \textit{@xml:space})  (att.global.rendition (\textit{@rend}, \textit{@style}, \textit{@rendition})) (att.global.linking (\textit{@corresp}, \textit{@synch}, \textit{@sameAs}, \textit{@copyOf}, \textit{@next}, \textit{@prev}, \textit{@exclude}, \textit{@select})) (att.global.analytic (\textit{@ana})) (att.global.facs (\textit{@facs})) (att.global.change (\textit{@change})) (att.global.responsibility (\textit{@cert}, \textit{@resp})) (att.global.source (\textit{@source})) att.naming (\textit{@role}, \textit{@nymRef})  (att.canonical (\textit{@key}, \textit{@ref}))
    \item[{Member of}]
  model.imprintPart model.publicationStmtPart.detail
    \item[{Contained by}]
  
    \item[core: ]
   bibl imprint\par 
    \item[header: ]
   publicationStmt\par 
    \item[textstructure: ]
   docImprint
    \item[{May contain}]
  
    \item[analysis: ]
   c cl interp interpGrp m pc phr s span spanGrp w\par 
    \item[core: ]
   abbr add address cb choice corr date del distinct email emph expan foreign gap gb gloss graphic hi index lb measure measureGrp media mentioned milestone name note num orig pb ptr ref reg rs sic soCalled term time title unclear\par 
    \item[figures: ]
   figure formula notatedMusic\par 
    \item[gaiji: ]
   g\par 
    \item[header: ]
   idno\par 
    \item[linking: ]
   alt altGrp anchor join joinGrp link linkGrp seg timeline\par 
    \item[msdescription: ]
   catchwords depth dim dimensions height heraldry locus locusGrp material objectType origDate origPlace secFol signatures stamp watermark width\par 
    \item[namesdates: ]
   addName affiliation bloc climate country district forename genName geo geogFeat geogName location nameLink offset orgName persName placeName population region roleName settlement state surname terrain trait\par 
    \item[textcrit: ]
   app witDetail\par 
    \item[transcr: ]
   addSpan am damage damageSpan delSpan ex fw handShift listTranspose metamark mod redo restore retrace secl space subst substJoin supplied surplus undo\par character data
    \item[{Example}]
  \leavevmode\bgroup\exampleFont \begin{shaded}\noindent\mbox{}{<\textbf{publicationStmt}>}\mbox{}\newline 
\hspace*{6pt}{<\textbf{publisher}>}Oxford University Press{</\textbf{publisher}>}\mbox{}\newline 
\hspace*{6pt}{<\textbf{pubPlace}>}Oxford{</\textbf{pubPlace}>}\mbox{}\newline 
\hspace*{6pt}{<\textbf{date}>}1989{</\textbf{date}>}\mbox{}\newline 
{</\textbf{publicationStmt}>}\end{shaded}\egroup 


    \item[{Content model}]
  \mbox{}\hfill\\[-10pt]\begin{Verbatim}[fontsize=\small]
<content>
 <macroRef key="macro.phraseSeq"/>
</content>
    
\end{Verbatim}

    \item[{Schema Declaration}]
  \mbox{}\hfill\\[-10pt]\begin{Verbatim}[fontsize=\small]
element pubPlace
{
   att.global.attributes,
   att.naming.attributes,
   macro.phraseSeq}
\end{Verbatim}

\end{reflist}  \index{publicationStmt=<publicationStmt>|oddindex}
\begin{reflist}
\item[]\begin{specHead}{TEI.publicationStmt}{<publicationStmt> }(publication statement) groups information concerning the publication or distribution of an electronic or other text. [\xref{http://www.tei-c.org/release/doc/tei-p5-doc/en/html/HD.html\#HD24}{2.2.4. Publication, Distribution, Licensing, etc.} \xref{http://www.tei-c.org/release/doc/tei-p5-doc/en/html/HD.html\#HD2}{2.2. The File Description}]\end{specHead} 
    \item[{Module}]
  header
    \item[{Attributes}]
  Attributes att.global (\textit{@xml:id}, \textit{@n}, \textit{@xml:lang}, \textit{@xml:base}, \textit{@xml:space})  (att.global.rendition (\textit{@rend}, \textit{@style}, \textit{@rendition})) (att.global.linking (\textit{@corresp}, \textit{@synch}, \textit{@sameAs}, \textit{@copyOf}, \textit{@next}, \textit{@prev}, \textit{@exclude}, \textit{@select})) (att.global.analytic (\textit{@ana})) (att.global.facs (\textit{@facs})) (att.global.change (\textit{@change})) (att.global.responsibility (\textit{@cert}, \textit{@resp})) (att.global.source (\textit{@source}))
    \item[{Contained by}]
  
    \item[header: ]
   biblFull fileDesc
    \item[{May contain}]
  
    \item[core: ]
   address date p pubPlace publisher\par 
    \item[header: ]
   authority availability distributor idno\par 
    \item[linking: ]
   ab
    \item[{Note}]
  \par
Where a publication statement contains several members of the \textsf{model.publicationStmtPart.agency} or \textsf{model.publicationStmtPart.detail} classes rather than one or more paragraphs or anonymous blocks, care should be taken to ensure that the repeated elements are presented in a meaningful order. It is a conformance requirement that elements supplying information about publication place, address, identifier, availability, and date be given following the name of the publisher, distributor, or authority concerned, and preferably in that order.
    \item[{Example}]
  \leavevmode\bgroup\exampleFont \begin{shaded}\noindent\mbox{}{<\textbf{publicationStmt}>}\mbox{}\newline 
\hspace*{6pt}{<\textbf{publisher}>}C. Muquardt {</\textbf{publisher}>}\mbox{}\newline 
\hspace*{6pt}{<\textbf{pubPlace}>}Bruxelles \& Leipzig{</\textbf{pubPlace}>}\mbox{}\newline 
\hspace*{6pt}{<\textbf{date}\hspace*{6pt}{when}="{1846}"/>}\mbox{}\newline 
{</\textbf{publicationStmt}>}\end{shaded}\egroup 


    \item[{Example}]
  \leavevmode\bgroup\exampleFont \begin{shaded}\noindent\mbox{}{<\textbf{publicationStmt}>}\mbox{}\newline 
\hspace*{6pt}{<\textbf{publisher}>}Chadwyck Healey{</\textbf{publisher}>}\mbox{}\newline 
\hspace*{6pt}{<\textbf{pubPlace}>}Cambridge{</\textbf{pubPlace}>}\mbox{}\newline 
\hspace*{6pt}{<\textbf{availability}>}\mbox{}\newline 
\hspace*{6pt}\hspace*{6pt}{<\textbf{p}>}Available under licence only{</\textbf{p}>}\mbox{}\newline 
\hspace*{6pt}{</\textbf{availability}>}\mbox{}\newline 
\hspace*{6pt}{<\textbf{date}\hspace*{6pt}{when}="{1992}">}1992{</\textbf{date}>}\mbox{}\newline 
{</\textbf{publicationStmt}>}\end{shaded}\egroup 


    \item[{Content model}]
  \mbox{}\hfill\\[-10pt]\begin{Verbatim}[fontsize=\small]
<content>
 <alternate>
  <sequence maxOccurs="unbounded"
   minOccurs="1">
   <classRef key="model.publicationStmtPart.agency"/>
   <classRef key="model.publicationStmtPart.detail"
    maxOccurs="unbounded" minOccurs="0"/>
  </sequence>
  <classRef key="model.pLike"
   maxOccurs="unbounded" minOccurs="1"/>
 </alternate>
</content>
    
\end{Verbatim}

    \item[{Schema Declaration}]
  \mbox{}\hfill\\[-10pt]\begin{Verbatim}[fontsize=\small]
element publicationStmt
{
   att.global.attributes,
   (
      ( model.publicationStmtPart.agency, model.publicationStmtPart.detail* )+
    | model.pLike+
   )
}
\end{Verbatim}

\end{reflist}  \index{publisher=<publisher>|oddindex}
\begin{reflist}
\item[]\begin{specHead}{TEI.publisher}{<publisher> }provides the name of the organization responsible for the publication or distribution of a bibliographic item. [\xref{http://www.tei-c.org/release/doc/tei-p5-doc/en/html/CO.html\#COBICOI}{3.11.2.4. Imprint, Size of a Document, and Reprint Information} \xref{http://www.tei-c.org/release/doc/tei-p5-doc/en/html/HD.html\#HD24}{2.2.4. Publication, Distribution, Licensing, etc.}]\end{specHead} 
    \item[{Module}]
  core
    \item[{Attributes}]
  Attributes att.global (\textit{@xml:id}, \textit{@n}, \textit{@xml:lang}, \textit{@xml:base}, \textit{@xml:space})  (att.global.rendition (\textit{@rend}, \textit{@style}, \textit{@rendition})) (att.global.linking (\textit{@corresp}, \textit{@synch}, \textit{@sameAs}, \textit{@copyOf}, \textit{@next}, \textit{@prev}, \textit{@exclude}, \textit{@select})) (att.global.analytic (\textit{@ana})) (att.global.facs (\textit{@facs})) (att.global.change (\textit{@change})) (att.global.responsibility (\textit{@cert}, \textit{@resp})) (att.global.source (\textit{@source}))
    \item[{Member of}]
  model.imprintPart model.publicationStmtPart.agency
    \item[{Contained by}]
  
    \item[core: ]
   bibl imprint\par 
    \item[header: ]
   publicationStmt\par 
    \item[textstructure: ]
   docImprint
    \item[{May contain}]
  
    \item[analysis: ]
   c cl interp interpGrp m pc phr s span spanGrp w\par 
    \item[core: ]
   abbr add address cb choice corr date del distinct email emph expan foreign gap gb gloss graphic hi index lb measure measureGrp media mentioned milestone name note num orig pb ptr ref reg rs sic soCalled term time title unclear\par 
    \item[figures: ]
   figure formula notatedMusic\par 
    \item[gaiji: ]
   g\par 
    \item[header: ]
   idno\par 
    \item[linking: ]
   alt altGrp anchor join joinGrp link linkGrp seg timeline\par 
    \item[msdescription: ]
   catchwords depth dim dimensions height heraldry locus locusGrp material objectType origDate origPlace secFol signatures stamp watermark width\par 
    \item[namesdates: ]
   addName affiliation bloc climate country district forename genName geo geogFeat geogName location nameLink offset orgName persName placeName population region roleName settlement state surname terrain trait\par 
    \item[textcrit: ]
   app witDetail\par 
    \item[transcr: ]
   addSpan am damage damageSpan delSpan ex fw handShift listTranspose metamark mod redo restore retrace secl space subst substJoin supplied surplus undo\par character data
    \item[{Note}]
  \par
Use the full form of the name by which a company is usually referred to, rather than any abbreviation of it which may appear on a title page
    \item[{Example}]
  \leavevmode\bgroup\exampleFont \begin{shaded}\noindent\mbox{}{<\textbf{imprint}>}\mbox{}\newline 
\hspace*{6pt}{<\textbf{pubPlace}>}Oxford{</\textbf{pubPlace}>}\mbox{}\newline 
\hspace*{6pt}{<\textbf{publisher}>}Clarendon Press{</\textbf{publisher}>}\mbox{}\newline 
\hspace*{6pt}{<\textbf{date}>}1987{</\textbf{date}>}\mbox{}\newline 
{</\textbf{imprint}>}\end{shaded}\egroup 


    \item[{Content model}]
  \mbox{}\hfill\\[-10pt]\begin{Verbatim}[fontsize=\small]
<content>
 <macroRef key="macro.phraseSeq"/>
</content>
    
\end{Verbatim}

    \item[{Schema Declaration}]
  \mbox{}\hfill\\[-10pt]\begin{Verbatim}[fontsize=\small]
element publisher { att.global.attributes, macro.phraseSeq }
\end{Verbatim}

\end{reflist}  \index{punctuation=<punctuation>|oddindex}\index{marks=@marks!<punctuation>|oddindex}\index{placement=@placement!<punctuation>|oddindex}
\begin{reflist}
\item[]\begin{specHead}{TEI.punctuation}{<punctuation> }specifies editorial practice adopted with respect to punctuation marks in the original. [\xref{http://www.tei-c.org/release/doc/tei-p5-doc/en/html/HD.html\#HD53}{2.3.3. The Editorial Practices Declaration} \xref{http://www.tei-c.org/release/doc/tei-p5-doc/en/html/CO.html\#COPU}{3.2. Treatment of Punctuation}]\end{specHead} 
    \item[{Module}]
  header
    \item[{Attributes}]
  Attributes att.declarable (\textit{@default}) att.global (\textit{@xml:id}, \textit{@n}, \textit{@xml:lang}, \textit{@xml:base}, \textit{@xml:space})  (att.global.rendition (\textit{@rend}, \textit{@style}, \textit{@rendition})) (att.global.linking (\textit{@corresp}, \textit{@synch}, \textit{@sameAs}, \textit{@copyOf}, \textit{@next}, \textit{@prev}, \textit{@exclude}, \textit{@select})) (att.global.analytic (\textit{@ana})) (att.global.facs (\textit{@facs})) (att.global.change (\textit{@change})) (att.global.responsibility (\textit{@cert}, \textit{@resp})) (att.global.source (\textit{@source})) \hfil\\[-10pt]\begin{sansreflist}
    \item[@marks]
  indicates whether or not punctation marks have been retained as content within the text.
\begin{reflist}
    \item[{Status}]
  Optional
    \item[{Datatype}]
  teidata.enumerated
    \item[{Legal values are:}]
  \begin{description}

\item[{none}]no punctuation marks have been retained
\item[{some}]some punctuation marks have been retained
\item[{all}]all punctuation marks have been retained
\end{description} 
\end{reflist}  
    \item[@placement]
  indicates whether punctation marks have been captured inside or outside of an adjacent element.
\begin{reflist}
    \item[{Status}]
  Optional
    \item[{Datatype}]
  teidata.enumerated
    \item[{Legal values are:}]
  \begin{description}

\item[{internal}]punctuation marks are captured inside adjacent elements
\item[{external}]punctuation marks are captured outside adjacent elements
\end{description} 
\end{reflist}  
\end{sansreflist}  
    \item[{Member of}]
  model.editorialDeclPart
    \item[{Contained by}]
  
    \item[header: ]
   editorialDecl
    \item[{May contain}]
  
    \item[core: ]
   p\par 
    \item[linking: ]
   ab
    \item[{Example}]
  \leavevmode\bgroup\exampleFont \begin{shaded}\noindent\mbox{}{<\textbf{punctuation}\hspace*{6pt}{marks}="{all}"\mbox{}\newline 
\hspace*{6pt}{placement}="{internal}">}\mbox{}\newline 
\hspace*{6pt}{<\textbf{p}>}All punctuation marks in the source text have been retained and represented using the\mbox{}\newline 
\hspace*{6pt}\hspace*{6pt} appropriate Unicode code point. In cases where a punctuation mark and nearby markup convey\mbox{}\newline 
\hspace*{6pt}\hspace*{6pt} the same information (for example, a sentence ends with a question mark and is also tagged\mbox{}\newline 
\hspace*{6pt}\hspace*{6pt} as {<\textbf{gi}>}s{</\textbf{gi}>}) the punctuation mark is captured as content within the element.\mbox{}\newline 
\hspace*{6pt}{</\textbf{p}>}\mbox{}\newline 
{</\textbf{punctuation}>}\end{shaded}\egroup 


    \item[{Content model}]
  \mbox{}\hfill\\[-10pt]\begin{Verbatim}[fontsize=\small]
<content>
 <classRef key="model.pLike"
  maxOccurs="unbounded" minOccurs="0"/>
</content>
    
\end{Verbatim}

    \item[{Schema Declaration}]
  \mbox{}\hfill\\[-10pt]\begin{Verbatim}[fontsize=\small]
element punctuation
{
   att.declarable.attributes,
   att.global.attributes,
   attribute marks { "none" | "some" | "all" }?,
   attribute placement { "internal" | "external" }?,
   model.pLike*
}
\end{Verbatim}

\end{reflist}  \index{q=<q>|oddindex}\index{type=@type!<q>|oddindex}
\begin{reflist}
\item[]\begin{specHead}{TEI.q}{<q> }(quoted) contains material which is distinguished from the surrounding text using quotation marks or a similar method, for any one of a variety of reasons including, but not limited to: direct speech or thought, technical terms or jargon, authorial distance, quotations from elsewhere, and passages that are mentioned but not used. [\xref{http://www.tei-c.org/release/doc/tei-p5-doc/en/html/CO.html\#COHQQ}{3.3.3. Quotation}]\end{specHead} 
    \item[{Module}]
  core
    \item[{Attributes}]
  Attributes att.global (\textit{@xml:id}, \textit{@n}, \textit{@xml:lang}, \textit{@xml:base}, \textit{@xml:space})  (att.global.rendition (\textit{@rend}, \textit{@style}, \textit{@rendition})) (att.global.linking (\textit{@corresp}, \textit{@synch}, \textit{@sameAs}, \textit{@copyOf}, \textit{@next}, \textit{@prev}, \textit{@exclude}, \textit{@select})) (att.global.analytic (\textit{@ana})) (att.global.facs (\textit{@facs})) (att.global.change (\textit{@change})) (att.global.responsibility (\textit{@cert}, \textit{@resp})) (att.global.source (\textit{@source})) att.ascribed (\textit{@who}) \hfil\\[-10pt]\begin{sansreflist}
    \item[@type]
  may be used to indicate whether the offset passage is spoken or thought, or to characterize it more finely.
\begin{reflist}
    \item[{Status}]
  Optional
    \item[{Datatype}]
  teidata.enumerated
    \item[{Suggested values include:}]
  \begin{description}

\item[{spoken}]representation of speech
\item[{thought}]representation of thought, e.g. internal monologue
\item[{written}]quotation from a written source
\item[{soCalled}]authorial distance
\item[{foreign}]
\item[{distinct}]linguistically distinct
\item[{term}]technical term
\item[{emph}]rhetorically emphasized
\item[{mentioned}]refering to itself, not its normal referent
\end{description} 
\end{reflist}  
\end{sansreflist}  
    \item[{Member of}]
  model.qLike
    \item[{Contained by}]
  
    \item[core: ]
   add cit corr del desc emph head hi item l meeting note orig p q quote ref reg said sic sp stage title unclear\par 
    \item[figures: ]
   cell figDesc figure\par 
    \item[header: ]
   change handNote licence rendition scriptNote tagUsage typeNote\par 
    \item[linking: ]
   ab seg\par 
    \item[msdescription: ]
   accMat acquisition additions collation condition custEvent decoNote filiation foliation layout musicNotation origin provenance signatures source summary support surrogates\par 
    \item[namesdates: ]
   occupation\par 
    \item[textcrit: ]
   lem rdg witness\par 
    \item[textstructure: ]
   argument body div docEdition epigraph imprimatur postscript salute signed titlePart trailer\par 
    \item[transcr: ]
   damage metamark mod restore retrace secl supplied surplus
    \item[{May contain}]
  
    \item[analysis: ]
   c cl interp interpGrp m pc phr s span spanGrp w\par 
    \item[core: ]
   abbr add address bibl biblStruct cb choice cit corr date del desc distinct email emph expan foreign gap gb gloss graphic hi index l label lb lg list listBibl measure measureGrp media mentioned milestone name note num orig p pb ptr q quote ref reg rs said sic soCalled sp stage term time title unclear\par 
    \item[figures: ]
   figure formula notatedMusic table\par 
    \item[gaiji: ]
   g\par 
    \item[header: ]
   biblFull idno\par 
    \item[linking: ]
   ab alt altGrp anchor join joinGrp link linkGrp seg timeline\par 
    \item[msdescription: ]
   catchwords depth dim dimensions height heraldry locus locusGrp material msDesc objectType origDate origPlace secFol signatures stamp watermark width\par 
    \item[namesdates: ]
   addName affiliation bloc climate country district forename genName geo geogFeat geogName listEvent listNym listOrg listPerson listPlace location nameLink offset orgName persName placeName population region roleName settlement state surname terrain trait\par 
    \item[textcrit: ]
   app listApp listWit witDetail\par 
    \item[textstructure: ]
   floatingText\par 
    \item[transcr: ]
   addSpan am damage damageSpan delSpan ex fw handShift listTranspose metamark mod redo restore retrace secl space subst substJoin supplied surplus undo\par character data
    \item[{Note}]
  \par
May be used to indicate that a passage is distinguished from the surrounding text for reasons concerning which no claim is made. When used in this manner, <q> may be thought of as syntactic sugar for <hi> with a value of {\itshape rend} that indicates the use of such mechanisms as quotation marks.
    \item[{Example}]
  \leavevmode\bgroup\exampleFont \begin{shaded}\noindent\mbox{}It is spelled {<\textbf{q}>}Tübingen{</\textbf{q}>} — to enter the\mbox{}\newline 
 letter {<\textbf{q}>}u{</\textbf{q}>} with an umlaut hold down the {<\textbf{q}>}option{</\textbf{q}>} key and press \mbox{}\newline 
{<\textbf{q}>}0 0 f c{</\textbf{q}>}\end{shaded}\egroup 


    \item[{Content model}]
  \mbox{}\hfill\\[-10pt]\begin{Verbatim}[fontsize=\small]
<content>
 <macroRef key="macro.specialPara"/>
</content>
    
\end{Verbatim}

    \item[{Schema Declaration}]
  \mbox{}\hfill\\[-10pt]\begin{Verbatim}[fontsize=\small]
element q
{
   att.global.attributes,
   att.ascribed.attributes,
   attribute type
   {
      "spoken"
    | "thought"
    | "written"
    | "soCalled"
    | "foreign"
    | "distinct"
    | "term"
    | "emph"
    | "mentioned"
   }?,
   macro.specialPara}
\end{Verbatim}

\end{reflist}  \index{quotation=<quotation>|oddindex}\index{marks=@marks!<quotation>|oddindex}
\begin{reflist}
\item[]\begin{specHead}{TEI.quotation}{<quotation> }specifies editorial practice adopted with respect to quotation marks in the original. [\xref{http://www.tei-c.org/release/doc/tei-p5-doc/en/html/HD.html\#HD53}{2.3.3. The Editorial Practices Declaration} \xref{http://www.tei-c.org/release/doc/tei-p5-doc/en/html/CC.html\#CCAS2}{15.3.2. Declarable Elements}]\end{specHead} 
    \item[{Module}]
  header
    \item[{Attributes}]
  Attributes att.global (\textit{@xml:id}, \textit{@n}, \textit{@xml:lang}, \textit{@xml:base}, \textit{@xml:space})  (att.global.rendition (\textit{@rend}, \textit{@style}, \textit{@rendition})) (att.global.linking (\textit{@corresp}, \textit{@synch}, \textit{@sameAs}, \textit{@copyOf}, \textit{@next}, \textit{@prev}, \textit{@exclude}, \textit{@select})) (att.global.analytic (\textit{@ana})) (att.global.facs (\textit{@facs})) (att.global.change (\textit{@change})) (att.global.responsibility (\textit{@cert}, \textit{@resp})) (att.global.source (\textit{@source})) att.declarable (\textit{@default}) \hfil\\[-10pt]\begin{sansreflist}
    \item[@marks]
  (quotation marks) indicates whether or not quotation marks have been retained as content within the text.
\begin{reflist}
    \item[{Status}]
  Optional
    \item[{Datatype}]
  teidata.enumerated
    \item[{Legal values are:}]
  \begin{description}

\item[{none}]no quotation marks have been retained
\item[{some}]some quotation marks have been retained
\item[{all}]all quotation marks have been retained
\end{description} 
\end{reflist}  
\end{sansreflist}  
    \item[{Member of}]
  model.editorialDeclPart
    \item[{Contained by}]
  
    \item[header: ]
   editorialDecl
    \item[{May contain}]
  
    \item[core: ]
   p\par 
    \item[linking: ]
   ab
    \item[{Example}]
  \leavevmode\bgroup\exampleFont \begin{shaded}\noindent\mbox{}{<\textbf{quotation}\hspace*{6pt}{marks}="{none}">}\mbox{}\newline 
\hspace*{6pt}{<\textbf{p}>}No quotation marks have been retained. Instead, the {<\textbf{att}>}rend{</\textbf{att}>} attribute on the\mbox{}\newline 
\hspace*{6pt}{<\textbf{gi}>}q{</\textbf{gi}>} element is used to specify what kinds of quotation mark was used, according\mbox{}\newline 
\hspace*{6pt}\hspace*{6pt} to the following list: {<\textbf{list}\hspace*{6pt}{type}="{gloss}">}\mbox{}\newline 
\hspace*{6pt}\hspace*{6pt}\hspace*{6pt}{<\textbf{label}>}dq{</\textbf{label}>}\mbox{}\newline 
\hspace*{6pt}\hspace*{6pt}\hspace*{6pt}{<\textbf{item}>}double quotes, open and close{</\textbf{item}>}\mbox{}\newline 
\hspace*{6pt}\hspace*{6pt}\hspace*{6pt}{<\textbf{label}>}sq{</\textbf{label}>}\mbox{}\newline 
\hspace*{6pt}\hspace*{6pt}\hspace*{6pt}{<\textbf{item}>}single quotes, open and close{</\textbf{item}>}\mbox{}\newline 
\hspace*{6pt}\hspace*{6pt}\hspace*{6pt}{<\textbf{label}>}dash{</\textbf{label}>}\mbox{}\newline 
\hspace*{6pt}\hspace*{6pt}\hspace*{6pt}{<\textbf{item}>}long dash open, no close{</\textbf{item}>}\mbox{}\newline 
\hspace*{6pt}\hspace*{6pt}\hspace*{6pt}{<\textbf{label}>}dg{</\textbf{label}>}\mbox{}\newline 
\hspace*{6pt}\hspace*{6pt}\hspace*{6pt}{<\textbf{item}>}double guillemets, open and close{</\textbf{item}>}\mbox{}\newline 
\hspace*{6pt}\hspace*{6pt}{</\textbf{list}>}\mbox{}\newline 
\hspace*{6pt}{</\textbf{p}>}\mbox{}\newline 
{</\textbf{quotation}>}\end{shaded}\egroup 


    \item[{Example}]
  \leavevmode\bgroup\exampleFont \begin{shaded}\noindent\mbox{}{<\textbf{quotation}\hspace*{6pt}{marks}="{all}">}\mbox{}\newline 
\hspace*{6pt}{<\textbf{p}>}All quotation marks are retained in the text and are represented by appropriate Unicode\mbox{}\newline 
\hspace*{6pt}\hspace*{6pt} characters.{</\textbf{p}>}\mbox{}\newline 
{</\textbf{quotation}>}\end{shaded}\egroup 


    \item[{Schematron}]
   <s:report test="not(@marks) and not (tei:p)">On <s:name/>, either the @marks attribute should be used, or a paragraph of description provided</s:report>
    \item[{Content model}]
  \mbox{}\hfill\\[-10pt]\begin{Verbatim}[fontsize=\small]
<content>
 <classRef key="model.pLike"
  maxOccurs="unbounded" minOccurs="0"/>
</content>
    
\end{Verbatim}

    \item[{Schema Declaration}]
  \mbox{}\hfill\\[-10pt]\begin{Verbatim}[fontsize=\small]
element quotation
{
   att.global.attributes,
   att.declarable.attributes,
   attribute marks { "none" | "some" | "all" }?,
   model.pLike*
}
\end{Verbatim}

\end{reflist}  \index{quote=<quote>|oddindex}
\begin{reflist}
\item[]\begin{specHead}{TEI.quote}{<quote> }(quotation) contains a phrase or passage attributed by the narrator or author to some agency external to the text. [\xref{http://www.tei-c.org/release/doc/tei-p5-doc/en/html/CO.html\#COHQQ}{3.3.3. Quotation} \xref{http://www.tei-c.org/release/doc/tei-p5-doc/en/html/DS.html\#DSGRP}{4.3.1. Grouped Texts}]\end{specHead} 
    \item[{Module}]
  core
    \item[{Attributes}]
  Attributes att.global (\textit{@xml:id}, \textit{@n}, \textit{@xml:lang}, \textit{@xml:base}, \textit{@xml:space})  (att.global.rendition (\textit{@rend}, \textit{@style}, \textit{@rendition})) (att.global.linking (\textit{@corresp}, \textit{@synch}, \textit{@sameAs}, \textit{@copyOf}, \textit{@next}, \textit{@prev}, \textit{@exclude}, \textit{@select})) (att.global.analytic (\textit{@ana})) (att.global.facs (\textit{@facs})) (att.global.change (\textit{@change})) (att.global.responsibility (\textit{@cert}, \textit{@resp})) (att.global.source (\textit{@source})) att.typed (\textit{@type}, \textit{@subtype}) att.msExcerpt (\textit{@defective}) 
    \item[{Member of}]
  model.quoteLike
    \item[{Contained by}]
  
    \item[core: ]
   add cit corr del desc emph head hi item l meeting note orig p q quote ref reg said sic sp stage title unclear\par 
    \item[figures: ]
   cell figDesc figure\par 
    \item[header: ]
   change handNote licence rendition scriptNote tagUsage typeNote\par 
    \item[linking: ]
   ab seg\par 
    \item[msdescription: ]
   accMat acquisition additions collation condition custEvent decoNote filiation foliation layout msItem musicNotation origin provenance signatures source summary support surrogates\par 
    \item[namesdates: ]
   occupation\par 
    \item[textcrit: ]
   lem rdg witness\par 
    \item[textstructure: ]
   argument body div docEdition epigraph imprimatur postscript salute signed titlePart trailer\par 
    \item[transcr: ]
   damage metamark mod restore retrace secl supplied surplus
    \item[{May contain}]
  
    \item[analysis: ]
   c cl interp interpGrp m pc phr s span spanGrp w\par 
    \item[core: ]
   abbr add address bibl biblStruct cb choice cit corr date del desc distinct email emph expan foreign gap gb gloss graphic hi index l label lb lg list listBibl measure measureGrp media mentioned milestone name note num orig p pb ptr q quote ref reg rs said sic soCalled sp stage term time title unclear\par 
    \item[figures: ]
   figure formula notatedMusic table\par 
    \item[gaiji: ]
   g\par 
    \item[header: ]
   biblFull idno\par 
    \item[linking: ]
   ab alt altGrp anchor join joinGrp link linkGrp seg timeline\par 
    \item[msdescription: ]
   catchwords depth dim dimensions height heraldry locus locusGrp material msDesc objectType origDate origPlace secFol signatures stamp watermark width\par 
    \item[namesdates: ]
   addName affiliation bloc climate country district forename genName geo geogFeat geogName listEvent listNym listOrg listPerson listPlace location nameLink offset orgName persName placeName population region roleName settlement state surname terrain trait\par 
    \item[textcrit: ]
   app listApp listWit witDetail\par 
    \item[textstructure: ]
   floatingText\par 
    \item[transcr: ]
   addSpan am damage damageSpan delSpan ex fw handShift listTranspose metamark mod redo restore retrace secl space subst substJoin supplied surplus undo\par character data
    \item[{Note}]
  \par
If a bibliographic citation is supplied for the source of a quotation, the two may be grouped using the <cit> element.
    \item[{Example}]
  \leavevmode\bgroup\exampleFont \begin{shaded}\noindent\mbox{}Lexicography has shown little sign of being affected by the\mbox{}\newline 
 work of followers of J.R. Firth, probably best summarized in his\mbox{}\newline 
 slogan, {<\textbf{quote}>}You shall know a word by the company it\mbox{}\newline 
 keeps{</\textbf{quote}>}\mbox{}\newline 
{<\textbf{ref}>}(Firth, 1957){</\textbf{ref}>}\end{shaded}\egroup 


    \item[{Content model}]
  \mbox{}\hfill\\[-10pt]\begin{Verbatim}[fontsize=\small]
<content>
 <macroRef key="macro.specialPara"/>
</content>
    
\end{Verbatim}

    \item[{Schema Declaration}]
  \mbox{}\hfill\\[-10pt]\begin{Verbatim}[fontsize=\small]
element quote
{
   att.global.attributes,
   att.typed.attributes,
   att.msExcerpt.attributes,
   macro.specialPara}
\end{Verbatim}

\end{reflist}  \index{rdg=<rdg>|oddindex}
\begin{reflist}
\item[]\begin{specHead}{TEI.rdg}{<rdg> }(reading) contains a single reading within a textual variation. [\xref{http://www.tei-c.org/release/doc/tei-p5-doc/en/html/TC.html\#TCAPLL}{12.1. The Apparatus Entry, Readings, and Witnesses}]\end{specHead} 
    \item[{Module}]
  textcrit
    \item[{Attributes}]
  Attributes att.global (\textit{@xml:id}, \textit{@n}, \textit{@xml:lang}, \textit{@xml:base}, \textit{@xml:space})  (att.global.rendition (\textit{@rend}, \textit{@style}, \textit{@rendition})) (att.global.linking (\textit{@corresp}, \textit{@synch}, \textit{@sameAs}, \textit{@copyOf}, \textit{@next}, \textit{@prev}, \textit{@exclude}, \textit{@select})) (att.global.analytic (\textit{@ana})) (att.global.facs (\textit{@facs})) (att.global.change (\textit{@change})) (att.global.responsibility (\textit{@cert}, \textit{@resp})) (att.global.source (\textit{@source})) att.textCritical (\textit{@type}, \textit{@cause}, \textit{@varSeq}, \textit{@require})  (att.written (\textit{@hand})) att.witnessed (\textit{@wit}) 
    \item[{Member of}]
  model.rdgLike
    \item[{Contained by}]
  
    \item[textcrit: ]
   app rdgGrp
    \item[{May contain}]
  
    \item[analysis: ]
   c cl interp interpGrp m pc phr s span spanGrp w\par 
    \item[core: ]
   abbr add address bibl biblStruct cb choice cit corr date del desc distinct email emph expan foreign gap gb gloss graphic hi index l label lb lg list listBibl measure measureGrp media mentioned milestone name note num orig p pb ptr q quote ref reg rs said sic soCalled sp stage term time title unclear\par 
    \item[figures: ]
   figure formula notatedMusic table\par 
    \item[gaiji: ]
   g\par 
    \item[header: ]
   biblFull idno\par 
    \item[linking: ]
   ab alt altGrp anchor join joinGrp link linkGrp seg timeline\par 
    \item[msdescription: ]
   catchwords depth dim dimensions height heraldry locus locusGrp material msDesc objectType origDate origPlace secFol signatures stamp watermark width\par 
    \item[namesdates: ]
   addName affiliation bloc climate country district forename genName geo geogFeat geogName listEvent listNym listOrg listPerson listPlace location nameLink offset orgName persName placeName population region roleName settlement state surname terrain trait\par 
    \item[textcrit: ]
   app lacunaEnd lacunaStart listApp listWit wit witDetail witEnd witStart\par 
    \item[textstructure: ]
   div floatingText\par 
    \item[transcr: ]
   addSpan am damage damageSpan delSpan ex fw handShift listTranspose metamark mod redo restore retrace secl space subst substJoin supplied surplus undo\par character data
    \item[{Example}]
  \leavevmode\bgroup\exampleFont \begin{shaded}\noindent\mbox{}{<\textbf{rdg}\hspace*{6pt}{wit}="{\#Ra2}">}Eryment{</\textbf{rdg}>}\end{shaded}\egroup 


    \item[{Content model}]
  \mbox{}\hfill\\[-10pt]\begin{Verbatim}[fontsize=\small]
<content>
 <alternate maxOccurs="unbounded"
  minOccurs="0">
  <textNode/>
  <classRef key="model.divLike"/>
  <classRef key="model.divPart"/>
  <classRef key="model.gLike"/>
  <classRef key="model.phrase"/>
  <classRef key="model.inter"/>
  <classRef key="model.global"/>
  <classRef key="model.rdgPart"/>
 </alternate>
</content>
    
\end{Verbatim}

    \item[{Schema Declaration}]
  \mbox{}\hfill\\[-10pt]\begin{Verbatim}[fontsize=\small]
element rdg
{
   att.global.attributes,
   att.textCritical.attributes,
   att.witnessed.attributes,
   (
      text
    | model.divLike    | model.divPart    | model.gLike    | model.phrase    | model.inter    | model.global    | model.rdgPart   )*
}
\end{Verbatim}

\end{reflist}  \index{rdgGrp=<rdgGrp>|oddindex}
\begin{reflist}
\item[]\begin{specHead}{TEI.rdgGrp}{<rdgGrp> }(reading group) within a textual variation, groups two or more readings perceived to have a genetic relationship or other affinity. [\xref{http://www.tei-c.org/release/doc/tei-p5-doc/en/html/TC.html\#TCAPLL}{12.1. The Apparatus Entry, Readings, and Witnesses}]\end{specHead} 
    \item[{Module}]
  textcrit
    \item[{Attributes}]
  Attributes att.global (\textit{@xml:id}, \textit{@n}, \textit{@xml:lang}, \textit{@xml:base}, \textit{@xml:space})  (att.global.rendition (\textit{@rend}, \textit{@style}, \textit{@rendition})) (att.global.linking (\textit{@corresp}, \textit{@synch}, \textit{@sameAs}, \textit{@copyOf}, \textit{@next}, \textit{@prev}, \textit{@exclude}, \textit{@select})) (att.global.analytic (\textit{@ana})) (att.global.facs (\textit{@facs})) (att.global.change (\textit{@change})) (att.global.responsibility (\textit{@cert}, \textit{@resp})) (att.global.source (\textit{@source})) att.textCritical (\textit{@type}, \textit{@cause}, \textit{@varSeq}, \textit{@require})  (att.written (\textit{@hand}))
    \item[{Contained by}]
  
    \item[textcrit: ]
   app rdgGrp
    \item[{May contain}]
  
    \item[core: ]
   note\par 
    \item[textcrit: ]
   lem rdg rdgGrp wit witDetail
    \item[{Note}]
  \par
May contain readings and nested reading groups.\par
Note that only one <lem> element may appear within a single apparatus entry, whether it appears outside a <rdgGrp> element or within it.
    \item[{Example}]
  \leavevmode\bgroup\exampleFont \begin{shaded}\noindent\mbox{}{<\textbf{app}>}\mbox{}\newline 
\hspace*{6pt}{<\textbf{lem}\hspace*{6pt}{wit}="{\#El \#Ra2}">}though{</\textbf{lem}>}\mbox{}\newline 
\hspace*{6pt}{<\textbf{rdgGrp}\hspace*{6pt}{type}="{orthographic}">}\mbox{}\newline 
\hspace*{6pt}\hspace*{6pt}{<\textbf{rdg}\hspace*{6pt}{wit}="{\#Hg}">}thogh{</\textbf{rdg}>}\mbox{}\newline 
\hspace*{6pt}\hspace*{6pt}{<\textbf{rdg}\hspace*{6pt}{wit}="{\#La}">}thouhe{</\textbf{rdg}>}\mbox{}\newline 
\hspace*{6pt}{</\textbf{rdgGrp}>}\mbox{}\newline 
{</\textbf{app}>}\end{shaded}\egroup 


    \item[{Schematron}]
   <sch:assert test="count(tei:lem) < 2">Only one <lem> element may appear within a <rdgGrp></sch:assert>
    \item[{Content model}]
  \mbox{}\hfill\\[-10pt]\begin{Verbatim}[fontsize=\small]
<content>
 <alternate maxOccurs="unbounded"
  minOccurs="1">
  <sequence>
   <elementRef key="rdgGrp"/>
   <elementRef key="wit" minOccurs="0"/>
  </sequence>
  <sequence maxOccurs="unbounded"
   minOccurs="0">
   <sequence minOccurs="0">
    <elementRef key="lem"/>
    <elementRef key="wit" minOccurs="0"/>
   </sequence>
   <sequence>
    <classRef key="model.rdgLike"/>
    <elementRef key="wit" minOccurs="0"/>
   </sequence>
  </sequence>
  <classRef key="model.noteLike"
   maxOccurs="unbounded" minOccurs="0"/>
 </alternate>
</content>
    
\end{Verbatim}

    \item[{Schema Declaration}]
  \mbox{}\hfill\\[-10pt]\begin{Verbatim}[fontsize=\small]
element rdgGrp
{
   att.global.attributes,
   att.textCritical.attributes,
   (
      ( rdgGrp, wit? )
    | ( ( lem, wit? )?, ( model.rdgLike, wit? ) )*
    | model.noteLike*
   )+
}
\end{Verbatim}

\end{reflist}  \index{recordHist=<recordHist>|oddindex}
\begin{reflist}
\item[]\begin{specHead}{TEI.recordHist}{<recordHist> }(recorded history) provides information about the source and revision status of the parent manuscript description itself. [\xref{http://www.tei-c.org/release/doc/tei-p5-doc/en/html/MS.html\#msadad}{10.9.1. Administrative Information}]\end{specHead} 
    \item[{Module}]
  msdescription
    \item[{Attributes}]
  Attributes att.global (\textit{@xml:id}, \textit{@n}, \textit{@xml:lang}, \textit{@xml:base}, \textit{@xml:space})  (att.global.rendition (\textit{@rend}, \textit{@style}, \textit{@rendition})) (att.global.linking (\textit{@corresp}, \textit{@synch}, \textit{@sameAs}, \textit{@copyOf}, \textit{@next}, \textit{@prev}, \textit{@exclude}, \textit{@select})) (att.global.analytic (\textit{@ana})) (att.global.facs (\textit{@facs})) (att.global.change (\textit{@change})) (att.global.responsibility (\textit{@cert}, \textit{@resp})) (att.global.source (\textit{@source}))
    \item[{Contained by}]
  
    \item[msdescription: ]
   adminInfo
    \item[{May contain}]
  
    \item[core: ]
   p\par 
    \item[header: ]
   change\par 
    \item[linking: ]
   ab\par 
    \item[msdescription: ]
   source
    \item[{Example}]
  \leavevmode\bgroup\exampleFont \begin{shaded}\noindent\mbox{}{<\textbf{recordHist}>}\mbox{}\newline 
\hspace*{6pt}{<\textbf{source}>}\mbox{}\newline 
\hspace*{6pt}\hspace*{6pt}{<\textbf{p}>}Derived from {<\textbf{ref}\hspace*{6pt}{target}="{\#IMEV}">}IMEV 123{</\textbf{ref}>} with additional research\mbox{}\newline 
\hspace*{6pt}\hspace*{6pt}\hspace*{6pt}\hspace*{6pt} by P.M.W.Robinson{</\textbf{p}>}\mbox{}\newline 
\hspace*{6pt}{</\textbf{source}>}\mbox{}\newline 
\hspace*{6pt}{<\textbf{change}\hspace*{6pt}{when}="{1999-06-23}">}\mbox{}\newline 
\hspace*{6pt}\hspace*{6pt}{<\textbf{name}>}LDB{</\textbf{name}>} (editor)\mbox{}\newline 
\hspace*{6pt}\hspace*{6pt} checked examples against DTD version 3.6\mbox{}\newline 
\hspace*{6pt}{</\textbf{change}>}\mbox{}\newline 
{</\textbf{recordHist}>}\end{shaded}\egroup 


    \item[{Content model}]
  \mbox{}\hfill\\[-10pt]\begin{Verbatim}[fontsize=\small]
<content>
 <alternate>
  <classRef key="model.pLike"
   maxOccurs="unbounded" minOccurs="1"/>
  <sequence>
   <elementRef key="source"/>
   <elementRef key="change"
    maxOccurs="unbounded" minOccurs="0"/>
  </sequence>
 </alternate>
</content>
    
\end{Verbatim}

    \item[{Schema Declaration}]
  \mbox{}\hfill\\[-10pt]\begin{Verbatim}[fontsize=\small]
element recordHist
{
   att.global.attributes,
   ( model.pLike+ | ( source, change* ) )
}
\end{Verbatim}

\end{reflist}  \index{redo=<redo>|oddindex}\index{target=@target!<redo>|oddindex}
\begin{reflist}
\item[]\begin{specHead}{TEI.redo}{<redo> }indicates one or more cancelled interventions in a document which have subsequently been marked as reaffirmed or repeated. [\xref{http://www.tei-c.org/release/doc/tei-p5-doc/en/html/PH.html\#undo}{11.3.4.4. Confirmation, Cancellation, and Reinstatement of Modifications}]\end{specHead} 
    \item[{Module}]
  transcr
    \item[{Attributes}]
  Attributes att.global (\textit{@xml:id}, \textit{@n}, \textit{@xml:lang}, \textit{@xml:base}, \textit{@xml:space})  (att.global.rendition (\textit{@rend}, \textit{@style}, \textit{@rendition})) (att.global.linking (\textit{@corresp}, \textit{@synch}, \textit{@sameAs}, \textit{@copyOf}, \textit{@next}, \textit{@prev}, \textit{@exclude}, \textit{@select})) (att.global.analytic (\textit{@ana})) (att.global.facs (\textit{@facs})) (att.global.change (\textit{@change})) (att.global.responsibility (\textit{@cert}, \textit{@resp})) (att.global.source (\textit{@source})) att.spanning (\textit{@spanTo}) att.transcriptional (\textit{@status}, \textit{@cause}, \textit{@seq})  (att.editLike (\textit{@evidence}, \textit{@instant}) (att.dimensions (\textit{@unit}, \textit{@quantity}, \textit{@extent}, \textit{@precision}, \textit{@scope}) (att.ranging (\textit{@atLeast}, \textit{@atMost}, \textit{@min}, \textit{@max}, \textit{@confidence})) ) ) (att.written (\textit{@hand})) \hfil\\[-10pt]\begin{sansreflist}
    \item[@target]
  points to one or more elements representing the interventions which are being reasserted.
\begin{reflist}
    \item[{Status}]
  Optional
    \item[{Datatype}]
  1–∞ occurrences of teidata.pointer separated by whitespace
\end{reflist}  
\end{sansreflist}  
    \item[{Member of}]
  model.linePart model.pPart.transcriptional
    \item[{Contained by}]
  
    \item[analysis: ]
   cl pc phr s w\par 
    \item[core: ]
   abbr add addrLine author bibl biblScope citedRange corr date del distinct editor email emph expan foreign gloss head headItem headLabel hi item l label measure mentioned name note num orig p pubPlace publisher q quote ref reg rs said sic soCalled speaker stage street term textLang time title unclear\par 
    \item[figures: ]
   cell\par 
    \item[header: ]
   change distributor edition extent geoDecl handNote licence scriptNote typeNote\par 
    \item[linking: ]
   ab seg\par 
    \item[msdescription: ]
   accMat acquisition additions catchwords collation colophon condition custEvent decoNote explicit filiation finalRubric foliation heraldry incipit layout material musicNotation objectType origDate origPlace origin provenance rubric secFol signatures source stamp summary support surrogates watermark\par 
    \item[namesdates: ]
   addName affiliation birth bloc country death district education faith floruit forename genName geogFeat geogName nameLink nationality occupation offset orgName persName placeName region residence roleName settlement sex socecStatus surname\par 
    \item[textcrit: ]
   lem rdg wit witDetail\par 
    \item[textstructure: ]
   byline closer dateline docAuthor docDate docEdition docImprint imprimatur opener salute signed titlePart trailer\par 
    \item[transcr: ]
   am damage fw line metamark mod restore retrace secl supplied surplus zone
    \item[{May contain}]
  Empty element
    \item[{Example}]
  \leavevmode\bgroup\exampleFont \begin{shaded}\noindent\mbox{}{<\textbf{line}>}\mbox{}\newline 
\hspace*{6pt}{<\textbf{redo}\hspace*{6pt}{cause}="{fix}"\hspace*{6pt}{hand}="{\#g\textunderscore t}"\mbox{}\newline 
\hspace*{6pt}\hspace*{6pt}{target}="{\#redo-1}"/>}\mbox{}\newline 
\hspace*{6pt}{<\textbf{mod}\hspace*{6pt}{hand}="{\#g\textunderscore bl}"\hspace*{6pt}{rend}="{strikethrough}"\mbox{}\newline 
\hspace*{6pt}\hspace*{6pt}{spanTo}="{\#anchor-1}"\hspace*{6pt}{xml:id}="{redo-1}"/>}Ihr hagren, triſten, krummgezog{<\textbf{mod}\hspace*{6pt}{rend}="{strikethrough}">}nen{</\textbf{mod}>}ener Nacken\mbox{}\newline 
{</\textbf{line}>}\mbox{}\newline 
{<\textbf{line}>}Wenn ihr nur piepſet iſt die Welt ſchon matt.{<\textbf{anchor}\hspace*{6pt}{xml:id}="{anchor-1}"/>}\mbox{}\newline 
{</\textbf{line}>}\end{shaded}\egroup 

This encoding represents the following sequence of events: \begin{itemize}
\item "Ihr hagren, tristen, krummgezog nenener Nacken/ Wenn ihr nur piepset ist die Welt schon matt." is written 
\item the redundant letters "nen" in "nenener" are deleted
\item the whole passage is deleted by hand \texttt{g\textunderscore bl} using strikethrough
\item the deletion is reasserted by another hand (identified here as \texttt{g\textunderscore t})
\end{itemize} 
    \item[{Content model}]
  \fbox{\ttfamily <content>\newline
</content>\newline
    } 
    \item[{Schema Declaration}]
  \mbox{}\hfill\\[-10pt]\begin{Verbatim}[fontsize=\small]
element redo
{
   att.global.attributes,
   att.spanning.attributes,
   att.transcriptional.attributes,
   attribute target { list { + } }?,
   empty
}
\end{Verbatim}

\end{reflist}  \index{ref=<ref>|oddindex}
\begin{reflist}
\item[]\begin{specHead}{TEI.ref}{<ref> }(reference) defines a reference to another location, possibly modified by additional text or comment. [\xref{http://www.tei-c.org/release/doc/tei-p5-doc/en/html/CO.html\#COXR}{3.6. Simple Links and Cross-References} \xref{http://www.tei-c.org/release/doc/tei-p5-doc/en/html/SA.html\#SAPT}{16.1. Links}]\end{specHead} 
    \item[{Module}]
  core
    \item[{Attributes}]
  Attributes att.global (\textit{@xml:id}, \textit{@n}, \textit{@xml:lang}, \textit{@xml:base}, \textit{@xml:space})  (att.global.rendition (\textit{@rend}, \textit{@style}, \textit{@rendition})) (att.global.linking (\textit{@corresp}, \textit{@synch}, \textit{@sameAs}, \textit{@copyOf}, \textit{@next}, \textit{@prev}, \textit{@exclude}, \textit{@select})) (att.global.analytic (\textit{@ana})) (att.global.facs (\textit{@facs})) (att.global.change (\textit{@change})) (att.global.responsibility (\textit{@cert}, \textit{@resp})) (att.global.source (\textit{@source})) att.pointing (\textit{@targetLang}, \textit{@target}, \textit{@evaluate}) att.internetMedia (\textit{@mimeType}) att.typed (\textit{@type}, \textit{@subtype}) att.declaring (\textit{@decls}) att.cReferencing (\textit{@cRef}) 
    \item[{Member of}]
  model.ptrLike
    \item[{Contained by}]
  
    \item[analysis: ]
   cl phr s span\par 
    \item[core: ]
   abbr add addrLine analytic author bibl biblScope biblStruct cit citedRange corr date del desc distinct editor email emph expan foreign gloss head headItem headLabel hi item l label measure meeting mentioned monogr name note num orig p pubPlace publisher q quote ref reg relatedItem resp rs said series sic soCalled speaker stage street term textLang time title unclear\par 
    \item[figures: ]
   cell figDesc notatedMusic\par 
    \item[header: ]
   application authority catDesc change classCode correspContext creation distributor edition extent funder geoDecl handNote language licence principal rendition scriptNote sponsor tagUsage typeNote\par 
    \item[linking: ]
   ab seg\par 
    \item[msdescription: ]
   accMat acquisition additions catchwords collation colophon condition custEvent decoNote explicit filiation finalRubric foliation heraldry incipit layout material musicNotation objectType origDate origPlace origin provenance rubric secFol signatures source stamp summary support surrogates watermark\par 
    \item[namesdates: ]
   addName affiliation age birth bloc country death district education faith floruit forename genName geogFeat geogName langKnown nameLink nationality occupation offset orgName persName placeName region residence roleName settlement sex socecStatus surname\par 
    \item[textcrit: ]
   lem rdg wit witDetail witness\par 
    \item[textstructure: ]
   byline closer dateline docAuthor docDate docEdition docImprint imprimatur opener salute signed titlePart trailer\par 
    \item[transcr: ]
   damage fw metamark mod restore retrace secl supplied surplus
    \item[{May contain}]
  
    \item[analysis: ]
   c cl interp interpGrp m pc phr s span spanGrp w\par 
    \item[core: ]
   abbr add address bibl biblStruct cb choice cit corr date del desc distinct email emph expan foreign gap gb gloss graphic hi index l label lb lg list listBibl measure measureGrp media mentioned milestone name note num orig pb ptr q quote ref reg rs said sic soCalled stage term time title unclear\par 
    \item[figures: ]
   figure formula notatedMusic table\par 
    \item[gaiji: ]
   g\par 
    \item[header: ]
   biblFull idno\par 
    \item[linking: ]
   alt altGrp anchor join joinGrp link linkGrp seg timeline\par 
    \item[msdescription: ]
   catchwords depth dim dimensions height heraldry locus locusGrp material msDesc objectType origDate origPlace secFol signatures stamp watermark width\par 
    \item[namesdates: ]
   addName affiliation bloc climate country district forename genName geo geogFeat geogName listEvent listNym listOrg listPerson listPlace location nameLink offset orgName persName placeName population region roleName settlement state surname terrain trait\par 
    \item[textcrit: ]
   app listApp listWit witDetail\par 
    \item[textstructure: ]
   floatingText\par 
    \item[transcr: ]
   addSpan am damage damageSpan delSpan ex fw handShift listTranspose metamark mod redo restore retrace secl space subst substJoin supplied surplus undo\par character data
    \item[{Note}]
  \par
The {\itshape target} and {\itshape cRef} attributes are mutually exclusive.
    \item[{Example}]
  \leavevmode\bgroup\exampleFont \begin{shaded}\noindent\mbox{}See especially {<\textbf{ref}\hspace*{6pt}{target}="{http://www.natcorp.ox.ac.uk/Texts/A02.xml\#s2}">}the second\mbox{}\newline 
 sentence{</\textbf{ref}>}\end{shaded}\egroup 


    \item[{Example}]
  \leavevmode\bgroup\exampleFont \begin{shaded}\noindent\mbox{}See also {<\textbf{ref}\hspace*{6pt}{target}="{\#locution}">}s.v. {<\textbf{term}>}locution{</\textbf{term}>}\mbox{}\newline 
{</\textbf{ref}>}.\end{shaded}\egroup 


    \item[{Schematron}]
   <s:report test="@target and @cRef">Only one of the  attributes @target' and @cRef' may be supplied on <s:name/> </s:report>
    \item[{Content model}]
  \mbox{}\hfill\\[-10pt]\begin{Verbatim}[fontsize=\small]
<content>
 <macroRef key="macro.paraContent"/>
</content>
    
\end{Verbatim}

    \item[{Schema Declaration}]
  \mbox{}\hfill\\[-10pt]\begin{Verbatim}[fontsize=\small]
element ref
{
   att.global.attributes,
   att.pointing.attributes,
   att.internetMedia.attributes,
   att.typed.attributes,
   att.declaring.attributes,
   att.cReferencing.attributes,
   macro.paraContent}
\end{Verbatim}

\end{reflist}  \index{refState=<refState>|oddindex}\index{length=@length!<refState>|oddindex}\index{delim=@delim!<refState>|oddindex}
\begin{reflist}
\item[]\begin{specHead}{TEI.refState}{<refState> }(reference state) specifies one component of a canonical reference defined by the milestone method. [\xref{http://www.tei-c.org/release/doc/tei-p5-doc/en/html/HD.html\#HD54M}{2.3.6.3. Milestone Method} \xref{http://www.tei-c.org/release/doc/tei-p5-doc/en/html/HD.html\#HD54}{2.3.6. The Reference System Declaration}]\end{specHead} 
    \item[{Module}]
  header
    \item[{Attributes}]
  Attributes att.global (\textit{@xml:id}, \textit{@n}, \textit{@xml:lang}, \textit{@xml:base}, \textit{@xml:space})  (att.global.rendition (\textit{@rend}, \textit{@style}, \textit{@rendition})) (att.global.linking (\textit{@corresp}, \textit{@synch}, \textit{@sameAs}, \textit{@copyOf}, \textit{@next}, \textit{@prev}, \textit{@exclude}, \textit{@select})) (att.global.analytic (\textit{@ana})) (att.global.facs (\textit{@facs})) (att.global.change (\textit{@change})) (att.global.responsibility (\textit{@cert}, \textit{@resp})) (att.global.source (\textit{@source})) att.milestoneUnit (\textit{@unit}) att.edition (\textit{@ed}, \textit{@edRef}) \hfil\\[-10pt]\begin{sansreflist}
    \item[@length]
  specifies the fixed length of the reference component.
\begin{reflist}
    \item[{Status}]
  Optional
    \item[{Datatype}]
  teidata.count
    \item[{Note}]
  \par
When constructing a reference, if the reference component found is of numeric type, the length is made up by inserting leading zeros; if it is not, by inserting trailing blanks. In either case, reference components are truncated if necessary at the right hand side. \par
When seeking a reference, the length indicates the number of characters which should be compared. Values longer than this will be regarded as matching, if they start correctly. If no value is provided, the length is unlimited and goes to the next delimiter or to the end of the value. 
\end{reflist}  
    \item[@delim]
  (delimiter) supplies a delimiting string following the reference component.
\begin{reflist}
    \item[{Status}]
  Optional
    \item[{Datatype}]
  teidata.text
\end{reflist}  
\end{sansreflist}  
    \item[{Contained by}]
  
    \item[header: ]
   refsDecl
    \item[{May contain}]
  Empty element
    \item[{Example}]
  \leavevmode\bgroup\exampleFont \begin{shaded}\noindent\mbox{}{<\textbf{refState}\hspace*{6pt}{delim}="{:}"\hspace*{6pt}{unit}="{book}"/>}\mbox{}\newline 
{<\textbf{refState}\hspace*{6pt}{length}="{4}"\hspace*{6pt}{unit}="{line}"/>}\end{shaded}\egroup 


    \item[{Content model}]
  \fbox{\ttfamily <content>\newline
</content>\newline
    } 
    \item[{Schema Declaration}]
  \mbox{}\hfill\\[-10pt]\begin{Verbatim}[fontsize=\small]
element refState
{
   att.global.attributes,
   att.milestoneUnit.attributes,
   att.edition.attributes,
   attribute length { text }?,
   attribute delim { text }?,
   empty
}
\end{Verbatim}

\end{reflist}  \index{refsDecl=<refsDecl>|oddindex}
\begin{reflist}
\item[]\begin{specHead}{TEI.refsDecl}{<refsDecl> }(references declaration) specifies how canonical references are constructed for this text. [\xref{http://www.tei-c.org/release/doc/tei-p5-doc/en/html/HD.html\#HD54M}{2.3.6.3. Milestone Method} \xref{http://www.tei-c.org/release/doc/tei-p5-doc/en/html/HD.html\#HD5}{2.3. The Encoding Description} \xref{http://www.tei-c.org/release/doc/tei-p5-doc/en/html/HD.html\#HD54}{2.3.6. The Reference System Declaration}]\end{specHead} 
    \item[{Module}]
  header
    \item[{Attributes}]
  Attributes att.global (\textit{@xml:id}, \textit{@n}, \textit{@xml:lang}, \textit{@xml:base}, \textit{@xml:space})  (att.global.rendition (\textit{@rend}, \textit{@style}, \textit{@rendition})) (att.global.linking (\textit{@corresp}, \textit{@synch}, \textit{@sameAs}, \textit{@copyOf}, \textit{@next}, \textit{@prev}, \textit{@exclude}, \textit{@select})) (att.global.analytic (\textit{@ana})) (att.global.facs (\textit{@facs})) (att.global.change (\textit{@change})) (att.global.responsibility (\textit{@cert}, \textit{@resp})) (att.global.source (\textit{@source})) att.declarable (\textit{@default}) 
    \item[{Member of}]
  model.encodingDescPart
    \item[{Contained by}]
  
    \item[header: ]
   encodingDesc
    \item[{May contain}]
  
    \item[core: ]
   p\par 
    \item[header: ]
   cRefPattern refState\par 
    \item[linking: ]
   ab
    \item[{Example}]
  \leavevmode\bgroup\exampleFont \begin{shaded}\noindent\mbox{}{<\textbf{refsDecl}>}\mbox{}\newline 
\hspace*{6pt}{<\textbf{cRefPattern}\hspace*{6pt}{matchPattern}="{([A-Za-z0-9]+) ([0-9]+):([0-9]+)}"\mbox{}\newline 
\hspace*{6pt}\hspace*{6pt}{replacementPattern}="{\#xpath(//body/div[@n='\$1']/div[\$2]/div3[\$3])}"/>}\mbox{}\newline 
{</\textbf{refsDecl}>}\end{shaded}\egroup 

This example is a formal representation for the referencing scheme described informally in the following example.
    \item[{Example}]
  \leavevmode\bgroup\exampleFont \begin{shaded}\noindent\mbox{}{<\textbf{refsDecl}>}\mbox{}\newline 
\hspace*{6pt}{<\textbf{p}>}References are made up by concatenating the value for the\mbox{}\newline 
\hspace*{6pt}{<\textbf{att}>}n{</\textbf{att}>} attribute on the highest level {<\textbf{gi}>}div{</\textbf{gi}>}\mbox{}\newline 
\hspace*{6pt}\hspace*{6pt} element, followed by a space, followed by the sequential\mbox{}\newline 
\hspace*{6pt}\hspace*{6pt} number of the next level {<\textbf{gi}>}div{</\textbf{gi}>} followed by a colon\mbox{}\newline 
\hspace*{6pt}\hspace*{6pt} followed by the sequential number of the next (and lowest)\mbox{}\newline 
\hspace*{6pt}\hspace*{6pt} level {<\textbf{gi}>}div{</\textbf{gi}>}.{</\textbf{p}>}\mbox{}\newline 
{</\textbf{refsDecl}>}\end{shaded}\egroup 


    \item[{Content model}]
  \mbox{}\hfill\\[-10pt]\begin{Verbatim}[fontsize=\small]
<content>
 <alternate>
  <classRef key="model.pLike"
   maxOccurs="unbounded" minOccurs="1"/>
  <elementRef key="cRefPattern"
   maxOccurs="unbounded" minOccurs="1"/>
  <elementRef key="refState"
   maxOccurs="unbounded" minOccurs="1"/>
 </alternate>
</content>
    
\end{Verbatim}

    \item[{Schema Declaration}]
  \mbox{}\hfill\\[-10pt]\begin{Verbatim}[fontsize=\small]
element refsDecl
{
   att.global.attributes,
   att.declarable.attributes,
   ( model.pLike+ | cRefPattern+ | refState+ )
}
\end{Verbatim}

\end{reflist}  \index{reg=<reg>|oddindex}
\begin{reflist}
\item[]\begin{specHead}{TEI.reg}{<reg> }(regularization) contains a reading which has been regularized or normalized in some sense. [\xref{http://www.tei-c.org/release/doc/tei-p5-doc/en/html/CO.html\#COEDREG}{3.4.2. Regularization and Normalization} \xref{http://www.tei-c.org/release/doc/tei-p5-doc/en/html/TC.html\#TC}{12. Critical Apparatus}]\end{specHead} 
    \item[{Module}]
  core
    \item[{Attributes}]
  Attributes att.global (\textit{@xml:id}, \textit{@n}, \textit{@xml:lang}, \textit{@xml:base}, \textit{@xml:space})  (att.global.rendition (\textit{@rend}, \textit{@style}, \textit{@rendition})) (att.global.linking (\textit{@corresp}, \textit{@synch}, \textit{@sameAs}, \textit{@copyOf}, \textit{@next}, \textit{@prev}, \textit{@exclude}, \textit{@select})) (att.global.analytic (\textit{@ana})) (att.global.facs (\textit{@facs})) (att.global.change (\textit{@change})) (att.global.responsibility (\textit{@cert}, \textit{@resp})) (att.global.source (\textit{@source})) att.editLike (\textit{@evidence}, \textit{@instant})  (att.dimensions (\textit{@unit}, \textit{@quantity}, \textit{@extent}, \textit{@precision}, \textit{@scope}) (att.ranging (\textit{@atLeast}, \textit{@atMost}, \textit{@min}, \textit{@max}, \textit{@confidence})) ) att.typed (\textit{@type}, \textit{@subtype}) 
    \item[{Member of}]
  model.choicePart model.pPart.transcriptional
    \item[{Contained by}]
  
    \item[analysis: ]
   cl pc phr s w\par 
    \item[core: ]
   abbr add addrLine author bibl biblScope choice citedRange corr date del distinct editor email emph expan foreign gloss head headItem headLabel hi item l label measure mentioned name note num orig p pubPlace publisher q quote ref reg rs said sic soCalled speaker stage street term textLang time title unclear\par 
    \item[figures: ]
   cell\par 
    \item[header: ]
   change distributor edition extent geoDecl handNote licence scriptNote typeNote\par 
    \item[linking: ]
   ab seg\par 
    \item[msdescription: ]
   accMat acquisition additions catchwords collation colophon condition custEvent decoNote explicit filiation finalRubric foliation heraldry incipit layout material musicNotation objectType origDate origPlace origin provenance rubric secFol signatures source stamp summary support surrogates watermark\par 
    \item[namesdates: ]
   addName affiliation birth bloc country death district education faith floruit forename genName geogFeat geogName nameLink nationality occupation offset orgName persName placeName region residence roleName settlement sex socecStatus surname\par 
    \item[textcrit: ]
   lem rdg wit witDetail\par 
    \item[textstructure: ]
   byline closer dateline docAuthor docDate docEdition docImprint imprimatur opener salute signed titlePart trailer\par 
    \item[transcr: ]
   am damage fw metamark mod restore retrace secl supplied surplus
    \item[{May contain}]
  
    \item[analysis: ]
   c cl interp interpGrp m pc phr s span spanGrp w\par 
    \item[core: ]
   abbr add address bibl biblStruct cb choice cit corr date del desc distinct email emph expan foreign gap gb gloss graphic hi index l label lb lg list listBibl measure measureGrp media mentioned milestone name note num orig pb ptr q quote ref reg rs said sic soCalled stage term time title unclear\par 
    \item[figures: ]
   figure formula notatedMusic table\par 
    \item[gaiji: ]
   g\par 
    \item[header: ]
   biblFull idno\par 
    \item[linking: ]
   alt altGrp anchor join joinGrp link linkGrp seg timeline\par 
    \item[msdescription: ]
   catchwords depth dim dimensions height heraldry locus locusGrp material msDesc objectType origDate origPlace secFol signatures stamp watermark width\par 
    \item[namesdates: ]
   addName affiliation bloc climate country district forename genName geo geogFeat geogName listEvent listNym listOrg listPerson listPlace location nameLink offset orgName persName placeName population region roleName settlement state surname terrain trait\par 
    \item[textcrit: ]
   app listApp listWit witDetail\par 
    \item[textstructure: ]
   floatingText\par 
    \item[transcr: ]
   addSpan am damage damageSpan delSpan ex fw handShift listTranspose metamark mod redo restore retrace secl space subst substJoin supplied surplus undo\par character data
    \item[{Example}]
  If all that is desired is to call attention to the fact that the copy text has been regularized, <reg> may be used alone:\leavevmode\bgroup\exampleFont \begin{shaded}\noindent\mbox{}{<\textbf{q}>}Please {<\textbf{reg}>}knock{</\textbf{reg}>} if an {<\textbf{reg}>}answer{</\textbf{reg}>} is {<\textbf{reg}>}required{</\textbf{reg}>}\mbox{}\newline 
{</\textbf{q}>}\end{shaded}\egroup 


    \item[{Example}]
  It is also possible to identify the individual responsible for the regularization, and, using the <choice> and <orig> elements, to provide both the original and regularized readings:\leavevmode\bgroup\exampleFont \begin{shaded}\noindent\mbox{}{<\textbf{q}>}Please {<\textbf{choice}>}\mbox{}\newline 
\hspace*{6pt}\hspace*{6pt}{<\textbf{reg}\hspace*{6pt}{resp}="{\#LB}">}knock{</\textbf{reg}>}\mbox{}\newline 
\hspace*{6pt}\hspace*{6pt}{<\textbf{orig}>}cnk{</\textbf{orig}>}\mbox{}\newline 
\hspace*{6pt}{</\textbf{choice}>} if an {<\textbf{choice}>}\mbox{}\newline 
\hspace*{6pt}\hspace*{6pt}{<\textbf{reg}>}answer{</\textbf{reg}>}\mbox{}\newline 
\hspace*{6pt}\hspace*{6pt}{<\textbf{orig}>}nsr{</\textbf{orig}>}\mbox{}\newline 
\hspace*{6pt}{</\textbf{choice}>} is {<\textbf{choice}>}\mbox{}\newline 
\hspace*{6pt}\hspace*{6pt}{<\textbf{reg}>}required{</\textbf{reg}>}\mbox{}\newline 
\hspace*{6pt}\hspace*{6pt}{<\textbf{orig}>}reqd{</\textbf{orig}>}\mbox{}\newline 
\hspace*{6pt}{</\textbf{choice}>}\mbox{}\newline 
{</\textbf{q}>}\end{shaded}\egroup 


    \item[{Content model}]
  \mbox{}\hfill\\[-10pt]\begin{Verbatim}[fontsize=\small]
<content>
 <macroRef key="macro.paraContent"/>
</content>
    
\end{Verbatim}

    \item[{Schema Declaration}]
  \mbox{}\hfill\\[-10pt]\begin{Verbatim}[fontsize=\small]
element reg
{
   att.global.attributes,
   att.editLike.attributes,
   att.typed.attributes,
   macro.paraContent}
\end{Verbatim}

\end{reflist}  \index{region=<region>|oddindex}
\begin{reflist}
\item[]\begin{specHead}{TEI.region}{<region> }contains the name of an administrative unit such as a state, province, or county, larger than a settlement, but smaller than a country. [\xref{http://www.tei-c.org/release/doc/tei-p5-doc/en/html/ND.html\#NDPLAC}{13.2.3. Place Names}]\end{specHead} 
    \item[{Module}]
  namesdates
    \item[{Attributes}]
  Attributes att.global (\textit{@xml:id}, \textit{@n}, \textit{@xml:lang}, \textit{@xml:base}, \textit{@xml:space})  (att.global.rendition (\textit{@rend}, \textit{@style}, \textit{@rendition})) (att.global.linking (\textit{@corresp}, \textit{@synch}, \textit{@sameAs}, \textit{@copyOf}, \textit{@next}, \textit{@prev}, \textit{@exclude}, \textit{@select})) (att.global.analytic (\textit{@ana})) (att.global.facs (\textit{@facs})) (att.global.change (\textit{@change})) (att.global.responsibility (\textit{@cert}, \textit{@resp})) (att.global.source (\textit{@source})) att.naming (\textit{@role}, \textit{@nymRef})  (att.canonical (\textit{@key}, \textit{@ref})) att.typed (\textit{@type}, \textit{@subtype}) att.datable (\textit{@calendar}, \textit{@period})  (att.datable.w3c (\textit{@when}, \textit{@notBefore}, \textit{@notAfter}, \textit{@from}, \textit{@to})) (att.datable.iso (\textit{@when-iso}, \textit{@notBefore-iso}, \textit{@notAfter-iso}, \textit{@from-iso}, \textit{@to-iso})) (att.datable.custom (\textit{@when-custom}, \textit{@notBefore-custom}, \textit{@notAfter-custom}, \textit{@from-custom}, \textit{@to-custom}, \textit{@datingPoint}, \textit{@datingMethod}))
    \item[{Member of}]
  model.placeNamePart
    \item[{Contained by}]
  
    \item[analysis: ]
   cl phr s span\par 
    \item[core: ]
   abbr add addrLine address author bibl biblScope citedRange corr date del desc distinct editor email emph expan foreign gloss head headItem headLabel hi item l label measure meeting mentioned name note num orig p pubPlace publisher q quote ref reg resp rs said sic soCalled speaker stage street term textLang time title unclear\par 
    \item[figures: ]
   cell figDesc\par 
    \item[header: ]
   authority catDesc change classCode correspAction creation distributor edition extent funder geoDecl handNote language licence principal rendition scriptNote sponsor tagUsage typeNote\par 
    \item[linking: ]
   ab seg\par 
    \item[msdescription: ]
   accMat acquisition additions altIdentifier catchwords collation colophon condition custEvent decoNote explicit filiation finalRubric foliation heraldry incipit layout material msIdentifier musicNotation objectType origDate origPlace origin provenance rubric secFol signatures source stamp summary support surrogates watermark\par 
    \item[namesdates: ]
   addName affiliation age birth bloc country death district education faith floruit forename genName geogFeat geogName langKnown location nameLink nationality occupation offset org orgName persName place placeName region residence roleName settlement sex socecStatus surname\par 
    \item[textcrit: ]
   lem rdg wit witDetail witness\par 
    \item[textstructure: ]
   byline closer dateline docAuthor docDate docEdition docImprint imprimatur opener salute signed titlePart trailer\par 
    \item[transcr: ]
   damage fw metamark mod restore retrace secl supplied surplus
    \item[{May contain}]
  
    \item[analysis: ]
   c cl interp interpGrp m pc phr s span spanGrp w\par 
    \item[core: ]
   abbr add address cb choice corr date del distinct email emph expan foreign gap gb gloss graphic hi index lb measure measureGrp media mentioned milestone name note num orig pb ptr ref reg rs sic soCalled term time title unclear\par 
    \item[figures: ]
   figure formula notatedMusic\par 
    \item[gaiji: ]
   g\par 
    \item[header: ]
   idno\par 
    \item[linking: ]
   alt altGrp anchor join joinGrp link linkGrp seg timeline\par 
    \item[msdescription: ]
   catchwords depth dim dimensions height heraldry locus locusGrp material objectType origDate origPlace secFol signatures stamp watermark width\par 
    \item[namesdates: ]
   addName affiliation bloc climate country district forename genName geo geogFeat geogName location nameLink offset orgName persName placeName population region roleName settlement state surname terrain trait\par 
    \item[textcrit: ]
   app witDetail\par 
    \item[transcr: ]
   addSpan am damage damageSpan delSpan ex fw handShift listTranspose metamark mod redo restore retrace secl space subst substJoin supplied surplus undo\par character data
    \item[{Example}]
  \leavevmode\bgroup\exampleFont \begin{shaded}\noindent\mbox{}{<\textbf{placeName}>}\mbox{}\newline 
\hspace*{6pt}{<\textbf{region}\hspace*{6pt}{n}="{IL}"\hspace*{6pt}{type}="{state}">}Illinois{</\textbf{region}>}\mbox{}\newline 
{</\textbf{placeName}>}\end{shaded}\egroup 


    \item[{Content model}]
  \mbox{}\hfill\\[-10pt]\begin{Verbatim}[fontsize=\small]
<content>
 <macroRef key="macro.phraseSeq"/>
</content>
    
\end{Verbatim}

    \item[{Schema Declaration}]
  \mbox{}\hfill\\[-10pt]\begin{Verbatim}[fontsize=\small]
element region
{
   att.global.attributes,
   att.naming.attributes,
   att.typed.attributes,
   att.datable.attributes,
   macro.phraseSeq}
\end{Verbatim}

\end{reflist}  \index{relatedItem=<relatedItem>|oddindex}\index{target=@target!<relatedItem>|oddindex}
\begin{reflist}
\item[]\begin{specHead}{TEI.relatedItem}{<relatedItem> }contains or references some other bibliographic item which is related to the present one in some specified manner, for example as a constituent or alternative version of it. [\xref{http://www.tei-c.org/release/doc/tei-p5-doc/en/html/CO.html\#COBIRI}{3.11.2.7. Related Items}]\end{specHead} 
    \item[{Module}]
  core
    \item[{Attributes}]
  Attributes att.global (\textit{@xml:id}, \textit{@n}, \textit{@xml:lang}, \textit{@xml:base}, \textit{@xml:space})  (att.global.rendition (\textit{@rend}, \textit{@style}, \textit{@rendition})) (att.global.linking (\textit{@corresp}, \textit{@synch}, \textit{@sameAs}, \textit{@copyOf}, \textit{@next}, \textit{@prev}, \textit{@exclude}, \textit{@select})) (att.global.analytic (\textit{@ana})) (att.global.facs (\textit{@facs})) (att.global.change (\textit{@change})) (att.global.responsibility (\textit{@cert}, \textit{@resp})) (att.global.source (\textit{@source})) att.typed (\textit{@type}, \textit{@subtype}) \hfil\\[-10pt]\begin{sansreflist}
    \item[@target]
  points to the related bibliographic element by means of an absolute or relative URI reference
\begin{reflist}
    \item[{Status}]
  Optional
    \item[{Datatype}]
  teidata.pointer
\end{reflist}  
\end{sansreflist}  
    \item[{Member of}]
  model.biblPart 
    \item[{Contained by}]
  
    \item[core: ]
   bibl biblStruct\par 
    \item[header: ]
   notesStmt
    \item[{May contain}]
  
    \item[core: ]
   bibl biblStruct listBibl ptr ref\par 
    \item[header: ]
   biblFull\par 
    \item[msdescription: ]
   msDesc
    \item[{Note}]
  \par
If the {\itshape target} attribute is used to reference the related bibliographic item, the element should be empty.
    \item[{Example}]
  \leavevmode\bgroup\exampleFont \begin{shaded}\noindent\mbox{}{<\textbf{biblStruct}>}\mbox{}\newline 
\hspace*{6pt}{<\textbf{monogr}>}\mbox{}\newline 
\hspace*{6pt}\hspace*{6pt}{<\textbf{author}>}Shirley, James{</\textbf{author}>}\mbox{}\newline 
\hspace*{6pt}\hspace*{6pt}{<\textbf{title}\hspace*{6pt}{type}="{main}">}The gentlemen of Venice{</\textbf{title}>}\mbox{}\newline 
\hspace*{6pt}\hspace*{6pt}{<\textbf{imprint}>}\mbox{}\newline 
\hspace*{6pt}\hspace*{6pt}\hspace*{6pt}{<\textbf{pubPlace}>}New York{</\textbf{pubPlace}>}\mbox{}\newline 
\hspace*{6pt}\hspace*{6pt}\hspace*{6pt}{<\textbf{publisher}>}Readex Microprint{</\textbf{publisher}>}\mbox{}\newline 
\hspace*{6pt}\hspace*{6pt}\hspace*{6pt}{<\textbf{date}>}1953{</\textbf{date}>}\mbox{}\newline 
\hspace*{6pt}\hspace*{6pt}{</\textbf{imprint}>}\mbox{}\newline 
\hspace*{6pt}\hspace*{6pt}{<\textbf{extent}>}1 microprint card, 23 x 15 cm.{</\textbf{extent}>}\mbox{}\newline 
\hspace*{6pt}{</\textbf{monogr}>}\mbox{}\newline 
\hspace*{6pt}{<\textbf{series}>}\mbox{}\newline 
\hspace*{6pt}\hspace*{6pt}{<\textbf{title}>}Three centuries of drama: English, 1642–1700{</\textbf{title}>}\mbox{}\newline 
\hspace*{6pt}{</\textbf{series}>}\mbox{}\newline 
\hspace*{6pt}{<\textbf{relatedItem}\hspace*{6pt}{type}="{otherForm}">}\mbox{}\newline 
\hspace*{6pt}\hspace*{6pt}{<\textbf{biblStruct}>}\mbox{}\newline 
\hspace*{6pt}\hspace*{6pt}\hspace*{6pt}{<\textbf{monogr}>}\mbox{}\newline 
\hspace*{6pt}\hspace*{6pt}\hspace*{6pt}\hspace*{6pt}{<\textbf{author}>}Shirley, James{</\textbf{author}>}\mbox{}\newline 
\hspace*{6pt}\hspace*{6pt}\hspace*{6pt}\hspace*{6pt}{<\textbf{title}\hspace*{6pt}{type}="{main}">}The gentlemen of Venice{</\textbf{title}>}\mbox{}\newline 
\hspace*{6pt}\hspace*{6pt}\hspace*{6pt}\hspace*{6pt}{<\textbf{title}\hspace*{6pt}{type}="{sub}">}a tragi-comedie presented at the private house in Salisbury\mbox{}\newline 
\hspace*{6pt}\hspace*{6pt}\hspace*{6pt}\hspace*{6pt}\hspace*{6pt}\hspace*{6pt}\hspace*{6pt}\hspace*{6pt} Court by Her Majesties servants{</\textbf{title}>}\mbox{}\newline 
\hspace*{6pt}\hspace*{6pt}\hspace*{6pt}\hspace*{6pt}{<\textbf{imprint}>}\mbox{}\newline 
\hspace*{6pt}\hspace*{6pt}\hspace*{6pt}\hspace*{6pt}\hspace*{6pt}{<\textbf{pubPlace}>}London{</\textbf{pubPlace}>}\mbox{}\newline 
\hspace*{6pt}\hspace*{6pt}\hspace*{6pt}\hspace*{6pt}\hspace*{6pt}{<\textbf{publisher}>}H. Moseley{</\textbf{publisher}>}\mbox{}\newline 
\hspace*{6pt}\hspace*{6pt}\hspace*{6pt}\hspace*{6pt}\hspace*{6pt}{<\textbf{date}>}1655{</\textbf{date}>}\mbox{}\newline 
\hspace*{6pt}\hspace*{6pt}\hspace*{6pt}\hspace*{6pt}{</\textbf{imprint}>}\mbox{}\newline 
\hspace*{6pt}\hspace*{6pt}\hspace*{6pt}\hspace*{6pt}{<\textbf{extent}>}78 p.{</\textbf{extent}>}\mbox{}\newline 
\hspace*{6pt}\hspace*{6pt}\hspace*{6pt}{</\textbf{monogr}>}\mbox{}\newline 
\hspace*{6pt}\hspace*{6pt}{</\textbf{biblStruct}>}\mbox{}\newline 
\hspace*{6pt}{</\textbf{relatedItem}>}\mbox{}\newline 
{</\textbf{biblStruct}>}\end{shaded}\egroup 


    \item[{Schematron}]
   <sch:report test="@target and count( child::* ) > 0">If the @target attribute on <sch:name/> is used, the  relatedItem element must be empty</sch:report> <sch:assert test="@target or child::*">A relatedItem element should have either a 'target' attribute  or a child element to indicate the related bibliographic item</sch:assert>
    \item[{Content model}]
  \mbox{}\hfill\\[-10pt]\begin{Verbatim}[fontsize=\small]
<content>
 <alternate minOccurs="0">
  <classRef key="model.biblLike"/>
  <classRef key="model.ptrLike"/>
 </alternate>
</content>
    
\end{Verbatim}

    \item[{Schema Declaration}]
  \mbox{}\hfill\\[-10pt]\begin{Verbatim}[fontsize=\small]
element relatedItem
{
   att.global.attributes,
   att.typed.attributes,
   attribute target { text }?,
   ( model.biblLike | model.ptrLike )?
}
\end{Verbatim}

\end{reflist}  \index{relation=<relation>|oddindex}\index{name=@name!<relation>|oddindex}\index{active=@active!<relation>|oddindex}\index{mutual=@mutual!<relation>|oddindex}\index{passive=@passive!<relation>|oddindex}
\begin{reflist}
\item[]\begin{specHead}{TEI.relation}{<relation> }(relationship) describes any kind of relationship or linkage amongst a specified group of places, events, persons, objects or other items. [\xref{http://www.tei-c.org/release/doc/tei-p5-doc/en/html/ND.html\#NDPERSREL}{13.3.2.3. Personal Relationships}]\end{specHead} 
    \item[{Module}]
  namesdates
    \item[{Attributes}]
  Attributes att.global (\textit{@xml:id}, \textit{@n}, \textit{@xml:lang}, \textit{@xml:base}, \textit{@xml:space})  (att.global.rendition (\textit{@rend}, \textit{@style}, \textit{@rendition})) (att.global.linking (\textit{@corresp}, \textit{@synch}, \textit{@sameAs}, \textit{@copyOf}, \textit{@next}, \textit{@prev}, \textit{@exclude}, \textit{@select})) (att.global.analytic (\textit{@ana})) (att.global.facs (\textit{@facs})) (att.global.change (\textit{@change})) (att.global.responsibility (\textit{@cert}, \textit{@resp})) (att.global.source (\textit{@source})) att.datable (\textit{@calendar}, \textit{@period})  (att.datable.w3c (\textit{@when}, \textit{@notBefore}, \textit{@notAfter}, \textit{@from}, \textit{@to})) (att.datable.iso (\textit{@when-iso}, \textit{@notBefore-iso}, \textit{@notAfter-iso}, \textit{@from-iso}, \textit{@to-iso})) (att.datable.custom (\textit{@when-custom}, \textit{@notBefore-custom}, \textit{@notAfter-custom}, \textit{@from-custom}, \textit{@to-custom}, \textit{@datingPoint}, \textit{@datingMethod})) att.editLike (\textit{@evidence}, \textit{@instant})  (att.dimensions (\textit{@unit}, \textit{@quantity}, \textit{@extent}, \textit{@precision}, \textit{@scope}) (att.ranging (\textit{@atLeast}, \textit{@atMost}, \textit{@min}, \textit{@max}, \textit{@confidence})) ) att.canonical (\textit{@key}, \textit{@ref}) att.sortable (\textit{@sortKey}) att.typed (\textit{@type}, \textit{@subtype}) \hfil\\[-10pt]\begin{sansreflist}
    \item[@name]
  supplies a name for the kind of relationship of which this is an instance.
\begin{reflist}
    \item[{Status}]
  Optional
    \item[{Datatype}]
  teidata.enumerated
\end{reflist}  
    \item[@active]
  identifies the ‘active’ participants in a non-mutual relationship, or all the participants in a mutual one.
\begin{reflist}
    \item[{Status}]
  Optional
    \item[{Datatype}]
  1–∞ occurrences of teidata.pointer separated by whitespace
\end{reflist}  
    \item[@mutual]
  supplies a list of participants amongst all of whom the relationship holds equally.
\begin{reflist}
    \item[{Status}]
  Optional
    \item[{Datatype}]
  1–∞ occurrences of teidata.pointer separated by whitespace
\end{reflist}  
    \item[@passive]
  identifies the ‘passive’ participants in a non-mutual relationship.
\begin{reflist}
    \item[{Status}]
  Optional
    \item[{Datatype}]
  1–∞ occurrences of teidata.pointer separated by whitespace
\end{reflist}  
\end{sansreflist}  
    \item[{Contained by}]
  
    \item[core: ]
   listBibl\par 
    \item[namesdates: ]
   listEvent listNym listOrg listPerson listPlace listRelation
    \item[{May contain}]
  
    \item[core: ]
   desc
    \item[{Note}]
  \par
Only one of the attributes {\itshape active} and {\itshape mutual} may be supplied; the attribute {\itshape passive} may be supplied only if the attribute {\itshape active} is supplied. Not all of these constraints can be enforced in all schema languages.
    \item[{Example}]
  \leavevmode\bgroup\exampleFont \begin{shaded}\noindent\mbox{}{<\textbf{relation}\hspace*{6pt}{active}="{\#p1}"\hspace*{6pt}{name}="{supervisor}"\mbox{}\newline 
\hspace*{6pt}{passive}="{\#p2 \#p3 \#p4}"\hspace*{6pt}{type}="{social}"/>}\end{shaded}\egroup 

This indicates that the person with identifier p1 is supervisor of persons p2, p3, and p4.
    \item[{Example}]
  \leavevmode\bgroup\exampleFont \begin{shaded}\noindent\mbox{}{<\textbf{relation}\hspace*{6pt}{mutual}="{\#p2 \#p3 \#p4}"\mbox{}\newline 
\hspace*{6pt}{name}="{friends}"\hspace*{6pt}{type}="{personal}"/>}\end{shaded}\egroup 

This indicates that p2, p3, and p4 are all friends.
    \item[{Example}]
  \leavevmode\bgroup\exampleFont \begin{shaded}\noindent\mbox{}{<\textbf{relation}\hspace*{6pt}{active}="{http://id.clarosnet.org/places/metamorphoses/place/italy-orvieto}"\mbox{}\newline 
\hspace*{6pt}{name}="{P89\textunderscore falls\textunderscore within}"\mbox{}\newline 
\hspace*{6pt}{passive}="{http://id.clarosnet.org/places/metamorphoses/country/IT}"\hspace*{6pt}{type}="{CRM}"/>}\end{shaded}\egroup 

This indicates that there is a relation, defined by CIDOC CRM, between two resources identified by URLs.
    \item[{Example}]
  \leavevmode\bgroup\exampleFont \begin{shaded}\noindent\mbox{}{<\textbf{relation}\hspace*{6pt}{active}="{http://www.ancientwisdoms.ac.uk/cts/urn:cts:greekLit:tlg3017.Syno298.sawsGrc01:divedition.divsection1.o14.a107}"\mbox{}\newline 
\hspace*{6pt}{passive}="{http://data.perseus.org/citations/urn:cts:greekLit:tlg0031.tlg002.perseus-grc1:9.35}"\mbox{}\newline 
\hspace*{6pt}{ref}="{http://purl.org/saws/ontology\#isVariantOf}"\hspace*{6pt}{resp}="{http://viaf.org/viaf/44335536/}"/>}\end{shaded}\egroup 

This example records a relationship, defined by the SAWS ontology, between a passage of text identified by a CTS URN, and a variant passage of text in the Perseus Digital Library, and assigns the identification of the relationship to a particular editor (all using resolvable URIs).
    \item[{Schematron}]
   <s:assert test="@ref or @key or @name">One of the attributes 'name', 'ref' or 'key' must be supplied</s:assert>
    \item[{Schematron}]
   <s:report test="@active and @mutual">Only one of the attributes  @active and @mutual may be supplied</s:report>
    \item[{Schematron}]
   <s:report test="@passive and not(@active)">the attribute 'passive'  may be supplied only if the attribute 'active' is  supplied</s:report>
    \item[{Content model}]
  \mbox{}\hfill\\[-10pt]\begin{Verbatim}[fontsize=\small]
<content>
 <elementRef key="desc" minOccurs="0"/>
</content>
    
\end{Verbatim}

    \item[{Schema Declaration}]
  \mbox{}\hfill\\[-10pt]\begin{Verbatim}[fontsize=\small]
element relation
{
   att.global.attributes,
   att.datable.attributes,
   att.editLike.attributes,
   att.canonical.attributes,
   att.sortable.attributes,
   att.typed.attributes,
   attribute name { text }?,
   ( attribute active { list { + } }? | attribute mutual { list { + } }? ),
   attribute passive { list { + } }?,
   desc?
}
\end{Verbatim}

\end{reflist}  \index{rendition=<rendition>|oddindex}\index{scope=@scope!<rendition>|oddindex}\index{selector=@selector!<rendition>|oddindex}
\begin{reflist}
\item[]\begin{specHead}{TEI.rendition}{<rendition> }supplies information about the rendition or appearance of one or more elements in the source text. [\xref{http://www.tei-c.org/release/doc/tei-p5-doc/en/html/HD.html\#HD57}{2.3.4. The Tagging Declaration}]\end{specHead} 
    \item[{Module}]
  header
    \item[{Attributes}]
  Attributes att.global (\textit{@xml:id}, \textit{@n}, \textit{@xml:lang}, \textit{@xml:base}, \textit{@xml:space})  (att.global.rendition (\textit{@rend}, \textit{@style}, \textit{@rendition})) (att.global.linking (\textit{@corresp}, \textit{@synch}, \textit{@sameAs}, \textit{@copyOf}, \textit{@next}, \textit{@prev}, \textit{@exclude}, \textit{@select})) (att.global.analytic (\textit{@ana})) (att.global.facs (\textit{@facs})) (att.global.change (\textit{@change})) (att.global.responsibility (\textit{@cert}, \textit{@resp})) (att.global.source (\textit{@source})) att.styleDef (\textit{@scheme}, \textit{@schemeVersion}) \hfil\\[-10pt]\begin{sansreflist}
    \item[@scope]
  where CSS is used, provides a way of defining ‘pseudo-elements’, that is, styling rules applicable to specific sub-portions of an element.
\begin{reflist}
    \item[{Status}]
  Optional
    \item[{Datatype}]
  teidata.enumerated
    \item[{Sample values include:}]
  \begin{description}

\item[{first-line}]styling applies to the first line of the target element
\item[{first-letter}]styling applies to the first letter of the target element
\item[{before}]styling should be applied immediately before the content of the target element
\item[{after}]styling should be applied immediately after the content of the target element
\end{description} 
\end{reflist}  
    \item[@selector]
  contains a selector or series of selectors specifying the elements to which the contained style description applies, expressed in the language specified in the {\itshape scheme} attribute.
\begin{reflist}
    \item[{Status}]
  Optional
    \item[{Datatype}]
  teidata.text
    \item[]\exampleFont {<\textbf{rendition}\hspace*{6pt}{scheme}="{css}"\mbox{}\newline 
\hspace*{6pt}{selector}="{text, front, back, body, div, p, ab}">} \mbox{}\newline 
 display: block;\mbox{}\newline 
{</\textbf{rendition}>}
    \item[]\exampleFont {<\textbf{rendition}\hspace*{6pt}{scheme}="{css}"\mbox{}\newline 
\hspace*{6pt}{selector}="{*[rend*=italic]}">} font-style: italic;\mbox{}\newline 
{</\textbf{rendition}>}
    \item[{Note}]
  \par
Since the default value of the {\itshape scheme} attribute is assumed to be CSS, the default expectation for this attribute, in the absence of {\itshape scheme}, is that CSS selector syntax will be used.
    \item[{Note}]
  \par
While {\itshape rendition} is used to point from an element in the transcribed source to a <rendition> element in the header which describes how it appears, the {\itshape selector} attribute allows the encoder to point in the other direction: from a <rendition> in the header to a collection of elements which all share the same renditional features. In both cases, the intention is to record the appearance of the source text, not to prescribe any particular output rendering.
\end{reflist}  
\end{sansreflist}  
    \item[{Contained by}]
  
    \item[header: ]
   tagsDecl
    \item[{May contain}]
  
    \item[core: ]
   abbr address bibl biblStruct choice cit date desc distinct email emph expan foreign gloss hi label list listBibl measure measureGrp mentioned name num ptr q quote ref rs said soCalled stage term time title\par 
    \item[figures: ]
   table\par 
    \item[header: ]
   biblFull idno\par 
    \item[msdescription: ]
   catchwords depth dim dimensions height heraldry locus locusGrp material msDesc objectType origDate origPlace secFol signatures stamp watermark width\par 
    \item[namesdates: ]
   addName affiliation bloc climate country district forename genName geo geogFeat geogName listEvent listNym listOrg listPerson listPlace location nameLink offset orgName persName placeName population region roleName settlement state surname terrain trait\par 
    \item[textcrit: ]
   listApp listWit\par 
    \item[textstructure: ]
   floatingText\par 
    \item[transcr: ]
   am ex subst\par character data
    \item[{Example}]
  \leavevmode\bgroup\exampleFont \begin{shaded}\noindent\mbox{}{<\textbf{tagsDecl}>}\mbox{}\newline 
\hspace*{6pt}{<\textbf{rendition}\hspace*{6pt}{scheme}="{css}"\hspace*{6pt}{xml:id}="{r-center}">}text-align: center;{</\textbf{rendition}>}\mbox{}\newline 
\hspace*{6pt}{<\textbf{rendition}\hspace*{6pt}{scheme}="{css}"\hspace*{6pt}{xml:id}="{r-small}">}font-size: small;{</\textbf{rendition}>}\mbox{}\newline 
\hspace*{6pt}{<\textbf{rendition}\hspace*{6pt}{scheme}="{css}"\hspace*{6pt}{xml:id}="{r-large}">}font-size: large;{</\textbf{rendition}>}\mbox{}\newline 
\hspace*{6pt}{<\textbf{rendition}\hspace*{6pt}{scheme}="{css}"\mbox{}\newline 
\hspace*{6pt}\hspace*{6pt}{scope}="{first-letter}"\hspace*{6pt}{xml:id}="{initcaps}">}font-size: xx-large{</\textbf{rendition}>}\mbox{}\newline 
{</\textbf{tagsDecl}>}\end{shaded}\egroup 


    \item[{Content model}]
  \mbox{}\hfill\\[-10pt]\begin{Verbatim}[fontsize=\small]
<content>
 <macroRef key="macro.limitedContent"/>
</content>
    
\end{Verbatim}

    \item[{Schema Declaration}]
  \mbox{}\hfill\\[-10pt]\begin{Verbatim}[fontsize=\small]
element rendition
{
   att.global.attributes,
   att.styleDef.attributes,
   attribute scope { text }?,
   attribute selector { text }?,
   macro.limitedContent}
\end{Verbatim}

\end{reflist}  \index{repository=<repository>|oddindex}
\begin{reflist}
\item[]\begin{specHead}{TEI.repository}{<repository> }contains the name of a repository within which manuscripts are stored, possibly forming part of an institution. [\xref{http://www.tei-c.org/release/doc/tei-p5-doc/en/html/MS.html\#msid}{10.4. The Manuscript Identifier}]\end{specHead} 
    \item[{Module}]
  msdescription
    \item[{Attributes}]
  Attributes att.global (\textit{@xml:id}, \textit{@n}, \textit{@xml:lang}, \textit{@xml:base}, \textit{@xml:space})  (att.global.rendition (\textit{@rend}, \textit{@style}, \textit{@rendition})) (att.global.linking (\textit{@corresp}, \textit{@synch}, \textit{@sameAs}, \textit{@copyOf}, \textit{@next}, \textit{@prev}, \textit{@exclude}, \textit{@select})) (att.global.analytic (\textit{@ana})) (att.global.facs (\textit{@facs})) (att.global.change (\textit{@change})) (att.global.responsibility (\textit{@cert}, \textit{@resp})) (att.global.source (\textit{@source})) att.naming (\textit{@role}, \textit{@nymRef})  (att.canonical (\textit{@key}, \textit{@ref}))
    \item[{Contained by}]
  
    \item[msdescription: ]
   altIdentifier msIdentifier
    \item[{May contain}]
  
    \item[gaiji: ]
   g\par character data
    \item[{Example}]
  \leavevmode\bgroup\exampleFont \begin{shaded}\noindent\mbox{}{<\textbf{msIdentifier}>}\mbox{}\newline 
\hspace*{6pt}{<\textbf{settlement}>}Oxford{</\textbf{settlement}>}\mbox{}\newline 
\hspace*{6pt}{<\textbf{institution}>}University of Oxford{</\textbf{institution}>}\mbox{}\newline 
\hspace*{6pt}{<\textbf{repository}>}Bodleian Library{</\textbf{repository}>}\mbox{}\newline 
\hspace*{6pt}{<\textbf{idno}>}MS. Bodley 406{</\textbf{idno}>}\mbox{}\newline 
{</\textbf{msIdentifier}>}\end{shaded}\egroup 


    \item[{Content model}]
  \fbox{\ttfamily <content>\newline
 <macroRef key="macro.xtext"/>\newline
</content>\newline
    } 
    \item[{Schema Declaration}]
  \mbox{}\hfill\\[-10pt]\begin{Verbatim}[fontsize=\small]
element repository
{
   att.global.attributes,
   att.naming.attributes,
   macro.xtext}
\end{Verbatim}

\end{reflist}  \index{residence=<residence>|oddindex}
\begin{reflist}
\item[]\begin{specHead}{TEI.residence}{<residence> }describes a person's present or past places of residence. [\xref{http://www.tei-c.org/release/doc/tei-p5-doc/en/html/CC.html\#CCAHPA}{15.2.2. The Participant Description}]\end{specHead} 
    \item[{Module}]
  namesdates
    \item[{Attributes}]
  Attributes att.global (\textit{@xml:id}, \textit{@n}, \textit{@xml:lang}, \textit{@xml:base}, \textit{@xml:space})  (att.global.rendition (\textit{@rend}, \textit{@style}, \textit{@rendition})) (att.global.linking (\textit{@corresp}, \textit{@synch}, \textit{@sameAs}, \textit{@copyOf}, \textit{@next}, \textit{@prev}, \textit{@exclude}, \textit{@select})) (att.global.analytic (\textit{@ana})) (att.global.facs (\textit{@facs})) (att.global.change (\textit{@change})) (att.global.responsibility (\textit{@cert}, \textit{@resp})) (att.global.source (\textit{@source})) att.datable (\textit{@calendar}, \textit{@period})  (att.datable.w3c (\textit{@when}, \textit{@notBefore}, \textit{@notAfter}, \textit{@from}, \textit{@to})) (att.datable.iso (\textit{@when-iso}, \textit{@notBefore-iso}, \textit{@notAfter-iso}, \textit{@from-iso}, \textit{@to-iso})) (att.datable.custom (\textit{@when-custom}, \textit{@notBefore-custom}, \textit{@notAfter-custom}, \textit{@from-custom}, \textit{@to-custom}, \textit{@datingPoint}, \textit{@datingMethod})) att.editLike (\textit{@evidence}, \textit{@instant})  (att.dimensions (\textit{@unit}, \textit{@quantity}, \textit{@extent}, \textit{@precision}, \textit{@scope}) (att.ranging (\textit{@atLeast}, \textit{@atMost}, \textit{@min}, \textit{@max}, \textit{@confidence})) ) att.naming (\textit{@role}, \textit{@nymRef})  (att.canonical (\textit{@key}, \textit{@ref}))
    \item[{Member of}]
  model.persStateLike
    \item[{Contained by}]
  
    \item[namesdates: ]
   person personGrp
    \item[{May contain}]
  
    \item[analysis: ]
   c cl interp interpGrp m pc phr s span spanGrp w\par 
    \item[core: ]
   abbr add address cb choice corr date del distinct email emph expan foreign gap gb gloss graphic hi index lb measure measureGrp media mentioned milestone name note num orig pb ptr ref reg rs sic soCalled term time title unclear\par 
    \item[figures: ]
   figure formula notatedMusic\par 
    \item[gaiji: ]
   g\par 
    \item[header: ]
   idno\par 
    \item[linking: ]
   alt altGrp anchor join joinGrp link linkGrp seg timeline\par 
    \item[msdescription: ]
   catchwords depth dim dimensions height heraldry locus locusGrp material objectType origDate origPlace secFol signatures stamp watermark width\par 
    \item[namesdates: ]
   addName affiliation bloc climate country district forename genName geo geogFeat geogName location nameLink offset orgName persName placeName population region roleName settlement state surname terrain trait\par 
    \item[textcrit: ]
   app witDetail\par 
    \item[transcr: ]
   addSpan am damage damageSpan delSpan ex fw handShift listTranspose metamark mod redo restore retrace secl space subst substJoin supplied surplus undo\par character data
    \item[{Example}]
  \leavevmode\bgroup\exampleFont \begin{shaded}\noindent\mbox{}{<\textbf{residence}>}Childhood in East Africa and long term resident of Glasgow, Scotland.{</\textbf{residence}>}\end{shaded}\egroup 


    \item[{Example}]
  \leavevmode\bgroup\exampleFont \begin{shaded}\noindent\mbox{}{<\textbf{residence}\hspace*{6pt}{notAfter}="{1997}">}Mbeni estate, Dzukumura region, Matabele land{</\textbf{residence}>}\mbox{}\newline 
{<\textbf{residence}\hspace*{6pt}{notAfter}="{1996}"\hspace*{6pt}{notBefore}="{1903}">}\mbox{}\newline 
\hspace*{6pt}{<\textbf{placeName}>}\mbox{}\newline 
\hspace*{6pt}\hspace*{6pt}{<\textbf{settlement}>}Glasgow{</\textbf{settlement}>}\mbox{}\newline 
\hspace*{6pt}\hspace*{6pt}{<\textbf{region}>}Scotland{</\textbf{region}>}\mbox{}\newline 
\hspace*{6pt}{</\textbf{placeName}>}\mbox{}\newline 
{</\textbf{residence}>}\end{shaded}\egroup 


    \item[{Content model}]
  \mbox{}\hfill\\[-10pt]\begin{Verbatim}[fontsize=\small]
<content>
 <macroRef key="macro.phraseSeq"/>
</content>
    
\end{Verbatim}

    \item[{Schema Declaration}]
  \mbox{}\hfill\\[-10pt]\begin{Verbatim}[fontsize=\small]
element residence
{
   att.global.attributes,
   att.datable.attributes,
   att.editLike.attributes,
   att.naming.attributes,
   macro.phraseSeq}
\end{Verbatim}

\end{reflist}  \index{resp=<resp>|oddindex}
\begin{reflist}
\item[]\begin{specHead}{TEI.resp}{<resp> }(responsibility) contains a phrase describing the nature of a person's intellectual responsibility, or an organization's role in the production or distribution of a work. [\xref{http://www.tei-c.org/release/doc/tei-p5-doc/en/html/CO.html\#COBICOR}{3.11.2.2. Titles, Authors, and Editors} \xref{http://www.tei-c.org/release/doc/tei-p5-doc/en/html/HD.html\#HD21}{2.2.1. The Title Statement} \xref{http://www.tei-c.org/release/doc/tei-p5-doc/en/html/HD.html\#HD22}{2.2.2. The Edition Statement} \xref{http://www.tei-c.org/release/doc/tei-p5-doc/en/html/HD.html\#HD26}{2.2.5. The Series Statement}]\end{specHead} 
    \item[{Module}]
  core
    \item[{Attributes}]
  Attributes att.global (\textit{@xml:id}, \textit{@n}, \textit{@xml:lang}, \textit{@xml:base}, \textit{@xml:space})  (att.global.rendition (\textit{@rend}, \textit{@style}, \textit{@rendition})) (att.global.linking (\textit{@corresp}, \textit{@synch}, \textit{@sameAs}, \textit{@copyOf}, \textit{@next}, \textit{@prev}, \textit{@exclude}, \textit{@select})) (att.global.analytic (\textit{@ana})) (att.global.facs (\textit{@facs})) (att.global.change (\textit{@change})) (att.global.responsibility (\textit{@cert}, \textit{@resp})) (att.global.source (\textit{@source})) att.canonical (\textit{@key}, \textit{@ref}) att.datable (\textit{@calendar}, \textit{@period})  (att.datable.w3c (\textit{@when}, \textit{@notBefore}, \textit{@notAfter}, \textit{@from}, \textit{@to})) (att.datable.iso (\textit{@when-iso}, \textit{@notBefore-iso}, \textit{@notAfter-iso}, \textit{@from-iso}, \textit{@to-iso})) (att.datable.custom (\textit{@when-custom}, \textit{@notBefore-custom}, \textit{@notAfter-custom}, \textit{@from-custom}, \textit{@to-custom}, \textit{@datingPoint}, \textit{@datingMethod}))
    \item[{Contained by}]
  
    \item[core: ]
   respStmt
    \item[{May contain}]
  
    \item[analysis: ]
   interp interpGrp span spanGrp\par 
    \item[core: ]
   abbr address cb choice date distinct email emph expan foreign gap gb gloss hi index lb measure measureGrp mentioned milestone name note num pb ptr ref rs soCalled term time title\par 
    \item[figures: ]
   figure notatedMusic\par 
    \item[header: ]
   idno\par 
    \item[linking: ]
   alt altGrp anchor join joinGrp link linkGrp timeline\par 
    \item[msdescription: ]
   catchwords depth dim dimensions height heraldry locus locusGrp material objectType origDate origPlace secFol signatures stamp watermark width\par 
    \item[namesdates: ]
   addName affiliation bloc climate country district forename genName geo geogFeat geogName location nameLink offset orgName persName placeName population region roleName settlement state surname terrain trait\par 
    \item[textcrit: ]
   app witDetail\par 
    \item[transcr: ]
   addSpan am damageSpan delSpan ex fw listTranspose metamark space subst substJoin\par character data
    \item[{Note}]
  \par
The attribute {\itshape ref}, inherited from the class \textsf{att.canonical} may be used to indicate the kind of responsibility in a normalized form by referring directly to a standardized list of responsibility types, such as that maintained by a naming authority, for example the list maintained at \url{http://www.loc.gov/marc/relators/relacode.html} for bibliographic usage.
    \item[{Example}]
  \leavevmode\bgroup\exampleFont \begin{shaded}\noindent\mbox{}{<\textbf{respStmt}>}\mbox{}\newline 
\hspace*{6pt}{<\textbf{resp}\hspace*{6pt}{ref}="{http://id.loc.gov/vocabulary/relators/com.html}">}compiler{</\textbf{resp}>}\mbox{}\newline 
\hspace*{6pt}{<\textbf{name}>}Edward Child{</\textbf{name}>}\mbox{}\newline 
{</\textbf{respStmt}>}\end{shaded}\egroup 


    \item[{Content model}]
  \mbox{}\hfill\\[-10pt]\begin{Verbatim}[fontsize=\small]
<content>
 <macroRef key="macro.phraseSeq.limited"/>
</content>
    
\end{Verbatim}

    \item[{Schema Declaration}]
  \mbox{}\hfill\\[-10pt]\begin{Verbatim}[fontsize=\small]
element resp
{
   att.global.attributes,
   att.canonical.attributes,
   att.datable.attributes,
   macro.phraseSeq.limited}
\end{Verbatim}

\end{reflist}  \index{respStmt=<respStmt>|oddindex}
\begin{reflist}
\item[]\begin{specHead}{TEI.respStmt}{<respStmt> }(statement of responsibility) supplies a statement of responsibility for the intellectual content of a text, edition, recording, or series, where the specialized elements for authors, editors, etc. do not suffice or do not apply. May also be used to encode information about individuals or organizations which have played a role in the production or distribution of a bibliographic work. [\xref{http://www.tei-c.org/release/doc/tei-p5-doc/en/html/CO.html\#COBICOR}{3.11.2.2. Titles, Authors, and Editors} \xref{http://www.tei-c.org/release/doc/tei-p5-doc/en/html/HD.html\#HD21}{2.2.1. The Title Statement} \xref{http://www.tei-c.org/release/doc/tei-p5-doc/en/html/HD.html\#HD22}{2.2.2. The Edition Statement} \xref{http://www.tei-c.org/release/doc/tei-p5-doc/en/html/HD.html\#HD26}{2.2.5. The Series Statement}]\end{specHead} 
    \item[{Module}]
  core
    \item[{Attributes}]
  Attributes att.global (\textit{@xml:id}, \textit{@n}, \textit{@xml:lang}, \textit{@xml:base}, \textit{@xml:space})  (att.global.rendition (\textit{@rend}, \textit{@style}, \textit{@rendition})) (att.global.linking (\textit{@corresp}, \textit{@synch}, \textit{@sameAs}, \textit{@copyOf}, \textit{@next}, \textit{@prev}, \textit{@exclude}, \textit{@select})) (att.global.analytic (\textit{@ana})) (att.global.facs (\textit{@facs})) (att.global.change (\textit{@change})) (att.global.responsibility (\textit{@cert}, \textit{@resp})) (att.global.source (\textit{@source})) att.canonical (\textit{@key}, \textit{@ref}) 
    \item[{Member of}]
  model.respLike 
    \item[{Contained by}]
  
    \item[core: ]
   analytic bibl imprint monogr series\par 
    \item[header: ]
   editionStmt seriesStmt titleStmt\par 
    \item[msdescription: ]
   msItem msItemStruct
    \item[{May contain}]
  
    \item[core: ]
   name resp\par 
    \item[namesdates: ]
   orgName persName
    \item[{Example}]
  \leavevmode\bgroup\exampleFont \begin{shaded}\noindent\mbox{}{<\textbf{respStmt}>}\mbox{}\newline 
\hspace*{6pt}{<\textbf{resp}>}transcribed from original ms{</\textbf{resp}>}\mbox{}\newline 
\hspace*{6pt}{<\textbf{persName}>}Claus Huitfeldt{</\textbf{persName}>}\mbox{}\newline 
{</\textbf{respStmt}>}\end{shaded}\egroup 


    \item[{Example}]
  \leavevmode\bgroup\exampleFont \begin{shaded}\noindent\mbox{}{<\textbf{respStmt}>}\mbox{}\newline 
\hspace*{6pt}{<\textbf{resp}>}converted to XML encoding{</\textbf{resp}>}\mbox{}\newline 
\hspace*{6pt}{<\textbf{name}>}Alan Morrison{</\textbf{name}>}\mbox{}\newline 
{</\textbf{respStmt}>}\end{shaded}\egroup 


    \item[{Content model}]
  \mbox{}\hfill\\[-10pt]\begin{Verbatim}[fontsize=\small]
<content>
 <alternate>
  <sequence>
   <elementRef key="resp"
    maxOccurs="unbounded" minOccurs="1"/>
   <classRef key="model.nameLike.agent"
    maxOccurs="unbounded" minOccurs="1"/>
  </sequence>
  <sequence>
   <classRef key="model.nameLike.agent"
    maxOccurs="unbounded" minOccurs="1"/>
   <elementRef key="resp"
    maxOccurs="unbounded" minOccurs="1"/>
  </sequence>
 </alternate>
</content>
    
\end{Verbatim}

    \item[{Schema Declaration}]
  \mbox{}\hfill\\[-10pt]\begin{Verbatim}[fontsize=\small]
element respStmt
{
   att.global.attributes,
   att.canonical.attributes,
   ( ( resp+, model.nameLike.agent+ ) | ( model.nameLike.agent+, resp+ ) )
}
\end{Verbatim}

\end{reflist}  \index{restore=<restore>|oddindex}
\begin{reflist}
\item[]\begin{specHead}{TEI.restore}{<restore> }indicates restoration of text to an earlier state by cancellation of an editorial or authorial marking or instruction. [\xref{http://www.tei-c.org/release/doc/tei-p5-doc/en/html/PH.html\#PHCD}{11.3.1.6. Cancellation of Deletions and Other Markings}]\end{specHead} 
    \item[{Module}]
  transcr
    \item[{Attributes}]
  Attributes att.global (\textit{@xml:id}, \textit{@n}, \textit{@xml:lang}, \textit{@xml:base}, \textit{@xml:space})  (att.global.rendition (\textit{@rend}, \textit{@style}, \textit{@rendition})) (att.global.linking (\textit{@corresp}, \textit{@synch}, \textit{@sameAs}, \textit{@copyOf}, \textit{@next}, \textit{@prev}, \textit{@exclude}, \textit{@select})) (att.global.analytic (\textit{@ana})) (att.global.facs (\textit{@facs})) (att.global.change (\textit{@change})) (att.global.responsibility (\textit{@cert}, \textit{@resp})) (att.global.source (\textit{@source})) att.transcriptional (\textit{@status}, \textit{@cause}, \textit{@seq})  (att.editLike (\textit{@evidence}, \textit{@instant}) (att.dimensions (\textit{@unit}, \textit{@quantity}, \textit{@extent}, \textit{@precision}, \textit{@scope}) (att.ranging (\textit{@atLeast}, \textit{@atMost}, \textit{@min}, \textit{@max}, \textit{@confidence})) ) ) (att.written (\textit{@hand})) att.typed (\textit{@type}, \textit{@subtype}) 
    \item[{Member of}]
  model.linePart model.pPart.transcriptional
    \item[{Contained by}]
  
    \item[analysis: ]
   cl pc phr s w\par 
    \item[core: ]
   abbr add addrLine author bibl biblScope citedRange corr date del distinct editor email emph expan foreign gloss head headItem headLabel hi item l label measure mentioned name note num orig p pubPlace publisher q quote ref reg rs said sic soCalled speaker stage street term textLang time title unclear\par 
    \item[figures: ]
   cell\par 
    \item[header: ]
   change distributor edition extent geoDecl handNote licence scriptNote typeNote\par 
    \item[linking: ]
   ab seg\par 
    \item[msdescription: ]
   accMat acquisition additions catchwords collation colophon condition custEvent decoNote explicit filiation finalRubric foliation heraldry incipit layout material musicNotation objectType origDate origPlace origin provenance rubric secFol signatures source stamp summary support surrogates watermark\par 
    \item[namesdates: ]
   addName affiliation birth bloc country death district education faith floruit forename genName geogFeat geogName nameLink nationality occupation offset orgName persName placeName region residence roleName settlement sex socecStatus surname\par 
    \item[textcrit: ]
   lem rdg wit witDetail\par 
    \item[textstructure: ]
   byline closer dateline docAuthor docDate docEdition docImprint imprimatur opener salute signed titlePart trailer\par 
    \item[transcr: ]
   am damage fw line metamark mod restore retrace secl supplied surplus zone
    \item[{May contain}]
  
    \item[analysis: ]
   c cl interp interpGrp m pc phr s span spanGrp w\par 
    \item[core: ]
   abbr add address bibl biblStruct cb choice cit corr date del desc distinct email emph expan foreign gap gb gloss graphic hi index l label lb lg list listBibl measure measureGrp media mentioned milestone name note num orig pb ptr q quote ref reg rs said sic soCalled stage term time title unclear\par 
    \item[figures: ]
   figure formula notatedMusic table\par 
    \item[gaiji: ]
   g\par 
    \item[header: ]
   biblFull idno\par 
    \item[linking: ]
   alt altGrp anchor join joinGrp link linkGrp seg timeline\par 
    \item[msdescription: ]
   catchwords depth dim dimensions height heraldry locus locusGrp material msDesc objectType origDate origPlace secFol signatures stamp watermark width\par 
    \item[namesdates: ]
   addName affiliation bloc climate country district forename genName geo geogFeat geogName listEvent listNym listOrg listPerson listPlace location nameLink offset orgName persName placeName population region roleName settlement state surname terrain trait\par 
    \item[textcrit: ]
   app listApp listWit witDetail\par 
    \item[textstructure: ]
   floatingText\par 
    \item[transcr: ]
   addSpan am damage damageSpan delSpan ex fw handShift listTranspose metamark mod redo restore retrace secl space subst substJoin supplied surplus undo\par character data
    \item[{Note}]
  \par
On this element, the {\itshape type} attribute categorizes the way that the cancelled intervention has been indicated in some way, for example by means of a marginal note, over-inking, additional markup, etc.
    \item[{Example}]
  \leavevmode\bgroup\exampleFont \begin{shaded}\noindent\mbox{}For I hate this\mbox{}\newline 
{<\textbf{restore}\hspace*{6pt}{hand}="{\#dhl}"\mbox{}\newline 
\hspace*{6pt}{type}="{marginalStetNote}">}\mbox{}\newline 
\hspace*{6pt}{<\textbf{del}>}my{</\textbf{del}>}\mbox{}\newline 
{</\textbf{restore}>} body \end{shaded}\egroup 


    \item[{Content model}]
  \mbox{}\hfill\\[-10pt]\begin{Verbatim}[fontsize=\small]
<content>
 <macroRef key="macro.paraContent"/>
</content>
    
\end{Verbatim}

    \item[{Schema Declaration}]
  \mbox{}\hfill\\[-10pt]\begin{Verbatim}[fontsize=\small]
element restore
{
   att.global.attributes,
   att.transcriptional.attributes,
   att.typed.attributes,
   macro.paraContent}
\end{Verbatim}

\end{reflist}  \index{retrace=<retrace>|oddindex}
\begin{reflist}
\item[]\begin{specHead}{TEI.retrace}{<retrace> }contains a sequence of writing which has been retraced, for example by over-inking, to clarify or fix it. [\xref{http://www.tei-c.org/release/doc/tei-p5-doc/en/html/PH.html\#PH-fix}{11.3.4.3. Fixation and Clarification}]\end{specHead} 
    \item[{Module}]
  transcr
    \item[{Attributes}]
  Attributes att.global (\textit{@xml:id}, \textit{@n}, \textit{@xml:lang}, \textit{@xml:base}, \textit{@xml:space})  (att.global.rendition (\textit{@rend}, \textit{@style}, \textit{@rendition})) (att.global.linking (\textit{@corresp}, \textit{@synch}, \textit{@sameAs}, \textit{@copyOf}, \textit{@next}, \textit{@prev}, \textit{@exclude}, \textit{@select})) (att.global.analytic (\textit{@ana})) (att.global.facs (\textit{@facs})) (att.global.change (\textit{@change})) (att.global.responsibility (\textit{@cert}, \textit{@resp})) (att.global.source (\textit{@source})) att.spanning (\textit{@spanTo}) att.transcriptional (\textit{@status}, \textit{@cause}, \textit{@seq})  (att.editLike (\textit{@evidence}, \textit{@instant}) (att.dimensions (\textit{@unit}, \textit{@quantity}, \textit{@extent}, \textit{@precision}, \textit{@scope}) (att.ranging (\textit{@atLeast}, \textit{@atMost}, \textit{@min}, \textit{@max}, \textit{@confidence})) ) ) (att.written (\textit{@hand}))
    \item[{Member of}]
  model.linePart model.pPart.transcriptional
    \item[{Contained by}]
  
    \item[analysis: ]
   cl pc phr s w\par 
    \item[core: ]
   abbr add addrLine author bibl biblScope citedRange corr date del distinct editor email emph expan foreign gloss head headItem headLabel hi item l label measure mentioned name note num orig p pubPlace publisher q quote ref reg rs said sic soCalled speaker stage street term textLang time title unclear\par 
    \item[figures: ]
   cell\par 
    \item[header: ]
   change distributor edition extent geoDecl handNote licence scriptNote typeNote\par 
    \item[linking: ]
   ab seg\par 
    \item[msdescription: ]
   accMat acquisition additions catchwords collation colophon condition custEvent decoNote explicit filiation finalRubric foliation heraldry incipit layout material musicNotation objectType origDate origPlace origin provenance rubric secFol signatures source stamp summary support surrogates watermark\par 
    \item[namesdates: ]
   addName affiliation birth bloc country death district education faith floruit forename genName geogFeat geogName nameLink nationality occupation offset orgName persName placeName region residence roleName settlement sex socecStatus surname\par 
    \item[textcrit: ]
   lem rdg wit witDetail\par 
    \item[textstructure: ]
   byline closer dateline docAuthor docDate docEdition docImprint imprimatur opener salute signed titlePart trailer\par 
    \item[transcr: ]
   am damage fw line metamark mod restore retrace secl supplied surplus zone
    \item[{May contain}]
  
    \item[analysis: ]
   c cl interp interpGrp m pc phr s span spanGrp w\par 
    \item[core: ]
   abbr add address bibl biblStruct cb choice cit corr date del desc distinct email emph expan foreign gap gb gloss graphic hi index l label lb lg list listBibl measure measureGrp media mentioned milestone name note num orig pb ptr q quote ref reg rs said sic soCalled stage term time title unclear\par 
    \item[figures: ]
   figure formula notatedMusic table\par 
    \item[gaiji: ]
   g\par 
    \item[header: ]
   biblFull idno\par 
    \item[linking: ]
   alt altGrp anchor join joinGrp link linkGrp seg timeline\par 
    \item[msdescription: ]
   catchwords depth dim dimensions height heraldry locus locusGrp material msDesc objectType origDate origPlace secFol signatures stamp watermark width\par 
    \item[namesdates: ]
   addName affiliation bloc climate country district forename genName geo geogFeat geogName listEvent listNym listOrg listPerson listPlace location nameLink offset orgName persName placeName population region roleName settlement state surname terrain trait\par 
    \item[textcrit: ]
   app listApp listWit witDetail\par 
    \item[textstructure: ]
   floatingText\par 
    \item[transcr: ]
   addSpan am damage damageSpan delSpan ex fw handShift listTranspose metamark mod redo restore retrace secl space subst substJoin supplied surplus undo\par character data
    \item[{Note}]
  \par
Multiple retraces are indicated by nesting one <retrace> within another. In principle, a retrace differs from a substitution in that second and subsequent rewrites do not materially alter the content of an element. Where minor changes have been made during the retracing action however these may be marked up using <del>, <add>, etc. with an appropriate value for the {\itshape change} attribute.
    \item[{Content model}]
  \mbox{}\hfill\\[-10pt]\begin{Verbatim}[fontsize=\small]
<content>
 <macroRef key="macro.paraContent"/>
</content>
    
\end{Verbatim}

    \item[{Schema Declaration}]
  \mbox{}\hfill\\[-10pt]\begin{Verbatim}[fontsize=\small]
element retrace
{
   att.global.attributes,
   att.spanning.attributes,
   att.transcriptional.attributes,
   macro.paraContent}
\end{Verbatim}

\end{reflist}  \index{revisionDesc=<revisionDesc>|oddindex}
\begin{reflist}
\item[]\begin{specHead}{TEI.revisionDesc}{<revisionDesc> }(revision description) summarizes the revision history for a file. [\xref{http://www.tei-c.org/release/doc/tei-p5-doc/en/html/HD.html\#HD6}{2.6. The Revision Description} \xref{http://www.tei-c.org/release/doc/tei-p5-doc/en/html/HD.html\#HD11}{2.1.1. The TEI Header and Its Components}]\end{specHead} 
    \item[{Module}]
  header
    \item[{Attributes}]
  Attributes att.global (\textit{@xml:id}, \textit{@n}, \textit{@xml:lang}, \textit{@xml:base}, \textit{@xml:space})  (att.global.rendition (\textit{@rend}, \textit{@style}, \textit{@rendition})) (att.global.linking (\textit{@corresp}, \textit{@synch}, \textit{@sameAs}, \textit{@copyOf}, \textit{@next}, \textit{@prev}, \textit{@exclude}, \textit{@select})) (att.global.analytic (\textit{@ana})) (att.global.facs (\textit{@facs})) (att.global.change (\textit{@change})) (att.global.responsibility (\textit{@cert}, \textit{@resp})) (att.global.source (\textit{@source})) att.docStatus (\textit{@status}) 
    \item[{Contained by}]
  
    \item[header: ]
   teiHeader
    \item[{May contain}]
  
    \item[core: ]
   list\par 
    \item[header: ]
   change listChange
    \item[{Note}]
  \par
If present on this element, the {\itshape status} attribute should indicate the current status of the document. The same attribute may appear on any <change> to record the status at the time of that change. Conventionally <change> elements should be given in reverse date order, with the most recent change at the start of the list.
    \item[{Example}]
  \leavevmode\bgroup\exampleFont \begin{shaded}\noindent\mbox{}{<\textbf{revisionDesc}\hspace*{6pt}{status}="{embargoed}">}\mbox{}\newline 
\hspace*{6pt}{<\textbf{change}\hspace*{6pt}{when}="{1991-11-11}"\hspace*{6pt}{who}="{\#LB}">} deleted chapter 10 {</\textbf{change}>}\mbox{}\newline 
{</\textbf{revisionDesc}>}\end{shaded}\egroup 


    \item[{Content model}]
  \mbox{}\hfill\\[-10pt]\begin{Verbatim}[fontsize=\small]
<content>
 <alternate>
  <elementRef key="list"/>
  <elementRef key="listChange"/>
  <elementRef key="change"
   maxOccurs="unbounded" minOccurs="1"/>
 </alternate>
</content>
    
\end{Verbatim}

    \item[{Schema Declaration}]
  \mbox{}\hfill\\[-10pt]\begin{Verbatim}[fontsize=\small]
element revisionDesc
{
   att.global.attributes,
   att.docStatus.attributes,
   ( list | listChange | change+ )
}
\end{Verbatim}

\end{reflist}  \index{roleName=<roleName>|oddindex}
\begin{reflist}
\item[]\begin{specHead}{TEI.roleName}{<roleName> }contains a name component which indicates that the referent has a particular role or position in society, such as an official title or rank. [\xref{http://www.tei-c.org/release/doc/tei-p5-doc/en/html/ND.html\#NDPER}{13.2.1. Personal Names}]\end{specHead} 
    \item[{Module}]
  namesdates
    \item[{Attributes}]
  Attributes att.global (\textit{@xml:id}, \textit{@n}, \textit{@xml:lang}, \textit{@xml:base}, \textit{@xml:space})  (att.global.rendition (\textit{@rend}, \textit{@style}, \textit{@rendition})) (att.global.linking (\textit{@corresp}, \textit{@synch}, \textit{@sameAs}, \textit{@copyOf}, \textit{@next}, \textit{@prev}, \textit{@exclude}, \textit{@select})) (att.global.analytic (\textit{@ana})) (att.global.facs (\textit{@facs})) (att.global.change (\textit{@change})) (att.global.responsibility (\textit{@cert}, \textit{@resp})) (att.global.source (\textit{@source})) att.personal (\textit{@full}, \textit{@sort})  (att.naming (\textit{@role}, \textit{@nymRef}) (att.canonical (\textit{@key}, \textit{@ref})) ) att.typed (\textit{@type}, \textit{@subtype}) 
    \item[{Member of}]
  model.persNamePart
    \item[{Contained by}]
  
    \item[analysis: ]
   cl phr s span\par 
    \item[core: ]
   abbr add addrLine address author bibl biblScope citedRange corr date del desc distinct editor email emph expan foreign gloss head headItem headLabel hi item l label measure meeting mentioned name note num orig p pubPlace publisher q quote ref reg resp rs said sic soCalled speaker stage street term textLang time title unclear\par 
    \item[figures: ]
   cell figDesc\par 
    \item[header: ]
   authority catDesc change classCode correspAction creation distributor edition extent funder geoDecl handNote language licence principal rendition scriptNote sponsor tagUsage typeNote\par 
    \item[linking: ]
   ab seg\par 
    \item[msdescription: ]
   accMat acquisition additions catchwords collation colophon condition custEvent decoNote explicit filiation finalRubric foliation heraldry incipit layout material musicNotation objectType origDate origPlace origin provenance rubric secFol signatures source stamp summary support surrogates watermark\par 
    \item[namesdates: ]
   addName affiliation age birth bloc country death district education faith floruit forename genName geogFeat geogName langKnown nameLink nationality occupation offset org orgName persName placeName region residence roleName settlement sex socecStatus surname\par 
    \item[textcrit: ]
   lem rdg wit witDetail witness\par 
    \item[textstructure: ]
   byline closer dateline docAuthor docDate docEdition docImprint imprimatur opener salute signed titlePart trailer\par 
    \item[transcr: ]
   damage fw metamark mod restore retrace secl supplied surplus
    \item[{May contain}]
  
    \item[analysis: ]
   c cl interp interpGrp m pc phr s span spanGrp w\par 
    \item[core: ]
   abbr add address cb choice corr date del distinct email emph expan foreign gap gb gloss graphic hi index lb measure measureGrp media mentioned milestone name note num orig pb ptr ref reg rs sic soCalled term time title unclear\par 
    \item[figures: ]
   figure formula notatedMusic\par 
    \item[gaiji: ]
   g\par 
    \item[header: ]
   idno\par 
    \item[linking: ]
   alt altGrp anchor join joinGrp link linkGrp seg timeline\par 
    \item[msdescription: ]
   catchwords depth dim dimensions height heraldry locus locusGrp material objectType origDate origPlace secFol signatures stamp watermark width\par 
    \item[namesdates: ]
   addName affiliation bloc climate country district forename genName geo geogFeat geogName location nameLink offset orgName persName placeName population region roleName settlement state surname terrain trait\par 
    \item[textcrit: ]
   app witDetail\par 
    \item[transcr: ]
   addSpan am damage damageSpan delSpan ex fw handShift listTranspose metamark mod redo restore retrace secl space subst substJoin supplied surplus undo\par character data
    \item[{Note}]
  \par
A <roleName> may be distinguished from an <addName> by virtue of the fact that, like a title, it typically exists independently of its holder.
    \item[{Example}]
  \leavevmode\bgroup\exampleFont \begin{shaded}\noindent\mbox{}{<\textbf{persName}>}\mbox{}\newline 
\hspace*{6pt}{<\textbf{forename}>}William{</\textbf{forename}>}\mbox{}\newline 
\hspace*{6pt}{<\textbf{surname}>}Poulteny{</\textbf{surname}>}\mbox{}\newline 
\hspace*{6pt}{<\textbf{roleName}>}Earl of Bath{</\textbf{roleName}>}\mbox{}\newline 
{</\textbf{persName}>}\end{shaded}\egroup 


    \item[{Content model}]
  \mbox{}\hfill\\[-10pt]\begin{Verbatim}[fontsize=\small]
<content>
 <macroRef key="macro.phraseSeq"/>
</content>
    
\end{Verbatim}

    \item[{Schema Declaration}]
  \mbox{}\hfill\\[-10pt]\begin{Verbatim}[fontsize=\small]
element roleName
{
   att.global.attributes,
   att.personal.attributes,
   att.typed.attributes,
   macro.phraseSeq}
\end{Verbatim}

\end{reflist}  \index{row=<row>|oddindex}
\begin{reflist}
\item[]\begin{specHead}{TEI.row}{<row> }contains one row of a table. [\xref{http://www.tei-c.org/release/doc/tei-p5-doc/en/html/FT.html\#FTTAB1}{14.1.1. TEI Tables}]\end{specHead} 
    \item[{Module}]
  figures
    \item[{Attributes}]
  Attributes att.global (\textit{@xml:id}, \textit{@n}, \textit{@xml:lang}, \textit{@xml:base}, \textit{@xml:space})  (att.global.rendition (\textit{@rend}, \textit{@style}, \textit{@rendition})) (att.global.linking (\textit{@corresp}, \textit{@synch}, \textit{@sameAs}, \textit{@copyOf}, \textit{@next}, \textit{@prev}, \textit{@exclude}, \textit{@select})) (att.global.analytic (\textit{@ana})) (att.global.facs (\textit{@facs})) (att.global.change (\textit{@change})) (att.global.responsibility (\textit{@cert}, \textit{@resp})) (att.global.source (\textit{@source})) att.tableDecoration (\textit{@role}, \textit{@rows}, \textit{@cols}) 
    \item[{Contained by}]
  
    \item[figures: ]
   table
    \item[{May contain}]
  
    \item[figures: ]
   cell
    \item[{Example}]
  \leavevmode\bgroup\exampleFont \begin{shaded}\noindent\mbox{}{<\textbf{row}\hspace*{6pt}{role}="{data}">}\mbox{}\newline 
\hspace*{6pt}{<\textbf{cell}\hspace*{6pt}{role}="{label}">}Classics{</\textbf{cell}>}\mbox{}\newline 
\hspace*{6pt}{<\textbf{cell}>}Idle listless and unimproving{</\textbf{cell}>}\mbox{}\newline 
{</\textbf{row}>}\end{shaded}\egroup 


    \item[{Content model}]
  \mbox{}\hfill\\[-10pt]\begin{Verbatim}[fontsize=\small]
<content>
 <elementRef key="cell"
  maxOccurs="unbounded" minOccurs="1"/>
</content>
    
\end{Verbatim}

    \item[{Schema Declaration}]
  \mbox{}\hfill\\[-10pt]\begin{Verbatim}[fontsize=\small]
element row { att.global.attributes, att.tableDecoration.attributes, cell+ }
\end{Verbatim}

\end{reflist}  \index{rs=<rs>|oddindex}
\begin{reflist}
\item[]\begin{specHead}{TEI.rs}{<rs> }(referencing string) contains a general purpose name or referring string. [\xref{http://www.tei-c.org/release/doc/tei-p5-doc/en/html/ND.html\#NDPER}{13.2.1. Personal Names} \xref{http://www.tei-c.org/release/doc/tei-p5-doc/en/html/CO.html\#CONARS}{3.5.1. Referring Strings}]\end{specHead} 
    \item[{Module}]
  core
    \item[{Attributes}]
  Attributes att.global (\textit{@xml:id}, \textit{@n}, \textit{@xml:lang}, \textit{@xml:base}, \textit{@xml:space})  (att.global.rendition (\textit{@rend}, \textit{@style}, \textit{@rendition})) (att.global.linking (\textit{@corresp}, \textit{@synch}, \textit{@sameAs}, \textit{@copyOf}, \textit{@next}, \textit{@prev}, \textit{@exclude}, \textit{@select})) (att.global.analytic (\textit{@ana})) (att.global.facs (\textit{@facs})) (att.global.change (\textit{@change})) (att.global.responsibility (\textit{@cert}, \textit{@resp})) (att.global.source (\textit{@source})) att.naming (\textit{@role}, \textit{@nymRef})  (att.canonical (\textit{@key}, \textit{@ref})) att.typed (\textit{@type}, \textit{@subtype}) 
    \item[{Member of}]
  model.nameLike
    \item[{Contained by}]
  
    \item[analysis: ]
   cl phr s span\par 
    \item[core: ]
   abbr add addrLine address author bibl biblScope citedRange corr date del desc distinct editor email emph expan foreign gloss head headItem headLabel hi item l label measure meeting mentioned name note num orig p pubPlace publisher q quote ref reg resp rs said sic soCalled speaker stage street term textLang time title unclear\par 
    \item[figures: ]
   cell figDesc\par 
    \item[header: ]
   authority catDesc change classCode correspAction creation distributor edition extent funder geoDecl handNote language licence principal rendition scriptNote sponsor tagUsage typeNote\par 
    \item[linking: ]
   ab seg\par 
    \item[msdescription: ]
   accMat acquisition additions catchwords collation colophon condition custEvent decoNote explicit filiation finalRubric foliation heraldry incipit layout material musicNotation objectType origDate origPlace origin provenance rubric secFol signatures source stamp summary support surrogates watermark\par 
    \item[namesdates: ]
   addName affiliation age birth bloc country death district education faith floruit forename genName geogFeat geogName langKnown nameLink nationality occupation offset org orgName persName placeName region residence roleName settlement sex socecStatus surname\par 
    \item[textcrit: ]
   lem rdg wit witDetail witness\par 
    \item[textstructure: ]
   byline closer dateline docAuthor docDate docEdition docImprint imprimatur opener salute signed titlePart trailer\par 
    \item[transcr: ]
   damage fw metamark mod restore retrace secl supplied surplus
    \item[{May contain}]
  
    \item[analysis: ]
   c cl interp interpGrp m pc phr s span spanGrp w\par 
    \item[core: ]
   abbr add address cb choice corr date del distinct email emph expan foreign gap gb gloss graphic hi index lb measure measureGrp media mentioned milestone name note num orig pb ptr ref reg rs sic soCalled term time title unclear\par 
    \item[figures: ]
   figure formula notatedMusic\par 
    \item[gaiji: ]
   g\par 
    \item[header: ]
   idno\par 
    \item[linking: ]
   alt altGrp anchor join joinGrp link linkGrp seg timeline\par 
    \item[msdescription: ]
   catchwords depth dim dimensions height heraldry locus locusGrp material objectType origDate origPlace secFol signatures stamp watermark width\par 
    \item[namesdates: ]
   addName affiliation bloc climate country district forename genName geo geogFeat geogName location nameLink offset orgName persName placeName population region roleName settlement state surname terrain trait\par 
    \item[textcrit: ]
   app witDetail\par 
    \item[transcr: ]
   addSpan am damage damageSpan delSpan ex fw handShift listTranspose metamark mod redo restore retrace secl space subst substJoin supplied surplus undo\par character data
    \item[{Example}]
  \leavevmode\bgroup\exampleFont \begin{shaded}\noindent\mbox{}{<\textbf{q}>}My dear {<\textbf{rs}\hspace*{6pt}{type}="{person}">}Mr. Bennet{</\textbf{rs}>}, {</\textbf{q}>} said {<\textbf{rs}\hspace*{6pt}{type}="{person}">}his lady{</\textbf{rs}>}\mbox{}\newline 
 to him one day, \mbox{}\newline 
{<\textbf{q}>}have you heard that {<\textbf{rs}\hspace*{6pt}{type}="{place}">}Netherfield Park{</\textbf{rs}>} is let at\mbox{}\newline 
 last?{</\textbf{q}>}\end{shaded}\egroup 


    \item[{Content model}]
  \mbox{}\hfill\\[-10pt]\begin{Verbatim}[fontsize=\small]
<content>
 <macroRef key="macro.phraseSeq"/>
</content>
    
\end{Verbatim}

    \item[{Schema Declaration}]
  \mbox{}\hfill\\[-10pt]\begin{Verbatim}[fontsize=\small]
element rs
{
   att.global.attributes,
   att.naming.attributes,
   att.typed.attributes,
   macro.phraseSeq}
\end{Verbatim}

\end{reflist}  \index{rubric=<rubric>|oddindex}
\begin{reflist}
\item[]\begin{specHead}{TEI.rubric}{<rubric> }contains the text of any \textit{rubric} or heading attached to a particular manuscript item, that is, a string of words through which a manuscript signals the beginning of a text division, often with an assertion as to its author and title, which is in some way set off from the text itself, usually in red ink, or by use of different size or type of script, or some other such visual device. [\xref{http://www.tei-c.org/release/doc/tei-p5-doc/en/html/MS.html\#mscoit}{10.6.1. The msItem and msItemStruct Elements}]\end{specHead} 
    \item[{Module}]
  msdescription
    \item[{Attributes}]
  Attributes att.global (\textit{@xml:id}, \textit{@n}, \textit{@xml:lang}, \textit{@xml:base}, \textit{@xml:space})  (att.global.rendition (\textit{@rend}, \textit{@style}, \textit{@rendition})) (att.global.linking (\textit{@corresp}, \textit{@synch}, \textit{@sameAs}, \textit{@copyOf}, \textit{@next}, \textit{@prev}, \textit{@exclude}, \textit{@select})) (att.global.analytic (\textit{@ana})) (att.global.facs (\textit{@facs})) (att.global.change (\textit{@change})) (att.global.responsibility (\textit{@cert}, \textit{@resp})) (att.global.source (\textit{@source})) att.typed (\textit{@type}, \textit{@subtype}) 
    \item[{Member of}]
  model.msQuoteLike 
    \item[{Contained by}]
  
    \item[msdescription: ]
   msItem msItemStruct
    \item[{May contain}]
  
    \item[analysis: ]
   c cl interp interpGrp m pc phr s span spanGrp w\par 
    \item[core: ]
   abbr add address cb choice corr date del distinct email emph expan foreign gap gb gloss graphic hi index lb measure measureGrp media mentioned milestone name note num orig pb ptr ref reg rs sic soCalled term time title unclear\par 
    \item[figures: ]
   figure formula notatedMusic\par 
    \item[gaiji: ]
   g\par 
    \item[header: ]
   idno\par 
    \item[linking: ]
   alt altGrp anchor join joinGrp link linkGrp seg timeline\par 
    \item[msdescription: ]
   catchwords depth dim dimensions height heraldry locus locusGrp material objectType origDate origPlace secFol signatures stamp watermark width\par 
    \item[namesdates: ]
   addName affiliation bloc climate country district forename genName geo geogFeat geogName location nameLink offset orgName persName placeName population region roleName settlement state surname terrain trait\par 
    \item[textcrit: ]
   app witDetail\par 
    \item[transcr: ]
   addSpan am damage damageSpan delSpan ex fw handShift listTranspose metamark mod redo restore retrace secl space subst substJoin supplied surplus undo\par character data
    \item[{Example}]
  \leavevmode\bgroup\exampleFont \begin{shaded}\noindent\mbox{}{<\textbf{rubric}>}Nu koma Skyckiu Rym{<\textbf{ex}>}ur{</\textbf{ex}>}.{</\textbf{rubric}>}\mbox{}\newline 
{<\textbf{rubric}>}Incipit liber de consciencia humana a beatissimo Bernardo editus.{</\textbf{rubric}>}\mbox{}\newline 
{<\textbf{rubric}>}\mbox{}\newline 
\hspace*{6pt}{<\textbf{locus}>}16. f. 28v in margin: {</\textbf{locus}>}Dicta Cassiodori\mbox{}\newline 
{</\textbf{rubric}>}\end{shaded}\egroup 


    \item[{Content model}]
  \mbox{}\hfill\\[-10pt]\begin{Verbatim}[fontsize=\small]
<content>
 <macroRef key="macro.phraseSeq"/>
</content>
    
\end{Verbatim}

    \item[{Schema Declaration}]
  \mbox{}\hfill\\[-10pt]\begin{Verbatim}[fontsize=\small]
element rubric { att.global.attributes, att.typed.attributes, macro.phraseSeq }
\end{Verbatim}

\end{reflist}  \index{s=<s>|oddindex}
\begin{reflist}
\item[]\begin{specHead}{TEI.s}{<s> }(s-unit) contains a sentence-like division of a text. [\xref{http://www.tei-c.org/release/doc/tei-p5-doc/en/html/AI.html\#AILC}{17.1. Linguistic Segment Categories} \xref{http://www.tei-c.org/release/doc/tei-p5-doc/en/html/TS.html\#TSSASE}{8.4.1. Segmentation}]\end{specHead} 
    \item[{Module}]
  analysis
    \item[{Attributes}]
  Attributes att.global (\textit{@xml:id}, \textit{@n}, \textit{@xml:lang}, \textit{@xml:base}, \textit{@xml:space})  (att.global.rendition (\textit{@rend}, \textit{@style}, \textit{@rendition})) (att.global.linking (\textit{@corresp}, \textit{@synch}, \textit{@sameAs}, \textit{@copyOf}, \textit{@next}, \textit{@prev}, \textit{@exclude}, \textit{@select})) (att.global.analytic (\textit{@ana})) (att.global.facs (\textit{@facs})) (att.global.change (\textit{@change})) (att.global.responsibility (\textit{@cert}, \textit{@resp})) (att.global.source (\textit{@source})) att.segLike (\textit{@function})  (att.datcat (\textit{@datcat}, \textit{@valueDatcat})) (att.fragmentable (\textit{@part})) att.typed (\textit{@type}, \textit{@subtype}) 
    \item[{Member of}]
  model.segLike
    \item[{Contained by}]
  
    \item[analysis: ]
   cl phr s\par 
    \item[core: ]
   abbr add addrLine author bibl biblScope citedRange corr date del distinct editor email emph expan foreign gloss head headItem headLabel hi item l label measure mentioned name note num orig p pubPlace publisher q quote ref reg rs said sic soCalled speaker stage street term textLang time title unclear\par 
    \item[figures: ]
   cell\par 
    \item[header: ]
   change distributor edition extent geoDecl handNote licence scriptNote typeNote\par 
    \item[linking: ]
   ab seg\par 
    \item[msdescription: ]
   accMat acquisition additions catchwords collation colophon condition custEvent decoNote explicit filiation finalRubric foliation heraldry incipit layout material musicNotation objectType origDate origPlace origin provenance rubric secFol signatures source stamp summary support surrogates watermark\par 
    \item[namesdates: ]
   addName affiliation birth bloc country death district education faith floruit forename genName geogFeat geogName nameLink nationality occupation offset orgName persName placeName region residence roleName settlement sex socecStatus surname\par 
    \item[textcrit: ]
   lem rdg wit witDetail\par 
    \item[textstructure: ]
   byline closer dateline docAuthor docDate docEdition docImprint imprimatur opener salute signed titlePart trailer\par 
    \item[transcr: ]
   damage fw metamark mod restore retrace secl supplied surplus
    \item[{May contain}]
  
    \item[analysis: ]
   c cl interp interpGrp m pc phr s span spanGrp w\par 
    \item[core: ]
   abbr add address cb choice corr date del distinct email emph expan foreign gap gb gloss graphic hi index lb measure measureGrp media mentioned milestone name note num orig pb ptr ref reg rs sic soCalled term time title unclear\par 
    \item[figures: ]
   figure formula notatedMusic\par 
    \item[gaiji: ]
   g\par 
    \item[header: ]
   idno\par 
    \item[linking: ]
   alt altGrp anchor join joinGrp link linkGrp seg timeline\par 
    \item[msdescription: ]
   catchwords depth dim dimensions height heraldry locus locusGrp material objectType origDate origPlace secFol signatures stamp watermark width\par 
    \item[namesdates: ]
   addName affiliation bloc climate country district forename genName geo geogFeat geogName location nameLink offset orgName persName placeName population region roleName settlement state surname terrain trait\par 
    \item[textcrit: ]
   app witDetail\par 
    \item[transcr: ]
   addSpan am damage damageSpan delSpan ex fw handShift listTranspose metamark mod redo restore retrace secl space subst substJoin supplied surplus undo\par character data
    \item[{Note}]
  \par
The <s> element may be used to mark orthographic sentences, or any other segmentation of a text, provided that the segmentation is end-to-end, complete, and non-nesting. For segmentation which is partial or recursive, the <seg> should be used instead.\par
The {\itshape type} attribute may be used to indicate the type of segmentation intended, according to any convenient typology.
    \item[{Example}]
  \leavevmode\bgroup\exampleFont \begin{shaded}\noindent\mbox{}{<\textbf{head}>}\mbox{}\newline 
\hspace*{6pt}{<\textbf{s}>}A short affair{</\textbf{s}>}\mbox{}\newline 
{</\textbf{head}>}\mbox{}\newline 
{<\textbf{s}>}When are you leaving?{</\textbf{s}>}\mbox{}\newline 
{<\textbf{s}>}Tomorrow.{</\textbf{s}>}\end{shaded}\egroup 


    \item[{Schematron}]
   <s:report test="tei:s">You may not nest one s element within  another: use seg instead</s:report>
    \item[{Content model}]
  \mbox{}\hfill\\[-10pt]\begin{Verbatim}[fontsize=\small]
<content>
 <macroRef key="macro.phraseSeq"/>
</content>
    
\end{Verbatim}

    \item[{Schema Declaration}]
  \mbox{}\hfill\\[-10pt]\begin{Verbatim}[fontsize=\small]
element s
{
   att.global.attributes,
   att.segLike.attributes,
   att.typed.attributes,
   macro.phraseSeq}
\end{Verbatim}

\end{reflist}  \index{said=<said>|oddindex}\index{aloud=@aloud!<said>|oddindex}\index{direct=@direct!<said>|oddindex}
\begin{reflist}
\item[]\begin{specHead}{TEI.said}{<said> }(speech or thought) indicates passages thought or spoken aloud, whether explicitly indicated in the source or not, whether directly or indirectly reported, whether by real people or fictional characters. [\xref{http://www.tei-c.org/release/doc/tei-p5-doc/en/html/CO.html\#COHQQ}{3.3.3. Quotation}]\end{specHead} 
    \item[{Module}]
  core
    \item[{Attributes}]
  Attributes att.global (\textit{@xml:id}, \textit{@n}, \textit{@xml:lang}, \textit{@xml:base}, \textit{@xml:space})  (att.global.rendition (\textit{@rend}, \textit{@style}, \textit{@rendition})) (att.global.linking (\textit{@corresp}, \textit{@synch}, \textit{@sameAs}, \textit{@copyOf}, \textit{@next}, \textit{@prev}, \textit{@exclude}, \textit{@select})) (att.global.analytic (\textit{@ana})) (att.global.facs (\textit{@facs})) (att.global.change (\textit{@change})) (att.global.responsibility (\textit{@cert}, \textit{@resp})) (att.global.source (\textit{@source})) att.ascribed (\textit{@who}) \hfil\\[-10pt]\begin{sansreflist}
    \item[@aloud]
  may be used to indicate whether the quoted matter is regarded as having been vocalized or signed.
\begin{reflist}
    \item[{Status}]
  Optional
    \item[{Datatype}]
  teidata.xTruthValue
    \item[{Default}]
  unknown \par \begin{tabular}{P{0.4969230769230769\textwidth}P{0.35307692307692307\textwidth}}
\xref{http://www.tei-c.org/Activities/Council/Working/tcw27.xml}{Deprecated}\tabcellsep The value will no longer be a default after 2017-09-05\end{tabular}
    \item[]\exampleFont {<\textbf{p}>} Celia thought privately, {<\textbf{said}\hspace*{6pt}{aloud}="{false}">}Dorothea quite despises Sir James Chettam;\mbox{}\newline 
\hspace*{6pt}\hspace*{6pt} I believe she would not accept him.{</\textbf{said}>} Celia felt that this was a pity.\mbox{}\newline 
\mbox{}\newline 
\textit{<!-- ... -->}\mbox{}\newline 
{</\textbf{p}>}
    \item[{Note}]
  \par
The value true indicates the encoded passage was expressed outwardly (whether spoken, signed, sung, screamed, chanted, etc.); the value false indicates that the encoded passage was thought, but not outwardly expressed.
\end{reflist}  
    \item[@direct]
  may be used to indicate whether the quoted matter is regarded as direct or indirect speech.
\begin{reflist}
    \item[{Status}]
  Optional
    \item[{Datatype}]
  teidata.xTruthValue
    \item[{Default}]
  true
    \item[]\exampleFont \mbox{}\newline 
\textit{<!-- in the header -->}{<\textbf{editorialDecl}>}\mbox{}\newline 
\hspace*{6pt}{<\textbf{quotation}\hspace*{6pt}{marks}="{none}"/>}\mbox{}\newline 
{</\textbf{editorialDecl}>}\mbox{}\newline 
\textit{<!-- ... -->}\mbox{}\newline 
{<\textbf{p}>} Tantripp had brought a card, and said that there was a gentleman waiting in the lobby.\mbox{}\newline 
 The courier had told him that {<\textbf{said}\hspace*{6pt}{direct}="{false}">}only Mrs. Casaubon was at home{</\textbf{said}>},\mbox{}\newline 
 but he said {<\textbf{said}\hspace*{6pt}{direct}="{false}">}he was a relation of Mr. Casaubon's: would she see him?{</\textbf{said}>}\mbox{}\newline 
{</\textbf{p}>}
    \item[{Note}]
  \par
The value true indicates the speech or thought is represented directly; the value false that speech or thought is represented indirectly, e.g. by use of a marked verbal aspect.
\end{reflist}  
\end{sansreflist}  
    \item[{Member of}]
  model.qLike
    \item[{Contained by}]
  
    \item[core: ]
   add cit corr del desc emph head hi item l meeting note orig p q quote ref reg said sic sp stage title unclear\par 
    \item[figures: ]
   cell figDesc figure\par 
    \item[header: ]
   change handNote licence rendition scriptNote tagUsage typeNote\par 
    \item[linking: ]
   ab seg\par 
    \item[msdescription: ]
   accMat acquisition additions collation condition custEvent decoNote filiation foliation layout musicNotation origin provenance signatures source summary support surrogates\par 
    \item[namesdates: ]
   occupation\par 
    \item[textcrit: ]
   lem rdg witness\par 
    \item[textstructure: ]
   argument body div docEdition epigraph imprimatur postscript salute signed titlePart trailer\par 
    \item[transcr: ]
   damage metamark mod restore retrace secl supplied surplus
    \item[{May contain}]
  
    \item[analysis: ]
   c cl interp interpGrp m pc phr s span spanGrp w\par 
    \item[core: ]
   abbr add address bibl biblStruct cb choice cit corr date del desc distinct email emph expan foreign gap gb gloss graphic hi index l label lb lg list listBibl measure measureGrp media mentioned milestone name note num orig p pb ptr q quote ref reg rs said sic soCalled sp stage term time title unclear\par 
    \item[figures: ]
   figure formula notatedMusic table\par 
    \item[gaiji: ]
   g\par 
    \item[header: ]
   biblFull idno\par 
    \item[linking: ]
   ab alt altGrp anchor join joinGrp link linkGrp seg timeline\par 
    \item[msdescription: ]
   catchwords depth dim dimensions height heraldry locus locusGrp material msDesc objectType origDate origPlace secFol signatures stamp watermark width\par 
    \item[namesdates: ]
   addName affiliation bloc climate country district forename genName geo geogFeat geogName listEvent listNym listOrg listPerson listPlace location nameLink offset orgName persName placeName population region roleName settlement state surname terrain trait\par 
    \item[textcrit: ]
   app listApp listWit witDetail\par 
    \item[textstructure: ]
   floatingText\par 
    \item[transcr: ]
   addSpan am damage damageSpan delSpan ex fw handShift listTranspose metamark mod redo restore retrace secl space subst substJoin supplied surplus undo\par character data
    \item[{Example}]
  \leavevmode\bgroup\exampleFont \begin{shaded}\noindent\mbox{}\mbox{}\newline 
\textit{<!-- in the header -->}{<\textbf{editorialDecl}>}\mbox{}\newline 
\hspace*{6pt}{<\textbf{quotation}\hspace*{6pt}{marks}="{all}"/>}\mbox{}\newline 
{</\textbf{editorialDecl}>}\mbox{}\newline 
\textit{<!-- ... -->}\mbox{}\newline 
{<\textbf{p}>}\mbox{}\newline 
\hspace*{6pt}{<\textbf{said}>}"Our minstrel here will warm the old man's heart with song, dazzle him with jewels and\mbox{}\newline 
\hspace*{6pt}\hspace*{6pt} gold"{</\textbf{said}>}, a troublemaker simpered. {<\textbf{said}>}"He'll trample on the Duke's camellias, spill\mbox{}\newline 
\hspace*{6pt}\hspace*{6pt} his wine, and blunt his sword, and say his name begins with X, and in the end the Duke\mbox{}\newline 
\hspace*{6pt}\hspace*{6pt} will say, {<\textbf{said}>}'Take Saralinda, with my blessing, O lordly Prince of Rags and Tags, O\mbox{}\newline 
\hspace*{6pt}\hspace*{6pt}\hspace*{6pt}\hspace*{6pt} rider of the sun!'{</\textbf{said}>}"{</\textbf{said}>}\mbox{}\newline 
{</\textbf{p}>}\end{shaded}\egroup 


    \item[{Example}]
  \leavevmode\bgroup\exampleFont \begin{shaded}\noindent\mbox{}{<\textbf{p}>}\mbox{}\newline 
\hspace*{6pt}{<\textbf{said}\hspace*{6pt}{aloud}="{true}"\hspace*{6pt}{rend}="{pre(“) post(”)}">}Hmmm{</\textbf{said}>}, said a small voice in his ear.\mbox{}\newline 
{<\textbf{said}\hspace*{6pt}{aloud}="{true}"\hspace*{6pt}{rend}="{pre(“) post(”)}">}Difficult. Very difficult. Plenty of courage, I see.\mbox{}\newline 
\hspace*{6pt}\hspace*{6pt} Not a bad mind either. there's talent, oh my goodness, yes — and a nice thirst to prove\mbox{}\newline 
\hspace*{6pt}\hspace*{6pt} yourself, now that's interesting. … So where shall I put you?{</\textbf{said}>}\mbox{}\newline 
{</\textbf{p}>}\mbox{}\newline 
{<\textbf{p}>}Harry gripped the edges of the stool and thought, {<\textbf{said}\hspace*{6pt}{aloud}="{false}"\hspace*{6pt}{rend}="{italic}">}Not\mbox{}\newline 
\hspace*{6pt}\hspace*{6pt} Slytherin, not Slytherin{</\textbf{said}>}.{</\textbf{p}>}\end{shaded}\egroup 


    \item[{Content model}]
  \mbox{}\hfill\\[-10pt]\begin{Verbatim}[fontsize=\small]
<content>
 <macroRef key="macro.specialPara"/>
</content>
    
\end{Verbatim}

    \item[{Schema Declaration}]
  \mbox{}\hfill\\[-10pt]\begin{Verbatim}[fontsize=\small]
element said
{
   att.global.attributes,
   att.ascribed.attributes,
   attribute aloud { text }?,
   attribute direct { text }?,
   macro.specialPara}
\end{Verbatim}

\end{reflist}  \index{salute=<salute>|oddindex}
\begin{reflist}
\item[]\begin{specHead}{TEI.salute}{<salute> }(salutation) contains a salutation or greeting prefixed to a foreword, dedicatory epistle, or other division of a text, or the salutation in the closing of a letter, preface, etc. [\xref{http://www.tei-c.org/release/doc/tei-p5-doc/en/html/DS.html\#DSOC}{4.2.2. Openers and Closers}]\end{specHead} 
    \item[{Module}]
  textstructure
    \item[{Attributes}]
  Attributes att.global (\textit{@xml:id}, \textit{@n}, \textit{@xml:lang}, \textit{@xml:base}, \textit{@xml:space})  (att.global.rendition (\textit{@rend}, \textit{@style}, \textit{@rendition})) (att.global.linking (\textit{@corresp}, \textit{@synch}, \textit{@sameAs}, \textit{@copyOf}, \textit{@next}, \textit{@prev}, \textit{@exclude}, \textit{@select})) (att.global.analytic (\textit{@ana})) (att.global.facs (\textit{@facs})) (att.global.change (\textit{@change})) (att.global.responsibility (\textit{@cert}, \textit{@resp})) (att.global.source (\textit{@source})) att.written (\textit{@hand}) 
    \item[{Member of}]
  model.divWrapper 
    \item[{Contained by}]
  
    \item[core: ]
   lg list\par 
    \item[figures: ]
   figure table\par 
    \item[textstructure: ]
   body closer div front group opener
    \item[{May contain}]
  
    \item[analysis: ]
   c cl interp interpGrp m pc phr s span spanGrp w\par 
    \item[core: ]
   abbr add address bibl biblStruct cb choice cit corr date del desc distinct email emph expan foreign gap gb gloss graphic hi index l label lb lg list listBibl measure measureGrp media mentioned milestone name note num orig pb ptr q quote ref reg rs said sic soCalled stage term time title unclear\par 
    \item[figures: ]
   figure formula notatedMusic table\par 
    \item[gaiji: ]
   g\par 
    \item[header: ]
   biblFull idno\par 
    \item[linking: ]
   alt altGrp anchor join joinGrp link linkGrp seg timeline\par 
    \item[msdescription: ]
   catchwords depth dim dimensions height heraldry locus locusGrp material msDesc objectType origDate origPlace secFol signatures stamp watermark width\par 
    \item[namesdates: ]
   addName affiliation bloc climate country district forename genName geo geogFeat geogName listEvent listNym listOrg listPerson listPlace location nameLink offset orgName persName placeName population region roleName settlement state surname terrain trait\par 
    \item[textcrit: ]
   app listApp listWit witDetail\par 
    \item[textstructure: ]
   floatingText\par 
    \item[transcr: ]
   addSpan am damage damageSpan delSpan ex fw handShift listTranspose metamark mod redo restore retrace secl space subst substJoin supplied surplus undo\par character data
    \item[{Example}]
  \leavevmode\bgroup\exampleFont \begin{shaded}\noindent\mbox{}{<\textbf{salute}>}To all courteous mindes, that will voutchsafe the readinge.{</\textbf{salute}>}\end{shaded}\egroup 


    \item[{Content model}]
  \mbox{}\hfill\\[-10pt]\begin{Verbatim}[fontsize=\small]
<content>
 <macroRef key="macro.paraContent"/>
</content>
    
\end{Verbatim}

    \item[{Schema Declaration}]
  \mbox{}\hfill\\[-10pt]\begin{Verbatim}[fontsize=\small]
element salute
{
   att.global.attributes,
   att.written.attributes,
   macro.paraContent}
\end{Verbatim}

\end{reflist}  \index{samplingDecl=<samplingDecl>|oddindex}
\begin{reflist}
\item[]\begin{specHead}{TEI.samplingDecl}{<samplingDecl> }(sampling declaration) contains a prose description of the rationale and methods used in sampling texts in the creation of a corpus or collection. [\xref{http://www.tei-c.org/release/doc/tei-p5-doc/en/html/HD.html\#HD52}{2.3.2. The Sampling Declaration} \xref{http://www.tei-c.org/release/doc/tei-p5-doc/en/html/HD.html\#HD5}{2.3. The Encoding Description} \xref{http://www.tei-c.org/release/doc/tei-p5-doc/en/html/CC.html\#CCAS2}{15.3.2. Declarable Elements}]\end{specHead} 
    \item[{Module}]
  header
    \item[{Attributes}]
  Attributes att.global (\textit{@xml:id}, \textit{@n}, \textit{@xml:lang}, \textit{@xml:base}, \textit{@xml:space})  (att.global.rendition (\textit{@rend}, \textit{@style}, \textit{@rendition})) (att.global.linking (\textit{@corresp}, \textit{@synch}, \textit{@sameAs}, \textit{@copyOf}, \textit{@next}, \textit{@prev}, \textit{@exclude}, \textit{@select})) (att.global.analytic (\textit{@ana})) (att.global.facs (\textit{@facs})) (att.global.change (\textit{@change})) (att.global.responsibility (\textit{@cert}, \textit{@resp})) (att.global.source (\textit{@source})) att.declarable (\textit{@default}) 
    \item[{Member of}]
  model.encodingDescPart
    \item[{Contained by}]
  
    \item[header: ]
   encodingDesc
    \item[{May contain}]
  
    \item[core: ]
   p\par 
    \item[linking: ]
   ab
    \item[{Note}]
  \par
This element records all information about systematic inclusion or omission of portions of the text, whether a reflection of sampling procedures in the pure sense or of systematic omission of material deemed either too difficult to transcribe or not of sufficient interest.
    \item[{Example}]
  \leavevmode\bgroup\exampleFont \begin{shaded}\noindent\mbox{}{<\textbf{samplingDecl}>}\mbox{}\newline 
\hspace*{6pt}{<\textbf{p}>}Samples of up to 2000 words taken at random from the beginning, middle, or end of each\mbox{}\newline 
\hspace*{6pt}\hspace*{6pt} text identified as relevant by respondents.{</\textbf{p}>}\mbox{}\newline 
{</\textbf{samplingDecl}>}\end{shaded}\egroup 


    \item[{Content model}]
  \mbox{}\hfill\\[-10pt]\begin{Verbatim}[fontsize=\small]
<content>
 <classRef key="model.pLike"
  maxOccurs="unbounded" minOccurs="1"/>
</content>
    
\end{Verbatim}

    \item[{Schema Declaration}]
  \mbox{}\hfill\\[-10pt]\begin{Verbatim}[fontsize=\small]
element samplingDecl
{
   att.global.attributes,
   att.declarable.attributes,
   model.pLike+
}
\end{Verbatim}

\end{reflist}  \index{schemaRef=<schemaRef>|oddindex}\index{key=@key!<schemaRef>|oddindex}
\begin{reflist}
\item[]\begin{specHead}{TEI.schemaRef}{<schemaRef> }(schema reference) describes or points to a related customization or schema file [\xref{http://www.tei-c.org/release/doc/tei-p5-doc/en/html/HD.html\#HDSCHSPEC}{2.3.9. The Schema Specification}]\end{specHead} 
    \item[{Module}]
  header
    \item[{Attributes}]
  Attributes att.global (\textit{@xml:id}, \textit{@n}, \textit{@xml:lang}, \textit{@xml:base}, \textit{@xml:space})  (att.global.rendition (\textit{@rend}, \textit{@style}, \textit{@rendition})) (att.global.linking (\textit{@corresp}, \textit{@synch}, \textit{@sameAs}, \textit{@copyOf}, \textit{@next}, \textit{@prev}, \textit{@exclude}, \textit{@select})) (att.global.analytic (\textit{@ana})) (att.global.facs (\textit{@facs})) (att.global.change (\textit{@change})) (att.global.responsibility (\textit{@cert}, \textit{@resp})) (att.global.source (\textit{@source})) att.typed (\textit{@type}, \textit{@subtype}) att.resourced (\textit{@url}) \hfil\\[-10pt]\begin{sansreflist}
    \item[@key]
  the identifier used for the customization or schema
\begin{reflist}
    \item[{Status}]
  Optional
    \item[{Datatype}]
  teidata.xmlName
\end{reflist}  
\end{sansreflist}  
    \item[{Member of}]
  model.encodingDescPart
    \item[{Contained by}]
  
    \item[header: ]
   encodingDesc
    \item[{May contain}]
  
    \item[core: ]
   desc
    \item[{Example}]
  \leavevmode\bgroup\exampleFont \begin{shaded}\noindent\mbox{}{<\textbf{schemaRef}\hspace*{6pt}{type}="{interchangeODD}"\mbox{}\newline 
\hspace*{6pt}{url}="{http://www.tei-c.org/release/xml/tei/custom/odd/tei\textunderscore lite.odd}"/>}\mbox{}\newline 
{<\textbf{schemaRef}\hspace*{6pt}{type}="{interchangeRNG}"\mbox{}\newline 
\hspace*{6pt}{url}="{http://www.tei-c.org/release/xml/tei/custom/odd/tei\textunderscore lite.rng}"/>}\mbox{}\newline 
{<\textbf{schemaRef}\hspace*{6pt}{type}="{projectODD}"\mbox{}\newline 
\hspace*{6pt}{url}="{file:///schema/project.odd}"/>}\end{shaded}\egroup 


    \item[{Content model}]
  \mbox{}\hfill\\[-10pt]\begin{Verbatim}[fontsize=\small]
<content>
 <classRef key="model.descLike"
  minOccurs="0"/>
</content>
    
\end{Verbatim}

    \item[{Schema Declaration}]
  \mbox{}\hfill\\[-10pt]\begin{Verbatim}[fontsize=\small]
element schemaRef
{
   att.global.attributes,
   att.typed.attributes,
   att.resourced.attributes,
   attribute key { text }?,
   model.descLike?
}
\end{Verbatim}

\end{reflist}  \index{scriptDesc=<scriptDesc>|oddindex}
\begin{reflist}
\item[]\begin{specHead}{TEI.scriptDesc}{<scriptDesc> }contains a description of the scripts used in a manuscript or similar source. [\xref{http://www.tei-c.org/release/doc/tei-p5-doc/en/html/MS.html\#msphwr}{10.7.2.1. Writing}]\end{specHead} 
    \item[{Module}]
  msdescription
    \item[{Attributes}]
  Attributes att.global (\textit{@xml:id}, \textit{@n}, \textit{@xml:lang}, \textit{@xml:base}, \textit{@xml:space})  (att.global.rendition (\textit{@rend}, \textit{@style}, \textit{@rendition})) (att.global.linking (\textit{@corresp}, \textit{@synch}, \textit{@sameAs}, \textit{@copyOf}, \textit{@next}, \textit{@prev}, \textit{@exclude}, \textit{@select})) (att.global.analytic (\textit{@ana})) (att.global.facs (\textit{@facs})) (att.global.change (\textit{@change})) (att.global.responsibility (\textit{@cert}, \textit{@resp})) (att.global.source (\textit{@source}))
    \item[{Member of}]
  model.physDescPart
    \item[{Contained by}]
  
    \item[msdescription: ]
   physDesc
    \item[{May contain}]
  
    \item[core: ]
   p\par 
    \item[header: ]
   scriptNote\par 
    \item[linking: ]
   ab\par 
    \item[msdescription: ]
   summary
    \item[{Example}]
  \leavevmode\bgroup\exampleFont \begin{shaded}\noindent\mbox{}{<\textbf{scriptDesc}>}\mbox{}\newline 
\hspace*{6pt}{<\textbf{p}/>}\mbox{}\newline 
{</\textbf{scriptDesc}>}\end{shaded}\egroup 


    \item[{Example}]
  \leavevmode\bgroup\exampleFont \begin{shaded}\noindent\mbox{}{<\textbf{scriptDesc}>}\mbox{}\newline 
\hspace*{6pt}{<\textbf{summary}>}Contains two distinct styles of scripts {</\textbf{summary}>}\mbox{}\newline 
\hspace*{6pt}{<\textbf{scriptNote}\hspace*{6pt}{xml:id}="{style-1}">}.{</\textbf{scriptNote}>}\mbox{}\newline 
\hspace*{6pt}{<\textbf{scriptNote}\hspace*{6pt}{xml:id}="{style-2}">}.{</\textbf{scriptNote}>}\mbox{}\newline 
{</\textbf{scriptDesc}>}\end{shaded}\egroup 


    \item[{Content model}]
  \mbox{}\hfill\\[-10pt]\begin{Verbatim}[fontsize=\small]
<content>
 <alternate>
  <classRef key="model.pLike"
   maxOccurs="unbounded" minOccurs="1"/>
  <sequence>
   <elementRef key="summary" minOccurs="0"/>
   <elementRef key="scriptNote"
    maxOccurs="unbounded" minOccurs="1"/>
  </sequence>
 </alternate>
</content>
    
\end{Verbatim}

    \item[{Schema Declaration}]
  \mbox{}\hfill\\[-10pt]\begin{Verbatim}[fontsize=\small]
element scriptDesc
{
   att.global.attributes,
   ( model.pLike+ | ( summary?, scriptNote+ ) )
}
\end{Verbatim}

\end{reflist}  \index{scriptNote=<scriptNote>|oddindex}
\begin{reflist}
\item[]\begin{specHead}{TEI.scriptNote}{<scriptNote> }describes a particular script distinguished within the description of a manuscript or similar resource. [\xref{http://www.tei-c.org/release/doc/tei-p5-doc/en/html/MS.html\#msph2}{10.7.2. Writing, Decoration, and Other Notations}]\end{specHead} 
    \item[{Module}]
  header
    \item[{Attributes}]
  Attributes att.global (\textit{@xml:id}, \textit{@n}, \textit{@xml:lang}, \textit{@xml:base}, \textit{@xml:space})  (att.global.rendition (\textit{@rend}, \textit{@style}, \textit{@rendition})) (att.global.linking (\textit{@corresp}, \textit{@synch}, \textit{@sameAs}, \textit{@copyOf}, \textit{@next}, \textit{@prev}, \textit{@exclude}, \textit{@select})) (att.global.analytic (\textit{@ana})) (att.global.facs (\textit{@facs})) (att.global.change (\textit{@change})) (att.global.responsibility (\textit{@cert}, \textit{@resp})) (att.global.source (\textit{@source})) att.handFeatures (\textit{@scribe}, \textit{@scribeRef}, \textit{@script}, \textit{@scriptRef}, \textit{@medium}, \textit{@scope}) 
    \item[{Contained by}]
  
    \item[msdescription: ]
   scriptDesc
    \item[{May contain}]
  
    \item[analysis: ]
   c cl interp interpGrp m pc phr s span spanGrp w\par 
    \item[core: ]
   abbr add address bibl biblStruct cb choice cit corr date del desc distinct email emph expan foreign gap gb gloss graphic hi index l label lb lg list listBibl measure measureGrp media mentioned milestone name note num orig p pb ptr q quote ref reg rs said sic soCalled sp stage term time title unclear\par 
    \item[figures: ]
   figure formula notatedMusic table\par 
    \item[gaiji: ]
   g\par 
    \item[header: ]
   biblFull idno\par 
    \item[linking: ]
   ab alt altGrp anchor join joinGrp link linkGrp seg timeline\par 
    \item[msdescription: ]
   catchwords depth dim dimensions height heraldry locus locusGrp material msDesc objectType origDate origPlace secFol signatures stamp watermark width\par 
    \item[namesdates: ]
   addName affiliation bloc climate country district forename genName geo geogFeat geogName listEvent listNym listOrg listPerson listPlace location nameLink offset orgName persName placeName population region roleName settlement state surname terrain trait\par 
    \item[textcrit: ]
   app listApp listWit witDetail\par 
    \item[textstructure: ]
   floatingText\par 
    \item[transcr: ]
   addSpan am damage damageSpan delSpan ex fw handShift listTranspose metamark mod redo restore retrace secl space subst substJoin supplied surplus undo\par character data
    \item[{Example}]
  \leavevmode\bgroup\exampleFont \begin{shaded}\noindent\mbox{}{<\textbf{scriptNote}\hspace*{6pt}{scope}="{sole}"/>}\end{shaded}\egroup 


    \item[{Content model}]
  \mbox{}\hfill\\[-10pt]\begin{Verbatim}[fontsize=\small]
<content>
 <macroRef key="macro.specialPara"/>
</content>
    
\end{Verbatim}

    \item[{Schema Declaration}]
  \mbox{}\hfill\\[-10pt]\begin{Verbatim}[fontsize=\small]
element scriptNote
{
   att.global.attributes,
   att.handFeatures.attributes,
   macro.specialPara}
\end{Verbatim}

\end{reflist}  \index{seal=<seal>|oddindex}\index{contemporary=@contemporary!<seal>|oddindex}
\begin{reflist}
\item[]\begin{specHead}{TEI.seal}{<seal> }contains a description of one seal or similar attachment applied to a manuscript. [\xref{http://www.tei-c.org/release/doc/tei-p5-doc/en/html/MS.html\#msphse}{10.7.3.2. Seals}]\end{specHead} 
    \item[{Module}]
  msdescription
    \item[{Attributes}]
  Attributes att.global (\textit{@xml:id}, \textit{@n}, \textit{@xml:lang}, \textit{@xml:base}, \textit{@xml:space})  (att.global.rendition (\textit{@rend}, \textit{@style}, \textit{@rendition})) (att.global.linking (\textit{@corresp}, \textit{@synch}, \textit{@sameAs}, \textit{@copyOf}, \textit{@next}, \textit{@prev}, \textit{@exclude}, \textit{@select})) (att.global.analytic (\textit{@ana})) (att.global.facs (\textit{@facs})) (att.global.change (\textit{@change})) (att.global.responsibility (\textit{@cert}, \textit{@resp})) (att.global.source (\textit{@source})) att.typed (\textit{@type}, \textit{@subtype}) att.datable (\textit{@calendar}, \textit{@period})  (att.datable.w3c (\textit{@when}, \textit{@notBefore}, \textit{@notAfter}, \textit{@from}, \textit{@to})) (att.datable.iso (\textit{@when-iso}, \textit{@notBefore-iso}, \textit{@notAfter-iso}, \textit{@from-iso}, \textit{@to-iso})) (att.datable.custom (\textit{@when-custom}, \textit{@notBefore-custom}, \textit{@notAfter-custom}, \textit{@from-custom}, \textit{@to-custom}, \textit{@datingPoint}, \textit{@datingMethod})) \hfil\\[-10pt]\begin{sansreflist}
    \item[@contemporary]
  specifies whether or not the seal is contemporary with the item to which it is affixed
\begin{reflist}
    \item[{Status}]
  Optional
    \item[{Datatype}]
  teidata.xTruthValue
\end{reflist}  
\end{sansreflist}  
    \item[{Contained by}]
  
    \item[msdescription: ]
   sealDesc
    \item[{May contain}]
  
    \item[core: ]
   p\par 
    \item[linking: ]
   ab\par 
    \item[msdescription: ]
   decoNote
    \item[{Example}]
  \leavevmode\bgroup\exampleFont \begin{shaded}\noindent\mbox{}{<\textbf{seal}\hspace*{6pt}{n}="{2}"\hspace*{6pt}{subtype}="{cauda\textunderscore duplex}"\mbox{}\newline 
\hspace*{6pt}{type}="{pendant}">}\mbox{}\newline 
\hspace*{6pt}{<\textbf{p}>}The seal of {<\textbf{name}>}Jens Olufsen{</\textbf{name}>} in black wax.\mbox{}\newline 
\hspace*{6pt}\hspace*{6pt} ({<\textbf{ref}>}DAS 1061{</\textbf{ref}>}). Legend: {<\textbf{q}>}S IOHANNES OLAVI{</\textbf{q}>}.\mbox{}\newline 
\hspace*{6pt}\hspace*{6pt} Parchment tag on which is written: {<\textbf{q}>}Woldorp Iohanne G{</\textbf{q}>}.{</\textbf{p}>}\mbox{}\newline 
{</\textbf{seal}>}\end{shaded}\egroup 


    \item[{Content model}]
  \mbox{}\hfill\\[-10pt]\begin{Verbatim}[fontsize=\small]
<content>
 <alternate maxOccurs="unbounded"
  minOccurs="1">
  <classRef key="model.pLike"/>
  <elementRef key="decoNote"/>
 </alternate>
</content>
    
\end{Verbatim}

    \item[{Schema Declaration}]
  \mbox{}\hfill\\[-10pt]\begin{Verbatim}[fontsize=\small]
element seal
{
   att.global.attributes,
   att.typed.attributes,
   att.datable.attributes,
   attribute contemporary { text }?,
   ( model.pLike | decoNote )+
}
\end{Verbatim}

\end{reflist}  \index{sealDesc=<sealDesc>|oddindex}
\begin{reflist}
\item[]\begin{specHead}{TEI.sealDesc}{<sealDesc> }(seal description) describes the seals or other external items attached to a manuscript, either as a series of paragraphs or as a series of distinct <seal> elements, possibly with additional <decoNote>s. [\xref{http://www.tei-c.org/release/doc/tei-p5-doc/en/html/MS.html\#msphse}{10.7.3.2. Seals}]\end{specHead} 
    \item[{Module}]
  msdescription
    \item[{Attributes}]
  Attributes att.global (\textit{@xml:id}, \textit{@n}, \textit{@xml:lang}, \textit{@xml:base}, \textit{@xml:space})  (att.global.rendition (\textit{@rend}, \textit{@style}, \textit{@rendition})) (att.global.linking (\textit{@corresp}, \textit{@synch}, \textit{@sameAs}, \textit{@copyOf}, \textit{@next}, \textit{@prev}, \textit{@exclude}, \textit{@select})) (att.global.analytic (\textit{@ana})) (att.global.facs (\textit{@facs})) (att.global.change (\textit{@change})) (att.global.responsibility (\textit{@cert}, \textit{@resp})) (att.global.source (\textit{@source}))
    \item[{Member of}]
  model.physDescPart
    \item[{Contained by}]
  
    \item[msdescription: ]
   physDesc
    \item[{May contain}]
  
    \item[core: ]
   p\par 
    \item[linking: ]
   ab\par 
    \item[msdescription: ]
   condition decoNote seal summary
    \item[{Example}]
  \leavevmode\bgroup\exampleFont \begin{shaded}\noindent\mbox{}{<\textbf{sealDesc}>}\mbox{}\newline 
\hspace*{6pt}{<\textbf{seal}\hspace*{6pt}{contemporary}="{true}"\hspace*{6pt}{type}="{pendant}">}\mbox{}\newline 
\hspace*{6pt}\hspace*{6pt}{<\textbf{p}>}Green wax vertical oval seal attached at base.{</\textbf{p}>}\mbox{}\newline 
\hspace*{6pt}{</\textbf{seal}>}\mbox{}\newline 
{</\textbf{sealDesc}>}\end{shaded}\egroup 


    \item[{Example}]
  \leavevmode\bgroup\exampleFont \begin{shaded}\noindent\mbox{}{<\textbf{sealDesc}>}\mbox{}\newline 
\hspace*{6pt}{<\textbf{p}>}Parchment strip for seal in place; seal missing.{</\textbf{p}>}\mbox{}\newline 
{</\textbf{sealDesc}>}\end{shaded}\egroup 


    \item[{Content model}]
  \mbox{}\hfill\\[-10pt]\begin{Verbatim}[fontsize=\small]
<content>
 <alternate>
  <classRef key="model.pLike"
   maxOccurs="unbounded" minOccurs="1"/>
  <sequence>
   <elementRef key="summary" minOccurs="0"/>
   <alternate maxOccurs="unbounded"
    minOccurs="1">
    <elementRef key="decoNote"/>
    <elementRef key="seal"/>
    <elementRef key="condition"/>
   </alternate>
  </sequence>
 </alternate>
</content>
    
\end{Verbatim}

    \item[{Schema Declaration}]
  \mbox{}\hfill\\[-10pt]\begin{Verbatim}[fontsize=\small]
element sealDesc
{
   att.global.attributes,
   ( model.pLike+ | ( summary?, ( decoNote | seal | condition )+ ) )
}
\end{Verbatim}

\end{reflist}  \index{secFol=<secFol>|oddindex}
\begin{reflist}
\item[]\begin{specHead}{TEI.secFol}{<secFol> }(second folio) marks the word or words taken from a fixed point in a codex (typically the beginning of the second leaf) in order to provide a unique identifier for it.  [\xref{http://www.tei-c.org/release/doc/tei-p5-doc/en/html/MS.html\#msmisc}{10.3.7. Catchwords, Signatures, Secundo Folio}]\end{specHead} 
    \item[{Module}]
  msdescription
    \item[{Attributes}]
  Attributes att.global (\textit{@xml:id}, \textit{@n}, \textit{@xml:lang}, \textit{@xml:base}, \textit{@xml:space})  (att.global.rendition (\textit{@rend}, \textit{@style}, \textit{@rendition})) (att.global.linking (\textit{@corresp}, \textit{@synch}, \textit{@sameAs}, \textit{@copyOf}, \textit{@next}, \textit{@prev}, \textit{@exclude}, \textit{@select})) (att.global.analytic (\textit{@ana})) (att.global.facs (\textit{@facs})) (att.global.change (\textit{@change})) (att.global.responsibility (\textit{@cert}, \textit{@resp})) (att.global.source (\textit{@source}))
    \item[{Member of}]
  model.pPart.msdesc
    \item[{Contained by}]
  
    \item[analysis: ]
   cl phr s span\par 
    \item[core: ]
   abbr add addrLine author biblScope citedRange corr date del desc distinct editor email emph expan foreign gloss head headItem headLabel hi item l label measure meeting mentioned name note num orig p pubPlace publisher q quote ref reg resp rs said sic soCalled speaker stage street term textLang time title unclear\par 
    \item[figures: ]
   cell figDesc\par 
    \item[header: ]
   authority catDesc change classCode creation distributor edition extent funder geoDecl handNote language licence principal rendition scriptNote sponsor tagUsage typeNote\par 
    \item[linking: ]
   ab seg\par 
    \item[msdescription: ]
   accMat acquisition additions catchwords collation colophon condition custEvent decoNote explicit filiation finalRubric foliation heraldry incipit layout material musicNotation objectType origDate origPlace origin provenance rubric secFol signatures source stamp summary support surrogates watermark\par 
    \item[namesdates: ]
   addName affiliation age birth bloc country death district education faith floruit forename genName geogFeat geogName langKnown nameLink nationality occupation offset orgName persName placeName region residence roleName settlement sex socecStatus surname\par 
    \item[textcrit: ]
   lem rdg wit witDetail witness\par 
    \item[textstructure: ]
   byline closer dateline docAuthor docDate docEdition docImprint imprimatur opener salute signed titlePart trailer\par 
    \item[transcr: ]
   damage fw metamark mod restore retrace secl supplied surplus
    \item[{May contain}]
  
    \item[analysis: ]
   c cl interp interpGrp m pc phr s span spanGrp w\par 
    \item[core: ]
   abbr add address cb choice corr date del distinct email emph expan foreign gap gb gloss graphic hi index lb measure measureGrp media mentioned milestone name note num orig pb ptr ref reg rs sic soCalled term time title unclear\par 
    \item[figures: ]
   figure formula notatedMusic\par 
    \item[gaiji: ]
   g\par 
    \item[header: ]
   idno\par 
    \item[linking: ]
   alt altGrp anchor join joinGrp link linkGrp seg timeline\par 
    \item[msdescription: ]
   catchwords depth dim dimensions height heraldry locus locusGrp material objectType origDate origPlace secFol signatures stamp watermark width\par 
    \item[namesdates: ]
   addName affiliation bloc climate country district forename genName geo geogFeat geogName location nameLink offset orgName persName placeName population region roleName settlement state surname terrain trait\par 
    \item[textcrit: ]
   app witDetail\par 
    \item[transcr: ]
   addSpan am damage damageSpan delSpan ex fw handShift listTranspose metamark mod redo restore retrace secl space subst substJoin supplied surplus undo\par character data
    \item[{Example}]
  \leavevmode\bgroup\exampleFont \begin{shaded}\noindent\mbox{}{<\textbf{secFol}>}(con-)versio morum{</\textbf{secFol}>}\end{shaded}\egroup 


    \item[{Schematron}]
   <sch:assert role="nonfatal"  test="ancestor::tei:msDesc">WARNING: deprecated use of element — The <sch:name/> element will not be allowed outside of msDesc as of 2018-10-01.</sch:assert>
    \item[{Content model}]
  \mbox{}\hfill\\[-10pt]\begin{Verbatim}[fontsize=\small]
<content>
 <macroRef key="macro.phraseSeq"/>
</content>
    
\end{Verbatim}

    \item[{Schema Declaration}]
  \mbox{}\hfill\\[-10pt]\begin{Verbatim}[fontsize=\small]
element secFol { att.global.attributes, macro.phraseSeq }
\end{Verbatim}

\end{reflist}  \index{secl=<secl>|oddindex}\index{reason=@reason!<secl>|oddindex}
\begin{reflist}
\item[]\begin{specHead}{TEI.secl}{<secl> }(secluded text) Secluded. Marks text present in the source which the editor believes to be genuine but out of its original place (which is unknown). [\xref{http://www.tei-c.org/release/doc/tei-p5-doc/en/html/PH.html\#PHOM}{11.3.1.7. Text Omitted from or Supplied in the Transcription}]\end{specHead} 
    \item[{Module}]
  transcr
    \item[{Attributes}]
  Attributes att.global (\textit{@xml:id}, \textit{@n}, \textit{@xml:lang}, \textit{@xml:base}, \textit{@xml:space})  (att.global.rendition (\textit{@rend}, \textit{@style}, \textit{@rendition})) (att.global.linking (\textit{@corresp}, \textit{@synch}, \textit{@sameAs}, \textit{@copyOf}, \textit{@next}, \textit{@prev}, \textit{@exclude}, \textit{@select})) (att.global.analytic (\textit{@ana})) (att.global.facs (\textit{@facs})) (att.global.change (\textit{@change})) (att.global.responsibility (\textit{@cert}, \textit{@resp})) (att.global.source (\textit{@source})) att.editLike (\textit{@evidence}, \textit{@instant})  (att.dimensions (\textit{@unit}, \textit{@quantity}, \textit{@extent}, \textit{@precision}, \textit{@scope}) (att.ranging (\textit{@atLeast}, \textit{@atMost}, \textit{@min}, \textit{@max}, \textit{@confidence})) ) \hfil\\[-10pt]\begin{sansreflist}
    \item[@reason]
  one or more words indicating why this text has been secluded, e.g. \textit{interpolated} etc.
\begin{reflist}
    \item[{Status}]
  Optional
    \item[{Datatype}]
  1–∞ occurrences of teidata.word separated by whitespace
\end{reflist}  
\end{sansreflist}  
    \item[{Member of}]
  model.pPart.transcriptional
    \item[{Contained by}]
  
    \item[analysis: ]
   cl pc phr s w\par 
    \item[core: ]
   abbr add addrLine author bibl biblScope citedRange corr date del distinct editor email emph expan foreign gloss head headItem headLabel hi item l label measure mentioned name note num orig p pubPlace publisher q quote ref reg rs said sic soCalled speaker stage street term textLang time title unclear\par 
    \item[figures: ]
   cell\par 
    \item[header: ]
   change distributor edition extent geoDecl handNote licence scriptNote typeNote\par 
    \item[linking: ]
   ab seg\par 
    \item[msdescription: ]
   accMat acquisition additions catchwords collation colophon condition custEvent decoNote explicit filiation finalRubric foliation heraldry incipit layout material musicNotation objectType origDate origPlace origin provenance rubric secFol signatures source stamp summary support surrogates watermark\par 
    \item[namesdates: ]
   addName affiliation birth bloc country death district education faith floruit forename genName geogFeat geogName nameLink nationality occupation offset orgName persName placeName region residence roleName settlement sex socecStatus surname\par 
    \item[textcrit: ]
   lem rdg wit witDetail\par 
    \item[textstructure: ]
   byline closer dateline docAuthor docDate docEdition docImprint imprimatur opener salute signed titlePart trailer\par 
    \item[transcr: ]
   am damage fw metamark mod restore retrace secl supplied surplus
    \item[{May contain}]
  
    \item[analysis: ]
   c cl interp interpGrp m pc phr s span spanGrp w\par 
    \item[core: ]
   abbr add address bibl biblStruct cb choice cit corr date del desc distinct email emph expan foreign gap gb gloss graphic hi index l label lb lg list listBibl measure measureGrp media mentioned milestone name note num orig pb ptr q quote ref reg rs said sic soCalled stage term time title unclear\par 
    \item[figures: ]
   figure formula notatedMusic table\par 
    \item[gaiji: ]
   g\par 
    \item[header: ]
   biblFull idno\par 
    \item[linking: ]
   alt altGrp anchor join joinGrp link linkGrp seg timeline\par 
    \item[msdescription: ]
   catchwords depth dim dimensions height heraldry locus locusGrp material msDesc objectType origDate origPlace secFol signatures stamp watermark width\par 
    \item[namesdates: ]
   addName affiliation bloc climate country district forename genName geo geogFeat geogName listEvent listNym listOrg listPerson listPlace location nameLink offset orgName persName placeName population region roleName settlement state surname terrain trait\par 
    \item[textcrit: ]
   app listApp listWit witDetail\par 
    \item[textstructure: ]
   floatingText\par 
    \item[transcr: ]
   addSpan am damage damageSpan delSpan ex fw handShift listTranspose metamark mod redo restore retrace secl space subst substJoin supplied surplus undo\par character data
    \item[{Example}]
  \leavevmode\bgroup\exampleFont \begin{shaded}\noindent\mbox{}{<\textbf{rdg}\hspace*{6pt}{source}="{\#Pescani}">}\mbox{}\newline 
\hspace*{6pt}{<\textbf{secl}>}\mbox{}\newline 
\hspace*{6pt}\hspace*{6pt}{<\textbf{l}\hspace*{6pt}{n}="{15}"\hspace*{6pt}{xml:id}="{l15}">}Alphesiboea suos ulta est pro coniuge fratres,{</\textbf{l}>}\mbox{}\newline 
\hspace*{6pt}\hspace*{6pt}{<\textbf{l}\hspace*{6pt}{n}="{16}"\hspace*{6pt}{xml:id}="{l16}">}sanguinis et cari vincula rupit amor.{</\textbf{l}>}\mbox{}\newline 
\hspace*{6pt}{</\textbf{secl}>}\mbox{}\newline 
{</\textbf{rdg}>}\mbox{}\newline 
{<\textbf{wit}>}secl. Pescani{</\textbf{wit}>}\end{shaded}\egroup 


    \item[{Content model}]
  \mbox{}\hfill\\[-10pt]\begin{Verbatim}[fontsize=\small]
<content>
 <macroRef key="macro.paraContent"/>
</content>
    
\end{Verbatim}

    \item[{Schema Declaration}]
  \mbox{}\hfill\\[-10pt]\begin{Verbatim}[fontsize=\small]
element secl
{
   att.global.attributes,
   att.editLike.attributes,
   attribute reason { list { + } }?,
   macro.paraContent}
\end{Verbatim}

\end{reflist}  \index{seg=<seg>|oddindex}
\begin{reflist}
\item[]\begin{specHead}{TEI.seg}{<seg> }(arbitrary segment) represents any segmentation of text below the ‘chunk’ level. [\xref{http://www.tei-c.org/release/doc/tei-p5-doc/en/html/SA.html\#SASE}{16.3. Blocks, Segments, and Anchors} \xref{http://www.tei-c.org/release/doc/tei-p5-doc/en/html/VE.html\#VESE}{6.2. Components of the Verse Line} \xref{http://www.tei-c.org/release/doc/tei-p5-doc/en/html/DR.html\#DRPAL}{7.2.5. Speech Contents}]\end{specHead} 
    \item[{Module}]
  linking
    \item[{Attributes}]
  Attributes att.global (\textit{@xml:id}, \textit{@n}, \textit{@xml:lang}, \textit{@xml:base}, \textit{@xml:space})  (att.global.rendition (\textit{@rend}, \textit{@style}, \textit{@rendition})) (att.global.linking (\textit{@corresp}, \textit{@synch}, \textit{@sameAs}, \textit{@copyOf}, \textit{@next}, \textit{@prev}, \textit{@exclude}, \textit{@select})) (att.global.analytic (\textit{@ana})) (att.global.facs (\textit{@facs})) (att.global.change (\textit{@change})) (att.global.responsibility (\textit{@cert}, \textit{@resp})) (att.global.source (\textit{@source})) att.segLike (\textit{@function})  (att.datcat (\textit{@datcat}, \textit{@valueDatcat})) (att.fragmentable (\textit{@part})) att.typed (\textit{@type}, \textit{@subtype}) att.written (\textit{@hand}) 
    \item[{Member of}]
  model.choicePart model.linePart model.segLike 
    \item[{Contained by}]
  
    \item[analysis: ]
   cl m phr s w\par 
    \item[core: ]
   abbr add addrLine author bibl biblScope choice citedRange corr date del distinct editor email emph expan foreign gloss head headItem headLabel hi item l label measure mentioned name note num orig p pubPlace publisher q quote ref reg rs said sic soCalled speaker stage street term textLang time title unclear\par 
    \item[figures: ]
   cell notatedMusic\par 
    \item[header: ]
   change distributor edition extent geoDecl handNote licence scriptNote typeNote\par 
    \item[linking: ]
   ab seg\par 
    \item[msdescription: ]
   accMat acquisition additions catchwords collation colophon condition custEvent decoNote explicit filiation finalRubric foliation heraldry incipit layout material musicNotation objectType origDate origPlace origin provenance rubric secFol signatures source stamp summary support surrogates watermark\par 
    \item[namesdates: ]
   addName affiliation birth bloc country death district education faith floruit forename genName geogFeat geogName nameLink nationality occupation offset orgName persName placeName region residence roleName settlement sex socecStatus surname\par 
    \item[textcrit: ]
   lem rdg wit witDetail\par 
    \item[textstructure: ]
   byline closer dateline docAuthor docDate docEdition docImprint imprimatur opener salute signed titlePart trailer\par 
    \item[transcr: ]
   damage fw line metamark mod restore retrace secl supplied surplus zone
    \item[{May contain}]
  
    \item[analysis: ]
   c cl interp interpGrp m pc phr s span spanGrp w\par 
    \item[core: ]
   abbr add address bibl biblStruct cb choice cit corr date del desc distinct email emph expan foreign gap gb gloss graphic hi index l label lb lg list listBibl measure measureGrp media mentioned milestone name note num orig pb ptr q quote ref reg rs said sic soCalled stage term time title unclear\par 
    \item[figures: ]
   figure formula notatedMusic table\par 
    \item[gaiji: ]
   g\par 
    \item[header: ]
   biblFull idno\par 
    \item[linking: ]
   alt altGrp anchor join joinGrp link linkGrp seg timeline\par 
    \item[msdescription: ]
   catchwords depth dim dimensions height heraldry locus locusGrp material msDesc objectType origDate origPlace secFol signatures stamp watermark width\par 
    \item[namesdates: ]
   addName affiliation bloc climate country district forename genName geo geogFeat geogName listEvent listNym listOrg listPerson listPlace location nameLink offset orgName persName placeName population region roleName settlement state surname terrain trait\par 
    \item[textcrit: ]
   app listApp listWit witDetail\par 
    \item[textstructure: ]
   floatingText\par 
    \item[transcr: ]
   addSpan am damage damageSpan delSpan ex fw handShift listTranspose metamark mod redo restore retrace secl space subst substJoin supplied surplus undo\par character data
    \item[{Note}]
  \par
The <seg> element may be used at the encoder's discretion to mark any segments of the text of interest for processing. One use of the element is to mark text features for which no appropriate markup is otherwise defined. Another use is to provide an identifier for some segment which is to be pointed at by some other element—i.e. to provide a target, or a part of a target, for a <ptr> or other similar element.
    \item[{Example}]
  \leavevmode\bgroup\exampleFont \begin{shaded}\noindent\mbox{}{<\textbf{seg}>}When are you leaving?{</\textbf{seg}>}\mbox{}\newline 
{<\textbf{seg}>}Tomorrow.{</\textbf{seg}>}\end{shaded}\egroup 


    \item[{Example}]
  \leavevmode\bgroup\exampleFont \begin{shaded}\noindent\mbox{}{<\textbf{s}>}\mbox{}\newline 
\hspace*{6pt}{<\textbf{seg}\hspace*{6pt}{rend}="{caps}"\hspace*{6pt}{type}="{initial-cap}">}So father's only{</\textbf{seg}>} glory was the ballfield. \mbox{}\newline 
{</\textbf{s}>}\end{shaded}\egroup 


    \item[{Example}]
  \leavevmode\bgroup\exampleFont \begin{shaded}\noindent\mbox{}{<\textbf{seg}\hspace*{6pt}{type}="{preamble}">}\mbox{}\newline 
\hspace*{6pt}{<\textbf{seg}>}Sigmund, {<\textbf{seg}\hspace*{6pt}{type}="{patronym}">}the son of Volsung{</\textbf{seg}>}, was a king in Frankish country.{</\textbf{seg}>}\mbox{}\newline 
\hspace*{6pt}{<\textbf{seg}>}Sinfiotli was the eldest of his sons ...{</\textbf{seg}>}\mbox{}\newline 
\hspace*{6pt}{<\textbf{seg}>}Borghild, Sigmund's wife, had a brother ... {</\textbf{seg}>}\mbox{}\newline 
{</\textbf{seg}>}\end{shaded}\egroup 


    \item[{Content model}]
  \mbox{}\hfill\\[-10pt]\begin{Verbatim}[fontsize=\small]
<content>
 <macroRef key="macro.paraContent"/>
</content>
    
\end{Verbatim}

    \item[{Schema Declaration}]
  \mbox{}\hfill\\[-10pt]\begin{Verbatim}[fontsize=\small]
element seg
{
   att.global.attributes,
   att.segLike.attributes,
   att.typed.attributes,
   att.written.attributes,
   macro.paraContent}
\end{Verbatim}

\end{reflist}  \index{segmentation=<segmentation>|oddindex}
\begin{reflist}
\item[]\begin{specHead}{TEI.segmentation}{<segmentation> }describes the principles according to which the text has been segmented, for example into sentences, tone-units, graphemic strata, etc. [\xref{http://www.tei-c.org/release/doc/tei-p5-doc/en/html/HD.html\#HD53}{2.3.3. The Editorial Practices Declaration} \xref{http://www.tei-c.org/release/doc/tei-p5-doc/en/html/CC.html\#CCAS2}{15.3.2. Declarable Elements}]\end{specHead} 
    \item[{Module}]
  header
    \item[{Attributes}]
  Attributes att.global (\textit{@xml:id}, \textit{@n}, \textit{@xml:lang}, \textit{@xml:base}, \textit{@xml:space})  (att.global.rendition (\textit{@rend}, \textit{@style}, \textit{@rendition})) (att.global.linking (\textit{@corresp}, \textit{@synch}, \textit{@sameAs}, \textit{@copyOf}, \textit{@next}, \textit{@prev}, \textit{@exclude}, \textit{@select})) (att.global.analytic (\textit{@ana})) (att.global.facs (\textit{@facs})) (att.global.change (\textit{@change})) (att.global.responsibility (\textit{@cert}, \textit{@resp})) (att.global.source (\textit{@source})) att.declarable (\textit{@default}) 
    \item[{Member of}]
  model.editorialDeclPart
    \item[{Contained by}]
  
    \item[header: ]
   editorialDecl
    \item[{May contain}]
  
    \item[core: ]
   p\par 
    \item[linking: ]
   ab
    \item[{Example}]
  \leavevmode\bgroup\exampleFont \begin{shaded}\noindent\mbox{}{<\textbf{segmentation}>}\mbox{}\newline 
\hspace*{6pt}{<\textbf{p}>}\mbox{}\newline 
\hspace*{6pt}\hspace*{6pt}{<\textbf{gi}>}s{</\textbf{gi}>} elements mark orthographic sentences and are numbered sequentially within\mbox{}\newline 
\hspace*{6pt}\hspace*{6pt} their parent {<\textbf{gi}>}div{</\textbf{gi}>} element {</\textbf{p}>}\mbox{}\newline 
{</\textbf{segmentation}>}\end{shaded}\egroup 


    \item[{Example}]
  \leavevmode\bgroup\exampleFont \begin{shaded}\noindent\mbox{}{<\textbf{p}>}\mbox{}\newline 
\hspace*{6pt}{<\textbf{gi}>}seg{</\textbf{gi}>} elements are used to mark functional constituents of various types within each\mbox{}\newline 
{<\textbf{gi}>}s{</\textbf{gi}>}; the typology used is defined by a {<\textbf{gi}>}taxonomy{</\textbf{gi}>} element in the corpus\mbox{}\newline 
 header {<\textbf{gi}>}classDecl{</\textbf{gi}>}\mbox{}\newline 
{</\textbf{p}>}\end{shaded}\egroup 


    \item[{Content model}]
  \mbox{}\hfill\\[-10pt]\begin{Verbatim}[fontsize=\small]
<content>
 <classRef key="model.pLike"
  maxOccurs="unbounded" minOccurs="1"/>
</content>
    
\end{Verbatim}

    \item[{Schema Declaration}]
  \mbox{}\hfill\\[-10pt]\begin{Verbatim}[fontsize=\small]
element segmentation
{
   att.global.attributes,
   att.declarable.attributes,
   model.pLike+
}
\end{Verbatim}

\end{reflist}  \index{series=<series>|oddindex}
\begin{reflist}
\item[]\begin{specHead}{TEI.series}{<series> }(series information) contains information about the series in which a book or other bibliographic item has appeared. [\xref{http://www.tei-c.org/release/doc/tei-p5-doc/en/html/CO.html\#COBICOL}{3.11.2.1. Analytic, Monographic, and Series Levels}]\end{specHead} 
    \item[{Module}]
  core
    \item[{Attributes}]
  Attributes att.global (\textit{@xml:id}, \textit{@n}, \textit{@xml:lang}, \textit{@xml:base}, \textit{@xml:space})  (att.global.rendition (\textit{@rend}, \textit{@style}, \textit{@rendition})) (att.global.linking (\textit{@corresp}, \textit{@synch}, \textit{@sameAs}, \textit{@copyOf}, \textit{@next}, \textit{@prev}, \textit{@exclude}, \textit{@select})) (att.global.analytic (\textit{@ana})) (att.global.facs (\textit{@facs})) (att.global.change (\textit{@change})) (att.global.responsibility (\textit{@cert}, \textit{@resp})) (att.global.source (\textit{@source}))
    \item[{Member of}]
  model.biblPart
    \item[{Contained by}]
  
    \item[core: ]
   bibl biblStruct
    \item[{May contain}]
  
    \item[analysis: ]
   interp interpGrp span spanGrp\par 
    \item[core: ]
   biblScope cb editor gap gb index lb milestone note pb ptr ref respStmt textLang title\par 
    \item[figures: ]
   figure notatedMusic\par 
    \item[gaiji: ]
   g\par 
    \item[header: ]
   availability idno\par 
    \item[linking: ]
   alt altGrp anchor join joinGrp link linkGrp timeline\par 
    \item[textcrit: ]
   app witDetail\par 
    \item[transcr: ]
   addSpan damageSpan delSpan fw listTranspose metamark space substJoin\par character data
    \item[{Example}]
  \leavevmode\bgroup\exampleFont \begin{shaded}\noindent\mbox{}{<\textbf{series}\hspace*{6pt}{xml:lang}="{de}">}\mbox{}\newline 
\hspace*{6pt}{<\textbf{title}\hspace*{6pt}{level}="{s}">}Halbgraue Reihe zur Historischen Fachinformatik{</\textbf{title}>}\mbox{}\newline 
\hspace*{6pt}{<\textbf{respStmt}>}\mbox{}\newline 
\hspace*{6pt}\hspace*{6pt}{<\textbf{resp}>}Herausgegeben von{</\textbf{resp}>}\mbox{}\newline 
\hspace*{6pt}\hspace*{6pt}{<\textbf{name}\hspace*{6pt}{type}="{person}">}Manfred Thaller{</\textbf{name}>}\mbox{}\newline 
\hspace*{6pt}\hspace*{6pt}{<\textbf{name}\hspace*{6pt}{type}="{org}">}Max-Planck-Institut für Geschichte{</\textbf{name}>}\mbox{}\newline 
\hspace*{6pt}{</\textbf{respStmt}>}\mbox{}\newline 
\hspace*{6pt}{<\textbf{title}\hspace*{6pt}{level}="{s}">}Serie A: Historische Quellenkunden{</\textbf{title}>}\mbox{}\newline 
\hspace*{6pt}{<\textbf{biblScope}>}Band 11{</\textbf{biblScope}>}\mbox{}\newline 
{</\textbf{series}>}\end{shaded}\egroup 


    \item[{Content model}]
  \mbox{}\hfill\\[-10pt]\begin{Verbatim}[fontsize=\small]
<content>
 <alternate maxOccurs="unbounded"
  minOccurs="0">
  <textNode/>
  <classRef key="model.gLike"/>
  <elementRef key="title"/>
  <classRef key="model.ptrLike"/>
  <elementRef key="editor"/>
  <elementRef key="respStmt"/>
  <elementRef key="biblScope"/>
  <elementRef key="idno"/>
  <elementRef key="textLang"/>
  <classRef key="model.global"/>
  <elementRef key="availability"/>
 </alternate>
</content>
    
\end{Verbatim}

    \item[{Schema Declaration}]
  \mbox{}\hfill\\[-10pt]\begin{Verbatim}[fontsize=\small]
element series
{
   att.global.attributes,
   (
      text
    | model.gLike    | title    | model.ptrLike    | editor    | respStmt    | biblScope    | idno    | textLang    | model.global    | availability   )*
}
\end{Verbatim}

\end{reflist}  \index{seriesStmt=<seriesStmt>|oddindex}
\begin{reflist}
\item[]\begin{specHead}{TEI.seriesStmt}{<seriesStmt> }(series statement) groups information about the series, if any, to which a publication belongs. [\xref{http://www.tei-c.org/release/doc/tei-p5-doc/en/html/HD.html\#HD26}{2.2.5. The Series Statement} \xref{http://www.tei-c.org/release/doc/tei-p5-doc/en/html/HD.html\#HD2}{2.2. The File Description}]\end{specHead} 
    \item[{Module}]
  header
    \item[{Attributes}]
  Attributes att.global (\textit{@xml:id}, \textit{@n}, \textit{@xml:lang}, \textit{@xml:base}, \textit{@xml:space})  (att.global.rendition (\textit{@rend}, \textit{@style}, \textit{@rendition})) (att.global.linking (\textit{@corresp}, \textit{@synch}, \textit{@sameAs}, \textit{@copyOf}, \textit{@next}, \textit{@prev}, \textit{@exclude}, \textit{@select})) (att.global.analytic (\textit{@ana})) (att.global.facs (\textit{@facs})) (att.global.change (\textit{@change})) (att.global.responsibility (\textit{@cert}, \textit{@resp})) (att.global.source (\textit{@source}))
    \item[{Contained by}]
  
    \item[header: ]
   biblFull fileDesc
    \item[{May contain}]
  
    \item[core: ]
   biblScope editor p respStmt title\par 
    \item[header: ]
   idno\par 
    \item[linking: ]
   ab
    \item[{Example}]
  \leavevmode\bgroup\exampleFont \begin{shaded}\noindent\mbox{}{<\textbf{seriesStmt}>}\mbox{}\newline 
\hspace*{6pt}{<\textbf{title}>}Machine-Readable Texts for the Study of Indian Literature{</\textbf{title}>}\mbox{}\newline 
\hspace*{6pt}{<\textbf{respStmt}>}\mbox{}\newline 
\hspace*{6pt}\hspace*{6pt}{<\textbf{resp}>}ed. by{</\textbf{resp}>}\mbox{}\newline 
\hspace*{6pt}\hspace*{6pt}{<\textbf{name}>}Jan Gonda{</\textbf{name}>}\mbox{}\newline 
\hspace*{6pt}{</\textbf{respStmt}>}\mbox{}\newline 
\hspace*{6pt}{<\textbf{biblScope}\hspace*{6pt}{unit}="{volume}">}1.2{</\textbf{biblScope}>}\mbox{}\newline 
\hspace*{6pt}{<\textbf{idno}\hspace*{6pt}{type}="{ISSN}">}0 345 6789{</\textbf{idno}>}\mbox{}\newline 
{</\textbf{seriesStmt}>}\end{shaded}\egroup 


    \item[{Content model}]
  \mbox{}\hfill\\[-10pt]\begin{Verbatim}[fontsize=\small]
<content>
 <alternate>
  <classRef key="model.pLike"
   maxOccurs="unbounded" minOccurs="1"/>
  <sequence>
   <elementRef key="title"
    maxOccurs="unbounded" minOccurs="1"/>
   <alternate maxOccurs="unbounded"
    minOccurs="0">
    <elementRef key="editor"/>
    <elementRef key="respStmt"/>
   </alternate>
   <alternate maxOccurs="unbounded"
    minOccurs="0">
    <elementRef key="idno"/>
    <elementRef key="biblScope"/>
   </alternate>
  </sequence>
 </alternate>
</content>
    
\end{Verbatim}

    \item[{Schema Declaration}]
  \mbox{}\hfill\\[-10pt]\begin{Verbatim}[fontsize=\small]
element seriesStmt
{
   att.global.attributes,
   ( model.pLike+ | ( title+, ( editor | respStmt )*, ( idno | biblScope )* ) )
}
\end{Verbatim}

\end{reflist}  \index{settlement=<settlement>|oddindex}
\begin{reflist}
\item[]\begin{specHead}{TEI.settlement}{<settlement> }contains the name of a settlement such as a city, town, or village identified as a single geo-political or administrative unit. [\xref{http://www.tei-c.org/release/doc/tei-p5-doc/en/html/ND.html\#NDPLAC}{13.2.3. Place Names}]\end{specHead} 
    \item[{Module}]
  namesdates
    \item[{Attributes}]
  Attributes att.global (\textit{@xml:id}, \textit{@n}, \textit{@xml:lang}, \textit{@xml:base}, \textit{@xml:space})  (att.global.rendition (\textit{@rend}, \textit{@style}, \textit{@rendition})) (att.global.linking (\textit{@corresp}, \textit{@synch}, \textit{@sameAs}, \textit{@copyOf}, \textit{@next}, \textit{@prev}, \textit{@exclude}, \textit{@select})) (att.global.analytic (\textit{@ana})) (att.global.facs (\textit{@facs})) (att.global.change (\textit{@change})) (att.global.responsibility (\textit{@cert}, \textit{@resp})) (att.global.source (\textit{@source})) att.naming (\textit{@role}, \textit{@nymRef})  (att.canonical (\textit{@key}, \textit{@ref})) att.typed (\textit{@type}, \textit{@subtype}) att.datable (\textit{@calendar}, \textit{@period})  (att.datable.w3c (\textit{@when}, \textit{@notBefore}, \textit{@notAfter}, \textit{@from}, \textit{@to})) (att.datable.iso (\textit{@when-iso}, \textit{@notBefore-iso}, \textit{@notAfter-iso}, \textit{@from-iso}, \textit{@to-iso})) (att.datable.custom (\textit{@when-custom}, \textit{@notBefore-custom}, \textit{@notAfter-custom}, \textit{@from-custom}, \textit{@to-custom}, \textit{@datingPoint}, \textit{@datingMethod}))
    \item[{Member of}]
  model.placeNamePart
    \item[{Contained by}]
  
    \item[analysis: ]
   cl phr s span\par 
    \item[core: ]
   abbr add addrLine address author bibl biblScope citedRange corr date del desc distinct editor email emph expan foreign gloss head headItem headLabel hi item l label measure meeting mentioned name note num orig p pubPlace publisher q quote ref reg resp rs said sic soCalled speaker stage street term textLang time title unclear\par 
    \item[figures: ]
   cell figDesc\par 
    \item[header: ]
   authority catDesc change classCode correspAction creation distributor edition extent funder geoDecl handNote language licence principal rendition scriptNote sponsor tagUsage typeNote\par 
    \item[linking: ]
   ab seg\par 
    \item[msdescription: ]
   accMat acquisition additions altIdentifier catchwords collation colophon condition custEvent decoNote explicit filiation finalRubric foliation heraldry incipit layout material msIdentifier musicNotation objectType origDate origPlace origin provenance rubric secFol signatures source stamp summary support surrogates watermark\par 
    \item[namesdates: ]
   addName affiliation age birth bloc country death district education faith floruit forename genName geogFeat geogName langKnown location nameLink nationality occupation offset org orgName persName place placeName region residence roleName settlement sex socecStatus surname\par 
    \item[textcrit: ]
   lem rdg wit witDetail witness\par 
    \item[textstructure: ]
   byline closer dateline docAuthor docDate docEdition docImprint imprimatur opener salute signed titlePart trailer\par 
    \item[transcr: ]
   damage fw metamark mod restore retrace secl supplied surplus
    \item[{May contain}]
  
    \item[analysis: ]
   c cl interp interpGrp m pc phr s span spanGrp w\par 
    \item[core: ]
   abbr add address cb choice corr date del distinct email emph expan foreign gap gb gloss graphic hi index lb measure measureGrp media mentioned milestone name note num orig pb ptr ref reg rs sic soCalled term time title unclear\par 
    \item[figures: ]
   figure formula notatedMusic\par 
    \item[gaiji: ]
   g\par 
    \item[header: ]
   idno\par 
    \item[linking: ]
   alt altGrp anchor join joinGrp link linkGrp seg timeline\par 
    \item[msdescription: ]
   catchwords depth dim dimensions height heraldry locus locusGrp material objectType origDate origPlace secFol signatures stamp watermark width\par 
    \item[namesdates: ]
   addName affiliation bloc climate country district forename genName geo geogFeat geogName location nameLink offset orgName persName placeName population region roleName settlement state surname terrain trait\par 
    \item[textcrit: ]
   app witDetail\par 
    \item[transcr: ]
   addSpan am damage damageSpan delSpan ex fw handShift listTranspose metamark mod redo restore retrace secl space subst substJoin supplied surplus undo\par character data
    \item[{Example}]
  \leavevmode\bgroup\exampleFont \begin{shaded}\noindent\mbox{}{<\textbf{placeName}>}\mbox{}\newline 
\hspace*{6pt}{<\textbf{settlement}\hspace*{6pt}{type}="{town}">}Glasgow{</\textbf{settlement}>}\mbox{}\newline 
\hspace*{6pt}{<\textbf{region}>}Scotland{</\textbf{region}>}\mbox{}\newline 
{</\textbf{placeName}>}\end{shaded}\egroup 


    \item[{Content model}]
  \mbox{}\hfill\\[-10pt]\begin{Verbatim}[fontsize=\small]
<content>
 <macroRef key="macro.phraseSeq"/>
</content>
    
\end{Verbatim}

    \item[{Schema Declaration}]
  \mbox{}\hfill\\[-10pt]\begin{Verbatim}[fontsize=\small]
element settlement
{
   att.global.attributes,
   att.naming.attributes,
   att.typed.attributes,
   att.datable.attributes,
   macro.phraseSeq}
\end{Verbatim}

\end{reflist}  \index{sex=<sex>|oddindex}\index{value=@value!<sex>|oddindex}
\begin{reflist}
\item[]\begin{specHead}{TEI.sex}{<sex> }specifies the sex of a person. [\xref{http://www.tei-c.org/release/doc/tei-p5-doc/en/html/ND.html\#NDPERSEpc}{13.3.2.1. Personal Characteristics}]\end{specHead} 
    \item[{Module}]
  namesdates
    \item[{Attributes}]
  Attributes att.global (\textit{@xml:id}, \textit{@n}, \textit{@xml:lang}, \textit{@xml:base}, \textit{@xml:space})  (att.global.rendition (\textit{@rend}, \textit{@style}, \textit{@rendition})) (att.global.linking (\textit{@corresp}, \textit{@synch}, \textit{@sameAs}, \textit{@copyOf}, \textit{@next}, \textit{@prev}, \textit{@exclude}, \textit{@select})) (att.global.analytic (\textit{@ana})) (att.global.facs (\textit{@facs})) (att.global.change (\textit{@change})) (att.global.responsibility (\textit{@cert}, \textit{@resp})) (att.global.source (\textit{@source})) att.editLike (\textit{@evidence}, \textit{@instant})  (att.dimensions (\textit{@unit}, \textit{@quantity}, \textit{@extent}, \textit{@precision}, \textit{@scope}) (att.ranging (\textit{@atLeast}, \textit{@atMost}, \textit{@min}, \textit{@max}, \textit{@confidence})) ) att.datable (\textit{@calendar}, \textit{@period})  (att.datable.w3c (\textit{@when}, \textit{@notBefore}, \textit{@notAfter}, \textit{@from}, \textit{@to})) (att.datable.iso (\textit{@when-iso}, \textit{@notBefore-iso}, \textit{@notAfter-iso}, \textit{@from-iso}, \textit{@to-iso})) (att.datable.custom (\textit{@when-custom}, \textit{@notBefore-custom}, \textit{@notAfter-custom}, \textit{@from-custom}, \textit{@to-custom}, \textit{@datingPoint}, \textit{@datingMethod})) \hfil\\[-10pt]\begin{sansreflist}
    \item[@value]
  supplies a coded value for sex
\begin{reflist}
    \item[{Status}]
  Optional
    \item[{Datatype}]
  1–∞ occurrences of teidata.sex separated by whitespace
    \item[{Note}]
  \par
Values for this attribute may be locally defined by a project, or may refer to an external standard, such as vCard's sex property \url{http://microformats.org/wiki/gender-formats} (in which M indicates male, F female, O other, N none or not applicable, U unknown), or the often used ISO 5218:2004 \textit{Representation of Human Sexes} \url{http://standards.iso.org/ittf/PubliclyAvailableStandards/c036266\textunderscore ISO\textunderscore IEC\textunderscore 5218\textunderscore 2004(E\textunderscore F).zip} (in which 0 indicates unknown; 1 male; 2 female; and 9 not applicable, although the ISO standard is widely considered inadequate); cf. CETH's \textit{Recommendations for Inclusive Data Collection of Trans People} \url{http://transhealth.ucsf.edu/trans?page=lib-data-collection}.
\end{reflist}  
\end{sansreflist}  
    \item[{Member of}]
  model.persStateLike
    \item[{Contained by}]
  
    \item[namesdates: ]
   person personGrp
    \item[{May contain}]
  
    \item[analysis: ]
   c cl interp interpGrp m pc phr s span spanGrp w\par 
    \item[core: ]
   abbr add address cb choice corr date del distinct email emph expan foreign gap gb gloss graphic hi index lb measure measureGrp media mentioned milestone name note num orig pb ptr ref reg rs sic soCalled term time title unclear\par 
    \item[figures: ]
   figure formula notatedMusic\par 
    \item[gaiji: ]
   g\par 
    \item[header: ]
   idno\par 
    \item[linking: ]
   alt altGrp anchor join joinGrp link linkGrp seg timeline\par 
    \item[msdescription: ]
   catchwords depth dim dimensions height heraldry locus locusGrp material objectType origDate origPlace secFol signatures stamp watermark width\par 
    \item[namesdates: ]
   addName affiliation bloc climate country district forename genName geo geogFeat geogName location nameLink offset orgName persName placeName population region roleName settlement state surname terrain trait\par 
    \item[textcrit: ]
   app witDetail\par 
    \item[transcr: ]
   addSpan am damage damageSpan delSpan ex fw handShift listTranspose metamark mod redo restore retrace secl space subst substJoin supplied surplus undo\par character data
    \item[{Note}]
  \par
As with other culturally-constructed traits such as age, the way in which this concept is described in different cultural contexts may vary. The normalizing attributes are provided only as an optional means of simplifying that variety to one or more external standards for purposes of interoperability, or project-internal taxonomies for consistency, and should not be used where that is inappropriate or unhelpful. The content of the element may be used to describe the intended concept in more detail, using plain text. 
    \item[{Example}]
  \leavevmode\bgroup\exampleFont \begin{shaded}\noindent\mbox{}{<\textbf{sex}\hspace*{6pt}{value}="{M}">}male{</\textbf{sex}>}\end{shaded}\egroup 


    \item[{Example}]
  \leavevmode\bgroup\exampleFont \begin{shaded}\noindent\mbox{}{<\textbf{sex}\hspace*{6pt}{value}="{2}">}female{</\textbf{sex}>}\end{shaded}\egroup 


    \item[{Example}]
  \leavevmode\bgroup\exampleFont \begin{shaded}\noindent\mbox{}{<\textbf{sex}\hspace*{6pt}{value}="{I}">}Intersex{</\textbf{sex}>}\end{shaded}\egroup 


    \item[{Example}]
  \leavevmode\bgroup\exampleFont \begin{shaded}\noindent\mbox{}{<\textbf{sex}\hspace*{6pt}{value}="{TG F}">}Female (TransWoman){</\textbf{sex}>}\end{shaded}\egroup 


    \item[{Content model}]
  \mbox{}\hfill\\[-10pt]\begin{Verbatim}[fontsize=\small]
<content>
 <macroRef key="macro.phraseSeq"/>
</content>
    
\end{Verbatim}

    \item[{Schema Declaration}]
  \mbox{}\hfill\\[-10pt]\begin{Verbatim}[fontsize=\small]
element sex
{
   att.global.attributes,
   att.editLike.attributes,
   att.datable.attributes,
   attribute value { list { + } }?,
   macro.phraseSeq}
\end{Verbatim}

\end{reflist}  \index{sic=<sic>|oddindex}
\begin{reflist}
\item[]\begin{specHead}{TEI.sic}{<sic> }(Latin for thus or so) contains text reproduced although apparently incorrect or inaccurate. [\xref{http://www.tei-c.org/release/doc/tei-p5-doc/en/html/CO.html\#COEDCOR}{3.4.1. Apparent Errors}]\end{specHead} 
    \item[{Module}]
  core
    \item[{Attributes}]
  Attributes att.global (\textit{@xml:id}, \textit{@n}, \textit{@xml:lang}, \textit{@xml:base}, \textit{@xml:space})  (att.global.rendition (\textit{@rend}, \textit{@style}, \textit{@rendition})) (att.global.linking (\textit{@corresp}, \textit{@synch}, \textit{@sameAs}, \textit{@copyOf}, \textit{@next}, \textit{@prev}, \textit{@exclude}, \textit{@select})) (att.global.analytic (\textit{@ana})) (att.global.facs (\textit{@facs})) (att.global.change (\textit{@change})) (att.global.responsibility (\textit{@cert}, \textit{@resp})) (att.global.source (\textit{@source}))
    \item[{Member of}]
  model.choicePart model.pPart.transcriptional
    \item[{Contained by}]
  
    \item[analysis: ]
   cl pc phr s w\par 
    \item[core: ]
   abbr add addrLine author bibl biblScope choice citedRange corr date del distinct editor email emph expan foreign gloss head headItem headLabel hi item l label measure mentioned name note num orig p pubPlace publisher q quote ref reg rs said sic soCalled speaker stage street term textLang time title unclear\par 
    \item[figures: ]
   cell\par 
    \item[header: ]
   change distributor edition extent geoDecl handNote licence scriptNote typeNote\par 
    \item[linking: ]
   ab seg\par 
    \item[msdescription: ]
   accMat acquisition additions catchwords collation colophon condition custEvent decoNote explicit filiation finalRubric foliation heraldry incipit layout material musicNotation objectType origDate origPlace origin provenance rubric secFol signatures source stamp summary support surrogates watermark\par 
    \item[namesdates: ]
   addName affiliation birth bloc country death district education faith floruit forename genName geogFeat geogName nameLink nationality occupation offset orgName persName placeName region residence roleName settlement sex socecStatus surname\par 
    \item[textcrit: ]
   lem rdg wit witDetail\par 
    \item[textstructure: ]
   byline closer dateline docAuthor docDate docEdition docImprint imprimatur opener salute signed titlePart trailer\par 
    \item[transcr: ]
   am damage fw metamark mod restore retrace secl supplied surplus
    \item[{May contain}]
  
    \item[analysis: ]
   c cl interp interpGrp m pc phr s span spanGrp w\par 
    \item[core: ]
   abbr add address bibl biblStruct cb choice cit corr date del desc distinct email emph expan foreign gap gb gloss graphic hi index l label lb lg list listBibl measure measureGrp media mentioned milestone name note num orig pb ptr q quote ref reg rs said sic soCalled stage term time title unclear\par 
    \item[figures: ]
   figure formula notatedMusic table\par 
    \item[gaiji: ]
   g\par 
    \item[header: ]
   biblFull idno\par 
    \item[linking: ]
   alt altGrp anchor join joinGrp link linkGrp seg timeline\par 
    \item[msdescription: ]
   catchwords depth dim dimensions height heraldry locus locusGrp material msDesc objectType origDate origPlace secFol signatures stamp watermark width\par 
    \item[namesdates: ]
   addName affiliation bloc climate country district forename genName geo geogFeat geogName listEvent listNym listOrg listPerson listPlace location nameLink offset orgName persName placeName population region roleName settlement state surname terrain trait\par 
    \item[textcrit: ]
   app listApp listWit witDetail\par 
    \item[textstructure: ]
   floatingText\par 
    \item[transcr: ]
   addSpan am damage damageSpan delSpan ex fw handShift listTranspose metamark mod redo restore retrace secl space subst substJoin supplied surplus undo\par character data
    \item[{Example}]
  \leavevmode\bgroup\exampleFont \begin{shaded}\noindent\mbox{}for his nose was as sharp as\mbox{}\newline 
 a pen, and {<\textbf{sic}>}a Table{</\textbf{sic}>} of green fields.\end{shaded}\egroup 


    \item[{Example}]
  If all that is desired is to call attention to the apparent problem in the copy text, <sic> may be used alone:\leavevmode\bgroup\exampleFont \begin{shaded}\noindent\mbox{}I don't know, Juan. It's so far in the past now\mbox{}\newline 
 — how {<\textbf{sic}>}we can{</\textbf{sic}>} prove or disprove anyone's theories?\end{shaded}\egroup 


    \item[{Example}]
  It is also possible, using the <choice> and <corr> elements, to provide a corrected reading:\leavevmode\bgroup\exampleFont \begin{shaded}\noindent\mbox{}I don't know, Juan. It's so far in the past now\mbox{}\newline 
 — how {<\textbf{choice}>}\mbox{}\newline 
\hspace*{6pt}{<\textbf{sic}>}we can{</\textbf{sic}>}\mbox{}\newline 
\hspace*{6pt}{<\textbf{corr}>}can we{</\textbf{corr}>}\mbox{}\newline 
{</\textbf{choice}>} prove or disprove anyone's theories?\end{shaded}\egroup 


    \item[{Example}]
  \leavevmode\bgroup\exampleFont \begin{shaded}\noindent\mbox{}for his nose was as sharp as\mbox{}\newline 
 a pen, and {<\textbf{choice}>}\mbox{}\newline 
\hspace*{6pt}{<\textbf{sic}>}a Table{</\textbf{sic}>}\mbox{}\newline 
\hspace*{6pt}{<\textbf{corr}>}a' babbld{</\textbf{corr}>}\mbox{}\newline 
{</\textbf{choice}>} of green fields.\end{shaded}\egroup 


    \item[{Content model}]
  \mbox{}\hfill\\[-10pt]\begin{Verbatim}[fontsize=\small]
<content>
 <macroRef key="macro.paraContent"/>
</content>
    
\end{Verbatim}

    \item[{Schema Declaration}]
  \mbox{}\hfill\\[-10pt]\begin{Verbatim}[fontsize=\small]
element sic { att.global.attributes, macro.paraContent }
\end{Verbatim}

\end{reflist}  \index{signatures=<signatures>|oddindex}
\begin{reflist}
\item[]\begin{specHead}{TEI.signatures}{<signatures> }contains discussion of the leaf or quire signatures found within a codex. [\xref{http://www.tei-c.org/release/doc/tei-p5-doc/en/html/MS.html\#msmisc}{10.3.7. Catchwords, Signatures, Secundo Folio}]\end{specHead} 
    \item[{Module}]
  msdescription
    \item[{Attributes}]
  Attributes att.global (\textit{@xml:id}, \textit{@n}, \textit{@xml:lang}, \textit{@xml:base}, \textit{@xml:space})  (att.global.rendition (\textit{@rend}, \textit{@style}, \textit{@rendition})) (att.global.linking (\textit{@corresp}, \textit{@synch}, \textit{@sameAs}, \textit{@copyOf}, \textit{@next}, \textit{@prev}, \textit{@exclude}, \textit{@select})) (att.global.analytic (\textit{@ana})) (att.global.facs (\textit{@facs})) (att.global.change (\textit{@change})) (att.global.responsibility (\textit{@cert}, \textit{@resp})) (att.global.source (\textit{@source}))
    \item[{Member of}]
  model.pPart.msdesc
    \item[{Contained by}]
  
    \item[analysis: ]
   cl phr s span\par 
    \item[core: ]
   abbr add addrLine author biblScope citedRange corr date del desc distinct editor email emph expan foreign gloss head headItem headLabel hi item l label measure meeting mentioned name note num orig p pubPlace publisher q quote ref reg resp rs said sic soCalled speaker stage street term textLang time title unclear\par 
    \item[figures: ]
   cell figDesc\par 
    \item[header: ]
   authority catDesc change classCode creation distributor edition extent funder geoDecl handNote language licence principal rendition scriptNote sponsor tagUsage typeNote\par 
    \item[linking: ]
   ab seg\par 
    \item[msdescription: ]
   accMat acquisition additions catchwords collation colophon condition custEvent decoNote explicit filiation finalRubric foliation heraldry incipit layout material musicNotation objectType origDate origPlace origin provenance rubric secFol signatures source stamp summary support surrogates watermark\par 
    \item[namesdates: ]
   addName affiliation age birth bloc country death district education faith floruit forename genName geogFeat geogName langKnown nameLink nationality occupation offset orgName persName placeName region residence roleName settlement sex socecStatus surname\par 
    \item[textcrit: ]
   lem rdg wit witDetail witness\par 
    \item[textstructure: ]
   byline closer dateline docAuthor docDate docEdition docImprint imprimatur opener salute signed titlePart trailer\par 
    \item[transcr: ]
   damage fw metamark mod restore retrace secl supplied surplus
    \item[{May contain}]
  
    \item[analysis: ]
   c cl interp interpGrp m pc phr s span spanGrp w\par 
    \item[core: ]
   abbr add address bibl biblStruct cb choice cit corr date del desc distinct email emph expan foreign gap gb gloss graphic hi index l label lb lg list listBibl measure measureGrp media mentioned milestone name note num orig p pb ptr q quote ref reg rs said sic soCalled sp stage term time title unclear\par 
    \item[figures: ]
   figure formula notatedMusic table\par 
    \item[gaiji: ]
   g\par 
    \item[header: ]
   biblFull idno\par 
    \item[linking: ]
   ab alt altGrp anchor join joinGrp link linkGrp seg timeline\par 
    \item[msdescription: ]
   catchwords depth dim dimensions height heraldry locus locusGrp material msDesc objectType origDate origPlace secFol signatures stamp watermark width\par 
    \item[namesdates: ]
   addName affiliation bloc climate country district forename genName geo geogFeat geogName listEvent listNym listOrg listPerson listPlace location nameLink offset orgName persName placeName population region roleName settlement state surname terrain trait\par 
    \item[textcrit: ]
   app listApp listWit witDetail\par 
    \item[textstructure: ]
   floatingText\par 
    \item[transcr: ]
   addSpan am damage damageSpan delSpan ex fw handShift listTranspose metamark mod redo restore retrace secl space subst substJoin supplied surplus undo\par character data
    \item[{Example}]
  \leavevmode\bgroup\exampleFont \begin{shaded}\noindent\mbox{}{<\textbf{signatures}>}Quire and leaf signatures in letters, [b]-v, and roman\mbox{}\newline 
 numerals; those in quires 10 (1) and 17 (s) in red ink and different\mbox{}\newline 
 from others; every third quire also signed with red crayon in arabic\mbox{}\newline 
 numerals in the center lower margin of the first leaf recto: "2" for\mbox{}\newline 
 quire 4 (f. 19), "3" for quire 7 (f. 43); "4," barely visible, for\mbox{}\newline 
 quire 10 (f. 65), "5," in a later hand, for quire 13 (f. 89), "6," in\mbox{}\newline 
 a later hand, for quire 16 (f. 113).{</\textbf{signatures}>}\end{shaded}\egroup 


    \item[{Schematron}]
   <sch:assert role="nonfatal"  test="ancestor::tei:msDesc">WARNING: deprecated use of element — The <sch:name/> element will not be allowed outside of msDesc as of 2018-10-01.</sch:assert>
    \item[{Content model}]
  \mbox{}\hfill\\[-10pt]\begin{Verbatim}[fontsize=\small]
<content>
 <macroRef key="macro.specialPara"/>
</content>
    
\end{Verbatim}

    \item[{Schema Declaration}]
  \mbox{}\hfill\\[-10pt]\begin{Verbatim}[fontsize=\small]
element signatures { att.global.attributes, macro.specialPara }
\end{Verbatim}

\end{reflist}  \index{signed=<signed>|oddindex}
\begin{reflist}
\item[]\begin{specHead}{TEI.signed}{<signed> }(signature) contains the closing salutation, etc., appended to a foreword, dedicatory epistle, or other division of a text. [\xref{http://www.tei-c.org/release/doc/tei-p5-doc/en/html/DS.html\#DSOC}{4.2.2. Openers and Closers}]\end{specHead} 
    \item[{Module}]
  textstructure
    \item[{Attributes}]
  Attributes att.global (\textit{@xml:id}, \textit{@n}, \textit{@xml:lang}, \textit{@xml:base}, \textit{@xml:space})  (att.global.rendition (\textit{@rend}, \textit{@style}, \textit{@rendition})) (att.global.linking (\textit{@corresp}, \textit{@synch}, \textit{@sameAs}, \textit{@copyOf}, \textit{@next}, \textit{@prev}, \textit{@exclude}, \textit{@select})) (att.global.analytic (\textit{@ana})) (att.global.facs (\textit{@facs})) (att.global.change (\textit{@change})) (att.global.responsibility (\textit{@cert}, \textit{@resp})) (att.global.source (\textit{@source})) att.written (\textit{@hand}) 
    \item[{Member of}]
  model.divBottomPart model.divTopPart 
    \item[{Contained by}]
  
    \item[core: ]
   lg list\par 
    \item[figures: ]
   figure table\par 
    \item[textstructure: ]
   back body closer div front group opener postscript
    \item[{May contain}]
  
    \item[analysis: ]
   c cl interp interpGrp m pc phr s span spanGrp w\par 
    \item[core: ]
   abbr add address bibl biblStruct cb choice cit corr date del desc distinct email emph expan foreign gap gb gloss graphic hi index l label lb lg list listBibl measure measureGrp media mentioned milestone name note num orig pb ptr q quote ref reg rs said sic soCalled stage term time title unclear\par 
    \item[figures: ]
   figure formula notatedMusic table\par 
    \item[gaiji: ]
   g\par 
    \item[header: ]
   biblFull idno\par 
    \item[linking: ]
   alt altGrp anchor join joinGrp link linkGrp seg timeline\par 
    \item[msdescription: ]
   catchwords depth dim dimensions height heraldry locus locusGrp material msDesc objectType origDate origPlace secFol signatures stamp watermark width\par 
    \item[namesdates: ]
   addName affiliation bloc climate country district forename genName geo geogFeat geogName listEvent listNym listOrg listPerson listPlace location nameLink offset orgName persName placeName population region roleName settlement state surname terrain trait\par 
    \item[textcrit: ]
   app listApp listWit witDetail\par 
    \item[textstructure: ]
   floatingText\par 
    \item[transcr: ]
   addSpan am damage damageSpan delSpan ex fw handShift listTranspose metamark mod redo restore retrace secl space subst substJoin supplied surplus undo\par character data
    \item[{Example}]
  \leavevmode\bgroup\exampleFont \begin{shaded}\noindent\mbox{}{<\textbf{signed}>}Thine to command {<\textbf{name}>}Humph. Moseley{</\textbf{name}>}\mbox{}\newline 
{</\textbf{signed}>}\end{shaded}\egroup 


    \item[{Example}]
  \leavevmode\bgroup\exampleFont \begin{shaded}\noindent\mbox{}{<\textbf{closer}>}\mbox{}\newline 
\hspace*{6pt}{<\textbf{signed}>}Sign'd and Seal'd,\mbox{}\newline 
\hspace*{6pt}{<\textbf{list}>}\mbox{}\newline 
\hspace*{6pt}\hspace*{6pt}\hspace*{6pt}{<\textbf{item}>}John Bull,{</\textbf{item}>}\mbox{}\newline 
\hspace*{6pt}\hspace*{6pt}\hspace*{6pt}{<\textbf{item}>}Nic. Frog.{</\textbf{item}>}\mbox{}\newline 
\hspace*{6pt}\hspace*{6pt}{</\textbf{list}>}\mbox{}\newline 
\hspace*{6pt}{</\textbf{signed}>}\mbox{}\newline 
{</\textbf{closer}>}\end{shaded}\egroup 


    \item[{Content model}]
  \mbox{}\hfill\\[-10pt]\begin{Verbatim}[fontsize=\small]
<content>
 <macroRef key="macro.paraContent"/>
</content>
    
\end{Verbatim}

    \item[{Schema Declaration}]
  \mbox{}\hfill\\[-10pt]\begin{Verbatim}[fontsize=\small]
element signed
{
   att.global.attributes,
   att.written.attributes,
   macro.paraContent}
\end{Verbatim}

\end{reflist}  \index{soCalled=<soCalled>|oddindex}
\begin{reflist}
\item[]\begin{specHead}{TEI.soCalled}{<soCalled> }contains a word or phrase for which the author or narrator indicates a disclaiming of responsibility, for example by the use of scare quotes or italics. [\xref{http://www.tei-c.org/release/doc/tei-p5-doc/en/html/CO.html\#COHQQ}{3.3.3. Quotation}]\end{specHead} 
    \item[{Module}]
  core
    \item[{Attributes}]
  Attributes att.global (\textit{@xml:id}, \textit{@n}, \textit{@xml:lang}, \textit{@xml:base}, \textit{@xml:space})  (att.global.rendition (\textit{@rend}, \textit{@style}, \textit{@rendition})) (att.global.linking (\textit{@corresp}, \textit{@synch}, \textit{@sameAs}, \textit{@copyOf}, \textit{@next}, \textit{@prev}, \textit{@exclude}, \textit{@select})) (att.global.analytic (\textit{@ana})) (att.global.facs (\textit{@facs})) (att.global.change (\textit{@change})) (att.global.responsibility (\textit{@cert}, \textit{@resp})) (att.global.source (\textit{@source}))
    \item[{Member of}]
  model.emphLike
    \item[{Contained by}]
  
    \item[analysis: ]
   cl phr s span\par 
    \item[core: ]
   abbr add addrLine author bibl biblScope citedRange corr date del desc distinct editor email emph expan foreign gloss head headItem headLabel hi item l label measure meeting mentioned name note num orig p pubPlace publisher q quote ref reg resp rs said sic soCalled speaker stage street term textLang time title unclear\par 
    \item[figures: ]
   cell figDesc\par 
    \item[header: ]
   authority catDesc change classCode creation distributor edition extent funder geoDecl handNote language licence principal rendition scriptNote sponsor tagUsage typeNote\par 
    \item[linking: ]
   ab seg\par 
    \item[msdescription: ]
   accMat acquisition additions catchwords collation colophon condition custEvent decoNote explicit filiation finalRubric foliation heraldry incipit layout material musicNotation objectType origDate origPlace origin provenance rubric secFol signatures source stamp summary support surrogates watermark\par 
    \item[namesdates: ]
   addName affiliation age birth bloc country death district education faith floruit forename genName geogFeat geogName langKnown nameLink nationality occupation offset orgName persName placeName region residence roleName settlement sex socecStatus surname\par 
    \item[textcrit: ]
   lem rdg wit witDetail witness\par 
    \item[textstructure: ]
   byline closer dateline docAuthor docDate docEdition docImprint imprimatur opener salute signed titlePart trailer\par 
    \item[transcr: ]
   damage fw metamark mod restore retrace secl supplied surplus
    \item[{May contain}]
  
    \item[analysis: ]
   c cl interp interpGrp m pc phr s span spanGrp w\par 
    \item[core: ]
   abbr add address cb choice corr date del distinct email emph expan foreign gap gb gloss graphic hi index lb measure measureGrp media mentioned milestone name note num orig pb ptr ref reg rs sic soCalled term time title unclear\par 
    \item[figures: ]
   figure formula notatedMusic\par 
    \item[gaiji: ]
   g\par 
    \item[header: ]
   idno\par 
    \item[linking: ]
   alt altGrp anchor join joinGrp link linkGrp seg timeline\par 
    \item[msdescription: ]
   catchwords depth dim dimensions height heraldry locus locusGrp material objectType origDate origPlace secFol signatures stamp watermark width\par 
    \item[namesdates: ]
   addName affiliation bloc climate country district forename genName geo geogFeat geogName location nameLink offset orgName persName placeName population region roleName settlement state surname terrain trait\par 
    \item[textcrit: ]
   app witDetail\par 
    \item[transcr: ]
   addSpan am damage damageSpan delSpan ex fw handShift listTranspose metamark mod redo restore retrace secl space subst substJoin supplied surplus undo\par character data
    \item[{Example}]
  \leavevmode\bgroup\exampleFont \begin{shaded}\noindent\mbox{}To edge his way along\mbox{}\newline 
 the crowded paths of life, warning all human sympathy to keep its distance, was what the\mbox{}\newline 
 knowing ones call {<\textbf{soCalled}>}nuts{</\textbf{soCalled}>} to Scrooge.\end{shaded}\egroup 


    \item[{Content model}]
  \mbox{}\hfill\\[-10pt]\begin{Verbatim}[fontsize=\small]
<content>
 <macroRef key="macro.phraseSeq"/>
</content>
    
\end{Verbatim}

    \item[{Schema Declaration}]
  \mbox{}\hfill\\[-10pt]\begin{Verbatim}[fontsize=\small]
element soCalled { att.global.attributes, macro.phraseSeq }
\end{Verbatim}

\end{reflist}  \index{socecStatus=<socecStatus>|oddindex}\index{scheme=@scheme!<socecStatus>|oddindex}\index{code=@code!<socecStatus>|oddindex}
\begin{reflist}
\item[]\begin{specHead}{TEI.socecStatus}{<socecStatus> }(socio-economic status) contains an informal description of a person's perceived social or economic status. [\xref{http://www.tei-c.org/release/doc/tei-p5-doc/en/html/CC.html\#CCAHPA}{15.2.2. The Participant Description}]\end{specHead} 
    \item[{Module}]
  namesdates
    \item[{Attributes}]
  Attributes att.global (\textit{@xml:id}, \textit{@n}, \textit{@xml:lang}, \textit{@xml:base}, \textit{@xml:space})  (att.global.rendition (\textit{@rend}, \textit{@style}, \textit{@rendition})) (att.global.linking (\textit{@corresp}, \textit{@synch}, \textit{@sameAs}, \textit{@copyOf}, \textit{@next}, \textit{@prev}, \textit{@exclude}, \textit{@select})) (att.global.analytic (\textit{@ana})) (att.global.facs (\textit{@facs})) (att.global.change (\textit{@change})) (att.global.responsibility (\textit{@cert}, \textit{@resp})) (att.global.source (\textit{@source})) att.datable (\textit{@calendar}, \textit{@period})  (att.datable.w3c (\textit{@when}, \textit{@notBefore}, \textit{@notAfter}, \textit{@from}, \textit{@to})) (att.datable.iso (\textit{@when-iso}, \textit{@notBefore-iso}, \textit{@notAfter-iso}, \textit{@from-iso}, \textit{@to-iso})) (att.datable.custom (\textit{@when-custom}, \textit{@notBefore-custom}, \textit{@notAfter-custom}, \textit{@from-custom}, \textit{@to-custom}, \textit{@datingPoint}, \textit{@datingMethod})) att.editLike (\textit{@evidence}, \textit{@instant})  (att.dimensions (\textit{@unit}, \textit{@quantity}, \textit{@extent}, \textit{@precision}, \textit{@scope}) (att.ranging (\textit{@atLeast}, \textit{@atMost}, \textit{@min}, \textit{@max}, \textit{@confidence})) ) att.naming (\textit{@role}, \textit{@nymRef})  (att.canonical (\textit{@key}, \textit{@ref})) \hfil\\[-10pt]\begin{sansreflist}
    \item[@scheme]
  identifies the classification system or taxonomy in use, for example by pointing to a locally-defined <taxonomy> element or by supplying a URI for an externally-defined system.
\begin{reflist}
    \item[{Status}]
  Optional
    \item[{Datatype}]
  teidata.pointer
\end{reflist}  
    \item[@code]
  identifies a status code defined within the classification system or taxonomy defined by the {\itshape scheme} attribute.
\begin{reflist}
    \item[{Status}]
  Optional
    \item[{Datatype}]
  teidata.pointer
\end{reflist}  
\end{sansreflist}  
    \item[{Member of}]
  model.persStateLike
    \item[{Contained by}]
  
    \item[namesdates: ]
   person personGrp
    \item[{May contain}]
  
    \item[analysis: ]
   c cl interp interpGrp m pc phr s span spanGrp w\par 
    \item[core: ]
   abbr add address cb choice corr date del distinct email emph expan foreign gap gb gloss graphic hi index lb measure measureGrp media mentioned milestone name note num orig pb ptr ref reg rs sic soCalled term time title unclear\par 
    \item[figures: ]
   figure formula notatedMusic\par 
    \item[gaiji: ]
   g\par 
    \item[header: ]
   idno\par 
    \item[linking: ]
   alt altGrp anchor join joinGrp link linkGrp seg timeline\par 
    \item[msdescription: ]
   catchwords depth dim dimensions height heraldry locus locusGrp material objectType origDate origPlace secFol signatures stamp watermark width\par 
    \item[namesdates: ]
   addName affiliation bloc climate country district forename genName geo geogFeat geogName location nameLink offset orgName persName placeName population region roleName settlement state surname terrain trait\par 
    \item[textcrit: ]
   app witDetail\par 
    \item[transcr: ]
   addSpan am damage damageSpan delSpan ex fw handShift listTranspose metamark mod redo restore retrace secl space subst substJoin supplied surplus undo\par character data
    \item[{Note}]
  \par
The content of this element may be used as an alternative to the more formal specification made possible by its attributes; it may also be used to supplement the formal specification with commentary or clarification.
    \item[{Example}]
  \leavevmode\bgroup\exampleFont \begin{shaded}\noindent\mbox{}{<\textbf{socecStatus}\hspace*{6pt}{code}="{\#ab1}"\hspace*{6pt}{scheme}="{\#rg}"/>}\end{shaded}\egroup 


    \item[{Example}]
  \leavevmode\bgroup\exampleFont \begin{shaded}\noindent\mbox{}{<\textbf{socecStatus}>}Status AB1 in the RG Classification scheme{</\textbf{socecStatus}>}\end{shaded}\egroup 


    \item[{Content model}]
  \mbox{}\hfill\\[-10pt]\begin{Verbatim}[fontsize=\small]
<content>
 <macroRef key="macro.phraseSeq"/>
</content>
    
\end{Verbatim}

    \item[{Schema Declaration}]
  \mbox{}\hfill\\[-10pt]\begin{Verbatim}[fontsize=\small]
element socecStatus
{
   att.global.attributes,
   att.datable.attributes,
   att.editLike.attributes,
   att.naming.attributes,
   attribute scheme { text }?,
   attribute code { text }?,
   macro.phraseSeq}
\end{Verbatim}

\end{reflist}  \index{source=<source>|oddindex}
\begin{reflist}
\item[]\begin{specHead}{TEI.source}{<source> }describes the original source for the information contained with a manuscript description. [\xref{http://www.tei-c.org/release/doc/tei-p5-doc/en/html/MS.html\#msrh}{10.9.1.1. Record History}]\end{specHead} 
    \item[{Module}]
  msdescription
    \item[{Attributes}]
  Attributes att.global (\textit{@xml:id}, \textit{@n}, \textit{@xml:lang}, \textit{@xml:base}, \textit{@xml:space})  (att.global.rendition (\textit{@rend}, \textit{@style}, \textit{@rendition})) (att.global.linking (\textit{@corresp}, \textit{@synch}, \textit{@sameAs}, \textit{@copyOf}, \textit{@next}, \textit{@prev}, \textit{@exclude}, \textit{@select})) (att.global.analytic (\textit{@ana})) (att.global.facs (\textit{@facs})) (att.global.change (\textit{@change})) (att.global.responsibility (\textit{@cert}, \textit{@resp})) (att.global.source (\textit{@source}))
    \item[{Contained by}]
  
    \item[msdescription: ]
   recordHist
    \item[{May contain}]
  
    \item[analysis: ]
   c cl interp interpGrp m pc phr s span spanGrp w\par 
    \item[core: ]
   abbr add address bibl biblStruct cb choice cit corr date del desc distinct email emph expan foreign gap gb gloss graphic hi index l label lb lg list listBibl measure measureGrp media mentioned milestone name note num orig p pb ptr q quote ref reg rs said sic soCalled sp stage term time title unclear\par 
    \item[figures: ]
   figure formula notatedMusic table\par 
    \item[gaiji: ]
   g\par 
    \item[header: ]
   biblFull idno\par 
    \item[linking: ]
   ab alt altGrp anchor join joinGrp link linkGrp seg timeline\par 
    \item[msdescription: ]
   catchwords depth dim dimensions height heraldry locus locusGrp material msDesc objectType origDate origPlace secFol signatures stamp watermark width\par 
    \item[namesdates: ]
   addName affiliation bloc climate country district forename genName geo geogFeat geogName listEvent listNym listOrg listPerson listPlace location nameLink offset orgName persName placeName population region roleName settlement state surname terrain trait\par 
    \item[textcrit: ]
   app listApp listWit witDetail\par 
    \item[textstructure: ]
   floatingText\par 
    \item[transcr: ]
   addSpan am damage damageSpan delSpan ex fw handShift listTranspose metamark mod redo restore retrace secl space subst substJoin supplied surplus undo\par character data
    \item[{Example}]
  \leavevmode\bgroup\exampleFont \begin{shaded}\noindent\mbox{}{<\textbf{source}>}Derived from {<\textbf{ref}>}Stanley (1960){</\textbf{ref}>}\mbox{}\newline 
{</\textbf{source}>}\end{shaded}\egroup 


    \item[{Content model}]
  \mbox{}\hfill\\[-10pt]\begin{Verbatim}[fontsize=\small]
<content>
 <macroRef key="macro.specialPara"/>
</content>
    
\end{Verbatim}

    \item[{Schema Declaration}]
  \mbox{}\hfill\\[-10pt]\begin{Verbatim}[fontsize=\small]
element source { att.global.attributes, macro.specialPara }
\end{Verbatim}

\end{reflist}  \index{sourceDesc=<sourceDesc>|oddindex}
\begin{reflist}
\item[]\begin{specHead}{TEI.sourceDesc}{<sourceDesc> }(source description) describes the source from which an electronic text was derived or generated, typically a bibliographic description in the case of a digitized text, or a phrase such as "born digital" for a text which has no previous existence. [\xref{http://www.tei-c.org/release/doc/tei-p5-doc/en/html/HD.html\#HD3}{2.2.7. The Source Description}]\end{specHead} 
    \item[{Module}]
  header
    \item[{Attributes}]
  Attributes att.global (\textit{@xml:id}, \textit{@n}, \textit{@xml:lang}, \textit{@xml:base}, \textit{@xml:space})  (att.global.rendition (\textit{@rend}, \textit{@style}, \textit{@rendition})) (att.global.linking (\textit{@corresp}, \textit{@synch}, \textit{@sameAs}, \textit{@copyOf}, \textit{@next}, \textit{@prev}, \textit{@exclude}, \textit{@select})) (att.global.analytic (\textit{@ana})) (att.global.facs (\textit{@facs})) (att.global.change (\textit{@change})) (att.global.responsibility (\textit{@cert}, \textit{@resp})) (att.global.source (\textit{@source})) att.declarable (\textit{@default}) 
    \item[{Contained by}]
  
    \item[header: ]
   biblFull fileDesc
    \item[{May contain}]
  
    \item[core: ]
   bibl biblStruct list listBibl p\par 
    \item[figures: ]
   table\par 
    \item[header: ]
   biblFull\par 
    \item[linking: ]
   ab\par 
    \item[msdescription: ]
   msDesc\par 
    \item[namesdates: ]
   listEvent listNym listOrg listPerson listPlace\par 
    \item[textcrit: ]
   listApp listWit
    \item[{Example}]
  \leavevmode\bgroup\exampleFont \begin{shaded}\noindent\mbox{}{<\textbf{sourceDesc}>}\mbox{}\newline 
\hspace*{6pt}{<\textbf{bibl}>}\mbox{}\newline 
\hspace*{6pt}\hspace*{6pt}{<\textbf{title}\hspace*{6pt}{level}="{a}">}The Interesting story of the Children in the Wood{</\textbf{title}>}. In\mbox{}\newline 
\hspace*{6pt}{<\textbf{author}>}Victor E Neuberg{</\textbf{author}>}, {<\textbf{title}>}The Penny Histories{</\textbf{title}>}.\mbox{}\newline 
\hspace*{6pt}{<\textbf{publisher}>}OUP{</\textbf{publisher}>}\mbox{}\newline 
\hspace*{6pt}\hspace*{6pt}{<\textbf{date}>}1968{</\textbf{date}>}. {</\textbf{bibl}>}\mbox{}\newline 
{</\textbf{sourceDesc}>}\end{shaded}\egroup 


    \item[{Example}]
  \leavevmode\bgroup\exampleFont \begin{shaded}\noindent\mbox{}{<\textbf{sourceDesc}>}\mbox{}\newline 
\hspace*{6pt}{<\textbf{p}>}Born digital: no previous source exists.{</\textbf{p}>}\mbox{}\newline 
{</\textbf{sourceDesc}>}\end{shaded}\egroup 


    \item[{Content model}]
  \mbox{}\hfill\\[-10pt]\begin{Verbatim}[fontsize=\small]
<content>
 <alternate>
  <classRef key="model.pLike"
   maxOccurs="unbounded" minOccurs="1"/>
  <alternate maxOccurs="unbounded"
   minOccurs="1">
   <classRef key="model.biblLike"/>
   <classRef key="model.sourceDescPart"/>
   <classRef key="model.listLike"/>
  </alternate>
 </alternate>
</content>
    
\end{Verbatim}

    \item[{Schema Declaration}]
  \mbox{}\hfill\\[-10pt]\begin{Verbatim}[fontsize=\small]
element sourceDesc
{
   att.global.attributes,
   att.declarable.attributes,
   (
      model.pLike+
    | ( model.biblLike | model.sourceDescPart | model.listLike )+
   )
}
\end{Verbatim}

\end{reflist}  \index{sourceDoc=<sourceDoc>|oddindex}
\begin{reflist}
\item[]\begin{specHead}{TEI.sourceDoc}{<sourceDoc> }contains a transcription or other representation of a single source document potentially forming part of a \textit{dossier génétique} or collection of sources. [\xref{http://www.tei-c.org/release/doc/tei-p5-doc/en/html/PH.html\#PHFAX}{11.1. Digital Facsimiles} \xref{http://www.tei-c.org/release/doc/tei-p5-doc/en/html/PH.html\#PHZLAB}{11.2.2. Embedded Transcription}]\end{specHead} 
    \item[{Module}]
  transcr
    \item[{Attributes}]
  Attributes att.global (\textit{@xml:id}, \textit{@n}, \textit{@xml:lang}, \textit{@xml:base}, \textit{@xml:space})  (att.global.rendition (\textit{@rend}, \textit{@style}, \textit{@rendition})) (att.global.linking (\textit{@corresp}, \textit{@synch}, \textit{@sameAs}, \textit{@copyOf}, \textit{@next}, \textit{@prev}, \textit{@exclude}, \textit{@select})) (att.global.analytic (\textit{@ana})) (att.global.facs (\textit{@facs})) (att.global.change (\textit{@change})) (att.global.responsibility (\textit{@cert}, \textit{@resp})) (att.global.source (\textit{@source})) att.declaring (\textit{@decls}) 
    \item[{Member of}]
  model.resourceLike
    \item[{Contained by}]
  
    \item[core: ]
   teiCorpus\par 
    \item[textstructure: ]
   TEI
    \item[{May contain}]
  
    \item[analysis: ]
   interp interpGrp span spanGrp\par 
    \item[core: ]
   cb gap gb graphic index lb media milestone note pb\par 
    \item[figures: ]
   figure formula notatedMusic\par 
    \item[linking: ]
   alt altGrp anchor join joinGrp link linkGrp timeline\par 
    \item[textcrit: ]
   app witDetail\par 
    \item[transcr: ]
   addSpan damageSpan delSpan fw listTranspose metamark space substJoin surface surfaceGrp
    \item[{Note}]
  \par
This element may be used as an alternative to <facsimile> for TEI documents containing only page images, or for documents containing both images and transcriptions. Transcriptions may be provided within the <surface> elements making up a source document, in parallel with them as part of a <text> element, or in both places if the encoder wishes to distinguish these two modes of transcription.
    \item[{Example}]
  \leavevmode\bgroup\exampleFont \begin{shaded}\noindent\mbox{}{<\textbf{sourceDoc}>}\mbox{}\newline 
\hspace*{6pt}{<\textbf{surfaceGrp}\hspace*{6pt}{n}="{leaf1}">}\mbox{}\newline 
\hspace*{6pt}\hspace*{6pt}{<\textbf{surface}\hspace*{6pt}{facs}="{page1.png}">}\mbox{}\newline 
\hspace*{6pt}\hspace*{6pt}\hspace*{6pt}{<\textbf{zone}>}All the writing on page 1{</\textbf{zone}>}\mbox{}\newline 
\hspace*{6pt}\hspace*{6pt}{</\textbf{surface}>}\mbox{}\newline 
\hspace*{6pt}\hspace*{6pt}{<\textbf{surface}>}\mbox{}\newline 
\hspace*{6pt}\hspace*{6pt}\hspace*{6pt}{<\textbf{graphic}\hspace*{6pt}{url}="{page2-highRes.png}"/>}\mbox{}\newline 
\hspace*{6pt}\hspace*{6pt}\hspace*{6pt}{<\textbf{graphic}\hspace*{6pt}{url}="{page2-lowRes.png}"/>}\mbox{}\newline 
\hspace*{6pt}\hspace*{6pt}\hspace*{6pt}{<\textbf{zone}>}\mbox{}\newline 
\hspace*{6pt}\hspace*{6pt}\hspace*{6pt}\hspace*{6pt}{<\textbf{line}>}A line of writing on page 2{</\textbf{line}>}\mbox{}\newline 
\hspace*{6pt}\hspace*{6pt}\hspace*{6pt}\hspace*{6pt}{<\textbf{line}>}Another line of writing on page 2{</\textbf{line}>}\mbox{}\newline 
\hspace*{6pt}\hspace*{6pt}\hspace*{6pt}{</\textbf{zone}>}\mbox{}\newline 
\hspace*{6pt}\hspace*{6pt}{</\textbf{surface}>}\mbox{}\newline 
\hspace*{6pt}{</\textbf{surfaceGrp}>}\mbox{}\newline 
{</\textbf{sourceDoc}>}\end{shaded}\egroup 


    \item[{Content model}]
  \mbox{}\hfill\\[-10pt]\begin{Verbatim}[fontsize=\small]
<content>
 <alternate maxOccurs="unbounded"
  minOccurs="1">
  <classRef key="model.global"/>
  <classRef key="model.graphicLike"/>
  <elementRef key="surface"/>
  <elementRef key="surfaceGrp"/>
 </alternate>
</content>
    
\end{Verbatim}

    \item[{Schema Declaration}]
  \mbox{}\hfill\\[-10pt]\begin{Verbatim}[fontsize=\small]
element sourceDoc
{
   att.global.attributes,
   att.declaring.attributes,
   ( model.global | model.graphicLike | surface | surfaceGrp )+
}
\end{Verbatim}

\end{reflist}  \index{sp=<sp>|oddindex}
\begin{reflist}
\item[]\begin{specHead}{TEI.sp}{<sp> }(speech) contains an individual speech in a performance text, or a passage presented as such in a prose or verse text. [\xref{http://www.tei-c.org/release/doc/tei-p5-doc/en/html/CO.html\#CODR}{3.12.2. Core Tags for Drama} \xref{http://www.tei-c.org/release/doc/tei-p5-doc/en/html/CO.html\#CODV}{3.12. Passages of Verse or Drama} \xref{http://www.tei-c.org/release/doc/tei-p5-doc/en/html/DR.html\#DRSP}{7.2.2. Speeches and Speakers}]\end{specHead} 
    \item[{Module}]
  core
    \item[{Attributes}]
  Attributes att.global (\textit{@xml:id}, \textit{@n}, \textit{@xml:lang}, \textit{@xml:base}, \textit{@xml:space})  (att.global.rendition (\textit{@rend}, \textit{@style}, \textit{@rendition})) (att.global.linking (\textit{@corresp}, \textit{@synch}, \textit{@sameAs}, \textit{@copyOf}, \textit{@next}, \textit{@prev}, \textit{@exclude}, \textit{@select})) (att.global.analytic (\textit{@ana})) (att.global.facs (\textit{@facs})) (att.global.change (\textit{@change})) (att.global.responsibility (\textit{@cert}, \textit{@resp})) (att.global.source (\textit{@source})) att.ascribed (\textit{@who}) 
    \item[{Member of}]
  model.divPart
    \item[{Contained by}]
  
    \item[core: ]
   item note q quote said stage\par 
    \item[figures: ]
   cell figure\par 
    \item[header: ]
   change handNote licence scriptNote typeNote\par 
    \item[msdescription: ]
   accMat acquisition additions collation condition custEvent decoNote filiation foliation layout musicNotation origin provenance signatures source summary support surrogates\par 
    \item[namesdates: ]
   occupation\par 
    \item[textcrit: ]
   lem rdg\par 
    \item[textstructure: ]
   argument body div epigraph postscript\par 
    \item[transcr: ]
   metamark
    \item[{May contain}]
  
    \item[analysis: ]
   interp interpGrp span spanGrp\par 
    \item[core: ]
   cb cit gap gb index l lb lg list milestone note p pb q quote said speaker stage\par 
    \item[figures: ]
   figure notatedMusic table\par 
    \item[linking: ]
   ab alt altGrp anchor join joinGrp link linkGrp timeline\par 
    \item[namesdates: ]
   listEvent listNym listOrg listPerson listPlace\par 
    \item[textcrit: ]
   app listApp listWit witDetail\par 
    \item[textstructure: ]
   floatingText\par 
    \item[transcr: ]
   addSpan damageSpan delSpan fw listTranspose metamark space substJoin
    \item[{Note}]
  \par
The {\itshape who} attribute on this element may be used either in addition to the <speaker> element or as an alternative.
    \item[{Example}]
  \leavevmode\bgroup\exampleFont \begin{shaded}\noindent\mbox{}{<\textbf{sp}>}\mbox{}\newline 
\hspace*{6pt}{<\textbf{speaker}>}The reverend Doctor Opimian{</\textbf{speaker}>}\mbox{}\newline 
\hspace*{6pt}{<\textbf{p}>}I do not think I have named a single unpresentable fish.{</\textbf{p}>}\mbox{}\newline 
{</\textbf{sp}>}\mbox{}\newline 
{<\textbf{sp}>}\mbox{}\newline 
\hspace*{6pt}{<\textbf{speaker}>}Mr Gryll{</\textbf{speaker}>}\mbox{}\newline 
\hspace*{6pt}{<\textbf{p}>}Bream, Doctor: there is not much to be said for bream.{</\textbf{p}>}\mbox{}\newline 
{</\textbf{sp}>}\mbox{}\newline 
{<\textbf{sp}>}\mbox{}\newline 
\hspace*{6pt}{<\textbf{speaker}>}The Reverend Doctor Opimian{</\textbf{speaker}>}\mbox{}\newline 
\hspace*{6pt}{<\textbf{p}>}On the contrary, sir, I think there is much to be said for him. In the first place [...]{</\textbf{p}>}\mbox{}\newline 
\hspace*{6pt}{<\textbf{p}>}Fish, Miss Gryll — I could discourse to you on fish by the hour: but for the present I\mbox{}\newline 
\hspace*{6pt}\hspace*{6pt} will forbear [...]{</\textbf{p}>}\mbox{}\newline 
{</\textbf{sp}>}\end{shaded}\egroup 


    \item[{Content model}]
  \mbox{}\hfill\\[-10pt]\begin{Verbatim}[fontsize=\small]
<content>
 <sequence>
  <classRef key="model.global"
   maxOccurs="unbounded" minOccurs="0"/>
  <sequence minOccurs="0">
   <elementRef key="speaker"/>
   <classRef key="model.global"
    maxOccurs="unbounded" minOccurs="0"/>
  </sequence>
  <sequence maxOccurs="unbounded"
   minOccurs="1">
   <alternate>
    <elementRef key="lg"/>
    <classRef key="model.lLike"/>
    <classRef key="model.pLike"/>
    <classRef key="model.listLike"/>
    <classRef key="model.stageLike"/>
    <classRef key="model.qLike"/>
   </alternate>
   <classRef key="model.global"
    maxOccurs="unbounded" minOccurs="0"/>
  </sequence>
 </sequence>
</content>
    
\end{Verbatim}

    \item[{Schema Declaration}]
  \mbox{}\hfill\\[-10pt]\begin{Verbatim}[fontsize=\small]
element sp
{
   att.global.attributes,
   att.ascribed.attributes,
   (
      model.global*,
      ( speaker, model.global* )?,
      (
         (
            lg          | model.lLike          | model.pLike          | model.listLike          | model.stageLike          | model.qLike         ),
         model.global*
      )+
   )
}
\end{Verbatim}

\end{reflist}  \index{space=<space>|oddindex}\index{resp=@resp!<space>|oddindex}\index{dim=@dim!<space>|oddindex}
\begin{reflist}
\item[]\begin{specHead}{TEI.space}{<space> }indicates the location of a significant space in the text. [\xref{http://www.tei-c.org/release/doc/tei-p5-doc/en/html/PH.html\#PHSP}{11.5.1. Space}]\end{specHead} 
    \item[{Module}]
  transcr
    \item[{Attributes}]
  Attributes att.typed (\textit{@type}, \textit{@subtype}) att.dimensions (\textit{@unit}, \textit{@quantity}, \textit{@extent}, \textit{@precision}, \textit{@scope})  (att.ranging (\textit{@atLeast}, \textit{@atMost}, \textit{@min}, \textit{@max}, \textit{@confidence})) att.global (@xml:id, @n, @xml:lang, @xml:base, @xml:space) att.global.rendition (@rend, @style, @rendition) att.global.linking (@corresp, @synch, @sameAs, @copyOf, @next, @prev, @exclude, @select) att.global.analytic (@ana) att.global.facs (@facs) att.global.change (@change) att.global.responsibility (\unusedattribute{resp}, @cert) att.global.source (@source) \hfil\\[-10pt]\begin{sansreflist}
    \item[@resp]
  (responsible party) (responsible party) indicates the individual responsible for identifying and measuring the space
\begin{reflist}
    \item[{Derived from}]
  att.global.responsibility
    \item[{Status}]
  Optional
    \item[{Datatype}]
  1–∞ occurrences of teidata.pointer separated by whitespace
\end{reflist}  
    \item[@dim]
  (dimension) indicates whether the space is horizontal or vertical.
\begin{reflist}
    \item[{Status}]
  Recommended
    \item[{Datatype}]
  teidata.enumerated
    \item[{Legal values are:}]
  \begin{description}

\item[{horizontal}]the space is horizontal.
\item[{vertical}]the space is vertical.
\end{description} 
    \item[{Note}]
  \par
For irregular shapes in two dimensions, the value for this attribute should reflect the more important of the two dimensions. In conventional left-right scripts, a space with both vertical and horizontal components should be classed as vertical.
\end{reflist}  
\end{sansreflist}  
    \item[{Member of}]
  model.global.edit
    \item[{Contained by}]
  
    \item[analysis: ]
   cl m phr s span w\par 
    \item[core: ]
   abbr add addrLine address author bibl biblScope cit citedRange corr date del distinct editor email emph expan foreign gloss head headItem headLabel hi imprint item l label lg list measure mentioned name note num orig p pubPlace publisher q quote ref reg resp rs said series sic soCalled sp speaker stage street term textLang time title unclear\par 
    \item[figures: ]
   cell figure table\par 
    \item[header: ]
   authority change classCode distributor edition extent funder geoDecl handNote language licence principal scriptNote sponsor typeNote\par 
    \item[linking: ]
   ab seg\par 
    \item[msdescription: ]
   accMat acquisition additions catchwords collation colophon condition custEvent decoNote explicit filiation finalRubric foliation heraldry incipit layout material msItem musicNotation objectType origDate origPlace origin provenance rubric secFol signatures source stamp summary support surrogates watermark\par 
    \item[namesdates: ]
   addName affiliation age birth bloc country death district education faith floruit forename genName geogFeat geogName langKnown nameLink nationality occupation offset orgName persName person personGrp placeName region residence roleName settlement sex socecStatus surname\par 
    \item[textcrit: ]
   lem rdg wit witDetail\par 
    \item[textstructure: ]
   argument back body byline closer dateline div docAuthor docDate docEdition docImprint docTitle epigraph floatingText front group imprimatur opener postscript salute signed text titlePage titlePart trailer\par 
    \item[transcr: ]
   damage fw line metamark mod restore retrace secl sourceDoc supplied surface surfaceGrp surplus zone
    \item[{May contain}]
  
    \item[core: ]
   desc
    \item[{Note}]
  \par
This element should be used wherever it is desired to record an unusual space in the source text, e.g. space left for a word to be filled in later, for later rubrication, etc. It is not intended to be used to mark normal inter-word space or the like.
    \item[{Example}]
  \leavevmode\bgroup\exampleFont \begin{shaded}\noindent\mbox{}By god if wommen had writen storyes\mbox{}\newline 
 As {<\textbf{space}\hspace*{6pt}{quantity}="{7}"\hspace*{6pt}{unit}="{minims}"/>} han within her oratoryes\end{shaded}\egroup 


    \item[{Example}]
  \leavevmode\bgroup\exampleFont \begin{shaded}\noindent\mbox{}στρατηλάτ{<\textbf{space}\hspace*{6pt}{quantity}="{1}"\hspace*{6pt}{unit}="{chars}"/>}ου\end{shaded}\egroup 


    \item[{Content model}]
  \mbox{}\hfill\\[-10pt]\begin{Verbatim}[fontsize=\small]
<content>
 <alternate maxOccurs="unbounded"
  minOccurs="0">
  <classRef key="model.descLike"/>
  <classRef key="model.certLike"/>
 </alternate>
</content>
    
\end{Verbatim}

    \item[{Schema Declaration}]
  \mbox{}\hfill\\[-10pt]\begin{Verbatim}[fontsize=\small]
element space
{
   att.global.attribute.xmlid,
   att.global.attribute.n,
   att.global.attribute.xmllang,
   att.global.attribute.xmlbase,
   att.global.attribute.xmlspace,
   att.global.rendition.attribute.rend,
   att.global.rendition.attribute.style,
   att.global.rendition.attribute.rendition,
   att.global.linking.attribute.corresp,
   att.global.linking.attribute.synch,
   att.global.linking.attribute.sameAs,
   att.global.linking.attribute.copyOf,
   att.global.linking.attribute.next,
   att.global.linking.attribute.prev,
   att.global.linking.attribute.exclude,
   att.global.linking.attribute.select,
   att.global.analytic.attribute.ana,
   att.global.facs.attribute.facs,
   att.global.change.attribute.change,
   att.global.responsibility.attribute.cert,
   att.global.source.attribute.source,
   att.typed.attributes,
   att.dimensions.attributes,
   attribute resp { list { + } }?,
   attribute dim { "horizontal" | "vertical" }?,
   ( model.descLike | model.certLike )*
}
\end{Verbatim}

\end{reflist}  \index{span=<span>|oddindex}\index{from=@from!<span>|oddindex}\index{to=@to!<span>|oddindex}
\begin{reflist}
\item[]\begin{specHead}{TEI.span}{<span> }associates an interpretative annotation directly with a span of text. [\xref{http://www.tei-c.org/release/doc/tei-p5-doc/en/html/AI.html\#AISP}{17.3. Spans and Interpretations}]\end{specHead} 
    \item[{Module}]
  analysis
    \item[{Attributes}]
  Attributes att.global (\textit{@xml:id}, \textit{@n}, \textit{@xml:lang}, \textit{@xml:base}, \textit{@xml:space})  (att.global.rendition (\textit{@rend}, \textit{@style}, \textit{@rendition})) (att.global.linking (\textit{@corresp}, \textit{@synch}, \textit{@sameAs}, \textit{@copyOf}, \textit{@next}, \textit{@prev}, \textit{@exclude}, \textit{@select})) (att.global.analytic (\textit{@ana})) (att.global.facs (\textit{@facs})) (att.global.change (\textit{@change})) (att.global.responsibility (\textit{@cert}, \textit{@resp})) (att.global.source (\textit{@source})) att.interpLike (\textit{@type}, \textit{@inst}) att.pointing (\textit{@targetLang}, \textit{@target}, \textit{@evaluate}) \hfil\\[-10pt]\begin{sansreflist}
    \item[@from]
  gives the identifier of the node which is the starting point of the span of text being annotated; if not accompanied by a {\itshape to} attribute, gives the identifier of the node of the entire span of text being annotated.
\begin{reflist}
    \item[{Status}]
  Optional
    \item[{Datatype}]
  teidata.pointer
\end{reflist}  
    \item[@to]
  gives the identifier of the node which is the end-point of the span of text being annotated.
\begin{reflist}
    \item[{Status}]
  Optional
    \item[{Datatype}]
  teidata.pointer
\end{reflist}  
\end{sansreflist}  
    \item[{Member of}]
  model.global.meta 
    \item[{Contained by}]
  
    \item[analysis: ]
   cl m phr s span spanGrp w\par 
    \item[core: ]
   abbr add addrLine address author bibl biblScope cit citedRange corr date del distinct editor email emph expan foreign gloss head headItem headLabel hi imprint item l label lg list measure mentioned name note num orig p pubPlace publisher q quote ref reg resp rs said series sic soCalled sp speaker stage street term textLang time title unclear\par 
    \item[figures: ]
   cell figure table\par 
    \item[header: ]
   authority change classCode distributor edition extent funder geoDecl handNote language licence principal scriptNote sponsor typeNote\par 
    \item[linking: ]
   ab seg\par 
    \item[msdescription: ]
   accMat acquisition additions catchwords collation colophon condition custEvent decoNote explicit filiation finalRubric foliation heraldry incipit layout material msItem musicNotation objectType origDate origPlace origin provenance rubric secFol signatures source stamp summary support surrogates watermark\par 
    \item[namesdates: ]
   addName affiliation age birth bloc country death district education faith floruit forename genName geogFeat geogName langKnown nameLink nationality occupation offset orgName persName person personGrp placeName region residence roleName settlement sex socecStatus surname\par 
    \item[textcrit: ]
   lem rdg wit witDetail\par 
    \item[textstructure: ]
   argument back body byline closer dateline div docAuthor docDate docEdition docImprint docTitle epigraph floatingText front group imprimatur opener postscript salute signed text titlePage titlePart trailer\par 
    \item[transcr: ]
   damage fw line metamark mod restore retrace secl sourceDoc supplied surface surfaceGrp surplus zone
    \item[{May contain}]
  
    \item[analysis: ]
   interp interpGrp span spanGrp\par 
    \item[core: ]
   abbr address cb choice date distinct email emph expan foreign gap gb gloss hi index lb measure measureGrp mentioned milestone name note num pb ptr ref rs soCalled term time title\par 
    \item[figures: ]
   figure notatedMusic\par 
    \item[header: ]
   idno\par 
    \item[linking: ]
   alt altGrp anchor join joinGrp link linkGrp timeline\par 
    \item[msdescription: ]
   catchwords depth dim dimensions height heraldry locus locusGrp material objectType origDate origPlace secFol signatures stamp watermark width\par 
    \item[namesdates: ]
   addName affiliation bloc climate country district forename genName geo geogFeat geogName location nameLink offset orgName persName placeName population region roleName settlement state surname terrain trait\par 
    \item[textcrit: ]
   app witDetail\par 
    \item[transcr: ]
   addSpan am damageSpan delSpan ex fw listTranspose metamark space subst substJoin\par character data
    \item[{Example}]
  \leavevmode\bgroup\exampleFont \begin{shaded}\noindent\mbox{}{<\textbf{p}\hspace*{6pt}{xml:id}="{para2}">}(The "aftermath" starts here){</\textbf{p}>}\mbox{}\newline 
{<\textbf{p}\hspace*{6pt}{xml:id}="{para3}">}(The "aftermath" continues here){</\textbf{p}>}\mbox{}\newline 
{<\textbf{p}\hspace*{6pt}{xml:id}="{para4}">}(The "aftermath" ends in this paragraph){</\textbf{p}>}\mbox{}\newline 
\textit{<!-- ... -->}\mbox{}\newline 
{<\textbf{span}\hspace*{6pt}{from}="{\#para2}"\hspace*{6pt}{to}="{\#para4}"\mbox{}\newline 
\hspace*{6pt}{type}="{structure}">}aftermath{</\textbf{span}>}\end{shaded}\egroup 


    \item[{Schematron}]
   <s:report test="@from and @target">Only one of the attributes @target and @from may be supplied on <s:name/> </s:report>
    \item[{Schematron}]
   <s:report test="@to and @target">Only one of the attributes @target and @to may be supplied on <s:name/> </s:report>
    \item[{Schematron}]
   <s:report test="@to and not(@from)">If @to is supplied on <s:name/>, @from must be supplied as well</s:report>
    \item[{Schematron}]
   <s:report test="contains(normalize-space(@to),' ') or contains(normalize-space(@from),'   ')">The attributes @to and @from on <s:name/> may each contain only a single value</s:report>
    \item[{Content model}]
  \mbox{}\hfill\\[-10pt]\begin{Verbatim}[fontsize=\small]
<content>
 <macroRef key="macro.phraseSeq.limited"/>
</content>
    
\end{Verbatim}

    \item[{Schema Declaration}]
  \mbox{}\hfill\\[-10pt]\begin{Verbatim}[fontsize=\small]
element span
{
   att.global.attributes,
   att.interpLike.attributes,
   att.pointing.attributes,
   attribute from { text }?,
   attribute to { text }?,
   macro.phraseSeq.limited}
\end{Verbatim}

\end{reflist}  \index{spanGrp=<spanGrp>|oddindex}
\begin{reflist}
\item[]\begin{specHead}{TEI.spanGrp}{<spanGrp> }(span group) collects together span tags. [\xref{http://www.tei-c.org/release/doc/tei-p5-doc/en/html/AI.html\#AISP}{17.3. Spans and Interpretations}]\end{specHead} 
    \item[{Module}]
  analysis
    \item[{Attributes}]
  Attributes att.global (\textit{@xml:id}, \textit{@n}, \textit{@xml:lang}, \textit{@xml:base}, \textit{@xml:space})  (att.global.rendition (\textit{@rend}, \textit{@style}, \textit{@rendition})) (att.global.linking (\textit{@corresp}, \textit{@synch}, \textit{@sameAs}, \textit{@copyOf}, \textit{@next}, \textit{@prev}, \textit{@exclude}, \textit{@select})) (att.global.analytic (\textit{@ana})) (att.global.facs (\textit{@facs})) (att.global.change (\textit{@change})) (att.global.responsibility (\textit{@cert}, \textit{@resp})) (att.global.source (\textit{@source})) att.interpLike (\textit{@type}, \textit{@inst}) 
    \item[{Member of}]
  model.global.meta
    \item[{Contained by}]
  
    \item[analysis: ]
   cl m phr s span w\par 
    \item[core: ]
   abbr add addrLine address author bibl biblScope cit citedRange corr date del distinct editor email emph expan foreign gloss head headItem headLabel hi imprint item l label lg list measure mentioned name note num orig p pubPlace publisher q quote ref reg resp rs said series sic soCalled sp speaker stage street term textLang time title unclear\par 
    \item[figures: ]
   cell figure table\par 
    \item[header: ]
   authority change classCode distributor edition extent funder geoDecl handNote language licence principal scriptNote sponsor typeNote\par 
    \item[linking: ]
   ab seg\par 
    \item[msdescription: ]
   accMat acquisition additions catchwords collation colophon condition custEvent decoNote explicit filiation finalRubric foliation heraldry incipit layout material msItem musicNotation objectType origDate origPlace origin provenance rubric secFol signatures source stamp summary support surrogates watermark\par 
    \item[namesdates: ]
   addName affiliation age birth bloc country death district education faith floruit forename genName geogFeat geogName langKnown nameLink nationality occupation offset orgName persName person personGrp placeName region residence roleName settlement sex socecStatus surname\par 
    \item[textcrit: ]
   lem rdg wit witDetail\par 
    \item[textstructure: ]
   argument back body byline closer dateline div docAuthor docDate docEdition docImprint docTitle epigraph floatingText front group imprimatur opener postscript salute signed text titlePage titlePart trailer\par 
    \item[transcr: ]
   damage fw line metamark mod restore retrace secl sourceDoc supplied surface surfaceGrp surplus zone
    \item[{May contain}]
  
    \item[analysis: ]
   span
    \item[{Example}]
  \leavevmode\bgroup\exampleFont \begin{shaded}\noindent\mbox{}{<\textbf{u}\hspace*{6pt}{xml:id}="{UU1}">}Can I have ten oranges and a kilo of bananas please?{</\textbf{u}>}\mbox{}\newline 
{<\textbf{u}\hspace*{6pt}{xml:id}="{UU2}">}Yes, anything else?{</\textbf{u}>}\mbox{}\newline 
{<\textbf{u}\hspace*{6pt}{xml:id}="{UU3}">}No thanks.{</\textbf{u}>}\mbox{}\newline 
{<\textbf{u}\hspace*{6pt}{xml:id}="{UU4}">}That'll be dollar forty.{</\textbf{u}>}\mbox{}\newline 
{<\textbf{u}\hspace*{6pt}{xml:id}="{UU5}">}Two dollars{</\textbf{u}>}\mbox{}\newline 
{<\textbf{u}\hspace*{6pt}{xml:id}="{UU6}">}Sixty, eighty, two dollars.\mbox{}\newline 
{<\textbf{anchor}\hspace*{6pt}{xml:id}="{UU6e}"/>}Thank you.{<\textbf{anchor}\hspace*{6pt}{xml:id}="{UU6f}"/>}\mbox{}\newline 
{</\textbf{u}>}\mbox{}\newline 
{<\textbf{spanGrp}\hspace*{6pt}{type}="{transactions}">}\mbox{}\newline 
\hspace*{6pt}{<\textbf{span}\hspace*{6pt}{from}="{\#UU1}">}sale request{</\textbf{span}>}\mbox{}\newline 
\hspace*{6pt}{<\textbf{span}\hspace*{6pt}{from}="{\#UU2}"\hspace*{6pt}{to}="{\#UU3}">}sale compliance{</\textbf{span}>}\mbox{}\newline 
\hspace*{6pt}{<\textbf{span}\hspace*{6pt}{from}="{\#UU4}">}sale{</\textbf{span}>}\mbox{}\newline 
\hspace*{6pt}{<\textbf{span}\hspace*{6pt}{from}="{\#UU5}"\hspace*{6pt}{to}="{\#UU6}">}purchase{</\textbf{span}>}\mbox{}\newline 
\hspace*{6pt}{<\textbf{span}\hspace*{6pt}{from}="{\#UU6e}"\hspace*{6pt}{to}="{\#UU6f}">}purchase closure{</\textbf{span}>}\mbox{}\newline 
{</\textbf{spanGrp}>}\end{shaded}\egroup 


    \item[{Content model}]
  \mbox{}\hfill\\[-10pt]\begin{Verbatim}[fontsize=\small]
<content>
 <elementRef key="span"
  maxOccurs="unbounded" minOccurs="0"/>
</content>
    
\end{Verbatim}

    \item[{Schema Declaration}]
  \mbox{}\hfill\\[-10pt]\begin{Verbatim}[fontsize=\small]
element spanGrp { att.global.attributes, att.interpLike.attributes, span* }
\end{Verbatim}

\end{reflist}  \index{speaker=<speaker>|oddindex}
\begin{reflist}
\item[]\begin{specHead}{TEI.speaker}{<speaker> }contains a specialized form of heading or label, giving the name of one or more speakers in a dramatic text or fragment. [\xref{http://www.tei-c.org/release/doc/tei-p5-doc/en/html/CO.html\#CODR}{3.12.2. Core Tags for Drama}]\end{specHead} 
    \item[{Module}]
  core
    \item[{Attributes}]
  Attributes att.global (\textit{@xml:id}, \textit{@n}, \textit{@xml:lang}, \textit{@xml:base}, \textit{@xml:space})  (att.global.rendition (\textit{@rend}, \textit{@style}, \textit{@rendition})) (att.global.linking (\textit{@corresp}, \textit{@synch}, \textit{@sameAs}, \textit{@copyOf}, \textit{@next}, \textit{@prev}, \textit{@exclude}, \textit{@select})) (att.global.analytic (\textit{@ana})) (att.global.facs (\textit{@facs})) (att.global.change (\textit{@change})) (att.global.responsibility (\textit{@cert}, \textit{@resp})) (att.global.source (\textit{@source}))
    \item[{Contained by}]
  
    \item[core: ]
   sp
    \item[{May contain}]
  
    \item[analysis: ]
   c cl interp interpGrp m pc phr s span spanGrp w\par 
    \item[core: ]
   abbr add address cb choice corr date del distinct email emph expan foreign gap gb gloss graphic hi index lb measure measureGrp media mentioned milestone name note num orig pb ptr ref reg rs sic soCalled term time title unclear\par 
    \item[figures: ]
   figure formula notatedMusic\par 
    \item[gaiji: ]
   g\par 
    \item[header: ]
   idno\par 
    \item[linking: ]
   alt altGrp anchor join joinGrp link linkGrp seg timeline\par 
    \item[msdescription: ]
   catchwords depth dim dimensions height heraldry locus locusGrp material objectType origDate origPlace secFol signatures stamp watermark width\par 
    \item[namesdates: ]
   addName affiliation bloc climate country district forename genName geo geogFeat geogName location nameLink offset orgName persName placeName population region roleName settlement state surname terrain trait\par 
    \item[textcrit: ]
   app witDetail\par 
    \item[transcr: ]
   addSpan am damage damageSpan delSpan ex fw handShift listTranspose metamark mod redo restore retrace secl space subst substJoin supplied surplus undo\par character data
    \item[{Example}]
  \leavevmode\bgroup\exampleFont \begin{shaded}\noindent\mbox{}{<\textbf{sp}\hspace*{6pt}{who}="{\#ni \#rsa}">}\mbox{}\newline 
\hspace*{6pt}{<\textbf{speaker}>}Nancy and Robert{</\textbf{speaker}>}\mbox{}\newline 
\hspace*{6pt}{<\textbf{stage}\hspace*{6pt}{type}="{delivery}">}(speaking simultaneously){</\textbf{stage}>}\mbox{}\newline 
\hspace*{6pt}{<\textbf{p}>}The future? ...{</\textbf{p}>}\mbox{}\newline 
{</\textbf{sp}>}\mbox{}\newline 
{<\textbf{list}\hspace*{6pt}{type}="{speakers}">}\mbox{}\newline 
\hspace*{6pt}{<\textbf{item}\hspace*{6pt}{xml:id}="{ni}"/>}\mbox{}\newline 
\hspace*{6pt}{<\textbf{item}\hspace*{6pt}{xml:id}="{rsa}"/>}\mbox{}\newline 
{</\textbf{list}>}\end{shaded}\egroup 


    \item[{Content model}]
  \mbox{}\hfill\\[-10pt]\begin{Verbatim}[fontsize=\small]
<content>
 <macroRef key="macro.phraseSeq"/>
</content>
    
\end{Verbatim}

    \item[{Schema Declaration}]
  \mbox{}\hfill\\[-10pt]\begin{Verbatim}[fontsize=\small]
element speaker { att.global.attributes, macro.phraseSeq }
\end{Verbatim}

\end{reflist}  \index{sponsor=<sponsor>|oddindex}
\begin{reflist}
\item[]\begin{specHead}{TEI.sponsor}{<sponsor> }specifies the name of a sponsoring organization or institution. [\xref{http://www.tei-c.org/release/doc/tei-p5-doc/en/html/HD.html\#HD21}{2.2.1. The Title Statement}]\end{specHead} 
    \item[{Module}]
  header
    \item[{Attributes}]
  Attributes att.global (\textit{@xml:id}, \textit{@n}, \textit{@xml:lang}, \textit{@xml:base}, \textit{@xml:space})  (att.global.rendition (\textit{@rend}, \textit{@style}, \textit{@rendition})) (att.global.linking (\textit{@corresp}, \textit{@synch}, \textit{@sameAs}, \textit{@copyOf}, \textit{@next}, \textit{@prev}, \textit{@exclude}, \textit{@select})) (att.global.analytic (\textit{@ana})) (att.global.facs (\textit{@facs})) (att.global.change (\textit{@change})) (att.global.responsibility (\textit{@cert}, \textit{@resp})) (att.global.source (\textit{@source})) att.canonical (\textit{@key}, \textit{@ref}) 
    \item[{Member of}]
  model.respLike 
    \item[{Contained by}]
  
    \item[core: ]
   bibl monogr\par 
    \item[header: ]
   editionStmt titleStmt\par 
    \item[msdescription: ]
   msItem
    \item[{May contain}]
  
    \item[analysis: ]
   interp interpGrp span spanGrp\par 
    \item[core: ]
   abbr address cb choice date distinct email emph expan foreign gap gb gloss hi index lb measure measureGrp mentioned milestone name note num pb ptr ref rs soCalled term time title\par 
    \item[figures: ]
   figure notatedMusic\par 
    \item[header: ]
   idno\par 
    \item[linking: ]
   alt altGrp anchor join joinGrp link linkGrp timeline\par 
    \item[msdescription: ]
   catchwords depth dim dimensions height heraldry locus locusGrp material objectType origDate origPlace secFol signatures stamp watermark width\par 
    \item[namesdates: ]
   addName affiliation bloc climate country district forename genName geo geogFeat geogName location nameLink offset orgName persName placeName population region roleName settlement state surname terrain trait\par 
    \item[textcrit: ]
   app witDetail\par 
    \item[transcr: ]
   addSpan am damageSpan delSpan ex fw listTranspose metamark space subst substJoin\par character data
    \item[{Note}]
  \par
Sponsors give their intellectual authority to a project; they are to be distinguished from \textit{funders} (see element <funder>), who provide the funding but do not necessarily take intellectual responsibility.
    \item[{Example}]
  \leavevmode\bgroup\exampleFont \begin{shaded}\noindent\mbox{}{<\textbf{sponsor}>}Association for Computers and the Humanities{</\textbf{sponsor}>}\mbox{}\newline 
{<\textbf{sponsor}>}Association for Computational Linguistics{</\textbf{sponsor}>}\mbox{}\newline 
{<\textbf{sponsor}\hspace*{6pt}{ref}="{http://www.allc.org/}">}Association for Literary and Linguistic Computing{</\textbf{sponsor}>}\end{shaded}\egroup 


    \item[{Content model}]
  \mbox{}\hfill\\[-10pt]\begin{Verbatim}[fontsize=\small]
<content>
 <macroRef key="macro.phraseSeq.limited"/>
</content>
    
\end{Verbatim}

    \item[{Schema Declaration}]
  \mbox{}\hfill\\[-10pt]\begin{Verbatim}[fontsize=\small]
element sponsor
{
   att.global.attributes,
   att.canonical.attributes,
   macro.phraseSeq.limited}
\end{Verbatim}

\end{reflist}  \index{stage=<stage>|oddindex}\index{type=@type!<stage>|oddindex}
\begin{reflist}
\item[]\begin{specHead}{TEI.stage}{<stage> }(stage direction) contains any kind of stage direction within a dramatic text or fragment. [\xref{http://www.tei-c.org/release/doc/tei-p5-doc/en/html/CO.html\#CODR}{3.12.2. Core Tags for Drama} \xref{http://www.tei-c.org/release/doc/tei-p5-doc/en/html/CO.html\#CODV}{3.12. Passages of Verse or Drama} \xref{http://www.tei-c.org/release/doc/tei-p5-doc/en/html/DR.html\#DRSTA}{7.2.4. Stage Directions}]\end{specHead} 
    \item[{Module}]
  core
    \item[{Attributes}]
  Attributes att.ascribed (\textit{@who}) att.global (\textit{@xml:id}, \textit{@n}, \textit{@xml:lang}, \textit{@xml:base}, \textit{@xml:space})  (att.global.rendition (\textit{@rend}, \textit{@style}, \textit{@rendition})) (att.global.linking (\textit{@corresp}, \textit{@synch}, \textit{@sameAs}, \textit{@copyOf}, \textit{@next}, \textit{@prev}, \textit{@exclude}, \textit{@select})) (att.global.analytic (\textit{@ana})) (att.global.facs (\textit{@facs})) (att.global.change (\textit{@change})) (att.global.responsibility (\textit{@cert}, \textit{@resp})) (att.global.source (\textit{@source})) att.placement (\textit{@place}) \hfil\\[-10pt]\begin{sansreflist}
    \item[@type]
  indicates the kind of stage direction.
\begin{reflist}
    \item[{Status}]
  Recommended
    \item[{Datatype}]
  0–∞ occurrences of teidata.enumerated separated by whitespace
    \item[{Suggested values include:}]
  \begin{description}

\item[{setting}]describes a setting.
\item[{entrance}]describes an entrance.
\item[{exit}]describes an exit.
\item[{business}]describes stage business.
\item[{novelistic}]is a narrative, motivating stage direction.
\item[{delivery}]describes how a character speaks.
\item[{modifier}]gives some detail about a character.
\item[{location}]describes a location.
\item[{mixed}]more than one of the above
\end{description} 
    \item[{Note}]
  \par
If the value mixed is used, it must be the only value. Multiple values may however be supplied if a single stage direction performs multiple functions, for example is both an entrance and a modifier.
\end{reflist}  
\end{sansreflist}  
    \item[{Member of}]
  model.stageLike
    \item[{Contained by}]
  
    \item[core: ]
   add corr del desc emph head hi item l lg meeting note orig p q quote ref reg said sic sp stage title unclear\par 
    \item[figures: ]
   cell figDesc figure\par 
    \item[header: ]
   change handNote licence rendition scriptNote tagUsage typeNote\par 
    \item[linking: ]
   ab seg\par 
    \item[msdescription: ]
   accMat acquisition additions collation condition custEvent decoNote filiation foliation layout musicNotation origin provenance signatures source summary support surrogates\par 
    \item[namesdates: ]
   occupation\par 
    \item[textcrit: ]
   lem rdg witness\par 
    \item[textstructure: ]
   argument body div docEdition epigraph imprimatur postscript salute signed titlePart trailer\par 
    \item[transcr: ]
   damage metamark mod restore retrace secl supplied surplus
    \item[{May contain}]
  
    \item[analysis: ]
   c cl interp interpGrp m pc phr s span spanGrp w\par 
    \item[core: ]
   abbr add address bibl biblStruct cb choice cit corr date del desc distinct email emph expan foreign gap gb gloss graphic hi index l label lb lg list listBibl measure measureGrp media mentioned milestone name note num orig p pb ptr q quote ref reg rs said sic soCalled sp stage term time title unclear\par 
    \item[figures: ]
   figure formula notatedMusic table\par 
    \item[gaiji: ]
   g\par 
    \item[header: ]
   biblFull idno\par 
    \item[linking: ]
   ab alt altGrp anchor join joinGrp link linkGrp seg timeline\par 
    \item[msdescription: ]
   catchwords depth dim dimensions height heraldry locus locusGrp material msDesc objectType origDate origPlace secFol signatures stamp watermark width\par 
    \item[namesdates: ]
   addName affiliation bloc climate country district forename genName geo geogFeat geogName listEvent listNym listOrg listPerson listPlace location nameLink offset orgName persName placeName population region roleName settlement state surname terrain trait\par 
    \item[textcrit: ]
   app listApp listWit witDetail\par 
    \item[textstructure: ]
   floatingText\par 
    \item[transcr: ]
   addSpan am damage damageSpan delSpan ex fw handShift listTranspose metamark mod redo restore retrace secl space subst substJoin supplied surplus undo\par character data
    \item[{Note}]
  \par
The {\itshape who} attribute may be used to indicate more precisely the person or persons participating in the action described by the stage direction.
    \item[{Example}]
  \leavevmode\bgroup\exampleFont \begin{shaded}\noindent\mbox{}{<\textbf{stage}\hspace*{6pt}{type}="{setting}">}A curtain being drawn.{</\textbf{stage}>}\mbox{}\newline 
{<\textbf{stage}\hspace*{6pt}{type}="{setting}">}Music{</\textbf{stage}>}\mbox{}\newline 
{<\textbf{stage}\hspace*{6pt}{type}="{entrance}">}Enter Husband as being thrown off his horse and falls.{</\textbf{stage}>}\mbox{}\newline 
\textit{<!-- Middleton : Yorkshire Tragedy -->}\mbox{}\newline 
{<\textbf{stage}\hspace*{6pt}{type}="{exit}">}Exit pursued by a bear.{</\textbf{stage}>}\mbox{}\newline 
{<\textbf{stage}\hspace*{6pt}{type}="{business}">}He quickly takes the stone out.{</\textbf{stage}>}\mbox{}\newline 
{<\textbf{stage}\hspace*{6pt}{type}="{delivery}">}To Lussurioso.{</\textbf{stage}>}\mbox{}\newline 
{<\textbf{stage}\hspace*{6pt}{type}="{novelistic}">}Having had enough, and embarrassed for the family.{</\textbf{stage}>}\mbox{}\newline 
\textit{<!-- Lorraine Hansbury : a raisin in in the sun -->}\mbox{}\newline 
{<\textbf{stage}\hspace*{6pt}{type}="{modifier}">}Disguised as Ansaldo.{</\textbf{stage}>}\mbox{}\newline 
{<\textbf{stage}\hspace*{6pt}{type}="{entrance modifier}">}Enter Latrocinio disguised as an empiric{</\textbf{stage}>}\mbox{}\newline 
\textit{<!-- Middleton: The Widow -->}\mbox{}\newline 
{<\textbf{stage}\hspace*{6pt}{type}="{location}">}At a window.{</\textbf{stage}>}\mbox{}\newline 
{<\textbf{stage}\hspace*{6pt}{rend}="{inline}"\hspace*{6pt}{type}="{delivery}">}Aside.{</\textbf{stage}>}\end{shaded}\egroup 


    \item[{Example}]
  \leavevmode\bgroup\exampleFont \begin{shaded}\noindent\mbox{}{<\textbf{l}>}Behold. {<\textbf{stage}\hspace*{6pt}{n}="{*}"\hspace*{6pt}{place}="{margin}">}Here the vp{<\textbf{lb}/>}per part of the {<\textbf{hi}>}Scene{</\textbf{hi}>} open'd; when\mbox{}\newline 
\hspace*{6pt}\hspace*{6pt} straight appear'd a Heauen, and all the {<\textbf{hi}>}Pure Artes{</\textbf{hi}>} sitting on\mbox{}\newline 
\hspace*{6pt}\hspace*{6pt} two semi{<\textbf{lb}/>}circular ben{<\textbf{lb}/>}ches, one a{<\textbf{lb}/>}boue another: who sate thus till the rest of the\mbox{}\newline 
\hspace*{6pt}{<\textbf{hi}>}Prologue{</\textbf{hi}>} was spoken, which being ended, they descended in\mbox{}\newline 
\hspace*{6pt}\hspace*{6pt} order within the {<\textbf{hi}>}Scene,{</\textbf{hi}>} whiles the Musicke plaid{</\textbf{stage}>} Our\mbox{}\newline 
 Poet knowing our free hearts{</\textbf{l}>}\end{shaded}\egroup 


    \item[{Content model}]
  \mbox{}\hfill\\[-10pt]\begin{Verbatim}[fontsize=\small]
<content>
 <macroRef key="macro.specialPara"/>
</content>
    
\end{Verbatim}

    \item[{Schema Declaration}]
  \mbox{}\hfill\\[-10pt]\begin{Verbatim}[fontsize=\small]
element stage
{
   att.ascribed.attributes,
   att.global.attributes,
   att.placement.attributes,
   attribute type
   {
      list
      {
         (
            "setting"
          | "entrance"
          | "exit"
          | "business"
          | "novelistic"
          | "delivery"
          | "modifier"
          | "location"
          | "mixed"
         )*
      }
   }?,
   macro.specialPara}
\end{Verbatim}

\end{reflist}  \index{stamp=<stamp>|oddindex}
\begin{reflist}
\item[]\begin{specHead}{TEI.stamp}{<stamp> }contains a word or phrase describing a stamp or similar device. [\xref{http://www.tei-c.org/release/doc/tei-p5-doc/en/html/MS.html\#mswat}{10.3.3. Watermarks and Stamps}]\end{specHead} 
    \item[{Module}]
  msdescription
    \item[{Attributes}]
  Attributes att.global (\textit{@xml:id}, \textit{@n}, \textit{@xml:lang}, \textit{@xml:base}, \textit{@xml:space})  (att.global.rendition (\textit{@rend}, \textit{@style}, \textit{@rendition})) (att.global.linking (\textit{@corresp}, \textit{@synch}, \textit{@sameAs}, \textit{@copyOf}, \textit{@next}, \textit{@prev}, \textit{@exclude}, \textit{@select})) (att.global.analytic (\textit{@ana})) (att.global.facs (\textit{@facs})) (att.global.change (\textit{@change})) (att.global.responsibility (\textit{@cert}, \textit{@resp})) (att.global.source (\textit{@source})) att.typed (\textit{@type}, \textit{@subtype}) att.datable (\textit{@calendar}, \textit{@period})  (att.datable.w3c (\textit{@when}, \textit{@notBefore}, \textit{@notAfter}, \textit{@from}, \textit{@to})) (att.datable.iso (\textit{@when-iso}, \textit{@notBefore-iso}, \textit{@notAfter-iso}, \textit{@from-iso}, \textit{@to-iso})) (att.datable.custom (\textit{@when-custom}, \textit{@notBefore-custom}, \textit{@notAfter-custom}, \textit{@from-custom}, \textit{@to-custom}, \textit{@datingPoint}, \textit{@datingMethod}))
    \item[{Member of}]
  model.pPart.msdesc
    \item[{Contained by}]
  
    \item[analysis: ]
   cl phr s span\par 
    \item[core: ]
   abbr add addrLine author biblScope citedRange corr date del desc distinct editor email emph expan foreign gloss head headItem headLabel hi item l label measure meeting mentioned name note num orig p pubPlace publisher q quote ref reg resp rs said sic soCalled speaker stage street term textLang time title unclear\par 
    \item[figures: ]
   cell figDesc\par 
    \item[header: ]
   authority catDesc change classCode creation distributor edition extent funder geoDecl handNote language licence principal rendition scriptNote sponsor tagUsage typeNote\par 
    \item[linking: ]
   ab seg\par 
    \item[msdescription: ]
   accMat acquisition additions catchwords collation colophon condition custEvent decoNote explicit filiation finalRubric foliation heraldry incipit layout material musicNotation objectType origDate origPlace origin provenance rubric secFol signatures source stamp summary support surrogates watermark\par 
    \item[namesdates: ]
   addName affiliation age birth bloc country death district education faith floruit forename genName geogFeat geogName langKnown nameLink nationality occupation offset orgName persName placeName region residence roleName settlement sex socecStatus surname\par 
    \item[textcrit: ]
   lem rdg wit witDetail witness\par 
    \item[textstructure: ]
   byline closer dateline docAuthor docDate docEdition docImprint imprimatur opener salute signed titlePart trailer\par 
    \item[transcr: ]
   damage fw metamark mod restore retrace secl supplied surplus
    \item[{May contain}]
  
    \item[analysis: ]
   c cl interp interpGrp m pc phr s span spanGrp w\par 
    \item[core: ]
   abbr add address cb choice corr date del distinct email emph expan foreign gap gb gloss graphic hi index lb measure measureGrp media mentioned milestone name note num orig pb ptr ref reg rs sic soCalled term time title unclear\par 
    \item[figures: ]
   figure formula notatedMusic\par 
    \item[gaiji: ]
   g\par 
    \item[header: ]
   idno\par 
    \item[linking: ]
   alt altGrp anchor join joinGrp link linkGrp seg timeline\par 
    \item[msdescription: ]
   catchwords depth dim dimensions height heraldry locus locusGrp material objectType origDate origPlace secFol signatures stamp watermark width\par 
    \item[namesdates: ]
   addName affiliation bloc climate country district forename genName geo geogFeat geogName location nameLink offset orgName persName placeName population region roleName settlement state surname terrain trait\par 
    \item[textcrit: ]
   app witDetail\par 
    \item[transcr: ]
   addSpan am damage damageSpan delSpan ex fw handShift listTranspose metamark mod redo restore retrace secl space subst substJoin supplied surplus undo\par character data
    \item[{Example}]
  \leavevmode\bgroup\exampleFont \begin{shaded}\noindent\mbox{}{<\textbf{rubric}>}Apologyticu TTVLLIANI AC IGNORATIA IN XPO IHV{<\textbf{lb}/>}\mbox{}\newline 
 SI NON LICET{<\textbf{lb}/>}\mbox{}\newline 
 NOBIS RO{<\textbf{lb}/>}\mbox{}\newline 
 manii imperii {<\textbf{stamp}>}Bodleian stamp{</\textbf{stamp}>}\mbox{}\newline 
\hspace*{6pt}{<\textbf{lb}/>}\mbox{}\newline 
{</\textbf{rubric}>}\end{shaded}\egroup 


    \item[{Content model}]
  \mbox{}\hfill\\[-10pt]\begin{Verbatim}[fontsize=\small]
<content>
 <macroRef key="macro.phraseSeq"/>
</content>
    
\end{Verbatim}

    \item[{Schema Declaration}]
  \mbox{}\hfill\\[-10pt]\begin{Verbatim}[fontsize=\small]
element stamp
{
   att.global.attributes,
   att.typed.attributes,
   att.datable.attributes,
   macro.phraseSeq}
\end{Verbatim}

\end{reflist}  \index{state=<state>|oddindex}
\begin{reflist}
\item[]\begin{specHead}{TEI.state}{<state> }contains a description of some status or quality attributed to a person, place, or organization often at some specific time or for a specific date range. [\xref{http://www.tei-c.org/release/doc/tei-p5-doc/en/html/ND.html\#NDPERSbp}{13.3.1. Basic Principles} \xref{http://www.tei-c.org/release/doc/tei-p5-doc/en/html/ND.html\#NDPERSEpc}{13.3.2.1. Personal Characteristics}]\end{specHead} 
    \item[{Module}]
  namesdates
    \item[{Attributes}]
  Attributes att.global (\textit{@xml:id}, \textit{@n}, \textit{@xml:lang}, \textit{@xml:base}, \textit{@xml:space})  (att.global.rendition (\textit{@rend}, \textit{@style}, \textit{@rendition})) (att.global.linking (\textit{@corresp}, \textit{@synch}, \textit{@sameAs}, \textit{@copyOf}, \textit{@next}, \textit{@prev}, \textit{@exclude}, \textit{@select})) (att.global.analytic (\textit{@ana})) (att.global.facs (\textit{@facs})) (att.global.change (\textit{@change})) (att.global.responsibility (\textit{@cert}, \textit{@resp})) (att.global.source (\textit{@source})) att.datable (\textit{@calendar}, \textit{@period})  (att.datable.w3c (\textit{@when}, \textit{@notBefore}, \textit{@notAfter}, \textit{@from}, \textit{@to})) (att.datable.iso (\textit{@when-iso}, \textit{@notBefore-iso}, \textit{@notAfter-iso}, \textit{@from-iso}, \textit{@to-iso})) (att.datable.custom (\textit{@when-custom}, \textit{@notBefore-custom}, \textit{@notAfter-custom}, \textit{@from-custom}, \textit{@to-custom}, \textit{@datingPoint}, \textit{@datingMethod})) att.editLike (\textit{@evidence}, \textit{@instant})  (att.dimensions (\textit{@unit}, \textit{@quantity}, \textit{@extent}, \textit{@precision}, \textit{@scope}) (att.ranging (\textit{@atLeast}, \textit{@atMost}, \textit{@min}, \textit{@max}, \textit{@confidence})) ) att.typed (\textit{@type}, \textit{@subtype}) att.naming (\textit{@role}, \textit{@nymRef})  (att.canonical (\textit{@key}, \textit{@ref}))
    \item[{Member of}]
  model.persStateLike model.placeStateLike 
    \item[{Contained by}]
  
    \item[analysis: ]
   cl phr s span\par 
    \item[core: ]
   abbr add addrLine address author bibl biblScope citedRange corr date del desc distinct editor email emph expan foreign gloss head headItem headLabel hi item l label measure meeting mentioned name note num orig p pubPlace publisher q quote ref reg resp rs said sic soCalled speaker stage street term textLang time title unclear\par 
    \item[figures: ]
   cell figDesc\par 
    \item[header: ]
   authority catDesc change classCode correspAction creation distributor edition extent funder geoDecl handNote language licence principal rendition scriptNote sponsor tagUsage typeNote\par 
    \item[linking: ]
   ab seg\par 
    \item[msdescription: ]
   accMat acquisition additions catchwords collation colophon condition custEvent decoNote explicit filiation finalRubric foliation heraldry incipit layout material musicNotation objectType origDate origPlace origin provenance rubric secFol signatures source stamp summary support surrogates watermark\par 
    \item[namesdates: ]
   addName affiliation age birth bloc country death district education faith floruit forename genName geogFeat geogName langKnown nameLink nationality occupation offset org orgName persName person personGrp place placeName region residence roleName settlement sex socecStatus state surname\par 
    \item[textcrit: ]
   lem rdg wit witDetail witness\par 
    \item[textstructure: ]
   byline closer dateline docAuthor docDate docEdition docImprint imprimatur opener salute signed titlePart trailer\par 
    \item[transcr: ]
   damage fw metamark mod restore retrace secl supplied surplus
    \item[{May contain}]
  
    \item[core: ]
   bibl biblStruct desc head label listBibl note p\par 
    \item[header: ]
   biblFull\par 
    \item[linking: ]
   ab\par 
    \item[msdescription: ]
   msDesc\par 
    \item[namesdates: ]
   state\par 
    \item[textcrit: ]
   witDetail
    \item[{Note}]
  \par
Where there is confusion between <trait> and <state> the more general purpose element <state> should be used even for unchanging characteristics. If you wish to distinguish between characteristics that are generally perceived to be time-bound states and those assumed to be fixed traits, then <trait> is available for the more static of these. The <state> element encodes characteristics which are sometimes assumed to change, often at specific times or over a date range, whereas the <trait> elements are used to record characteristics, such as eye-colour, which are less subject to change. Traits are typically, but not necessarily, independent of the volition or action of the holder.
    \item[{Example}]
  \leavevmode\bgroup\exampleFont \begin{shaded}\noindent\mbox{}{<\textbf{state}\hspace*{6pt}{ref}="{\#SCHOL}"\hspace*{6pt}{type}="{status}">}\mbox{}\newline 
\hspace*{6pt}{<\textbf{label}>}scholar{</\textbf{label}>}\mbox{}\newline 
{</\textbf{state}>}\end{shaded}\egroup 


    \item[{Example}]
  \leavevmode\bgroup\exampleFont \begin{shaded}\noindent\mbox{}{<\textbf{org}>}\mbox{}\newline 
\hspace*{6pt}{<\textbf{orgName}\hspace*{6pt}{notAfter}="{1960}">}The Silver Beetles{</\textbf{orgName}>}\mbox{}\newline 
\hspace*{6pt}{<\textbf{orgName}\hspace*{6pt}{notBefore}="{1960}">}The Beatles{</\textbf{orgName}>}\mbox{}\newline 
\hspace*{6pt}{<\textbf{state}\hspace*{6pt}{from}="{1960-08}"\hspace*{6pt}{to}="{1962-05}"\mbox{}\newline 
\hspace*{6pt}\hspace*{6pt}{type}="{membership}">}\mbox{}\newline 
\hspace*{6pt}\hspace*{6pt}{<\textbf{desc}>}\mbox{}\newline 
\hspace*{6pt}\hspace*{6pt}\hspace*{6pt}{<\textbf{persName}>}John Lennon{</\textbf{persName}>}\mbox{}\newline 
\hspace*{6pt}\hspace*{6pt}\hspace*{6pt}{<\textbf{persName}>}Paul McCartney{</\textbf{persName}>}\mbox{}\newline 
\hspace*{6pt}\hspace*{6pt}\hspace*{6pt}{<\textbf{persName}>}George Harrison{</\textbf{persName}>}\mbox{}\newline 
\hspace*{6pt}\hspace*{6pt}\hspace*{6pt}{<\textbf{persName}>}Stuart Sutcliffe{</\textbf{persName}>}\mbox{}\newline 
\hspace*{6pt}\hspace*{6pt}\hspace*{6pt}{<\textbf{persName}>}Pete Best{</\textbf{persName}>}\mbox{}\newline 
\hspace*{6pt}\hspace*{6pt}{</\textbf{desc}>}\mbox{}\newline 
\hspace*{6pt}{</\textbf{state}>}\mbox{}\newline 
\hspace*{6pt}{<\textbf{state}\hspace*{6pt}{notBefore}="{1963}"\hspace*{6pt}{type}="{membership}">}\mbox{}\newline 
\hspace*{6pt}\hspace*{6pt}{<\textbf{desc}>}\mbox{}\newline 
\hspace*{6pt}\hspace*{6pt}\hspace*{6pt}{<\textbf{persName}>}John Lennon{</\textbf{persName}>}\mbox{}\newline 
\hspace*{6pt}\hspace*{6pt}\hspace*{6pt}{<\textbf{persName}>}Paul McCartney{</\textbf{persName}>}\mbox{}\newline 
\hspace*{6pt}\hspace*{6pt}\hspace*{6pt}{<\textbf{persName}>}George Harrison{</\textbf{persName}>}\mbox{}\newline 
\hspace*{6pt}\hspace*{6pt}\hspace*{6pt}{<\textbf{persName}>}Ringo Starr{</\textbf{persName}>}\mbox{}\newline 
\hspace*{6pt}\hspace*{6pt}{</\textbf{desc}>}\mbox{}\newline 
\hspace*{6pt}{</\textbf{state}>}\mbox{}\newline 
{</\textbf{org}>}\end{shaded}\egroup 


    \item[{Content model}]
  \mbox{}\hfill\\[-10pt]\begin{Verbatim}[fontsize=\small]
<content>
 <sequence>
  <elementRef key="precision"
   maxOccurs="unbounded" minOccurs="0"/>
  <alternate>
   <elementRef key="state"
    maxOccurs="unbounded" minOccurs="1"/>
   <sequence>
    <classRef key="model.headLike"
     maxOccurs="unbounded" minOccurs="0"/>
    <classRef key="model.pLike"
     maxOccurs="unbounded" minOccurs="1"/>
    <alternate maxOccurs="unbounded"
     minOccurs="0">
     <classRef key="model.noteLike"/>
     <classRef key="model.biblLike"/>
    </alternate>
   </sequence>
   <alternate maxOccurs="unbounded"
    minOccurs="0">
    <classRef key="model.labelLike"/>
    <classRef key="model.noteLike"/>
    <classRef key="model.biblLike"/>
   </alternate>
  </alternate>
 </sequence>
</content>
    
\end{Verbatim}

    \item[{Schema Declaration}]
  \mbox{}\hfill\\[-10pt]\begin{Verbatim}[fontsize=\small]
element state
{
   att.global.attributes,
   att.datable.attributes,
   att.editLike.attributes,
   att.typed.attributes,
   att.naming.attributes,
   (
      precision*,
      (
         state+
       | (
            model.headLike*,
            model.pLike+,
            ( model.noteLike | model.biblLike )*
         )
       | ( model.labelLike | model.noteLike | model.biblLike )*
      )
   )
}
\end{Verbatim}

\end{reflist}  \index{stdVals=<stdVals>|oddindex}
\begin{reflist}
\item[]\begin{specHead}{TEI.stdVals}{<stdVals> }(standard values) specifies the format used when standardized date or number values are supplied. [\xref{http://www.tei-c.org/release/doc/tei-p5-doc/en/html/HD.html\#HD53}{2.3.3. The Editorial Practices Declaration} \xref{http://www.tei-c.org/release/doc/tei-p5-doc/en/html/CC.html\#CCAS2}{15.3.2. Declarable Elements}]\end{specHead} 
    \item[{Module}]
  header
    \item[{Attributes}]
  Attributes att.global (\textit{@xml:id}, \textit{@n}, \textit{@xml:lang}, \textit{@xml:base}, \textit{@xml:space})  (att.global.rendition (\textit{@rend}, \textit{@style}, \textit{@rendition})) (att.global.linking (\textit{@corresp}, \textit{@synch}, \textit{@sameAs}, \textit{@copyOf}, \textit{@next}, \textit{@prev}, \textit{@exclude}, \textit{@select})) (att.global.analytic (\textit{@ana})) (att.global.facs (\textit{@facs})) (att.global.change (\textit{@change})) (att.global.responsibility (\textit{@cert}, \textit{@resp})) (att.global.source (\textit{@source})) att.declarable (\textit{@default}) 
    \item[{Member of}]
  model.editorialDeclPart
    \item[{Contained by}]
  
    \item[header: ]
   editorialDecl
    \item[{May contain}]
  
    \item[core: ]
   p\par 
    \item[linking: ]
   ab
    \item[{Example}]
  \leavevmode\bgroup\exampleFont \begin{shaded}\noindent\mbox{}{<\textbf{stdVals}>}\mbox{}\newline 
\hspace*{6pt}{<\textbf{p}>}All integer numbers are left-filled with zeroes to 8 digits.{</\textbf{p}>}\mbox{}\newline 
{</\textbf{stdVals}>}\end{shaded}\egroup 


    \item[{Content model}]
  \mbox{}\hfill\\[-10pt]\begin{Verbatim}[fontsize=\small]
<content>
 <classRef key="model.pLike"
  maxOccurs="unbounded" minOccurs="1"/>
</content>
    
\end{Verbatim}

    \item[{Schema Declaration}]
  \mbox{}\hfill\\[-10pt]\begin{Verbatim}[fontsize=\small]
element stdVals
{
   att.global.attributes,
   att.declarable.attributes,
   model.pLike+
}
\end{Verbatim}

\end{reflist}  \index{street=<street>|oddindex}
\begin{reflist}
\item[]\begin{specHead}{TEI.street}{<street> }contains a full street address including any name or number identifying a building as well as the name of the street or route on which it is located. [\xref{http://www.tei-c.org/release/doc/tei-p5-doc/en/html/CO.html\#CONAAD}{3.5.2. Addresses}]\end{specHead} 
    \item[{Module}]
  core
    \item[{Attributes}]
  Attributes att.global (\textit{@xml:id}, \textit{@n}, \textit{@xml:lang}, \textit{@xml:base}, \textit{@xml:space})  (att.global.rendition (\textit{@rend}, \textit{@style}, \textit{@rendition})) (att.global.linking (\textit{@corresp}, \textit{@synch}, \textit{@sameAs}, \textit{@copyOf}, \textit{@next}, \textit{@prev}, \textit{@exclude}, \textit{@select})) (att.global.analytic (\textit{@ana})) (att.global.facs (\textit{@facs})) (att.global.change (\textit{@change})) (att.global.responsibility (\textit{@cert}, \textit{@resp})) (att.global.source (\textit{@source}))
    \item[{Member of}]
  model.addrPart
    \item[{Contained by}]
  
    \item[core: ]
   address
    \item[{May contain}]
  
    \item[analysis: ]
   c cl interp interpGrp m pc phr s span spanGrp w\par 
    \item[core: ]
   abbr add address cb choice corr date del distinct email emph expan foreign gap gb gloss graphic hi index lb measure measureGrp media mentioned milestone name note num orig pb ptr ref reg rs sic soCalled term time title unclear\par 
    \item[figures: ]
   figure formula notatedMusic\par 
    \item[gaiji: ]
   g\par 
    \item[header: ]
   idno\par 
    \item[linking: ]
   alt altGrp anchor join joinGrp link linkGrp seg timeline\par 
    \item[msdescription: ]
   catchwords depth dim dimensions height heraldry locus locusGrp material objectType origDate origPlace secFol signatures stamp watermark width\par 
    \item[namesdates: ]
   addName affiliation bloc climate country district forename genName geo geogFeat geogName location nameLink offset orgName persName placeName population region roleName settlement state surname terrain trait\par 
    \item[textcrit: ]
   app witDetail\par 
    \item[transcr: ]
   addSpan am damage damageSpan delSpan ex fw handShift listTranspose metamark mod redo restore retrace secl space subst substJoin supplied surplus undo\par character data
    \item[{Note}]
  \par
The order and presentation of house names and numbers and street names, etc., may vary considerably in different countries. The encoding should reflect the order which is appropriate in the country concerned. 
    \item[{Example}]
  \leavevmode\bgroup\exampleFont \begin{shaded}\noindent\mbox{}{<\textbf{street}>}via della Faggiola, 36{</\textbf{street}>}\end{shaded}\egroup 


    \item[{Example}]
  \leavevmode\bgroup\exampleFont \begin{shaded}\noindent\mbox{}{<\textbf{street}>}\mbox{}\newline 
\hspace*{6pt}{<\textbf{name}>}Duntaggin{</\textbf{name}>}, 110 Southmoor Road\mbox{}\newline 
{</\textbf{street}>}\end{shaded}\egroup 


    \item[{Content model}]
  \mbox{}\hfill\\[-10pt]\begin{Verbatim}[fontsize=\small]
<content>
 <macroRef key="macro.phraseSeq"/>
</content>
    
\end{Verbatim}

    \item[{Schema Declaration}]
  \mbox{}\hfill\\[-10pt]\begin{Verbatim}[fontsize=\small]
element street { att.global.attributes, macro.phraseSeq }
\end{Verbatim}

\end{reflist}  \index{styleDefDecl=<styleDefDecl>|oddindex}
\begin{reflist}
\item[]\begin{specHead}{TEI.styleDefDecl}{<styleDefDecl> }(style definition language declaration) specifies the name of the formal language in which style or renditional information is supplied elsewhere in the document. The specific version of the scheme may also be supplied. [\xref{http://www.tei-c.org/release/doc/tei-p5-doc/en/html/HD.html\#HD57-1a}{2.3.5. The Default Style Definition Language Declaration}]\end{specHead} 
    \item[{Module}]
  header
    \item[{Attributes}]
  Attributes att.global (\textit{@xml:id}, \textit{@n}, \textit{@xml:lang}, \textit{@xml:base}, \textit{@xml:space})  (att.global.rendition (\textit{@rend}, \textit{@style}, \textit{@rendition})) (att.global.linking (\textit{@corresp}, \textit{@synch}, \textit{@sameAs}, \textit{@copyOf}, \textit{@next}, \textit{@prev}, \textit{@exclude}, \textit{@select})) (att.global.analytic (\textit{@ana})) (att.global.facs (\textit{@facs})) (att.global.change (\textit{@change})) (att.global.responsibility (\textit{@cert}, \textit{@resp})) (att.global.source (\textit{@source})) att.declarable (\textit{@default}) att.styleDef (\textit{@scheme}, \textit{@schemeVersion}) 
    \item[{Member of}]
  model.encodingDescPart
    \item[{Contained by}]
  
    \item[header: ]
   encodingDesc
    \item[{May contain}]
  
    \item[core: ]
   p\par 
    \item[linking: ]
   ab
    \item[{Example}]
  \leavevmode\bgroup\exampleFont \begin{shaded}\noindent\mbox{}{<\textbf{styleDefDecl}\hspace*{6pt}{scheme}="{css}"\mbox{}\newline 
\hspace*{6pt}{schemeVersion}="{2.1}"/>}\mbox{}\newline 
\textit{<!-- ... -->}\mbox{}\newline 
{<\textbf{tagsDecl}>}\mbox{}\newline 
\hspace*{6pt}{<\textbf{rendition}\hspace*{6pt}{xml:id}="{boldface}">}font-weight: bold;{</\textbf{rendition}>}\mbox{}\newline 
\hspace*{6pt}{<\textbf{rendition}\hspace*{6pt}{xml:id}="{italicstyle}">}font-style: italic;{</\textbf{rendition}>}\mbox{}\newline 
{</\textbf{tagsDecl}>}\end{shaded}\egroup 


    \item[{Content model}]
  \mbox{}\hfill\\[-10pt]\begin{Verbatim}[fontsize=\small]
<content>
 <classRef key="model.pLike"
  maxOccurs="unbounded" minOccurs="0"/>
</content>
    
\end{Verbatim}

    \item[{Schema Declaration}]
  \mbox{}\hfill\\[-10pt]\begin{Verbatim}[fontsize=\small]
element styleDefDecl
{
   att.global.attributes,
   att.declarable.attributes,
   att.styleDef.attributes,
   model.pLike*
}
\end{Verbatim}

\end{reflist}  \index{subst=<subst>|oddindex}
\begin{reflist}
\item[]\begin{specHead}{TEI.subst}{<subst> }(substitution) groups one or more deletions with one or more additions when the combination is to be regarded as a single intervention in the text. [\xref{http://www.tei-c.org/release/doc/tei-p5-doc/en/html/PH.html\#PHSU}{11.3.1.5. Substitutions}]\end{specHead} 
    \item[{Module}]
  transcr
    \item[{Attributes}]
  Attributes att.global (\textit{@xml:id}, \textit{@n}, \textit{@xml:lang}, \textit{@xml:base}, \textit{@xml:space})  (att.global.rendition (\textit{@rend}, \textit{@style}, \textit{@rendition})) (att.global.linking (\textit{@corresp}, \textit{@synch}, \textit{@sameAs}, \textit{@copyOf}, \textit{@next}, \textit{@prev}, \textit{@exclude}, \textit{@select})) (att.global.analytic (\textit{@ana})) (att.global.facs (\textit{@facs})) (att.global.change (\textit{@change})) (att.global.responsibility (\textit{@cert}, \textit{@resp})) (att.global.source (\textit{@source})) att.transcriptional (\textit{@status}, \textit{@cause}, \textit{@seq})  (att.editLike (\textit{@evidence}, \textit{@instant}) (att.dimensions (\textit{@unit}, \textit{@quantity}, \textit{@extent}, \textit{@precision}, \textit{@scope}) (att.ranging (\textit{@atLeast}, \textit{@atMost}, \textit{@min}, \textit{@max}, \textit{@confidence})) ) ) (att.written (\textit{@hand}))
    \item[{Member of}]
  model.pPart.editorial
    \item[{Contained by}]
  
    \item[analysis: ]
   cl pc phr s span w\par 
    \item[core: ]
   abbr add addrLine author bibl biblScope citedRange corr date del desc distinct editor email emph expan foreign gloss head headItem headLabel hi item l label measure meeting mentioned name note num orig p pubPlace publisher q quote ref reg resp rs said sic soCalled speaker stage street term textLang time title unclear\par 
    \item[figures: ]
   cell figDesc\par 
    \item[header: ]
   authority catDesc change classCode creation distributor edition extent funder geoDecl handNote language licence principal rendition scriptNote sponsor tagUsage typeNote\par 
    \item[linking: ]
   ab seg\par 
    \item[msdescription: ]
   accMat acquisition additions catchwords collation colophon condition custEvent decoNote explicit filiation finalRubric foliation heraldry incipit layout material musicNotation objectType origDate origPlace origin provenance rubric secFol signatures source stamp summary support surrogates watermark\par 
    \item[namesdates: ]
   addName affiliation age birth bloc country death district education faith floruit forename genName geogFeat geogName langKnown nameLink nationality occupation offset orgName persName placeName region residence roleName settlement sex socecStatus surname\par 
    \item[textcrit: ]
   lem rdg wit witDetail witness\par 
    \item[textstructure: ]
   byline closer dateline docAuthor docDate docEdition docImprint imprimatur opener salute signed titlePart trailer\par 
    \item[transcr: ]
   damage fw metamark mod restore retrace secl supplied surplus
    \item[{May contain}]
  
    \item[core: ]
   add cb del gb lb milestone pb\par 
    \item[linking: ]
   anchor\par 
    \item[transcr: ]
   fw
    \item[{Example}]
  \leavevmode\bgroup\exampleFont \begin{shaded}\noindent\mbox{}... are all included. {<\textbf{del}\hspace*{6pt}{hand}="{\#RG}">}It is{</\textbf{del}>}\mbox{}\newline 
{<\textbf{subst}>}\mbox{}\newline 
\hspace*{6pt}{<\textbf{add}>}T{</\textbf{add}>}\mbox{}\newline 
\hspace*{6pt}{<\textbf{del}>}t{</\textbf{del}>}\mbox{}\newline 
{</\textbf{subst}>}he expressed\mbox{}\newline 
\end{shaded}\egroup 


    \item[{Example}]
  \leavevmode\bgroup\exampleFont \begin{shaded}\noindent\mbox{} that he and his Sister Miſs D — {<\textbf{lb}/>}who always lived with him, wd. be {<\textbf{subst}>}\mbox{}\newline 
\hspace*{6pt}{<\textbf{del}>}very{</\textbf{del}>}\mbox{}\newline 
\hspace*{6pt}{<\textbf{lb}/>}\mbox{}\newline 
\hspace*{6pt}{<\textbf{add}>}principally{</\textbf{add}>}\mbox{}\newline 
{</\textbf{subst}>} remembered in her Will.\mbox{}\newline 
\end{shaded}\egroup 


    \item[{Example}]
  \leavevmode\bgroup\exampleFont \begin{shaded}\noindent\mbox{}{<\textbf{ab}>}τ{<\textbf{subst}>}\mbox{}\newline 
\hspace*{6pt}\hspace*{6pt}{<\textbf{add}\hspace*{6pt}{place}="{above}">}ῶν{</\textbf{add}>}\mbox{}\newline 
\hspace*{6pt}\hspace*{6pt}{<\textbf{del}>}α{</\textbf{del}>}\mbox{}\newline 
\hspace*{6pt}{</\textbf{subst}>}\mbox{}\newline 
 συνκυρόντ{<\textbf{subst}>}\mbox{}\newline 
\hspace*{6pt}\hspace*{6pt}{<\textbf{add}\hspace*{6pt}{place}="{above}">}ων{</\textbf{add}>}\mbox{}\newline 
\hspace*{6pt}\hspace*{6pt}{<\textbf{del}>}α{</\textbf{del}>}\mbox{}\newline 
\hspace*{6pt}{</\textbf{subst}>}\mbox{}\newline 
 ἐργαστηρί{<\textbf{subst}>}\mbox{}\newline 
\hspace*{6pt}\hspace*{6pt}{<\textbf{add}\hspace*{6pt}{place}="{above}">}ων{</\textbf{add}>}\mbox{}\newline 
\hspace*{6pt}\hspace*{6pt}{<\textbf{del}>}α{</\textbf{del}>}\mbox{}\newline 
\hspace*{6pt}{</\textbf{subst}>}\mbox{}\newline 
{</\textbf{ab}>}\end{shaded}\egroup 


    \item[{Example}]
  \leavevmode\bgroup\exampleFont \begin{shaded}\noindent\mbox{}{<\textbf{subst}>}\mbox{}\newline 
\hspace*{6pt}{<\textbf{del}>}\mbox{}\newline 
\hspace*{6pt}\hspace*{6pt}{<\textbf{gap}\hspace*{6pt}{quantity}="{5}"\hspace*{6pt}{reason}="{illegible}"\mbox{}\newline 
\hspace*{6pt}\hspace*{6pt}\hspace*{6pt}{unit}="{character}"/>}\mbox{}\newline 
\hspace*{6pt}{</\textbf{del}>}\mbox{}\newline 
\hspace*{6pt}{<\textbf{add}>}apple{</\textbf{add}>}\mbox{}\newline 
{</\textbf{subst}>}\end{shaded}\egroup 


    \item[{Schematron}]
   <s:assert test="child::tei:add and child::tei:del"> <s:name/> must have at least one child add and at least one child del</s:assert>
    \item[{Content model}]
  \mbox{}\hfill\\[-10pt]\begin{Verbatim}[fontsize=\small]
<content>
 <alternate maxOccurs="unbounded"
  minOccurs="1">
  <elementRef key="add"/>
  <elementRef key="del"/>
  <classRef key="model.milestoneLike"/>
 </alternate>
</content>
    
\end{Verbatim}

    \item[{Schema Declaration}]
  \mbox{}\hfill\\[-10pt]\begin{Verbatim}[fontsize=\small]
element subst
{
   att.global.attributes,
   att.transcriptional.attributes,
   ( add | del | model.milestoneLike )+
}
\end{Verbatim}

\end{reflist}  \index{substJoin=<substJoin>|oddindex}
\begin{reflist}
\item[]\begin{specHead}{TEI.substJoin}{<substJoin> }(substitution join) identifies a series of possibly fragmented additions, deletions or other revisions on a manuscript that combine to make up a single intervention in the text [\xref{http://www.tei-c.org/release/doc/tei-p5-doc/en/html/PH.html\#PHSU}{11.3.1.5. Substitutions}]\end{specHead} 
    \item[{Module}]
  transcr
    \item[{Attributes}]
  Attributes att.global (\textit{@xml:id}, \textit{@n}, \textit{@xml:lang}, \textit{@xml:base}, \textit{@xml:space})  (att.global.rendition (\textit{@rend}, \textit{@style}, \textit{@rendition})) (att.global.linking (\textit{@corresp}, \textit{@synch}, \textit{@sameAs}, \textit{@copyOf}, \textit{@next}, \textit{@prev}, \textit{@exclude}, \textit{@select})) (att.global.analytic (\textit{@ana})) (att.global.facs (\textit{@facs})) (att.global.change (\textit{@change})) (att.global.responsibility (\textit{@cert}, \textit{@resp})) (att.global.source (\textit{@source})) att.pointing (\textit{@targetLang}, \textit{@target}, \textit{@evaluate}) att.transcriptional (\textit{@status}, \textit{@cause}, \textit{@seq})  (att.editLike (\textit{@evidence}, \textit{@instant}) (att.dimensions (\textit{@unit}, \textit{@quantity}, \textit{@extent}, \textit{@precision}, \textit{@scope}) (att.ranging (\textit{@atLeast}, \textit{@atMost}, \textit{@min}, \textit{@max}, \textit{@confidence})) ) ) (att.written (\textit{@hand}))
    \item[{Member of}]
  model.global.meta
    \item[{Contained by}]
  
    \item[analysis: ]
   cl m phr s span w\par 
    \item[core: ]
   abbr add addrLine address author bibl biblScope cit citedRange corr date del distinct editor email emph expan foreign gloss head headItem headLabel hi imprint item l label lg list measure mentioned name note num orig p pubPlace publisher q quote ref reg resp rs said series sic soCalled sp speaker stage street term textLang time title unclear\par 
    \item[figures: ]
   cell figure table\par 
    \item[header: ]
   authority change classCode distributor edition extent funder geoDecl handNote language licence principal scriptNote sponsor typeNote\par 
    \item[linking: ]
   ab seg\par 
    \item[msdescription: ]
   accMat acquisition additions catchwords collation colophon condition custEvent decoNote explicit filiation finalRubric foliation heraldry incipit layout material msItem musicNotation objectType origDate origPlace origin provenance rubric secFol signatures source stamp summary support surrogates watermark\par 
    \item[namesdates: ]
   addName affiliation age birth bloc country death district education faith floruit forename genName geogFeat geogName langKnown nameLink nationality occupation offset orgName persName person personGrp placeName region residence roleName settlement sex socecStatus surname\par 
    \item[textcrit: ]
   lem rdg wit witDetail\par 
    \item[textstructure: ]
   argument back body byline closer dateline div docAuthor docDate docEdition docImprint docTitle epigraph floatingText front group imprimatur opener postscript salute signed text titlePage titlePart trailer\par 
    \item[transcr: ]
   damage fw line metamark mod restore retrace secl sourceDoc supplied surface surfaceGrp surplus zone
    \item[{May contain}]
  
    \item[core: ]
   desc
    \item[{Example}]
  \leavevmode\bgroup\exampleFont \begin{shaded}\noindent\mbox{} While {<\textbf{del}\hspace*{6pt}{xml:id}="{r112}">}pondering{</\textbf{del}>} thus {<\textbf{add}\hspace*{6pt}{xml:id}="{r113}">}she mus'd{</\textbf{add}>}, her pinions fann'd\mbox{}\newline 
{<\textbf{substJoin}\hspace*{6pt}{target}="{\#r112 \#r113}"/>}\end{shaded}\egroup 


    \item[{Content model}]
  \mbox{}\hfill\\[-10pt]\begin{Verbatim}[fontsize=\small]
<content>
 <alternate maxOccurs="unbounded"
  minOccurs="0">
  <classRef key="model.descLike"/>
  <classRef key="model.certLike"/>
 </alternate>
</content>
    
\end{Verbatim}

    \item[{Schema Declaration}]
  \mbox{}\hfill\\[-10pt]\begin{Verbatim}[fontsize=\small]
element substJoin
{
   att.global.attributes,
   att.pointing.attributes,
   att.transcriptional.attributes,
   ( model.descLike | model.certLike )*
}
\end{Verbatim}

\end{reflist}  \index{summary=<summary>|oddindex}
\begin{reflist}
\item[]\begin{specHead}{TEI.summary}{<summary> }contains an overview of the available information concerning some aspect of an item (for example, its intellectual content, history, layout, typography etc.) as a complement or alternative to the more detailed information carried by more specific elements. [\xref{http://www.tei-c.org/release/doc/tei-p5-doc/en/html/MS.html\#msco}{10.6. Intellectual Content}]\end{specHead} 
    \item[{Module}]
  msdescription
    \item[{Attributes}]
  Attributes att.global (\textit{@xml:id}, \textit{@n}, \textit{@xml:lang}, \textit{@xml:base}, \textit{@xml:space})  (att.global.rendition (\textit{@rend}, \textit{@style}, \textit{@rendition})) (att.global.linking (\textit{@corresp}, \textit{@synch}, \textit{@sameAs}, \textit{@copyOf}, \textit{@next}, \textit{@prev}, \textit{@exclude}, \textit{@select})) (att.global.analytic (\textit{@ana})) (att.global.facs (\textit{@facs})) (att.global.change (\textit{@change})) (att.global.responsibility (\textit{@cert}, \textit{@resp})) (att.global.source (\textit{@source}))
    \item[{Contained by}]
  
    \item[msdescription: ]
   decoDesc handDesc history layoutDesc msContents scriptDesc sealDesc typeDesc
    \item[{May contain}]
  
    \item[analysis: ]
   c cl interp interpGrp m pc phr s span spanGrp w\par 
    \item[core: ]
   abbr add address bibl biblStruct cb choice cit corr date del desc distinct email emph expan foreign gap gb gloss graphic hi index l label lb lg list listBibl measure measureGrp media mentioned milestone name note num orig p pb ptr q quote ref reg rs said sic soCalled sp stage term time title unclear\par 
    \item[figures: ]
   figure formula notatedMusic table\par 
    \item[gaiji: ]
   g\par 
    \item[header: ]
   biblFull idno\par 
    \item[linking: ]
   ab alt altGrp anchor join joinGrp link linkGrp seg timeline\par 
    \item[msdescription: ]
   catchwords depth dim dimensions height heraldry locus locusGrp material msDesc objectType origDate origPlace secFol signatures stamp watermark width\par 
    \item[namesdates: ]
   addName affiliation bloc climate country district forename genName geo geogFeat geogName listEvent listNym listOrg listPerson listPlace location nameLink offset orgName persName placeName population region roleName settlement state surname terrain trait\par 
    \item[textcrit: ]
   app listApp listWit witDetail\par 
    \item[textstructure: ]
   floatingText\par 
    \item[transcr: ]
   addSpan am damage damageSpan delSpan ex fw handShift listTranspose metamark mod redo restore retrace secl space subst substJoin supplied surplus undo\par character data
    \item[{Example}]
  \leavevmode\bgroup\exampleFont \begin{shaded}\noindent\mbox{}{<\textbf{summary}>}This item consists of three books with a prologue and an epilogue.\mbox{}\newline 
{</\textbf{summary}>}\end{shaded}\egroup 


    \item[{Example}]
  \leavevmode\bgroup\exampleFont \begin{shaded}\noindent\mbox{}{<\textbf{typeDesc}>}\mbox{}\newline 
\hspace*{6pt}{<\textbf{summary}>}Uses a mixture of Roman and Black Letter types.{</\textbf{summary}>}\mbox{}\newline 
\hspace*{6pt}{<\textbf{typeNote}>}Antiqua typeface, showing influence of Jenson's Venetian\mbox{}\newline 
\hspace*{6pt}\hspace*{6pt} fonts.{</\textbf{typeNote}>}\mbox{}\newline 
\hspace*{6pt}{<\textbf{typeNote}>}The black letter face is a variant of Schwabacher.{</\textbf{typeNote}>}\mbox{}\newline 
{</\textbf{typeDesc}>}\end{shaded}\egroup 


    \item[{Content model}]
  \mbox{}\hfill\\[-10pt]\begin{Verbatim}[fontsize=\small]
<content>
 <macroRef key="macro.specialPara"/>
</content>
    
\end{Verbatim}

    \item[{Schema Declaration}]
  \mbox{}\hfill\\[-10pt]\begin{Verbatim}[fontsize=\small]
element summary { att.global.attributes, macro.specialPara }
\end{Verbatim}

\end{reflist}  \index{supplied=<supplied>|oddindex}\index{reason=@reason!<supplied>|oddindex}
\begin{reflist}
\item[]\begin{specHead}{TEI.supplied}{<supplied> }signifies text supplied by the transcriber or editor for any reason; for example because the original cannot be read due to physical damage, or because of an obvious omission by the author or scribe. [\xref{http://www.tei-c.org/release/doc/tei-p5-doc/en/html/PH.html\#PHDA}{11.3.3.1. Damage, Illegibility, and Supplied Text}]\end{specHead} 
    \item[{Module}]
  transcr
    \item[{Attributes}]
  Attributes att.global (\textit{@xml:id}, \textit{@n}, \textit{@xml:lang}, \textit{@xml:base}, \textit{@xml:space})  (att.global.rendition (\textit{@rend}, \textit{@style}, \textit{@rendition})) (att.global.linking (\textit{@corresp}, \textit{@synch}, \textit{@sameAs}, \textit{@copyOf}, \textit{@next}, \textit{@prev}, \textit{@exclude}, \textit{@select})) (att.global.analytic (\textit{@ana})) (att.global.facs (\textit{@facs})) (att.global.change (\textit{@change})) (att.global.responsibility (\textit{@cert}, \textit{@resp})) (att.global.source (\textit{@source})) att.editLike (\textit{@evidence}, \textit{@instant})  (att.dimensions (\textit{@unit}, \textit{@quantity}, \textit{@extent}, \textit{@precision}, \textit{@scope}) (att.ranging (\textit{@atLeast}, \textit{@atMost}, \textit{@min}, \textit{@max}, \textit{@confidence})) ) \hfil\\[-10pt]\begin{sansreflist}
    \item[@reason]
  one or more words indicating why the text has had to be supplied, e.g. \textit{overbinding}, \textit{faded-ink}, \textit{lost-folio}, \textit{omitted-in-original}.
\begin{reflist}
    \item[{Status}]
  Optional
    \item[{Datatype}]
  1–∞ occurrences of teidata.word separated by whitespace
\end{reflist}  
\end{sansreflist}  
    \item[{Member of}]
  model.choicePart model.pPart.transcriptional
    \item[{Contained by}]
  
    \item[analysis: ]
   cl pc phr s w\par 
    \item[core: ]
   abbr add addrLine author bibl biblScope choice citedRange corr date del distinct editor email emph expan foreign gloss head headItem headLabel hi item l label measure mentioned name note num orig p pubPlace publisher q quote ref reg rs said sic soCalled speaker stage street term textLang time title unclear\par 
    \item[figures: ]
   cell\par 
    \item[header: ]
   change distributor edition extent geoDecl handNote licence scriptNote typeNote\par 
    \item[linking: ]
   ab seg\par 
    \item[msdescription: ]
   accMat acquisition additions catchwords collation colophon condition custEvent decoNote explicit filiation finalRubric foliation heraldry incipit layout material musicNotation objectType origDate origPlace origin provenance rubric secFol signatures source stamp summary support surrogates watermark\par 
    \item[namesdates: ]
   addName affiliation birth bloc country death district education faith floruit forename genName geogFeat geogName nameLink nationality occupation offset orgName persName placeName region residence roleName settlement sex socecStatus surname\par 
    \item[textcrit: ]
   lem rdg wit witDetail\par 
    \item[textstructure: ]
   byline closer dateline docAuthor docDate docEdition docImprint imprimatur opener salute signed titlePart trailer\par 
    \item[transcr: ]
   am damage fw metamark mod restore retrace secl supplied surplus
    \item[{May contain}]
  
    \item[analysis: ]
   c cl interp interpGrp m pc phr s span spanGrp w\par 
    \item[core: ]
   abbr add address bibl biblStruct cb choice cit corr date del desc distinct email emph expan foreign gap gb gloss graphic hi index l label lb lg list listBibl measure measureGrp media mentioned milestone name note num orig pb ptr q quote ref reg rs said sic soCalled stage term time title unclear\par 
    \item[figures: ]
   figure formula notatedMusic table\par 
    \item[gaiji: ]
   g\par 
    \item[header: ]
   biblFull idno\par 
    \item[linking: ]
   alt altGrp anchor join joinGrp link linkGrp seg timeline\par 
    \item[msdescription: ]
   catchwords depth dim dimensions height heraldry locus locusGrp material msDesc objectType origDate origPlace secFol signatures stamp watermark width\par 
    \item[namesdates: ]
   addName affiliation bloc climate country district forename genName geo geogFeat geogName listEvent listNym listOrg listPerson listPlace location nameLink offset orgName persName placeName population region roleName settlement state surname terrain trait\par 
    \item[textcrit: ]
   app listApp listWit witDetail\par 
    \item[textstructure: ]
   floatingText\par 
    \item[transcr: ]
   addSpan am damage damageSpan delSpan ex fw handShift listTranspose metamark mod redo restore retrace secl space subst substJoin supplied surplus undo\par character data
    \item[{Note}]
  \par
The <damage>, <gap>, <del>, <unclear> and <supplied> elements may be closely allied in use. See section \xref{http://www.tei-c.org/release/doc/tei-p5-doc/en/html/PH.html\#PHCOMB}{11.3.3.2. Use of the gap, del, damage, unclear, and supplied Elements in Combination} for discussion of which element is appropriate for which circumstance.
    \item[{Example}]
  \leavevmode\bgroup\exampleFont \begin{shaded}\noindent\mbox{}I am dr Sr yr\mbox{}\newline 
{<\textbf{supplied}\hspace*{6pt}{reason}="{illegible}"\mbox{}\newline 
\hspace*{6pt}{source}="{\#amanuensis\textunderscore copy}">}very humble Servt{</\textbf{supplied}>}\mbox{}\newline 
 Sydney Smith\end{shaded}\egroup 


    \item[{Example}]
  \leavevmode\bgroup\exampleFont \begin{shaded}\noindent\mbox{}{<\textbf{supplied}\hspace*{6pt}{reason}="{omitted-in-original}">}Dedication{</\textbf{supplied}>} to the duke of Bejar\end{shaded}\egroup 


    \item[{Content model}]
  \mbox{}\hfill\\[-10pt]\begin{Verbatim}[fontsize=\small]
<content>
 <macroRef key="macro.paraContent"/>
</content>
    
\end{Verbatim}

    \item[{Schema Declaration}]
  \mbox{}\hfill\\[-10pt]\begin{Verbatim}[fontsize=\small]
element supplied
{
   att.global.attributes,
   att.editLike.attributes,
   attribute reason { list { + } }?,
   macro.paraContent}
\end{Verbatim}

\end{reflist}  \index{support=<support>|oddindex}
\begin{reflist}
\item[]\begin{specHead}{TEI.support}{<support> }contains a description of the materials etc. which make up the physical support for the written part of a manuscript. [\xref{http://www.tei-c.org/release/doc/tei-p5-doc/en/html/MS.html\#msph1}{10.7.1. Object Description}]\end{specHead} 
    \item[{Module}]
  msdescription
    \item[{Attributes}]
  Attributes att.global (\textit{@xml:id}, \textit{@n}, \textit{@xml:lang}, \textit{@xml:base}, \textit{@xml:space})  (att.global.rendition (\textit{@rend}, \textit{@style}, \textit{@rendition})) (att.global.linking (\textit{@corresp}, \textit{@synch}, \textit{@sameAs}, \textit{@copyOf}, \textit{@next}, \textit{@prev}, \textit{@exclude}, \textit{@select})) (att.global.analytic (\textit{@ana})) (att.global.facs (\textit{@facs})) (att.global.change (\textit{@change})) (att.global.responsibility (\textit{@cert}, \textit{@resp})) (att.global.source (\textit{@source}))
    \item[{Contained by}]
  
    \item[msdescription: ]
   supportDesc
    \item[{May contain}]
  
    \item[analysis: ]
   c cl interp interpGrp m pc phr s span spanGrp w\par 
    \item[core: ]
   abbr add address bibl biblStruct cb choice cit corr date del desc distinct email emph expan foreign gap gb gloss graphic hi index l label lb lg list listBibl measure measureGrp media mentioned milestone name note num orig p pb ptr q quote ref reg rs said sic soCalled sp stage term time title unclear\par 
    \item[figures: ]
   figure formula notatedMusic table\par 
    \item[gaiji: ]
   g\par 
    \item[header: ]
   biblFull idno\par 
    \item[linking: ]
   ab alt altGrp anchor join joinGrp link linkGrp seg timeline\par 
    \item[msdescription: ]
   catchwords depth dim dimensions height heraldry locus locusGrp material msDesc objectType origDate origPlace secFol signatures stamp watermark width\par 
    \item[namesdates: ]
   addName affiliation bloc climate country district forename genName geo geogFeat geogName listEvent listNym listOrg listPerson listPlace location nameLink offset orgName persName placeName population region roleName settlement state surname terrain trait\par 
    \item[textcrit: ]
   app listApp listWit witDetail\par 
    \item[textstructure: ]
   floatingText\par 
    \item[transcr: ]
   addSpan am damage damageSpan delSpan ex fw handShift listTranspose metamark mod redo restore retrace secl space subst substJoin supplied surplus undo\par character data
    \item[{Example}]
  \leavevmode\bgroup\exampleFont \begin{shaded}\noindent\mbox{}{<\textbf{objectDesc}\hspace*{6pt}{form}="{roll}">}\mbox{}\newline 
\hspace*{6pt}{<\textbf{supportDesc}>}\mbox{}\newline 
\hspace*{6pt}\hspace*{6pt}{<\textbf{support}>} Parchment roll with {<\textbf{material}>}silk{</\textbf{material}>} ribbons.\mbox{}\newline 
\hspace*{6pt}\hspace*{6pt}{</\textbf{support}>}\mbox{}\newline 
\hspace*{6pt}{</\textbf{supportDesc}>}\mbox{}\newline 
{</\textbf{objectDesc}>}\end{shaded}\egroup 


    \item[{Content model}]
  \mbox{}\hfill\\[-10pt]\begin{Verbatim}[fontsize=\small]
<content>
 <macroRef key="macro.specialPara"/>
</content>
    
\end{Verbatim}

    \item[{Schema Declaration}]
  \mbox{}\hfill\\[-10pt]\begin{Verbatim}[fontsize=\small]
element support { att.global.attributes, macro.specialPara }
\end{Verbatim}

\end{reflist}  \index{supportDesc=<supportDesc>|oddindex}\index{material=@material!<supportDesc>|oddindex}
\begin{reflist}
\item[]\begin{specHead}{TEI.supportDesc}{<supportDesc> }(support description) groups elements describing the physical support for the written part of a manuscript. [\xref{http://www.tei-c.org/release/doc/tei-p5-doc/en/html/MS.html\#msph1}{10.7.1. Object Description}]\end{specHead} 
    \item[{Module}]
  msdescription
    \item[{Attributes}]
  Attributes att.global (\textit{@xml:id}, \textit{@n}, \textit{@xml:lang}, \textit{@xml:base}, \textit{@xml:space})  (att.global.rendition (\textit{@rend}, \textit{@style}, \textit{@rendition})) (att.global.linking (\textit{@corresp}, \textit{@synch}, \textit{@sameAs}, \textit{@copyOf}, \textit{@next}, \textit{@prev}, \textit{@exclude}, \textit{@select})) (att.global.analytic (\textit{@ana})) (att.global.facs (\textit{@facs})) (att.global.change (\textit{@change})) (att.global.responsibility (\textit{@cert}, \textit{@resp})) (att.global.source (\textit{@source})) \hfil\\[-10pt]\begin{sansreflist}
    \item[@material]
  a short project-defined name for the material composing the majority of the support
\begin{reflist}
    \item[{Status}]
  Optional
    \item[{Datatype}]
  teidata.enumerated
    \item[{Suggested values include:}]
  \begin{description}

\item[{paper}]
\item[{parch}](parchment)
\item[{mixed}]
\end{description} 
\end{reflist}  
\end{sansreflist}  
    \item[{Contained by}]
  
    \item[msdescription: ]
   objectDesc
    \item[{May contain}]
  
    \item[core: ]
   p\par 
    \item[header: ]
   extent\par 
    \item[linking: ]
   ab\par 
    \item[msdescription: ]
   collation condition foliation support
    \item[{Example}]
  \leavevmode\bgroup\exampleFont \begin{shaded}\noindent\mbox{}{<\textbf{supportDesc}>}\mbox{}\newline 
\hspace*{6pt}{<\textbf{support}>} Parchment roll with {<\textbf{material}>}silk{</\textbf{material}>} ribbons.\mbox{}\newline 
\hspace*{6pt}{</\textbf{support}>}\mbox{}\newline 
{</\textbf{supportDesc}>}\end{shaded}\egroup 


    \item[{Content model}]
  \mbox{}\hfill\\[-10pt]\begin{Verbatim}[fontsize=\small]
<content>
 <alternate>
  <classRef key="model.pLike"
   maxOccurs="unbounded" minOccurs="1"/>
  <sequence>
   <elementRef key="support" minOccurs="0"/>
   <elementRef key="extent" minOccurs="0"/>
   <elementRef key="foliation"
    maxOccurs="unbounded" minOccurs="0"/>
   <elementRef key="collation"
    minOccurs="0"/>
   <elementRef key="condition"
    minOccurs="0"/>
  </sequence>
 </alternate>
</content>
    
\end{Verbatim}

    \item[{Schema Declaration}]
  \mbox{}\hfill\\[-10pt]\begin{Verbatim}[fontsize=\small]
element supportDesc
{
   att.global.attributes,
   attribute material { "paper" | "parch" | "mixed" }?,
   ( model.pLike+ | ( support?, extent?, foliation*, collation?, condition? ) )
}
\end{Verbatim}

\end{reflist}  \index{surface=<surface>|oddindex}\index{attachment=@attachment!<surface>|oddindex}\index{flipping=@flipping!<surface>|oddindex}
\begin{reflist}
\item[]\begin{specHead}{TEI.surface}{<surface> }defines a written surface as a two-dimensional coordinate space, optionally grouping one or more graphic representations of that space, zones of interest within that space, and transcriptions of the writing within them. [\xref{http://www.tei-c.org/release/doc/tei-p5-doc/en/html/PH.html\#PHFAX}{11.1. Digital Facsimiles} \xref{http://www.tei-c.org/release/doc/tei-p5-doc/en/html/PH.html\#PHZLAB}{11.2.2. Embedded Transcription}]\end{specHead} 
    \item[{Module}]
  transcr
    \item[{Attributes}]
  Attributes att.global (\textit{@xml:id}, \textit{@n}, \textit{@xml:lang}, \textit{@xml:base}, \textit{@xml:space})  (att.global.rendition (\textit{@rend}, \textit{@style}, \textit{@rendition})) (att.global.linking (\textit{@corresp}, \textit{@synch}, \textit{@sameAs}, \textit{@copyOf}, \textit{@next}, \textit{@prev}, \textit{@exclude}, \textit{@select})) (att.global.analytic (\textit{@ana})) (att.global.facs (\textit{@facs})) (att.global.change (\textit{@change})) (att.global.responsibility (\textit{@cert}, \textit{@resp})) (att.global.source (\textit{@source})) att.coordinated (\textit{@start}, \textit{@ulx}, \textit{@uly}, \textit{@lrx}, \textit{@lry}, \textit{@points}) att.declaring (\textit{@decls}) att.typed (\textit{@type}, \textit{@subtype}) \hfil\\[-10pt]\begin{sansreflist}
    \item[@attachment]
  describes the method by which this surface is or was connected to the main surface
\begin{reflist}
    \item[{Status}]
  Optional
    \item[{Datatype}]
  teidata.enumerated
    \item[{Sample values include:}]
  \begin{description}

\item[{glued}]glued in place
\item[{pinned}]pinned or stapled in place
\item[{sewn}]sewn in place
\end{description} 
\end{reflist}  
    \item[@flipping]
  indicates whether the surface is attached and folded in such a way as to provide two writing surfaces
\begin{reflist}
    \item[{Status}]
  Optional
    \item[{Datatype}]
  teidata.truthValue
\end{reflist}  
\end{sansreflist}  
    \item[{Contained by}]
  
    \item[transcr: ]
   facsimile sourceDoc surface surfaceGrp zone
    \item[{May contain}]
  
    \item[analysis: ]
   interp interpGrp span spanGrp\par 
    \item[core: ]
   cb desc gap gb graphic index label lb media milestone note pb\par 
    \item[figures: ]
   figure formula notatedMusic\par 
    \item[linking: ]
   alt altGrp anchor join joinGrp link linkGrp timeline\par 
    \item[textcrit: ]
   app witDetail\par 
    \item[transcr: ]
   addSpan damageSpan delSpan fw line listTranspose metamark space substJoin surface surfaceGrp zone
    \item[{Note}]
  \par
The <surface> element represents any two-dimensional space on some physical surface forming part of the source material, such as a piece of paper, a face of a monument, a billboard, a scroll, a leaf etc.\par
The coordinate space defined by this element may be thought of as a grid {\itshape lrx} - {\itshape ulx} units wide and {\itshape uly} - {\itshape lry} units high.\par
The <surface> element may contain graphic representations or transcriptions of written zones, or both. The coordinate values used by every <zone> element contained by this element are to be understood with reference to the same grid.\par
Where it is useful or meaningful to do so, any grouping of multiple <surface> elements may be indicated using the <surfaceGrp> elements.
    \item[{Example}]
  \leavevmode\bgroup\exampleFont \begin{shaded}\noindent\mbox{}{<\textbf{facsimile}>}\mbox{}\newline 
\hspace*{6pt}{<\textbf{surface}\hspace*{6pt}{lrx}="{200}"\hspace*{6pt}{lry}="{300}"\hspace*{6pt}{ulx}="{0}"\hspace*{6pt}{uly}="{0}">}\mbox{}\newline 
\hspace*{6pt}\hspace*{6pt}{<\textbf{graphic}\hspace*{6pt}{url}="{Bovelles-49r.png}"/>}\mbox{}\newline 
\hspace*{6pt}{</\textbf{surface}>}\mbox{}\newline 
{</\textbf{facsimile}>}\end{shaded}\egroup 


    \item[{Content model}]
  \mbox{}\hfill\\[-10pt]\begin{Verbatim}[fontsize=\small]
<content>
 <sequence>
  <alternate maxOccurs="unbounded"
   minOccurs="0">
   <classRef key="model.global"/>
   <classRef key="model.labelLike"/>
   <classRef key="model.graphicLike"/>
  </alternate>
  <sequence maxOccurs="unbounded"
   minOccurs="0">
   <alternate>
    <elementRef key="zone"/>
    <elementRef key="line"/>
    <elementRef key="surface"/>
    <elementRef key="surfaceGrp"/>
   </alternate>
   <classRef key="model.global"
    maxOccurs="unbounded" minOccurs="0"/>
  </sequence>
 </sequence>
</content>
    
\end{Verbatim}

    \item[{Schema Declaration}]
  \mbox{}\hfill\\[-10pt]\begin{Verbatim}[fontsize=\small]
element surface
{
   att.global.attributes,
   att.coordinated.attributes,
   att.declaring.attributes,
   att.typed.attributes,
   attribute attachment { text }?,
   attribute flipping { text }?,
   (
      ( model.global | model.labelLike | model.graphicLike )*,
      ( ( zone | line | surface | surfaceGrp ), model.global* )*
   )
}
\end{Verbatim}

\end{reflist}  \index{surfaceGrp=<surfaceGrp>|oddindex}
\begin{reflist}
\item[]\begin{specHead}{TEI.surfaceGrp}{<surfaceGrp> }defines any kind of useful grouping of written surfaces, for example the recto and verso of a single leaf, which the encoder wishes to treat as a single unit. [\xref{http://www.tei-c.org/release/doc/tei-p5-doc/en/html/PH.html\#PHFAX}{11.1. Digital Facsimiles}]\end{specHead} 
    \item[{Module}]
  transcr
    \item[{Attributes}]
  Attributes att.global (\textit{@xml:id}, \textit{@n}, \textit{@xml:lang}, \textit{@xml:base}, \textit{@xml:space})  (att.global.rendition (\textit{@rend}, \textit{@style}, \textit{@rendition})) (att.global.linking (\textit{@corresp}, \textit{@synch}, \textit{@sameAs}, \textit{@copyOf}, \textit{@next}, \textit{@prev}, \textit{@exclude}, \textit{@select})) (att.global.analytic (\textit{@ana})) (att.global.facs (\textit{@facs})) (att.global.change (\textit{@change})) (att.global.responsibility (\textit{@cert}, \textit{@resp})) (att.global.source (\textit{@source})) att.declaring (\textit{@decls}) att.typed (\textit{@type}, \textit{@subtype}) 
    \item[{Contained by}]
  
    \item[transcr: ]
   facsimile sourceDoc surface surfaceGrp
    \item[{May contain}]
  
    \item[analysis: ]
   interp interpGrp span spanGrp\par 
    \item[core: ]
   cb gap gb index lb milestone note pb\par 
    \item[figures: ]
   figure notatedMusic\par 
    \item[linking: ]
   alt altGrp anchor join joinGrp link linkGrp timeline\par 
    \item[textcrit: ]
   app witDetail\par 
    \item[transcr: ]
   addSpan damageSpan delSpan fw listTranspose metamark space substJoin surface surfaceGrp
    \item[{Note}]
  \par
Where it is useful or meaningful to do so, any grouping of multiple <surface> elements may be indicated using the <surfaceGrp> elements.
    \item[{Example}]
  \leavevmode\bgroup\exampleFont \begin{shaded}\noindent\mbox{}{<\textbf{sourceDoc}>}\mbox{}\newline 
\hspace*{6pt}{<\textbf{surfaceGrp}>}\mbox{}\newline 
\hspace*{6pt}\hspace*{6pt}{<\textbf{surface}\hspace*{6pt}{lrx}="{200}"\hspace*{6pt}{lry}="{300}"\hspace*{6pt}{ulx}="{0}"\mbox{}\newline 
\hspace*{6pt}\hspace*{6pt}\hspace*{6pt}{uly}="{0}">}\mbox{}\newline 
\hspace*{6pt}\hspace*{6pt}\hspace*{6pt}{<\textbf{graphic}\hspace*{6pt}{url}="{Bovelles-49r.png}"/>}\mbox{}\newline 
\hspace*{6pt}\hspace*{6pt}{</\textbf{surface}>}\mbox{}\newline 
\hspace*{6pt}\hspace*{6pt}{<\textbf{surface}\hspace*{6pt}{lrx}="{200}"\hspace*{6pt}{lry}="{300}"\hspace*{6pt}{ulx}="{0}"\mbox{}\newline 
\hspace*{6pt}\hspace*{6pt}\hspace*{6pt}{uly}="{0}">}\mbox{}\newline 
\hspace*{6pt}\hspace*{6pt}\hspace*{6pt}{<\textbf{graphic}\hspace*{6pt}{url}="{Bovelles-49v.png}"/>}\mbox{}\newline 
\hspace*{6pt}\hspace*{6pt}{</\textbf{surface}>}\mbox{}\newline 
\hspace*{6pt}{</\textbf{surfaceGrp}>}\mbox{}\newline 
{</\textbf{sourceDoc}>}\end{shaded}\egroup 


    \item[{Content model}]
  \mbox{}\hfill\\[-10pt]\begin{Verbatim}[fontsize=\small]
<content>
 <alternate maxOccurs="unbounded"
  minOccurs="1">
  <classRef key="model.global"/>
  <elementRef key="surface"/>
  <elementRef key="surfaceGrp"/>
 </alternate>
</content>
    
\end{Verbatim}

    \item[{Schema Declaration}]
  \mbox{}\hfill\\[-10pt]\begin{Verbatim}[fontsize=\small]
element surfaceGrp
{
   att.global.attributes,
   att.declaring.attributes,
   att.typed.attributes,
   ( model.global | surface | surfaceGrp )+
}
\end{Verbatim}

\end{reflist}  \index{surname=<surname>|oddindex}
\begin{reflist}
\item[]\begin{specHead}{TEI.surname}{<surname> }contains a family (inherited) name, as opposed to a given, baptismal, or nick name. [\xref{http://www.tei-c.org/release/doc/tei-p5-doc/en/html/ND.html\#NDPER}{13.2.1. Personal Names}]\end{specHead} 
    \item[{Module}]
  namesdates
    \item[{Attributes}]
  Attributes att.global (\textit{@xml:id}, \textit{@n}, \textit{@xml:lang}, \textit{@xml:base}, \textit{@xml:space})  (att.global.rendition (\textit{@rend}, \textit{@style}, \textit{@rendition})) (att.global.linking (\textit{@corresp}, \textit{@synch}, \textit{@sameAs}, \textit{@copyOf}, \textit{@next}, \textit{@prev}, \textit{@exclude}, \textit{@select})) (att.global.analytic (\textit{@ana})) (att.global.facs (\textit{@facs})) (att.global.change (\textit{@change})) (att.global.responsibility (\textit{@cert}, \textit{@resp})) (att.global.source (\textit{@source})) att.personal (\textit{@full}, \textit{@sort})  (att.naming (\textit{@role}, \textit{@nymRef}) (att.canonical (\textit{@key}, \textit{@ref})) ) att.typed (\textit{@type}, \textit{@subtype}) 
    \item[{Member of}]
  model.persNamePart
    \item[{Contained by}]
  
    \item[analysis: ]
   cl phr s span\par 
    \item[core: ]
   abbr add addrLine address author bibl biblScope citedRange corr date del desc distinct editor email emph expan foreign gloss head headItem headLabel hi item l label measure meeting mentioned name note num orig p pubPlace publisher q quote ref reg resp rs said sic soCalled speaker stage street term textLang time title unclear\par 
    \item[figures: ]
   cell figDesc\par 
    \item[header: ]
   authority catDesc change classCode correspAction creation distributor edition extent funder geoDecl handNote language licence principal rendition scriptNote sponsor tagUsage typeNote\par 
    \item[linking: ]
   ab seg\par 
    \item[msdescription: ]
   accMat acquisition additions catchwords collation colophon condition custEvent decoNote explicit filiation finalRubric foliation heraldry incipit layout material musicNotation objectType origDate origPlace origin provenance rubric secFol signatures source stamp summary support surrogates watermark\par 
    \item[namesdates: ]
   addName affiliation age birth bloc country death district education faith floruit forename genName geogFeat geogName langKnown nameLink nationality occupation offset org orgName persName placeName region residence roleName settlement sex socecStatus surname\par 
    \item[textcrit: ]
   lem rdg wit witDetail witness\par 
    \item[textstructure: ]
   byline closer dateline docAuthor docDate docEdition docImprint imprimatur opener salute signed titlePart trailer\par 
    \item[transcr: ]
   damage fw metamark mod restore retrace secl supplied surplus
    \item[{May contain}]
  
    \item[analysis: ]
   c cl interp interpGrp m pc phr s span spanGrp w\par 
    \item[core: ]
   abbr add address cb choice corr date del distinct email emph expan foreign gap gb gloss graphic hi index lb measure measureGrp media mentioned milestone name note num orig pb ptr ref reg rs sic soCalled term time title unclear\par 
    \item[figures: ]
   figure formula notatedMusic\par 
    \item[gaiji: ]
   g\par 
    \item[header: ]
   idno\par 
    \item[linking: ]
   alt altGrp anchor join joinGrp link linkGrp seg timeline\par 
    \item[msdescription: ]
   catchwords depth dim dimensions height heraldry locus locusGrp material objectType origDate origPlace secFol signatures stamp watermark width\par 
    \item[namesdates: ]
   addName affiliation bloc climate country district forename genName geo geogFeat geogName location nameLink offset orgName persName placeName population region roleName settlement state surname terrain trait\par 
    \item[textcrit: ]
   app witDetail\par 
    \item[transcr: ]
   addSpan am damage damageSpan delSpan ex fw handShift listTranspose metamark mod redo restore retrace secl space subst substJoin supplied surplus undo\par character data
    \item[{Example}]
  \leavevmode\bgroup\exampleFont \begin{shaded}\noindent\mbox{}{<\textbf{surname}\hspace*{6pt}{type}="{combine}">}St John Stevas{</\textbf{surname}>}\end{shaded}\egroup 


    \item[{Content model}]
  \mbox{}\hfill\\[-10pt]\begin{Verbatim}[fontsize=\small]
<content>
 <macroRef key="macro.phraseSeq"/>
</content>
    
\end{Verbatim}

    \item[{Schema Declaration}]
  \mbox{}\hfill\\[-10pt]\begin{Verbatim}[fontsize=\small]
element surname
{
   att.global.attributes,
   att.personal.attributes,
   att.typed.attributes,
   macro.phraseSeq}
\end{Verbatim}

\end{reflist}  \index{surplus=<surplus>|oddindex}\index{reason=@reason!<surplus>|oddindex}
\begin{reflist}
\item[]\begin{specHead}{TEI.surplus}{<surplus> }marks text present in the source which the editor believes to be superfluous or redundant. [\xref{http://www.tei-c.org/release/doc/tei-p5-doc/en/html/PH.html\#PHDA}{11.3.3.1. Damage, Illegibility, and Supplied Text}]\end{specHead} 
    \item[{Module}]
  transcr
    \item[{Attributes}]
  Attributes att.global (\textit{@xml:id}, \textit{@n}, \textit{@xml:lang}, \textit{@xml:base}, \textit{@xml:space})  (att.global.rendition (\textit{@rend}, \textit{@style}, \textit{@rendition})) (att.global.linking (\textit{@corresp}, \textit{@synch}, \textit{@sameAs}, \textit{@copyOf}, \textit{@next}, \textit{@prev}, \textit{@exclude}, \textit{@select})) (att.global.analytic (\textit{@ana})) (att.global.facs (\textit{@facs})) (att.global.change (\textit{@change})) (att.global.responsibility (\textit{@cert}, \textit{@resp})) (att.global.source (\textit{@source})) att.editLike (\textit{@evidence}, \textit{@instant})  (att.dimensions (\textit{@unit}, \textit{@quantity}, \textit{@extent}, \textit{@precision}, \textit{@scope}) (att.ranging (\textit{@atLeast}, \textit{@atMost}, \textit{@min}, \textit{@max}, \textit{@confidence})) ) \hfil\\[-10pt]\begin{sansreflist}
    \item[@reason]
  one or more words indicating why this text is believed to be superfluous, e.g. \textit{repeated}, \textit{interpolated} etc.
\begin{reflist}
    \item[{Status}]
  Optional
    \item[{Datatype}]
  1–∞ occurrences of teidata.word separated by whitespace
\end{reflist}  
\end{sansreflist}  
    \item[{Member of}]
  model.pPart.transcriptional
    \item[{Contained by}]
  
    \item[analysis: ]
   cl pc phr s w\par 
    \item[core: ]
   abbr add addrLine author bibl biblScope citedRange corr date del distinct editor email emph expan foreign gloss head headItem headLabel hi item l label measure mentioned name note num orig p pubPlace publisher q quote ref reg rs said sic soCalled speaker stage street term textLang time title unclear\par 
    \item[figures: ]
   cell\par 
    \item[header: ]
   change distributor edition extent geoDecl handNote licence scriptNote typeNote\par 
    \item[linking: ]
   ab seg\par 
    \item[msdescription: ]
   accMat acquisition additions catchwords collation colophon condition custEvent decoNote explicit filiation finalRubric foliation heraldry incipit layout material musicNotation objectType origDate origPlace origin provenance rubric secFol signatures source stamp summary support surrogates watermark\par 
    \item[namesdates: ]
   addName affiliation birth bloc country death district education faith floruit forename genName geogFeat geogName nameLink nationality occupation offset orgName persName placeName region residence roleName settlement sex socecStatus surname\par 
    \item[textcrit: ]
   lem rdg wit witDetail\par 
    \item[textstructure: ]
   byline closer dateline docAuthor docDate docEdition docImprint imprimatur opener salute signed titlePart trailer\par 
    \item[transcr: ]
   am damage fw metamark mod restore retrace secl supplied surplus
    \item[{May contain}]
  
    \item[analysis: ]
   c cl interp interpGrp m pc phr s span spanGrp w\par 
    \item[core: ]
   abbr add address bibl biblStruct cb choice cit corr date del desc distinct email emph expan foreign gap gb gloss graphic hi index l label lb lg list listBibl measure measureGrp media mentioned milestone name note num orig pb ptr q quote ref reg rs said sic soCalled stage term time title unclear\par 
    \item[figures: ]
   figure formula notatedMusic table\par 
    \item[gaiji: ]
   g\par 
    \item[header: ]
   biblFull idno\par 
    \item[linking: ]
   alt altGrp anchor join joinGrp link linkGrp seg timeline\par 
    \item[msdescription: ]
   catchwords depth dim dimensions height heraldry locus locusGrp material msDesc objectType origDate origPlace secFol signatures stamp watermark width\par 
    \item[namesdates: ]
   addName affiliation bloc climate country district forename genName geo geogFeat geogName listEvent listNym listOrg listPerson listPlace location nameLink offset orgName persName placeName population region roleName settlement state surname terrain trait\par 
    \item[textcrit: ]
   app listApp listWit witDetail\par 
    \item[textstructure: ]
   floatingText\par 
    \item[transcr: ]
   addSpan am damage damageSpan delSpan ex fw handShift listTranspose metamark mod redo restore retrace secl space subst substJoin supplied surplus undo\par character data
    \item[{Example}]
  \leavevmode\bgroup\exampleFont \begin{shaded}\noindent\mbox{}I am dr Sr yrs\mbox{}\newline 
{<\textbf{surplus}\hspace*{6pt}{reason}="{repeated}">}yrs{</\textbf{surplus}>}\mbox{}\newline 
 Sydney Smith\end{shaded}\egroup 


    \item[{Content model}]
  \mbox{}\hfill\\[-10pt]\begin{Verbatim}[fontsize=\small]
<content>
 <macroRef key="macro.paraContent"/>
</content>
    
\end{Verbatim}

    \item[{Schema Declaration}]
  \mbox{}\hfill\\[-10pt]\begin{Verbatim}[fontsize=\small]
element surplus
{
   att.global.attributes,
   att.editLike.attributes,
   attribute reason { list { + } }?,
   macro.paraContent}
\end{Verbatim}

\end{reflist}  \index{surrogates=<surrogates>|oddindex}
\begin{reflist}
\item[]\begin{specHead}{TEI.surrogates}{<surrogates> }contains information about any representations of the manuscript being described which may exist in the holding institution or elsewhere. [\xref{http://www.tei-c.org/release/doc/tei-p5-doc/en/html/MS.html\#msad}{10.9. Additional Information}]\end{specHead} 
    \item[{Module}]
  msdescription
    \item[{Attributes}]
  Attributes att.global (\textit{@xml:id}, \textit{@n}, \textit{@xml:lang}, \textit{@xml:base}, \textit{@xml:space})  (att.global.rendition (\textit{@rend}, \textit{@style}, \textit{@rendition})) (att.global.linking (\textit{@corresp}, \textit{@synch}, \textit{@sameAs}, \textit{@copyOf}, \textit{@next}, \textit{@prev}, \textit{@exclude}, \textit{@select})) (att.global.analytic (\textit{@ana})) (att.global.facs (\textit{@facs})) (att.global.change (\textit{@change})) (att.global.responsibility (\textit{@cert}, \textit{@resp})) (att.global.source (\textit{@source}))
    \item[{Contained by}]
  
    \item[msdescription: ]
   additional
    \item[{May contain}]
  
    \item[analysis: ]
   c cl interp interpGrp m pc phr s span spanGrp w\par 
    \item[core: ]
   abbr add address bibl biblStruct cb choice cit corr date del desc distinct email emph expan foreign gap gb gloss graphic hi index l label lb lg list listBibl measure measureGrp media mentioned milestone name note num orig p pb ptr q quote ref reg rs said sic soCalled sp stage term time title unclear\par 
    \item[figures: ]
   figure formula notatedMusic table\par 
    \item[gaiji: ]
   g\par 
    \item[header: ]
   biblFull idno\par 
    \item[linking: ]
   ab alt altGrp anchor join joinGrp link linkGrp seg timeline\par 
    \item[msdescription: ]
   catchwords depth dim dimensions height heraldry locus locusGrp material msDesc objectType origDate origPlace secFol signatures stamp watermark width\par 
    \item[namesdates: ]
   addName affiliation bloc climate country district forename genName geo geogFeat geogName listEvent listNym listOrg listPerson listPlace location nameLink offset orgName persName placeName population region roleName settlement state surname terrain trait\par 
    \item[textcrit: ]
   app listApp listWit witDetail\par 
    \item[textstructure: ]
   floatingText\par 
    \item[transcr: ]
   addSpan am damage damageSpan delSpan ex fw handShift listTranspose metamark mod redo restore retrace secl space subst substJoin supplied surplus undo\par character data
    \item[{Example}]
  \leavevmode\bgroup\exampleFont \begin{shaded}\noindent\mbox{}{<\textbf{surrogates}>}\mbox{}\newline 
\hspace*{6pt}{<\textbf{bibl}>}\mbox{}\newline 
\hspace*{6pt}\hspace*{6pt}{<\textbf{title}\hspace*{6pt}{type}="{gmd}">}diapositive{</\textbf{title}>}\mbox{}\newline 
\hspace*{6pt}\hspace*{6pt}{<\textbf{idno}>}AM 74 a, fol.{</\textbf{idno}>}\mbox{}\newline 
\hspace*{6pt}\hspace*{6pt}{<\textbf{date}>}May 1984{</\textbf{date}>}\mbox{}\newline 
\hspace*{6pt}{</\textbf{bibl}>}\mbox{}\newline 
\hspace*{6pt}{<\textbf{bibl}>}\mbox{}\newline 
\hspace*{6pt}\hspace*{6pt}{<\textbf{title}\hspace*{6pt}{type}="{gmd}">}b/w prints{</\textbf{title}>}\mbox{}\newline 
\hspace*{6pt}\hspace*{6pt}{<\textbf{idno}>}AM 75 a, fol.{</\textbf{idno}>}\mbox{}\newline 
\hspace*{6pt}\hspace*{6pt}{<\textbf{date}>}1972{</\textbf{date}>}\mbox{}\newline 
\hspace*{6pt}{</\textbf{bibl}>}\mbox{}\newline 
{</\textbf{surrogates}>}\end{shaded}\egroup 


    \item[{Content model}]
  \mbox{}\hfill\\[-10pt]\begin{Verbatim}[fontsize=\small]
<content>
 <macroRef key="macro.specialPara"/>
</content>
    
\end{Verbatim}

    \item[{Schema Declaration}]
  \mbox{}\hfill\\[-10pt]\begin{Verbatim}[fontsize=\small]
element surrogates { att.global.attributes, macro.specialPara }
\end{Verbatim}

\end{reflist}  \index{table=<table>|oddindex}\index{rows=@rows!<table>|oddindex}\index{cols=@cols!<table>|oddindex}
\begin{reflist}
\item[]\begin{specHead}{TEI.table}{<table> }contains text displayed in tabular form, in rows and columns. [\xref{http://www.tei-c.org/release/doc/tei-p5-doc/en/html/FT.html\#FTTAB1}{14.1.1. TEI Tables}]\end{specHead} 
    \item[{Module}]
  figures
    \item[{Attributes}]
  Attributes att.global (\textit{@xml:id}, \textit{@n}, \textit{@xml:lang}, \textit{@xml:base}, \textit{@xml:space})  (att.global.rendition (\textit{@rend}, \textit{@style}, \textit{@rendition})) (att.global.linking (\textit{@corresp}, \textit{@synch}, \textit{@sameAs}, \textit{@copyOf}, \textit{@next}, \textit{@prev}, \textit{@exclude}, \textit{@select})) (att.global.analytic (\textit{@ana})) (att.global.facs (\textit{@facs})) (att.global.change (\textit{@change})) (att.global.responsibility (\textit{@cert}, \textit{@resp})) (att.global.source (\textit{@source})) att.typed (\textit{@type}, \textit{@subtype}) \hfil\\[-10pt]\begin{sansreflist}
    \item[@rows]
  indicates the number of rows in the table.
\begin{reflist}
    \item[{Status}]
  Optional
    \item[{Datatype}]
  teidata.count
    \item[{Note}]
  \par
If no number is supplied, an application must calculate the number of rows.\par
Rows should be presented from top to bottom.
\end{reflist}  
    \item[@cols]
  (columns) indicates the number of columns in each row of the table.
\begin{reflist}
    \item[{Status}]
  Optional
    \item[{Datatype}]
  teidata.count
    \item[{Note}]
  \par
If no number is supplied, an application must calculate the number of columns.\par
Within each row, columns should be presented left to right.
\end{reflist}  
\end{sansreflist}  
    \item[{Member of}]
  model.listLike
    \item[{Contained by}]
  
    \item[core: ]
   add corr del desc emph head hi item l meeting note orig p q quote ref reg said sic sp stage title unclear\par 
    \item[figures: ]
   cell figDesc figure\par 
    \item[header: ]
   abstract change handNote licence rendition scriptNote sourceDesc tagUsage typeNote\par 
    \item[linking: ]
   ab seg\par 
    \item[msdescription: ]
   accMat acquisition additions collation condition custEvent decoNote filiation foliation layout musicNotation origin provenance signatures source summary support surrogates\par 
    \item[namesdates: ]
   occupation\par 
    \item[textcrit: ]
   lem rdg witness\par 
    \item[textstructure: ]
   argument back body div docEdition epigraph imprimatur postscript salute signed titlePart trailer\par 
    \item[transcr: ]
   damage metamark mod restore retrace secl supplied surplus
    \item[{May contain}]
  
    \item[analysis: ]
   interp interpGrp span spanGrp\par 
    \item[core: ]
   cb gap gb graphic head index lb media meeting milestone note pb\par 
    \item[figures: ]
   figure formula notatedMusic row\par 
    \item[linking: ]
   alt altGrp anchor join joinGrp link linkGrp timeline\par 
    \item[textcrit: ]
   app witDetail\par 
    \item[textstructure: ]
   argument byline closer dateline docAuthor docDate epigraph postscript salute signed trailer\par 
    \item[transcr: ]
   addSpan damageSpan delSpan fw listTranspose metamark space substJoin
    \item[{Note}]
  \par
Contains an optional heading and a series of rows.\par
Any rendition information should be supplied using the global {\itshape rend} attribute, at the table, row, or cell level as appropriate.
    \item[{Example}]
  \leavevmode\bgroup\exampleFont \begin{shaded}\noindent\mbox{}{<\textbf{table}\hspace*{6pt}{cols}="{4}"\hspace*{6pt}{rows}="{4}">}\mbox{}\newline 
\hspace*{6pt}{<\textbf{head}>}Poor Men's Lodgings in Norfolk (Mayhew, 1843){</\textbf{head}>}\mbox{}\newline 
\hspace*{6pt}{<\textbf{row}\hspace*{6pt}{role}="{label}">}\mbox{}\newline 
\hspace*{6pt}\hspace*{6pt}{<\textbf{cell}\hspace*{6pt}{role}="{data}"/>}\mbox{}\newline 
\hspace*{6pt}\hspace*{6pt}{<\textbf{cell}\hspace*{6pt}{role}="{data}">}Dossing Cribs or Lodging Houses{</\textbf{cell}>}\mbox{}\newline 
\hspace*{6pt}\hspace*{6pt}{<\textbf{cell}\hspace*{6pt}{role}="{data}">}Beds{</\textbf{cell}>}\mbox{}\newline 
\hspace*{6pt}\hspace*{6pt}{<\textbf{cell}\hspace*{6pt}{role}="{data}">}Needys or Nightly Lodgers{</\textbf{cell}>}\mbox{}\newline 
\hspace*{6pt}{</\textbf{row}>}\mbox{}\newline 
\hspace*{6pt}{<\textbf{row}\hspace*{6pt}{role}="{data}">}\mbox{}\newline 
\hspace*{6pt}\hspace*{6pt}{<\textbf{cell}\hspace*{6pt}{role}="{label}">}Bury St Edmund's{</\textbf{cell}>}\mbox{}\newline 
\hspace*{6pt}\hspace*{6pt}{<\textbf{cell}\hspace*{6pt}{role}="{data}">}5{</\textbf{cell}>}\mbox{}\newline 
\hspace*{6pt}\hspace*{6pt}{<\textbf{cell}\hspace*{6pt}{role}="{data}">}8{</\textbf{cell}>}\mbox{}\newline 
\hspace*{6pt}\hspace*{6pt}{<\textbf{cell}\hspace*{6pt}{role}="{data}">}128{</\textbf{cell}>}\mbox{}\newline 
\hspace*{6pt}{</\textbf{row}>}\mbox{}\newline 
\hspace*{6pt}{<\textbf{row}\hspace*{6pt}{role}="{data}">}\mbox{}\newline 
\hspace*{6pt}\hspace*{6pt}{<\textbf{cell}\hspace*{6pt}{role}="{label}">}Thetford{</\textbf{cell}>}\mbox{}\newline 
\hspace*{6pt}\hspace*{6pt}{<\textbf{cell}\hspace*{6pt}{role}="{data}">}3{</\textbf{cell}>}\mbox{}\newline 
\hspace*{6pt}\hspace*{6pt}{<\textbf{cell}\hspace*{6pt}{role}="{data}">}6{</\textbf{cell}>}\mbox{}\newline 
\hspace*{6pt}\hspace*{6pt}{<\textbf{cell}\hspace*{6pt}{role}="{data}">}36{</\textbf{cell}>}\mbox{}\newline 
\hspace*{6pt}{</\textbf{row}>}\mbox{}\newline 
\hspace*{6pt}{<\textbf{row}\hspace*{6pt}{role}="{data}">}\mbox{}\newline 
\hspace*{6pt}\hspace*{6pt}{<\textbf{cell}\hspace*{6pt}{role}="{label}">}Attleboro'{</\textbf{cell}>}\mbox{}\newline 
\hspace*{6pt}\hspace*{6pt}{<\textbf{cell}\hspace*{6pt}{role}="{data}">}3{</\textbf{cell}>}\mbox{}\newline 
\hspace*{6pt}\hspace*{6pt}{<\textbf{cell}\hspace*{6pt}{role}="{data}">}5{</\textbf{cell}>}\mbox{}\newline 
\hspace*{6pt}\hspace*{6pt}{<\textbf{cell}\hspace*{6pt}{role}="{data}">}20{</\textbf{cell}>}\mbox{}\newline 
\hspace*{6pt}{</\textbf{row}>}\mbox{}\newline 
\hspace*{6pt}{<\textbf{row}\hspace*{6pt}{role}="{data}">}\mbox{}\newline 
\hspace*{6pt}\hspace*{6pt}{<\textbf{cell}\hspace*{6pt}{role}="{label}">}Wymondham{</\textbf{cell}>}\mbox{}\newline 
\hspace*{6pt}\hspace*{6pt}{<\textbf{cell}\hspace*{6pt}{role}="{data}">}1{</\textbf{cell}>}\mbox{}\newline 
\hspace*{6pt}\hspace*{6pt}{<\textbf{cell}\hspace*{6pt}{role}="{data}">}11{</\textbf{cell}>}\mbox{}\newline 
\hspace*{6pt}\hspace*{6pt}{<\textbf{cell}\hspace*{6pt}{role}="{data}">}22{</\textbf{cell}>}\mbox{}\newline 
\hspace*{6pt}{</\textbf{row}>}\mbox{}\newline 
{</\textbf{table}>}\end{shaded}\egroup 


    \item[{Content model}]
  \mbox{}\hfill\\[-10pt]\begin{Verbatim}[fontsize=\small]
<content>
 <sequence>
  <alternate maxOccurs="unbounded"
   minOccurs="0">
   <classRef key="model.headLike"/>
   <classRef key="model.global"/>
  </alternate>
  <alternate>
   <sequence maxOccurs="unbounded"
    minOccurs="1">
    <elementRef key="row"/>
    <classRef key="model.global"
     maxOccurs="unbounded" minOccurs="0"/>
   </sequence>
   <sequence maxOccurs="unbounded"
    minOccurs="1">
    <classRef key="model.graphicLike"/>
    <classRef key="model.global"
     maxOccurs="unbounded" minOccurs="0"/>
   </sequence>
  </alternate>
  <sequence maxOccurs="unbounded"
   minOccurs="0">
   <classRef key="model.divBottom"/>
   <classRef key="model.global"
    maxOccurs="unbounded" minOccurs="0"/>
  </sequence>
 </sequence>
</content>
    
\end{Verbatim}

    \item[{Schema Declaration}]
  \mbox{}\hfill\\[-10pt]\begin{Verbatim}[fontsize=\small]
element table
{
   att.global.attributes,
   att.typed.attributes,
   attribute rows { text }?,
   attribute cols { text }?,
   (
      ( model.headLike | model.global )*,
      ( ( row, model.global* )+ | ( model.graphicLike, model.global* )+ ),
      ( model.divBottom, model.global* )*
   )
}
\end{Verbatim}

\end{reflist}  \index{tagUsage=<tagUsage>|oddindex}\index{gi=@gi!<tagUsage>|oddindex}\index{occurs=@occurs!<tagUsage>|oddindex}\index{withId=@withId!<tagUsage>|oddindex}\index{render=@render!<tagUsage>|oddindex}
\begin{reflist}
\item[]\begin{specHead}{TEI.tagUsage}{<tagUsage> }documents the usage of a specific element within a specified document. [\xref{http://www.tei-c.org/release/doc/tei-p5-doc/en/html/HD.html\#HD57}{2.3.4. The Tagging Declaration}]\end{specHead} 
    \item[{Module}]
  header
    \item[{Attributes}]
  Attributes att.global (\textit{@xml:id}, \textit{@n}, \textit{@xml:lang}, \textit{@xml:base}, \textit{@xml:space})  (att.global.rendition (\textit{@rend}, \textit{@style}, \textit{@rendition})) (att.global.linking (\textit{@corresp}, \textit{@synch}, \textit{@sameAs}, \textit{@copyOf}, \textit{@next}, \textit{@prev}, \textit{@exclude}, \textit{@select})) (att.global.analytic (\textit{@ana})) (att.global.facs (\textit{@facs})) (att.global.change (\textit{@change})) (att.global.responsibility (\textit{@cert}, \textit{@resp})) (att.global.source (\textit{@source})) \hfil\\[-10pt]\begin{sansreflist}
    \item[@gi]
  (generic identifier) specifies the name (generic identifier) of the element indicated by the tag, within the namespace indicated by the parent <namespace> element.
\begin{reflist}
    \item[{Status}]
  Required
    \item[{Datatype}]
  teidata.name
\end{reflist}  
    \item[@occurs]
  specifies the number of occurrences of this element within the text.
\begin{reflist}
    \item[{Status}]
  Recommended
    \item[{Datatype}]
  teidata.count
\end{reflist}  
    \item[@withId]
  (with unique identifier) specifies the number of occurrences of this element within the text which bear a distinct value for the global {\itshape xml:id} attribute.
\begin{reflist}
    \item[{Status}]
  Recommended
    \item[{Datatype}]
  teidata.count
\end{reflist}  
    \item[@render]
  specifies the identifier of a <rendition> element which defines how this element was rendered in the source text.
\begin{reflist}
    \item[\xref{http://www.tei-c.org/Activities/Council/Working/tcw27.xml}{Deprecated}]
  will be removed on 2017-01-01
    \item[{Status}]
  Optional
    \item[{Datatype}]
  1–∞ occurrences of teidata.pointer separated by whitespace
    \item[{Note}]
  \par
The recommended way of specifying a default rendition for a set of elements is to use the {\itshape selector} attribute on the <rendition> element.
\end{reflist}  
\end{sansreflist}  
    \item[{Contained by}]
  
    \item[header: ]
   namespace
    \item[{May contain}]
  
    \item[core: ]
   abbr address bibl biblStruct choice cit date desc distinct email emph expan foreign gloss hi label list listBibl measure measureGrp mentioned name num ptr q quote ref rs said soCalled stage term time title\par 
    \item[figures: ]
   table\par 
    \item[header: ]
   biblFull idno\par 
    \item[msdescription: ]
   catchwords depth dim dimensions height heraldry locus locusGrp material msDesc objectType origDate origPlace secFol signatures stamp watermark width\par 
    \item[namesdates: ]
   addName affiliation bloc climate country district forename genName geo geogFeat geogName listEvent listNym listOrg listPerson listPlace location nameLink offset orgName persName placeName population region roleName settlement state surname terrain trait\par 
    \item[textcrit: ]
   listApp listWit\par 
    \item[textstructure: ]
   floatingText\par 
    \item[transcr: ]
   am ex subst\par character data
    \item[{Example}]
  \leavevmode\bgroup\exampleFont \begin{shaded}\noindent\mbox{}{<\textbf{tagsDecl}\hspace*{6pt}{partial}="{true}">}\mbox{}\newline 
\hspace*{6pt}{<\textbf{rendition}\hspace*{6pt}{scheme}="{css}"\mbox{}\newline 
\hspace*{6pt}\hspace*{6pt}{selector}="{foreign hi}"\hspace*{6pt}{xml:id}="{it}">} font-style: italic; {</\textbf{rendition}>}\mbox{}\newline 
\textit{<!-- ... -->}\mbox{}\newline 
\hspace*{6pt}{<\textbf{namespace}\hspace*{6pt}{name}="{http://www.tei-c.org/ns/1.0}">}\mbox{}\newline 
\hspace*{6pt}\hspace*{6pt}{<\textbf{tagUsage}\hspace*{6pt}{gi}="{hi}"\hspace*{6pt}{occurs}="{28}"\hspace*{6pt}{withId}="{2}">} Used to mark English words italicized in the copy text.{</\textbf{tagUsage}>}\mbox{}\newline 
\hspace*{6pt}\hspace*{6pt}{<\textbf{tagUsage}\hspace*{6pt}{gi}="{foreign}">}Used to mark non-English words in the copy text.{</\textbf{tagUsage}>}\mbox{}\newline 
\textit{<!-- ... -->}\mbox{}\newline 
\hspace*{6pt}{</\textbf{namespace}>}\mbox{}\newline 
{</\textbf{tagsDecl}>}\end{shaded}\egroup 


    \item[{Example}]
  \leavevmode\bgroup\exampleFont \begin{shaded}\noindent\mbox{}{<\textbf{tagsDecl}>}\mbox{}\newline 
\textit{<!-- deprecated usage -->}\mbox{}\newline 
\hspace*{6pt}{<\textbf{rendition}\hspace*{6pt}{xml:id}="{dit}">}Render using a slant or italic variant on the current font{</\textbf{rendition}>}\mbox{}\newline 
\textit{<!-- ... -->}\mbox{}\newline 
\hspace*{6pt}{<\textbf{namespace}\hspace*{6pt}{name}="{http://www.tei-c.org/ns/1.0}">}\mbox{}\newline 
\hspace*{6pt}\hspace*{6pt}{<\textbf{tagUsage}\hspace*{6pt}{gi}="{hi}"\hspace*{6pt}{occurs}="{28}"\mbox{}\newline 
\hspace*{6pt}\hspace*{6pt}\hspace*{6pt}{render}="{\#dit}"\hspace*{6pt}{withId}="{2}">} Used to mark English words\mbox{}\newline 
\hspace*{6pt}\hspace*{6pt}\hspace*{6pt}\hspace*{6pt} italicized in the copy text.{</\textbf{tagUsage}>}\mbox{}\newline 
\hspace*{6pt}\hspace*{6pt}{<\textbf{tagUsage}\hspace*{6pt}{gi}="{foreign}"\hspace*{6pt}{render}="{\#dit}">}Used to mark non-English words in the copy text.{</\textbf{tagUsage}>}\mbox{}\newline 
\textit{<!-- ... -->}\mbox{}\newline 
\hspace*{6pt}{</\textbf{namespace}>}\mbox{}\newline 
{</\textbf{tagsDecl}>}\end{shaded}\egroup 


    \item[{Content model}]
  \mbox{}\hfill\\[-10pt]\begin{Verbatim}[fontsize=\small]
<content>
 <macroRef key="macro.limitedContent"/>
</content>
    
\end{Verbatim}

    \item[{Schema Declaration}]
  \mbox{}\hfill\\[-10pt]\begin{Verbatim}[fontsize=\small]
element tagUsage
{
   att.global.attributes,
   attribute gi { text },
   attribute occurs { text }?,
   attribute withId { text }?,
   attribute render { list { + } }?,
   macro.limitedContent}
\end{Verbatim}

\end{reflist}  \index{tagsDecl=<tagsDecl>|oddindex}\index{partial=@partial!<tagsDecl>|oddindex}
\begin{reflist}
\item[]\begin{specHead}{TEI.tagsDecl}{<tagsDecl> }(tagging declaration) provides detailed information about the tagging applied to a document. [\xref{http://www.tei-c.org/release/doc/tei-p5-doc/en/html/HD.html\#HD57}{2.3.4. The Tagging Declaration} \xref{http://www.tei-c.org/release/doc/tei-p5-doc/en/html/HD.html\#HD5}{2.3. The Encoding Description}]\end{specHead} 
    \item[{Module}]
  header
    \item[{Attributes}]
  Attributes att.global (\textit{@xml:id}, \textit{@n}, \textit{@xml:lang}, \textit{@xml:base}, \textit{@xml:space})  (att.global.rendition (\textit{@rend}, \textit{@style}, \textit{@rendition})) (att.global.linking (\textit{@corresp}, \textit{@synch}, \textit{@sameAs}, \textit{@copyOf}, \textit{@next}, \textit{@prev}, \textit{@exclude}, \textit{@select})) (att.global.analytic (\textit{@ana})) (att.global.facs (\textit{@facs})) (att.global.change (\textit{@change})) (att.global.responsibility (\textit{@cert}, \textit{@resp})) (att.global.source (\textit{@source})) \hfil\\[-10pt]\begin{sansreflist}
    \item[@partial]
  indicates whether the element types listed exhaustively include all those found within <text>, or represent only a subset.
\begin{reflist}
    \item[{Status}]
  Recommended
    \item[{Datatype}]
  teidata.truthValue
    \item[{Note}]
  \par
TEI recommended practice is to specify this attribute. When the <tagUsage> elements inside <tagsDecl> are used to list each of the element types in the associated <text>, the value should be given as false. When the <tagUsage> elements inside <tagsDecl> are used to provide usage information or default renditions for only a subset of the elements types within the associated <text>, the value should be true.
\end{reflist}  
\end{sansreflist}  
    \item[{Member of}]
  model.encodingDescPart
    \item[{Contained by}]
  
    \item[header: ]
   encodingDesc
    \item[{May contain}]
  
    \item[header: ]
   namespace rendition
    \item[{Example}]
  \leavevmode\bgroup\exampleFont \begin{shaded}\noindent\mbox{}{<\textbf{tagsDecl}\hspace*{6pt}{partial}="{true}">}\mbox{}\newline 
\hspace*{6pt}{<\textbf{rendition}\hspace*{6pt}{scheme}="{css}"\mbox{}\newline 
\hspace*{6pt}\hspace*{6pt}{selector}="{emph hi name title}"\hspace*{6pt}{xml:id}="{rend-it}">}font-style: italic;{</\textbf{rendition}>}\mbox{}\newline 
\hspace*{6pt}{<\textbf{namespace}\hspace*{6pt}{name}="{http://www.tei-c.org/ns/1.0}">}\mbox{}\newline 
\hspace*{6pt}\hspace*{6pt}{<\textbf{tagUsage}\hspace*{6pt}{gi}="{hi}"\hspace*{6pt}{occurs}="{467}"/>}\mbox{}\newline 
\hspace*{6pt}\hspace*{6pt}{<\textbf{tagUsage}\hspace*{6pt}{gi}="{title}"\hspace*{6pt}{occurs}="{45}"/>}\mbox{}\newline 
\hspace*{6pt}{</\textbf{namespace}>}\mbox{}\newline 
\hspace*{6pt}{<\textbf{namespace}\hspace*{6pt}{name}="{http://docbook.org/ns/docbook}">}\mbox{}\newline 
\hspace*{6pt}\hspace*{6pt}{<\textbf{tagUsage}\hspace*{6pt}{gi}="{para}"\hspace*{6pt}{occurs}="{10}"/>}\mbox{}\newline 
\hspace*{6pt}{</\textbf{namespace}>}\mbox{}\newline 
{</\textbf{tagsDecl}>}\end{shaded}\egroup 

If the {\itshape partial} attribute were not specified here, the implication would be that the document in question contains only <hi>, <title>, and \texttt{<para>} elements.
    \item[{Content model}]
  \mbox{}\hfill\\[-10pt]\begin{Verbatim}[fontsize=\small]
<content>
 <sequence>
  <elementRef key="rendition"
   maxOccurs="unbounded" minOccurs="0"/>
  <elementRef key="namespace"
   maxOccurs="unbounded" minOccurs="0"/>
 </sequence>
</content>
    
\end{Verbatim}

    \item[{Schema Declaration}]
  \mbox{}\hfill\\[-10pt]\begin{Verbatim}[fontsize=\small]
element tagsDecl
{
   att.global.attributes,
   attribute partial { text }?,
   ( rendition*, namespace* )
}
\end{Verbatim}

\end{reflist}  \index{taxonomy=<taxonomy>|oddindex}
\begin{reflist}
\item[]\begin{specHead}{TEI.taxonomy}{<taxonomy> }defines a typology either implicitly, by means of a bibliographic citation, or explicitly by a structured taxonomy. [\xref{http://www.tei-c.org/release/doc/tei-p5-doc/en/html/HD.html\#HD55}{2.3.7. The Classification Declaration}]\end{specHead} 
    \item[{Module}]
  header
    \item[{Attributes}]
  Attributes att.global (\textit{@xml:id}, \textit{@n}, \textit{@xml:lang}, \textit{@xml:base}, \textit{@xml:space})  (att.global.rendition (\textit{@rend}, \textit{@style}, \textit{@rendition})) (att.global.linking (\textit{@corresp}, \textit{@synch}, \textit{@sameAs}, \textit{@copyOf}, \textit{@next}, \textit{@prev}, \textit{@exclude}, \textit{@select})) (att.global.analytic (\textit{@ana})) (att.global.facs (\textit{@facs})) (att.global.change (\textit{@change})) (att.global.responsibility (\textit{@cert}, \textit{@resp})) (att.global.source (\textit{@source}))
    \item[{Contained by}]
  
    \item[header: ]
   classDecl taxonomy
    \item[{May contain}]
  
    \item[core: ]
   bibl biblStruct desc gloss listBibl\par 
    \item[header: ]
   biblFull category taxonomy\par 
    \item[msdescription: ]
   msDesc
    \item[{Note}]
  \par
Nested taxonomies are common in many fields, so the <taxonomy> element can be nested.
    \item[{Example}]
  \leavevmode\bgroup\exampleFont \begin{shaded}\noindent\mbox{}{<\textbf{taxonomy}\hspace*{6pt}{xml:id}="{tax.b}">}\mbox{}\newline 
\hspace*{6pt}{<\textbf{bibl}>}Brown Corpus{</\textbf{bibl}>}\mbox{}\newline 
\hspace*{6pt}{<\textbf{category}\hspace*{6pt}{xml:id}="{tax.b.a}">}\mbox{}\newline 
\hspace*{6pt}\hspace*{6pt}{<\textbf{catDesc}>}Press Reportage{</\textbf{catDesc}>}\mbox{}\newline 
\hspace*{6pt}\hspace*{6pt}{<\textbf{category}\hspace*{6pt}{xml:id}="{tax.b.a1}">}\mbox{}\newline 
\hspace*{6pt}\hspace*{6pt}\hspace*{6pt}{<\textbf{catDesc}>}Daily{</\textbf{catDesc}>}\mbox{}\newline 
\hspace*{6pt}\hspace*{6pt}{</\textbf{category}>}\mbox{}\newline 
\hspace*{6pt}\hspace*{6pt}{<\textbf{category}\hspace*{6pt}{xml:id}="{tax.b.a2}">}\mbox{}\newline 
\hspace*{6pt}\hspace*{6pt}\hspace*{6pt}{<\textbf{catDesc}>}Sunday{</\textbf{catDesc}>}\mbox{}\newline 
\hspace*{6pt}\hspace*{6pt}{</\textbf{category}>}\mbox{}\newline 
\hspace*{6pt}\hspace*{6pt}{<\textbf{category}\hspace*{6pt}{xml:id}="{tax.b.a3}">}\mbox{}\newline 
\hspace*{6pt}\hspace*{6pt}\hspace*{6pt}{<\textbf{catDesc}>}National{</\textbf{catDesc}>}\mbox{}\newline 
\hspace*{6pt}\hspace*{6pt}{</\textbf{category}>}\mbox{}\newline 
\hspace*{6pt}\hspace*{6pt}{<\textbf{category}\hspace*{6pt}{xml:id}="{tax.b.a4}">}\mbox{}\newline 
\hspace*{6pt}\hspace*{6pt}\hspace*{6pt}{<\textbf{catDesc}>}Provincial{</\textbf{catDesc}>}\mbox{}\newline 
\hspace*{6pt}\hspace*{6pt}{</\textbf{category}>}\mbox{}\newline 
\hspace*{6pt}\hspace*{6pt}{<\textbf{category}\hspace*{6pt}{xml:id}="{tax.b.a5}">}\mbox{}\newline 
\hspace*{6pt}\hspace*{6pt}\hspace*{6pt}{<\textbf{catDesc}>}Political{</\textbf{catDesc}>}\mbox{}\newline 
\hspace*{6pt}\hspace*{6pt}{</\textbf{category}>}\mbox{}\newline 
\hspace*{6pt}\hspace*{6pt}{<\textbf{category}\hspace*{6pt}{xml:id}="{tax.b.a6}">}\mbox{}\newline 
\hspace*{6pt}\hspace*{6pt}\hspace*{6pt}{<\textbf{catDesc}>}Sports{</\textbf{catDesc}>}\mbox{}\newline 
\hspace*{6pt}\hspace*{6pt}{</\textbf{category}>}\mbox{}\newline 
\hspace*{6pt}{</\textbf{category}>}\mbox{}\newline 
\hspace*{6pt}{<\textbf{category}\hspace*{6pt}{xml:id}="{tax.b.d}">}\mbox{}\newline 
\hspace*{6pt}\hspace*{6pt}{<\textbf{catDesc}>}Religion{</\textbf{catDesc}>}\mbox{}\newline 
\hspace*{6pt}\hspace*{6pt}{<\textbf{category}\hspace*{6pt}{xml:id}="{tax.b.d1}">}\mbox{}\newline 
\hspace*{6pt}\hspace*{6pt}\hspace*{6pt}{<\textbf{catDesc}>}Books{</\textbf{catDesc}>}\mbox{}\newline 
\hspace*{6pt}\hspace*{6pt}{</\textbf{category}>}\mbox{}\newline 
\hspace*{6pt}\hspace*{6pt}{<\textbf{category}\hspace*{6pt}{xml:id}="{tax.b.d2}">}\mbox{}\newline 
\hspace*{6pt}\hspace*{6pt}\hspace*{6pt}{<\textbf{catDesc}>}Periodicals and tracts{</\textbf{catDesc}>}\mbox{}\newline 
\hspace*{6pt}\hspace*{6pt}{</\textbf{category}>}\mbox{}\newline 
\hspace*{6pt}{</\textbf{category}>}\mbox{}\newline 
{</\textbf{taxonomy}>}\end{shaded}\egroup 


    \item[{Example}]
  \leavevmode\bgroup\exampleFont \begin{shaded}\noindent\mbox{}{<\textbf{taxonomy}>}\mbox{}\newline 
\hspace*{6pt}{<\textbf{category}\hspace*{6pt}{xml:id}="{literature}">}\mbox{}\newline 
\hspace*{6pt}\hspace*{6pt}{<\textbf{catDesc}>}Literature{</\textbf{catDesc}>}\mbox{}\newline 
\hspace*{6pt}\hspace*{6pt}{<\textbf{category}\hspace*{6pt}{xml:id}="{poetry}">}\mbox{}\newline 
\hspace*{6pt}\hspace*{6pt}\hspace*{6pt}{<\textbf{catDesc}>}Poetry{</\textbf{catDesc}>}\mbox{}\newline 
\hspace*{6pt}\hspace*{6pt}\hspace*{6pt}{<\textbf{category}\hspace*{6pt}{xml:id}="{sonnet}">}\mbox{}\newline 
\hspace*{6pt}\hspace*{6pt}\hspace*{6pt}\hspace*{6pt}{<\textbf{catDesc}>}Sonnet{</\textbf{catDesc}>}\mbox{}\newline 
\hspace*{6pt}\hspace*{6pt}\hspace*{6pt}\hspace*{6pt}{<\textbf{category}\hspace*{6pt}{xml:id}="{shakesSonnet}">}\mbox{}\newline 
\hspace*{6pt}\hspace*{6pt}\hspace*{6pt}\hspace*{6pt}\hspace*{6pt}{<\textbf{catDesc}>}Shakespearean Sonnet{</\textbf{catDesc}>}\mbox{}\newline 
\hspace*{6pt}\hspace*{6pt}\hspace*{6pt}\hspace*{6pt}{</\textbf{category}>}\mbox{}\newline 
\hspace*{6pt}\hspace*{6pt}\hspace*{6pt}\hspace*{6pt}{<\textbf{category}\hspace*{6pt}{xml:id}="{petraSonnet}">}\mbox{}\newline 
\hspace*{6pt}\hspace*{6pt}\hspace*{6pt}\hspace*{6pt}\hspace*{6pt}{<\textbf{catDesc}>}Petrarchan Sonnet{</\textbf{catDesc}>}\mbox{}\newline 
\hspace*{6pt}\hspace*{6pt}\hspace*{6pt}\hspace*{6pt}{</\textbf{category}>}\mbox{}\newline 
\hspace*{6pt}\hspace*{6pt}\hspace*{6pt}{</\textbf{category}>}\mbox{}\newline 
\hspace*{6pt}\hspace*{6pt}\hspace*{6pt}{<\textbf{category}\hspace*{6pt}{xml:id}="{haiku}">}\mbox{}\newline 
\hspace*{6pt}\hspace*{6pt}\hspace*{6pt}\hspace*{6pt}{<\textbf{catDesc}>}Haiku{</\textbf{catDesc}>}\mbox{}\newline 
\hspace*{6pt}\hspace*{6pt}\hspace*{6pt}{</\textbf{category}>}\mbox{}\newline 
\hspace*{6pt}\hspace*{6pt}{</\textbf{category}>}\mbox{}\newline 
\hspace*{6pt}\hspace*{6pt}{<\textbf{category}\hspace*{6pt}{xml:id}="{drama}">}\mbox{}\newline 
\hspace*{6pt}\hspace*{6pt}\hspace*{6pt}{<\textbf{catDesc}>}Drama{</\textbf{catDesc}>}\mbox{}\newline 
\hspace*{6pt}\hspace*{6pt}{</\textbf{category}>}\mbox{}\newline 
\hspace*{6pt}{</\textbf{category}>}\mbox{}\newline 
\hspace*{6pt}{<\textbf{category}\hspace*{6pt}{xml:id}="{meter}">}\mbox{}\newline 
\hspace*{6pt}\hspace*{6pt}{<\textbf{catDesc}>}Metrical Categories{</\textbf{catDesc}>}\mbox{}\newline 
\hspace*{6pt}\hspace*{6pt}{<\textbf{category}\hspace*{6pt}{xml:id}="{feet}">}\mbox{}\newline 
\hspace*{6pt}\hspace*{6pt}\hspace*{6pt}{<\textbf{catDesc}>}Metrical Feet{</\textbf{catDesc}>}\mbox{}\newline 
\hspace*{6pt}\hspace*{6pt}\hspace*{6pt}{<\textbf{category}\hspace*{6pt}{xml:id}="{iambic}">}\mbox{}\newline 
\hspace*{6pt}\hspace*{6pt}\hspace*{6pt}\hspace*{6pt}{<\textbf{catDesc}>}Iambic{</\textbf{catDesc}>}\mbox{}\newline 
\hspace*{6pt}\hspace*{6pt}\hspace*{6pt}{</\textbf{category}>}\mbox{}\newline 
\hspace*{6pt}\hspace*{6pt}\hspace*{6pt}{<\textbf{category}\hspace*{6pt}{xml:id}="{trochaic}">}\mbox{}\newline 
\hspace*{6pt}\hspace*{6pt}\hspace*{6pt}\hspace*{6pt}{<\textbf{catDesc}>}trochaic{</\textbf{catDesc}>}\mbox{}\newline 
\hspace*{6pt}\hspace*{6pt}\hspace*{6pt}{</\textbf{category}>}\mbox{}\newline 
\hspace*{6pt}\hspace*{6pt}{</\textbf{category}>}\mbox{}\newline 
\hspace*{6pt}\hspace*{6pt}{<\textbf{category}\hspace*{6pt}{xml:id}="{feetNumber}">}\mbox{}\newline 
\hspace*{6pt}\hspace*{6pt}\hspace*{6pt}{<\textbf{catDesc}>}Number of feet{</\textbf{catDesc}>}\mbox{}\newline 
\hspace*{6pt}\hspace*{6pt}\hspace*{6pt}{<\textbf{category}\hspace*{6pt}{xml:id}="{pentameter}">}\mbox{}\newline 
\hspace*{6pt}\hspace*{6pt}\hspace*{6pt}\hspace*{6pt}{<\textbf{catDesc}>}>Pentameter{</\textbf{catDesc}>}\mbox{}\newline 
\hspace*{6pt}\hspace*{6pt}\hspace*{6pt}{</\textbf{category}>}\mbox{}\newline 
\hspace*{6pt}\hspace*{6pt}\hspace*{6pt}{<\textbf{category}\hspace*{6pt}{xml:id}="{tetrameter}">}\mbox{}\newline 
\hspace*{6pt}\hspace*{6pt}\hspace*{6pt}\hspace*{6pt}{<\textbf{catDesc}>}>Tetrameter{</\textbf{catDesc}>}\mbox{}\newline 
\hspace*{6pt}\hspace*{6pt}\hspace*{6pt}{</\textbf{category}>}\mbox{}\newline 
\hspace*{6pt}\hspace*{6pt}{</\textbf{category}>}\mbox{}\newline 
\hspace*{6pt}{</\textbf{category}>}\mbox{}\newline 
{</\textbf{taxonomy}>}\mbox{}\newline 
\textit{<!-- elsewhere in document -->}\mbox{}\newline 
{<\textbf{lg}\hspace*{6pt}{ana}="{\#shakesSonnet \#iambic \#pentameter}">}\mbox{}\newline 
\hspace*{6pt}{<\textbf{l}>}Shall I compare thee to a summer's day{</\textbf{l}>}\mbox{}\newline 
\textit{<!-- ... -->}\mbox{}\newline 
{</\textbf{lg}>}\end{shaded}\egroup 


    \item[{Content model}]
  \mbox{}\hfill\\[-10pt]\begin{Verbatim}[fontsize=\small]
<content>
 <alternate>
  <alternate>
   <alternate maxOccurs="unbounded"
    minOccurs="1">
    <elementRef key="category"/>
    <elementRef key="taxonomy"/>
   </alternate>
   <sequence>
    <alternate maxOccurs="unbounded"
     minOccurs="1">
     <classRef key="model.glossLike"/>
     <classRef key="model.descLike"/>
    </alternate>
    <alternate maxOccurs="unbounded"
     minOccurs="0">
     <elementRef key="category"/>
     <elementRef key="taxonomy"/>
    </alternate>
   </sequence>
  </alternate>
  <sequence>
   <classRef key="model.biblLike"/>
   <alternate maxOccurs="unbounded"
    minOccurs="0">
    <elementRef key="category"/>
    <elementRef key="taxonomy"/>
   </alternate>
  </sequence>
 </alternate>
</content>
    
\end{Verbatim}

    \item[{Schema Declaration}]
  \mbox{}\hfill\\[-10pt]\begin{Verbatim}[fontsize=\small]
element taxonomy
{
   att.global.attributes,
   (
      (
         ( category | taxonomy )+
       | ( ( model.glossLike | model.descLike )+, ( category | taxonomy )* )
      )
    | ( model.biblLike, ( category | taxonomy )* )
   )
}
\end{Verbatim}

\end{reflist}  \index{teiCorpus=<teiCorpus>|oddindex}\index{version=@version!<teiCorpus>|oddindex}
\begin{reflist}
\item[]\begin{specHead}{TEI.teiCorpus}{<teiCorpus> }contains the whole of a TEI encoded corpus, comprising a single corpus header and one or more TEI elements, each containing a single text header and a text. [\xref{http://www.tei-c.org/release/doc/tei-p5-doc/en/html/DS.html\#DS}{4. Default Text Structure} \xref{http://www.tei-c.org/release/doc/tei-p5-doc/en/html/CC.html\#CCDEF}{15.1. Varieties of Composite Text}]\end{specHead} 
    \item[{Module}]
  core
    \item[{Attributes}]
  Attributes att.global (\textit{@xml:id}, \textit{@n}, \textit{@xml:lang}, \textit{@xml:base}, \textit{@xml:space})  (att.global.rendition (\textit{@rend}, \textit{@style}, \textit{@rendition})) (att.global.linking (\textit{@corresp}, \textit{@synch}, \textit{@sameAs}, \textit{@copyOf}, \textit{@next}, \textit{@prev}, \textit{@exclude}, \textit{@select})) (att.global.analytic (\textit{@ana})) (att.global.facs (\textit{@facs})) (att.global.change (\textit{@change})) (att.global.responsibility (\textit{@cert}, \textit{@resp})) (att.global.source (\textit{@source})) att.typed (\textit{@type}, \textit{@subtype}) \hfil\\[-10pt]\begin{sansreflist}
    \item[@version]
  The version of the TEI scheme
\begin{reflist}
    \item[{Status}]
  Optional
    \item[{Datatype}]
  teidata.version
    \item[{Default}]
  5.0
\end{reflist}  
\end{sansreflist}  
    \item[{Contained by}]
  
    \item[core: ]
   teiCorpus
    \item[{May contain}]
  
    \item[core: ]
   teiCorpus\par 
    \item[header: ]
   teiHeader\par 
    \item[textstructure: ]
   TEI text\par 
    \item[transcr: ]
   facsimile sourceDoc
    \item[{Note}]
  \par
Must contain one TEI header for the corpus, and a series of <TEI> elements, one for each text.\par
This element is mandatory when applicable.
    \item[{Example}]
  \leavevmode\bgroup\exampleFont \begin{shaded}\noindent\mbox{}{<\textbf{teiCorpus}\hspace*{6pt}{version}="{5.2}" xmlns="http://www.tei-c.org/ns/1.0">}\mbox{}\newline 
\hspace*{6pt}{<\textbf{teiHeader}>}\mbox{}\newline 
\textit{<!-- header for corpus -->}\mbox{}\newline 
\hspace*{6pt}{</\textbf{teiHeader}>}\mbox{}\newline 
\hspace*{6pt}{<\textbf{TEI}>}\mbox{}\newline 
\hspace*{6pt}\hspace*{6pt}{<\textbf{teiHeader}>}\mbox{}\newline 
\textit{<!-- header for first text -->}\mbox{}\newline 
\hspace*{6pt}\hspace*{6pt}{</\textbf{teiHeader}>}\mbox{}\newline 
\hspace*{6pt}\hspace*{6pt}{<\textbf{text}>}\mbox{}\newline 
\textit{<!-- content of first text -->}\mbox{}\newline 
\hspace*{6pt}\hspace*{6pt}{</\textbf{text}>}\mbox{}\newline 
\hspace*{6pt}{</\textbf{TEI}>}\mbox{}\newline 
\hspace*{6pt}{<\textbf{TEI}>}\mbox{}\newline 
\hspace*{6pt}\hspace*{6pt}{<\textbf{teiHeader}>}\mbox{}\newline 
\textit{<!-- header for second text -->}\mbox{}\newline 
\hspace*{6pt}\hspace*{6pt}{</\textbf{teiHeader}>}\mbox{}\newline 
\hspace*{6pt}\hspace*{6pt}{<\textbf{text}>}\mbox{}\newline 
\textit{<!-- content of second text -->}\mbox{}\newline 
\hspace*{6pt}\hspace*{6pt}{</\textbf{text}>}\mbox{}\newline 
\hspace*{6pt}{</\textbf{TEI}>}\mbox{}\newline 
\textit{<!-- more TEI elements here -->}\mbox{}\newline 
{</\textbf{teiCorpus}>}\end{shaded}\egroup 


    \item[{Content model}]
  \mbox{}\hfill\\[-10pt]\begin{Verbatim}[fontsize=\small]
<content>
 <sequence>
  <elementRef key="teiHeader"/>
  <alternate>
   <sequence>
    <classRef key="model.resourceLike"
     maxOccurs="unbounded" minOccurs="1"/>
    <alternate maxOccurs="unbounded"
     minOccurs="0">
     <elementRef key="TEI"/>
     <elementRef key="teiCorpus"/>
    </alternate>
   </sequence>
   <alternate maxOccurs="unbounded"
    minOccurs="1">
    <elementRef key="TEI"/>
    <elementRef key="teiCorpus"/>
   </alternate>
  </alternate>
 </sequence>
</content>
    
\end{Verbatim}

    \item[{Schema Declaration}]
  \mbox{}\hfill\\[-10pt]\begin{Verbatim}[fontsize=\small]
element teiCorpus
{
   att.global.attributes,
   att.typed.attributes,
   attribute version { text }?,
   (
      teiHeader,
      ( ( model.resourceLike+, ( TEI | teiCorpus )* ) | ( TEI | teiCorpus )+ )
   )
}
\end{Verbatim}

\end{reflist}  \index{teiHeader=<teiHeader>|oddindex}
\begin{reflist}
\item[]\begin{specHead}{TEI.teiHeader}{<teiHeader> }(TEI header) supplies descriptive and declarative metadata associated with a digital resource or set of resources. [\xref{http://www.tei-c.org/release/doc/tei-p5-doc/en/html/HD.html\#HD11}{2.1.1. The TEI Header and Its Components} \xref{http://www.tei-c.org/release/doc/tei-p5-doc/en/html/CC.html\#CCDEF}{15.1. Varieties of Composite Text}]\end{specHead} 
    \item[{Module}]
  header
    \item[{Attributes}]
  Attributes att.global (\textit{@xml:id}, \textit{@n}, \textit{@xml:lang}, \textit{@xml:base}, \textit{@xml:space})  (att.global.rendition (\textit{@rend}, \textit{@style}, \textit{@rendition})) (att.global.linking (\textit{@corresp}, \textit{@synch}, \textit{@sameAs}, \textit{@copyOf}, \textit{@next}, \textit{@prev}, \textit{@exclude}, \textit{@select})) (att.global.analytic (\textit{@ana})) (att.global.facs (\textit{@facs})) (att.global.change (\textit{@change})) (att.global.responsibility (\textit{@cert}, \textit{@resp})) (att.global.source (\textit{@source}))
    \item[{Contained by}]
  
    \item[core: ]
   teiCorpus\par 
    \item[textstructure: ]
   TEI
    \item[{May contain}]
  
    \item[header: ]
   encodingDesc fileDesc profileDesc revisionDesc xenoData
    \item[{Note}]
  \par
One of the few elements unconditionally required in any TEI document.
    \item[{Example}]
  \leavevmode\bgroup\exampleFont \begin{shaded}\noindent\mbox{}{<\textbf{teiHeader}>}\mbox{}\newline 
\hspace*{6pt}{<\textbf{fileDesc}>}\mbox{}\newline 
\hspace*{6pt}\hspace*{6pt}{<\textbf{titleStmt}>}\mbox{}\newline 
\hspace*{6pt}\hspace*{6pt}\hspace*{6pt}{<\textbf{title}>}Shakespeare: the first folio (1623) in electronic form{</\textbf{title}>}\mbox{}\newline 
\hspace*{6pt}\hspace*{6pt}\hspace*{6pt}{<\textbf{author}>}Shakespeare, William (1564–1616){</\textbf{author}>}\mbox{}\newline 
\hspace*{6pt}\hspace*{6pt}\hspace*{6pt}{<\textbf{respStmt}>}\mbox{}\newline 
\hspace*{6pt}\hspace*{6pt}\hspace*{6pt}\hspace*{6pt}{<\textbf{resp}>}Originally prepared by{</\textbf{resp}>}\mbox{}\newline 
\hspace*{6pt}\hspace*{6pt}\hspace*{6pt}\hspace*{6pt}{<\textbf{name}>}Trevor Howard-Hill{</\textbf{name}>}\mbox{}\newline 
\hspace*{6pt}\hspace*{6pt}\hspace*{6pt}{</\textbf{respStmt}>}\mbox{}\newline 
\hspace*{6pt}\hspace*{6pt}\hspace*{6pt}{<\textbf{respStmt}>}\mbox{}\newline 
\hspace*{6pt}\hspace*{6pt}\hspace*{6pt}\hspace*{6pt}{<\textbf{resp}>}Revised and edited by{</\textbf{resp}>}\mbox{}\newline 
\hspace*{6pt}\hspace*{6pt}\hspace*{6pt}\hspace*{6pt}{<\textbf{name}>}Christine Avern-Carr{</\textbf{name}>}\mbox{}\newline 
\hspace*{6pt}\hspace*{6pt}\hspace*{6pt}{</\textbf{respStmt}>}\mbox{}\newline 
\hspace*{6pt}\hspace*{6pt}{</\textbf{titleStmt}>}\mbox{}\newline 
\hspace*{6pt}\hspace*{6pt}{<\textbf{publicationStmt}>}\mbox{}\newline 
\hspace*{6pt}\hspace*{6pt}\hspace*{6pt}{<\textbf{distributor}>}Oxford Text Archive{</\textbf{distributor}>}\mbox{}\newline 
\hspace*{6pt}\hspace*{6pt}\hspace*{6pt}{<\textbf{address}>}\mbox{}\newline 
\hspace*{6pt}\hspace*{6pt}\hspace*{6pt}\hspace*{6pt}{<\textbf{addrLine}>}13 Banbury Road, Oxford OX2 6NN, UK{</\textbf{addrLine}>}\mbox{}\newline 
\hspace*{6pt}\hspace*{6pt}\hspace*{6pt}{</\textbf{address}>}\mbox{}\newline 
\hspace*{6pt}\hspace*{6pt}\hspace*{6pt}{<\textbf{idno}\hspace*{6pt}{type}="{OTA}">}119{</\textbf{idno}>}\mbox{}\newline 
\hspace*{6pt}\hspace*{6pt}\hspace*{6pt}{<\textbf{availability}>}\mbox{}\newline 
\hspace*{6pt}\hspace*{6pt}\hspace*{6pt}\hspace*{6pt}{<\textbf{p}>}Freely available on a non-commercial basis.{</\textbf{p}>}\mbox{}\newline 
\hspace*{6pt}\hspace*{6pt}\hspace*{6pt}{</\textbf{availability}>}\mbox{}\newline 
\hspace*{6pt}\hspace*{6pt}\hspace*{6pt}{<\textbf{date}\hspace*{6pt}{when}="{1968}">}1968{</\textbf{date}>}\mbox{}\newline 
\hspace*{6pt}\hspace*{6pt}{</\textbf{publicationStmt}>}\mbox{}\newline 
\hspace*{6pt}\hspace*{6pt}{<\textbf{sourceDesc}>}\mbox{}\newline 
\hspace*{6pt}\hspace*{6pt}\hspace*{6pt}{<\textbf{bibl}>}The first folio of Shakespeare, prepared by Charlton Hinman (The Norton Facsimile,\mbox{}\newline 
\hspace*{6pt}\hspace*{6pt}\hspace*{6pt}\hspace*{6pt}\hspace*{6pt}\hspace*{6pt} 1968){</\textbf{bibl}>}\mbox{}\newline 
\hspace*{6pt}\hspace*{6pt}{</\textbf{sourceDesc}>}\mbox{}\newline 
\hspace*{6pt}{</\textbf{fileDesc}>}\mbox{}\newline 
\hspace*{6pt}{<\textbf{encodingDesc}>}\mbox{}\newline 
\hspace*{6pt}\hspace*{6pt}{<\textbf{projectDesc}>}\mbox{}\newline 
\hspace*{6pt}\hspace*{6pt}\hspace*{6pt}{<\textbf{p}>}Originally prepared for use in the production of a series of old-spelling\mbox{}\newline 
\hspace*{6pt}\hspace*{6pt}\hspace*{6pt}\hspace*{6pt}\hspace*{6pt}\hspace*{6pt} concordances in 1968, this text was extensively checked and revised for use during the\mbox{}\newline 
\hspace*{6pt}\hspace*{6pt}\hspace*{6pt}\hspace*{6pt}\hspace*{6pt}\hspace*{6pt} editing of the new Oxford Shakespeare (Wells and Taylor, 1989).{</\textbf{p}>}\mbox{}\newline 
\hspace*{6pt}\hspace*{6pt}{</\textbf{projectDesc}>}\mbox{}\newline 
\hspace*{6pt}\hspace*{6pt}{<\textbf{editorialDecl}>}\mbox{}\newline 
\hspace*{6pt}\hspace*{6pt}\hspace*{6pt}{<\textbf{correction}>}\mbox{}\newline 
\hspace*{6pt}\hspace*{6pt}\hspace*{6pt}\hspace*{6pt}{<\textbf{p}>}Turned letters are silently corrected.{</\textbf{p}>}\mbox{}\newline 
\hspace*{6pt}\hspace*{6pt}\hspace*{6pt}{</\textbf{correction}>}\mbox{}\newline 
\hspace*{6pt}\hspace*{6pt}\hspace*{6pt}{<\textbf{normalization}>}\mbox{}\newline 
\hspace*{6pt}\hspace*{6pt}\hspace*{6pt}\hspace*{6pt}{<\textbf{p}>}Original spelling and typography is retained, except that long s and ligatured\mbox{}\newline 
\hspace*{6pt}\hspace*{6pt}\hspace*{6pt}\hspace*{6pt}\hspace*{6pt}\hspace*{6pt}\hspace*{6pt}\hspace*{6pt} forms are not encoded.{</\textbf{p}>}\mbox{}\newline 
\hspace*{6pt}\hspace*{6pt}\hspace*{6pt}{</\textbf{normalization}>}\mbox{}\newline 
\hspace*{6pt}\hspace*{6pt}{</\textbf{editorialDecl}>}\mbox{}\newline 
\hspace*{6pt}\hspace*{6pt}{<\textbf{refsDecl}\hspace*{6pt}{xml:id}="{ASLREF}">}\mbox{}\newline 
\hspace*{6pt}\hspace*{6pt}\hspace*{6pt}{<\textbf{cRefPattern}\hspace*{6pt}{matchPattern}="{(⃥S+) ([\textasciicircum .]+)⃥.(.*)}"\mbox{}\newline 
\hspace*{6pt}\hspace*{6pt}\hspace*{6pt}\hspace*{6pt}{replacementPattern}="{\#xpath(//div1[@n='\$1']/div2/[@n='\$2']//lb[@n='\$3'])}">}\mbox{}\newline 
\hspace*{6pt}\hspace*{6pt}\hspace*{6pt}\hspace*{6pt}{<\textbf{p}>}A reference is created by assembling the following, in the reverse order as that\mbox{}\newline 
\hspace*{6pt}\hspace*{6pt}\hspace*{6pt}\hspace*{6pt}\hspace*{6pt}\hspace*{6pt}\hspace*{6pt}\hspace*{6pt} listed here: {<\textbf{list}>}\mbox{}\newline 
\hspace*{6pt}\hspace*{6pt}\hspace*{6pt}\hspace*{6pt}\hspace*{6pt}\hspace*{6pt}{<\textbf{item}>}the {<\textbf{att}>}n{</\textbf{att}>} value of the preceding {<\textbf{gi}>}lb{</\textbf{gi}>}\mbox{}\newline 
\hspace*{6pt}\hspace*{6pt}\hspace*{6pt}\hspace*{6pt}\hspace*{6pt}\hspace*{6pt}{</\textbf{item}>}\mbox{}\newline 
\hspace*{6pt}\hspace*{6pt}\hspace*{6pt}\hspace*{6pt}\hspace*{6pt}\hspace*{6pt}{<\textbf{item}>}a period{</\textbf{item}>}\mbox{}\newline 
\hspace*{6pt}\hspace*{6pt}\hspace*{6pt}\hspace*{6pt}\hspace*{6pt}\hspace*{6pt}{<\textbf{item}>}the {<\textbf{att}>}n{</\textbf{att}>} value of the ancestor {<\textbf{gi}>}div2{</\textbf{gi}>}\mbox{}\newline 
\hspace*{6pt}\hspace*{6pt}\hspace*{6pt}\hspace*{6pt}\hspace*{6pt}\hspace*{6pt}{</\textbf{item}>}\mbox{}\newline 
\hspace*{6pt}\hspace*{6pt}\hspace*{6pt}\hspace*{6pt}\hspace*{6pt}\hspace*{6pt}{<\textbf{item}>}a space{</\textbf{item}>}\mbox{}\newline 
\hspace*{6pt}\hspace*{6pt}\hspace*{6pt}\hspace*{6pt}\hspace*{6pt}\hspace*{6pt}{<\textbf{item}>}the {<\textbf{att}>}n{</\textbf{att}>} value of the parent {<\textbf{gi}>}div1{</\textbf{gi}>}\mbox{}\newline 
\hspace*{6pt}\hspace*{6pt}\hspace*{6pt}\hspace*{6pt}\hspace*{6pt}\hspace*{6pt}{</\textbf{item}>}\mbox{}\newline 
\hspace*{6pt}\hspace*{6pt}\hspace*{6pt}\hspace*{6pt}\hspace*{6pt}{</\textbf{list}>}\mbox{}\newline 
\hspace*{6pt}\hspace*{6pt}\hspace*{6pt}\hspace*{6pt}{</\textbf{p}>}\mbox{}\newline 
\hspace*{6pt}\hspace*{6pt}\hspace*{6pt}{</\textbf{cRefPattern}>}\mbox{}\newline 
\hspace*{6pt}\hspace*{6pt}{</\textbf{refsDecl}>}\mbox{}\newline 
\hspace*{6pt}{</\textbf{encodingDesc}>}\mbox{}\newline 
\hspace*{6pt}{<\textbf{revisionDesc}>}\mbox{}\newline 
\hspace*{6pt}\hspace*{6pt}{<\textbf{list}>}\mbox{}\newline 
\hspace*{6pt}\hspace*{6pt}\hspace*{6pt}{<\textbf{item}>}\mbox{}\newline 
\hspace*{6pt}\hspace*{6pt}\hspace*{6pt}\hspace*{6pt}{<\textbf{date}\hspace*{6pt}{when}="{1989-04-12}">}12 Apr 89{</\textbf{date}>} Last checked by CAC{</\textbf{item}>}\mbox{}\newline 
\hspace*{6pt}\hspace*{6pt}\hspace*{6pt}{<\textbf{item}>}\mbox{}\newline 
\hspace*{6pt}\hspace*{6pt}\hspace*{6pt}\hspace*{6pt}{<\textbf{date}\hspace*{6pt}{when}="{1989-03-01}">}1 Mar 89{</\textbf{date}>} LB made new file{</\textbf{item}>}\mbox{}\newline 
\hspace*{6pt}\hspace*{6pt}{</\textbf{list}>}\mbox{}\newline 
\hspace*{6pt}{</\textbf{revisionDesc}>}\mbox{}\newline 
{</\textbf{teiHeader}>}\end{shaded}\egroup 


    \item[{Content model}]
  \mbox{}\hfill\\[-10pt]\begin{Verbatim}[fontsize=\small]
<content>
 <sequence>
  <elementRef key="fileDesc"/>
  <classRef key="model.teiHeaderPart"
   maxOccurs="unbounded" minOccurs="0"/>
  <elementRef key="revisionDesc"
   minOccurs="0"/>
 </sequence>
</content>
    
\end{Verbatim}

    \item[{Schema Declaration}]
  \mbox{}\hfill\\[-10pt]\begin{Verbatim}[fontsize=\small]
element teiHeader
{
   att.global.attributes,
   ( fileDesc, model.teiHeaderPart*, revisionDesc? )
}
\end{Verbatim}

\end{reflist}  \index{term=<term>|oddindex}
\begin{reflist}
\item[]\begin{specHead}{TEI.term}{<term> }contains a single-word, multi-word, or symbolic designation which is regarded as a technical term. [\xref{http://www.tei-c.org/release/doc/tei-p5-doc/en/html/CO.html\#COHQU}{3.3.4. Terms, Glosses, Equivalents, and Descriptions}]\end{specHead} 
    \item[{Module}]
  core
    \item[{Attributes}]
  Attributes att.global (\textit{@xml:id}, \textit{@n}, \textit{@xml:lang}, \textit{@xml:base}, \textit{@xml:space})  (att.global.rendition (\textit{@rend}, \textit{@style}, \textit{@rendition})) (att.global.linking (\textit{@corresp}, \textit{@synch}, \textit{@sameAs}, \textit{@copyOf}, \textit{@next}, \textit{@prev}, \textit{@exclude}, \textit{@select})) (att.global.analytic (\textit{@ana})) (att.global.facs (\textit{@facs})) (att.global.change (\textit{@change})) (att.global.responsibility (\textit{@cert}, \textit{@resp})) (att.global.source (\textit{@source})) att.declaring (\textit{@decls}) att.pointing (\textit{@targetLang}, \textit{@target}, \textit{@evaluate}) att.typed (\textit{@type}, \textit{@subtype}) att.canonical (\textit{@key}, \textit{@ref}) att.sortable (\textit{@sortKey}) att.cReferencing (\textit{@cRef}) 
    \item[{Member of}]
  model.emphLike
    \item[{Contained by}]
  
    \item[analysis: ]
   cl phr s span\par 
    \item[core: ]
   abbr add addrLine author bibl biblScope citedRange corr date del desc distinct editor email emph expan foreign gloss head headItem headLabel hi index item l label measure meeting mentioned name note num orig p pubPlace publisher q quote ref reg resp rs said sic soCalled speaker stage street term textLang time title unclear\par 
    \item[figures: ]
   cell figDesc\par 
    \item[header: ]
   authority catDesc change classCode creation distributor edition extent funder geoDecl handNote keywords language licence principal rendition scriptNote sponsor tagUsage typeNote\par 
    \item[linking: ]
   ab seg\par 
    \item[msdescription: ]
   accMat acquisition additions catchwords collation colophon condition custEvent decoNote explicit filiation finalRubric foliation heraldry incipit layout material musicNotation objectType origDate origPlace origin provenance rubric secFol signatures source stamp summary support surrogates watermark\par 
    \item[namesdates: ]
   addName affiliation age birth bloc country death district education faith floruit forename genName geogFeat geogName langKnown nameLink nationality occupation offset orgName persName placeName region residence roleName settlement sex socecStatus surname\par 
    \item[textcrit: ]
   lem rdg wit witDetail witness\par 
    \item[textstructure: ]
   byline closer dateline docAuthor docDate docEdition docImprint imprimatur opener salute signed titlePart trailer\par 
    \item[transcr: ]
   damage fw metamark mod restore retrace secl supplied surplus
    \item[{May contain}]
  
    \item[analysis: ]
   c cl interp interpGrp m pc phr s span spanGrp w\par 
    \item[core: ]
   abbr add address cb choice corr date del distinct email emph expan foreign gap gb gloss graphic hi index lb measure measureGrp media mentioned milestone name note num orig pb ptr ref reg rs sic soCalled term time title unclear\par 
    \item[figures: ]
   figure formula notatedMusic\par 
    \item[gaiji: ]
   g\par 
    \item[header: ]
   idno\par 
    \item[linking: ]
   alt altGrp anchor join joinGrp link linkGrp seg timeline\par 
    \item[msdescription: ]
   catchwords depth dim dimensions height heraldry locus locusGrp material objectType origDate origPlace secFol signatures stamp watermark width\par 
    \item[namesdates: ]
   addName affiliation bloc climate country district forename genName geo geogFeat geogName location nameLink offset orgName persName placeName population region roleName settlement state surname terrain trait\par 
    \item[textcrit: ]
   app witDetail\par 
    \item[transcr: ]
   addSpan am damage damageSpan delSpan ex fw handShift listTranspose metamark mod redo restore retrace secl space subst substJoin supplied surplus undo\par character data
    \item[{Note}]
  \par
When this element appears within an <index> element, it is understood to supply the form under which an index entry is to be made for that location. Elsewhere, it is understood simply to indicate that its content is to be regarded as a technical or specialised term. It may be associated with a <gloss> element by means of its {\itshape ref} attribute; alternatively a <gloss> element may point to a <term> element by means of its {\itshape target} attribute.\par
In formal terminological work, there is frequently discussion over whether terms must be atomic or may include multi-word lexical items, symbolic designations, or phraseological units. The <term> element may be used to mark any of these. No position is taken on the philosophical issue of what a term can be; the looser definition simply allows the <term> element to be used by practitioners of any persuasion.\par
As with other members of the \textsf{att.canonical} class, instances of this element occuring in a text may be associated with a canonical definition, either by means of a URI (using the {\itshape ref} attribute), or by means of some system-specific code value (using the {\itshape key} attribute). Because the mutually exclusive {\itshape target} and {\itshape cRef} attributes overlap with the function of the {\itshape ref} attribute, they are deprecated and may be removed at a subsequent release.
    \item[{Example}]
  \leavevmode\bgroup\exampleFont \begin{shaded}\noindent\mbox{}A computational device that infers structure\mbox{}\newline 
 from grammatical strings of words is known as a {<\textbf{term}>}parser{</\textbf{term}>}, and much of the history\mbox{}\newline 
 of NLP over the last 20 years has been occupied with the design of parsers.\end{shaded}\egroup 


    \item[{Example}]
  \leavevmode\bgroup\exampleFont \begin{shaded}\noindent\mbox{}We may define {<\textbf{term}\hspace*{6pt}{rend}="{sc}"\hspace*{6pt}{xml:id}="{TDPV1}">}discoursal point of view{</\textbf{term}>} as \mbox{}\newline 
{<\textbf{gloss}\hspace*{6pt}{target}="{\#TDPV1}">}the relationship, expressed\mbox{}\newline 
 through discourse structure, between the implied author or some other addresser, and the\mbox{}\newline 
 fiction.{</\textbf{gloss}>}\end{shaded}\egroup 


    \item[{Example}]
  \leavevmode\bgroup\exampleFont \begin{shaded}\noindent\mbox{}We may define {<\textbf{term}\hspace*{6pt}{ref}="{\#TDPV2}"\hspace*{6pt}{rend}="{sc}">}discoursal point of view{</\textbf{term}>} as \mbox{}\newline 
{<\textbf{gloss}\hspace*{6pt}{xml:id}="{TDPV2}">}the relationship, expressed\mbox{}\newline 
 through discourse structure, between the implied author or some other addresser, and the\mbox{}\newline 
 fiction.{</\textbf{gloss}>}\end{shaded}\egroup 


    \item[{Example}]
  \leavevmode\bgroup\exampleFont \begin{shaded}\noindent\mbox{}We discuss Leech's concept of {<\textbf{term}\hspace*{6pt}{ref}="{myGlossary.xml\#TDPV2}"\hspace*{6pt}{rend}="{sc}">}discoursal point of view{</\textbf{term}>} below. \end{shaded}\egroup 


    \item[{Content model}]
  \mbox{}\hfill\\[-10pt]\begin{Verbatim}[fontsize=\small]
<content>
 <macroRef key="macro.phraseSeq"/>
</content>
    
\end{Verbatim}

    \item[{Schema Declaration}]
  \mbox{}\hfill\\[-10pt]\begin{Verbatim}[fontsize=\small]
element term
{
   att.global.attributes,
   att.declaring.attributes,
   att.pointing.attributes,
   att.typed.attributes,
   att.canonical.attributes,
   att.sortable.attributes,
   att.cReferencing.attributes,
   macro.phraseSeq}
\end{Verbatim}

\end{reflist}  \index{terrain=<terrain>|oddindex}
\begin{reflist}
\item[]\begin{specHead}{TEI.terrain}{<terrain> }contains information about the physical terrain of a place. [\xref{http://www.tei-c.org/release/doc/tei-p5-doc/en/html/ND.html\#NDGEOGste}{13.3.4.3. States, Traits, and Events}]\end{specHead} 
    \item[{Module}]
  namesdates
    \item[{Attributes}]
  Attributes att.global (\textit{@xml:id}, \textit{@n}, \textit{@xml:lang}, \textit{@xml:base}, \textit{@xml:space})  (att.global.rendition (\textit{@rend}, \textit{@style}, \textit{@rendition})) (att.global.linking (\textit{@corresp}, \textit{@synch}, \textit{@sameAs}, \textit{@copyOf}, \textit{@next}, \textit{@prev}, \textit{@exclude}, \textit{@select})) (att.global.analytic (\textit{@ana})) (att.global.facs (\textit{@facs})) (att.global.change (\textit{@change})) (att.global.responsibility (\textit{@cert}, \textit{@resp})) (att.global.source (\textit{@source})) att.datable (\textit{@calendar}, \textit{@period})  (att.datable.w3c (\textit{@when}, \textit{@notBefore}, \textit{@notAfter}, \textit{@from}, \textit{@to})) (att.datable.iso (\textit{@when-iso}, \textit{@notBefore-iso}, \textit{@notAfter-iso}, \textit{@from-iso}, \textit{@to-iso})) (att.datable.custom (\textit{@when-custom}, \textit{@notBefore-custom}, \textit{@notAfter-custom}, \textit{@from-custom}, \textit{@to-custom}, \textit{@datingPoint}, \textit{@datingMethod})) att.editLike (\textit{@evidence}, \textit{@instant})  (att.dimensions (\textit{@unit}, \textit{@quantity}, \textit{@extent}, \textit{@precision}, \textit{@scope}) (att.ranging (\textit{@atLeast}, \textit{@atMost}, \textit{@min}, \textit{@max}, \textit{@confidence})) ) att.naming (\textit{@role}, \textit{@nymRef})  (att.canonical (\textit{@key}, \textit{@ref})) att.typed (\textit{@type}, \textit{@subtype}) 
    \item[{Member of}]
  model.placeStateLike 
    \item[{Contained by}]
  
    \item[analysis: ]
   cl phr s span\par 
    \item[core: ]
   abbr add addrLine address author bibl biblScope citedRange corr date del desc distinct editor email emph expan foreign gloss head headItem headLabel hi item l label measure meeting mentioned name note num orig p pubPlace publisher q quote ref reg resp rs said sic soCalled speaker stage street term textLang time title unclear\par 
    \item[figures: ]
   cell figDesc\par 
    \item[header: ]
   authority catDesc change classCode correspAction creation distributor edition extent funder geoDecl handNote language licence principal rendition scriptNote sponsor tagUsage typeNote\par 
    \item[linking: ]
   ab seg\par 
    \item[msdescription: ]
   accMat acquisition additions catchwords collation colophon condition custEvent decoNote explicit filiation finalRubric foliation heraldry incipit layout material musicNotation objectType origDate origPlace origin provenance rubric secFol signatures source stamp summary support surrogates watermark\par 
    \item[namesdates: ]
   addName affiliation age birth bloc country death district education faith floruit forename genName geogFeat geogName langKnown nameLink nationality occupation offset org orgName persName place placeName region residence roleName settlement sex socecStatus surname terrain\par 
    \item[textcrit: ]
   lem rdg wit witDetail witness\par 
    \item[textstructure: ]
   byline closer dateline docAuthor docDate docEdition docImprint imprimatur opener salute signed titlePart trailer\par 
    \item[transcr: ]
   damage fw metamark mod restore retrace secl supplied surplus
    \item[{May contain}]
  
    \item[core: ]
   bibl biblStruct desc head label listBibl note p\par 
    \item[header: ]
   biblFull\par 
    \item[linking: ]
   ab\par 
    \item[msdescription: ]
   msDesc\par 
    \item[namesdates: ]
   terrain\par 
    \item[textcrit: ]
   witDetail
    \item[{Example}]
  \leavevmode\bgroup\exampleFont \begin{shaded}\noindent\mbox{}{<\textbf{place}\hspace*{6pt}{xml:id}="{KERG}">}\mbox{}\newline 
\hspace*{6pt}{<\textbf{placeName}>}Kerguelen Islands{</\textbf{placeName}>}\mbox{}\newline 
\textit{<!-- ... -->}\mbox{}\newline 
\hspace*{6pt}{<\textbf{terrain}>}\mbox{}\newline 
\hspace*{6pt}\hspace*{6pt}{<\textbf{desc}>}antarctic tundra{</\textbf{desc}>}\mbox{}\newline 
\hspace*{6pt}{</\textbf{terrain}>}\mbox{}\newline 
\textit{<!-- ... -->}\mbox{}\newline 
{</\textbf{place}>}\end{shaded}\egroup 


    \item[{Content model}]
  \mbox{}\hfill\\[-10pt]\begin{Verbatim}[fontsize=\small]
<content>
 <sequence>
  <elementRef key="precision"
   maxOccurs="unbounded" minOccurs="0"/>
  <classRef key="model.headLike"
   maxOccurs="unbounded" minOccurs="0"/>
  <alternate>
   <classRef key="model.pLike"
    maxOccurs="unbounded" minOccurs="1"/>
   <classRef key="model.labelLike"
    maxOccurs="unbounded" minOccurs="1"/>
  </alternate>
  <alternate maxOccurs="unbounded"
   minOccurs="0">
   <classRef key="model.noteLike"/>
   <classRef key="model.biblLike"/>
  </alternate>
  <elementRef key="terrain"
   maxOccurs="unbounded" minOccurs="0"/>
 </sequence>
</content>
    
\end{Verbatim}

    \item[{Schema Declaration}]
  \mbox{}\hfill\\[-10pt]\begin{Verbatim}[fontsize=\small]
element terrain
{
   att.global.attributes,
   att.datable.attributes,
   att.editLike.attributes,
   att.naming.attributes,
   att.typed.attributes,
   (
      precision*,
      model.headLike*,
      ( model.pLike+ | model.labelLike+ ),
      ( model.noteLike | model.biblLike )*,
      terrain*
   )
}
\end{Verbatim}

\end{reflist}  \index{text=<text>|oddindex}
\begin{reflist}
\item[]\begin{specHead}{TEI.text}{<text> }contains a single text of any kind, whether unitary or composite, for example a poem or drama, a collection of essays, a novel, a dictionary, or a corpus sample. [\xref{http://www.tei-c.org/release/doc/tei-p5-doc/en/html/DS.html\#DS}{4. Default Text Structure} \xref{http://www.tei-c.org/release/doc/tei-p5-doc/en/html/CC.html\#CCDEF}{15.1. Varieties of Composite Text}]\end{specHead} 
    \item[{Module}]
  textstructure
    \item[{Attributes}]
  Attributes att.global (\textit{@xml:id}, \textit{@n}, \textit{@xml:lang}, \textit{@xml:base}, \textit{@xml:space})  (att.global.rendition (\textit{@rend}, \textit{@style}, \textit{@rendition})) (att.global.linking (\textit{@corresp}, \textit{@synch}, \textit{@sameAs}, \textit{@copyOf}, \textit{@next}, \textit{@prev}, \textit{@exclude}, \textit{@select})) (att.global.analytic (\textit{@ana})) (att.global.facs (\textit{@facs})) (att.global.change (\textit{@change})) (att.global.responsibility (\textit{@cert}, \textit{@resp})) (att.global.source (\textit{@source})) att.declaring (\textit{@decls}) att.typed (\textit{@type}, \textit{@subtype}) att.written (\textit{@hand}) 
    \item[{Member of}]
  model.resourceLike
    \item[{Contained by}]
  
    \item[core: ]
   teiCorpus\par 
    \item[textstructure: ]
   TEI group
    \item[{May contain}]
  
    \item[analysis: ]
   interp interpGrp span spanGrp\par 
    \item[core: ]
   cb gap gb index lb milestone note pb\par 
    \item[figures: ]
   figure notatedMusic\par 
    \item[linking: ]
   alt altGrp anchor join joinGrp link linkGrp timeline\par 
    \item[textcrit: ]
   app witDetail\par 
    \item[textstructure: ]
   back body front group\par 
    \item[transcr: ]
   addSpan damageSpan delSpan fw listTranspose metamark space substJoin
    \item[{Note}]
  \par
This element should not be used to represent a text which is inserted at an arbitrary point within the structure of another, for example as in an embedded or quoted narrative; the <floatingText> is provided for this purpose.
    \item[{Example}]
  \leavevmode\bgroup\exampleFont \begin{shaded}\noindent\mbox{}{<\textbf{text}>}\mbox{}\newline 
\hspace*{6pt}{<\textbf{front}>}\mbox{}\newline 
\hspace*{6pt}\hspace*{6pt}{<\textbf{docTitle}>}\mbox{}\newline 
\hspace*{6pt}\hspace*{6pt}\hspace*{6pt}{<\textbf{titlePart}>}Autumn Haze{</\textbf{titlePart}>}\mbox{}\newline 
\hspace*{6pt}\hspace*{6pt}{</\textbf{docTitle}>}\mbox{}\newline 
\hspace*{6pt}{</\textbf{front}>}\mbox{}\newline 
\hspace*{6pt}{<\textbf{body}>}\mbox{}\newline 
\hspace*{6pt}\hspace*{6pt}{<\textbf{l}>}Is it a dragonfly or a maple leaf{</\textbf{l}>}\mbox{}\newline 
\hspace*{6pt}\hspace*{6pt}{<\textbf{l}>}That settles softly down upon the water?{</\textbf{l}>}\mbox{}\newline 
\hspace*{6pt}{</\textbf{body}>}\mbox{}\newline 
{</\textbf{text}>}\end{shaded}\egroup 


    \item[{Example}]
  The body of a text may be replaced by a group of nested texts, as in the following schematic:\leavevmode\bgroup\exampleFont \begin{shaded}\noindent\mbox{}{<\textbf{text}>}\mbox{}\newline 
\hspace*{6pt}{<\textbf{front}>}\mbox{}\newline 
\textit{<!-- front matter for the whole group -->}\mbox{}\newline 
\hspace*{6pt}{</\textbf{front}>}\mbox{}\newline 
\hspace*{6pt}{<\textbf{group}>}\mbox{}\newline 
\hspace*{6pt}\hspace*{6pt}{<\textbf{text}>}\mbox{}\newline 
\textit{<!-- first text -->}\mbox{}\newline 
\hspace*{6pt}\hspace*{6pt}{</\textbf{text}>}\mbox{}\newline 
\hspace*{6pt}\hspace*{6pt}{<\textbf{text}>}\mbox{}\newline 
\textit{<!-- second text -->}\mbox{}\newline 
\hspace*{6pt}\hspace*{6pt}{</\textbf{text}>}\mbox{}\newline 
\hspace*{6pt}{</\textbf{group}>}\mbox{}\newline 
{</\textbf{text}>}\end{shaded}\egroup 


    \item[{Content model}]
  \mbox{}\hfill\\[-10pt]\begin{Verbatim}[fontsize=\small]
<content>
 <sequence>
  <classRef key="model.global"
   maxOccurs="unbounded" minOccurs="0"/>
  <sequence minOccurs="0">
   <elementRef key="front"/>
   <classRef key="model.global"
    maxOccurs="unbounded" minOccurs="0"/>
  </sequence>
  <alternate>
   <elementRef key="body"/>
   <elementRef key="group"/>
  </alternate>
  <classRef key="model.global"
   maxOccurs="unbounded" minOccurs="0"/>
  <sequence minOccurs="0">
   <elementRef key="back"/>
   <classRef key="model.global"
    maxOccurs="unbounded" minOccurs="0"/>
  </sequence>
 </sequence>
</content>
    
\end{Verbatim}

    \item[{Schema Declaration}]
  \mbox{}\hfill\\[-10pt]\begin{Verbatim}[fontsize=\small]
element text
{
   att.global.attributes,
   att.declaring.attributes,
   att.typed.attributes,
   att.written.attributes,
   (
      model.global*,
      ( front, model.global* )?,
      ( body | group ),
      model.global*,
      ( back, model.global* )?
   )
}
\end{Verbatim}

\end{reflist}  \index{textClass=<textClass>|oddindex}
\begin{reflist}
\item[]\begin{specHead}{TEI.textClass}{<textClass> }(text classification) groups information which describes the nature or topic of a text in terms of a standard classification scheme, thesaurus, etc. [\xref{http://www.tei-c.org/release/doc/tei-p5-doc/en/html/HD.html\#HD43}{2.4.3. The Text Classification}]\end{specHead} 
    \item[{Module}]
  header
    \item[{Attributes}]
  Attributes att.global (\textit{@xml:id}, \textit{@n}, \textit{@xml:lang}, \textit{@xml:base}, \textit{@xml:space})  (att.global.rendition (\textit{@rend}, \textit{@style}, \textit{@rendition})) (att.global.linking (\textit{@corresp}, \textit{@synch}, \textit{@sameAs}, \textit{@copyOf}, \textit{@next}, \textit{@prev}, \textit{@exclude}, \textit{@select})) (att.global.analytic (\textit{@ana})) (att.global.facs (\textit{@facs})) (att.global.change (\textit{@change})) (att.global.responsibility (\textit{@cert}, \textit{@resp})) (att.global.source (\textit{@source})) att.declarable (\textit{@default}) 
    \item[{Member of}]
  model.profileDescPart
    \item[{Contained by}]
  
    \item[header: ]
   profileDesc
    \item[{May contain}]
  
    \item[header: ]
   catRef classCode keywords
    \item[{Example}]
  \leavevmode\bgroup\exampleFont \begin{shaded}\noindent\mbox{}{<\textbf{taxonomy}>}\mbox{}\newline 
\hspace*{6pt}{<\textbf{category}\hspace*{6pt}{xml:id}="{acprose}">}\mbox{}\newline 
\hspace*{6pt}\hspace*{6pt}{<\textbf{catDesc}>}Academic prose{</\textbf{catDesc}>}\mbox{}\newline 
\hspace*{6pt}{</\textbf{category}>}\mbox{}\newline 
\textit{<!-- other categories here -->}\mbox{}\newline 
{</\textbf{taxonomy}>}\mbox{}\newline 
\textit{<!-- ... -->}\mbox{}\newline 
{<\textbf{textClass}>}\mbox{}\newline 
\hspace*{6pt}{<\textbf{catRef}\hspace*{6pt}{target}="{\#acprose}"/>}\mbox{}\newline 
\hspace*{6pt}{<\textbf{classCode}\hspace*{6pt}{scheme}="{http://www.udcc.org}">}001.9{</\textbf{classCode}>}\mbox{}\newline 
\hspace*{6pt}{<\textbf{keywords}\hspace*{6pt}{scheme}="{http://authorities.loc.gov}">}\mbox{}\newline 
\hspace*{6pt}\hspace*{6pt}{<\textbf{list}>}\mbox{}\newline 
\hspace*{6pt}\hspace*{6pt}\hspace*{6pt}{<\textbf{item}>}End of the world{</\textbf{item}>}\mbox{}\newline 
\hspace*{6pt}\hspace*{6pt}\hspace*{6pt}{<\textbf{item}>}History - philosophy{</\textbf{item}>}\mbox{}\newline 
\hspace*{6pt}\hspace*{6pt}{</\textbf{list}>}\mbox{}\newline 
\hspace*{6pt}{</\textbf{keywords}>}\mbox{}\newline 
{</\textbf{textClass}>}\end{shaded}\egroup 


    \item[{Content model}]
  \mbox{}\hfill\\[-10pt]\begin{Verbatim}[fontsize=\small]
<content>
 <alternate maxOccurs="unbounded"
  minOccurs="0">
  <elementRef key="classCode"/>
  <elementRef key="catRef"/>
  <elementRef key="keywords"/>
 </alternate>
</content>
    
\end{Verbatim}

    \item[{Schema Declaration}]
  \mbox{}\hfill\\[-10pt]\begin{Verbatim}[fontsize=\small]
element textClass
{
   att.global.attributes,
   att.declarable.attributes,
   ( classCode | catRef | keywords )*
}
\end{Verbatim}

\end{reflist}  \index{textLang=<textLang>|oddindex}\index{mainLang=@mainLang!<textLang>|oddindex}\index{otherLangs=@otherLangs!<textLang>|oddindex}
\begin{reflist}
\item[]\begin{specHead}{TEI.textLang}{<textLang> }(text language) describes the languages and writing systems identified within the bibliographic work being described, rather than its description. [\xref{http://www.tei-c.org/release/doc/tei-p5-doc/en/html/CO.html\#COBICOI}{3.11.2.4. Imprint, Size of a Document, and Reprint Information} \xref{http://www.tei-c.org/release/doc/tei-p5-doc/en/html/MS.html\#mslangs}{10.6.6. Languages and Writing Systems}]\end{specHead} 
    \item[{Module}]
  core
    \item[{Attributes}]
  Attributes att.global (\textit{@xml:id}, \textit{@n}, \textit{@xml:lang}, \textit{@xml:base}, \textit{@xml:space})  (att.global.rendition (\textit{@rend}, \textit{@style}, \textit{@rendition})) (att.global.linking (\textit{@corresp}, \textit{@synch}, \textit{@sameAs}, \textit{@copyOf}, \textit{@next}, \textit{@prev}, \textit{@exclude}, \textit{@select})) (att.global.analytic (\textit{@ana})) (att.global.facs (\textit{@facs})) (att.global.change (\textit{@change})) (att.global.responsibility (\textit{@cert}, \textit{@resp})) (att.global.source (\textit{@source})) \hfil\\[-10pt]\begin{sansreflist}
    \item[@mainLang]
  (main language) supplies a code which identifies the chief language used in the bibliographic work.
\begin{reflist}
    \item[{Status}]
  Optional
    \item[{Datatype}]
  teidata.language
\end{reflist}  
    \item[@otherLangs]
  (other languages) one or more codes identifying any other languages used in the bibliographic work.
\begin{reflist}
    \item[{Status}]
  Optional
    \item[{Datatype}]
  0–∞ occurrences of teidata.language separated by whitespace
\end{reflist}  
\end{sansreflist}  
    \item[{Member of}]
  model.biblPart model.msItemPart 
    \item[{Contained by}]
  
    \item[core: ]
   analytic bibl monogr series\par 
    \item[msdescription: ]
   msContents msItem msItemStruct
    \item[{May contain}]
  
    \item[analysis: ]
   c cl interp interpGrp m pc phr s span spanGrp w\par 
    \item[core: ]
   abbr add address cb choice corr date del distinct email emph expan foreign gap gb gloss graphic hi index lb measure measureGrp media mentioned milestone name note num orig pb ptr ref reg rs sic soCalled term time title unclear\par 
    \item[figures: ]
   figure formula notatedMusic\par 
    \item[gaiji: ]
   g\par 
    \item[header: ]
   idno\par 
    \item[linking: ]
   alt altGrp anchor join joinGrp link linkGrp seg timeline\par 
    \item[msdescription: ]
   catchwords depth dim dimensions height heraldry locus locusGrp material objectType origDate origPlace secFol signatures stamp watermark width\par 
    \item[namesdates: ]
   addName affiliation bloc climate country district forename genName geo geogFeat geogName location nameLink offset orgName persName placeName population region roleName settlement state surname terrain trait\par 
    \item[textcrit: ]
   app witDetail\par 
    \item[transcr: ]
   addSpan am damage damageSpan delSpan ex fw handShift listTranspose metamark mod redo restore retrace secl space subst substJoin supplied surplus undo\par character data
    \item[{Note}]
  \par
This element should not be used to document the languages or writing systems used for the bibliographic or manuscript description itself: as for all other TEI elements, such information should be provided by means of the global {\itshape xml:lang} attribute attached to the element containing the description.\par
In all cases, languages should be identified by means of a standardized ‘language tag’ generated according to \xref{https://tools.ietf.org/html/bcp47}{BCP 47}. Additional documentation for the language may be provided by a <language> element in the TEI Header.
    \item[{Example}]
  \leavevmode\bgroup\exampleFont \begin{shaded}\noindent\mbox{}{<\textbf{textLang}\hspace*{6pt}{mainLang}="{en}"\hspace*{6pt}{otherLangs}="{la}">} Predominantly in English with Latin\mbox{}\newline 
 glosses{</\textbf{textLang}>}\end{shaded}\egroup 


    \item[{Content model}]
  \mbox{}\hfill\\[-10pt]\begin{Verbatim}[fontsize=\small]
<content>
 <macroRef key="macro.phraseSeq"/>
</content>
    
\end{Verbatim}

    \item[{Schema Declaration}]
  \mbox{}\hfill\\[-10pt]\begin{Verbatim}[fontsize=\small]
element textLang
{
   att.global.attributes,
   attribute mainLang { text }?,
   attribute otherLangs { list { * } }?,
   macro.phraseSeq}
\end{Verbatim}

\end{reflist}  \index{time=<time>|oddindex}
\begin{reflist}
\item[]\begin{specHead}{TEI.time}{<time> }contains a phrase defining a time of day in any format. [\xref{http://www.tei-c.org/release/doc/tei-p5-doc/en/html/CO.html\#CONADA}{3.5.4. Dates and Times}]\end{specHead} 
    \item[{Module}]
  core
    \item[{Attributes}]
  Attributes att.global (\textit{@xml:id}, \textit{@n}, \textit{@xml:lang}, \textit{@xml:base}, \textit{@xml:space})  (att.global.rendition (\textit{@rend}, \textit{@style}, \textit{@rendition})) (att.global.linking (\textit{@corresp}, \textit{@synch}, \textit{@sameAs}, \textit{@copyOf}, \textit{@next}, \textit{@prev}, \textit{@exclude}, \textit{@select})) (att.global.analytic (\textit{@ana})) (att.global.facs (\textit{@facs})) (att.global.change (\textit{@change})) (att.global.responsibility (\textit{@cert}, \textit{@resp})) (att.global.source (\textit{@source})) att.datable (\textit{@calendar}, \textit{@period})  (att.datable.w3c (\textit{@when}, \textit{@notBefore}, \textit{@notAfter}, \textit{@from}, \textit{@to})) (att.datable.iso (\textit{@when-iso}, \textit{@notBefore-iso}, \textit{@notAfter-iso}, \textit{@from-iso}, \textit{@to-iso})) (att.datable.custom (\textit{@when-custom}, \textit{@notBefore-custom}, \textit{@notAfter-custom}, \textit{@from-custom}, \textit{@to-custom}, \textit{@datingPoint}, \textit{@datingMethod})) att.editLike (\textit{@evidence}, \textit{@instant})  (att.dimensions (\textit{@unit}, \textit{@quantity}, \textit{@extent}, \textit{@precision}, \textit{@scope}) (att.ranging (\textit{@atLeast}, \textit{@atMost}, \textit{@min}, \textit{@max}, \textit{@confidence})) ) att.typed (\textit{@type}, \textit{@subtype}) 
    \item[{Member of}]
  model.dateLike
    \item[{Contained by}]
  
    \item[analysis: ]
   cl phr s span\par 
    \item[core: ]
   abbr add addrLine author bibl biblScope citedRange corr date del desc distinct editor email emph expan foreign gloss head headItem headLabel hi imprint item l label measure meeting mentioned name note num orig p pubPlace publisher q quote ref reg resp rs said sic soCalled speaker stage street term textLang time title unclear\par 
    \item[figures: ]
   cell figDesc\par 
    \item[header: ]
   authority catDesc change classCode correspAction creation distributor edition extent funder geoDecl handNote language licence principal rendition scriptNote sponsor tagUsage typeNote\par 
    \item[linking: ]
   ab seg\par 
    \item[msdescription: ]
   accMat acquisition additions catchwords collation colophon condition custEvent decoNote explicit filiation finalRubric foliation heraldry incipit layout material musicNotation objectType origDate origPlace origin provenance rubric secFol signatures source stamp summary support surrogates watermark\par 
    \item[namesdates: ]
   addName affiliation age birth bloc country death district education faith floruit forename genName geogFeat geogName langKnown nameLink nationality occupation offset orgName persName placeName region residence roleName settlement sex socecStatus surname\par 
    \item[textcrit: ]
   lem rdg wit witDetail witness\par 
    \item[textstructure: ]
   byline closer dateline docAuthor docDate docEdition docImprint imprimatur opener salute signed titlePart trailer\par 
    \item[transcr: ]
   damage fw metamark mod restore retrace secl supplied surplus
    \item[{May contain}]
  
    \item[analysis: ]
   c cl interp interpGrp m pc phr s span spanGrp w\par 
    \item[core: ]
   abbr add address cb choice corr date del distinct email emph expan foreign gap gb gloss graphic hi index lb measure measureGrp media mentioned milestone name note num orig pb ptr ref reg rs sic soCalled term time title unclear\par 
    \item[figures: ]
   figure formula notatedMusic\par 
    \item[gaiji: ]
   g\par 
    \item[header: ]
   idno\par 
    \item[linking: ]
   alt altGrp anchor join joinGrp link linkGrp seg timeline\par 
    \item[msdescription: ]
   catchwords depth dim dimensions height heraldry locus locusGrp material objectType origDate origPlace secFol signatures stamp watermark width\par 
    \item[namesdates: ]
   addName affiliation bloc climate country district forename genName geo geogFeat geogName location nameLink offset orgName persName placeName population region roleName settlement state surname terrain trait\par 
    \item[textcrit: ]
   app witDetail\par 
    \item[transcr: ]
   addSpan am damage damageSpan delSpan ex fw handShift listTranspose metamark mod redo restore retrace secl space subst substJoin supplied surplus undo\par character data
    \item[{Example}]
  \leavevmode\bgroup\exampleFont \begin{shaded}\noindent\mbox{}As he sat smiling, the\mbox{}\newline 
 quarter struck — {<\textbf{time}\hspace*{6pt}{when}="{11:45:00}">}the quarter to twelve{</\textbf{time}>}.\end{shaded}\egroup 


    \item[{Content model}]
  \mbox{}\hfill\\[-10pt]\begin{Verbatim}[fontsize=\small]
<content>
 <alternate maxOccurs="unbounded"
  minOccurs="0">
  <textNode/>
  <classRef key="model.gLike"/>
  <classRef key="model.phrase"/>
  <classRef key="model.global"/>
 </alternate>
</content>
    
\end{Verbatim}

    \item[{Schema Declaration}]
  \mbox{}\hfill\\[-10pt]\begin{Verbatim}[fontsize=\small]
element time
{
   att.global.attributes,
   att.datable.attributes,
   att.editLike.attributes,
   att.typed.attributes,
   ( text | model.gLike | model.phrase | model.global )*
}
\end{Verbatim}

\end{reflist}  \index{timeline=<timeline>|oddindex}\index{origin=@origin!<timeline>|oddindex}\index{unit=@unit!<timeline>|oddindex}\index{interval=@interval!<timeline>|oddindex}
\begin{reflist}
\item[]\begin{specHead}{TEI.timeline}{<timeline> }provides a set of ordered points in time which can be linked to elements of a spoken text to create a temporal alignment of that text. [\xref{http://www.tei-c.org/release/doc/tei-p5-doc/en/html/SA.html\#SASYMP}{16.4.2. Placing Synchronous Events in Time}]\end{specHead} 
    \item[{Module}]
  linking
    \item[{Attributes}]
  Attributes att.global (\textit{@xml:id}, \textit{@n}, \textit{@xml:lang}, \textit{@xml:base}, \textit{@xml:space})  (att.global.rendition (\textit{@rend}, \textit{@style}, \textit{@rendition})) (att.global.linking (\textit{@corresp}, \textit{@synch}, \textit{@sameAs}, \textit{@copyOf}, \textit{@next}, \textit{@prev}, \textit{@exclude}, \textit{@select})) (att.global.analytic (\textit{@ana})) (att.global.facs (\textit{@facs})) (att.global.change (\textit{@change})) (att.global.responsibility (\textit{@cert}, \textit{@resp})) (att.global.source (\textit{@source})) \hfil\\[-10pt]\begin{sansreflist}
    \item[@origin]
  designates the origin of the timeline, i.e. the time at which it begins.
\begin{reflist}
    \item[{Status}]
  Optional
    \item[{Datatype}]
  teidata.pointer
    \item[{Note}]
  \par
If this attribute is not supplied, the implication is that the time of origin is not known. If it is supplied, it must point either to one of the <when> elements in its content, or to another <timeline> element.
\end{reflist}  
    \item[@unit]
  specifies the unit of time corresponding to the {\itshape interval} value of the timeline or of its constituent points in time.
\begin{reflist}
    \item[{Status}]
  Optional
    \item[{Datatype}]
  teidata.enumerated
    \item[{Suggested values include:}]
  \begin{description}

\item[{d}](days)
\item[{h}](hours)
\item[{min}](minutes)
\item[{s}](seconds)
\item[{ms}](milliseconds)
\end{description} 
\end{reflist}  
    \item[@interval]
  specifies a time interval either as a positive integral value or using one of a set of predefined codes.
\begin{reflist}
    \item[{Status}]
  Optional
    \item[{Datatype}]
  teidata.interval
    \item[{Note}]
  \par
The value irregular indicates uncertainty about all the intervals in the timeline; the value regular indicates that all the intervals are evenly spaced, but the size of the intervals is not known; numeric values indicate evenly spaced values of the size specified. If individual points in time in the timeline are given different values for the {\itshape interval} attribute, those values locally override the value given in the timeline.
\end{reflist}  
\end{sansreflist}  
    \item[{Member of}]
  model.global.meta
    \item[{Contained by}]
  
    \item[analysis: ]
   cl m phr s span w\par 
    \item[core: ]
   abbr add addrLine address author bibl biblScope cit citedRange corr date del distinct editor email emph expan foreign gloss head headItem headLabel hi imprint item l label lg list measure mentioned name note num orig p pubPlace publisher q quote ref reg resp rs said series sic soCalled sp speaker stage street term textLang time title unclear\par 
    \item[figures: ]
   cell figure table\par 
    \item[header: ]
   authority change classCode distributor edition extent funder geoDecl handNote language licence principal scriptNote sponsor typeNote\par 
    \item[linking: ]
   ab seg\par 
    \item[msdescription: ]
   accMat acquisition additions catchwords collation colophon condition custEvent decoNote explicit filiation finalRubric foliation heraldry incipit layout material msItem musicNotation objectType origDate origPlace origin provenance rubric secFol signatures source stamp summary support surrogates watermark\par 
    \item[namesdates: ]
   addName affiliation age birth bloc country death district education faith floruit forename genName geogFeat geogName langKnown nameLink nationality occupation offset orgName persName person personGrp placeName region residence roleName settlement sex socecStatus surname\par 
    \item[textcrit: ]
   lem rdg wit witDetail\par 
    \item[textstructure: ]
   argument back body byline closer dateline div docAuthor docDate docEdition docImprint docTitle epigraph floatingText front group imprimatur opener postscript salute signed text titlePage titlePart trailer\par 
    \item[transcr: ]
   damage fw line metamark mod restore retrace secl sourceDoc supplied surface surfaceGrp surplus zone
    \item[{May contain}]
  
    \item[linking: ]
   when
    \item[{Example}]
  \leavevmode\bgroup\exampleFont \begin{shaded}\noindent\mbox{}{<\textbf{timeline}\hspace*{6pt}{unit}="{ms}"\hspace*{6pt}{xml:id}="{TL01}">}\mbox{}\newline 
\hspace*{6pt}{<\textbf{when}\hspace*{6pt}{absolute}="{11:30:00}"\hspace*{6pt}{xml:id}="{TL-w0}"/>}\mbox{}\newline 
\hspace*{6pt}{<\textbf{when}\hspace*{6pt}{interval}="{unknown}"\hspace*{6pt}{since}="{\#TL-w0}"\mbox{}\newline 
\hspace*{6pt}\hspace*{6pt}{xml:id}="{TL-w1}"/>}\mbox{}\newline 
\hspace*{6pt}{<\textbf{when}\hspace*{6pt}{interval}="{100}"\hspace*{6pt}{since}="{\#TL-w1}"\mbox{}\newline 
\hspace*{6pt}\hspace*{6pt}{xml:id}="{TL-w2}"/>}\mbox{}\newline 
\hspace*{6pt}{<\textbf{when}\hspace*{6pt}{interval}="{200}"\hspace*{6pt}{since}="{\#TL-w2}"\mbox{}\newline 
\hspace*{6pt}\hspace*{6pt}{xml:id}="{TL-w3}"/>}\mbox{}\newline 
\hspace*{6pt}{<\textbf{when}\hspace*{6pt}{interval}="{150}"\hspace*{6pt}{since}="{\#TL-w3}"\mbox{}\newline 
\hspace*{6pt}\hspace*{6pt}{xml:id}="{TL-w4}"/>}\mbox{}\newline 
\hspace*{6pt}{<\textbf{when}\hspace*{6pt}{interval}="{250}"\hspace*{6pt}{since}="{\#TL-w4}"\mbox{}\newline 
\hspace*{6pt}\hspace*{6pt}{xml:id}="{TL-w5}"/>}\mbox{}\newline 
\hspace*{6pt}{<\textbf{when}\hspace*{6pt}{interval}="{100}"\hspace*{6pt}{since}="{\#TL-w5}"\mbox{}\newline 
\hspace*{6pt}\hspace*{6pt}{xml:id}="{TL-w6}"/>}\mbox{}\newline 
{</\textbf{timeline}>}\end{shaded}\egroup 


    \item[{Content model}]
  \mbox{}\hfill\\[-10pt]\begin{Verbatim}[fontsize=\small]
<content>
 <elementRef key="when"
  maxOccurs="unbounded" minOccurs="1"/>
</content>
    
\end{Verbatim}

    \item[{Schema Declaration}]
  \mbox{}\hfill\\[-10pt]\begin{Verbatim}[fontsize=\small]
element timeline
{
   att.global.attributes,
   attribute origin { text }?,
   attribute unit { "d" | "h" | "min" | "s" | "ms" }?,
   attribute interval { text }?,
   when+
}
\end{Verbatim}

\end{reflist}  \index{title=<title>|oddindex}\index{type=@type!<title>|oddindex}\index{level=@level!<title>|oddindex}
\begin{reflist}
\item[]\begin{specHead}{TEI.title}{<title> }contains a title for any kind of work. [\xref{http://www.tei-c.org/release/doc/tei-p5-doc/en/html/CO.html\#COBICOR}{3.11.2.2. Titles, Authors, and Editors} \xref{http://www.tei-c.org/release/doc/tei-p5-doc/en/html/HD.html\#HD21}{2.2.1. The Title Statement} \xref{http://www.tei-c.org/release/doc/tei-p5-doc/en/html/HD.html\#HD26}{2.2.5. The Series Statement}]\end{specHead} 
    \item[{Module}]
  core
    \item[{Attributes}]
  Attributes att.global (\textit{@xml:id}, \textit{@n}, \textit{@xml:lang}, \textit{@xml:base}, \textit{@xml:space})  (att.global.rendition (\textit{@rend}, \textit{@style}, \textit{@rendition})) (att.global.linking (\textit{@corresp}, \textit{@synch}, \textit{@sameAs}, \textit{@copyOf}, \textit{@next}, \textit{@prev}, \textit{@exclude}, \textit{@select})) (att.global.analytic (\textit{@ana})) (att.global.facs (\textit{@facs})) (att.global.change (\textit{@change})) (att.global.responsibility (\textit{@cert}, \textit{@resp})) (att.global.source (\textit{@source})) att.canonical (\textit{@key}, \textit{@ref}) att.datable (\textit{@calendar}, \textit{@period})  (att.datable.w3c (\textit{@when}, \textit{@notBefore}, \textit{@notAfter}, \textit{@from}, \textit{@to})) (att.datable.iso (\textit{@when-iso}, \textit{@notBefore-iso}, \textit{@notAfter-iso}, \textit{@from-iso}, \textit{@to-iso})) (att.datable.custom (\textit{@when-custom}, \textit{@notBefore-custom}, \textit{@notAfter-custom}, \textit{@from-custom}, \textit{@to-custom}, \textit{@datingPoint}, \textit{@datingMethod})) att.typed (\unusedattribute{type}, @subtype) \hfil\\[-10pt]\begin{sansreflist}
    \item[@type]
  classifies the title according to some convenient typology.
\begin{reflist}
    \item[{Derived from}]
  att.typed
    \item[{Status}]
  Optional
    \item[{Datatype}]
  teidata.enumerated
    \item[{Sample values include:}]
  \begin{description}

\item[{main}]main title
\item[{sub}](subordinate) subtitle, title of part
\item[{alt}](alternate) alternate title, often in another language, by which the work is also known
\item[{short}]abbreviated form of title
\item[{desc}](descriptive) descriptive paraphrase of the work functioning as a title
\end{description} 
    \item[{Note}]
  \par
This attribute is provided for convenience in analysing titles and processing them according to their type; where such specialized processing is not necessary, there is no need for such analysis, and the entire title, including subtitles and any parallel titles, may be enclosed within a single <title> element.
\end{reflist}  
    \item[@level]
  indicates the bibliographic level for a title, that is, whether it identifies an article, book, journal, series, or unpublished material.
\begin{reflist}
    \item[{Status}]
  Optional
    \item[{Datatype}]
  teidata.enumerated
    \item[{Legal values are:}]
  \begin{description}

\item[{a}](analytic) the title applies to an analytic item, such as an article, poem, or other work published as part of a larger item.
\item[{m}](monographic) the title applies to a monograph such as a book or other item considered to be a distinct publication, including single volumes of multi-volume works
\item[{j}](journal) the title applies to any serial or periodical publication such as a journal, magazine, or newspaper
\item[{s}](series) the title applies to a series of otherwise distinct publications such as a collection
\item[{u}](unpublished) the title applies to any unpublished material (including theses and dissertations unless published by a commercial press)
\end{description} 
    \item[{Note}]
  \par
The level of a title is sometimes implied by its context: for example, a title appearing directly within an <analytic> element is \textit{ipso facto} of level ‘a’, and one appearing within a <series> element of level ‘s’. For this reason, the {\itshape level} attribute is not required in contexts where its value can be unambiguously inferred. Where it is supplied in such contexts, its value should not contradict the value implied by its parent element.
\end{reflist}  
\end{sansreflist}  
    \item[{Member of}]
  model.emphLike model.msQuoteLike 
    \item[{Contained by}]
  
    \item[analysis: ]
   cl phr s span\par 
    \item[core: ]
   abbr add addrLine analytic author bibl biblScope citedRange corr date del desc distinct editor email emph expan foreign gloss head headItem headLabel hi item l label measure meeting mentioned monogr name note num orig p pubPlace publisher q quote ref reg resp rs said series sic soCalled speaker stage street term textLang time title unclear\par 
    \item[figures: ]
   cell figDesc\par 
    \item[header: ]
   authority catDesc change classCode creation distributor edition extent funder geoDecl handNote language licence principal rendition scriptNote seriesStmt sponsor tagUsage titleStmt typeNote\par 
    \item[linking: ]
   ab seg\par 
    \item[msdescription: ]
   accMat acquisition additions catchwords collation colophon condition custEvent decoNote explicit filiation finalRubric foliation heraldry incipit layout material msItem msItemStruct musicNotation objectType origDate origPlace origin provenance rubric secFol signatures source stamp summary support surrogates watermark\par 
    \item[namesdates: ]
   addName affiliation age birth bloc country death district education faith floruit forename genName geogFeat geogName langKnown nameLink nationality occupation offset orgName persName placeName region residence roleName settlement sex socecStatus surname\par 
    \item[textcrit: ]
   lem rdg wit witDetail witness\par 
    \item[textstructure: ]
   byline closer dateline docAuthor docDate docEdition docImprint imprimatur opener salute signed titlePart trailer\par 
    \item[transcr: ]
   damage fw metamark mod restore retrace secl supplied surplus
    \item[{May contain}]
  
    \item[analysis: ]
   c cl interp interpGrp m pc phr s span spanGrp w\par 
    \item[core: ]
   abbr add address bibl biblStruct cb choice cit corr date del desc distinct email emph expan foreign gap gb gloss graphic hi index l label lb lg list listBibl measure measureGrp media mentioned milestone name note num orig pb ptr q quote ref reg rs said sic soCalled stage term time title unclear\par 
    \item[figures: ]
   figure formula notatedMusic table\par 
    \item[gaiji: ]
   g\par 
    \item[header: ]
   biblFull idno\par 
    \item[linking: ]
   alt altGrp anchor join joinGrp link linkGrp seg timeline\par 
    \item[msdescription: ]
   catchwords depth dim dimensions height heraldry locus locusGrp material msDesc objectType origDate origPlace secFol signatures stamp watermark width\par 
    \item[namesdates: ]
   addName affiliation bloc climate country district forename genName geo geogFeat geogName listEvent listNym listOrg listPerson listPlace location nameLink offset orgName persName placeName population region roleName settlement state surname terrain trait\par 
    \item[textcrit: ]
   app listApp listWit witDetail\par 
    \item[textstructure: ]
   floatingText\par 
    \item[transcr: ]
   addSpan am damage damageSpan delSpan ex fw handShift listTranspose metamark mod redo restore retrace secl space subst substJoin supplied surplus undo\par character data
    \item[{Note}]
  \par
The attributes {\itshape key} and {\itshape ref}, inherited from the class \textsf{att.canonical} may be used to indicate the canonical form for the title; the former, by supplying (for example) the identifier of a record in some external library system; the latter by pointing to an XML element somewhere containing the canonical form of the title.
    \item[{Example}]
  \leavevmode\bgroup\exampleFont \begin{shaded}\noindent\mbox{}{<\textbf{title}>}Information Technology and the Research Process: Proceedings of\mbox{}\newline 
 a conference held at Cranfield Institute of Technology, UK,\mbox{}\newline 
 18–21 July 1989{</\textbf{title}>}\end{shaded}\egroup 


    \item[{Example}]
  \leavevmode\bgroup\exampleFont \begin{shaded}\noindent\mbox{}{<\textbf{title}>}Hardy's Tess of the D'Urbervilles: a machine readable\mbox{}\newline 
 edition{</\textbf{title}>}\end{shaded}\egroup 


    \item[{Example}]
  \leavevmode\bgroup\exampleFont \begin{shaded}\noindent\mbox{}{<\textbf{title}\hspace*{6pt}{type}="{full}">}\mbox{}\newline 
\hspace*{6pt}{<\textbf{title}\hspace*{6pt}{type}="{main}">}Synthèse{</\textbf{title}>}\mbox{}\newline 
\hspace*{6pt}{<\textbf{title}\hspace*{6pt}{type}="{sub}">}an international journal for\mbox{}\newline 
\hspace*{6pt}\hspace*{6pt} epistemology, methodology and history of\mbox{}\newline 
\hspace*{6pt}\hspace*{6pt} science{</\textbf{title}>}\mbox{}\newline 
{</\textbf{title}>}\end{shaded}\egroup 


    \item[{Content model}]
  \mbox{}\hfill\\[-10pt]\begin{Verbatim}[fontsize=\small]
<content>
 <macroRef key="macro.paraContent"/>
</content>
    
\end{Verbatim}

    \item[{Schema Declaration}]
  \mbox{}\hfill\\[-10pt]\begin{Verbatim}[fontsize=\small]
element title
{
   att.global.attributes,
   att.canonical.attributes,
   att.typed.attribute.subtype,
   att.datable.attributes,
   attribute type { text }?,
   attribute level { "a" | "m" | "j" | "s" | "u" }?,
   macro.paraContent}
\end{Verbatim}

\end{reflist}  \index{titlePage=<titlePage>|oddindex}\index{type=@type!<titlePage>|oddindex}
\begin{reflist}
\item[]\begin{specHead}{TEI.titlePage}{<titlePage> }(title page) contains the title page of a text, appearing within the front or back matter. [\xref{http://www.tei-c.org/release/doc/tei-p5-doc/en/html/DS.html\#DSTITL}{4.6. Title Pages}]\end{specHead} 
    \item[{Module}]
  textstructure
    \item[{Attributes}]
  Attributes att.global (\textit{@xml:id}, \textit{@n}, \textit{@xml:lang}, \textit{@xml:base}, \textit{@xml:space})  (att.global.rendition (\textit{@rend}, \textit{@style}, \textit{@rendition})) (att.global.linking (\textit{@corresp}, \textit{@synch}, \textit{@sameAs}, \textit{@copyOf}, \textit{@next}, \textit{@prev}, \textit{@exclude}, \textit{@select})) (att.global.analytic (\textit{@ana})) (att.global.facs (\textit{@facs})) (att.global.change (\textit{@change})) (att.global.responsibility (\textit{@cert}, \textit{@resp})) (att.global.source (\textit{@source})) \hfil\\[-10pt]\begin{sansreflist}
    \item[@type]
  classifies the title page according to any convenient typology.
\begin{reflist}
    \item[{Status}]
  Optional
    \item[{Datatype}]
  teidata.enumerated
    \item[{Note}]
  \par
This attribute allows the same element to be used for volume title pages, series title pages, etc., as well as for the ‘main’ title page of a work.
\end{reflist}  
\end{sansreflist}  
    \item[{Member of}]
  model.frontPart 
    \item[{Contained by}]
  
    \item[msdescription: ]
   msContents\par 
    \item[textstructure: ]
   back front
    \item[{May contain}]
  
    \item[analysis: ]
   interp interpGrp span spanGrp\par 
    \item[core: ]
   cb gap gb graphic index lb milestone note pb\par 
    \item[figures: ]
   figure notatedMusic\par 
    \item[linking: ]
   alt altGrp anchor join joinGrp link linkGrp timeline\par 
    \item[textcrit: ]
   app witDetail\par 
    \item[textstructure: ]
   argument byline docAuthor docDate docEdition docImprint docTitle epigraph imprimatur titlePart\par 
    \item[transcr: ]
   addSpan damageSpan delSpan fw listTranspose metamark space substJoin
    \item[{Example}]
  \leavevmode\bgroup\exampleFont \begin{shaded}\noindent\mbox{}{<\textbf{titlePage}>}\mbox{}\newline 
\hspace*{6pt}{<\textbf{docTitle}>}\mbox{}\newline 
\hspace*{6pt}\hspace*{6pt}{<\textbf{titlePart}\hspace*{6pt}{type}="{main}">}THOMAS OF Reading.{</\textbf{titlePart}>}\mbox{}\newline 
\hspace*{6pt}\hspace*{6pt}{<\textbf{titlePart}\hspace*{6pt}{type}="{alt}">}OR, The sixe worthy yeomen of the West.{</\textbf{titlePart}>}\mbox{}\newline 
\hspace*{6pt}{</\textbf{docTitle}>}\mbox{}\newline 
\hspace*{6pt}{<\textbf{docEdition}>}Now the fourth time corrected and enlarged{</\textbf{docEdition}>}\mbox{}\newline 
\hspace*{6pt}{<\textbf{byline}>}By T.D.{</\textbf{byline}>}\mbox{}\newline 
\hspace*{6pt}{<\textbf{figure}>}\mbox{}\newline 
\hspace*{6pt}\hspace*{6pt}{<\textbf{head}>}TP{</\textbf{head}>}\mbox{}\newline 
\hspace*{6pt}\hspace*{6pt}{<\textbf{p}>}Thou shalt labor till thou returne to duste{</\textbf{p}>}\mbox{}\newline 
\hspace*{6pt}\hspace*{6pt}{<\textbf{figDesc}>}Printers Ornament used by TP{</\textbf{figDesc}>}\mbox{}\newline 
\hspace*{6pt}{</\textbf{figure}>}\mbox{}\newline 
\hspace*{6pt}{<\textbf{docImprint}>}Printed at {<\textbf{name}\hspace*{6pt}{type}="{place}">}London{</\textbf{name}>} for {<\textbf{name}>}T.P.{</\textbf{name}>}\mbox{}\newline 
\hspace*{6pt}\hspace*{6pt}{<\textbf{date}>}1612.{</\textbf{date}>}\mbox{}\newline 
\hspace*{6pt}{</\textbf{docImprint}>}\mbox{}\newline 
{</\textbf{titlePage}>}\end{shaded}\egroup 


    \item[{Content model}]
  \mbox{}\hfill\\[-10pt]\begin{Verbatim}[fontsize=\small]
<content>
 <sequence>
  <classRef key="model.global"
   maxOccurs="unbounded" minOccurs="0"/>
  <classRef key="model.titlepagePart"/>
  <alternate maxOccurs="unbounded"
   minOccurs="0">
   <classRef key="model.titlepagePart"/>
   <classRef key="model.global"/>
  </alternate>
 </sequence>
</content>
    
\end{Verbatim}

    \item[{Schema Declaration}]
  \mbox{}\hfill\\[-10pt]\begin{Verbatim}[fontsize=\small]
element titlePage
{
   att.global.attributes,
   attribute type { text }?,
   (
      model.global*,
      model.titlepagePart,
      ( model.titlepagePart | model.global )*
   )
}
\end{Verbatim}

\end{reflist}  \index{titlePart=<titlePart>|oddindex}\index{type=@type!<titlePart>|oddindex}
\begin{reflist}
\item[]\begin{specHead}{TEI.titlePart}{<titlePart> }contains a subsection or division of the title of a work, as indicated on a title page. [\xref{http://www.tei-c.org/release/doc/tei-p5-doc/en/html/DS.html\#DSTITL}{4.6. Title Pages}]\end{specHead} 
    \item[{Module}]
  textstructure
    \item[{Attributes}]
  Attributes att.global (\textit{@xml:id}, \textit{@n}, \textit{@xml:lang}, \textit{@xml:base}, \textit{@xml:space})  (att.global.rendition (\textit{@rend}, \textit{@style}, \textit{@rendition})) (att.global.linking (\textit{@corresp}, \textit{@synch}, \textit{@sameAs}, \textit{@copyOf}, \textit{@next}, \textit{@prev}, \textit{@exclude}, \textit{@select})) (att.global.analytic (\textit{@ana})) (att.global.facs (\textit{@facs})) (att.global.change (\textit{@change})) (att.global.responsibility (\textit{@cert}, \textit{@resp})) (att.global.source (\textit{@source})) \hfil\\[-10pt]\begin{sansreflist}
    \item[@type]
  specifies the role of this subdivision of the title.
\begin{reflist}
    \item[{Status}]
  Optional
    \item[{Datatype}]
  teidata.enumerated
    \item[{Suggested values include:}]
  \begin{description}

\item[{main}]main title of the work{[Default] }
\item[{sub}](subordinate) subtitle of the work
\item[{alt}](alternate) alternative title of the work
\item[{short}]abbreviated form of title
\item[{desc}](descriptive) descriptive paraphrase of the work
\end{description} 
\end{reflist}  
\end{sansreflist}  
    \item[{Member of}]
  model.pLike.front model.titlepagePart
    \item[{Contained by}]
  
    \item[msdescription: ]
   msItem\par 
    \item[textstructure: ]
   back docTitle front titlePage
    \item[{May contain}]
  
    \item[analysis: ]
   c cl interp interpGrp m pc phr s span spanGrp w\par 
    \item[core: ]
   abbr add address bibl biblStruct cb choice cit corr date del desc distinct email emph expan foreign gap gb gloss graphic hi index l label lb lg list listBibl measure measureGrp media mentioned milestone name note num orig pb ptr q quote ref reg rs said sic soCalled stage term time title unclear\par 
    \item[figures: ]
   figure formula notatedMusic table\par 
    \item[gaiji: ]
   g\par 
    \item[header: ]
   biblFull idno\par 
    \item[linking: ]
   alt altGrp anchor join joinGrp link linkGrp seg timeline\par 
    \item[msdescription: ]
   catchwords depth dim dimensions height heraldry locus locusGrp material msDesc objectType origDate origPlace secFol signatures stamp watermark width\par 
    \item[namesdates: ]
   addName affiliation bloc climate country district forename genName geo geogFeat geogName listEvent listNym listOrg listPerson listPlace location nameLink offset orgName persName placeName population region roleName settlement state surname terrain trait\par 
    \item[textcrit: ]
   app listApp listWit witDetail\par 
    \item[textstructure: ]
   floatingText\par 
    \item[transcr: ]
   addSpan am damage damageSpan delSpan ex fw handShift listTranspose metamark mod redo restore retrace secl space subst substJoin supplied surplus undo\par character data
    \item[{Example}]
  \leavevmode\bgroup\exampleFont \begin{shaded}\noindent\mbox{}{<\textbf{docTitle}>}\mbox{}\newline 
\hspace*{6pt}{<\textbf{titlePart}\hspace*{6pt}{type}="{main}">}THE FORTUNES\mbox{}\newline 
\hspace*{6pt}\hspace*{6pt} AND MISFORTUNES Of the FAMOUS\mbox{}\newline 
\hspace*{6pt}\hspace*{6pt} Moll Flanders, \&c.\mbox{}\newline 
\hspace*{6pt}{</\textbf{titlePart}>}\mbox{}\newline 
\hspace*{6pt}{<\textbf{titlePart}\hspace*{6pt}{type}="{desc}">}Who was BORN in NEWGATE,\mbox{}\newline 
\hspace*{6pt}\hspace*{6pt} And during a Life of continu'd Variety for\mbox{}\newline 
\hspace*{6pt}\hspace*{6pt} Threescore Years, besides her Childhood, was\mbox{}\newline 
\hspace*{6pt}\hspace*{6pt} Twelve Year a {<\textbf{hi}>}Whore{</\textbf{hi}>}, five times a {<\textbf{hi}>}Wife{</\textbf{hi}>} (wherof\mbox{}\newline 
\hspace*{6pt}\hspace*{6pt} once to her own Brother) Twelve Year a {<\textbf{hi}>}Thief,{</\textbf{hi}>}\mbox{}\newline 
\hspace*{6pt}\hspace*{6pt} Eight Year a Transported {<\textbf{hi}>}Felon{</\textbf{hi}>} in {<\textbf{hi}>}Virginia{</\textbf{hi}>},\mbox{}\newline 
\hspace*{6pt}\hspace*{6pt} at last grew {<\textbf{hi}>}Rich{</\textbf{hi}>}, liv'd {<\textbf{hi}>}Honest{</\textbf{hi}>}, and died a\mbox{}\newline 
\hspace*{6pt}{<\textbf{hi}>}Penitent{</\textbf{hi}>}.{</\textbf{titlePart}>}\mbox{}\newline 
{</\textbf{docTitle}>}\end{shaded}\egroup 


    \item[{Content model}]
  \mbox{}\hfill\\[-10pt]\begin{Verbatim}[fontsize=\small]
<content>
 <macroRef key="macro.paraContent"/>
</content>
    
\end{Verbatim}

    \item[{Schema Declaration}]
  \mbox{}\hfill\\[-10pt]\begin{Verbatim}[fontsize=\small]
element titlePart
{
   att.global.attributes,
   attribute type { "main" | "sub" | "alt" | "short" | "desc" }?,
   macro.paraContent}
\end{Verbatim}

\end{reflist}  \index{titleStmt=<titleStmt>|oddindex}
\begin{reflist}
\item[]\begin{specHead}{TEI.titleStmt}{<titleStmt> }(title statement) groups information about the title of a work and those responsible for its content. [\xref{http://www.tei-c.org/release/doc/tei-p5-doc/en/html/HD.html\#HD21}{2.2.1. The Title Statement} \xref{http://www.tei-c.org/release/doc/tei-p5-doc/en/html/HD.html\#HD2}{2.2. The File Description}]\end{specHead} 
    \item[{Module}]
  header
    \item[{Attributes}]
  Attributes att.global (\textit{@xml:id}, \textit{@n}, \textit{@xml:lang}, \textit{@xml:base}, \textit{@xml:space})  (att.global.rendition (\textit{@rend}, \textit{@style}, \textit{@rendition})) (att.global.linking (\textit{@corresp}, \textit{@synch}, \textit{@sameAs}, \textit{@copyOf}, \textit{@next}, \textit{@prev}, \textit{@exclude}, \textit{@select})) (att.global.analytic (\textit{@ana})) (att.global.facs (\textit{@facs})) (att.global.change (\textit{@change})) (att.global.responsibility (\textit{@cert}, \textit{@resp})) (att.global.source (\textit{@source}))
    \item[{Contained by}]
  
    \item[header: ]
   biblFull fileDesc
    \item[{May contain}]
  
    \item[core: ]
   author editor meeting respStmt title\par 
    \item[header: ]
   funder principal sponsor
    \item[{Example}]
  \leavevmode\bgroup\exampleFont \begin{shaded}\noindent\mbox{}{<\textbf{titleStmt}>}\mbox{}\newline 
\hspace*{6pt}{<\textbf{title}>}Capgrave's Life of St. John Norbert: a machine-readable transcription{</\textbf{title}>}\mbox{}\newline 
\hspace*{6pt}{<\textbf{respStmt}>}\mbox{}\newline 
\hspace*{6pt}\hspace*{6pt}{<\textbf{resp}>}compiled by{</\textbf{resp}>}\mbox{}\newline 
\hspace*{6pt}\hspace*{6pt}{<\textbf{name}>}P.J. Lucas{</\textbf{name}>}\mbox{}\newline 
\hspace*{6pt}{</\textbf{respStmt}>}\mbox{}\newline 
{</\textbf{titleStmt}>}\end{shaded}\egroup 


    \item[{Content model}]
  \mbox{}\hfill\\[-10pt]\begin{Verbatim}[fontsize=\small]
<content>
 <sequence>
  <elementRef key="title"
   maxOccurs="unbounded" minOccurs="1"/>
  <classRef key="model.respLike"
   maxOccurs="unbounded" minOccurs="0"/>
 </sequence>
</content>
    
\end{Verbatim}

    \item[{Schema Declaration}]
  \mbox{}\hfill\\[-10pt]\begin{Verbatim}[fontsize=\small]
element titleStmt { att.global.attributes, ( title+, model.respLike* ) }
\end{Verbatim}

\end{reflist}  \index{trailer=<trailer>|oddindex}
\begin{reflist}
\item[]\begin{specHead}{TEI.trailer}{<trailer> }contains a closing title or footer appearing at the end of a division of a text. [\xref{http://www.tei-c.org/release/doc/tei-p5-doc/en/html/DS.html\#DSCO}{4.2.4. Content of Textual Divisions} \xref{http://www.tei-c.org/release/doc/tei-p5-doc/en/html/DS.html\#DSDTB}{4.2. Elements Common to All Divisions}]\end{specHead} 
    \item[{Module}]
  textstructure
    \item[{Attributes}]
  Attributes att.global (\textit{@xml:id}, \textit{@n}, \textit{@xml:lang}, \textit{@xml:base}, \textit{@xml:space})  (att.global.rendition (\textit{@rend}, \textit{@style}, \textit{@rendition})) (att.global.linking (\textit{@corresp}, \textit{@synch}, \textit{@sameAs}, \textit{@copyOf}, \textit{@next}, \textit{@prev}, \textit{@exclude}, \textit{@select})) (att.global.analytic (\textit{@ana})) (att.global.facs (\textit{@facs})) (att.global.change (\textit{@change})) (att.global.responsibility (\textit{@cert}, \textit{@resp})) (att.global.source (\textit{@source})) att.typed (\textit{@type}, \textit{@subtype}) 
    \item[{Member of}]
  model.divBottomPart
    \item[{Contained by}]
  
    \item[core: ]
   lg list\par 
    \item[figures: ]
   figure table\par 
    \item[textstructure: ]
   back body div front group postscript
    \item[{May contain}]
  
    \item[analysis: ]
   c cl interp interpGrp m pc phr s span spanGrp w\par 
    \item[core: ]
   abbr add address bibl biblStruct cb choice cit corr date del desc distinct email emph expan foreign gap gb gloss graphic hi index l label lb lg list listBibl measure measureGrp media mentioned milestone name note num orig pb ptr q quote ref reg rs said sic soCalled stage term time title unclear\par 
    \item[figures: ]
   figure formula notatedMusic table\par 
    \item[gaiji: ]
   g\par 
    \item[header: ]
   biblFull idno\par 
    \item[linking: ]
   alt altGrp anchor join joinGrp link linkGrp seg timeline\par 
    \item[msdescription: ]
   catchwords depth dim dimensions height heraldry locus locusGrp material msDesc objectType origDate origPlace secFol signatures stamp watermark width\par 
    \item[namesdates: ]
   addName affiliation bloc climate country district forename genName geo geogFeat geogName listEvent listNym listOrg listPerson listPlace location nameLink offset orgName persName placeName population region roleName settlement state surname terrain trait\par 
    \item[textcrit: ]
   app listApp listWit witDetail\par 
    \item[textstructure: ]
   floatingText\par 
    \item[transcr: ]
   addSpan am damage damageSpan delSpan ex fw handShift listTranspose metamark mod redo restore retrace secl space subst substJoin supplied surplus undo\par character data
    \item[{Example}]
  \leavevmode\bgroup\exampleFont \begin{shaded}\noindent\mbox{}{<\textbf{trailer}>}Explicit pars tertia{</\textbf{trailer}>}\end{shaded}\egroup 


    \item[{Example}]
  \leavevmode\bgroup\exampleFont \begin{shaded}\noindent\mbox{}{<\textbf{trailer}>}\mbox{}\newline 
\hspace*{6pt}{<\textbf{l}>}In stead of FINIS this advice {<\textbf{hi}>}I{</\textbf{hi}>} send,{</\textbf{l}>}\mbox{}\newline 
\hspace*{6pt}{<\textbf{l}>}Let Rogues and Thieves beware of {<\textbf{lb}/>}\mbox{}\newline 
\hspace*{6pt}\hspace*{6pt}{<\textbf{hi}>}Hamans{</\textbf{hi}>} END.{</\textbf{l}>}\mbox{}\newline 
{</\textbf{trailer}>}\end{shaded}\egroup 

From EEBO A87070
    \item[{Content model}]
  \mbox{}\hfill\\[-10pt]\begin{Verbatim}[fontsize=\small]
<content>
 <alternate maxOccurs="unbounded"
  minOccurs="0">
  <textNode/>
  <elementRef key="lg"/>
  <classRef key="model.gLike"/>
  <classRef key="model.phrase"/>
  <classRef key="model.inter"/>
  <classRef key="model.lLike"/>
  <classRef key="model.global"/>
 </alternate>
</content>
    
\end{Verbatim}

    \item[{Schema Declaration}]
  \mbox{}\hfill\\[-10pt]\begin{Verbatim}[fontsize=\small]
element trailer
{
   att.global.attributes,
   att.typed.attributes,
   (
      text
    | lg    | model.gLike    | model.phrase    | model.inter    | model.lLike    | model.global   )*
}
\end{Verbatim}

\end{reflist}  \index{trait=<trait>|oddindex}
\begin{reflist}
\item[]\begin{specHead}{TEI.trait}{<trait> }contains a description of some status or quality attributed to a person, place, or organization typically, but not necessarily, independent of the volition or action of the holder and usually not at some specific time or for a specific date range. [\xref{http://www.tei-c.org/release/doc/tei-p5-doc/en/html/ND.html\#NDPERSbp}{13.3.1. Basic Principles} \xref{http://www.tei-c.org/release/doc/tei-p5-doc/en/html/ND.html\#NDPERSEpc}{13.3.2.1. Personal Characteristics}]\end{specHead} 
    \item[{Module}]
  namesdates
    \item[{Attributes}]
  Attributes att.global (\textit{@xml:id}, \textit{@n}, \textit{@xml:lang}, \textit{@xml:base}, \textit{@xml:space})  (att.global.rendition (\textit{@rend}, \textit{@style}, \textit{@rendition})) (att.global.linking (\textit{@corresp}, \textit{@synch}, \textit{@sameAs}, \textit{@copyOf}, \textit{@next}, \textit{@prev}, \textit{@exclude}, \textit{@select})) (att.global.analytic (\textit{@ana})) (att.global.facs (\textit{@facs})) (att.global.change (\textit{@change})) (att.global.responsibility (\textit{@cert}, \textit{@resp})) (att.global.source (\textit{@source})) att.datable (\textit{@calendar}, \textit{@period})  (att.datable.w3c (\textit{@when}, \textit{@notBefore}, \textit{@notAfter}, \textit{@from}, \textit{@to})) (att.datable.iso (\textit{@when-iso}, \textit{@notBefore-iso}, \textit{@notAfter-iso}, \textit{@from-iso}, \textit{@to-iso})) (att.datable.custom (\textit{@when-custom}, \textit{@notBefore-custom}, \textit{@notAfter-custom}, \textit{@from-custom}, \textit{@to-custom}, \textit{@datingPoint}, \textit{@datingMethod})) att.editLike (\textit{@evidence}, \textit{@instant})  (att.dimensions (\textit{@unit}, \textit{@quantity}, \textit{@extent}, \textit{@precision}, \textit{@scope}) (att.ranging (\textit{@atLeast}, \textit{@atMost}, \textit{@min}, \textit{@max}, \textit{@confidence})) ) att.naming (\textit{@role}, \textit{@nymRef})  (att.canonical (\textit{@key}, \textit{@ref})) att.typed (\textit{@type}, \textit{@subtype}) 
    \item[{Member of}]
  model.persStateLike model.placeStateLike 
    \item[{Contained by}]
  
    \item[analysis: ]
   cl phr s span\par 
    \item[core: ]
   abbr add addrLine address author bibl biblScope citedRange corr date del desc distinct editor email emph expan foreign gloss head headItem headLabel hi item l label measure meeting mentioned name note num orig p pubPlace publisher q quote ref reg resp rs said sic soCalled speaker stage street term textLang time title unclear\par 
    \item[figures: ]
   cell figDesc\par 
    \item[header: ]
   authority catDesc change classCode correspAction creation distributor edition extent funder geoDecl handNote language licence principal rendition scriptNote sponsor tagUsage typeNote\par 
    \item[linking: ]
   ab seg\par 
    \item[msdescription: ]
   accMat acquisition additions catchwords collation colophon condition custEvent decoNote explicit filiation finalRubric foliation heraldry incipit layout material musicNotation objectType origDate origPlace origin provenance rubric secFol signatures source stamp summary support surrogates watermark\par 
    \item[namesdates: ]
   addName affiliation age birth bloc country death district education faith floruit forename genName geogFeat geogName langKnown nameLink nationality occupation offset org orgName persName person personGrp place placeName region residence roleName settlement sex socecStatus surname trait\par 
    \item[textcrit: ]
   lem rdg wit witDetail witness\par 
    \item[textstructure: ]
   byline closer dateline docAuthor docDate docEdition docImprint imprimatur opener salute signed titlePart trailer\par 
    \item[transcr: ]
   damage fw metamark mod restore retrace secl supplied surplus
    \item[{May contain}]
  
    \item[core: ]
   bibl biblStruct desc head label listBibl note p\par 
    \item[header: ]
   biblFull\par 
    \item[linking: ]
   ab\par 
    \item[msdescription: ]
   msDesc\par 
    \item[namesdates: ]
   trait\par 
    \item[textcrit: ]
   witDetail
    \item[{Note}]
  \par
Where there is confusion between <trait> and <state> the more general purpose element <state> should be used even for unchanging characteristics. If you wish to distinguish between characteristics that are generally perceived to be time-bound states and those assumed to be fixed traits, then <trait> is available for the more static of these. The <state> element encodes characteristics which are sometimes assumed to change, often at specific times or over a date range, whereas the <trait> elements are used to record characteristics, such as eye-colour, which are less subject to change. Traits are typically, but not necessarily, independent of the volition or action of the holder.
    \item[{Example}]
  \leavevmode\bgroup\exampleFont \begin{shaded}\noindent\mbox{}{<\textbf{trait}\hspace*{6pt}{type}="{physical}">}\mbox{}\newline 
\hspace*{6pt}{<\textbf{label}>}Eye colour{</\textbf{label}>}\mbox{}\newline 
\hspace*{6pt}{<\textbf{desc}>}Blue{</\textbf{desc}>}\mbox{}\newline 
{</\textbf{trait}>}\end{shaded}\egroup 


    \item[{Content model}]
  \mbox{}\hfill\\[-10pt]\begin{Verbatim}[fontsize=\small]
<content>
 <sequence>
  <elementRef key="precision"
   maxOccurs="unbounded" minOccurs="0"/>
  <alternate>
   <elementRef key="trait"
    maxOccurs="unbounded" minOccurs="1"/>
   <sequence>
    <classRef key="model.headLike"
     maxOccurs="unbounded" minOccurs="0"/>
    <classRef key="model.pLike"
     maxOccurs="unbounded" minOccurs="1"/>
    <alternate maxOccurs="unbounded"
     minOccurs="0">
     <classRef key="model.noteLike"/>
     <classRef key="model.biblLike"/>
    </alternate>
   </sequence>
   <alternate maxOccurs="unbounded"
    minOccurs="0">
    <classRef key="model.labelLike"/>
    <classRef key="model.noteLike"/>
    <classRef key="model.biblLike"/>
   </alternate>
  </alternate>
 </sequence>
</content>
    
\end{Verbatim}

    \item[{Schema Declaration}]
  \mbox{}\hfill\\[-10pt]\begin{Verbatim}[fontsize=\small]
element trait
{
   att.global.attributes,
   att.datable.attributes,
   att.editLike.attributes,
   att.naming.attributes,
   att.typed.attributes,
   (
      precision*,
      (
         trait+
       | (
            model.headLike*,
            model.pLike+,
            ( model.noteLike | model.biblLike )*
         )
       | ( model.labelLike | model.noteLike | model.biblLike )*
      )
   )
}
\end{Verbatim}

\end{reflist}  \index{transpose=<transpose>|oddindex}
\begin{reflist}
\item[]\begin{specHead}{TEI.transpose}{<transpose> }describes a single textual transposition as an ordered list of at least two pointers specifying the order in which the elements indicated should be re-combined. [\xref{http://www.tei-c.org/release/doc/tei-p5-doc/en/html/PH.html\#transpo}{11.3.4.5. Transpositions}]\end{specHead} 
    \item[{Module}]
  transcr
    \item[{Attributes}]
  Attributes att.global (\textit{@xml:id}, \textit{@n}, \textit{@xml:lang}, \textit{@xml:base}, \textit{@xml:space})  (att.global.rendition (\textit{@rend}, \textit{@style}, \textit{@rendition})) (att.global.linking (\textit{@corresp}, \textit{@synch}, \textit{@sameAs}, \textit{@copyOf}, \textit{@next}, \textit{@prev}, \textit{@exclude}, \textit{@select})) (att.global.analytic (\textit{@ana})) (att.global.facs (\textit{@facs})) (att.global.change (\textit{@change})) (att.global.responsibility (\textit{@cert}, \textit{@resp})) (att.global.source (\textit{@source}))
    \item[{Contained by}]
  
    \item[transcr: ]
   listTranspose
    \item[{May contain}]
  
    \item[core: ]
   ptr
    \item[{Note}]
  \par
Transposition is usually indicated in a document by a metamark such as a wavy line or numbering. \par
The order in which <ptr> elements appear within a <transpose> element should correspond with the desired order, as indicated by the metamark.
    \item[{Example}]
  \leavevmode\bgroup\exampleFont \begin{shaded}\noindent\mbox{}{<\textbf{transpose}>}\mbox{}\newline 
\hspace*{6pt}{<\textbf{ptr}\hspace*{6pt}{target}="{\#ib02}"/>}\mbox{}\newline 
\hspace*{6pt}{<\textbf{ptr}\hspace*{6pt}{target}="{\#ib01}"/>}\mbox{}\newline 
{</\textbf{transpose}>}\end{shaded}\egroup 

The transposition recorded here indicates that the content of the element with identifier \texttt{ib02} should appear before the content of the element with identifier \texttt{ib01}.
    \item[{Content model}]
  \mbox{}\hfill\\[-10pt]\begin{Verbatim}[fontsize=\small]
<content>
 <sequence>
  <elementRef key="ptr"/>
  <elementRef key="ptr"
   maxOccurs="unbounded" minOccurs="1"/>
 </sequence>
</content>
    
\end{Verbatim}

    \item[{Schema Declaration}]
  \mbox{}\hfill\\[-10pt]\begin{Verbatim}[fontsize=\small]
element transpose { att.global.attributes, ( ptr, ptr+ ) }
\end{Verbatim}

\end{reflist}  \index{typeDesc=<typeDesc>|oddindex}
\begin{reflist}
\item[]\begin{specHead}{TEI.typeDesc}{<typeDesc> }contains a description of the typefaces or other aspects of the printing of an incunable or other printed source. [\xref{http://www.tei-c.org/release/doc/tei-p5-doc/en/html/MS.html\#msphwr}{10.7.2.1. Writing}]\end{specHead} 
    \item[{Module}]
  msdescription
    \item[{Attributes}]
  Attributes att.global (\textit{@xml:id}, \textit{@n}, \textit{@xml:lang}, \textit{@xml:base}, \textit{@xml:space})  (att.global.rendition (\textit{@rend}, \textit{@style}, \textit{@rendition})) (att.global.linking (\textit{@corresp}, \textit{@synch}, \textit{@sameAs}, \textit{@copyOf}, \textit{@next}, \textit{@prev}, \textit{@exclude}, \textit{@select})) (att.global.analytic (\textit{@ana})) (att.global.facs (\textit{@facs})) (att.global.change (\textit{@change})) (att.global.responsibility (\textit{@cert}, \textit{@resp})) (att.global.source (\textit{@source}))
    \item[{Member of}]
  model.physDescPart
    \item[{Contained by}]
  
    \item[msdescription: ]
   physDesc
    \item[{May contain}]
  
    \item[core: ]
   p\par 
    \item[header: ]
   typeNote\par 
    \item[linking: ]
   ab\par 
    \item[msdescription: ]
   summary
    \item[{Example}]
  \leavevmode\bgroup\exampleFont \begin{shaded}\noindent\mbox{}{<\textbf{typeDesc}>}\mbox{}\newline 
\hspace*{6pt}{<\textbf{p}>}Uses an unidentified black letter font, probably from the\mbox{}\newline 
\hspace*{6pt}\hspace*{6pt} 15th century{</\textbf{p}>}\mbox{}\newline 
{</\textbf{typeDesc}>}\end{shaded}\egroup 


    \item[{Example}]
  \leavevmode\bgroup\exampleFont \begin{shaded}\noindent\mbox{}{<\textbf{typeDesc}>}\mbox{}\newline 
\hspace*{6pt}{<\textbf{summary}>}Contains a mixture of blackletter and Roman (antiqua) typefaces{</\textbf{summary}>}\mbox{}\newline 
\hspace*{6pt}{<\textbf{typeNote}\hspace*{6pt}{xml:id}="{Frak1}">}Blackletter face, showing\mbox{}\newline 
\hspace*{6pt}\hspace*{6pt} similarities to those produced in Wuerzburg after 1470.{</\textbf{typeNote}>}\mbox{}\newline 
\hspace*{6pt}{<\textbf{typeNote}\hspace*{6pt}{xml:id}="{Rom1}">}Roman face of Venetian origins.{</\textbf{typeNote}>}\mbox{}\newline 
{</\textbf{typeDesc}>}\end{shaded}\egroup 


    \item[{Content model}]
  \mbox{}\hfill\\[-10pt]\begin{Verbatim}[fontsize=\small]
<content>
 <alternate>
  <classRef key="model.pLike"
   maxOccurs="unbounded" minOccurs="1"/>
  <sequence>
   <elementRef key="summary" minOccurs="0"/>
   <elementRef key="typeNote"
    maxOccurs="unbounded" minOccurs="1"/>
  </sequence>
 </alternate>
</content>
    
\end{Verbatim}

    \item[{Schema Declaration}]
  \mbox{}\hfill\\[-10pt]\begin{Verbatim}[fontsize=\small]
element typeDesc
{
   att.global.attributes,
   ( model.pLike+ | ( summary?, typeNote+ ) )
}
\end{Verbatim}

\end{reflist}  \index{typeNote=<typeNote>|oddindex}
\begin{reflist}
\item[]\begin{specHead}{TEI.typeNote}{<typeNote> }describes a particular font or other significant typographic feature distinguished within the description of a printed resource. [\xref{http://www.tei-c.org/release/doc/tei-p5-doc/en/html/MS.html\#msph2}{10.7.2. Writing, Decoration, and Other Notations}]\end{specHead} 
    \item[{Module}]
  header
    \item[{Attributes}]
  Attributes att.global (\textit{@xml:id}, \textit{@n}, \textit{@xml:lang}, \textit{@xml:base}, \textit{@xml:space})  (att.global.rendition (\textit{@rend}, \textit{@style}, \textit{@rendition})) (att.global.linking (\textit{@corresp}, \textit{@synch}, \textit{@sameAs}, \textit{@copyOf}, \textit{@next}, \textit{@prev}, \textit{@exclude}, \textit{@select})) (att.global.analytic (\textit{@ana})) (att.global.facs (\textit{@facs})) (att.global.change (\textit{@change})) (att.global.responsibility (\textit{@cert}, \textit{@resp})) (att.global.source (\textit{@source})) att.handFeatures (\textit{@scribe}, \textit{@scribeRef}, \textit{@script}, \textit{@scriptRef}, \textit{@medium}, \textit{@scope}) 
    \item[{Contained by}]
  
    \item[msdescription: ]
   typeDesc
    \item[{May contain}]
  
    \item[analysis: ]
   c cl interp interpGrp m pc phr s span spanGrp w\par 
    \item[core: ]
   abbr add address bibl biblStruct cb choice cit corr date del desc distinct email emph expan foreign gap gb gloss graphic hi index l label lb lg list listBibl measure measureGrp media mentioned milestone name note num orig p pb ptr q quote ref reg rs said sic soCalled sp stage term time title unclear\par 
    \item[figures: ]
   figure formula notatedMusic table\par 
    \item[gaiji: ]
   g\par 
    \item[header: ]
   biblFull idno\par 
    \item[linking: ]
   ab alt altGrp anchor join joinGrp link linkGrp seg timeline\par 
    \item[msdescription: ]
   catchwords depth dim dimensions height heraldry locus locusGrp material msDesc objectType origDate origPlace secFol signatures stamp watermark width\par 
    \item[namesdates: ]
   addName affiliation bloc climate country district forename genName geo geogFeat geogName listEvent listNym listOrg listPerson listPlace location nameLink offset orgName persName placeName population region roleName settlement state surname terrain trait\par 
    \item[textcrit: ]
   app listApp listWit witDetail\par 
    \item[textstructure: ]
   floatingText\par 
    \item[transcr: ]
   addSpan am damage damageSpan delSpan ex fw handShift listTranspose metamark mod redo restore retrace secl space subst substJoin supplied surplus undo\par character data
    \item[{Example}]
  \leavevmode\bgroup\exampleFont \begin{shaded}\noindent\mbox{}{<\textbf{typeNote}\hspace*{6pt}{scope}="{sole}">} Printed in an Antiqua typeface showing strong Italianate influence.\mbox{}\newline 
{</\textbf{typeNote}>}\end{shaded}\egroup 


    \item[{Content model}]
  \mbox{}\hfill\\[-10pt]\begin{Verbatim}[fontsize=\small]
<content>
 <macroRef key="macro.specialPara"/>
</content>
    
\end{Verbatim}

    \item[{Schema Declaration}]
  \mbox{}\hfill\\[-10pt]\begin{Verbatim}[fontsize=\small]
element typeNote
{
   att.global.attributes,
   att.handFeatures.attributes,
   macro.specialPara}
\end{Verbatim}

\end{reflist}  \index{unclear=<unclear>|oddindex}\index{reason=@reason!<unclear>|oddindex}\index{hand=@hand!<unclear>|oddindex}\index{agent=@agent!<unclear>|oddindex}
\begin{reflist}
\item[]\begin{specHead}{TEI.unclear}{<unclear> }contains a word, phrase, or passage which cannot be transcribed with certainty because it is illegible or inaudible in the source. [\xref{http://www.tei-c.org/release/doc/tei-p5-doc/en/html/PH.html\#PHDA}{11.3.3.1. Damage, Illegibility, and Supplied Text} \xref{http://www.tei-c.org/release/doc/tei-p5-doc/en/html/CO.html\#COEDADD}{3.4.3. Additions, Deletions, and Omissions}]\end{specHead} 
    \item[{Module}]
  core
    \item[{Attributes}]
  Attributes att.global (\textit{@xml:id}, \textit{@n}, \textit{@xml:lang}, \textit{@xml:base}, \textit{@xml:space})  (att.global.rendition (\textit{@rend}, \textit{@style}, \textit{@rendition})) (att.global.linking (\textit{@corresp}, \textit{@synch}, \textit{@sameAs}, \textit{@copyOf}, \textit{@next}, \textit{@prev}, \textit{@exclude}, \textit{@select})) (att.global.analytic (\textit{@ana})) (att.global.facs (\textit{@facs})) (att.global.change (\textit{@change})) (att.global.responsibility (\textit{@cert}, \textit{@resp})) (att.global.source (\textit{@source})) att.editLike (\textit{@evidence}, \textit{@instant})  (att.dimensions (\textit{@unit}, \textit{@quantity}, \textit{@extent}, \textit{@precision}, \textit{@scope}) (att.ranging (\textit{@atLeast}, \textit{@atMost}, \textit{@min}, \textit{@max}, \textit{@confidence})) ) \hfil\\[-10pt]\begin{sansreflist}
    \item[@reason]
  indicates why the material is hard to transcribe.
\begin{reflist}
    \item[{Status}]
  Optional
    \item[{Datatype}]
  1–∞ occurrences of teidata.word separated by whitespace
    \item[]\exampleFont {<\textbf{div}>}\mbox{}\newline 
\hspace*{6pt}{<\textbf{head}>}Rx{</\textbf{head}>}\mbox{}\newline 
\hspace*{6pt}{<\textbf{p}>}500 mg {<\textbf{unclear}\hspace*{6pt}{reason}="{illegible}">}placebo{</\textbf{unclear}>}\mbox{}\newline 
\hspace*{6pt}{</\textbf{p}>}\mbox{}\newline 
{</\textbf{div}>}
    \item[{Note}]
  \par
One or more words may be used to describe the reason; usually each word will refer to a single cause. Typical examples might thus include \textit{faded}, \textit{illegible}, \textit{eccentric\textunderscore ductus} \textit{background\textunderscore noise}, \textit{passing\textunderscore truck},etc.
\end{reflist}  
    \item[@hand]
  Where the difficulty in transcription arises from action (partial deletion, etc.) assignable to an identifiable hand, signifies the hand responsible for the action.
\begin{reflist}
    \item[\xref{http://www.tei-c.org/Activities/Council/Working/tcw27.xml}{Deprecated}]
  will be removed on 2017-08-01
    \item[{Status}]
  Optional
    \item[{Datatype}]
  teidata.pointer
\end{reflist}  
    \item[@agent]
  Where the difficulty in transcription arises from damage, categorizes the cause of the damage, if it can be identified.
\begin{reflist}
    \item[{Status}]
  Optional
    \item[{Datatype}]
  teidata.enumerated
    \item[{Sample values include:}]
  \begin{description}

\item[{rubbing}]damage results from rubbing of the leaf edges
\item[{mildew}]damage results from mildew on the leaf surface
\item[{smoke}]damage results from smoke
\end{description} 
\end{reflist}  
\end{sansreflist}  
    \item[{Member of}]
  model.choicePart model.linePart model.pPart.transcriptional
    \item[{Contained by}]
  
    \item[analysis: ]
   cl pc phr s w\par 
    \item[core: ]
   abbr add addrLine author bibl biblScope choice citedRange corr date del distinct editor email emph expan foreign gloss head headItem headLabel hi item l label measure mentioned name note num orig p pubPlace publisher q quote ref reg rs said sic soCalled speaker stage street term textLang time title unclear\par 
    \item[figures: ]
   cell\par 
    \item[header: ]
   change distributor edition extent geoDecl handNote licence scriptNote typeNote\par 
    \item[linking: ]
   ab seg\par 
    \item[msdescription: ]
   accMat acquisition additions catchwords collation colophon condition custEvent decoNote explicit filiation finalRubric foliation heraldry incipit layout material musicNotation objectType origDate origPlace origin provenance rubric secFol signatures source stamp summary support surrogates watermark\par 
    \item[namesdates: ]
   addName affiliation birth bloc country death district education faith floruit forename genName geogFeat geogName nameLink nationality occupation offset orgName persName placeName region residence roleName settlement sex socecStatus surname\par 
    \item[textcrit: ]
   lem rdg wit witDetail\par 
    \item[textstructure: ]
   byline closer dateline docAuthor docDate docEdition docImprint imprimatur opener salute signed titlePart trailer\par 
    \item[transcr: ]
   am damage fw line metamark mod restore retrace secl supplied surplus zone
    \item[{May contain}]
  
    \item[analysis: ]
   c cl interp interpGrp m pc phr s span spanGrp w\par 
    \item[core: ]
   abbr add address bibl biblStruct cb choice cit corr date del desc distinct email emph expan foreign gap gb gloss graphic hi index l label lb lg list listBibl measure measureGrp media mentioned milestone name note num orig pb ptr q quote ref reg rs said sic soCalled stage term time title unclear\par 
    \item[figures: ]
   figure formula notatedMusic table\par 
    \item[gaiji: ]
   g\par 
    \item[header: ]
   biblFull idno\par 
    \item[linking: ]
   alt altGrp anchor join joinGrp link linkGrp seg timeline\par 
    \item[msdescription: ]
   catchwords depth dim dimensions height heraldry locus locusGrp material msDesc objectType origDate origPlace secFol signatures stamp watermark width\par 
    \item[namesdates: ]
   addName affiliation bloc climate country district forename genName geo geogFeat geogName listEvent listNym listOrg listPerson listPlace location nameLink offset orgName persName placeName population region roleName settlement state surname terrain trait\par 
    \item[textcrit: ]
   app listApp listWit witDetail\par 
    \item[textstructure: ]
   floatingText\par 
    \item[transcr: ]
   addSpan am damage damageSpan delSpan ex fw handShift listTranspose metamark mod redo restore retrace secl space subst substJoin supplied surplus undo\par character data
    \item[{Note}]
  \par
The same element is used for all cases of uncertainty in the transcription of element content, whether for written or spoken material. For other aspects of certainty, uncertainty, and reliability of tagging and transcription, see chapter \xref{http://www.tei-c.org/release/doc/tei-p5-doc/en/html/CE.html\#CE}{21. Certainty, Precision, and Responsibility}.\par
The <damage>, <gap>, <del>, <unclear> and <supplied> elements may be closely allied in use. See section \xref{http://www.tei-c.org/release/doc/tei-p5-doc/en/html/PH.html\#PHCOMB}{11.3.3.2. Use of the gap, del, damage, unclear, and supplied Elements in Combination} for discussion of which element is appropriate for which circumstance.\par
The {\itshape hand} attribute points to a definition of the hand concerned, as further discussed in section \xref{http://www.tei-c.org/release/doc/tei-p5-doc/en/html/PH.html\#PHDH}{11.3.2.1. Document Hands}.
    \item[{Example}]
  \leavevmode\bgroup\exampleFont \begin{shaded}\noindent\mbox{}{<\textbf{u}>} ...and then {<\textbf{unclear}\hspace*{6pt}{reason}="{background-noise}">}Nathalie{</\textbf{unclear}>} said ... {</\textbf{u}>}\end{shaded}\egroup 


    \item[{Content model}]
  \mbox{}\hfill\\[-10pt]\begin{Verbatim}[fontsize=\small]
<content>
 <macroRef key="macro.paraContent"/>
</content>
    
\end{Verbatim}

    \item[{Schema Declaration}]
  \mbox{}\hfill\\[-10pt]\begin{Verbatim}[fontsize=\small]
element unclear
{
   att.global.attributes,
   att.editLike.attributes,
   attribute reason { list { + } }?,
   attribute hand { text }?,
   attribute agent { text }?,
   macro.paraContent}
\end{Verbatim}

\end{reflist}  \index{undo=<undo>|oddindex}\index{target=@target!<undo>|oddindex}
\begin{reflist}
\item[]\begin{specHead}{TEI.undo}{<undo> }indicates one or more marked-up interventions in a document which have subsequently been marked for cancellation. [\xref{http://www.tei-c.org/release/doc/tei-p5-doc/en/html/PH.html\#undo}{11.3.4.4. Confirmation, Cancellation, and Reinstatement of Modifications}]\end{specHead} 
    \item[{Module}]
  transcr
    \item[{Attributes}]
  Attributes att.global (\textit{@xml:id}, \textit{@n}, \textit{@xml:lang}, \textit{@xml:base}, \textit{@xml:space})  (att.global.rendition (\textit{@rend}, \textit{@style}, \textit{@rendition})) (att.global.linking (\textit{@corresp}, \textit{@synch}, \textit{@sameAs}, \textit{@copyOf}, \textit{@next}, \textit{@prev}, \textit{@exclude}, \textit{@select})) (att.global.analytic (\textit{@ana})) (att.global.facs (\textit{@facs})) (att.global.change (\textit{@change})) (att.global.responsibility (\textit{@cert}, \textit{@resp})) (att.global.source (\textit{@source})) att.spanning (\textit{@spanTo}) att.transcriptional (\textit{@status}, \textit{@cause}, \textit{@seq})  (att.editLike (\textit{@evidence}, \textit{@instant}) (att.dimensions (\textit{@unit}, \textit{@quantity}, \textit{@extent}, \textit{@precision}, \textit{@scope}) (att.ranging (\textit{@atLeast}, \textit{@atMost}, \textit{@min}, \textit{@max}, \textit{@confidence})) ) ) (att.written (\textit{@hand})) \hfil\\[-10pt]\begin{sansreflist}
    \item[@target]
  points to one or more elements representing the interventions which are to be reverted or undone.
\begin{reflist}
    \item[{Status}]
  Optional
    \item[{Datatype}]
  1–∞ occurrences of teidata.pointer separated by whitespace
\end{reflist}  
\end{sansreflist}  
    \item[{Member of}]
  model.linePart model.pPart.transcriptional
    \item[{Contained by}]
  
    \item[analysis: ]
   cl pc phr s w\par 
    \item[core: ]
   abbr add addrLine author bibl biblScope citedRange corr date del distinct editor email emph expan foreign gloss head headItem headLabel hi item l label measure mentioned name note num orig p pubPlace publisher q quote ref reg rs said sic soCalled speaker stage street term textLang time title unclear\par 
    \item[figures: ]
   cell\par 
    \item[header: ]
   change distributor edition extent geoDecl handNote licence scriptNote typeNote\par 
    \item[linking: ]
   ab seg\par 
    \item[msdescription: ]
   accMat acquisition additions catchwords collation colophon condition custEvent decoNote explicit filiation finalRubric foliation heraldry incipit layout material musicNotation objectType origDate origPlace origin provenance rubric secFol signatures source stamp summary support surrogates watermark\par 
    \item[namesdates: ]
   addName affiliation birth bloc country death district education faith floruit forename genName geogFeat geogName nameLink nationality occupation offset orgName persName placeName region residence roleName settlement sex socecStatus surname\par 
    \item[textcrit: ]
   lem rdg wit witDetail\par 
    \item[textstructure: ]
   byline closer dateline docAuthor docDate docEdition docImprint imprimatur opener salute signed titlePart trailer\par 
    \item[transcr: ]
   am damage fw line metamark mod restore retrace secl supplied surplus zone
    \item[{May contain}]
  Empty element
    \item[{Example}]
  \leavevmode\bgroup\exampleFont \begin{shaded}\noindent\mbox{}{<\textbf{line}>}This is {<\textbf{del}\hspace*{6pt}{change}="{\#s2}"\hspace*{6pt}{rend}="{overstrike}">}\mbox{}\newline 
\hspace*{6pt}\hspace*{6pt}{<\textbf{seg}\hspace*{6pt}{xml:id}="{undo-a}">}just some{</\textbf{seg}>}\mbox{}\newline 
\hspace*{6pt}\hspace*{6pt} sample {<\textbf{seg}\hspace*{6pt}{xml:id}="{undo-b}">}text{</\textbf{seg}>},\mbox{}\newline 
\hspace*{6pt}\hspace*{6pt} we need{</\textbf{del}>}\mbox{}\newline 
\hspace*{6pt}{<\textbf{add}\hspace*{6pt}{change}="{\#s2}">}not{</\textbf{add}>}\mbox{}\newline 
 a real example.{</\textbf{line}>}\mbox{}\newline 
{<\textbf{undo}\hspace*{6pt}{change}="{\#s3}"\hspace*{6pt}{rend}="{dotted}"\mbox{}\newline 
\hspace*{6pt}{target}="{\#undo-a \#undo-b}"/>}\end{shaded}\egroup 

This encoding represents the following sequence of events: \begin{itemize}
\item "This is just some sample text, we need a real example" is written
\item At stage s2, "just some sample text, we need" is deleted by overstriking, and "not" is added 
\item At stage s3, parts of the deletion are cancelled by underdotting, thus reinstating the words "just some" and "text".
\end{itemize} 
    \item[{Content model}]
  \fbox{\ttfamily <content>\newline
</content>\newline
    } 
    \item[{Schema Declaration}]
  \mbox{}\hfill\\[-10pt]\begin{Verbatim}[fontsize=\small]
element undo
{
   att.global.attributes,
   att.spanning.attributes,
   att.transcriptional.attributes,
   attribute target { list { + } }?,
   empty
}
\end{Verbatim}

\end{reflist}  \index{unicodeName=<unicodeName>|oddindex}\index{version=@version!<unicodeName>|oddindex}
\begin{reflist}
\item[]\begin{specHead}{TEI.unicodeName}{<unicodeName> }(unicode property name) contains the name of a registered Unicode normative or informative property. [\xref{http://www.tei-c.org/release/doc/tei-p5-doc/en/html/WD.html\#ucsprops}{5.2.1. Character Properties}]\end{specHead} 
    \item[{Module}]
  gaiji
    \item[{Attributes}]
  Attributes att.global (\textit{@xml:id}, \textit{@n}, \textit{@xml:lang}, \textit{@xml:base}, \textit{@xml:space})  (att.global.rendition (\textit{@rend}, \textit{@style}, \textit{@rendition})) (att.global.linking (\textit{@corresp}, \textit{@synch}, \textit{@sameAs}, \textit{@copyOf}, \textit{@next}, \textit{@prev}, \textit{@exclude}, \textit{@select})) (att.global.analytic (\textit{@ana})) (att.global.facs (\textit{@facs})) (att.global.change (\textit{@change})) (att.global.responsibility (\textit{@cert}, \textit{@resp})) (att.global.source (\textit{@source})) \hfil\\[-10pt]\begin{sansreflist}
    \item[@version]
  specifies the version number of the Unicode Standard in which this property name is defined.
\begin{reflist}
    \item[{Status}]
  Optional
    \item[{Datatype}]
  teidata.version
\end{reflist}  
\end{sansreflist}  
    \item[{Contained by}]
  
    \item[gaiji: ]
   charProp
    \item[{May contain}]
  Character data only
    \item[{Note}]
  \par
A definitive list of current Unicode property names is provided in The Unicode Standard.
    \item[{Example}]
  \leavevmode\bgroup\exampleFont \begin{shaded}\noindent\mbox{}{<\textbf{unicodeName}>}character-decomposition-mapping{</\textbf{unicodeName}>}\mbox{}\newline 
{<\textbf{unicodeName}>}general-category{</\textbf{unicodeName}>}\end{shaded}\egroup 


    \item[{Content model}]
  \fbox{\ttfamily <content>\newline
 <textNode/>\newline
</content>\newline
    } 
    \item[{Schema Declaration}]
  \mbox{}\hfill\\[-10pt]\begin{Verbatim}[fontsize=\small]
element unicodeName
{
   att.global.attributes,
   attribute version { text }?,
   text
}
\end{Verbatim}

\end{reflist}  \index{value=<value>|oddindex}
\begin{reflist}
\item[]\begin{specHead}{TEI.value}{<value> }contains a single value for some property, attribute, or other analysis. [\xref{http://www.tei-c.org/release/doc/tei-p5-doc/en/html/WD.html\#ucsprops}{5.2.1. Character Properties}]\end{specHead} 
    \item[{Module}]
  gaiji
    \item[{Attributes}]
  Attributes att.global (\textit{@xml:id}, \textit{@n}, \textit{@xml:lang}, \textit{@xml:base}, \textit{@xml:space})  (att.global.rendition (\textit{@rend}, \textit{@style}, \textit{@rendition})) (att.global.linking (\textit{@corresp}, \textit{@synch}, \textit{@sameAs}, \textit{@copyOf}, \textit{@next}, \textit{@prev}, \textit{@exclude}, \textit{@select})) (att.global.analytic (\textit{@ana})) (att.global.facs (\textit{@facs})) (att.global.change (\textit{@change})) (att.global.responsibility (\textit{@cert}, \textit{@resp})) (att.global.source (\textit{@source}))
    \item[{Contained by}]
  
    \item[gaiji: ]
   charProp
    \item[{May contain}]
  
    \item[gaiji: ]
   g\par character data
    \item[{Example}]
  \leavevmode\bgroup\exampleFont \begin{shaded}\noindent\mbox{}{<\textbf{value}>}unknown{</\textbf{value}>}\end{shaded}\egroup 


    \item[{Content model}]
  \fbox{\ttfamily <content>\newline
 <macroRef key="macro.xtext"/>\newline
</content>\newline
    } 
    \item[{Schema Declaration}]
  \mbox{}\hfill\\[-10pt]\begin{Verbatim}[fontsize=\small]
element value { att.global.attributes, macro.xtext }
\end{Verbatim}

\end{reflist}  \index{variantEncoding=<variantEncoding>|oddindex}\index{method=@method!<variantEncoding>|oddindex}\index{location=@location!<variantEncoding>|oddindex}
\begin{reflist}
\item[]\begin{specHead}{TEI.variantEncoding}{<variantEncoding> }declares the method used to encode text-critical variants. [\xref{http://www.tei-c.org/release/doc/tei-p5-doc/en/html/TC.html\#TCAPEN}{12.1.1. The Apparatus Entry}]\end{specHead} 
    \item[{Module}]
  textcrit
    \item[{Attributes}]
  Attributes att.global (\textit{@xml:id}, \textit{@n}, \textit{@xml:lang}, \textit{@xml:base}, \textit{@xml:space})  (att.global.rendition (\textit{@rend}, \textit{@style}, \textit{@rendition})) (att.global.linking (\textit{@corresp}, \textit{@synch}, \textit{@sameAs}, \textit{@copyOf}, \textit{@next}, \textit{@prev}, \textit{@exclude}, \textit{@select})) (att.global.analytic (\textit{@ana})) (att.global.facs (\textit{@facs})) (att.global.change (\textit{@change})) (att.global.responsibility (\textit{@cert}, \textit{@resp})) (att.global.source (\textit{@source})) \hfil\\[-10pt]\begin{sansreflist}
    \item[@method]
  indicates which method is used to encode the apparatus of variants.
\begin{reflist}
    \item[{Status}]
  Required
    \item[{Datatype}]
  teidata.enumerated
    \item[{Legal values are:}]
  \begin{description}

\item[{location-referenced}]apparatus uses line numbers or other canonical reference scheme referenced in a base text.
\item[{double-end-point}]apparatus indicates the precise locations of the beginning and ending of each lemma relative to a base text.
\item[{parallel-segmentation}]alternate readings of a passage are given in parallel in the text; no notion of a base text is necessary.
\end{description} 
    \item[{Note}]
  \par
The value ‘parallel-segmentation’ requires in-line encoding of the apparatus.
\end{reflist}  
    \item[@location]
  indicates whether the apparatus appears within the running text or external to it.
\begin{reflist}
    \item[{Status}]
  Required
    \item[{Datatype}]
  teidata.enumerated
    \item[{Schematron}]
   <sch:rule context="tei:variantEncoding"> <sch:assert test="(@location != 'external') or (@method != 'parallel-segmentation')"> The @location value "external" is inconsistent with the  parallel-segmentation method of apparatus markup.</sch:assert> </sch:rule>
    \item[{Legal values are:}]
  \begin{description}

\item[{internal}]apparatus appears within the running text.
\item[{external}]apparatus appears outside the base text.
\end{description} 
    \item[{Note}]
  \par
The value ‘external’ is inconsistent with the parallel-segmentation method of apparatus markup.
\end{reflist}  
\end{sansreflist}  
    \item[{Member of}]
  model.encodingDescPart
    \item[{Contained by}]
  
    \item[header: ]
   encodingDesc
    \item[{May contain}]
  Empty element
    \item[{Example}]
  \leavevmode\bgroup\exampleFont \begin{shaded}\noindent\mbox{}{<\textbf{variantEncoding}\hspace*{6pt}{location}="{external}"\mbox{}\newline 
\hspace*{6pt}{method}="{location-referenced}"/>}\end{shaded}\egroup 


    \item[{Content model}]
  \fbox{\ttfamily <content>\newline
</content>\newline
    } 
    \item[{Schema Declaration}]
  \mbox{}\hfill\\[-10pt]\begin{Verbatim}[fontsize=\small]
element variantEncoding
{
   att.global.attributes,
   attribute method
   {
      "location-referenced" | "double-end-point" | "parallel-segmentation"
   },
   attribute location { "internal" | "external" }
   >>
   tei:constraintSpec
   [
      ident = "variantEncodingLocation"
      scheme = "isoschematron"
      " "
      """ The @location value "external" is inconsistent with the parallel-segmentation method of apparatus markup."""
   ],
   empty
}
\end{Verbatim}

\end{reflist}  \index{w=<w>|oddindex}\index{lemma=@lemma!<w>|oddindex}\index{lemmaRef=@lemmaRef!<w>|oddindex}
\begin{reflist}
\item[]\begin{specHead}{TEI.w}{<w> }(word) represents a grammatical (not necessarily orthographic) word. [\xref{http://www.tei-c.org/release/doc/tei-p5-doc/en/html/AI.html\#AILC}{17.1. Linguistic Segment Categories}]\end{specHead} 
    \item[{Module}]
  analysis
    \item[{Attributes}]
  Attributes att.global (\textit{@xml:id}, \textit{@n}, \textit{@xml:lang}, \textit{@xml:base}, \textit{@xml:space})  (att.global.rendition (\textit{@rend}, \textit{@style}, \textit{@rendition})) (att.global.linking (\textit{@corresp}, \textit{@synch}, \textit{@sameAs}, \textit{@copyOf}, \textit{@next}, \textit{@prev}, \textit{@exclude}, \textit{@select})) (att.global.analytic (\textit{@ana})) (att.global.facs (\textit{@facs})) (att.global.change (\textit{@change})) (att.global.responsibility (\textit{@cert}, \textit{@resp})) (att.global.source (\textit{@source})) att.segLike (\textit{@function})  (att.datcat (\textit{@datcat}, \textit{@valueDatcat})) (att.fragmentable (\textit{@part})) att.typed (\textit{@type}, \textit{@subtype}) \hfil\\[-10pt]\begin{sansreflist}
    \item[@lemma]
  provides a lemma for the word, such as an uninflected dictionary entry form.
\begin{reflist}
    \item[{Status}]
  Optional
    \item[{Datatype}]
  teidata.text
\end{reflist}  
    \item[@lemmaRef]
  provides a pointer to a definition of the lemma for the word, for example in an online lexicon.
\begin{reflist}
    \item[{Status}]
  Optional
    \item[{Datatype}]
  teidata.pointer
\end{reflist}  
\end{sansreflist}  
    \item[{Member of}]
  model.linePart model.segLike 
    \item[{Contained by}]
  
    \item[analysis: ]
   cl phr s w\par 
    \item[core: ]
   abbr add addrLine author bibl biblScope citedRange corr date del distinct editor email emph expan foreign gloss head headItem headLabel hi item l label measure mentioned name note num orig p pubPlace publisher q quote ref reg rs said sic soCalled speaker stage street term textLang time title unclear\par 
    \item[figures: ]
   cell\par 
    \item[header: ]
   change distributor edition extent geoDecl handNote licence scriptNote typeNote\par 
    \item[linking: ]
   ab seg\par 
    \item[msdescription: ]
   accMat acquisition additions catchwords collation colophon condition custEvent decoNote explicit filiation finalRubric foliation heraldry incipit layout material musicNotation objectType origDate origPlace origin provenance rubric secFol signatures source stamp summary support surrogates watermark\par 
    \item[namesdates: ]
   addName affiliation birth bloc country death district education faith floruit forename genName geogFeat geogName nameLink nationality occupation offset orgName persName placeName region residence roleName settlement sex socecStatus surname\par 
    \item[textcrit: ]
   lem rdg wit witDetail\par 
    \item[textstructure: ]
   byline closer dateline docAuthor docDate docEdition docImprint imprimatur opener salute signed titlePart trailer\par 
    \item[transcr: ]
   damage fw line metamark mod restore retrace secl supplied surplus zone
    \item[{May contain}]
  
    \item[analysis: ]
   c interp interpGrp m pc span spanGrp w\par 
    \item[core: ]
   abbr add cb choice corr del expan gap gb hi index lb milestone note orig pb reg sic unclear\par 
    \item[figures: ]
   figure notatedMusic\par 
    \item[gaiji: ]
   g\par 
    \item[linking: ]
   alt altGrp anchor join joinGrp link linkGrp seg timeline\par 
    \item[textcrit: ]
   app witDetail\par 
    \item[transcr: ]
   addSpan am damage damageSpan delSpan ex fw handShift listTranspose metamark mod redo restore retrace secl space subst substJoin supplied surplus undo\par character data
    \item[{Example}]
  \leavevmode\bgroup\exampleFont \begin{shaded}\noindent\mbox{}{<\textbf{w}\hspace*{6pt}{lemma}="{hit}"\mbox{}\newline 
\hspace*{6pt}{lemmaRef}="{http://www.example.com/lexicon/hitvb.xml}"\hspace*{6pt}{type}="{verb}">}hitt{<\textbf{m}\hspace*{6pt}{type}="{suffix}">}ing{</\textbf{m}>}\mbox{}\newline 
{</\textbf{w}>}\end{shaded}\egroup 


    \item[{Content model}]
  \mbox{}\hfill\\[-10pt]\begin{Verbatim}[fontsize=\small]
<content>
 <alternate maxOccurs="unbounded"
  minOccurs="0">
  <textNode/>
  <classRef key="model.gLike"/>
  <elementRef key="seg"/>
  <elementRef key="w"/>
  <elementRef key="m"/>
  <elementRef key="c"/>
  <elementRef key="pc"/>
  <classRef key="model.global"/>
  <classRef key="model.lPart"/>
  <classRef key="model.hiLike"/>
  <classRef key="model.pPart.edit"/>
 </alternate>
</content>
    
\end{Verbatim}

    \item[{Schema Declaration}]
  \mbox{}\hfill\\[-10pt]\begin{Verbatim}[fontsize=\small]
element w
{
   att.global.attributes,
   att.segLike.attributes,
   att.typed.attributes,
   attribute lemma { text }?,
   attribute lemmaRef { text }?,
   (
      text
    | model.gLike    | seg    | w    | m    | c    | pc    | model.global    | model.lPart    | model.hiLike    | model.pPart.edit   )*
}
\end{Verbatim}

\end{reflist}  \index{watermark=<watermark>|oddindex}
\begin{reflist}
\item[]\begin{specHead}{TEI.watermark}{<watermark> }contains a word or phrase describing a watermark or similar device. [\xref{http://www.tei-c.org/release/doc/tei-p5-doc/en/html/MS.html\#mswat}{10.3.3. Watermarks and Stamps}]\end{specHead} 
    \item[{Module}]
  msdescription
    \item[{Attributes}]
  Attributes att.global (\textit{@xml:id}, \textit{@n}, \textit{@xml:lang}, \textit{@xml:base}, \textit{@xml:space})  (att.global.rendition (\textit{@rend}, \textit{@style}, \textit{@rendition})) (att.global.linking (\textit{@corresp}, \textit{@synch}, \textit{@sameAs}, \textit{@copyOf}, \textit{@next}, \textit{@prev}, \textit{@exclude}, \textit{@select})) (att.global.analytic (\textit{@ana})) (att.global.facs (\textit{@facs})) (att.global.change (\textit{@change})) (att.global.responsibility (\textit{@cert}, \textit{@resp})) (att.global.source (\textit{@source}))
    \item[{Member of}]
  model.pPart.msdesc
    \item[{Contained by}]
  
    \item[analysis: ]
   cl phr s span\par 
    \item[core: ]
   abbr add addrLine author biblScope citedRange corr date del desc distinct editor email emph expan foreign gloss head headItem headLabel hi item l label measure meeting mentioned name note num orig p pubPlace publisher q quote ref reg resp rs said sic soCalled speaker stage street term textLang time title unclear\par 
    \item[figures: ]
   cell figDesc\par 
    \item[header: ]
   authority catDesc change classCode creation distributor edition extent funder geoDecl handNote language licence principal rendition scriptNote sponsor tagUsage typeNote\par 
    \item[linking: ]
   ab seg\par 
    \item[msdescription: ]
   accMat acquisition additions catchwords collation colophon condition custEvent decoNote explicit filiation finalRubric foliation heraldry incipit layout material musicNotation objectType origDate origPlace origin provenance rubric secFol signatures source stamp summary support surrogates watermark\par 
    \item[namesdates: ]
   addName affiliation age birth bloc country death district education faith floruit forename genName geogFeat geogName langKnown nameLink nationality occupation offset orgName persName placeName region residence roleName settlement sex socecStatus surname\par 
    \item[textcrit: ]
   lem rdg wit witDetail witness\par 
    \item[textstructure: ]
   byline closer dateline docAuthor docDate docEdition docImprint imprimatur opener salute signed titlePart trailer\par 
    \item[transcr: ]
   damage fw metamark mod restore retrace secl supplied surplus
    \item[{May contain}]
  
    \item[analysis: ]
   c cl interp interpGrp m pc phr s span spanGrp w\par 
    \item[core: ]
   abbr add address cb choice corr date del distinct email emph expan foreign gap gb gloss graphic hi index lb measure measureGrp media mentioned milestone name note num orig pb ptr ref reg rs sic soCalled term time title unclear\par 
    \item[figures: ]
   figure formula notatedMusic\par 
    \item[gaiji: ]
   g\par 
    \item[header: ]
   idno\par 
    \item[linking: ]
   alt altGrp anchor join joinGrp link linkGrp seg timeline\par 
    \item[msdescription: ]
   catchwords depth dim dimensions height heraldry locus locusGrp material objectType origDate origPlace secFol signatures stamp watermark width\par 
    \item[namesdates: ]
   addName affiliation bloc climate country district forename genName geo geogFeat geogName location nameLink offset orgName persName placeName population region roleName settlement state surname terrain trait\par 
    \item[textcrit: ]
   app witDetail\par 
    \item[transcr: ]
   addSpan am damage damageSpan delSpan ex fw handShift listTranspose metamark mod redo restore retrace secl space subst substJoin supplied surplus undo\par character data
    \item[{Example}]
  \leavevmode\bgroup\exampleFont \begin{shaded}\noindent\mbox{}{<\textbf{support}>}\mbox{}\newline 
\hspace*{6pt}{<\textbf{p}>}\mbox{}\newline 
\hspace*{6pt}\hspace*{6pt}{<\textbf{material}>}Rag paper{</\textbf{material}>} with {<\textbf{watermark}>}anchor{</\textbf{watermark}>} watermark{</\textbf{p}>}\mbox{}\newline 
{</\textbf{support}>}\end{shaded}\egroup 


    \item[{Content model}]
  \mbox{}\hfill\\[-10pt]\begin{Verbatim}[fontsize=\small]
<content>
 <macroRef key="macro.phraseSeq"/>
</content>
    
\end{Verbatim}

    \item[{Schema Declaration}]
  \mbox{}\hfill\\[-10pt]\begin{Verbatim}[fontsize=\small]
element watermark { att.global.attributes, macro.phraseSeq }
\end{Verbatim}

\end{reflist}  \index{when=<when>|oddindex}\index{absolute=@absolute!<when>|oddindex}\index{unit=@unit!<when>|oddindex}\index{interval=@interval!<when>|oddindex}\index{since=@since!<when>|oddindex}
\begin{reflist}
\item[]\begin{specHead}{TEI.when}{<when> }indicates a point in time either relative to other elements in the same timeline tag, or absolutely. [\xref{http://www.tei-c.org/release/doc/tei-p5-doc/en/html/SA.html\#SASYMP}{16.4.2. Placing Synchronous Events in Time}]\end{specHead} 
    \item[{Module}]
  linking
    \item[{Attributes}]
  Attributes att.global (\textit{@xml:id}, \textit{@n}, \textit{@xml:lang}, \textit{@xml:base}, \textit{@xml:space})  (att.global.rendition (\textit{@rend}, \textit{@style}, \textit{@rendition})) (att.global.linking (\textit{@corresp}, \textit{@synch}, \textit{@sameAs}, \textit{@copyOf}, \textit{@next}, \textit{@prev}, \textit{@exclude}, \textit{@select})) (att.global.analytic (\textit{@ana})) (att.global.facs (\textit{@facs})) (att.global.change (\textit{@change})) (att.global.responsibility (\textit{@cert}, \textit{@resp})) (att.global.source (\textit{@source})) \hfil\\[-10pt]\begin{sansreflist}
    \item[@absolute]
  supplies an absolute value for the time.
\begin{reflist}
    \item[{Status}]
  Optional
    \item[{Datatype}]
  teidata.temporal.w3c
    \item[{Note}]
  \par
This attribute should always be specified on a <when> element which serves as the target for the {\itshape origin} attribute of a \texttt{<timeLine>}.
\end{reflist}  
    \item[@unit]
  specifies the unit of time in which the {\itshape interval} value is expressed, if this is not inherited from the parent <timeline>.
\begin{reflist}
    \item[{Status}]
  Optional
    \item[{Datatype}]
  teidata.enumerated
    \item[{Suggested values include:}]
  \begin{description}

\item[{d}](days)
\item[{h}](hours)
\item[{min}](minutes)
\item[{s}](seconds)
\item[{ms}](milliseconds)
\end{description} 
\end{reflist}  
    \item[@interval]
  specifies a time interval either as a number or as one of the keywords defined by the datatype data.interval
\begin{reflist}
    \item[{Status}]
  Optional
    \item[{Datatype}]
  teidata.interval
\end{reflist}  
    \item[@since]
  identifies the reference point for determining the time of the current <when> element, which is obtained by adding the interval to the time of the reference point.
\begin{reflist}
    \item[{Status}]
  Optional
    \item[{Datatype}]
  teidata.pointer
    \item[{Note}]
  \par
This attribute should point to another <when> element in the same <timeline>. If no value is supplied, and the {\itshape absolute} attribute is also unspecified, then the reference point is understood to be the origin of the enclosing <timeline> tag.
\end{reflist}  
\end{sansreflist}  
    \item[{Contained by}]
  
    \item[linking: ]
   timeline
    \item[{May contain}]
  Empty element
    \item[{Note}]
  \par
On this element, the global {\itshape xml:id} attribute must be supplied to specify an identifier for this point in time. The value used may be chosen freely provided that it is unique within the document and is a syntactically valid name. There is no requirement for values containing numbers to be in sequence.
    \item[{Example}]
  \leavevmode\bgroup\exampleFont \begin{shaded}\noindent\mbox{}{<\textbf{when}\hspace*{6pt}{interval}="{20}"\hspace*{6pt}{since}="{\#w2}"\hspace*{6pt}{xml:id}="{TW3}"/>}\end{shaded}\egroup 


    \item[{Content model}]
  \fbox{\ttfamily <content>\newline
</content>\newline
    } 
    \item[{Schema Declaration}]
  \mbox{}\hfill\\[-10pt]\begin{Verbatim}[fontsize=\small]
element when
{
   att.global.attributes,
   attribute absolute { text }?,
   attribute unit { "d" | "h" | "min" | "s" | "ms" }?,
   attribute interval { text }?,
   attribute since { text }?,
   empty
}
\end{Verbatim}

\end{reflist}  \index{width=<width>|oddindex}
\begin{reflist}
\item[]\begin{specHead}{TEI.width}{<width> }contains a measurement measured along the axis parallel to the bottom of the written surface, i.e. perpendicular to the spine of a book or codex. [\xref{http://www.tei-c.org/release/doc/tei-p5-doc/en/html/MS.html\#msdim}{10.3.4. Dimensions}]\end{specHead} 
    \item[{Module}]
  msdescription
    \item[{Attributes}]
  Attributes att.global (\textit{@xml:id}, \textit{@n}, \textit{@xml:lang}, \textit{@xml:base}, \textit{@xml:space})  (att.global.rendition (\textit{@rend}, \textit{@style}, \textit{@rendition})) (att.global.linking (\textit{@corresp}, \textit{@synch}, \textit{@sameAs}, \textit{@copyOf}, \textit{@next}, \textit{@prev}, \textit{@exclude}, \textit{@select})) (att.global.analytic (\textit{@ana})) (att.global.facs (\textit{@facs})) (att.global.change (\textit{@change})) (att.global.responsibility (\textit{@cert}, \textit{@resp})) (att.global.source (\textit{@source})) att.dimensions (\textit{@unit}, \textit{@quantity}, \textit{@extent}, \textit{@precision}, \textit{@scope})  (att.ranging (\textit{@atLeast}, \textit{@atMost}, \textit{@min}, \textit{@max}, \textit{@confidence}))
    \item[{Member of}]
  model.dimLike model.measureLike
    \item[{Contained by}]
  
    \item[analysis: ]
   cl phr s span\par 
    \item[core: ]
   abbr add addrLine author bibl biblScope citedRange corr date del desc distinct editor email emph expan foreign gloss head headItem headLabel hi item l label measure measureGrp meeting mentioned name note num orig p pubPlace publisher q quote ref reg resp rs said sic soCalled speaker stage street term textLang time title unclear\par 
    \item[figures: ]
   cell figDesc\par 
    \item[header: ]
   authority catDesc change classCode creation distributor edition extent funder geoDecl handNote language licence principal rendition scriptNote sponsor tagUsage typeNote\par 
    \item[linking: ]
   ab seg\par 
    \item[msdescription: ]
   accMat acquisition additions catchwords collation colophon condition custEvent decoNote dimensions explicit filiation finalRubric foliation heraldry incipit layout material musicNotation objectType origDate origPlace origin provenance rubric secFol signatures source stamp summary support surrogates watermark\par 
    \item[namesdates: ]
   addName affiliation age birth bloc country death district education faith floruit forename genName geogFeat geogName langKnown location nameLink nationality occupation offset orgName persName placeName region residence roleName settlement sex socecStatus surname\par 
    \item[textcrit: ]
   lem rdg wit witDetail witness\par 
    \item[textstructure: ]
   byline closer dateline docAuthor docDate docEdition docImprint imprimatur opener salute signed titlePart trailer\par 
    \item[transcr: ]
   damage fw metamark mod restore retrace secl supplied surplus
    \item[{May contain}]
  
    \item[gaiji: ]
   g\par character data
    \item[{Note}]
  \par
If used to specify the depth of a non text-bearing portion of some object, for example a monument, this element conventionally refers to the axis facing the observer, and perpendicular to that indicated by the ‘depth’ axis.
    \item[{Example}]
  \leavevmode\bgroup\exampleFont \begin{shaded}\noindent\mbox{}{<\textbf{width}\hspace*{6pt}{unit}="{in}">}4{</\textbf{width}>}\end{shaded}\egroup 


    \item[{Content model}]
  \fbox{\ttfamily <content>\newline
 <macroRef key="macro.xtext"/>\newline
</content>\newline
    } 
    \item[{Schema Declaration}]
  \mbox{}\hfill\\[-10pt]\begin{Verbatim}[fontsize=\small]
element width { att.global.attributes, att.dimensions.attributes, macro.xtext }
\end{Verbatim}

\end{reflist}  \index{wit=<wit>|oddindex}
\begin{reflist}
\item[]\begin{specHead}{TEI.wit}{<wit> }contains a list of one or more sigla of witnesses attesting a given reading, in a textual variation. [\xref{http://www.tei-c.org/release/doc/tei-p5-doc/en/html/TC.html\#TCAPLW}{12.1.4. Witness Information}]\end{specHead} 
    \item[{Module}]
  textcrit
    \item[{Attributes}]
  Attributes att.global (\textit{@xml:id}, \textit{@n}, \textit{@xml:lang}, \textit{@xml:base}, \textit{@xml:space})  (att.global.rendition (\textit{@rend}, \textit{@style}, \textit{@rendition})) (att.global.linking (\textit{@corresp}, \textit{@synch}, \textit{@sameAs}, \textit{@copyOf}, \textit{@next}, \textit{@prev}, \textit{@exclude}, \textit{@select})) (att.global.analytic (\textit{@ana})) (att.global.facs (\textit{@facs})) (att.global.change (\textit{@change})) (att.global.responsibility (\textit{@cert}, \textit{@resp})) (att.global.source (\textit{@source})) att.rdgPart (\textit{@wit}) 
    \item[{Member of}]
  model.rdgPart 
    \item[{Contained by}]
  
    \item[textcrit: ]
   app lem rdg rdgGrp
    \item[{May contain}]
  
    \item[analysis: ]
   c cl interp interpGrp m pc phr s span spanGrp w\par 
    \item[core: ]
   abbr add address cb choice corr date del distinct email emph expan foreign gap gb gloss graphic hi index lb measure measureGrp media mentioned milestone name note num orig pb ptr ref reg rs sic soCalled term time title unclear\par 
    \item[figures: ]
   figure formula notatedMusic\par 
    \item[gaiji: ]
   g\par 
    \item[header: ]
   idno\par 
    \item[linking: ]
   alt altGrp anchor join joinGrp link linkGrp seg timeline\par 
    \item[msdescription: ]
   catchwords depth dim dimensions height heraldry locus locusGrp material objectType origDate origPlace secFol signatures stamp watermark width\par 
    \item[namesdates: ]
   addName affiliation bloc climate country district forename genName geo geogFeat geogName location nameLink offset orgName persName placeName population region roleName settlement state surname terrain trait\par 
    \item[textcrit: ]
   app witDetail\par 
    \item[transcr: ]
   addSpan am damage damageSpan delSpan ex fw handShift listTranspose metamark mod redo restore retrace secl space subst substJoin supplied surplus undo\par character data
    \item[{Note}]
  \par
This element represents the same information as that provided by the {\itshape wit} attribute of the reading; it may be used to record the exact form of the sigla given in the source edition, when that is of interest.
    \item[{Example}]
  \leavevmode\bgroup\exampleFont \begin{shaded}\noindent\mbox{}{<\textbf{rdg}\hspace*{6pt}{wit}="{\#El \#Hg}">}Experience{</\textbf{rdg}>}\mbox{}\newline 
{<\textbf{wit}>}Ellesmere, Hengwryt{</\textbf{wit}>}\end{shaded}\egroup 


    \item[{Content model}]
  \mbox{}\hfill\\[-10pt]\begin{Verbatim}[fontsize=\small]
<content>
 <macroRef key="macro.phraseSeq"/>
</content>
    
\end{Verbatim}

    \item[{Schema Declaration}]
  \mbox{}\hfill\\[-10pt]\begin{Verbatim}[fontsize=\small]
element wit { att.global.attributes, att.rdgPart.attributes, macro.phraseSeq }
\end{Verbatim}

\end{reflist}  \index{witDetail=<witDetail>|oddindex}\index{wit=@wit!<witDetail>|oddindex}\index{type=@type!<witDetail>|oddindex}
\begin{reflist}
\item[]\begin{specHead}{TEI.witDetail}{<witDetail> }(witness detail) gives further information about a particular witness, or witnesses, to a particular reading. [\xref{http://www.tei-c.org/release/doc/tei-p5-doc/en/html/TC.html\#TCAPLL}{12.1. The Apparatus Entry, Readings, and Witnesses}]\end{specHead} 
    \item[{Module}]
  textcrit
    \item[{Attributes}]
  Attributes att.global (\textit{@xml:id}, \textit{@n}, \textit{@xml:lang}, \textit{@xml:base}, \textit{@xml:space})  (att.global.rendition (\textit{@rend}, \textit{@style}, \textit{@rendition})) (att.global.linking (\textit{@corresp}, \textit{@synch}, \textit{@sameAs}, \textit{@copyOf}, \textit{@next}, \textit{@prev}, \textit{@exclude}, \textit{@select})) (att.global.analytic (\textit{@ana})) (att.global.facs (\textit{@facs})) (att.global.change (\textit{@change})) (att.global.responsibility (\textit{@cert}, \textit{@resp})) (att.global.source (\textit{@source})) att.placement (\textit{@place}) att.pointing (\textit{@targetLang}, \textit{@target}, \textit{@evaluate}) \hfil\\[-10pt]\begin{sansreflist}
    \item[@wit]
  (witnesses) indicates the sigil or sigla identifying the witness or witnesses to which the detail refers.
\begin{reflist}
    \item[{Status}]
  Required
    \item[{Datatype}]
  1–∞ occurrences of teidata.pointer separated by whitespace
\end{reflist}  
    \item[@type]
  describes the type of information given about the witness.
\begin{reflist}
    \item[{Status}]
  Optional
    \item[{Datatype}]
  teidata.enumerated
\end{reflist}  
\end{sansreflist}  
    \item[{Member of}]
  model.noteLike
    \item[{Contained by}]
  
    \item[analysis: ]
   cl m phr s span w\par 
    \item[core: ]
   abbr add addrLine address author bibl biblScope biblStruct cit citedRange corr date del distinct editor email emph expan foreign gloss head headItem headLabel hi imprint item l label lg list measure mentioned monogr name note num orig p pubPlace publisher q quote ref reg resp rs said series sic soCalled sp speaker stage street term textLang time title unclear\par 
    \item[figures: ]
   cell figure table\par 
    \item[gaiji: ]
   char glyph\par 
    \item[header: ]
   authority change classCode distributor edition extent funder geoDecl handNote language licence notesStmt principal scriptNote sponsor typeNote\par 
    \item[linking: ]
   ab seg\par 
    \item[msdescription: ]
   accMat acquisition additions adminInfo catchwords collation colophon condition custEvent decoNote explicit filiation finalRubric foliation heraldry incipit layout material msItem msItemStruct musicNotation objectType origDate origPlace origin provenance rubric secFol signatures source stamp summary support surrogates watermark\par 
    \item[namesdates: ]
   addName affiliation age birth bloc climate country death district education event faith floruit forename genName geogFeat geogName langKnown location nameLink nationality occupation offset org orgName persName person personGrp place placeName population region residence roleName settlement sex socecStatus state surname terrain trait\par 
    \item[textcrit: ]
   app lem rdg rdgGrp wit witDetail\par 
    \item[textstructure: ]
   argument back body byline closer dateline div docAuthor docDate docEdition docImprint docTitle epigraph floatingText front group imprimatur opener postscript salute signed text titlePage titlePart trailer\par 
    \item[transcr: ]
   damage fw line metamark mod restore retrace secl sourceDoc supplied surface surfaceGrp surplus zone
    \item[{May contain}]
  
    \item[analysis: ]
   c cl interp interpGrp m pc phr s span spanGrp w\par 
    \item[core: ]
   abbr add address cb choice corr date del distinct email emph expan foreign gap gb gloss graphic hi index lb measure measureGrp media mentioned milestone name note num orig pb ptr ref reg rs sic soCalled term time title unclear\par 
    \item[figures: ]
   figure formula notatedMusic\par 
    \item[gaiji: ]
   g\par 
    \item[header: ]
   idno\par 
    \item[linking: ]
   alt altGrp anchor join joinGrp link linkGrp seg timeline\par 
    \item[msdescription: ]
   catchwords depth dim dimensions height heraldry locus locusGrp material objectType origDate origPlace secFol signatures stamp watermark width\par 
    \item[namesdates: ]
   addName affiliation bloc climate country district forename genName geo geogFeat geogName location nameLink offset orgName persName placeName population region roleName settlement state surname terrain trait\par 
    \item[textcrit: ]
   app witDetail\par 
    \item[transcr: ]
   addSpan am damage damageSpan delSpan ex fw handShift listTranspose metamark mod redo restore retrace secl space subst substJoin supplied surplus undo\par character data
    \item[{Note}]
  \par
The <witDetail> element should be regarded as a specialized type of <note> element; it is synonymous with <note type='witnessDetail'>, but differs in the omission of some attributes seldom applicable to notes within critical apparatus, and in the provision of the {\itshape wit} attribute, which permits an application to extract all annotation concerning a particular witness or witnesses from the apparatus. It also differs in that the location of a <witDetail> element is not significant and may not be used to imply the point of attachment for the annotation; this must be explicitly given by means of the {\itshape target} attribute.
    \item[{Example}]
  \leavevmode\bgroup\exampleFont \begin{shaded}\noindent\mbox{}{<\textbf{app}\hspace*{6pt}{type}="{substantive}">}\mbox{}\newline 
\hspace*{6pt}{<\textbf{rdgGrp}\hspace*{6pt}{type}="{subvariants}">}\mbox{}\newline 
\hspace*{6pt}\hspace*{6pt}{<\textbf{lem}\hspace*{6pt}{wit}="{\#El \#HG}"\hspace*{6pt}{xml:id}="{W026x}">}Experience{</\textbf{lem}>}\mbox{}\newline 
\hspace*{6pt}\hspace*{6pt}{<\textbf{rdg}\hspace*{6pt}{wit}="{\#Ha4}">}Experiens{</\textbf{rdg}>}\mbox{}\newline 
\hspace*{6pt}{</\textbf{rdgGrp}>}\mbox{}\newline 
{</\textbf{app}>}\mbox{}\newline 
{<\textbf{witDetail}\hspace*{6pt}{resp}="{\#PR}"\hspace*{6pt}{target}="{\#W026x}"\mbox{}\newline 
\hspace*{6pt}{type}="{presentation}"\hspace*{6pt}{wit}="{\#El}">}Ornamental capital.{</\textbf{witDetail}>}\end{shaded}\egroup 


    \item[{Content model}]
  \mbox{}\hfill\\[-10pt]\begin{Verbatim}[fontsize=\small]
<content>
 <macroRef key="macro.phraseSeq"/>
</content>
    
\end{Verbatim}

    \item[{Schema Declaration}]
  \mbox{}\hfill\\[-10pt]\begin{Verbatim}[fontsize=\small]
element witDetail
{
   att.global.attributes,
   att.placement.attributes,
   att.pointing.attributes,
   attribute wit { list { + } },
   attribute type { text }?,
   macro.phraseSeq}
\end{Verbatim}

\end{reflist}  \index{witEnd=<witEnd>|oddindex}
\begin{reflist}
\item[]\begin{specHead}{TEI.witEnd}{<witEnd> }(fragmented witness end) indicates the end, or suspension, of the text of a fragmentary witness. [\xref{http://www.tei-c.org/release/doc/tei-p5-doc/en/html/TC.html\#TCAPMI}{12.1.5. Fragmentary Witnesses}]\end{specHead} 
    \item[{Module}]
  textcrit
    \item[{Attributes}]
  Attributes att.global (\textit{@xml:id}, \textit{@n}, \textit{@xml:lang}, \textit{@xml:base}, \textit{@xml:space})  (att.global.rendition (\textit{@rend}, \textit{@style}, \textit{@rendition})) (att.global.linking (\textit{@corresp}, \textit{@synch}, \textit{@sameAs}, \textit{@copyOf}, \textit{@next}, \textit{@prev}, \textit{@exclude}, \textit{@select})) (att.global.analytic (\textit{@ana})) (att.global.facs (\textit{@facs})) (att.global.change (\textit{@change})) (att.global.responsibility (\textit{@cert}, \textit{@resp})) (att.global.source (\textit{@source})) att.rdgPart (\textit{@wit}) 
    \item[{Member of}]
  model.rdgPart
    \item[{Contained by}]
  
    \item[textcrit: ]
   lem rdg
    \item[{May contain}]
  Empty element
    \item[{Example}]
  \leavevmode\bgroup\exampleFont \begin{shaded}\noindent\mbox{}{<\textbf{app}>}\mbox{}\newline 
\hspace*{6pt}{<\textbf{lem}\hspace*{6pt}{wit}="{\#El \#Hg}">}Experience{</\textbf{lem}>}\mbox{}\newline 
\hspace*{6pt}{<\textbf{rdg}\hspace*{6pt}{wit}="{\#Ha4}">}Ex{<\textbf{g}\hspace*{6pt}{ref}="{\#per}"/>}\mbox{}\newline 
\hspace*{6pt}\hspace*{6pt}{<\textbf{witEnd}/>}\mbox{}\newline 
\hspace*{6pt}{</\textbf{rdg}>}\mbox{}\newline 
{</\textbf{app}>}\end{shaded}\egroup 


    \item[{Content model}]
  \fbox{\ttfamily <content>\newline
</content>\newline
    } 
    \item[{Schema Declaration}]
  \mbox{}\hfill\\[-10pt]\begin{Verbatim}[fontsize=\small]
element witEnd { att.global.attributes, att.rdgPart.attributes, empty }
\end{Verbatim}

\end{reflist}  \index{witStart=<witStart>|oddindex}
\begin{reflist}
\item[]\begin{specHead}{TEI.witStart}{<witStart> }(fragmented witness start) indicates the beginning, or resumption, of the text of a fragmentary witness. [\xref{http://www.tei-c.org/release/doc/tei-p5-doc/en/html/TC.html\#TCAPMI}{12.1.5. Fragmentary Witnesses}]\end{specHead} 
    \item[{Module}]
  textcrit
    \item[{Attributes}]
  Attributes att.global (\textit{@xml:id}, \textit{@n}, \textit{@xml:lang}, \textit{@xml:base}, \textit{@xml:space})  (att.global.rendition (\textit{@rend}, \textit{@style}, \textit{@rendition})) (att.global.linking (\textit{@corresp}, \textit{@synch}, \textit{@sameAs}, \textit{@copyOf}, \textit{@next}, \textit{@prev}, \textit{@exclude}, \textit{@select})) (att.global.analytic (\textit{@ana})) (att.global.facs (\textit{@facs})) (att.global.change (\textit{@change})) (att.global.responsibility (\textit{@cert}, \textit{@resp})) (att.global.source (\textit{@source})) att.rdgPart (\textit{@wit}) 
    \item[{Member of}]
  model.rdgPart
    \item[{Contained by}]
  
    \item[textcrit: ]
   lem rdg
    \item[{May contain}]
  Empty element
    \item[{Example}]
  \leavevmode\bgroup\exampleFont \begin{shaded}\noindent\mbox{}{<\textbf{app}>}\mbox{}\newline 
\hspace*{6pt}{<\textbf{lem}\hspace*{6pt}{wit}="{\#El \#Hg}">}Auctoritee{</\textbf{lem}>}\mbox{}\newline 
\hspace*{6pt}{<\textbf{rdg}\hspace*{6pt}{wit}="{\#La \#Ra2}">}auctorite{</\textbf{rdg}>}\mbox{}\newline 
\hspace*{6pt}{<\textbf{rdg}\hspace*{6pt}{wit}="{\#X}">}\mbox{}\newline 
\hspace*{6pt}\hspace*{6pt}{<\textbf{witStart}/>}auctorite{</\textbf{rdg}>}\mbox{}\newline 
{</\textbf{app}>}\end{shaded}\egroup 


    \item[{Content model}]
  \fbox{\ttfamily <content>\newline
</content>\newline
    } 
    \item[{Schema Declaration}]
  \mbox{}\hfill\\[-10pt]\begin{Verbatim}[fontsize=\small]
element witStart { att.global.attributes, att.rdgPart.attributes, empty }
\end{Verbatim}

\end{reflist}  \index{witness=<witness>|oddindex}
\begin{reflist}
\item[]\begin{specHead}{TEI.witness}{<witness> }contains either a description of a single witness referred to within the critical apparatus, or a list of witnesses which is to be referred to by a single sigil. [\xref{http://www.tei-c.org/release/doc/tei-p5-doc/en/html/TC.html\#TCAPLL}{12.1. The Apparatus Entry, Readings, and Witnesses}]\end{specHead} 
    \item[{Module}]
  textcrit
    \item[{Attributes}]
  Attributes att.global (\textit{@xml:id}, \textit{@n}, \textit{@xml:lang}, \textit{@xml:base}, \textit{@xml:space})  (att.global.rendition (\textit{@rend}, \textit{@style}, \textit{@rendition})) (att.global.linking (\textit{@corresp}, \textit{@synch}, \textit{@sameAs}, \textit{@copyOf}, \textit{@next}, \textit{@prev}, \textit{@exclude}, \textit{@select})) (att.global.analytic (\textit{@ana})) (att.global.facs (\textit{@facs})) (att.global.change (\textit{@change})) (att.global.responsibility (\textit{@cert}, \textit{@resp})) (att.global.source (\textit{@source})) att.sortable (\textit{@sortKey}) 
    \item[{Contained by}]
  
    \item[textcrit: ]
   listWit
    \item[{May contain}]
  
    \item[core: ]
   abbr address bibl biblStruct choice cit date desc distinct email emph expan foreign gloss hi label list listBibl measure measureGrp mentioned name num ptr q quote ref rs said soCalled stage term time title\par 
    \item[figures: ]
   table\par 
    \item[header: ]
   biblFull idno\par 
    \item[msdescription: ]
   catchwords depth dim dimensions height heraldry locus locusGrp material msDesc objectType origDate origPlace secFol signatures stamp watermark width\par 
    \item[namesdates: ]
   addName affiliation bloc climate country district forename genName geo geogFeat geogName listEvent listNym listOrg listPerson listPlace location nameLink offset orgName persName placeName population region roleName settlement state surname terrain trait\par 
    \item[textcrit: ]
   listApp listWit\par 
    \item[textstructure: ]
   floatingText\par 
    \item[transcr: ]
   am ex subst\par character data
    \item[{Note}]
  \par
The content of the <witness> element may give bibliographic information about the witness or witness group, or it may be empty.
    \item[{Example}]
  \leavevmode\bgroup\exampleFont \begin{shaded}\noindent\mbox{}{<\textbf{listWit}>}\mbox{}\newline 
\hspace*{6pt}{<\textbf{witness}\hspace*{6pt}{xml:id}="{EL}">}Ellesmere, Huntingdon Library 26.C.9{</\textbf{witness}>}\mbox{}\newline 
\hspace*{6pt}{<\textbf{witness}\hspace*{6pt}{xml:id}="{HG}">}Hengwrt, National Library of Wales,\mbox{}\newline 
\hspace*{6pt}\hspace*{6pt} Aberystwyth, Peniarth 392D{</\textbf{witness}>}\mbox{}\newline 
\hspace*{6pt}{<\textbf{witness}\hspace*{6pt}{xml:id}="{RA2}">}Bodleian Library Rawlinson Poetic 149\mbox{}\newline 
\hspace*{6pt}\hspace*{6pt} (see further {<\textbf{ptr}\hspace*{6pt}{target}="{http://www.examples.com/MSdescs\#MSRP149}"/>}){</\textbf{witness}>}\mbox{}\newline 
{</\textbf{listWit}>}\end{shaded}\egroup 


    \item[{Content model}]
  \mbox{}\hfill\\[-10pt]\begin{Verbatim}[fontsize=\small]
<content>
 <macroRef key="macro.limitedContent"/>
</content>
    
\end{Verbatim}

    \item[{Schema Declaration}]
  \mbox{}\hfill\\[-10pt]\begin{Verbatim}[fontsize=\small]
element witness
{
   att.global.attributes,
   att.sortable.attributes,
   macro.limitedContent}
\end{Verbatim}

\end{reflist}  \index{xenoData=<xenoData>|oddindex}
\begin{reflist}
\item[]\begin{specHead}{TEI.xenoData}{<xenoData> }(non-TEI metadata) provides a container element into which metadata in non-TEI formats may be placed. [\xref{http://www.tei-c.org/release/doc/tei-p5-doc/en/html/HD.html\#HD9}{2.5. Non-TEI Metadata}]\end{specHead} 
    \item[{Module}]
  header
    \item[{Attributes}]
  Attributes att.global (\textit{@xml:id}, \textit{@n}, \textit{@xml:lang}, \textit{@xml:base}, \textit{@xml:space})  (att.global.rendition (\textit{@rend}, \textit{@style}, \textit{@rendition})) (att.global.linking (\textit{@corresp}, \textit{@synch}, \textit{@sameAs}, \textit{@copyOf}, \textit{@next}, \textit{@prev}, \textit{@exclude}, \textit{@select})) (att.global.analytic (\textit{@ana})) (att.global.facs (\textit{@facs})) (att.global.change (\textit{@change})) (att.global.responsibility (\textit{@cert}, \textit{@resp})) (att.global.source (\textit{@source})) att.declarable (\textit{@default}) att.typed (\textit{@type}, \textit{@subtype}) 
    \item[{Member of}]
  model.teiHeaderPart
    \item[{Contained by}]
  
    \item[header: ]
   teiHeader
    \item[{May contain}]
  Character data only
    \item[{Example}]
  This example presumes that the prefix \texttt{dc} has been bound to the namespace \texttt{http://purl.org/dc/elements/1.1/} and the prefix \texttt{rdf} is bound to the namespace \texttt{http://www.w3.org/1999/02/22-rdf-syntax-ns\#}. Note: The {\itshape about} attribute on the \texttt{rdf:Description} in this example gives a URI indicating the resource to which the metadata contained therein refer. The \texttt{rdf:Description} in the second <xenoData> block has a blank {\itshape about}, meaning it is pointing at the {\itshape current} document, so the RDF is about the document within which it is contained, i.e. the TEI document containing the <xenoData> block. Similarly, any kind of relative URI may be used, including fragment identifiers (see SG-id). Do note, however, that if the contents of the <xenoData> block are to be extracted and used elsewhere, any relative URIs will have to be resolved accordingly.\leavevmode\bgroup\exampleFont \begin{shaded}\noindent\mbox{}{<\textbf{xenoData}\mbox{}\newline 
   xmlns:dc="http://purl.org/dc/elements/1.1/"\mbox{}\newline 
   xmlns:rdf="http://www.w3.org/1999/02/22-rdf-syntax-ns\#">}\mbox{}\newline 
\hspace*{6pt}{<\textbf{rdf:RDF}>}\mbox{}\newline 
\hspace*{6pt}\hspace*{6pt}{<\textbf{rdf:Description}\hspace*{6pt}{rdf:about}="{http://www.worldcat.org/oclc/606621663}">}\mbox{}\newline 
\hspace*{6pt}\hspace*{6pt}\hspace*{6pt}{<\textbf{dc:title}>}The description of a new world, called the blazing-world{</\textbf{dc:title}>}\mbox{}\newline 
\hspace*{6pt}\hspace*{6pt}\hspace*{6pt}{<\textbf{dc:creator}>}The Duchess of Newcastle{</\textbf{dc:creator}>}\mbox{}\newline 
\hspace*{6pt}\hspace*{6pt}\hspace*{6pt}{<\textbf{dc:date}>}1667{</\textbf{dc:date}>}\mbox{}\newline 
\hspace*{6pt}\hspace*{6pt}\hspace*{6pt}{<\textbf{dc:identifier}>}British Library, 8407.h.10{</\textbf{dc:identifier}>}\mbox{}\newline 
\hspace*{6pt}\hspace*{6pt}\hspace*{6pt}{<\textbf{dc:subject}>}utopian fiction{</\textbf{dc:subject}>}\mbox{}\newline 
\hspace*{6pt}\hspace*{6pt}{</\textbf{rdf:Description}>}\mbox{}\newline 
\hspace*{6pt}{</\textbf{rdf:RDF}>}\mbox{}\newline 
{</\textbf{xenoData}>}\mbox{}\newline 
{<\textbf{xenoData}>}\mbox{}\newline 
\hspace*{6pt}{<\textbf{rdf:RDF}>}\mbox{}\newline 
\hspace*{6pt}\hspace*{6pt}{<\textbf{rdf:Description}\hspace*{6pt}{rdf:about}="{}">}\mbox{}\newline 
\hspace*{6pt}\hspace*{6pt}\hspace*{6pt}{<\textbf{dc:title}>}The Description of a New World, Called the Blazing-World, 1668{</\textbf{dc:title}>}\mbox{}\newline 
\hspace*{6pt}\hspace*{6pt}\hspace*{6pt}{<\textbf{dc:creator}>}Cavendish, Margaret (Lucas), Duchess of Newcastle{</\textbf{dc:creator}>}\mbox{}\newline 
\hspace*{6pt}\hspace*{6pt}\hspace*{6pt}{<\textbf{dc:publisher}>}Women Writers Project{</\textbf{dc:publisher}>}\mbox{}\newline 
\hspace*{6pt}\hspace*{6pt}\hspace*{6pt}{<\textbf{dc:date}>}2002-02-12{</\textbf{dc:date}>}\mbox{}\newline 
\hspace*{6pt}\hspace*{6pt}\hspace*{6pt}{<\textbf{dc:subject}>}utopian fiction{</\textbf{dc:subject}>}\mbox{}\newline 
\hspace*{6pt}\hspace*{6pt}{</\textbf{rdf:Description}>}\mbox{}\newline 
\hspace*{6pt}{</\textbf{rdf:RDF}>}\mbox{}\newline 
{</\textbf{xenoData}>}\end{shaded}\egroup 


    \item[{Example}]
  In this example, the prefix \texttt{rdf} is bound to the namespace \texttt{http://www.w3.org/1999/02/22-rdf-syntax-ns\#}, the prefix \texttt{dc} is bound to the namespace \texttt{http://purl.org/dc/elements/1.1/}, and the prefix \texttt{cc} is bound to the namespace \texttt{http://web.resource.org/cc/}.\leavevmode\bgroup\exampleFont \begin{shaded}\noindent\mbox{}{<\textbf{xenoData}\mbox{}\newline 
   xmlns:cc="http://web.resource.org/cc/"\mbox{}\newline 
   xmlns:dc="http://purl.org/dc/elements/1.1/"\mbox{}\newline 
   xmlns:rdf="http://www.w3.org/1999/02/22-rdf-syntax-ns\#">}\mbox{}\newline 
\hspace*{6pt}{<\textbf{rdf:RDF}>}\mbox{}\newline 
\hspace*{6pt}\hspace*{6pt}{<\textbf{cc:Work}\hspace*{6pt}{rdf:about}="{}">}\mbox{}\newline 
\hspace*{6pt}\hspace*{6pt}\hspace*{6pt}{<\textbf{dc:title}>}Applied Software Project Management - review{</\textbf{dc:title}>}\mbox{}\newline 
\hspace*{6pt}\hspace*{6pt}\hspace*{6pt}{<\textbf{dc:type}\hspace*{6pt}{rdf:resource}="{http://purl.org/dc/dcmitype/Text}"/>}\mbox{}\newline 
\hspace*{6pt}\hspace*{6pt}\hspace*{6pt}{<\textbf{dc:license}\hspace*{6pt}{rdf:resource}="{http://creativecommons.org/licenses/by-sa/2.0/uk/}"/>}\mbox{}\newline 
\hspace*{6pt}\hspace*{6pt}{</\textbf{cc:Work}>}\mbox{}\newline 
\hspace*{6pt}\hspace*{6pt}{<\textbf{cc:License}\hspace*{6pt}{rdf:about}="{http://creativecommons.org/licenses/by-sa/2.0/uk/}">}\mbox{}\newline 
\hspace*{6pt}\hspace*{6pt}\hspace*{6pt}{<\textbf{cc:permits}\hspace*{6pt}{rdf:resource}="{http://web.resource.org/cc/Reproduction}"/>}\mbox{}\newline 
\hspace*{6pt}\hspace*{6pt}\hspace*{6pt}{<\textbf{cc:permits}\hspace*{6pt}{rdf:resource}="{http://web.resource.org/cc/Distribution}"/>}\mbox{}\newline 
\hspace*{6pt}\hspace*{6pt}\hspace*{6pt}{<\textbf{cc:requires}\hspace*{6pt}{rdf:resource}="{http://web.resource.org/cc/Notice}"/>}\mbox{}\newline 
\hspace*{6pt}\hspace*{6pt}\hspace*{6pt}{<\textbf{cc:requires}\hspace*{6pt}{rdf:resource}="{http://web.resource.org/cc/Attribution}"/>}\mbox{}\newline 
\hspace*{6pt}\hspace*{6pt}\hspace*{6pt}{<\textbf{cc:permits}\hspace*{6pt}{rdf:resource}="{http://web.resource.org/cc/DerivativeWorks}"/>}\mbox{}\newline 
\hspace*{6pt}\hspace*{6pt}\hspace*{6pt}{<\textbf{cc:requires}\hspace*{6pt}{rdf:resource}="{http://web.resource.org/cc/ShareAlike}"/>}\mbox{}\newline 
\hspace*{6pt}\hspace*{6pt}{</\textbf{cc:License}>}\mbox{}\newline 
\hspace*{6pt}{</\textbf{rdf:RDF}>}\mbox{}\newline 
{</\textbf{xenoData}>}\end{shaded}\egroup 


    \item[{Example}]
  In this example, the prefix \texttt{dc} is again bound to the namespace \texttt{http://www.openarchives.org/OAI/2.0/oai\textunderscore dc/}, and the prefix \texttt{oai\textunderscore dc} is bound to the namespace \texttt{http://www.openarchives.org/OAI/2.0/oai\textunderscore dc/}.\leavevmode\bgroup\exampleFont \begin{shaded}\noindent\mbox{}{<\textbf{xenoData}\mbox{}\newline 
   xmlns:dc="http://purl.org/dc/elements/1.1/"\mbox{}\newline 
   xmlns:oai_dc="http://www.openarchives.org/OAI/2.0/oai\textunderscore dc/">}\mbox{}\newline 
\hspace*{6pt}{<\textbf{oai_dc:dc}>}\mbox{}\newline 
\hspace*{6pt}\hspace*{6pt}{<\textbf{dc:title}>}The colonial despatches of Vancouver Island and British\mbox{}\newline 
\hspace*{6pt}\hspace*{6pt}\hspace*{6pt}\hspace*{6pt} Columbia 1846-1871: 11566, CO 60/2, p. 291; received 13 November.\mbox{}\newline 
\hspace*{6pt}\hspace*{6pt}\hspace*{6pt}\hspace*{6pt} Trevelyan to Merivale (Permanent Under-Secretary){</\textbf{dc:title}>}\mbox{}\newline 
\hspace*{6pt}\hspace*{6pt}{<\textbf{dc:date}>}1858-11-12{</\textbf{dc:date}>}\mbox{}\newline 
\hspace*{6pt}\hspace*{6pt}{<\textbf{dc:creator}>}Trevelyan{</\textbf{dc:creator}>}\mbox{}\newline 
\hspace*{6pt}\hspace*{6pt}{<\textbf{dc:publisher}>}University of Victoria Humanities Computing and Media\mbox{}\newline 
\hspace*{6pt}\hspace*{6pt}\hspace*{6pt}\hspace*{6pt} Centre, and UVic Libraries{</\textbf{dc:publisher}>}\mbox{}\newline 
\hspace*{6pt}\hspace*{6pt}{<\textbf{dc:type}>}InteractiveResource{</\textbf{dc:type}>}\mbox{}\newline 
\hspace*{6pt}\hspace*{6pt}{<\textbf{dc:format}>}application/xhtml+xml{</\textbf{dc:format}>}\mbox{}\newline 
\hspace*{6pt}\hspace*{6pt}{<\textbf{dc:type}>}text{</\textbf{dc:type}>}\mbox{}\newline 
\hspace*{6pt}\hspace*{6pt}{<\textbf{dc:identifier}>}http://bcgenesis.uvic.ca/getDoc.htm?id=B585TE13.scx{</\textbf{dc:identifier}>}\mbox{}\newline 
\hspace*{6pt}\hspace*{6pt}{<\textbf{dc:rights}>}This document is licensed under a Creative Commons …{</\textbf{dc:rights}>}\mbox{}\newline 
\hspace*{6pt}\hspace*{6pt}{<\textbf{dc:language}>}(SCHEME=ISO639) en{</\textbf{dc:language}>}\mbox{}\newline 
\hspace*{6pt}\hspace*{6pt}{<\textbf{dc:source}>}Transcribed from microfilm and/or original documents, and\mbox{}\newline 
\hspace*{6pt}\hspace*{6pt}\hspace*{6pt}\hspace*{6pt} marked up in TEI P5 XML. The interactive XHTML resource is generated\mbox{}\newline 
\hspace*{6pt}\hspace*{6pt}\hspace*{6pt}\hspace*{6pt} from the XHTML using XQuery and XSLT.{</\textbf{dc:source}>}\mbox{}\newline 
\hspace*{6pt}\hspace*{6pt}{<\textbf{dc:source}>}repository: CO{</\textbf{dc:source}>}\mbox{}\newline 
\hspace*{6pt}\hspace*{6pt}{<\textbf{dc:source}>}coNumber: 60{</\textbf{dc:source}>}\mbox{}\newline 
\hspace*{6pt}\hspace*{6pt}{<\textbf{dc:source}>}coVol: 2{</\textbf{dc:source}>}\mbox{}\newline 
\hspace*{6pt}\hspace*{6pt}{<\textbf{dc:source}>}page: 291{</\textbf{dc:source}>}\mbox{}\newline 
\hspace*{6pt}\hspace*{6pt}{<\textbf{dc:source}>}coRegistration: 11566{</\textbf{dc:source}>}\mbox{}\newline 
\hspace*{6pt}\hspace*{6pt}{<\textbf{dc:source}>}received: received 13 November{</\textbf{dc:source}>}\mbox{}\newline 
\hspace*{6pt}\hspace*{6pt}{<\textbf{dc:subject}>}Trevelyan, Sir Charles Edward{</\textbf{dc:subject}>}\mbox{}\newline 
\hspace*{6pt}\hspace*{6pt}{<\textbf{dc:subject}>}Merivale, Herman{</\textbf{dc:subject}>}\mbox{}\newline 
\hspace*{6pt}\hspace*{6pt}{<\textbf{dc:subject}>}Elliot, T. Frederick{</\textbf{dc:subject}>}\mbox{}\newline 
\hspace*{6pt}\hspace*{6pt}{<\textbf{dc:subject}>}Moody, Colonel Richard Clement{</\textbf{dc:subject}>}\mbox{}\newline 
\hspace*{6pt}\hspace*{6pt}{<\textbf{dc:subject}>}Lytton, Sir Edward George Earle Bulwer{</\textbf{dc:subject}>}\mbox{}\newline 
\hspace*{6pt}\hspace*{6pt}{<\textbf{dc:subject}>}Jadis, Vane{</\textbf{dc:subject}>}\mbox{}\newline 
\hspace*{6pt}\hspace*{6pt}{<\textbf{dc:subject}>}Carnarvon, Earl{</\textbf{dc:subject}>}\mbox{}\newline 
\hspace*{6pt}\hspace*{6pt}{<\textbf{dc:subject}>}British Columbia{</\textbf{dc:subject}>}\mbox{}\newline 
\hspace*{6pt}\hspace*{6pt}{<\textbf{dc:description}>}British Columbia correspondence: Public Offices\mbox{}\newline 
\hspace*{6pt}\hspace*{6pt}\hspace*{6pt}\hspace*{6pt} document (normally correspondence between government\mbox{}\newline 
\hspace*{6pt}\hspace*{6pt}\hspace*{6pt}\hspace*{6pt} departments){</\textbf{dc:description}>}\mbox{}\newline 
\hspace*{6pt}{</\textbf{oai_dc:dc}>}\mbox{}\newline 
{</\textbf{xenoData}>}\end{shaded}\egroup 


    \item[{Example}]
  In this example, the prefix \texttt{mods} is bound to the namespace \texttt{http://www.loc.gov/mods/v3}.\leavevmode\bgroup\exampleFont \begin{shaded}\noindent\mbox{}{<\textbf{xenoData}\mbox{}\newline 
   xmlns:mods="http://www.loc.gov/mods/v3">}\mbox{}\newline 
\hspace*{6pt}{<\textbf{mods:mods}>}\mbox{}\newline 
\hspace*{6pt}\hspace*{6pt}{<\textbf{mods:titleInfo}>}\mbox{}\newline 
\hspace*{6pt}\hspace*{6pt}\hspace*{6pt}{<\textbf{mods:title}>}Academic adaptation and cross-cultural\mbox{}\newline 
\hspace*{6pt}\hspace*{6pt}\hspace*{6pt}\hspace*{6pt}\hspace*{6pt}\hspace*{6pt} learning experiences of Chinese students at American\mbox{}\newline 
\hspace*{6pt}\hspace*{6pt}\hspace*{6pt}\hspace*{6pt}\hspace*{6pt}\hspace*{6pt} universities{</\textbf{mods:title}>}\mbox{}\newline 
\hspace*{6pt}\hspace*{6pt}\hspace*{6pt}{<\textbf{mods:subTitle}>}a narrative inquiry{</\textbf{mods:subTitle}>}\mbox{}\newline 
\hspace*{6pt}\hspace*{6pt}{</\textbf{mods:titleInfo}>}\mbox{}\newline 
\hspace*{6pt}\hspace*{6pt}{<\textbf{mods:name}\hspace*{6pt}{authority}="{local}"\mbox{}\newline 
\hspace*{6pt}\hspace*{6pt}\hspace*{6pt}{type}="{personal}">}\mbox{}\newline 
\hspace*{6pt}\hspace*{6pt}\hspace*{6pt}{<\textbf{mods:namePart}/>}\mbox{}\newline 
\hspace*{6pt}\hspace*{6pt}\hspace*{6pt}{<\textbf{mods:role}>}\mbox{}\newline 
\hspace*{6pt}\hspace*{6pt}\hspace*{6pt}\hspace*{6pt}{<\textbf{mods:roleTerm}\hspace*{6pt}{authority}="{marcrelator}"\mbox{}\newline 
\hspace*{6pt}\hspace*{6pt}\hspace*{6pt}\hspace*{6pt}\hspace*{6pt}{type}="{text}">}Author{</\textbf{mods:roleTerm}>}\mbox{}\newline 
\hspace*{6pt}\hspace*{6pt}\hspace*{6pt}{</\textbf{mods:role}>}\mbox{}\newline 
\hspace*{6pt}\hspace*{6pt}\hspace*{6pt}{<\textbf{mods:affiliation}>}Northeastern University{</\textbf{mods:affiliation}>}\mbox{}\newline 
\hspace*{6pt}\hspace*{6pt}\hspace*{6pt}{<\textbf{mods:namePart}\hspace*{6pt}{type}="{given}">}Hong{</\textbf{mods:namePart}>}\mbox{}\newline 
\hspace*{6pt}\hspace*{6pt}\hspace*{6pt}{<\textbf{mods:namePart}\hspace*{6pt}{type}="{family}">}Zhang{</\textbf{mods:namePart}>}\mbox{}\newline 
\hspace*{6pt}\hspace*{6pt}{</\textbf{mods:name}>}\mbox{}\newline 
\hspace*{6pt}\hspace*{6pt}{<\textbf{mods:name}\hspace*{6pt}{authority}="{local}"\mbox{}\newline 
\hspace*{6pt}\hspace*{6pt}\hspace*{6pt}{type}="{personal}">}\mbox{}\newline 
\hspace*{6pt}\hspace*{6pt}\hspace*{6pt}{<\textbf{mods:namePart}/>}\mbox{}\newline 
\hspace*{6pt}\hspace*{6pt}\hspace*{6pt}{<\textbf{mods:role}>}\mbox{}\newline 
\hspace*{6pt}\hspace*{6pt}\hspace*{6pt}\hspace*{6pt}{<\textbf{mods:roleTerm}\hspace*{6pt}{authority}="{local}"\mbox{}\newline 
\hspace*{6pt}\hspace*{6pt}\hspace*{6pt}\hspace*{6pt}\hspace*{6pt}{type}="{text}">}Advisor{</\textbf{mods:roleTerm}>}\mbox{}\newline 
\hspace*{6pt}\hspace*{6pt}\hspace*{6pt}{</\textbf{mods:role}>}\mbox{}\newline 
\hspace*{6pt}\hspace*{6pt}\hspace*{6pt}{<\textbf{mods:namePart}\hspace*{6pt}{type}="{given}">}Liliana{</\textbf{mods:namePart}>}\mbox{}\newline 
\hspace*{6pt}\hspace*{6pt}\hspace*{6pt}{<\textbf{mods:namePart}\hspace*{6pt}{type}="{family}">}Meneses{</\textbf{mods:namePart}>}\mbox{}\newline 
\hspace*{6pt}\hspace*{6pt}{</\textbf{mods:name}>}\mbox{}\newline 
\hspace*{6pt}\hspace*{6pt}\mbox{}\newline 
\textit{<!-- ... -->}\mbox{}\newline 
\hspace*{6pt}\hspace*{6pt}{<\textbf{mods:typeOfResource}>}text{</\textbf{mods:typeOfResource}>}\mbox{}\newline 
\hspace*{6pt}\hspace*{6pt}{<\textbf{mods:genre}>}doctoral theses{</\textbf{mods:genre}>}\mbox{}\newline 
\hspace*{6pt}\hspace*{6pt}{<\textbf{mods:originInfo}>}\mbox{}\newline 
\hspace*{6pt}\hspace*{6pt}\hspace*{6pt}{<\textbf{mods:place}>}\mbox{}\newline 
\hspace*{6pt}\hspace*{6pt}\hspace*{6pt}\hspace*{6pt}{<\textbf{mods:placeTerm}\hspace*{6pt}{type}="{text}">}Boston (Mass.){</\textbf{mods:placeTerm}>}\mbox{}\newline 
\hspace*{6pt}\hspace*{6pt}\hspace*{6pt}{</\textbf{mods:place}>}\mbox{}\newline 
\hspace*{6pt}\hspace*{6pt}\hspace*{6pt}{<\textbf{mods:publisher}>}Northeastern University{</\textbf{mods:publisher}>}\mbox{}\newline 
\hspace*{6pt}\hspace*{6pt}\hspace*{6pt}{<\textbf{mods:copyrightDate}\hspace*{6pt}{encoding}="{w3cdtf}"\mbox{}\newline 
\hspace*{6pt}\hspace*{6pt}\hspace*{6pt}\hspace*{6pt}{keyDate}="{yes}">}2013{</\textbf{mods:copyrightDate}>}\mbox{}\newline 
\hspace*{6pt}\hspace*{6pt}{</\textbf{mods:originInfo}>}\mbox{}\newline 
\hspace*{6pt}\hspace*{6pt}{<\textbf{mods:language}>}\mbox{}\newline 
\hspace*{6pt}\hspace*{6pt}\hspace*{6pt}{<\textbf{mods:languageTerm}\hspace*{6pt}{authority}="{iso639-2b}"\mbox{}\newline 
\hspace*{6pt}\hspace*{6pt}\hspace*{6pt}\hspace*{6pt}{type}="{code}">}eng{</\textbf{mods:languageTerm}>}\mbox{}\newline 
\hspace*{6pt}\hspace*{6pt}{</\textbf{mods:language}>}\mbox{}\newline 
\hspace*{6pt}\hspace*{6pt}{<\textbf{mods:physicalDescription}>}\mbox{}\newline 
\hspace*{6pt}\hspace*{6pt}\hspace*{6pt}{<\textbf{mods:form}\hspace*{6pt}{authority}="{marcform}">}electronic{</\textbf{mods:form}>}\mbox{}\newline 
\hspace*{6pt}\hspace*{6pt}\hspace*{6pt}{<\textbf{mods:digitalOrigin}>}born digital{</\textbf{mods:digitalOrigin}>}\mbox{}\newline 
\hspace*{6pt}\hspace*{6pt}{</\textbf{mods:physicalDescription}>}\mbox{}\newline 
\textit{<!-- ... -->}\mbox{}\newline 
\hspace*{6pt}{</\textbf{mods:mods}>}\mbox{}\newline 
{</\textbf{xenoData}>}\end{shaded}\egroup 


    \item[{Content model}]
  \mbox{}\hfill\\[-10pt]\begin{Verbatim}[fontsize=\small]
<content>
 <alternate>
  <textNode/>
  <anyElement/>
 </alternate>
</content>
    
\end{Verbatim}

    \item[{Schema Declaration}]
  \mbox{}\hfill\\[-10pt]\begin{Verbatim}[fontsize=\small]
element xenoData
{
   att.global.attributes,
   att.declarable.attributes,
   att.typed.attributes,
   ( text | anyElement-xenoData )
}
\end{Verbatim}

\end{reflist}  \index{zone=<zone>|oddindex}\index{rotate=@rotate!<zone>|oddindex}
\begin{reflist}
\item[]\begin{specHead}{TEI.zone}{<zone> }defines any two-dimensional area within a <surface> element. [\xref{http://www.tei-c.org/release/doc/tei-p5-doc/en/html/PH.html\#PHFAX}{11.1. Digital Facsimiles} \xref{http://www.tei-c.org/release/doc/tei-p5-doc/en/html/PH.html\#PHZLAB}{11.2.2. Embedded Transcription}]\end{specHead} 
    \item[{Module}]
  transcr
    \item[{Attributes}]
  Attributes att.global (\textit{@xml:id}, \textit{@n}, \textit{@xml:lang}, \textit{@xml:base}, \textit{@xml:space})  (att.global.rendition (\textit{@rend}, \textit{@style}, \textit{@rendition})) (att.global.linking (\textit{@corresp}, \textit{@synch}, \textit{@sameAs}, \textit{@copyOf}, \textit{@next}, \textit{@prev}, \textit{@exclude}, \textit{@select})) (att.global.analytic (\textit{@ana})) (att.global.facs (\textit{@facs})) (att.global.change (\textit{@change})) (att.global.responsibility (\textit{@cert}, \textit{@resp})) (att.global.source (\textit{@source})) att.coordinated (\textit{@start}, \textit{@ulx}, \textit{@uly}, \textit{@lrx}, \textit{@lry}, \textit{@points}) att.typed (\textit{@type}, \textit{@subtype}) att.written (\textit{@hand}) \hfil\\[-10pt]\begin{sansreflist}
    \item[@rotate]
  indicates the amount by which this zone has been rotated clockwise, with respect to the normal orientation of the parent <surface> element as implied by the dimensions given in the <msDesc> element or by the coordinates of the <surface> itself. The orientation is expressed in arc degrees.
\begin{reflist}
    \item[{Status}]
  Optional
    \item[{Datatype}]
  teidata.count
    \item[{Default}]
  0
\end{reflist}  
\end{sansreflist}  
    \item[{Member of}]
  model.linePart 
    \item[{Contained by}]
  
    \item[transcr: ]
   line surface zone
    \item[{May contain}]
  
    \item[analysis: ]
   c interp interpGrp pc span spanGrp w\par 
    \item[core: ]
   add cb choice del gap gb graphic hi index lb media milestone note pb unclear\par 
    \item[figures: ]
   figure formula notatedMusic\par 
    \item[linking: ]
   alt altGrp anchor join joinGrp link linkGrp seg timeline\par 
    \item[textcrit: ]
   app witDetail\par 
    \item[transcr: ]
   addSpan damage damageSpan delSpan fw handShift line listTranspose metamark mod redo restore retrace space substJoin surface undo zone\par character data
    \item[{Note}]
  \par
The position of every zone for a given surface is always defined by reference to the coordinate system defined for that surface. \par
A graphic element contained by a zone represents the whole of the zone.\par
A zone may be of any shape. The attribute {\itshape points} may be used to define a polygonal zone, using the coordinate system defined by its parent surface.
    \item[{Example}]
  \leavevmode\bgroup\exampleFont \begin{shaded}\noindent\mbox{}{<\textbf{surface}\hspace*{6pt}{lrx}="{0}"\hspace*{6pt}{lry}="{0}"\hspace*{6pt}{ulx}="{14.54}"\mbox{}\newline 
\hspace*{6pt}{uly}="{16.14}">}\mbox{}\newline 
\hspace*{6pt}{<\textbf{graphic}\hspace*{6pt}{url}="{stone.jpg}"/>}\mbox{}\newline 
\hspace*{6pt}{<\textbf{zone}\hspace*{6pt}{points}="{4.6,6.3 5.25,5.85 6.2,6.6 8.19222,7.4125 9.89222,6.5875 10.9422,6.1375 
 11.4422,6.7125 8.21722,8.3125 6.2,7.65}"/>}\mbox{}\newline 
{</\textbf{surface}>}\end{shaded}\egroup 

This example defines a non-rectangular zone: see the illustration in section PH-surfzone.
    \item[{Example}]
  \leavevmode\bgroup\exampleFont \begin{shaded}\noindent\mbox{}{<\textbf{facsimile}>}\mbox{}\newline 
\hspace*{6pt}{<\textbf{surface}\hspace*{6pt}{lrx}="{400}"\hspace*{6pt}{lry}="{280}"\hspace*{6pt}{ulx}="{50}"\mbox{}\newline 
\hspace*{6pt}\hspace*{6pt}{uly}="{20}">}\mbox{}\newline 
\hspace*{6pt}\hspace*{6pt}{<\textbf{zone}\hspace*{6pt}{lrx}="{500}"\hspace*{6pt}{lry}="{321}"\hspace*{6pt}{ulx}="{0}"\hspace*{6pt}{uly}="{0}">}\mbox{}\newline 
\hspace*{6pt}\hspace*{6pt}\hspace*{6pt}{<\textbf{graphic}\hspace*{6pt}{url}="{graphic.png }"/>}\mbox{}\newline 
\hspace*{6pt}\hspace*{6pt}{</\textbf{zone}>}\mbox{}\newline 
\hspace*{6pt}{</\textbf{surface}>}\mbox{}\newline 
{</\textbf{facsimile}>}\end{shaded}\egroup 

This example defines a zone which has been defined as larger than its parent surface in order to match the dimensions of the graphic it contains.
    \item[{Content model}]
  \mbox{}\hfill\\[-10pt]\begin{Verbatim}[fontsize=\small]
<content>
 <alternate maxOccurs="unbounded"
  minOccurs="0">
  <textNode/>
  <classRef key="model.graphicLike"/>
  <classRef key="model.global"/>
  <elementRef key="surface"/>
  <classRef key="model.linePart"/>
 </alternate>
</content>
    
\end{Verbatim}

    \item[{Schema Declaration}]
  \mbox{}\hfill\\[-10pt]\begin{Verbatim}[fontsize=\small]
element zone
{
   att.global.attributes,
   att.coordinated.attributes,
   att.typed.attributes,
   att.written.attributes,
   attribute rotate { text }?,
   ( text | model.graphicLike | model.global | surface | model.linePart )*
}
\end{Verbatim}

\end{reflist}  
\section[{Model classes}]{Model classes}
\begin{reflist}
\item[]\begin{specHead}{TEI.model.addrPart}{model.addrPart}\index{model.addrPart (model class)|oddindex} groups elements such as names or postal codes which may appear as part of a postal address. [\xref{http://www.tei-c.org/release/doc/tei-p5-doc/en/html/CO.html\#CONAAD}{3.5.2. Addresses}]\end{specHead} 
    \item[{Module}]
  tei
    \item[{Used by}]
  address
    \item[{Members}]
  model.nameLike[model.nameLike.agent[name orgName persName] model.offsetLike[geogFeat offset] model.persNamePart[addName forename genName nameLink roleName surname] model.placeStateLike[model.placeNamePart[bloc country district geogName placeName region settlement] climate location population state terrain trait] idno rs] addrLine street
\end{reflist}  
\begin{reflist}
\item[]\begin{specHead}{TEI.model.addressLike}{model.addressLike}\index{model.addressLike (model class)|oddindex} groups elements used to represent a postal or email address. [\xref{http://www.tei-c.org/release/doc/tei-p5-doc/en/html/ST.html\#ST}{1. The TEI Infrastructure}]\end{specHead} 
    \item[{Module}]
  tei
    \item[{Used by}]
  location model.correspActionPart model.pPart.data
    \item[{Members}]
  address affiliation email
\end{reflist}  
\begin{reflist}
\item[]\begin{specHead}{TEI.model.applicationLike}{model.applicationLike}\index{model.applicationLike (model class)|oddindex} groups elements used to record application-specific information about a document in its header.\end{specHead} 
    \item[{Module}]
  tei
    \item[{Used by}]
  appInfo
    \item[{Members}]
  application
\end{reflist}  
\begin{reflist}
\item[]\begin{specHead}{TEI.model.availabilityPart}{model.availabilityPart}\index{model.availabilityPart (model class)|oddindex} groups elements such as licences and paragraphs of text which may appear as part of an availability statement [\xref{http://www.tei-c.org/release/doc/tei-p5-doc/en/html/HD.html\#HD24}{2.2.4. Publication, Distribution, Licensing, etc.}]\end{specHead} 
    \item[{Module}]
  tei
    \item[{Used by}]
  availability
    \item[{Members}]
  licence
\end{reflist}  
\begin{reflist}
\item[]\begin{specHead}{TEI.model.biblLike}{model.biblLike}\index{model.biblLike (model class)|oddindex} groups elements containing a bibliographic description. [\xref{http://www.tei-c.org/release/doc/tei-p5-doc/en/html/CO.html\#COBI}{3.11. Bibliographic Citations and References}]\end{specHead} 
    \item[{Module}]
  tei
    \item[{Used by}]
  cit climate event listBibl location model.inter model.msItemPart model.personPart org place population relatedItem sourceDesc state taxonomy terrain trait
    \item[{Members}]
  bibl biblFull biblStruct listBibl msDesc
\end{reflist}  
\begin{reflist}
\item[]\begin{specHead}{TEI.model.biblPart}{model.biblPart}\index{model.biblPart (model class)|oddindex} groups elements which represent components of a bibliographic description. [\xref{http://www.tei-c.org/release/doc/tei-p5-doc/en/html/CO.html\#COBI}{3.11. Bibliographic Citations and References}]\end{specHead} 
    \item[{Module}]
  tei
    \item[{Used by}]
  bibl
    \item[{Members}]
  model.imprintPart[biblScope distributor pubPlace publisher] model.respLike[author editor funder meeting principal respStmt sponsor] availability bibl citedRange edition extent listRelation msIdentifier relatedItem series textLang
\end{reflist}  
\begin{reflist}
\item[]\begin{specHead}{TEI.model.choicePart}{model.choicePart}\index{model.choicePart (model class)|oddindex} groups elements (other than <choice> itself) which can be used within a <choice> alternation. [\xref{http://www.tei-c.org/release/doc/tei-p5-doc/en/html/CO.html\#COED}{3.4. Simple Editorial Changes}]\end{specHead} 
    \item[{Module}]
  tei
    \item[{Used by}]
  choice
    \item[{Members}]
  abbr am corr ex expan orig reg seg sic supplied unclear
\end{reflist}  
\begin{reflist}
\item[]\begin{specHead}{TEI.model.common}{model.common}\index{model.common (model class)|oddindex} groups common chunk- and inter-level elements. [\xref{http://www.tei-c.org/release/doc/tei-p5-doc/en/html/ST.html\#STEC}{1.3. The TEI Class System}]\end{specHead} 
    \item[{Module}]
  tei
    \item[{Used by}]
  argument body div epigraph figure postscript
    \item[{Members}]
  model.divPart[model.lLike[l] model.pLike[ab p] lg sp] model.inter[model.biblLike[bibl biblFull biblStruct listBibl msDesc] model.egLike model.labelLike[desc label] model.listLike[list listApp listEvent listNym listOrg listPerson listPlace listWit table] model.oddDecl model.qLike[model.quoteLike[cit quote] floatingText q said] model.stageLike[stage]]
    \item[{Note}]
  \par
This class defines the set of chunk- and inter-level elements; it is used in many content models, including those for textual divisions.
\end{reflist}  
\begin{reflist}
\item[]\begin{specHead}{TEI.model.correspActionPart}{model.correspActionPart}\index{model.correspActionPart (model class)|oddindex} groups elements which define the parts (usually names, dates and places) of one action related to the correspondence.\end{specHead} 
    \item[{Module}]
  tei
    \item[{Used by}]
  correspAction
    \item[{Members}]
  model.addressLike[address affiliation email] model.dateLike[date time] model.nameLike[model.nameLike.agent[name orgName persName] model.offsetLike[geogFeat offset] model.persNamePart[addName forename genName nameLink roleName surname] model.placeStateLike[model.placeNamePart[bloc country district geogName placeName region settlement] climate location population state terrain trait] idno rs] note
\end{reflist}  
\begin{reflist}
\item[]\begin{specHead}{TEI.model.correspContextPart}{model.correspContextPart}\index{model.correspContextPart (model class)|oddindex} groups elements which may appear as part of the correspContext element\end{specHead} 
    \item[{Module}]
  tei
    \item[{Used by}]
  correspContext
    \item[{Members}]
  model.pLike[ab p] model.ptrLike[ptr ref] note
\end{reflist}  
\begin{reflist}
\item[]\begin{specHead}{TEI.model.correspDescPart}{model.correspDescPart}\index{model.correspDescPart (model class)|oddindex} groups together metadata elements for describing correspondence\end{specHead} 
    \item[{Module}]
  tei
    \item[{Used by}]
  correspDesc
    \item[{Members}]
  correspAction correspContext note
\end{reflist}  
\begin{reflist}
\item[]\begin{specHead}{TEI.model.dateLike}{model.dateLike}\index{model.dateLike (model class)|oddindex} groups elements containing temporal expressions. [\xref{http://www.tei-c.org/release/doc/tei-p5-doc/en/html/CO.html\#CONADA}{3.5.4. Dates and Times} \xref{http://www.tei-c.org/release/doc/tei-p5-doc/en/html/ND.html\#NDDATE}{13.3.6. Dates and Times}]\end{specHead} 
    \item[{Module}]
  tei
    \item[{Used by}]
  imprint model.correspActionPart model.pPart.data
    \item[{Members}]
  date time
\end{reflist}  
\begin{reflist}
\item[]\begin{specHead}{TEI.model.descLike}{model.descLike}\index{model.descLike (model class)|oddindex} groups elements which contain a description of their function.\end{specHead} 
    \item[{Module}]
  tei
    \item[{Used by}]
  category char gap glyph graphic interp interpGrp join media schemaRef space substJoin taxonomy
    \item[{Members}]
  desc
\end{reflist}  
\begin{reflist}
\item[]\begin{specHead}{TEI.model.dimLike}{model.dimLike}\index{model.dimLike (model class)|oddindex} groups elements which describe a measurement forming part of the physical dimensions of some object.\end{specHead} 
    \item[{Module}]
  tei
    \item[{Used by}]
  dimensions
    \item[{Members}]
  depth height width
\end{reflist}  
\begin{reflist}
\item[]\begin{specHead}{TEI.model.divBottom}{model.divBottom}\index{model.divBottom (model class)|oddindex} groups elements appearing at the end of a text division. [\xref{http://www.tei-c.org/release/doc/tei-p5-doc/en/html/DS.html\#DSDTB}{4.2. Elements Common to All Divisions}]\end{specHead} 
    \item[{Module}]
  tei
    \item[{Used by}]
  body div figure front group lg list table
    \item[{Members}]
  model.divBottomPart[closer postscript signed trailer] model.divWrapper[argument byline dateline docAuthor docDate epigraph meeting salute]
\end{reflist}  
\begin{reflist}
\item[]\begin{specHead}{TEI.model.divBottomPart}{model.divBottomPart}\index{model.divBottomPart (model class)|oddindex} groups elements which can occur only at the end of a text division. [\xref{http://www.tei-c.org/release/doc/tei-p5-doc/en/html/DS.html\#DSTITL}{4.6. Title Pages}]\end{specHead} 
    \item[{Module}]
  tei
    \item[{Used by}]
  back model.divBottom postscript
    \item[{Members}]
  closer postscript signed trailer
\end{reflist}  
\begin{reflist}
\item[]\begin{specHead}{TEI.model.divGenLike}{model.divGenLike}\index{model.divGenLike (model class)|oddindex} groups elements used to represent a structural division which is generated rather than explicitly present in the source.\end{specHead} 
    \item[{Module}]
  tei
    \item[{Used by}]
  body div
    \item[{Members}]
  divGen
\end{reflist}  
\begin{reflist}
\item[]\begin{specHead}{TEI.model.divLike}{model.divLike}\index{model.divLike (model class)|oddindex} groups elements used to represent un-numbered generic structural divisions.\end{specHead} 
    \item[{Module}]
  tei
    \item[{Used by}]
  back body div front lem rdg
    \item[{Members}]
  div
\end{reflist}  
\begin{reflist}
\item[]\begin{specHead}{TEI.model.divPart}{model.divPart}\index{model.divPart (model class)|oddindex} groups paragraph-level elements appearing directly within divisions. [\xref{http://www.tei-c.org/release/doc/tei-p5-doc/en/html/ST.html\#STEC}{1.3. The TEI Class System}]\end{specHead} 
    \item[{Module}]
  tei
    \item[{Used by}]
  lem macro.specialPara model.common rdg
    \item[{Members}]
  model.lLike[l] model.pLike[ab p] lg sp
    \item[{Note}]
  \par
Note that this element class does not include members of the \textsf{model.inter} class, which can appear either within or between paragraph-level items.
\end{reflist}  
\begin{reflist}
\item[]\begin{specHead}{TEI.model.divTop}{model.divTop}\index{model.divTop (model class)|oddindex} groups elements appearing at the beginning of a text division. [\xref{http://www.tei-c.org/release/doc/tei-p5-doc/en/html/DS.html\#DSDTB}{4.2. Elements Common to All Divisions}]\end{specHead} 
    \item[{Module}]
  tei
    \item[{Used by}]
  body div group lg list
    \item[{Members}]
  model.divTopPart[model.headLike[head] opener signed] model.divWrapper[argument byline dateline docAuthor docDate epigraph meeting salute]
\end{reflist}  
\begin{reflist}
\item[]\begin{specHead}{TEI.model.divTopPart}{model.divTopPart}\index{model.divTopPart (model class)|oddindex} groups elements which can occur only at the beginning of a text division. [\xref{http://www.tei-c.org/release/doc/tei-p5-doc/en/html/DS.html\#DSTITL}{4.6. Title Pages}]\end{specHead} 
    \item[{Module}]
  tei
    \item[{Used by}]
  model.divTop postscript
    \item[{Members}]
  model.headLike[head] opener signed
\end{reflist}  
\begin{reflist}
\item[]\begin{specHead}{TEI.model.divWrapper}{model.divWrapper}\index{model.divWrapper (model class)|oddindex} groups elements which can appear at either top or bottom of a textual division. [\xref{http://www.tei-c.org/release/doc/tei-p5-doc/en/html/DS.html\#DSDTB}{4.2. Elements Common to All Divisions}]\end{specHead} 
    \item[{Module}]
  tei
    \item[{Used by}]
  model.divBottom model.divTop
    \item[{Members}]
  argument byline dateline docAuthor docDate epigraph meeting salute
\end{reflist}  
\begin{reflist}
\item[]\begin{specHead}{TEI.model.editorialDeclPart}{model.editorialDeclPart}\index{model.editorialDeclPart (model class)|oddindex} groups elements which may be used inside <editorialDecl> and appear multiple times.\end{specHead} 
    \item[{Module}]
  tei
    \item[{Used by}]
  editorialDecl
    \item[{Members}]
  correction hyphenation interpretation normalization punctuation quotation segmentation stdVals
\end{reflist}  
\begin{reflist}
\item[]\begin{specHead}{TEI.model.emphLike}{model.emphLike}\index{model.emphLike (model class)|oddindex} groups phrase-level elements which are typographically distinct and to which a specific function can be attributed. [\xref{http://www.tei-c.org/release/doc/tei-p5-doc/en/html/CO.html\#COHQ}{3.3. Highlighting and Quotation}]\end{specHead} 
    \item[{Module}]
  tei
    \item[{Used by}]
  model.highlighted model.limitedPhrase
    \item[{Members}]
  distinct emph foreign gloss mentioned soCalled term title
\end{reflist}  
\begin{reflist}
\item[]\begin{specHead}{TEI.model.encodingDescPart}{model.encodingDescPart}\index{model.encodingDescPart (model class)|oddindex} groups elements which may be used inside <encodingDesc> and appear multiple times.\end{specHead} 
    \item[{Module}]
  tei
    \item[{Used by}]
  encodingDesc
    \item[{Members}]
  appInfo charDecl classDecl editorialDecl geoDecl listPrefixDef projectDesc refsDecl samplingDecl schemaRef styleDefDecl tagsDecl variantEncoding
\end{reflist}  
\begin{reflist}
\item[]\begin{specHead}{TEI.model.eventLike}{model.eventLike}\index{model.eventLike (model class)|oddindex} groups elements which describe events.\end{specHead} 
    \item[{Module}]
  tei
    \item[{Used by}]
  model.orgPart model.personPart place
    \item[{Members}]
  event listEvent
\end{reflist}  
\begin{reflist}
\item[]\begin{specHead}{TEI.model.frontPart}{model.frontPart}\index{model.frontPart (model class)|oddindex} groups elements which appear at the level of divisions within front or back matter. [\xref{http://www.tei-c.org/release/doc/tei-p5-doc/en/html/DR.html\#DRFAB}{7.1. Front and Back Matter }]\end{specHead} 
    \item[{Module}]
  tei
    \item[{Used by}]
  back front
    \item[{Members}]
  model.frontPart.drama divGen listBibl titlePage
\end{reflist}  
\begin{reflist}
\item[]\begin{specHead}{TEI.model.gLike}{model.gLike}\index{model.gLike (model class)|oddindex} groups elements used to represent individual non-Unicode characters or glyphs.\end{specHead} 
    \item[{Module}]
  tei
    \item[{Used by}]
  am bibl byline closer date dateline docImprint head idno interp l lem line m macro.paraContent macro.phraseSeq macro.specialPara macro.xtext measureGrp opener origDate pc rdg series time trailer w
    \item[{Members}]
  g
\end{reflist}  
\begin{reflist}
\item[]\begin{specHead}{TEI.model.global}{model.global}\index{model.global (model class)|oddindex} groups elements which may appear at any point within a TEI text. [\xref{http://www.tei-c.org/release/doc/tei-p5-doc/en/html/ST.html\#STEC}{1.3. The TEI Class System}]\end{specHead} 
    \item[{Module}]
  tei
    \item[{Used by}]
  address argument back bibl body byline cit closer date dateline div docImprint docTitle epigraph figure floatingText front group head imprint l lem lg line list m macro.paraContent macro.phraseSeq macro.phraseSeq.limited macro.specialPara msItem opener origDate person personGrp postscript rdg series sourceDoc sp surface surfaceGrp table text time titlePage trailer w zone
    \item[{Members}]
  model.global.edit[addSpan app damageSpan delSpan gap space] model.global.meta[alt altGrp index interp interpGrp join joinGrp link linkGrp listTranspose span spanGrp substJoin timeline] model.milestoneLike[anchor cb fw gb lb milestone pb] model.noteLike[note witDetail] figure metamark notatedMusic
\end{reflist}  
\begin{reflist}
\item[]\begin{specHead}{TEI.model.global.edit}{model.global.edit}\index{model.global.edit (model class)|oddindex} groups globally available elements which perform a specifically editorial function. [\xref{http://www.tei-c.org/release/doc/tei-p5-doc/en/html/ST.html\#STEC}{1.3. The TEI Class System}]\end{specHead} 
    \item[{Module}]
  tei
    \item[{Used by}]
  model.global
    \item[{Members}]
  addSpan app damageSpan delSpan gap space
\end{reflist}  
\begin{reflist}
\item[]\begin{specHead}{TEI.model.global.meta}{model.global.meta}\index{model.global.meta (model class)|oddindex} groups globally available elements which describe the status of other elements. [\xref{http://www.tei-c.org/release/doc/tei-p5-doc/en/html/ST.html\#STEC}{1.3. The TEI Class System}]\end{specHead} 
    \item[{Module}]
  tei
    \item[{Used by}]
  model.global
    \item[{Members}]
  alt altGrp index interp interpGrp join joinGrp link linkGrp listTranspose span spanGrp substJoin timeline
    \item[{Note}]
  \par
Elements in this class are typically used to hold groups of links or of abstract interpretations, or by provide indications of certainty etc. It may find be convenient to localize all metadata elements, for example to contain them within the same divison as the elements that they relate to; or to locate them all to a division of their own. They may however appear at any point in a TEI text.
\end{reflist}  
\begin{reflist}
\item[]\begin{specHead}{TEI.model.glossLike}{model.glossLike}\index{model.glossLike (model class)|oddindex} groups elements which provide an alternative name, explanation, or description for any markup construct.\end{specHead} 
    \item[{Module}]
  tei
    \item[{Used by}]
  category joinGrp taxonomy
    \item[{Members}]
  gloss
\end{reflist}  
\begin{reflist}
\item[]\begin{specHead}{TEI.model.graphicLike}{model.graphicLike}\index{model.graphicLike (model class)|oddindex} groups elements containing images, formulae, and similar objects. [\xref{http://www.tei-c.org/release/doc/tei-p5-doc/en/html/CO.html\#COGR}{3.9. Graphics and Other Non-textual Components}]\end{specHead} 
    \item[{Module}]
  tei
    \item[{Used by}]
  char facsimile figure formula glyph model.phrase sourceDoc surface table zone
    \item[{Members}]
  formula graphic media
\end{reflist}  
\begin{reflist}
\item[]\begin{specHead}{TEI.model.headLike}{model.headLike}\index{model.headLike (model class)|oddindex} groups elements used to provide a title or heading at the start of a text division.\end{specHead} 
    \item[{Module}]
  tei
    \item[{Used by}]
  argument climate divGen event figure listApp listBibl listEvent listNym listOrg listPerson listPlace listRelation listWit model.divTopPart msDesc msFrag msPart org place population state table terrain trait
    \item[{Members}]
  head
\end{reflist}  
\begin{reflist}
\item[]\begin{specHead}{TEI.model.hiLike}{model.hiLike}\index{model.hiLike (model class)|oddindex} groups phrase-level elements which are typographically distinct but to which no specific function can be attributed. [\xref{http://www.tei-c.org/release/doc/tei-p5-doc/en/html/CO.html\#COHQ}{3.3. Highlighting and Quotation}]\end{specHead} 
    \item[{Module}]
  tei
    \item[{Used by}]
  formula m model.highlighted model.limitedPhrase model.linePart w
    \item[{Members}]
  hi
\end{reflist}  
\begin{reflist}
\item[]\begin{specHead}{TEI.model.highlighted}{model.highlighted}\index{model.highlighted (model class)|oddindex} groups phrase-level elements which are typographically distinct. [\xref{http://www.tei-c.org/release/doc/tei-p5-doc/en/html/CO.html\#COHQ}{3.3. Highlighting and Quotation}]\end{specHead} 
    \item[{Module}]
  tei
    \item[{Used by}]
  bibl model.phrase
    \item[{Members}]
  model.emphLike[distinct emph foreign gloss mentioned soCalled term title] model.hiLike[hi]
\end{reflist}  
\begin{reflist}
\item[]\begin{specHead}{TEI.model.imprintPart}{model.imprintPart}\index{model.imprintPart (model class)|oddindex} groups the bibliographic elements which occur inside imprints. [\xref{http://www.tei-c.org/release/doc/tei-p5-doc/en/html/CO.html\#COBI}{3.11. Bibliographic Citations and References}]\end{specHead} 
    \item[{Module}]
  tei
    \item[{Used by}]
  imprint model.biblPart
    \item[{Members}]
  biblScope distributor pubPlace publisher
\end{reflist}  
\begin{reflist}
\item[]\begin{specHead}{TEI.model.inter}{model.inter}\index{model.inter (model class)|oddindex} groups elements which can appear either within or between paragraph-like elements. [\xref{http://www.tei-c.org/release/doc/tei-p5-doc/en/html/ST.html\#STEC}{1.3. The TEI Class System}]\end{specHead} 
    \item[{Module}]
  tei
    \item[{Used by}]
  head l lem macro.limitedContent macro.paraContent macro.specialPara model.common rdg trailer
    \item[{Members}]
  model.biblLike[bibl biblFull biblStruct listBibl msDesc] model.egLike model.labelLike[desc label] model.listLike[list listApp listEvent listNym listOrg listPerson listPlace listWit table] model.oddDecl model.qLike[model.quoteLike[cit quote] floatingText q said] model.stageLike[stage]
\end{reflist}  
\begin{reflist}
\item[]\begin{specHead}{TEI.model.lLike}{model.lLike}\index{model.lLike (model class)|oddindex} groups elements representing metrical components such as verse lines.\end{specHead} 
    \item[{Module}]
  tei
    \item[{Used by}]
  head lg macro.paraContent model.divPart sp trailer
    \item[{Members}]
  l
\end{reflist}  
\begin{reflist}
\item[]\begin{specHead}{TEI.model.labelLike}{model.labelLike}\index{model.labelLike (model class)|oddindex} groups elements used to gloss or explain other parts of a document.\end{specHead} 
    \item[{Module}]
  tei
    \item[{Used by}]
  application climate event lg location model.inter notatedMusic org place population state surface terrain trait
    \item[{Members}]
  desc label
\end{reflist}  
\begin{reflist}
\item[]\begin{specHead}{TEI.model.limitedPhrase}{model.limitedPhrase}\index{model.limitedPhrase (model class)|oddindex} groups phrase-level elements excluding those elements primarily intended for transcription of existing sources. [\xref{http://www.tei-c.org/release/doc/tei-p5-doc/en/html/ST.html\#STEC}{1.3. The TEI Class System}]\end{specHead} 
    \item[{Module}]
  tei
    \item[{Used by}]
  catDesc creation macro.limitedContent macro.phraseSeq.limited
    \item[{Members}]
  model.emphLike[distinct emph foreign gloss mentioned soCalled term title] model.hiLike[hi] model.pPart.data[model.addressLike[address affiliation email] model.dateLike[date time] model.measureLike[depth dim geo height measure measureGrp num width] model.nameLike[model.nameLike.agent[name orgName persName] model.offsetLike[geogFeat offset] model.persNamePart[addName forename genName nameLink roleName surname] model.placeStateLike[model.placeNamePart[bloc country district geogName placeName region settlement] climate location population state terrain trait] idno rs]] model.pPart.editorial[abbr am choice ex expan subst] model.pPart.msdesc[catchwords dimensions heraldry locus locusGrp material objectType origDate origPlace secFol signatures stamp watermark] model.phrase.xml model.ptrLike[ptr ref]
\end{reflist}  
\begin{reflist}
\item[]\begin{specHead}{TEI.model.linePart}{model.linePart}\index{model.linePart (model class)|oddindex} groups transcriptional elements which appear within lines or zones of a source-oriented transcription within a <sourceDoc> element.\end{specHead} 
    \item[{Module}]
  tei
    \item[{Used by}]
  line zone
    \item[{Members}]
  model.hiLike[hi] add c choice damage del handShift line mod pc redo restore retrace seg unclear undo w zone
\end{reflist}  
\begin{reflist}
\item[]\begin{specHead}{TEI.model.listLike}{model.listLike}\index{model.listLike (model class)|oddindex} groups list-like elements. [\xref{http://www.tei-c.org/release/doc/tei-p5-doc/en/html/CO.html\#COLI}{3.7. Lists}]\end{specHead} 
    \item[{Module}]
  tei
    \item[{Used by}]
  abstract back model.inter sourceDesc sp
    \item[{Members}]
  list listApp listEvent listNym listOrg listPerson listPlace listWit table
\end{reflist}  
\begin{reflist}
\item[]\begin{specHead}{TEI.model.measureLike}{model.measureLike}\index{model.measureLike (model class)|oddindex} groups elements which denote a number, a quantity, a measurement, or similar piece of text that conveys some numerical meaning. [\xref{http://www.tei-c.org/release/doc/tei-p5-doc/en/html/CO.html\#CONANU}{3.5.3. Numbers and Measures}]\end{specHead} 
    \item[{Module}]
  tei
    \item[{Used by}]
  location measureGrp model.pPart.data
    \item[{Members}]
  depth dim geo height measure measureGrp num width
\end{reflist}  
\begin{reflist}
\item[]\begin{specHead}{TEI.model.milestoneLike}{model.milestoneLike}\index{model.milestoneLike (model class)|oddindex} groups milestone-style elements used to represent reference systems. [\xref{http://www.tei-c.org/release/doc/tei-p5-doc/en/html/ST.html\#STEC}{1.3. The TEI Class System} \xref{http://www.tei-c.org/release/doc/tei-p5-doc/en/html/CO.html\#CORS5}{3.10.3. Milestone Elements}]\end{specHead} 
    \item[{Module}]
  tei
    \item[{Used by}]
  listBibl model.global org subst
    \item[{Members}]
  anchor cb fw gb lb milestone pb
\end{reflist}  
\begin{reflist}
\item[]\begin{specHead}{TEI.model.msItemPart}{model.msItemPart}\index{model.msItemPart (model class)|oddindex} groups elements which can appear within a manuscript item description.\end{specHead} 
    \item[{Module}]
  tei
    \item[{Used by}]
  msItem
    \item[{Members}]
  model.biblLike[bibl biblFull biblStruct listBibl msDesc] model.msQuoteLike[colophon explicit finalRubric incipit rubric title] model.quoteLike[cit quote] model.respLike[author editor funder meeting principal respStmt sponsor] decoNote filiation idno msItem msItemStruct textLang
\end{reflist}  
\begin{reflist}
\item[]\begin{specHead}{TEI.model.msQuoteLike}{model.msQuoteLike}\index{model.msQuoteLike (model class)|oddindex} groups elements which represent passages such as titles quoted from a manuscript as a part of its description.\end{specHead} 
    \item[{Module}]
  tei
    \item[{Used by}]
  model.msItemPart
    \item[{Members}]
  colophon explicit finalRubric incipit rubric title
\end{reflist}  
\begin{reflist}
\item[]\begin{specHead}{TEI.model.nameLike}{model.nameLike}\index{model.nameLike (model class)|oddindex} groups elements which name or refer to a person, place, or organization.\end{specHead} 
    \item[{Module}]
  tei
    \item[{Used by}]
  model.addrPart model.correspActionPart model.pPart.data org
    \item[{Members}]
  model.nameLike.agent[name orgName persName] model.offsetLike[geogFeat offset] model.persNamePart[addName forename genName nameLink roleName surname] model.placeStateLike[model.placeNamePart[bloc country district geogName placeName region settlement] climate location population state terrain trait] idno rs
    \item[{Note}]
  \par
A superset of the naming elements that may appear in datelines, addresses, statements of responsibility, etc.
\end{reflist}  
\begin{reflist}
\item[]\begin{specHead}{TEI.model.nameLike.agent}{model.nameLike.agent}\index{model.nameLike.agent (model class)|oddindex} groups elements which contain names of individuals or corporate bodies. [\xref{http://www.tei-c.org/release/doc/tei-p5-doc/en/html/CO.html\#CONA}{3.5. Names, Numbers, Dates, Abbreviations, and Addresses}]\end{specHead} 
    \item[{Module}]
  tei
    \item[{Used by}]
  model.nameLike respStmt
    \item[{Members}]
  name orgName persName
    \item[{Note}]
  \par
This class is used in the content model of elements which reference names of people or organizations.
\end{reflist}  
\begin{reflist}
\item[]\begin{specHead}{TEI.model.noteLike}{model.noteLike}\index{model.noteLike (model class)|oddindex} groups globally-available note-like elements. [\xref{http://www.tei-c.org/release/doc/tei-p5-doc/en/html/CO.html\#CONO}{3.8. Notes, Annotation, and Indexing}]\end{specHead} 
    \item[{Module}]
  tei
    \item[{Used by}]
  adminInfo app biblStruct char climate event glyph location model.global monogr msItemStruct notesStmt org place population rdgGrp state terrain trait
    \item[{Members}]
  note witDetail
\end{reflist}  
\begin{reflist}
\item[]\begin{specHead}{TEI.model.offsetLike}{model.offsetLike}\index{model.offsetLike (model class)|oddindex} groups elements which can appear only as part of a place name. [\xref{http://www.tei-c.org/release/doc/tei-p5-doc/en/html/ND.html\#NDPLAC}{13.2.3. Place Names}]\end{specHead} 
    \item[{Module}]
  tei
    \item[{Used by}]
  location model.nameLike
    \item[{Members}]
  geogFeat offset
\end{reflist}  
\begin{reflist}
\item[]\begin{specHead}{TEI.model.orgPart}{model.orgPart}\index{model.orgPart (model class)|oddindex} groups elements which form part of the description of an organization.\end{specHead} 
    \item[{Module}]
  tei
    \item[{Used by}]
  org
    \item[{Members}]
  model.eventLike[event listEvent] listOrg listPerson listPlace
\end{reflist}  
\begin{reflist}
\item[]\begin{specHead}{TEI.model.pLike}{model.pLike}\index{model.pLike (model class)|oddindex} groups paragraph-like elements.\end{specHead} 
    \item[{Module}]
  tei
    \item[{Used by}]
  abstract application availability back binding bindingDesc cRefPattern calendar climate correction correspAction correspDesc custodialHist decoDesc editionStmt editorialDecl encodingDesc event front handDesc history hyphenation interpretation langKnowledge langUsage layoutDesc listRelation model.correspContextPart model.divPart msContents msDesc msFrag msItem msItemStruct msPart normalization nym objectDesc org person personGrp physDesc place population prefixDef projectDesc publicationStmt punctuation quotation recordHist refsDecl samplingDecl scriptDesc seal sealDesc segmentation seriesStmt sourceDesc sp state stdVals styleDefDecl supportDesc terrain trait typeDesc
    \item[{Members}]
  ab p
\end{reflist}  
\begin{reflist}
\item[]\begin{specHead}{TEI.model.pLike.front}{model.pLike.front}\index{model.pLike.front (model class)|oddindex} groups paragraph-like elements which can occur as direct constituents of front matter. [\xref{http://www.tei-c.org/release/doc/tei-p5-doc/en/html/DS.html\#DSTITL}{4.6. Title Pages}]\end{specHead} 
    \item[{Module}]
  tei
    \item[{Used by}]
  back front
    \item[{Members}]
  argument byline docAuthor docDate docEdition docImprint docTitle epigraph head titlePart
\end{reflist}  
\begin{reflist}
\item[]\begin{specHead}{TEI.model.pPart.data}{model.pPart.data}\index{model.pPart.data (model class)|oddindex} groups phrase-level elements containing names, dates, numbers, measures, and similar data. [\xref{http://www.tei-c.org/release/doc/tei-p5-doc/en/html/CO.html\#CONA}{3.5. Names, Numbers, Dates, Abbreviations, and Addresses}]\end{specHead} 
    \item[{Module}]
  tei
    \item[{Used by}]
  bibl model.limitedPhrase model.phrase
    \item[{Members}]
  model.addressLike[address affiliation email] model.dateLike[date time] model.measureLike[depth dim geo height measure measureGrp num width] model.nameLike[model.nameLike.agent[name orgName persName] model.offsetLike[geogFeat offset] model.persNamePart[addName forename genName nameLink roleName surname] model.placeStateLike[model.placeNamePart[bloc country district geogName placeName region settlement] climate location population state terrain trait] idno rs]
\end{reflist}  
\begin{reflist}
\item[]\begin{specHead}{TEI.model.pPart.edit}{model.pPart.edit}\index{model.pPart.edit (model class)|oddindex} groups phrase-level elements for simple editorial correction and transcription. [\xref{http://www.tei-c.org/release/doc/tei-p5-doc/en/html/CO.html\#COED}{3.4. Simple Editorial Changes}]\end{specHead} 
    \item[{Module}]
  tei
    \item[{Used by}]
  bibl model.phrase pc w
    \item[{Members}]
  model.pPart.editorial[abbr am choice ex expan subst] model.pPart.transcriptional[add corr damage del handShift mod orig redo reg restore retrace secl sic supplied surplus unclear undo]
\end{reflist}  
\begin{reflist}
\item[]\begin{specHead}{TEI.model.pPart.editorial}{model.pPart.editorial}\index{model.pPart.editorial (model class)|oddindex} groups phrase-level elements for simple editorial interventions that may be useful both in transcribing and in authoring. [\xref{http://www.tei-c.org/release/doc/tei-p5-doc/en/html/CO.html\#COED}{3.4. Simple Editorial Changes}]\end{specHead} 
    \item[{Module}]
  tei
    \item[{Used by}]
  model.limitedPhrase model.pPart.edit
    \item[{Members}]
  abbr am choice ex expan subst
\end{reflist}  
\begin{reflist}
\item[]\begin{specHead}{TEI.model.pPart.msdesc}{model.pPart.msdesc}\index{model.pPart.msdesc (model class)|oddindex} groups phrase-level elements used in manuscript description. [\xref{http://www.tei-c.org/release/doc/tei-p5-doc/en/html/MS.html\#MS}{10. Manuscript Description}]\end{specHead} 
    \item[{Module}]
  tei
    \item[{Used by}]
  model.limitedPhrase model.phrase
    \item[{Members}]
  catchwords dimensions heraldry locus locusGrp material objectType origDate origPlace secFol signatures stamp watermark
\end{reflist}  
\begin{reflist}
\item[]\begin{specHead}{TEI.model.pPart.transcriptional}{model.pPart.transcriptional}\index{model.pPart.transcriptional (model class)|oddindex} groups phrase-level elements used for editorial transcription of pre-existing source materials. [\xref{http://www.tei-c.org/release/doc/tei-p5-doc/en/html/CO.html\#COED}{3.4. Simple Editorial Changes}]\end{specHead} 
    \item[{Module}]
  tei
    \item[{Used by}]
  am model.pPart.edit
    \item[{Members}]
  add corr damage del handShift mod orig redo reg restore retrace secl sic supplied surplus unclear undo
\end{reflist}  
\begin{reflist}
\item[]\begin{specHead}{TEI.model.persNamePart}{model.persNamePart}\index{model.persNamePart (model class)|oddindex} groups elements which form part of a personal name. [\xref{http://www.tei-c.org/release/doc/tei-p5-doc/en/html/ND.html\#NDPER}{13.2.1. Personal Names}]\end{specHead} 
    \item[{Module}]
  namesdates
    \item[{Used by}]
  model.nameLike
    \item[{Members}]
  addName forename genName nameLink roleName surname
\end{reflist}  
\begin{reflist}
\item[]\begin{specHead}{TEI.model.persStateLike}{model.persStateLike}\index{model.persStateLike (model class)|oddindex} groups elements describing changeable characteristics of a person which have a definite duration, for example occupation, residence, or name.\end{specHead} 
    \item[{Module}]
  tei
    \item[{Used by}]
  model.personPart
    \item[{Members}]
  affiliation age education faith floruit langKnowledge nationality occupation persName residence sex socecStatus state trait
    \item[{Note}]
  \par
These characteristics of an individual are typically a consequence of their own action or that of others.
\end{reflist}  
\begin{reflist}
\item[]\begin{specHead}{TEI.model.personLike}{model.personLike}\index{model.personLike (model class)|oddindex} groups elements which provide information about people and their relationships.\end{specHead} 
    \item[{Module}]
  tei
    \item[{Used by}]
  listPerson org
    \item[{Members}]
  org person personGrp
\end{reflist}  
\begin{reflist}
\item[]\begin{specHead}{TEI.model.personPart}{model.personPart}\index{model.personPart (model class)|oddindex} groups elements which form part of the description of a person. [\xref{http://www.tei-c.org/release/doc/tei-p5-doc/en/html/CC.html\#CCAHPA}{15.2.2. The Participant Description}]\end{specHead} 
    \item[{Module}]
  tei
    \item[{Used by}]
  person personGrp
    \item[{Members}]
  model.biblLike[bibl biblFull biblStruct listBibl msDesc] model.eventLike[event listEvent] model.persStateLike[affiliation age education faith floruit langKnowledge nationality occupation persName residence sex socecStatus state trait] birth death idno
\end{reflist}  
\begin{reflist}
\item[]\begin{specHead}{TEI.model.phrase}{model.phrase}\index{model.phrase (model class)|oddindex} groups elements which can occur at the level of individual words or phrases. [\xref{http://www.tei-c.org/release/doc/tei-p5-doc/en/html/ST.html\#STEC}{1.3. The TEI Class System}]\end{specHead} 
    \item[{Module}]
  tei
    \item[{Used by}]
  byline closer date dateline docImprint head l lem macro.paraContent macro.phraseSeq macro.specialPara opener origDate rdg time trailer
    \item[{Members}]
  model.graphicLike[formula graphic media] model.highlighted[model.emphLike[distinct emph foreign gloss mentioned soCalled term title] model.hiLike[hi]] model.lPart model.pPart.data[model.addressLike[address affiliation email] model.dateLike[date time] model.measureLike[depth dim geo height measure measureGrp num width] model.nameLike[model.nameLike.agent[name orgName persName] model.offsetLike[geogFeat offset] model.persNamePart[addName forename genName nameLink roleName surname] model.placeStateLike[model.placeNamePart[bloc country district geogName placeName region settlement] climate location population state terrain trait] idno rs]] model.pPart.edit[model.pPart.editorial[abbr am choice ex expan subst] model.pPart.transcriptional[add corr damage del handShift mod orig redo reg restore retrace secl sic supplied surplus unclear undo]] model.pPart.msdesc[catchwords dimensions heraldry locus locusGrp material objectType origDate origPlace secFol signatures stamp watermark] model.phrase.xml model.ptrLike[ptr ref] model.segLike[c cl m pc phr s seg w] model.specDescLike
    \item[{Note}]
  \par
This class of elements can occur within paragraphs, list items, lines of verse, etc.
\end{reflist}  
\begin{reflist}
\item[]\begin{specHead}{TEI.model.physDescPart}{model.physDescPart}\index{model.physDescPart (model class)|oddindex} groups specialized elements forming part of the physical description of a manuscript or similar written source.\end{specHead} 
    \item[{Module}]
  tei
    \item[{Used by}]
  physDesc
    \item[{Members}]
  accMat additions bindingDesc decoDesc handDesc musicNotation objectDesc scriptDesc sealDesc typeDesc
\end{reflist}  
\begin{reflist}
\item[]\begin{specHead}{TEI.model.placeLike}{model.placeLike}\index{model.placeLike (model class)|oddindex} groups elements used to provide information about places and their relationships.\end{specHead} 
    \item[{Module}]
  tei
    \item[{Used by}]
  listPlace org place
    \item[{Members}]
  place
\end{reflist}  
\begin{reflist}
\item[]\begin{specHead}{TEI.model.placeNamePart}{model.placeNamePart}\index{model.placeNamePart (model class)|oddindex} groups elements which form part of a place name. [\xref{http://www.tei-c.org/release/doc/tei-p5-doc/en/html/ND.html\#NDPLAC}{13.2.3. Place Names}]\end{specHead} 
    \item[{Module}]
  tei
    \item[{Used by}]
  altIdentifier location model.placeStateLike msIdentifier
    \item[{Members}]
  bloc country district geogName placeName region settlement
\end{reflist}  
\begin{reflist}
\item[]\begin{specHead}{TEI.model.placeStateLike}{model.placeStateLike}\index{model.placeStateLike (model class)|oddindex} groups elements which describe changing states of a place.\end{specHead} 
    \item[{Module}]
  tei
    \item[{Used by}]
  model.nameLike place
    \item[{Members}]
  model.placeNamePart[bloc country district geogName placeName region settlement] climate location population state terrain trait
\end{reflist}  
\begin{reflist}
\item[]\begin{specHead}{TEI.model.profileDescPart}{model.profileDescPart}\index{model.profileDescPart (model class)|oddindex} groups elements which may be used inside <profileDesc> and appear multiple times.\end{specHead} 
    \item[{Module}]
  tei
    \item[{Used by}]
  profileDesc
    \item[{Members}]
  abstract calendarDesc correspDesc creation handNotes langUsage listTranspose textClass
\end{reflist}  
\begin{reflist}
\item[]\begin{specHead}{TEI.model.ptrLike}{model.ptrLike}\index{model.ptrLike (model class)|oddindex} groups elements used for purposes of location and reference. [\xref{http://www.tei-c.org/release/doc/tei-p5-doc/en/html/CO.html\#COXR}{3.6. Simple Links and Cross-References}]\end{specHead} 
    \item[{Module}]
  tei
    \item[{Used by}]
  analytic application bibl biblStruct cit model.correspContextPart model.limitedPhrase model.phrase monogr notatedMusic relatedItem series
    \item[{Members}]
  ptr ref
\end{reflist}  
\begin{reflist}
\item[]\begin{specHead}{TEI.model.publicationStmtPart.agency}{model.publicationStmtPart.agency}\index{model.publicationStmtPart.agency (model class)|oddindex} groups the child elements of a <publicationStmt> element of the TEI header that indicate an authorising agent. [\xref{http://www.tei-c.org/release/doc/tei-p5-doc/en/html/HD.html\#HD24}{2.2.4. Publication, Distribution, Licensing, etc.}]\end{specHead} 
    \item[{Module}]
  tei
    \item[{Used by}]
  publicationStmt
    \item[{Members}]
  authority distributor publisher
    \item[{Note}]
  \par
The ‘agency’ child elements, while not required, are required if one of the ‘detail’ child elements is to be used. It is not valid to have a ‘detail’ child element without a preceding ‘agency’ child element.\par
See also \textsf{model.publicationStmtPart.detail}.
\end{reflist}  
\begin{reflist}
\item[]\begin{specHead}{TEI.model.publicationStmtPart.detail}{model.publicationStmtPart.detail}\index{model.publicationStmtPart.detail (model class)|oddindex} groups the agency-specific child elements of the <publicationStmt> element of the TEI header. [\xref{http://www.tei-c.org/release/doc/tei-p5-doc/en/html/HD.html\#HD24}{2.2.4. Publication, Distribution, Licensing, etc.}]\end{specHead} 
    \item[{Module}]
  tei
    \item[{Used by}]
  publicationStmt
    \item[{Members}]
  address availability date idno pubPlace
    \item[{Note}]
  \par
A ‘detail’ child element may not occur unless an ‘agency’ child element precedes it.\par
See also \textsf{model.publicationStmtPart.agency}.
\end{reflist}  
\begin{reflist}
\item[]\begin{specHead}{TEI.model.qLike}{model.qLike}\index{model.qLike (model class)|oddindex} groups elements related to highlighting which can appear either within or between chunk-level elements. [\xref{http://www.tei-c.org/release/doc/tei-p5-doc/en/html/CO.html\#COHQ}{3.3. Highlighting and Quotation}]\end{specHead} 
    \item[{Module}]
  tei
    \item[{Used by}]
  cit model.inter sp
    \item[{Members}]
  model.quoteLike[cit quote] floatingText q said
\end{reflist}  
\begin{reflist}
\item[]\begin{specHead}{TEI.model.quoteLike}{model.quoteLike}\index{model.quoteLike (model class)|oddindex} groups elements used to directly contain quotations.\end{specHead} 
    \item[{Module}]
  tei
    \item[{Used by}]
  model.msItemPart model.qLike
    \item[{Members}]
  cit quote
\end{reflist}  
\begin{reflist}
\item[]\begin{specHead}{TEI.model.rdgLike}{model.rdgLike}\index{model.rdgLike (model class)|oddindex} groups elements which contain a single reading, other than the lemma, within a textual variation. [\xref{http://www.tei-c.org/release/doc/tei-p5-doc/en/html/TC.html\#TCAPLL}{12.1. The Apparatus Entry, Readings, and Witnesses}]\end{specHead} 
    \item[{Module}]
  textcrit
    \item[{Used by}]
  app rdgGrp
    \item[{Members}]
  rdg
    \item[{Note}]
  \par
This class allows for variants of the <rdg> element to be easily created via TEI customizations.
\end{reflist}  
\begin{reflist}
\item[]\begin{specHead}{TEI.model.rdgPart}{model.rdgPart}\index{model.rdgPart (model class)|oddindex} groups elements which mark the beginning or ending of a fragmentary manuscript or other witness. [\xref{http://www.tei-c.org/release/doc/tei-p5-doc/en/html/TC.html\#TCAPMI}{12.1.5. Fragmentary Witnesses}]\end{specHead} 
    \item[{Module}]
  textcrit
    \item[{Used by}]
  lem rdg
    \item[{Members}]
  lacunaEnd lacunaStart wit witEnd witStart
    \item[{Note}]
  \par
These elements may appear anywhere within the elements <lem> and <rdg>, and also within any of their constituent elements.
\end{reflist}  
\begin{reflist}
\item[]\begin{specHead}{TEI.model.resourceLike}{model.resourceLike}\index{model.resourceLike (model class)|oddindex} groups separate elements which constitute the content of a digital resource, as opposed to its metadata. [\xref{http://www.tei-c.org/release/doc/tei-p5-doc/en/html/ST.html\#STEC}{1.3. The TEI Class System}]\end{specHead} 
    \item[{Module}]
  tei
    \item[{Used by}]
  TEI teiCorpus
    \item[{Members}]
  facsimile sourceDoc text
\end{reflist}  
\begin{reflist}
\item[]\begin{specHead}{TEI.model.respLike}{model.respLike}\index{model.respLike (model class)|oddindex} groups elements which are used to indicate intellectual or other significant responsibility, for example within a bibliographic element.\end{specHead} 
    \item[{Module}]
  tei
    \item[{Used by}]
  editionStmt model.biblPart model.msItemPart titleStmt
    \item[{Members}]
  author editor funder meeting principal respStmt sponsor
\end{reflist}  
\begin{reflist}
\item[]\begin{specHead}{TEI.model.segLike}{model.segLike}\index{model.segLike (model class)|oddindex} groups elements used for arbitrary segmentation. [\xref{http://www.tei-c.org/release/doc/tei-p5-doc/en/html/SA.html\#SASE}{16.3. Blocks, Segments, and Anchors} \xref{http://www.tei-c.org/release/doc/tei-p5-doc/en/html/AI.html\#AILC}{17.1. Linguistic Segment Categories}]\end{specHead} 
    \item[{Module}]
  tei
    \item[{Used by}]
  bibl model.phrase
    \item[{Members}]
  c cl m pc phr s seg w
    \item[{Note}]
  \par
The principles on which segmentation is carried out, and any special codes or attribute values used, should be defined explicitly in the <segmentation> element of the <encodingDesc> within the associated TEI header.
\end{reflist}  
\begin{reflist}
\item[]\begin{specHead}{TEI.model.stageLike}{model.stageLike}\index{model.stageLike (model class)|oddindex} groups elements containing stage directions or similar things defined by the module for performance texts. [\xref{http://www.tei-c.org/release/doc/tei-p5-doc/en/html/DR.html\#DROTH}{7.3. Other Types of Performance Text}]\end{specHead} 
    \item[{Module}]
  tei
    \item[{Used by}]
  lg model.inter sp
    \item[{Members}]
  stage
    \item[{Note}]
  \par
Stage directions are members of class \textit{inter}: that is, they can appear between or within component-level elements.
\end{reflist}  
\begin{reflist}
\item[]\begin{specHead}{TEI.model.teiHeaderPart}{model.teiHeaderPart}\index{model.teiHeaderPart (model class)|oddindex} groups high level elements which may appear more than once in a TEI header.\end{specHead} 
    \item[{Module}]
  tei
    \item[{Used by}]
  teiHeader
    \item[{Members}]
  encodingDesc profileDesc xenoData
\end{reflist}  
\begin{reflist}
\item[]\begin{specHead}{TEI.model.titlepagePart}{model.titlepagePart}\index{model.titlepagePart (model class)|oddindex} groups elements which can occur as direct constituents of a title page, such as <docTitle>, <docAuthor>, <docImprint>, or <epigraph>. [\xref{http://www.tei-c.org/release/doc/tei-p5-doc/en/html/DS.html\#DSTITL}{4.6. Title Pages}]\end{specHead} 
    \item[{Module}]
  tei
    \item[{Used by}]
  msItem titlePage
    \item[{Members}]
  argument byline docAuthor docDate docEdition docImprint docTitle epigraph graphic imprimatur titlePart
\end{reflist}  
\section[{Attribute classes}]{Attribute classes}
\begin{reflist}
\item[]\begin{specHead}{TEI.att.ascribed}{att.ascribed}\index{att.ascribed (attribute class)|oddindex}\index{who=@who!att.ascribed (attribute class)|oddindex} provides attributes for elements representing speech or action that can be ascribed to a specific individual. [\xref{http://www.tei-c.org/release/doc/tei-p5-doc/en/html/CO.html\#COHQQ}{3.3.3. Quotation} \xref{http://www.tei-c.org/release/doc/tei-p5-doc/en/html/TS.html\#TSBA}{8.3. Elements Unique to Spoken Texts}]\end{specHead} 
    \item[{Module}]
  tei
    \item[{Members}]
  change q said sp stage
    \item[{Attributes}]
  Attributes\hfil\\[-10pt]\begin{sansreflist}
    \item[@who]
  indicates the person, or group of people, to whom the element content is ascribed.
\begin{reflist}
    \item[{Status}]
  Optional
    \item[{Datatype}]
  1–∞ occurrences of teidata.pointer separated by whitespace
    \item[]In the following example from Hamlet, speeches (<sp>) in the body of the play are linked to \texttt{<castItem>} elements in the \texttt{<castList>} using the {\itshape who} attribute.\exampleFont {<\textbf{castItem}\hspace*{6pt}{type}="{role}">}\mbox{}\newline 
\hspace*{6pt}{<\textbf{role}\hspace*{6pt}{xml:id}="{Barnardo}">}Bernardo{</\textbf{role}>}\mbox{}\newline 
{</\textbf{castItem}>}\mbox{}\newline 
{<\textbf{castItem}\hspace*{6pt}{type}="{role}">}\mbox{}\newline 
\hspace*{6pt}{<\textbf{role}\hspace*{6pt}{xml:id}="{Francisco}">}Francisco{</\textbf{role}>}\mbox{}\newline 
\hspace*{6pt}{<\textbf{roleDesc}>}a soldier{</\textbf{roleDesc}>}\mbox{}\newline 
{</\textbf{castItem}>}\mbox{}\newline 
\textit{<!-- ... -->}\mbox{}\newline 
{<\textbf{sp}\hspace*{6pt}{who}="{\#Barnardo}">}\mbox{}\newline 
\hspace*{6pt}{<\textbf{speaker}>}Bernardo{</\textbf{speaker}>}\mbox{}\newline 
\hspace*{6pt}{<\textbf{l}\hspace*{6pt}{n}="{1}">}Who's there?{</\textbf{l}>}\mbox{}\newline 
{</\textbf{sp}>}\mbox{}\newline 
{<\textbf{sp}\hspace*{6pt}{who}="{\#Francisco}">}\mbox{}\newline 
\hspace*{6pt}{<\textbf{speaker}>}Francisco{</\textbf{speaker}>}\mbox{}\newline 
\hspace*{6pt}{<\textbf{l}\hspace*{6pt}{n}="{2}">}Nay, answer me: stand, and unfold yourself.{</\textbf{l}>}\mbox{}\newline 
{</\textbf{sp}>}
    \item[{Note}]
  \par
For transcribed speech, this will typically identify a participant or participant group; in other contexts, it will point to any identified <person> element.
\end{reflist}  
\end{sansreflist}  
\end{reflist}  
\begin{reflist}
\item[]\begin{specHead}{TEI.att.breaking}{att.breaking}\index{att.breaking (attribute class)|oddindex}\index{break=@break!att.breaking (attribute class)|oddindex} provides an attribute to indicate whether or not the element concerned is considered to mark the end of an orthographic token in the same way as whitespace. [\xref{http://www.tei-c.org/release/doc/tei-p5-doc/en/html/CO.html\#CORS5}{3.10.3. Milestone Elements}]\end{specHead} 
    \item[{Module}]
  tei
    \item[{Members}]
  cb gb lb milestone pb
    \item[{Attributes}]
  Attributes\hfil\\[-10pt]\begin{sansreflist}
    \item[@break]
  indicates whether or not the element bearing this attribute should be considered to mark the end of an orthographic token in the same way as whitespace.
\begin{reflist}
    \item[{Status}]
  Recommended
    \item[{Datatype}]
  teidata.enumerated
    \item[{Sample values include}]
  \begin{description}

\item[{yes}]the element bearing this attribute is considered to mark the end of any adjacent orthographic token irrespective of the presence of any adjacent whitespace
\item[{no}]the element bearing this attribute is considered not to mark the end of any adjacent orthographic token irrespective of the presence of any adjacent whitespace
\item[{maybe}]the encoding does not take any position on this issue.
\end{description} 
    \item[]In the following lines from the ‘Dream of the Rood’, linebreaks occur in the middle of the words \textit{lāðost} and \textit{reord-berendum}.\exampleFont {<\textbf{ab}>} ...eƿesa tome iu icƿæs ȝeƿorden ƿita heardoſt .\mbox{}\newline 
 leodum la{<\textbf{lb}\hspace*{6pt}{break}="{no}"/>} ðost ærþan ichim lifes\mbox{}\newline 
 ƿeȝ rihtne ȝerymde reord be{<\textbf{lb}\hspace*{6pt}{break}="{no}"/>}\mbox{}\newline 
 rendum hƿæt me þaȝeƿeorðode ƿuldres ealdor ofer...\mbox{}\newline 
{</\textbf{ab}>}
\end{reflist}  
\end{sansreflist}  
\end{reflist}  
\begin{reflist}
\item[]\begin{specHead}{TEI.att.cReferencing}{att.cReferencing}\index{att.cReferencing (attribute class)|oddindex}\index{cRef=@cRef!att.cReferencing (attribute class)|oddindex} provides an attribute which may be used to supply a \textit{canonical reference} as a means of identifying the target of a pointer.\end{specHead} 
    \item[{Module}]
  tei
    \item[{Members}]
  gloss ptr ref term
    \item[{Attributes}]
  Attributes\hfil\\[-10pt]\begin{sansreflist}
    \item[@cRef]
  (canonical reference) specifies the destination of the pointer by supplying a canonical reference expressed using the scheme defined in a <refsDecl> element in the TEI header
\begin{reflist}
    \item[{Status}]
  Optional
    \item[{Datatype}]
  teidata.text
    \item[{Note}]
  \par
The value of {\itshape cRef} should be constructed so that when the algorithm for the resolution of canonical references (described in section \xref{http://www.tei-c.org/release/doc/tei-p5-doc/en/html/SA.html\#SACR}{16.2.5. Canonical References}) is applied to it the result is a valid URI reference to the intended target\par
The <refsDecl> to use may be indicated with the {\itshape decls} attribute.\par
Currently these Guidelines only provide for a single canonical reference to be encoded on any given <ptr> element.
\end{reflist}  
\end{sansreflist}  
\end{reflist}  
\begin{reflist}
\item[]\begin{specHead}{TEI.att.canonical}{att.canonical}\index{att.canonical (attribute class)|oddindex}\index{key=@key!att.canonical (attribute class)|oddindex}\index{ref=@ref!att.canonical (attribute class)|oddindex} provides attributes which can be used to associate a representation such as a name or title with canonical information about the object being named or referenced.\end{specHead} 
    \item[{Module}]
  tei
    \item[{Members}]
  att.naming[att.personal[addName forename genName name orgName persName placeName roleName surname] affiliation author birth bloc climate collection country death district editor education event geogFeat geogName institution nationality occupation offset origPlace population pubPlace region repository residence rs settlement socecStatus state terrain trait] correspDesc docAuthor docTitle faith funder material meeting objectType principal relation resp respStmt sponsor term title
    \item[{Attributes}]
  Attributes\hfil\\[-10pt]\begin{sansreflist}
    \item[@key]
  provides an externally-defined means of identifying the entity (or entities) being named, using a coded value of some kind.
\begin{reflist}
    \item[{Status}]
  Optional
    \item[{Datatype}]
  teidata.text
    \item[]\exampleFont {<\textbf{author}>}\mbox{}\newline 
\hspace*{6pt}{<\textbf{name}\hspace*{6pt}{key}="{name 427308}"\mbox{}\newline 
\hspace*{6pt}\hspace*{6pt}{type}="{organisation}">}[New Zealand Parliament, Legislative Council]{</\textbf{name}>}\mbox{}\newline 
{</\textbf{author}>}
    \item[]\exampleFont {<\textbf{author}>}\mbox{}\newline 
\hspace*{6pt}{<\textbf{name}\hspace*{6pt}{key}="{Hugo, Victor (1802-1885)}"\mbox{}\newline 
\hspace*{6pt}\hspace*{6pt}{ref}="{http://www.idref.fr/026927608}">}Victor Hugo{</\textbf{name}>}\mbox{}\newline 
{</\textbf{author}>}
    \item[{Note}]
  \par
The value may be a unique identifier from a database, or any other externally-defined string identifying the referent. \par
No particular syntax is proposed for the values of the {\itshape key} attribute, since its form will depend entirely on practice within a given project. For the same reason, this attribute is not recommended in data interchange, since there is no way of ensuring that the values used by one project are distinct from those used by another. In such a situation, a preferable approach for magic tokens which follows standard practice on the Web is to use a {\itshape ref} attribute whose value is a tag URI as defined in RFC 4151.
\end{reflist}  
    \item[@ref]
  (reference) provides an explicit means of locating a full definition or identity for the entity being named by means of one or more URIs.
\begin{reflist}
    \item[{Status}]
  Optional
    \item[{Datatype}]
  1–∞ occurrences of teidata.pointer separated by whitespace
    \item[]\exampleFont {<\textbf{name}\hspace*{6pt}{ref}="{http://viaf.org/viaf/109557338}"\mbox{}\newline 
\hspace*{6pt}{type}="{person}">}Seamus Heaney{</\textbf{name}>}
    \item[{Note}]
  \par
The value must point directly to one or more XML elements or other resources by means of one or more URIs, separated by whitespace. If more than one is supplied the implication is that the name identifies several distinct entities.
\end{reflist}  
\end{sansreflist}  
\end{reflist}  
\begin{reflist}
\item[]\begin{specHead}{TEI.att.citing}{att.citing}\index{att.citing (attribute class)|oddindex}\index{unit=@unit!att.citing (attribute class)|oddindex}\index{from=@from!att.citing (attribute class)|oddindex}\index{to=@to!att.citing (attribute class)|oddindex} provides attributes for specifying the specific part of a bibliographic item being cited. [\xref{http://www.tei-c.org/release/doc/tei-p5-doc/en/html/ST.html\#STECAT}{1.3.1. Attribute Classes}]\end{specHead} 
    \item[{Module}]
  tei
    \item[{Members}]
  biblScope citedRange
    \item[{Attributes}]
  Attributes\hfil\\[-10pt]\begin{sansreflist}
    \item[@unit]
  identifies the unit of information conveyed by the element, e.g. columns, pages, volume.
\begin{reflist}
    \item[{Status}]
  Optional
    \item[{Datatype}]
  teidata.enumerated
    \item[{Suggested values include:}]
  \begin{description}

\item[{volume}]the element contains a volume number.
\item[{issue}]the element contains an issue number, or volume and issue numbers.
\item[{page}]the element contains a page number or page range.
\item[{line}]the element contains a line number or line range.
\item[{chapter}]the element contains a chapter indication (number and/or title)
\item[{part}]the element identifies a part of a book or collection.
\item[{column}]the element identifies a column.
\end{description} 
\end{reflist}  
    \item[@from]
  specifies the starting point of the range of units indicated by the {\itshape unit} attribute.
\begin{reflist}
    \item[{Status}]
  Optional
    \item[{Datatype}]
  teidata.word
\end{reflist}  
    \item[@to]
  specifies the end-point of the range of units indicated by the {\itshape unit} attribute.
\begin{reflist}
    \item[{Status}]
  Optional
    \item[{Datatype}]
  teidata.word
\end{reflist}  
\end{sansreflist}  
\end{reflist}  
\begin{reflist}
\item[]\begin{specHead}{TEI.att.coordinated}{att.coordinated}\index{att.coordinated (attribute class)|oddindex}\index{start=@start!att.coordinated (attribute class)|oddindex}\index{ulx=@ulx!att.coordinated (attribute class)|oddindex}\index{uly=@uly!att.coordinated (attribute class)|oddindex}\index{lrx=@lrx!att.coordinated (attribute class)|oddindex}\index{lry=@lry!att.coordinated (attribute class)|oddindex}\index{points=@points!att.coordinated (attribute class)|oddindex} provides attributes which can be used to position their parent element within a two dimensional coordinate system.\end{specHead} 
    \item[{Module}]
  transcr
    \item[{Members}]
  line surface zone
    \item[{Attributes}]
  Attributes\hfil\\[-10pt]\begin{sansreflist}
    \item[@start]
  indicates the element within a transcription of the text containing at least the start of the writing represented by this zone or surface.
\begin{reflist}
    \item[{Status}]
  Optional
    \item[{Datatype}]
  teidata.pointer
\end{reflist}  
    \item[@ulx]
  gives the x coordinate value for the upper left corner of a rectangular space.
\begin{reflist}
    \item[{Status}]
  Optional
    \item[{Datatype}]
  teidata.numeric
\end{reflist}  
    \item[@uly]
  gives the y coordinate value for the upper left corner of a rectangular space.
\begin{reflist}
    \item[{Status}]
  Optional
    \item[{Datatype}]
  teidata.numeric
\end{reflist}  
    \item[@lrx]
  gives the x coordinate value for the lower right corner of a rectangular space.
\begin{reflist}
    \item[{Status}]
  Optional
    \item[{Datatype}]
  teidata.numeric
\end{reflist}  
    \item[@lry]
  gives the y coordinate value for the lower right corner of a rectangular space.
\begin{reflist}
    \item[{Status}]
  Optional
    \item[{Datatype}]
  teidata.numeric
\end{reflist}  
    \item[@points]
  identifies a two dimensional area within the bounding box specified by the other attributes by means of a series of pairs of numbers, each of which gives the x,y coordinates of a point on a line enclosing the area.
\begin{reflist}
    \item[{Status}]
  Optional
    \item[{Datatype}]
  3–∞ occurrences of teidata.point separated by whitespace
\end{reflist}  
\end{sansreflist}  
\end{reflist}  
\begin{reflist}
\item[]\begin{specHead}{TEI.att.damaged}{att.damaged}\index{att.damaged (attribute class)|oddindex}\index{agent=@agent!att.damaged (attribute class)|oddindex}\index{degree=@degree!att.damaged (attribute class)|oddindex}\index{group=@group!att.damaged (attribute class)|oddindex} provides attributes describing the nature of any physical damage affecting a reading. [\xref{http://www.tei-c.org/release/doc/tei-p5-doc/en/html/PH.html\#PHDA}{11.3.3.1. Damage, Illegibility, and Supplied Text} \xref{http://www.tei-c.org/release/doc/tei-p5-doc/en/html/ST.html\#STECAT}{1.3.1. Attribute Classes}]\end{specHead} 
    \item[{Module}]
  tei
    \item[{Members}]
  damage damageSpan
    \item[{Attributes}]
  Attributes att.dimensions (\textit{@unit}, \textit{@quantity}, \textit{@extent}, \textit{@precision}, \textit{@scope})  (att.ranging (\textit{@atLeast}, \textit{@atMost}, \textit{@min}, \textit{@max}, \textit{@confidence})) att.written (\textit{@hand}) \hfil\\[-10pt]\begin{sansreflist}
    \item[@agent]
  categorizes the cause of the damage, if it can be identified.
\begin{reflist}
    \item[{Status}]
  Optional
    \item[{Datatype}]
  teidata.enumerated
    \item[{Sample values include:}]
  \begin{description}

\item[{rubbing}]damage results from rubbing of the leaf edges
\item[{mildew}]damage results from mildew on the leaf surface
\item[{smoke}]damage results from smoke
\end{description} 
\end{reflist}  
    \item[@degree]
  provides a coded representation of the degree of damage, either as a number between 0 (undamaged) and 1 (very extensively damaged), or as one of the codes high, medium, low, or unknown. The <damage> element with the {\itshape degree} attribute should only be used where the text may be read with some confidence; text supplied from other sources should be tagged as <supplied>.
\begin{reflist}
    \item[{Status}]
  Optional
    \item[{Datatype}]
  teidata.probCert
    \item[{Note}]
  \par
The <damage> element is appropriate where it is desired to record the fact of damage although this has not affected the readability of the text, for example a weathered inscription. Where the damage has rendered the text more or less illegible either the <unclear> tag (for partial illegibility) or the <gap> tag (for complete illegibility, with no text supplied) should be used, with the information concerning the damage given in the attribute values of these tags. See section \xref{http://www.tei-c.org/release/doc/tei-p5-doc/en/html/PH.html\#PHCOMB}{11.3.3.2. Use of the gap, del, damage, unclear, and supplied Elements in Combination} for discussion of the use of these tags in particular circumstances.
\end{reflist}  
    \item[@group]
  assigns an arbitrary number to each stretch of damage regarded as forming part of the same physical phenomenon.
\begin{reflist}
    \item[{Status}]
  Optional
    \item[{Datatype}]
  teidata.count
\end{reflist}  
\end{sansreflist}  
\end{reflist}  
\begin{reflist}
\item[]\begin{specHead}{TEI.att.datable}{att.datable}\index{att.datable (attribute class)|oddindex}\index{calendar=@calendar!att.datable (attribute class)|oddindex}\index{period=@period!att.datable (attribute class)|oddindex} provides attributes for normalization of elements that contain dates, times, or datable events. [\xref{http://www.tei-c.org/release/doc/tei-p5-doc/en/html/CO.html\#CONADA}{3.5.4. Dates and Times} \xref{http://www.tei-c.org/release/doc/tei-p5-doc/en/html/ND.html\#NDDATE}{13.3.6. Dates and Times}]\end{specHead} 
    \item[{Module}]
  tei
    \item[{Members}]
  acquisition affiliation age application binding birth bloc change climate country creation custEvent date death district education event faith floruit geogFeat geogName idno langKnowledge langKnown licence location name nationality occupation offset orgName origDate origPlace origin persName placeName population provenance region relation residence resp seal settlement sex socecStatus stamp state terrain time title trait
    \item[{Attributes}]
  Attributes att.datable.w3c (\textit{@when}, \textit{@notBefore}, \textit{@notAfter}, \textit{@from}, \textit{@to}) att.datable.iso (\textit{@when-iso}, \textit{@notBefore-iso}, \textit{@notAfter-iso}, \textit{@from-iso}, \textit{@to-iso}) att.datable.custom (\textit{@when-custom}, \textit{@notBefore-custom}, \textit{@notAfter-custom}, \textit{@from-custom}, \textit{@to-custom}, \textit{@datingPoint}, \textit{@datingMethod}) \hfil\\[-10pt]\begin{sansreflist}
    \item[@calendar]
  indicates the system or calendar to which the date represented by the content of this element belongs.
\begin{reflist}
    \item[{Status}]
  Optional
    \item[{Datatype}]
  teidata.pointer
    \item[{Schematron}]
   <sch:rule context="tei:*[@calendar]"> <sch:assert test="string-length(.) gt 0">@calendar indicates the system or calendar to which the date represented by the content of this element  belongs, but this <sch:name/> element has no textual content.</sch:assert> </sch:rule>
    \item[]\exampleFont He was born on {<\textbf{date}\hspace*{6pt}{calendar}="{\#gregorian}">}Feb. 22, 1732{</\textbf{date}>} \mbox{}\newline 
 ({<\textbf{date}\hspace*{6pt}{calendar}="{\#julian}"\mbox{}\newline 
\hspace*{6pt}{when}="{1732-02-22}">} Feb. 11, 1731/32, O.S.{</\textbf{date}>}).
    \item[{Note}]
  \par
Note that the {\itshape calendar} attribute (unlike {\itshape datingMethod} defined in \textsf{att.datable.custom}) defines the calendar system of the date in the original material defined by the parent element, \textit{not} the calendar to which the date is normalized.
\end{reflist}  
    \item[@period]
  supplies a pointer to some location defining a named period of time within which the datable item is understood to have occurred.
\begin{reflist}
    \item[{Status}]
  Optional
    \item[{Datatype}]
  teidata.pointer
\end{reflist}  
\end{sansreflist}  
    \item[{Note}]
  \par
This ‘superclass’ provides attributes that can be used to provide normalized values of temporal information. By default, the attributes from the \textsf{att.datable.w3c} class are provided. If the module for names \& dates is loaded, this class also provides attributes from the \textsf{att.datable.iso} and \textsf{att.datable.custom} classes. In general, the possible values of attributes restricted to the W3C datatypes form a subset of those values available via the ISO 8601 standard. However, the greater expressiveness of the ISO datatypes may not be needed, and there exists much greater software support for the W3C datatypes.
\end{reflist}  
\begin{reflist}
\item[]\begin{specHead}{TEI.att.datable.custom}{att.datable.custom}\index{att.datable.custom (attribute class)|oddindex}\index{when-custom=@when-custom!att.datable.custom (attribute class)|oddindex}\index{notBefore-custom=@notBefore-custom!att.datable.custom (attribute class)|oddindex}\index{notAfter-custom=@notAfter-custom!att.datable.custom (attribute class)|oddindex}\index{from-custom=@from-custom!att.datable.custom (attribute class)|oddindex}\index{to-custom=@to-custom!att.datable.custom (attribute class)|oddindex}\index{datingPoint=@datingPoint!att.datable.custom (attribute class)|oddindex}\index{datingMethod=@datingMethod!att.datable.custom (attribute class)|oddindex} provides attributes for normalization of elements that contain datable events to a custom dating system (i.e. other than the Gregorian used by W3 and ISO). [\xref{http://www.tei-c.org/release/doc/tei-p5-doc/en/html/ND.html\#NDDATE}{13.3.6. Dates and Times}]\end{specHead} 
    \item[{Module}]
  namesdates
    \item[{Members}]
  att.datable[acquisition affiliation age application binding birth bloc change climate country creation custEvent date death district education event faith floruit geogFeat geogName idno langKnowledge langKnown licence location name nationality occupation offset orgName origDate origPlace origin persName placeName population provenance region relation residence resp seal settlement sex socecStatus stamp state terrain time title trait]
    \item[{Attributes}]
  Attributes\hfil\\[-10pt]\begin{sansreflist}
    \item[@when-custom]
  supplies the value of a date or time in some custom standard form.
\begin{reflist}
    \item[{Status}]
  Optional
    \item[{Datatype}]
  1–∞ occurrences of teidata.word separated by whitespace
    \item[]The following are examples of custom date or time formats that are \textit{not} valid ISO or W3C format normalizations, normalized to a different dating system\exampleFont {<\textbf{p}>}Alhazen died in Cairo on the\mbox{}\newline 
{<\textbf{date}\hspace*{6pt}{when}="{1040-03-06}"\mbox{}\newline 
\hspace*{6pt}\hspace*{6pt}{when-custom}="{431-06-12}">} 12th day of Jumada t-Tania, 430 AH\mbox{}\newline 
\hspace*{6pt}{</\textbf{date}>}.{</\textbf{p}>}\mbox{}\newline 
{<\textbf{p}>}The current world will end at the\mbox{}\newline 
{<\textbf{date}\hspace*{6pt}{when}="{2012-12-21}"\mbox{}\newline 
\hspace*{6pt}\hspace*{6pt}{when-custom}="{13.0.0.0.0}">}end of B'ak'tun 13{</\textbf{date}>}.{</\textbf{p}>}\mbox{}\newline 
{<\textbf{p}>}The Battle of Meggidu\mbox{}\newline 
 ({<\textbf{date}\hspace*{6pt}{when-custom}="{Thutmose\textunderscore III:23}">}23rd year of reign of Thutmose III{</\textbf{date}>}).{</\textbf{p}>}\mbox{}\newline 
{<\textbf{p}>}Esidorus bixit in pace annos LXX plus minus sub\mbox{}\newline 
{<\textbf{date}\hspace*{6pt}{when-custom}="{Ind:4-10-11}">}die XI mensis Octobris indictione IIII{</\textbf{date}>}\mbox{}\newline 
{</\textbf{p}>}Not all custom date formulations will have Gregorian equivalents.The {\itshape when-custom} attribute and other custom dating are not contrained to a datatype by the TEI, but individual projects are recommended to regularize and document their dating formats.
\end{reflist}  
    \item[@notBefore-custom]
  specifies the earliest possible date for the event in some custom standard form.
\begin{reflist}
    \item[{Status}]
  Optional
    \item[{Datatype}]
  1–∞ occurrences of teidata.word separated by whitespace
\end{reflist}  
    \item[@notAfter-custom]
  specifies the latest possible date for the event in some custom standard form.
\begin{reflist}
    \item[{Status}]
  Optional
    \item[{Datatype}]
  1–∞ occurrences of teidata.word separated by whitespace
\end{reflist}  
    \item[@from-custom]
  indicates the starting point of the period in some custom standard form.
\begin{reflist}
    \item[{Status}]
  Optional
    \item[{Datatype}]
  1–∞ occurrences of teidata.word separated by whitespace
    \item[]\exampleFont {<\textbf{event}\hspace*{6pt}{datingMethod}="{\#julian}"\mbox{}\newline 
\hspace*{6pt}{from-custom}="{1666-09-02}"\mbox{}\newline 
\hspace*{6pt}{to-custom}="{1666-09-05}"\mbox{}\newline 
\hspace*{6pt}{xml:id}="{FIRE1}">}\mbox{}\newline 
\hspace*{6pt}{<\textbf{head}>}The Great Fire of London{</\textbf{head}>}\mbox{}\newline 
\hspace*{6pt}{<\textbf{p}>}The Great Fire of London burned through a large part\mbox{}\newline 
\hspace*{6pt}\hspace*{6pt} of the city of London.{</\textbf{p}>}\mbox{}\newline 
{</\textbf{event}>}
\end{reflist}  
    \item[@to-custom]
  indicates the ending point of the period in some custom standard form.
\begin{reflist}
    \item[{Status}]
  Optional
    \item[{Datatype}]
  1–∞ occurrences of teidata.word separated by whitespace
\end{reflist}  
    \item[@datingPoint]
  supplies a pointer to some location defining a named point in time with reference to which the datable item is understood to have occurred
\begin{reflist}
    \item[{Status}]
  Optional
    \item[{Datatype}]
  teidata.pointer
\end{reflist}  
    \item[@datingMethod]
  supplies a pointer to a <calendar> element or other means of interpreting the values of the custom dating attributes.
\begin{reflist}
    \item[{Status}]
  Optional
    \item[{Datatype}]
  teidata.pointer
    \item[]\exampleFont Contayning the Originall, Antiquity, Increaſe, Moderne\mbox{}\newline 
 eſtate, and deſcription of that Citie, written in the yeare\mbox{}\newline 
{<\textbf{date}\hspace*{6pt}{calendar}="{\#julian}"\mbox{}\newline 
\hspace*{6pt}{datingMethod}="{\#julian}"\mbox{}\newline 
\hspace*{6pt}{when-custom}="{1598}">}1598{</\textbf{date}>}. by Iohn Stow\mbox{}\newline 
 Citizen of London.In this example, the {\itshape calendar} attribute points to a <calendar> element for the Julian calendar, specifying that the text content of the <date> element is a Julian date, and the {\itshape datingMethod} attribute also points to the Julian calendar to indicate that the content of the {\itshape when-custom} attribute value is Julian too.
    \item[]\exampleFont {<\textbf{date}\hspace*{6pt}{datingMethod}="{\#creationOfWorld}"\mbox{}\newline 
\hspace*{6pt}{when}="{1382-06-28}"\mbox{}\newline 
\hspace*{6pt}{when-custom}="{6890-06-20}">} μηνὶ Ἰουνίου εἰς {<\textbf{num}>}κ{</\textbf{num}>} ἔτους {<\textbf{num}>}ςωϞ{</\textbf{num}>}\mbox{}\newline 
{</\textbf{date}>}In this example, a date is given in a Mediaeval text measured "from the creation of the world", which is normalised (in {\itshape when}) to the Gregorian date, but is also normalized (in {\itshape when-custom}) to a machine-actionable, numeric version of the date from the Creation.
    \item[{Note}]
  \par
Note that the {\itshape datingMethod} attribute (unlike {\itshape calendar} defined in \textsf{att.datable}) defines the calendar or dating system to which the date described by the parent element is normalized (i.e. in the {\itshape when-custom} or other {\itshape X-custom} attributes), \textit{not} the calendar of the original date in the element.
\end{reflist}  
\end{sansreflist}  
\end{reflist}  
\begin{reflist}
\item[]\begin{specHead}{TEI.att.datable.iso}{att.datable.iso}\index{att.datable.iso (attribute class)|oddindex}\index{when-iso=@when-iso!att.datable.iso (attribute class)|oddindex}\index{notBefore-iso=@notBefore-iso!att.datable.iso (attribute class)|oddindex}\index{notAfter-iso=@notAfter-iso!att.datable.iso (attribute class)|oddindex}\index{from-iso=@from-iso!att.datable.iso (attribute class)|oddindex}\index{to-iso=@to-iso!att.datable.iso (attribute class)|oddindex} provides attributes for normalization of elements that contain datable events using the ISO 8601 standard. [\xref{http://www.tei-c.org/release/doc/tei-p5-doc/en/html/CO.html\#CONADA}{3.5.4. Dates and Times} \xref{http://www.tei-c.org/release/doc/tei-p5-doc/en/html/ND.html\#NDDATE}{13.3.6. Dates and Times}]\end{specHead} 
    \item[{Module}]
  namesdates
    \item[{Members}]
  att.datable[acquisition affiliation age application binding birth bloc change climate country creation custEvent date death district education event faith floruit geogFeat geogName idno langKnowledge langKnown licence location name nationality occupation offset orgName origDate origPlace origin persName placeName population provenance region relation residence resp seal settlement sex socecStatus stamp state terrain time title trait]
    \item[{Attributes}]
  Attributes\hfil\\[-10pt]\begin{sansreflist}
    \item[@when-iso]
  supplies the value of a date or time in a standard form.
\begin{reflist}
    \item[{Status}]
  Optional
    \item[{Datatype}]
  teidata.temporal.iso
    \item[]The following are examples of ISO date, time, and date \& time formats that are \textit{not} valid W3C format normalizations.\exampleFont {<\textbf{date}\hspace*{6pt}{when-iso}="{1996-09-24T07:25+00}">}Sept. 24th, 1996 at 3:25 in the morning{</\textbf{date}>}\mbox{}\newline 
{<\textbf{date}\hspace*{6pt}{when-iso}="{1996-09-24T03:25-04}">}Sept. 24th, 1996 at 3:25 in the morning{</\textbf{date}>}\mbox{}\newline 
{<\textbf{time}\hspace*{6pt}{when-iso}="{1999-01-04T20:42-05}">}4 Jan 1999 at 8:42 pm{</\textbf{time}>}\mbox{}\newline 
{<\textbf{time}\hspace*{6pt}{when-iso}="{1999-W01-1T20,70-05}">}4 Jan 1999 at 8:42 pm{</\textbf{time}>}\mbox{}\newline 
{<\textbf{date}\hspace*{6pt}{when-iso}="{2006-05-18T10:03}">}a few minutes after ten in the morning on Thu 18 May{</\textbf{date}>}\mbox{}\newline 
{<\textbf{time}\hspace*{6pt}{when-iso}="{03:00}">}3 A.M.{</\textbf{time}>}\mbox{}\newline 
{<\textbf{time}\hspace*{6pt}{when-iso}="{14}">}around two{</\textbf{time}>}\mbox{}\newline 
{<\textbf{time}\hspace*{6pt}{when-iso}="{15,5}">}half past three{</\textbf{time}>}All of the examples of the {\itshape when} attribute in the \textsf{att.datable.w3c} class are also valid with respect to this attribute.
    \item[]\exampleFont He likes to be punctual. I said {<\textbf{q}>}\mbox{}\newline 
\hspace*{6pt}{<\textbf{time}\hspace*{6pt}{when-iso}="{12}">}around noon{</\textbf{time}>}\mbox{}\newline 
{</\textbf{q}>}, and he showed up at {<\textbf{time}\hspace*{6pt}{when-iso}="{12:00:00}">}12 O'clock{</\textbf{time}>} on the dot.The second occurence of <time> could have been encoded with the {\itshape when} attribute, as 12:00:00 is a valid time with respect to the W3C \textit{XML Schema Part 2: Datatypes Second Edition} specification. The first occurence could not.
\end{reflist}  
    \item[@notBefore-iso]
  specifies the earliest possible date for the event in standard form, e.g. yyyy-mm-dd.
\begin{reflist}
    \item[{Status}]
  Optional
    \item[{Datatype}]
  teidata.temporal.iso
\end{reflist}  
    \item[@notAfter-iso]
  specifies the latest possible date for the event in standard form, e.g. yyyy-mm-dd.
\begin{reflist}
    \item[{Status}]
  Optional
    \item[{Datatype}]
  teidata.temporal.iso
\end{reflist}  
    \item[@from-iso]
  indicates the starting point of the period in standard form.
\begin{reflist}
    \item[{Status}]
  Optional
    \item[{Datatype}]
  teidata.temporal.iso
\end{reflist}  
    \item[@to-iso]
  indicates the ending point of the period in standard form.
\begin{reflist}
    \item[{Status}]
  Optional
    \item[{Datatype}]
  teidata.temporal.iso
\end{reflist}  
\end{sansreflist}  
    \item[{Note}]
  \par
The value of these attributes should be a normalized representation of the date, time, or combined date \& time intended, in any of the standard formats specified by ISO 8601, using the Gregorian calendar.
    \item[{Note}]
  \par
If both {\itshape when-iso} and {\itshape dur-iso} are specified, the values should be interpreted as indicating a span of time by its starting time (or date) and duration. That is, \par\bgroup\exampleFont \begin{shaded}\noindent\mbox{}{<\textbf{date}\hspace*{6pt}{dur-iso}="{P8D}"\hspace*{6pt}{when-iso}="{2007-06-01}"/>}\end{shaded}\egroup\par \noindent  indicates the same time period as \par\bgroup\exampleFont \begin{shaded}\noindent\mbox{}{<\textbf{date}\hspace*{6pt}{when-iso}="{2007-06-01/P8D}"/>}\end{shaded}\egroup\par \par
In providing a ‘regularized’ form, no claim is made that the form in the source text is incorrect; the regularized form is simply that chosen as the main form for purposes of unifying variant forms under a single heading.
\end{reflist}  
\begin{reflist}
\item[]\begin{specHead}{TEI.att.datable.w3c}{att.datable.w3c}\index{att.datable.w3c (attribute class)|oddindex}\index{when=@when!att.datable.w3c (attribute class)|oddindex}\index{notBefore=@notBefore!att.datable.w3c (attribute class)|oddindex}\index{notAfter=@notAfter!att.datable.w3c (attribute class)|oddindex}\index{from=@from!att.datable.w3c (attribute class)|oddindex}\index{to=@to!att.datable.w3c (attribute class)|oddindex} provides attributes for normalization of elements that contain datable events conforming to the W3C \textit{XML Schema Part 2: Datatypes Second Edition}. [\xref{http://www.tei-c.org/release/doc/tei-p5-doc/en/html/CO.html\#CONADA}{3.5.4. Dates and Times} \xref{http://www.tei-c.org/release/doc/tei-p5-doc/en/html/ND.html\#NDDATE}{13.3.6. Dates and Times}]\end{specHead} 
    \item[{Module}]
  tei
    \item[{Members}]
  att.datable[acquisition affiliation age application binding birth bloc change climate country creation custEvent date death district education event faith floruit geogFeat geogName idno langKnowledge langKnown licence location name nationality occupation offset orgName origDate origPlace origin persName placeName population provenance region relation residence resp seal settlement sex socecStatus stamp state terrain time title trait]
    \item[{Attributes}]
  Attributes\hfil\\[-10pt]\begin{sansreflist}
    \item[@when]
  supplies the value of the date or time in a standard form, e.g. yyyy-mm-dd.
\begin{reflist}
    \item[{Status}]
  Optional
    \item[{Datatype}]
  teidata.temporal.w3c
    \item[]Examples of W3C date, time, and date \& time formats.\exampleFont {<\textbf{p}>}\mbox{}\newline 
\hspace*{6pt}{<\textbf{date}\hspace*{6pt}{when}="{1945-10-24}">}24 Oct 45{</\textbf{date}>}\mbox{}\newline 
\hspace*{6pt}{<\textbf{date}\hspace*{6pt}{when}="{1996-09-24T07:25:00Z}">}September 24th, 1996 at 3:25 in the morning{</\textbf{date}>}\mbox{}\newline 
\hspace*{6pt}{<\textbf{time}\hspace*{6pt}{when}="{1999-01-04T20:42:00-05:00}">}Jan 4 1999 at 8 pm{</\textbf{time}>}\mbox{}\newline 
\hspace*{6pt}{<\textbf{time}\hspace*{6pt}{when}="{14:12:38}">}fourteen twelve and 38 seconds{</\textbf{time}>}\mbox{}\newline 
\hspace*{6pt}{<\textbf{date}\hspace*{6pt}{when}="{1962-10}">}October of 1962{</\textbf{date}>}\mbox{}\newline 
\hspace*{6pt}{<\textbf{date}\hspace*{6pt}{when}="{--06-12}">}June 12th{</\textbf{date}>}\mbox{}\newline 
\hspace*{6pt}{<\textbf{date}\hspace*{6pt}{when}="{---01}">}the first of the month{</\textbf{date}>}\mbox{}\newline 
\hspace*{6pt}{<\textbf{date}\hspace*{6pt}{when}="{--08}">}August{</\textbf{date}>}\mbox{}\newline 
\hspace*{6pt}{<\textbf{date}\hspace*{6pt}{when}="{2006}">}MMVI{</\textbf{date}>}\mbox{}\newline 
\hspace*{6pt}{<\textbf{date}\hspace*{6pt}{when}="{0056}">}AD 56{</\textbf{date}>}\mbox{}\newline 
\hspace*{6pt}{<\textbf{date}\hspace*{6pt}{when}="{-0056}">}56 BC{</\textbf{date}>}\mbox{}\newline 
{</\textbf{p}>}
    \item[]\exampleFont This list begins in\mbox{}\newline 
 the year 1632, more precisely on Trinity Sunday, i.e. the Sunday after\mbox{}\newline 
 Pentecost, in that year the\mbox{}\newline 
{<\textbf{date}\hspace*{6pt}{calendar}="{\#julian}"\mbox{}\newline 
\hspace*{6pt}{when}="{1632-06-06}">}27th of May (old style){</\textbf{date}>}.
    \item[]\exampleFont {<\textbf{opener}>}\mbox{}\newline 
\hspace*{6pt}{<\textbf{dateline}>}\mbox{}\newline 
\hspace*{6pt}\hspace*{6pt}{<\textbf{placeName}>}Dorchester, Village,{</\textbf{placeName}>}\mbox{}\newline 
\hspace*{6pt}\hspace*{6pt}{<\textbf{date}\hspace*{6pt}{when}="{1828-03-02}">}March 2d. 1828.{</\textbf{date}>}\mbox{}\newline 
\hspace*{6pt}{</\textbf{dateline}>}\mbox{}\newline 
\hspace*{6pt}{<\textbf{salute}>}To\mbox{}\newline 
\hspace*{6pt}\hspace*{6pt} Mrs. Cornell,{</\textbf{salute}>} Sunday {<\textbf{time}\hspace*{6pt}{when}="{12:00:00}">}noon.{</\textbf{time}>}\mbox{}\newline 
{</\textbf{opener}>}
\end{reflist}  
    \item[@notBefore]
  specifies the earliest possible date for the event in standard form, e.g. yyyy-mm-dd.
\begin{reflist}
    \item[{Status}]
  Optional
    \item[{Datatype}]
  teidata.temporal.w3c
\end{reflist}  
    \item[@notAfter]
  specifies the latest possible date for the event in standard form, e.g. yyyy-mm-dd.
\begin{reflist}
    \item[{Status}]
  Optional
    \item[{Datatype}]
  teidata.temporal.w3c
\end{reflist}  
    \item[@from]
  indicates the starting point of the period in standard form, e.g. yyyy-mm-dd.
\begin{reflist}
    \item[{Status}]
  Optional
    \item[{Datatype}]
  teidata.temporal.w3c
\end{reflist}  
    \item[@to]
  indicates the ending point of the period in standard form, e.g. yyyy-mm-dd.
\begin{reflist}
    \item[{Status}]
  Optional
    \item[{Datatype}]
  teidata.temporal.w3c
\end{reflist}  
\end{sansreflist}  
    \item[{Schematron}]
   <sch:rule context="tei:*[@when]"> <sch:report role="nonfatal"  test="@notBefore|@notAfter|@from|@to">The @when attribute cannot be used with any other att.datable.w3c attributes.</sch:report> </sch:rule>
    \item[{Schematron}]
   <sch:rule context="tei:*[@from]"> <sch:report role="nonfatal"  test="@notBefore">The @from and @notBefore attributes cannot be used together.</sch:report> </sch:rule>
    \item[{Schematron}]
   <sch:rule context="tei:*[@to]"> <sch:report role="nonfatal"  test="@notAfter">The @to and @notAfter attributes cannot be used together.</sch:report> </sch:rule>
    \item[{Example}]
  \leavevmode\bgroup\exampleFont \begin{shaded}\noindent\mbox{}{<\textbf{date}\hspace*{6pt}{from}="{1863-05-28}"\hspace*{6pt}{to}="{1863-06-01}">}28 May through 1 June 1863{</\textbf{date}>}\end{shaded}\egroup 


    \item[{Note}]
  \par
The value of these attributes should be a normalized representation of the date, time, or combined date \& time intended, in any of the standard formats specified by \textit{XML Schema Part 2: Datatypes Second Edition}, using the Gregorian calendar.\par
The most commonly-encountered format for the date portion of a temporal attribute is \texttt{yyyy-mm-dd}, but \texttt{yyyy}, \texttt{--mm}, \texttt{---dd}, \texttt{yyyy-mm}, or \texttt{--mm-dd} may also be used. For the time part, the form \texttt{hh:mm:ss} is used.\par
Note that this format does not currently permit use of the value 0000 to represent the year 1 BCE; instead the value -0001 should be used.
\end{reflist}  
\begin{reflist}
\item[]\begin{specHead}{TEI.att.datcat}{att.datcat}\index{att.datcat (attribute class)|oddindex}\index{datcat=@datcat!att.datcat (attribute class)|oddindex}\index{valueDatcat=@valueDatcat!att.datcat (attribute class)|oddindex} provides the {\itshape dcr:datacat} and {\itshape dcr:ValueDatacat} attributes which are used to align XML elements or attributes with the appropriate Data Categories (DCs) defined by the ISO 12620:2009 standard and stored in the Web repository called ISOCat at \xref{http://www.isocat.org/}{http://www.isocat.org/}. [\xref{http://www.tei-c.org/release/doc/tei-p5-doc/en/html/DI.html\#DIMVLV}{9.5.2. Lexical View} \xref{http://www.tei-c.org/release/doc/tei-p5-doc/en/html/FS.html\#FSSY}{18.3. Other Atomic Feature Values}]\end{specHead} 
    \item[{Module}]
  tei
    \item[{Members}]
  att.segLike[c cl m pc phr s seg w]
    \item[{Attributes}]
  Attributes\hfil\\[-10pt]\begin{sansreflist}
    \item[@datcat]
  contains a PID (persistent identifier) that aligns the given element with the appropriate Data Category (or categories) in ISOcat.
\begin{reflist}
    \item[{Status}]
  Optional
    \item[{Datatype}]
  1–∞ occurrences of teidata.pointer separated by whitespace
\end{reflist}  
    \item[@valueDatcat]
  contains a PID (persistent identifier) that aligns the content of the given element or the value of the given attribute with the appropriate simple Data Category (or categories) in ISOcat.
\begin{reflist}
    \item[{Status}]
  Optional
    \item[{Datatype}]
  1–∞ occurrences of teidata.pointer separated by whitespace
\end{reflist}  
\end{sansreflist}  
    \item[{Example}]
  In this example {\itshape dcr:datcat} relates the feature name to the data category "partOfSpeech" and {\itshape dcr:valueDatcat} the feature value to the data category "commonNoun". Both these data categories reside in the ISOcat DCR at \xref{http://www.isocat.org}{www.isocat.org}, which is the DCR used by ISO TC37 and hosted by its registration authority, the MPI for Psycholinguistics in Nijmegen.\leavevmode\bgroup\exampleFont \begin{shaded}\noindent\mbox{}{<\textbf{fs}\mbox{}\newline 
   xmlns:dcr="http://www.isocat.org/ns/dcr">}\mbox{}\newline 
\hspace*{6pt}{<\textbf{f}\hspace*{6pt}{dcr:datcat}="{http://www.isocat.org/datcat/DC-1345}"\mbox{}\newline 
\hspace*{6pt}\hspace*{6pt}{dcr:valueDatcat}="{http://www.isocat.org/datcat/DC-1256}"\hspace*{6pt}{fVal}="{\#commonNoun}"\hspace*{6pt}{name}="{POS}"/>}\mbox{}\newline 
{</\textbf{fs}>}\end{shaded}\egroup 


    \item[{Note}]
  \par
ISO 12620:2009 is a standard describing the data model and procedures for a Data Category Registry (DCR). Data categories are defined as elementary descriptors in a linguistic structure. In the DCR data model each data category gets assigned a unique Peristent IDentifier (PID), i.e., an URI. Linguistic resources or preferably their schemas that make use of data categories from a DCR should refer to them using this PID. For XML-based resources, like TEI documents, ISO 12620:2009 normative Annex A gives a small Data Category Reference XML vocabulary (also available online at \xref{http://www.isocat.org/12620/}{http://www.isocat.org/12620/}), which provides two attributes, {\itshape dcr:datcat} and {\itshape dcr:valueDatcat}.
\end{reflist}  
\begin{reflist}
\item[]\begin{specHead}{TEI.att.declarable}{att.declarable}\index{att.declarable (attribute class)|oddindex}\index{default=@default!att.declarable (attribute class)|oddindex} provides attributes for those elements in the TEI header which may be independently selected by means of the special purpose {\itshape decls} attribute. [\xref{http://www.tei-c.org/release/doc/tei-p5-doc/en/html/CC.html\#CCAS}{15.3. Associating Contextual Information with a Text}]\end{specHead} 
    \item[{Module}]
  tei
    \item[{Members}]
  availability bibl biblFull biblStruct correction correspDesc editorialDecl geoDecl hyphenation interpretation langUsage listApp listBibl listEvent listNym listOrg listPerson listPlace normalization projectDesc punctuation quotation refsDecl samplingDecl segmentation sourceDesc stdVals styleDefDecl textClass xenoData
    \item[{Attributes}]
  Attributes\hfil\\[-10pt]\begin{sansreflist}
    \item[@default]
  indicates whether or not this element is selected by default when its parent is selected.
\begin{reflist}
    \item[{Status}]
  Optional
    \item[{Datatype}]
  teidata.truthValue
    \item[{Legal values are:}]
  \begin{description}

\item[{true}]This element is selected if its parent is selected
\item[{false}]This element can only be selected explicitly, unless it is the only one of its kind, in which case it is selected if its parent is selected.{[Default] }
\end{description} 
\end{reflist}  
\end{sansreflist}  
    \item[{Note}]
  \par
The rules governing the association of declarable elements with individual parts of a TEI text are fully defined in chapter \xref{http://www.tei-c.org/release/doc/tei-p5-doc/en/html/CC.html\#CCAS}{15.3. Associating Contextual Information with a Text}. Only one element of a particular type may have a {\itshape default} attribute with a value of true.
\end{reflist}  
\begin{reflist}
\item[]\begin{specHead}{TEI.att.declaring}{att.declaring}\index{att.declaring (attribute class)|oddindex}\index{decls=@decls!att.declaring (attribute class)|oddindex} provides attributes for elements which may be independently associated with a particular declarable element within the header, thus overriding the inherited default for that element. [\xref{http://www.tei-c.org/release/doc/tei-p5-doc/en/html/CC.html\#CCAS}{15.3. Associating Contextual Information with a Text}]\end{specHead} 
    \item[{Module}]
  tei
    \item[{Members}]
  ab back body div facsimile floatingText front geo gloss graphic group lg media msDesc p ptr ref sourceDoc surface surfaceGrp term text
    \item[{Attributes}]
  Attributes\hfil\\[-10pt]\begin{sansreflist}
    \item[@decls]
  identifies one or more \textit{declarable elements} within the header, which are understood to apply to the element bearing this attribute and its content.
\begin{reflist}
    \item[{Status}]
  Optional
    \item[{Datatype}]
  1–∞ occurrences of teidata.pointer separated by whitespace
\end{reflist}  
\end{sansreflist}  
    \item[{Note}]
  \par
The rules governing the association of declarable elements with individual parts of a TEI text are fully defined in chapter \xref{http://www.tei-c.org/release/doc/tei-p5-doc/en/html/CC.html\#CCAS}{15.3. Associating Contextual Information with a Text}.
\end{reflist}  
\begin{reflist}
\item[]\begin{specHead}{TEI.att.dimensions}{att.dimensions}\index{att.dimensions (attribute class)|oddindex}\index{unit=@unit!att.dimensions (attribute class)|oddindex}\index{quantity=@quantity!att.dimensions (attribute class)|oddindex}\index{extent=@extent!att.dimensions (attribute class)|oddindex}\index{precision=@precision!att.dimensions (attribute class)|oddindex}\index{scope=@scope!att.dimensions (attribute class)|oddindex} provides attributes for describing the size of physical objects.\end{specHead} 
    \item[{Module}]
  tei
    \item[{Members}]
  att.damaged[damage damageSpan] att.editLike[att.transcriptional[add addSpan del delSpan mod redo restore retrace subst substJoin undo] affiliation age am birth climate corr date death education event ex expan faith floruit gap geogFeat geogName langKnowledge langKnown location name nationality occupation offset org orgName origDate origPlace origin persName person place placeName population reg relation residence secl sex socecStatus state supplied surplus terrain time trait unclear] depth dim dimensions height space width
    \item[{Attributes}]
  Attributes att.ranging (\textit{@atLeast}, \textit{@atMost}, \textit{@min}, \textit{@max}, \textit{@confidence}) \hfil\\[-10pt]\begin{sansreflist}
    \item[@unit]
  names the unit used for the measurement
\begin{reflist}
    \item[{Status}]
  Optional
    \item[{Datatype}]
  teidata.enumerated
    \item[{Suggested values include:}]
  \begin{description}

\item[{cm}](centimetres)
\item[{mm}](millimetres)
\item[{in}](inches)
\item[{lines}]lines of text
\item[{chars}](characters) characters of text
\end{description} 
\end{reflist}  
    \item[@quantity]
  specifies the length in the units specified
\begin{reflist}
    \item[{Status}]
  Optional
    \item[{Datatype}]
  teidata.numeric
\end{reflist}  
    \item[@extent]
  indicates the size of the object concerned using a project-specific vocabulary combining quantity and units in a single string of words.
\begin{reflist}
    \item[{Status}]
  Optional
    \item[{Datatype}]
  teidata.text
    \item[]\exampleFont {<\textbf{gap}\hspace*{6pt}{extent}="{5 words}"/>}
    \item[]\exampleFont {<\textbf{height}\hspace*{6pt}{extent}="{half the page}"/>}
\end{reflist}  
    \item[@precision]
  characterizes the precision of the values specified by the other attributes.
\begin{reflist}
    \item[{Status}]
  Optional
    \item[{Datatype}]
  teidata.certainty
\end{reflist}  
    \item[@scope]
  where the measurement summarizes more than one observation, specifies the applicability of this measurement.
\begin{reflist}
    \item[{Status}]
  Optional
    \item[{Datatype}]
  teidata.enumerated
    \item[{Sample values include:}]
  \begin{description}

\item[{all}]measurement applies to all instances.
\item[{most}]measurement applies to most of the instances inspected.
\item[{range}]measurement applies to only the specified range of instances.
\end{description} 
\end{reflist}  
\end{sansreflist}  
\end{reflist}  
\begin{reflist}
\item[]\begin{specHead}{TEI.att.divLike}{att.divLike}\index{att.divLike (attribute class)|oddindex}\index{org=@org!att.divLike (attribute class)|oddindex}\index{sample=@sample!att.divLike (attribute class)|oddindex} provides attributes common to all elements which behave in the same way as divisions. [\xref{http://www.tei-c.org/release/doc/tei-p5-doc/en/html/DS.html\#DS}{4. Default Text Structure}]\end{specHead} 
    \item[{Module}]
  tei
    \item[{Members}]
  div lg
    \item[{Attributes}]
  Attributes att.fragmentable (\textit{@part}) \hfil\\[-10pt]\begin{sansreflist}
    \item[@org]
  (organization) specifies how the content of the division is organized.
\begin{reflist}
    \item[{Status}]
  Optional
    \item[{Datatype}]
  teidata.enumerated
    \item[{Legal values are:}]
  \begin{description}

\item[{composite}]no claim is made about the sequence in which the immediate contents of this division are to be processed, or their inter-relationships.
\item[{uniform}]the immediate contents of this element are regarded as forming a logical unit, to be processed in sequence.{[Default] }
\end{description} 
\end{reflist}  
    \item[@sample]
  indicates whether this division is a sample of the original source and if so, from which part.
\begin{reflist}
    \item[{Status}]
  Optional
    \item[{Datatype}]
  teidata.enumerated
    \item[{Legal values are:}]
  \begin{description}

\item[{initial}]division lacks material present at end in source.
\item[{medial}]division lacks material at start and end.
\item[{final}]division lacks material at start.
\item[{unknown}]position of sampled material within original unknown.
\item[{complete}]division is not a sample.{[Default] }
\end{description} 
\end{reflist}  
\end{sansreflist}  
\end{reflist}  
\begin{reflist}
\item[]\begin{specHead}{TEI.att.docStatus}{att.docStatus}\index{att.docStatus (attribute class)|oddindex}\index{status=@status!att.docStatus (attribute class)|oddindex} provides attributes for use on metadata elements describing the status of a document.\end{specHead} 
    \item[{Module}]
  tei
    \item[{Members}]
  bibl biblFull biblStruct change revisionDesc
    \item[{Attributes}]
  Attributes\hfil\\[-10pt]\begin{sansreflist}
    \item[@status]
  describes the status of a document either currently or, when associated with a dated element, at the time indicated.
\begin{reflist}
    \item[{Status}]
  Optional
    \item[{Datatype}]
  teidata.enumerated
    \item[{Sample values include:}]
  \begin{description}

\item[{approved}]
\item[{candidate}]
\item[{cleared}]
\item[{deprecated}]
\item[{draft}]{[Default] }
\item[{embargoed}]
\item[{expired}]
\item[{frozen}]
\item[{galley}]
\item[{proposed}]
\item[{published}]
\item[{recommendation}]
\item[{submitted}]
\item[{unfinished}]
\item[{withdrawn}]
\end{description} 
\end{reflist}  
\end{sansreflist}  
    \item[{Example}]
  \leavevmode\bgroup\exampleFont \begin{shaded}\noindent\mbox{}{<\textbf{revisionDesc}\hspace*{6pt}{status}="{published}">}\mbox{}\newline 
\hspace*{6pt}{<\textbf{change}\hspace*{6pt}{status}="{published}"\mbox{}\newline 
\hspace*{6pt}\hspace*{6pt}{when}="{2010-10-21}"/>}\mbox{}\newline 
\hspace*{6pt}{<\textbf{change}\hspace*{6pt}{status}="{cleared}"\hspace*{6pt}{when}="{2010-10-02}"/>}\mbox{}\newline 
\hspace*{6pt}{<\textbf{change}\hspace*{6pt}{status}="{embargoed}"\mbox{}\newline 
\hspace*{6pt}\hspace*{6pt}{when}="{2010-08-02}"/>}\mbox{}\newline 
\hspace*{6pt}{<\textbf{change}\hspace*{6pt}{status}="{frozen}"\hspace*{6pt}{when}="{2010-05-01}"\mbox{}\newline 
\hspace*{6pt}\hspace*{6pt}{who}="{\#MSM}"/>}\mbox{}\newline 
\hspace*{6pt}{<\textbf{change}\hspace*{6pt}{status}="{draft}"\hspace*{6pt}{when}="{2010-03-01}"\mbox{}\newline 
\hspace*{6pt}\hspace*{6pt}{who}="{\#LB}"/>}\mbox{}\newline 
{</\textbf{revisionDesc}>}\end{shaded}\egroup 


\end{reflist}  
\begin{reflist}
\item[]\begin{specHead}{TEI.att.editLike}{att.editLike}\index{att.editLike (attribute class)|oddindex}\index{evidence=@evidence!att.editLike (attribute class)|oddindex}\index{instant=@instant!att.editLike (attribute class)|oddindex} provides attributes describing the nature of an encoded scholarly intervention or interpretation of any kind. [\xref{http://www.tei-c.org/release/doc/tei-p5-doc/en/html/CO.html\#COED}{3.4. Simple Editorial Changes} \xref{http://www.tei-c.org/release/doc/tei-p5-doc/en/html/MS.html\#msdates}{10.3.1. Origination} \xref{http://www.tei-c.org/release/doc/tei-p5-doc/en/html/ND.html\#NDPERSE}{13.3.2. The Person Element} \xref{http://www.tei-c.org/release/doc/tei-p5-doc/en/html/PH.html\#PHCO}{11.3.1.1. Core Elements for Transcriptional Work}]\end{specHead} 
    \item[{Module}]
  tei
    \item[{Members}]
  att.transcriptional[add addSpan del delSpan mod redo restore retrace subst substJoin undo] affiliation age am birth climate corr date death education event ex expan faith floruit gap geogFeat geogName langKnowledge langKnown location name nationality occupation offset org orgName origDate origPlace origin persName person place placeName population reg relation residence secl sex socecStatus state supplied surplus terrain time trait unclear
    \item[{Attributes}]
  Attributes att.dimensions (\textit{@unit}, \textit{@quantity}, \textit{@extent}, \textit{@precision}, \textit{@scope})  (att.ranging (\textit{@atLeast}, \textit{@atMost}, \textit{@min}, \textit{@max}, \textit{@confidence})) \hfil\\[-10pt]\begin{sansreflist}
    \item[@evidence]
  indicates the nature of the evidence supporting the reliability or accuracy of the intervention or interpretation.
\begin{reflist}
    \item[{Status}]
  Optional
    \item[{Datatype}]
  1–∞ occurrences of teidata.enumerated separated by whitespace
    \item[{Suggested values include:}]
  \begin{description}

\item[{internal}]there is internal evidence to support the intervention.
\item[{external}]there is external evidence to support the intervention.
\item[{conjecture}]the intervention or interpretation has been made by the editor, cataloguer, or scholar on the basis of their expertise.
\end{description} 
\end{reflist}  
    \item[@instant]
  indicates whether this is an instant revision or not.
\begin{reflist}
    \item[{Status}]
  Optional
    \item[{Datatype}]
  teidata.xTruthValue
    \item[{Default}]
  false
\end{reflist}  
\end{sansreflist}  
    \item[{Note}]
  \par
The members of this attribute class are typically used to represent any kind of editorial intervention in a text, for example a correction or interpretation, or to date or localize manuscripts etc.
    \item[{Note}]
  \par
Each pointer on the {\itshape source} (if present) corresponding to a witness or witness group should reference a bibliographic citation such as a <witness>, <msDesc>, or <bibl> element, or another external bibliographic citation, documenting the source concerned.
\end{reflist}  
\begin{reflist}
\item[]\begin{specHead}{TEI.att.edition}{att.edition}\index{att.edition (attribute class)|oddindex}\index{ed=@ed!att.edition (attribute class)|oddindex}\index{edRef=@edRef!att.edition (attribute class)|oddindex} provides attributes identifying the source edition from which some encoded feature derives.\end{specHead} 
    \item[{Module}]
  tei
    \item[{Members}]
  cb lb milestone pb refState
    \item[{Attributes}]
  Attributes\hfil\\[-10pt]\begin{sansreflist}
    \item[@ed]
  (edition) supplies a sigil or other arbitrary identifier for the source edition in which the associated feature (for example, a page, column, or line break) occurs at this point in the text.
\begin{reflist}
    \item[{Status}]
  Optional
    \item[{Datatype}]
  1–∞ occurrences of teidata.word separated by whitespace
\end{reflist}  
    \item[@edRef]
  (edition reference) provides a pointer to the source edition in which the associated feature (for example, a page, column, or line break) occurs at this point in the text.
\begin{reflist}
    \item[{Status}]
  Optional
    \item[{Datatype}]
  1–∞ occurrences of teidata.pointer separated by whitespace
\end{reflist}  
\end{sansreflist}  
    \item[{Example}]
  \leavevmode\bgroup\exampleFont \begin{shaded}\noindent\mbox{}{<\textbf{l}>}Of Mans First Disobedience,{<\textbf{lb}\hspace*{6pt}{ed}="{1674}"/>} and{<\textbf{lb}\hspace*{6pt}{ed}="{1667}"/>} the Fruit{</\textbf{l}>}\mbox{}\newline 
{<\textbf{l}>}Of that Forbidden Tree, whose{<\textbf{lb}\hspace*{6pt}{ed}="{1667 1674}"/>} mortal tast{</\textbf{l}>}\mbox{}\newline 
{<\textbf{l}>}Brought Death into the World,{<\textbf{lb}\hspace*{6pt}{ed}="{1667}"/>} and all{<\textbf{lb}\hspace*{6pt}{ed}="{1674}"/>} our woe,{</\textbf{l}>}\end{shaded}\egroup 


    \item[{Example}]
  \leavevmode\bgroup\exampleFont \begin{shaded}\noindent\mbox{}{<\textbf{listBibl}>}\mbox{}\newline 
\hspace*{6pt}{<\textbf{bibl}\hspace*{6pt}{xml:id}="{stapledon1937}">}\mbox{}\newline 
\hspace*{6pt}\hspace*{6pt}{<\textbf{author}>}Olaf Stapledon{</\textbf{author}>},\mbox{}\newline 
\hspace*{6pt}{<\textbf{title}>}Starmaker{</\textbf{title}>}, {<\textbf{publisher}>}Methuen{</\textbf{publisher}>}, {<\textbf{date}>}1937{</\textbf{date}>}\mbox{}\newline 
\hspace*{6pt}{</\textbf{bibl}>}\mbox{}\newline 
\hspace*{6pt}{<\textbf{bibl}\hspace*{6pt}{xml:id}="{stapledon1968}">}\mbox{}\newline 
\hspace*{6pt}\hspace*{6pt}{<\textbf{author}>}Olaf Stapledon{</\textbf{author}>},\mbox{}\newline 
\hspace*{6pt}{<\textbf{title}>}Starmaker{</\textbf{title}>}, {<\textbf{publisher}>}Dover{</\textbf{publisher}>}, {<\textbf{date}>}1968{</\textbf{date}>}\mbox{}\newline 
\hspace*{6pt}{</\textbf{bibl}>}\mbox{}\newline 
{</\textbf{listBibl}>}\mbox{}\newline 
\textit{<!-- ... -->}\mbox{}\newline 
{<\textbf{p}>}Looking into the future aeons from the supreme moment of\mbox{}\newline 
 the cosmos, I saw the populations still with all their\mbox{}\newline 
 strength maintaining the{<\textbf{pb}\hspace*{6pt}{edRef}="{\#stapledon1968}"\hspace*{6pt}{n}="{411}"/>}essentials of their ancient culture,\mbox{}\newline 
 still living their personal lives in zest and endless\mbox{}\newline 
 novelty of action, … I saw myself still\mbox{}\newline 
 preserving, though with increasing difficulty, my lucid\mbox{}\newline 
 con-{<\textbf{pb}\hspace*{6pt}{edRef}="{\#stapledon1937}"\hspace*{6pt}{n}="{291}"/>}sciousness;{</\textbf{p}>}\end{shaded}\egroup 


\end{reflist}  
\begin{reflist}
\item[]\begin{specHead}{TEI.att.fragmentable}{att.fragmentable}\index{att.fragmentable (attribute class)|oddindex}\index{part=@part!att.fragmentable (attribute class)|oddindex} provides an attribute for representing fragmentation of a structural element, typically as a consequence of some overlapping hierarchy.\end{specHead} 
    \item[{Module}]
  tei
    \item[{Members}]
  att.divLike[div lg] att.segLike[c cl m pc phr s seg w] ab l p
    \item[{Attributes}]
  Attributes\hfil\\[-10pt]\begin{sansreflist}
    \item[@part]
  specifies whether or not its parent element is fragmented in some way, typically by some other overlapping structure: for example a speech which is divided between two or more verse stanzas, a paragraph which is split across a page division, a verse line which is divided between two speakers.
\begin{reflist}
    \item[{Status}]
  Optional
    \item[{Datatype}]
  teidata.enumerated
    \item[{Legal values are:}]
  \begin{description}

\item[{Y}](yes) the element is fragmented in some (unspecified) respect
\item[{N}](no) the element is not fragmented, or no claim is made as to its completeness{[Default] }
\item[{I}](initial) this is the initial part of a fragmented element
\item[{M}](medial) this is a medial part of a fragmented element
\item[{F}](final) this is the final part of a fragmented element
\end{description} 
    \item[{Note}]
  \par
The values I, M, or F should be used only where it is clear how the element may be be reconstituted.
\end{reflist}  
\end{sansreflist}  
\end{reflist}  
\begin{reflist}
\item[]\begin{specHead}{TEI.att.global}{att.global}\index{att.global (attribute class)|oddindex}\index{xml:id=@xml:id!att.global (attribute class)|oddindex}\index{n=@n!att.global (attribute class)|oddindex}\index{xml:lang=@xml:lang!att.global (attribute class)|oddindex}\index{xml:base=@xml:base!att.global (attribute class)|oddindex}\index{xml:space=@xml:space!att.global (attribute class)|oddindex} provides attributes common to all elements in the TEI encoding scheme. [\xref{http://www.tei-c.org/release/doc/tei-p5-doc/en/html/ST.html\#STGA}{1.3.1.1. Global Attributes}]\end{specHead} 
    \item[{Module}]
  tei
    \item[{Members}]
  TEI ab abbr abstract accMat acquisition add addName addSpan additional additions addrLine address adminInfo affiliation age alt altGrp altIdentifier am analytic anchor app appInfo application argument author authority availability back bibl biblFull biblScope biblStruct binding bindingDesc birth bloc body byline c cRefPattern calendar calendarDesc catDesc catRef catchwords category cb cell change char charDecl charName charProp choice cit citedRange cl classCode classDecl climate closer collation collection colophon condition corr correction correspAction correspContext correspDesc country creation custEvent custodialHist damage damageSpan date dateline death decoDesc decoNote del delSpan depth desc dim dimensions distinct distributor district div divGen docAuthor docDate docEdition docImprint docTitle edition editionStmt editor editorialDecl education email emph encodingDesc epigraph event ex expan explicit extent facsimile faith figDesc figure fileDesc filiation finalRubric floatingText floruit foliation foreign forename formula front funder fw g gap gb genName geo geoDecl geogFeat geogName gloss glyph glyphName graphic group handDesc handNote handNotes handShift head headItem headLabel height heraldry hi history hyphenation idno imprimatur imprint incipit index institution interp interpGrp interpretation item join joinGrp keywords l label lacunaEnd lacunaStart langKnowledge langKnown langUsage language layout layoutDesc lb lem lg licence line link linkGrp list listApp listBibl listChange listEvent listNym listOrg listPerson listPlace listPrefixDef listRelation listTranspose listWit localName location locus locusGrp m mapping material measure measureGrp media meeting mentioned metamark milestone mod monogr msContents msDesc msFrag msIdentifier msItem msItemStruct msName msPart musicNotation name nameLink namespace nationality normalization notatedMusic note notesStmt num nym objectDesc objectType occupation offset opener org orgName orig origDate origPlace origin p pb pc persName person personGrp phr physDesc place placeName population postscript prefixDef principal profileDesc projectDesc provenance ptr pubPlace publicationStmt publisher punctuation q quotation quote rdg rdgGrp recordHist redo ref refState refsDecl reg region relatedItem relation rendition repository residence resp respStmt restore retrace revisionDesc roleName row rs rubric s said salute samplingDecl schemaRef scriptDesc scriptNote seal sealDesc secFol secl seg segmentation series seriesStmt settlement sex sic signatures signed soCalled socecStatus source sourceDesc sourceDoc sp span spanGrp speaker sponsor stage stamp state stdVals street styleDefDecl subst substJoin summary supplied support supportDesc surface surfaceGrp surname surplus surrogates table tagUsage tagsDecl taxonomy teiCorpus teiHeader term terrain text textClass textLang time timeline title titlePage titlePart titleStmt trailer trait transpose typeDesc typeNote unclear undo unicodeName value variantEncoding w watermark when width wit witDetail witEnd witStart witness xenoData zone
    \item[{Attributes}]
  Attributes att.global.rendition (\textit{@rend}, \textit{@style}, \textit{@rendition}) att.global.linking (\textit{@corresp}, \textit{@synch}, \textit{@sameAs}, \textit{@copyOf}, \textit{@next}, \textit{@prev}, \textit{@exclude}, \textit{@select}) att.global.analytic (\textit{@ana}) att.global.facs (\textit{@facs}) att.global.change (\textit{@change}) att.global.responsibility (\textit{@cert}, \textit{@resp}) att.global.source (\textit{@source}) \hfil\\[-10pt]\begin{sansreflist}
    \item[@xml:id]
  (identifier) provides a unique identifier for the element bearing the attribute.
\begin{reflist}
    \item[{Status}]
  Optional
    \item[{Datatype}]
  
    \item[{Note}]
  \par
The {\itshape xml:id} attribute may be used to specify a canonical reference for an element; see section \xref{http://www.tei-c.org/release/doc/tei-p5-doc/en/html/CO.html\#CORS}{3.10. Reference Systems}.
\end{reflist}  
    \item[@n]
  (number) gives a number (or other label) for an element, which is not necessarily unique within the document.
\begin{reflist}
    \item[{Status}]
  Optional
    \item[{Datatype}]
  teidata.text
    \item[{Note}]
  \par
The value of this attribute is always understood to be a single token, even if it contains space or other punctuation characters, and need not be composed of numbers only. It is typically used to specify the numbering of chapters, sections, list items, etc.; it may also be used in the specification of a standard reference system for the text.
\end{reflist}  
    \item[@xml:lang]
  (language) indicates the language of the element content using a ‘tag’ generated according to \xref{http://www.rfc-editor.org/rfc/bcp/bcp47.txt}{BCP 47}.
\begin{reflist}
    \item[{Status}]
  Optional
    \item[{Datatype}]
  teidata.language
    \item[]\exampleFont {<\textbf{p}>} … The consequences of\mbox{}\newline 
 this rapid depopulation were the loss of the last\mbox{}\newline 
{<\textbf{foreign}\hspace*{6pt}{xml:lang}="{rap}">}ariki{</\textbf{foreign}>} or chief\mbox{}\newline 
 (Routledge 1920:205,210) and their connections to\mbox{}\newline 
 ancestral territorial organization.{</\textbf{p}>}
    \item[{Note}]
  \par
The xml:lang value will be inherited from the immediately enclosing element, or from its parent, and so on up the document hierarchy. It is generally good practice to specify xml:lang at the highest appropriate level, noticing that a different default may be needed for the teiHeader from that needed for the associated resource element or elements, and that a single TEI document may contain texts in many languages.\par
The authoritative list of registered language subtags is maintained by IANA and is available at \url{http://www.iana.org/assignments/language-subtag-registry}. For a good general overview of the construction of language tags, see \url{http://www.w3.org/International/articles/language-tags/}, and for a practical step-by-step guide, see \url{https://www.w3.org/International/questions/qa-choosing-language-tags.en.php}.\par
The value used must conform with BCP 47. If the value is a private use code (i.e., starts with x- or contains -x-), a <language> element with a matching value for its {\itshape ident} attribute should be supplied in the TEI header to document this value. Such documentation may also optionally be supplied for non-private-use codes, though these must remain consistent with their  ( {\abbr IETF}) {\expan Internet Engineering Task Force} definitions.
\end{reflist}  
    \item[@xml:base]
  provides a base URI reference with which applications can resolve relative URI references into absolute URI references.
\begin{reflist}
    \item[{Status}]
  Optional
    \item[{Datatype}]
  teidata.pointer
    \item[]\exampleFont {<\textbf{div}\hspace*{6pt}{type}="{bibl}">}\mbox{}\newline 
\hspace*{6pt}{<\textbf{head}>}Bibliography{</\textbf{head}>}\mbox{}\newline 
\hspace*{6pt}{<\textbf{listBibl}\hspace*{6pt}{xml:base}="{http://www.lib.ucdavis.edu/BWRP/Works/}">}\mbox{}\newline 
\hspace*{6pt}\hspace*{6pt}{<\textbf{bibl}>}\mbox{}\newline 
\hspace*{6pt}\hspace*{6pt}\hspace*{6pt}{<\textbf{author}>}\mbox{}\newline 
\hspace*{6pt}\hspace*{6pt}\hspace*{6pt}\hspace*{6pt}{<\textbf{name}>}Landon, Letitia Elizabeth{</\textbf{name}>}\mbox{}\newline 
\hspace*{6pt}\hspace*{6pt}\hspace*{6pt}{</\textbf{author}>}\mbox{}\newline 
\hspace*{6pt}\hspace*{6pt}\hspace*{6pt}{<\textbf{ref}\hspace*{6pt}{target}="{LandLVowOf.sgm}">}\mbox{}\newline 
\hspace*{6pt}\hspace*{6pt}\hspace*{6pt}\hspace*{6pt}{<\textbf{title}>}The Vow of the Peacock{</\textbf{title}>}\mbox{}\newline 
\hspace*{6pt}\hspace*{6pt}\hspace*{6pt}{</\textbf{ref}>}\mbox{}\newline 
\hspace*{6pt}\hspace*{6pt}{</\textbf{bibl}>}\mbox{}\newline 
\hspace*{6pt}\hspace*{6pt}{<\textbf{bibl}>}\mbox{}\newline 
\hspace*{6pt}\hspace*{6pt}\hspace*{6pt}{<\textbf{author}>}\mbox{}\newline 
\hspace*{6pt}\hspace*{6pt}\hspace*{6pt}\hspace*{6pt}{<\textbf{name}>}Compton, Margaret Clephane{</\textbf{name}>}\mbox{}\newline 
\hspace*{6pt}\hspace*{6pt}\hspace*{6pt}{</\textbf{author}>}\mbox{}\newline 
\hspace*{6pt}\hspace*{6pt}\hspace*{6pt}{<\textbf{ref}\hspace*{6pt}{target}="{NortMIrene.sgm}">}\mbox{}\newline 
\hspace*{6pt}\hspace*{6pt}\hspace*{6pt}\hspace*{6pt}{<\textbf{title}>}Irene, a Poem in Six Cantos{</\textbf{title}>}\mbox{}\newline 
\hspace*{6pt}\hspace*{6pt}\hspace*{6pt}{</\textbf{ref}>}\mbox{}\newline 
\hspace*{6pt}\hspace*{6pt}{</\textbf{bibl}>}\mbox{}\newline 
\hspace*{6pt}\hspace*{6pt}{<\textbf{bibl}>}\mbox{}\newline 
\hspace*{6pt}\hspace*{6pt}\hspace*{6pt}{<\textbf{author}>}\mbox{}\newline 
\hspace*{6pt}\hspace*{6pt}\hspace*{6pt}\hspace*{6pt}{<\textbf{name}>}Taylor, Jane{</\textbf{name}>}\mbox{}\newline 
\hspace*{6pt}\hspace*{6pt}\hspace*{6pt}{</\textbf{author}>}\mbox{}\newline 
\hspace*{6pt}\hspace*{6pt}\hspace*{6pt}{<\textbf{ref}\hspace*{6pt}{target}="{TaylJEssay.sgm}">}\mbox{}\newline 
\hspace*{6pt}\hspace*{6pt}\hspace*{6pt}\hspace*{6pt}{<\textbf{title}>}Essays in Rhyme on Morals and Manners{</\textbf{title}>}\mbox{}\newline 
\hspace*{6pt}\hspace*{6pt}\hspace*{6pt}{</\textbf{ref}>}\mbox{}\newline 
\hspace*{6pt}\hspace*{6pt}{</\textbf{bibl}>}\mbox{}\newline 
\hspace*{6pt}{</\textbf{listBibl}>}\mbox{}\newline 
{</\textbf{div}>}
\end{reflist}  
    \item[@xml:space]
  signals an intention about how white space should be managed by applications.
\begin{reflist}
    \item[{Status}]
  Optional
    \item[{Datatype}]
  teidata.enumerated
    \item[{Legal values are:}]
  \begin{description}

\item[{default}]signals that the application's default white-space processing modes are acceptable
\item[{preserve}]indicates the intent that applications preserve all white space
\end{description} 
    \item[{Note}]
  \par
The \xref{http://www.w3.org/TR/REC-xml/\#sec-white-space}{XML specification} provides further guidance on the use of this attribute. Note that many parsers may not handle xml:space correctly.
\end{reflist}  
\end{sansreflist}  
\end{reflist}  
\begin{reflist}
\item[]\begin{specHead}{TEI.att.global.analytic}{att.global.analytic}\index{att.global.analytic (attribute class)|oddindex}\index{ana=@ana!att.global.analytic (attribute class)|oddindex} provides additional global attributes for associating specific analyses or interpretations with appropriate portions of a text. [\xref{http://www.tei-c.org/release/doc/tei-p5-doc/en/html/AI.html\#AIATTS}{17.2. Global Attributes for Simple Analyses} \xref{http://www.tei-c.org/release/doc/tei-p5-doc/en/html/AI.html\#AISP}{17.3. Spans and Interpretations}]\end{specHead} 
    \item[{Module}]
  analysis
    \item[{Members}]
  att.global[TEI ab abbr abstract accMat acquisition add addName addSpan additional additions addrLine address adminInfo affiliation age alt altGrp altIdentifier am analytic anchor app appInfo application argument author authority availability back bibl biblFull biblScope biblStruct binding bindingDesc birth bloc body byline c cRefPattern calendar calendarDesc catDesc catRef catchwords category cb cell change char charDecl charName charProp choice cit citedRange cl classCode classDecl climate closer collation collection colophon condition corr correction correspAction correspContext correspDesc country creation custEvent custodialHist damage damageSpan date dateline death decoDesc decoNote del delSpan depth desc dim dimensions distinct distributor district div divGen docAuthor docDate docEdition docImprint docTitle edition editionStmt editor editorialDecl education email emph encodingDesc epigraph event ex expan explicit extent facsimile faith figDesc figure fileDesc filiation finalRubric floatingText floruit foliation foreign forename formula front funder fw g gap gb genName geo geoDecl geogFeat geogName gloss glyph glyphName graphic group handDesc handNote handNotes handShift head headItem headLabel height heraldry hi history hyphenation idno imprimatur imprint incipit index institution interp interpGrp interpretation item join joinGrp keywords l label lacunaEnd lacunaStart langKnowledge langKnown langUsage language layout layoutDesc lb lem lg licence line link linkGrp list listApp listBibl listChange listEvent listNym listOrg listPerson listPlace listPrefixDef listRelation listTranspose listWit localName location locus locusGrp m mapping material measure measureGrp media meeting mentioned metamark milestone mod monogr msContents msDesc msFrag msIdentifier msItem msItemStruct msName msPart musicNotation name nameLink namespace nationality normalization notatedMusic note notesStmt num nym objectDesc objectType occupation offset opener org orgName orig origDate origPlace origin p pb pc persName person personGrp phr physDesc place placeName population postscript prefixDef principal profileDesc projectDesc provenance ptr pubPlace publicationStmt publisher punctuation q quotation quote rdg rdgGrp recordHist redo ref refState refsDecl reg region relatedItem relation rendition repository residence resp respStmt restore retrace revisionDesc roleName row rs rubric s said salute samplingDecl schemaRef scriptDesc scriptNote seal sealDesc secFol secl seg segmentation series seriesStmt settlement sex sic signatures signed soCalled socecStatus source sourceDesc sourceDoc sp span spanGrp speaker sponsor stage stamp state stdVals street styleDefDecl subst substJoin summary supplied support supportDesc surface surfaceGrp surname surplus surrogates table tagUsage tagsDecl taxonomy teiCorpus teiHeader term terrain text textClass textLang time timeline title titlePage titlePart titleStmt trailer trait transpose typeDesc typeNote unclear undo unicodeName value variantEncoding w watermark when width wit witDetail witEnd witStart witness xenoData zone]
    \item[{Attributes}]
  Attributes\hfil\\[-10pt]\begin{sansreflist}
    \item[@ana]
  (analysis) indicates one or more elements containing interpretations of the element on which the {\itshape ana} attribute appears.
\begin{reflist}
    \item[{Status}]
  Optional
    \item[{Datatype}]
  1–∞ occurrences of teidata.pointer separated by whitespace
    \item[{Note}]
  \par
When multiple values are given, they may reflect either multiple divergent interpretations of an ambiguous text, or multiple mutually consistent interpretations of the same passage in different contexts.
\end{reflist}  
\end{sansreflist}  
\end{reflist}  
\begin{reflist}
\item[]\begin{specHead}{TEI.att.global.change}{att.global.change}\index{att.global.change (attribute class)|oddindex}\index{change=@change!att.global.change (attribute class)|oddindex} supplies the {\itshape change} attribute, allowing its member elements to specify one or more states or revision campaigns with which they are associated.\end{specHead} 
    \item[{Module}]
  transcr
    \item[{Members}]
  att.global[TEI ab abbr abstract accMat acquisition add addName addSpan additional additions addrLine address adminInfo affiliation age alt altGrp altIdentifier am analytic anchor app appInfo application argument author authority availability back bibl biblFull biblScope biblStruct binding bindingDesc birth bloc body byline c cRefPattern calendar calendarDesc catDesc catRef catchwords category cb cell change char charDecl charName charProp choice cit citedRange cl classCode classDecl climate closer collation collection colophon condition corr correction correspAction correspContext correspDesc country creation custEvent custodialHist damage damageSpan date dateline death decoDesc decoNote del delSpan depth desc dim dimensions distinct distributor district div divGen docAuthor docDate docEdition docImprint docTitle edition editionStmt editor editorialDecl education email emph encodingDesc epigraph event ex expan explicit extent facsimile faith figDesc figure fileDesc filiation finalRubric floatingText floruit foliation foreign forename formula front funder fw g gap gb genName geo geoDecl geogFeat geogName gloss glyph glyphName graphic group handDesc handNote handNotes handShift head headItem headLabel height heraldry hi history hyphenation idno imprimatur imprint incipit index institution interp interpGrp interpretation item join joinGrp keywords l label lacunaEnd lacunaStart langKnowledge langKnown langUsage language layout layoutDesc lb lem lg licence line link linkGrp list listApp listBibl listChange listEvent listNym listOrg listPerson listPlace listPrefixDef listRelation listTranspose listWit localName location locus locusGrp m mapping material measure measureGrp media meeting mentioned metamark milestone mod monogr msContents msDesc msFrag msIdentifier msItem msItemStruct msName msPart musicNotation name nameLink namespace nationality normalization notatedMusic note notesStmt num nym objectDesc objectType occupation offset opener org orgName orig origDate origPlace origin p pb pc persName person personGrp phr physDesc place placeName population postscript prefixDef principal profileDesc projectDesc provenance ptr pubPlace publicationStmt publisher punctuation q quotation quote rdg rdgGrp recordHist redo ref refState refsDecl reg region relatedItem relation rendition repository residence resp respStmt restore retrace revisionDesc roleName row rs rubric s said salute samplingDecl schemaRef scriptDesc scriptNote seal sealDesc secFol secl seg segmentation series seriesStmt settlement sex sic signatures signed soCalled socecStatus source sourceDesc sourceDoc sp span spanGrp speaker sponsor stage stamp state stdVals street styleDefDecl subst substJoin summary supplied support supportDesc surface surfaceGrp surname surplus surrogates table tagUsage tagsDecl taxonomy teiCorpus teiHeader term terrain text textClass textLang time timeline title titlePage titlePart titleStmt trailer trait transpose typeDesc typeNote unclear undo unicodeName value variantEncoding w watermark when width wit witDetail witEnd witStart witness xenoData zone]
    \item[{Attributes}]
  Attributes\hfil\\[-10pt]\begin{sansreflist}
    \item[@change]
  points to one or more <change> elements documenting a state or revision campaign to which the element bearing this attribute and its children have been assigned by the encoder.
\begin{reflist}
    \item[{Status}]
  Optional
    \item[{Datatype}]
  1–∞ occurrences of teidata.pointer separated by whitespace
\end{reflist}  
\end{sansreflist}  
\end{reflist}  
\begin{reflist}
\item[]\begin{specHead}{TEI.att.global.facs}{att.global.facs}\index{att.global.facs (attribute class)|oddindex}\index{facs=@facs!att.global.facs (attribute class)|oddindex} provides an attribute used to express correspondence between an element containing transcribed text and all or part of an image representing that text. [\xref{http://www.tei-c.org/release/doc/tei-p5-doc/en/html/PH.html\#PHFAX}{11.1. Digital Facsimiles}]\end{specHead} 
    \item[{Module}]
  transcr
    \item[{Members}]
  att.global[TEI ab abbr abstract accMat acquisition add addName addSpan additional additions addrLine address adminInfo affiliation age alt altGrp altIdentifier am analytic anchor app appInfo application argument author authority availability back bibl biblFull biblScope biblStruct binding bindingDesc birth bloc body byline c cRefPattern calendar calendarDesc catDesc catRef catchwords category cb cell change char charDecl charName charProp choice cit citedRange cl classCode classDecl climate closer collation collection colophon condition corr correction correspAction correspContext correspDesc country creation custEvent custodialHist damage damageSpan date dateline death decoDesc decoNote del delSpan depth desc dim dimensions distinct distributor district div divGen docAuthor docDate docEdition docImprint docTitle edition editionStmt editor editorialDecl education email emph encodingDesc epigraph event ex expan explicit extent facsimile faith figDesc figure fileDesc filiation finalRubric floatingText floruit foliation foreign forename formula front funder fw g gap gb genName geo geoDecl geogFeat geogName gloss glyph glyphName graphic group handDesc handNote handNotes handShift head headItem headLabel height heraldry hi history hyphenation idno imprimatur imprint incipit index institution interp interpGrp interpretation item join joinGrp keywords l label lacunaEnd lacunaStart langKnowledge langKnown langUsage language layout layoutDesc lb lem lg licence line link linkGrp list listApp listBibl listChange listEvent listNym listOrg listPerson listPlace listPrefixDef listRelation listTranspose listWit localName location locus locusGrp m mapping material measure measureGrp media meeting mentioned metamark milestone mod monogr msContents msDesc msFrag msIdentifier msItem msItemStruct msName msPart musicNotation name nameLink namespace nationality normalization notatedMusic note notesStmt num nym objectDesc objectType occupation offset opener org orgName orig origDate origPlace origin p pb pc persName person personGrp phr physDesc place placeName population postscript prefixDef principal profileDesc projectDesc provenance ptr pubPlace publicationStmt publisher punctuation q quotation quote rdg rdgGrp recordHist redo ref refState refsDecl reg region relatedItem relation rendition repository residence resp respStmt restore retrace revisionDesc roleName row rs rubric s said salute samplingDecl schemaRef scriptDesc scriptNote seal sealDesc secFol secl seg segmentation series seriesStmt settlement sex sic signatures signed soCalled socecStatus source sourceDesc sourceDoc sp span spanGrp speaker sponsor stage stamp state stdVals street styleDefDecl subst substJoin summary supplied support supportDesc surface surfaceGrp surname surplus surrogates table tagUsage tagsDecl taxonomy teiCorpus teiHeader term terrain text textClass textLang time timeline title titlePage titlePart titleStmt trailer trait transpose typeDesc typeNote unclear undo unicodeName value variantEncoding w watermark when width wit witDetail witEnd witStart witness xenoData zone]
    \item[{Attributes}]
  Attributes\hfil\\[-10pt]\begin{sansreflist}
    \item[@facs]
  (facsimile) points to all or part of an image which corresponds with the content of the element.
\begin{reflist}
    \item[{Status}]
  Optional
    \item[{Datatype}]
  1–∞ occurrences of teidata.pointer separated by whitespace
\end{reflist}  
\end{sansreflist}  
\end{reflist}  
\begin{reflist}
\item[]\begin{specHead}{TEI.att.global.linking}{att.global.linking}\index{att.global.linking (attribute class)|oddindex}\index{corresp=@corresp!att.global.linking (attribute class)|oddindex}\index{synch=@synch!att.global.linking (attribute class)|oddindex}\index{sameAs=@sameAs!att.global.linking (attribute class)|oddindex}\index{copyOf=@copyOf!att.global.linking (attribute class)|oddindex}\index{next=@next!att.global.linking (attribute class)|oddindex}\index{prev=@prev!att.global.linking (attribute class)|oddindex}\index{exclude=@exclude!att.global.linking (attribute class)|oddindex}\index{select=@select!att.global.linking (attribute class)|oddindex} provides a set of attributes for hypertextual linking. [\xref{http://www.tei-c.org/release/doc/tei-p5-doc/en/html/SA.html\#SA}{16. Linking, Segmentation, and Alignment}]\end{specHead} 
    \item[{Module}]
  linking
    \item[{Members}]
  att.global[TEI ab abbr abstract accMat acquisition add addName addSpan additional additions addrLine address adminInfo affiliation age alt altGrp altIdentifier am analytic anchor app appInfo application argument author authority availability back bibl biblFull biblScope biblStruct binding bindingDesc birth bloc body byline c cRefPattern calendar calendarDesc catDesc catRef catchwords category cb cell change char charDecl charName charProp choice cit citedRange cl classCode classDecl climate closer collation collection colophon condition corr correction correspAction correspContext correspDesc country creation custEvent custodialHist damage damageSpan date dateline death decoDesc decoNote del delSpan depth desc dim dimensions distinct distributor district div divGen docAuthor docDate docEdition docImprint docTitle edition editionStmt editor editorialDecl education email emph encodingDesc epigraph event ex expan explicit extent facsimile faith figDesc figure fileDesc filiation finalRubric floatingText floruit foliation foreign forename formula front funder fw g gap gb genName geo geoDecl geogFeat geogName gloss glyph glyphName graphic group handDesc handNote handNotes handShift head headItem headLabel height heraldry hi history hyphenation idno imprimatur imprint incipit index institution interp interpGrp interpretation item join joinGrp keywords l label lacunaEnd lacunaStart langKnowledge langKnown langUsage language layout layoutDesc lb lem lg licence line link linkGrp list listApp listBibl listChange listEvent listNym listOrg listPerson listPlace listPrefixDef listRelation listTranspose listWit localName location locus locusGrp m mapping material measure measureGrp media meeting mentioned metamark milestone mod monogr msContents msDesc msFrag msIdentifier msItem msItemStruct msName msPart musicNotation name nameLink namespace nationality normalization notatedMusic note notesStmt num nym objectDesc objectType occupation offset opener org orgName orig origDate origPlace origin p pb pc persName person personGrp phr physDesc place placeName population postscript prefixDef principal profileDesc projectDesc provenance ptr pubPlace publicationStmt publisher punctuation q quotation quote rdg rdgGrp recordHist redo ref refState refsDecl reg region relatedItem relation rendition repository residence resp respStmt restore retrace revisionDesc roleName row rs rubric s said salute samplingDecl schemaRef scriptDesc scriptNote seal sealDesc secFol secl seg segmentation series seriesStmt settlement sex sic signatures signed soCalled socecStatus source sourceDesc sourceDoc sp span spanGrp speaker sponsor stage stamp state stdVals street styleDefDecl subst substJoin summary supplied support supportDesc surface surfaceGrp surname surplus surrogates table tagUsage tagsDecl taxonomy teiCorpus teiHeader term terrain text textClass textLang time timeline title titlePage titlePart titleStmt trailer trait transpose typeDesc typeNote unclear undo unicodeName value variantEncoding w watermark when width wit witDetail witEnd witStart witness xenoData zone]
    \item[{Attributes}]
  Attributes\hfil\\[-10pt]\begin{sansreflist}
    \item[@corresp]
  (corresponds) points to elements that correspond to the current element in some way.
\begin{reflist}
    \item[{Status}]
  Optional
    \item[{Datatype}]
  1–∞ occurrences of teidata.pointer separated by whitespace
    \item[]\exampleFont {<\textbf{group}>}\mbox{}\newline 
\hspace*{6pt}{<\textbf{text}\hspace*{6pt}{xml:id}="{t1-g1-t1}"\mbox{}\newline 
\hspace*{6pt}\hspace*{6pt}{xml:lang}="{mi}">}\mbox{}\newline 
\hspace*{6pt}\hspace*{6pt}{<\textbf{body}\hspace*{6pt}{xml:id}="{t1-g1-t1-body1}">}\mbox{}\newline 
\hspace*{6pt}\hspace*{6pt}\hspace*{6pt}{<\textbf{div}\hspace*{6pt}{type}="{chapter}">}\mbox{}\newline 
\hspace*{6pt}\hspace*{6pt}\hspace*{6pt}\hspace*{6pt}{<\textbf{head}>}He Whakamaramatanga mo te Ture Hoko, Riihi hoki, i nga Whenua Maori, 1876.{</\textbf{head}>}\mbox{}\newline 
\hspace*{6pt}\hspace*{6pt}\hspace*{6pt}\hspace*{6pt}{<\textbf{p}>}…{</\textbf{p}>}\mbox{}\newline 
\hspace*{6pt}\hspace*{6pt}\hspace*{6pt}{</\textbf{div}>}\mbox{}\newline 
\hspace*{6pt}\hspace*{6pt}{</\textbf{body}>}\mbox{}\newline 
\hspace*{6pt}{</\textbf{text}>}\mbox{}\newline 
\hspace*{6pt}{<\textbf{text}\hspace*{6pt}{xml:id}="{t1-g1-t2}"\mbox{}\newline 
\hspace*{6pt}\hspace*{6pt}{xml:lang}="{en}">}\mbox{}\newline 
\hspace*{6pt}\hspace*{6pt}{<\textbf{body}\hspace*{6pt}{corresp}="{\#t1-g1-t1-body1}"\mbox{}\newline 
\hspace*{6pt}\hspace*{6pt}\hspace*{6pt}{xml:id}="{t1-g1-t2-body1}">}\mbox{}\newline 
\hspace*{6pt}\hspace*{6pt}\hspace*{6pt}{<\textbf{div}\hspace*{6pt}{type}="{chapter}">}\mbox{}\newline 
\hspace*{6pt}\hspace*{6pt}\hspace*{6pt}\hspace*{6pt}{<\textbf{head}>}An Act to regulate the Sale, Letting, and Disposal of Native Lands, 1876.{</\textbf{head}>}\mbox{}\newline 
\hspace*{6pt}\hspace*{6pt}\hspace*{6pt}\hspace*{6pt}{<\textbf{p}>}…{</\textbf{p}>}\mbox{}\newline 
\hspace*{6pt}\hspace*{6pt}\hspace*{6pt}{</\textbf{div}>}\mbox{}\newline 
\hspace*{6pt}\hspace*{6pt}{</\textbf{body}>}\mbox{}\newline 
\hspace*{6pt}{</\textbf{text}>}\mbox{}\newline 
{</\textbf{group}>}In this example a <group> contains two <text>s, each containing the same document in a different language. The correspondence is indicated using {\itshape corresp}. The language is indicated using \texttt{xml:lang}, whose value is inherited; both the tag with the {\itshape corresp} and the tag pointed to by the {\itshape corresp} inherit the value from their immediate parent.
    \item[]\exampleFont \mbox{}\newline 
\textit{<!-- In a placeography -->}{<\textbf{place}\hspace*{6pt}{corresp}="{\#LOND2 \#GENI1}"\mbox{}\newline 
\hspace*{6pt}{xml:id}="{LOND1}">}\mbox{}\newline 
\hspace*{6pt}{<\textbf{placeName}>}London{</\textbf{placeName}>}\mbox{}\newline 
\hspace*{6pt}{<\textbf{desc}>}The city of London...{</\textbf{desc}>}\mbox{}\newline 
{</\textbf{place}>}\mbox{}\newline 
\textit{<!-- In a literary personography -->}\mbox{}\newline 
{<\textbf{person}\hspace*{6pt}{corresp}="{\#LOND1 \#GENI1}"\mbox{}\newline 
\hspace*{6pt}{xml:id}="{LOND2}">}\mbox{}\newline 
\hspace*{6pt}{<\textbf{persName}\hspace*{6pt}{type}="{lit}">}London{</\textbf{persName}>}\mbox{}\newline 
\hspace*{6pt}{<\textbf{note}>}\mbox{}\newline 
\hspace*{6pt}\hspace*{6pt}{<\textbf{p}>}Allegorical character representing the city of {<\textbf{ref}\hspace*{6pt}{target}="{LOND1.xml}">}London{</\textbf{ref}>}.\mbox{}\newline 
\hspace*{6pt}\hspace*{6pt}{</\textbf{p}>}\mbox{}\newline 
\hspace*{6pt}{</\textbf{note}>}\mbox{}\newline 
{</\textbf{person}>}\mbox{}\newline 
{<\textbf{person}\hspace*{6pt}{corresp}="{\#LOND1 \#LOND2}"\mbox{}\newline 
\hspace*{6pt}{xml:id}="{GENI1}">}\mbox{}\newline 
\hspace*{6pt}{<\textbf{persName}\hspace*{6pt}{type}="{lit}">}London’s Genius{</\textbf{persName}>}\mbox{}\newline 
\hspace*{6pt}{<\textbf{note}>}\mbox{}\newline 
\hspace*{6pt}\hspace*{6pt}{<\textbf{p}>}Personification of London’s genius. Appears as an\mbox{}\newline 
\hspace*{6pt}\hspace*{6pt}\hspace*{6pt}\hspace*{6pt} allegorical character in mayoral shows.\mbox{}\newline 
\hspace*{6pt}\hspace*{6pt}{</\textbf{p}>}\mbox{}\newline 
\hspace*{6pt}{</\textbf{note}>}\mbox{}\newline 
{</\textbf{person}>}In this example, a <place> element containing information about the city of London is linked with two <person> elements in a literary personography. This correspondence represents a slightly looser relationship than the one in the preceding example; there is no sense in which an allegorical character could be substituted for the physical city, or vice versa, but there is obviously a correspondence between them.
\end{reflist}  
    \item[@synch]
  (synchronous) points to elements that are synchronous with the current element.
\begin{reflist}
    \item[{Status}]
  Optional
    \item[{Datatype}]
  1–∞ occurrences of teidata.pointer separated by whitespace
\end{reflist}  
    \item[@sameAs]
  points to an element that is the same as the current element.
\begin{reflist}
    \item[{Status}]
  Optional
    \item[{Datatype}]
  teidata.pointer
\end{reflist}  
    \item[@copyOf]
  points to an element of which the current element is a copy.
\begin{reflist}
    \item[{Status}]
  Optional
    \item[{Datatype}]
  teidata.pointer
    \item[{Note}]
  \par
Any content of the current element should be ignored. Its true content is that of the element being pointed at.
\end{reflist}  
    \item[@next]
  points to the next element of a virtual aggregate of which the current element is part.
\begin{reflist}
    \item[{Status}]
  Optional
    \item[{Datatype}]
  teidata.pointer
\end{reflist}  
    \item[@prev]
  (previous) points to the previous element of a virtual aggregate of which the current element is part.
\begin{reflist}
    \item[{Status}]
  Optional
    \item[{Datatype}]
  teidata.pointer
\end{reflist}  
    \item[@exclude]
  points to elements that are in exclusive alternation with the current element.
\begin{reflist}
    \item[{Status}]
  Optional
    \item[{Datatype}]
  1–∞ occurrences of teidata.pointer separated by whitespace
\end{reflist}  
    \item[@select]
  selects one or more alternants; if one alternant is selected, the ambiguity or uncertainty is marked as resolved. If more than one alternant is selected, the degree of ambiguity or uncertainty is marked as reduced by the number of alternants not selected.
\begin{reflist}
    \item[{Status}]
  Optional
    \item[{Datatype}]
  1–∞ occurrences of teidata.pointer separated by whitespace
    \item[{Note}]
  \par
This attribute should be placed on an element which is superordinate to all of the alternants from which the selection is being made.
\end{reflist}  
\end{sansreflist}  
\end{reflist}  
\begin{reflist}
\item[]\begin{specHead}{TEI.att.global.rendition}{att.global.rendition}\index{att.global.rendition (attribute class)|oddindex}\index{rend=@rend!att.global.rendition (attribute class)|oddindex}\index{style=@style!att.global.rendition (attribute class)|oddindex}\index{rendition=@rendition!att.global.rendition (attribute class)|oddindex} provides rendering attributes common to all elements in the TEI encoding scheme. [\xref{http://www.tei-c.org/release/doc/tei-p5-doc/en/html/ST.html\#STGAre}{1.3.1.1.3. Rendition Indicators}]\end{specHead} 
    \item[{Module}]
  tei
    \item[{Members}]
  att.global[TEI ab abbr abstract accMat acquisition add addName addSpan additional additions addrLine address adminInfo affiliation age alt altGrp altIdentifier am analytic anchor app appInfo application argument author authority availability back bibl biblFull biblScope biblStruct binding bindingDesc birth bloc body byline c cRefPattern calendar calendarDesc catDesc catRef catchwords category cb cell change char charDecl charName charProp choice cit citedRange cl classCode classDecl climate closer collation collection colophon condition corr correction correspAction correspContext correspDesc country creation custEvent custodialHist damage damageSpan date dateline death decoDesc decoNote del delSpan depth desc dim dimensions distinct distributor district div divGen docAuthor docDate docEdition docImprint docTitle edition editionStmt editor editorialDecl education email emph encodingDesc epigraph event ex expan explicit extent facsimile faith figDesc figure fileDesc filiation finalRubric floatingText floruit foliation foreign forename formula front funder fw g gap gb genName geo geoDecl geogFeat geogName gloss glyph glyphName graphic group handDesc handNote handNotes handShift head headItem headLabel height heraldry hi history hyphenation idno imprimatur imprint incipit index institution interp interpGrp interpretation item join joinGrp keywords l label lacunaEnd lacunaStart langKnowledge langKnown langUsage language layout layoutDesc lb lem lg licence line link linkGrp list listApp listBibl listChange listEvent listNym listOrg listPerson listPlace listPrefixDef listRelation listTranspose listWit localName location locus locusGrp m mapping material measure measureGrp media meeting mentioned metamark milestone mod monogr msContents msDesc msFrag msIdentifier msItem msItemStruct msName msPart musicNotation name nameLink namespace nationality normalization notatedMusic note notesStmt num nym objectDesc objectType occupation offset opener org orgName orig origDate origPlace origin p pb pc persName person personGrp phr physDesc place placeName population postscript prefixDef principal profileDesc projectDesc provenance ptr pubPlace publicationStmt publisher punctuation q quotation quote rdg rdgGrp recordHist redo ref refState refsDecl reg region relatedItem relation rendition repository residence resp respStmt restore retrace revisionDesc roleName row rs rubric s said salute samplingDecl schemaRef scriptDesc scriptNote seal sealDesc secFol secl seg segmentation series seriesStmt settlement sex sic signatures signed soCalled socecStatus source sourceDesc sourceDoc sp span spanGrp speaker sponsor stage stamp state stdVals street styleDefDecl subst substJoin summary supplied support supportDesc surface surfaceGrp surname surplus surrogates table tagUsage tagsDecl taxonomy teiCorpus teiHeader term terrain text textClass textLang time timeline title titlePage titlePart titleStmt trailer trait transpose typeDesc typeNote unclear undo unicodeName value variantEncoding w watermark when width wit witDetail witEnd witStart witness xenoData zone]
    \item[{Attributes}]
  Attributes\hfil\\[-10pt]\begin{sansreflist}
    \item[@rend]
  (rendition) indicates how the element in question was rendered or presented in the source text.
\begin{reflist}
    \item[{Status}]
  Optional
    \item[{Datatype}]
  1–∞ occurrences of teidata.word separated by whitespace
    \item[]\exampleFont {<\textbf{head}\hspace*{6pt}{rend}="{align(center) case(allcaps)}">}\mbox{}\newline 
\hspace*{6pt}{<\textbf{lb}/>}To The {<\textbf{lb}/>}Duchesse {<\textbf{lb}/>}of {<\textbf{lb}/>}Newcastle,\mbox{}\newline 
{<\textbf{lb}/>}On Her {<\textbf{lb}/>}\mbox{}\newline 
\hspace*{6pt}{<\textbf{hi}\hspace*{6pt}{rend}="{case(mixed)}">}New Blazing-World{</\textbf{hi}>}. \mbox{}\newline 
{</\textbf{head}>}
    \item[{Note}]
  \par
These Guidelines make no binding recommendations for the values of the {\itshape rend} attribute; the characteristics of visual presentation vary too much from text to text and the decision to record or ignore individual characteristics varies too much from project to project. Some potentially useful conventions are noted from time to time at appropriate points in the Guidelines. The values of the {\itshape rend} attribute are a set of sequence-indeterminate individual tokens separated by whitespace.
\end{reflist}  
    \item[@style]
  contains an expression in some formal style definition language which defines the rendering or presentation used for this element in the source text
\begin{reflist}
    \item[{Status}]
  Optional
    \item[{Datatype}]
  teidata.text
    \item[]\exampleFont {<\textbf{head}\hspace*{6pt}{style}="{text-align: center; font-variant: small-caps}">}\mbox{}\newline 
\hspace*{6pt}{<\textbf{lb}/>}To The {<\textbf{lb}/>}Duchesse {<\textbf{lb}/>}of {<\textbf{lb}/>}Newcastle, {<\textbf{lb}/>}On Her\mbox{}\newline 
{<\textbf{lb}/>}\mbox{}\newline 
\hspace*{6pt}{<\textbf{hi}\hspace*{6pt}{style}="{font-variant: normal}">}New Blazing-World{</\textbf{hi}>}. \mbox{}\newline 
{</\textbf{head}>}
    \item[{Note}]
  \par
Unlike the attribute values of {\itshape rend}, which uses whitespace as a separator, the {\itshape style} attribute may contain whitespace. This attribute is intended for recording inline stylistic information concerning the source, not any particular output.\par
The formal language in which values for this attribute are expressed may be specified using the <styleDefDecl> element in the TEI header.
\end{reflist}  
    \item[@rendition]
  points to a description of the rendering or presentation used for this element in the source text.
\begin{reflist}
    \item[{Status}]
  Optional
    \item[{Datatype}]
  1–∞ occurrences of teidata.pointer separated by whitespace
    \item[]\exampleFont {<\textbf{head}\hspace*{6pt}{rendition}="{\#ac \#sc}">}\mbox{}\newline 
\hspace*{6pt}{<\textbf{lb}/>}To The {<\textbf{lb}/>}Duchesse {<\textbf{lb}/>}of {<\textbf{lb}/>}Newcastle, {<\textbf{lb}/>}On Her\mbox{}\newline 
{<\textbf{lb}/>}\mbox{}\newline 
\hspace*{6pt}{<\textbf{hi}\hspace*{6pt}{rendition}="{\#normal}">}New Blazing-World{</\textbf{hi}>}. \mbox{}\newline 
{</\textbf{head}>}\mbox{}\newline 
\textit{<!-- elsewhere... -->}\mbox{}\newline 
{<\textbf{rendition}\hspace*{6pt}{scheme}="{css}"\mbox{}\newline 
\hspace*{6pt}{xml:id}="{sc}">}font-variant: small-caps{</\textbf{rendition}>}\mbox{}\newline 
{<\textbf{rendition}\hspace*{6pt}{scheme}="{css}"\mbox{}\newline 
\hspace*{6pt}{xml:id}="{normal}">}font-variant: normal{</\textbf{rendition}>}\mbox{}\newline 
{<\textbf{rendition}\hspace*{6pt}{scheme}="{css}"\mbox{}\newline 
\hspace*{6pt}{xml:id}="{ac}">}text-align: center{</\textbf{rendition}>}
    \item[{Note}]
  \par
The {\itshape rendition} attribute is used in a very similar way to the {\itshape class} attribute defined for XHTML but with the important distinction that its function is to describe the appearance of the source text, not necessarily to determine how that text should be presented on screen or paper.\par
Where both {\itshape rendition} and {\itshape rend} are supplied, the latter is understood to override or complement the former.\par
Each URI provided should indicate a <rendition> element defining the intended rendition in terms of some appropriate style language, as indicated by the {\itshape scheme} attribute.
\end{reflist}  
\end{sansreflist}  
\end{reflist}  
\begin{reflist}
\item[]\begin{specHead}{TEI.att.global.responsibility}{att.global.responsibility}\index{att.global.responsibility (attribute class)|oddindex}\index{cert=@cert!att.global.responsibility (attribute class)|oddindex}\index{resp=@resp!att.global.responsibility (attribute class)|oddindex} provides attributes indicating the agent responsible for some aspect of the text, the markup or something asserted by the markup, and the degree of certainty associated with it. [\xref{http://www.tei-c.org/release/doc/tei-p5-doc/en/html/ST.html\#STGAso}{1.3.1.1.4. Sources, certainty, and responsibility} \xref{http://www.tei-c.org/release/doc/tei-p5-doc/en/html/CO.html\#COED}{3.4. Simple Editorial Changes} \xref{http://www.tei-c.org/release/doc/tei-p5-doc/en/html/PH.html\#PHHR}{11.3.2.2. Hand, Responsibility, and Certainty Attributes} \xref{http://www.tei-c.org/release/doc/tei-p5-doc/en/html/AI.html\#AISP}{17.3. Spans and Interpretations} \xref{http://www.tei-c.org/release/doc/tei-p5-doc/en/html/ND.html\#NDATTSnr}{13.1.1. Linking Names and Their Referents}]\end{specHead} 
    \item[{Module}]
  tei
    \item[{Members}]
  att.global[TEI ab abbr abstract accMat acquisition add addName addSpan additional additions addrLine address adminInfo affiliation age alt altGrp altIdentifier am analytic anchor app appInfo application argument author authority availability back bibl biblFull biblScope biblStruct binding bindingDesc birth bloc body byline c cRefPattern calendar calendarDesc catDesc catRef catchwords category cb cell change char charDecl charName charProp choice cit citedRange cl classCode classDecl climate closer collation collection colophon condition corr correction correspAction correspContext correspDesc country creation custEvent custodialHist damage damageSpan date dateline death decoDesc decoNote del delSpan depth desc dim dimensions distinct distributor district div divGen docAuthor docDate docEdition docImprint docTitle edition editionStmt editor editorialDecl education email emph encodingDesc epigraph event ex expan explicit extent facsimile faith figDesc figure fileDesc filiation finalRubric floatingText floruit foliation foreign forename formula front funder fw g gap gb genName geo geoDecl geogFeat geogName gloss glyph glyphName graphic group handDesc handNote handNotes handShift head headItem headLabel height heraldry hi history hyphenation idno imprimatur imprint incipit index institution interp interpGrp interpretation item join joinGrp keywords l label lacunaEnd lacunaStart langKnowledge langKnown langUsage language layout layoutDesc lb lem lg licence line link linkGrp list listApp listBibl listChange listEvent listNym listOrg listPerson listPlace listPrefixDef listRelation listTranspose listWit localName location locus locusGrp m mapping material measure measureGrp media meeting mentioned metamark milestone mod monogr msContents msDesc msFrag msIdentifier msItem msItemStruct msName msPart musicNotation name nameLink namespace nationality normalization notatedMusic note notesStmt num nym objectDesc objectType occupation offset opener org orgName orig origDate origPlace origin p pb pc persName person personGrp phr physDesc place placeName population postscript prefixDef principal profileDesc projectDesc provenance ptr pubPlace publicationStmt publisher punctuation q quotation quote rdg rdgGrp recordHist redo ref refState refsDecl reg region relatedItem relation rendition repository residence resp respStmt restore retrace revisionDesc roleName row rs rubric s said salute samplingDecl schemaRef scriptDesc scriptNote seal sealDesc secFol secl seg segmentation series seriesStmt settlement sex sic signatures signed soCalled socecStatus source sourceDesc sourceDoc sp span spanGrp speaker sponsor stage stamp state stdVals street styleDefDecl subst substJoin summary supplied support supportDesc surface surfaceGrp surname surplus surrogates table tagUsage tagsDecl taxonomy teiCorpus teiHeader term terrain text textClass textLang time timeline title titlePage titlePart titleStmt trailer trait transpose typeDesc typeNote unclear undo unicodeName value variantEncoding w watermark when width wit witDetail witEnd witStart witness xenoData zone]
    \item[{Attributes}]
  Attributes\hfil\\[-10pt]\begin{sansreflist}
    \item[@cert]
  (certainty) signifies the degree of certainty associated with the intervention or interpretation.
\begin{reflist}
    \item[{Status}]
  Optional
    \item[{Datatype}]
  teidata.probCert
\end{reflist}  
    \item[@resp]
  (responsible party) indicates the agency responsible for the intervention or interpretation, for example an editor or transcriber.
\begin{reflist}
    \item[{Status}]
  Optional
    \item[{Datatype}]
  1–∞ occurrences of teidata.pointer separated by whitespace
    \item[{Note}]
  \par
To reduce the ambiguity of a {\itshape resp} pointing directly to a person or organization, we recommend that {\itshape resp} be used to point not to an agent (<person> or <org>) but to a <respStmt>, <author>, <editor> or similar element which clarifies the exact role played by the agent. Pointing to multiple <respStmt>s allows the encoder to specify clearly each of the roles played in part of a TEI file (creating, transcribing, encoding, editing, proofing etc.).
\end{reflist}  
\end{sansreflist}  
    \item[{Example}]
  \leavevmode\bgroup\exampleFont \begin{shaded}\noindent\mbox{}Blessed are the\mbox{}\newline 
{<\textbf{choice}>}\mbox{}\newline 
\hspace*{6pt}{<\textbf{sic}>}cheesemakers{</\textbf{sic}>}\mbox{}\newline 
\hspace*{6pt}{<\textbf{corr}\hspace*{6pt}{cert}="{high}"\hspace*{6pt}{resp}="{\#editor}">}peacemakers{</\textbf{corr}>}\mbox{}\newline 
{</\textbf{choice}>}: for they shall be called the children of God.\end{shaded}\egroup 


    \item[{Example}]
  \leavevmode\bgroup\exampleFont \begin{shaded}\noindent\mbox{}\mbox{}\newline 
\textit{<!-- in the <text> ... -->}{<\textbf{lg}>}\mbox{}\newline 
\textit{<!-- ... -->}\mbox{}\newline 
\hspace*{6pt}{<\textbf{l}>}Punkes, Panders, baſe extortionizing\mbox{}\newline 
\hspace*{6pt}\hspace*{6pt} sla{<\textbf{choice}>}\mbox{}\newline 
\hspace*{6pt}\hspace*{6pt}\hspace*{6pt}{<\textbf{sic}>}n{</\textbf{sic}>}\mbox{}\newline 
\hspace*{6pt}\hspace*{6pt}\hspace*{6pt}{<\textbf{corr}\hspace*{6pt}{resp}="{\#JENS1\textunderscore transcriber}">}u{</\textbf{corr}>}\mbox{}\newline 
\hspace*{6pt}\hspace*{6pt}{</\textbf{choice}>}es,{</\textbf{l}>}\mbox{}\newline 
\textit{<!-- ... -->}\mbox{}\newline 
{</\textbf{lg}>}\mbox{}\newline 
\textit{<!-- in the <teiHeader> ... -->}\mbox{}\newline 
\textit{<!-- ... -->}\mbox{}\newline 
{<\textbf{respStmt}\hspace*{6pt}{xml:id}="{JENS1\textunderscore transcriber}">}\mbox{}\newline 
\hspace*{6pt}{<\textbf{resp}\hspace*{6pt}{when}="{2014}">}Transcriber{</\textbf{resp}>}\mbox{}\newline 
\hspace*{6pt}{<\textbf{name}>}Janelle Jenstad{</\textbf{name}>}\mbox{}\newline 
{</\textbf{respStmt}>}\end{shaded}\egroup 


\end{reflist}  
\begin{reflist}
\item[]\begin{specHead}{TEI.att.global.source}{att.global.source}\index{att.global.source (attribute class)|oddindex}\index{source=@source!att.global.source (attribute class)|oddindex} provides an attribute used by elements to point to an external source. [\xref{http://www.tei-c.org/release/doc/tei-p5-doc/en/html/ST.html\#STGAso}{1.3.1.1.4. Sources, certainty, and responsibility} \xref{http://www.tei-c.org/release/doc/tei-p5-doc/en/html/CO.html\#COHQQ}{3.3.3. Quotation} \xref{http://www.tei-c.org/release/doc/tei-p5-doc/en/html/TS.html\#TSBAWR}{8.3.4. Writing}]\end{specHead} 
    \item[{Module}]
  tei
    \item[{Members}]
  att.global[TEI ab abbr abstract accMat acquisition add addName addSpan additional additions addrLine address adminInfo affiliation age alt altGrp altIdentifier am analytic anchor app appInfo application argument author authority availability back bibl biblFull biblScope biblStruct binding bindingDesc birth bloc body byline c cRefPattern calendar calendarDesc catDesc catRef catchwords category cb cell change char charDecl charName charProp choice cit citedRange cl classCode classDecl climate closer collation collection colophon condition corr correction correspAction correspContext correspDesc country creation custEvent custodialHist damage damageSpan date dateline death decoDesc decoNote del delSpan depth desc dim dimensions distinct distributor district div divGen docAuthor docDate docEdition docImprint docTitle edition editionStmt editor editorialDecl education email emph encodingDesc epigraph event ex expan explicit extent facsimile faith figDesc figure fileDesc filiation finalRubric floatingText floruit foliation foreign forename formula front funder fw g gap gb genName geo geoDecl geogFeat geogName gloss glyph glyphName graphic group handDesc handNote handNotes handShift head headItem headLabel height heraldry hi history hyphenation idno imprimatur imprint incipit index institution interp interpGrp interpretation item join joinGrp keywords l label lacunaEnd lacunaStart langKnowledge langKnown langUsage language layout layoutDesc lb lem lg licence line link linkGrp list listApp listBibl listChange listEvent listNym listOrg listPerson listPlace listPrefixDef listRelation listTranspose listWit localName location locus locusGrp m mapping material measure measureGrp media meeting mentioned metamark milestone mod monogr msContents msDesc msFrag msIdentifier msItem msItemStruct msName msPart musicNotation name nameLink namespace nationality normalization notatedMusic note notesStmt num nym objectDesc objectType occupation offset opener org orgName orig origDate origPlace origin p pb pc persName person personGrp phr physDesc place placeName population postscript prefixDef principal profileDesc projectDesc provenance ptr pubPlace publicationStmt publisher punctuation q quotation quote rdg rdgGrp recordHist redo ref refState refsDecl reg region relatedItem relation rendition repository residence resp respStmt restore retrace revisionDesc roleName row rs rubric s said salute samplingDecl schemaRef scriptDesc scriptNote seal sealDesc secFol secl seg segmentation series seriesStmt settlement sex sic signatures signed soCalled socecStatus source sourceDesc sourceDoc sp span spanGrp speaker sponsor stage stamp state stdVals street styleDefDecl subst substJoin summary supplied support supportDesc surface surfaceGrp surname surplus surrogates table tagUsage tagsDecl taxonomy teiCorpus teiHeader term terrain text textClass textLang time timeline title titlePage titlePart titleStmt trailer trait transpose typeDesc typeNote unclear undo unicodeName value variantEncoding w watermark when width wit witDetail witEnd witStart witness xenoData zone]
    \item[{Attributes}]
  Attributes\hfil\\[-10pt]\begin{sansreflist}
    \item[@source]
  specifies the source from which some aspect of this element is drawn.
\begin{reflist}
    \item[{Status}]
  Optional
    \item[{Datatype}]
  1–∞ occurrences of teidata.pointer separated by whitespace
    \item[{Note}]
  \par
The {\itshape source} attribute points to an external source. When used on elements describing schema components it locates the source for the ODD processor from which declarations and definitions for the components of the object being defined may be obtained. On other elements it provides a pointer to the bibliographical source from which a quotation or citation is drawn. These are provided as any form of URI, for example an absolute URI, a relative URI, or private scheme URI that is expanded to an absolute URI as documented in a <prefixDef>.
\end{reflist}  
\end{sansreflist}  
    \item[{Example}]
  \leavevmode\bgroup\exampleFont \begin{shaded}\noindent\mbox{}{<\textbf{p}>}\mbox{}\newline 
\textit{<!-- ... -->}\mbox{}\newline 
 As Willard McCarty ({<\textbf{bibl}\hspace*{6pt}{xml:id}="{mcc\textunderscore 2012}">}2012, p.2{</\textbf{bibl}>})\mbox{}\newline 
 tells us, {<\textbf{quote}\hspace*{6pt}{source}="{\#mcc\textunderscore 2012}">}‘Collaboration’ is a\mbox{}\newline 
\hspace*{6pt}\hspace*{6pt} problematic and should be a contested term.{</\textbf{quote}>}\mbox{}\newline 
\textit{<!-- ... -->}\mbox{}\newline 
{</\textbf{p}>}\end{shaded}\egroup 


    \item[{Example}]
  \leavevmode\bgroup\exampleFont \begin{shaded}\noindent\mbox{}{<\textbf{p}>}\mbox{}\newline 
\textit{<!-- ... -->}\mbox{}\newline 
\hspace*{6pt}{<\textbf{quote}\hspace*{6pt}{source}="{\#chicago\textunderscore 15\textunderscore ed}">}Grammatical theories\mbox{}\newline 
\hspace*{6pt}\hspace*{6pt} are in flux, and the more we learn, the less we\mbox{}\newline 
\hspace*{6pt}\hspace*{6pt} seem to know.{</\textbf{quote}>}\mbox{}\newline 
\textit{<!-- ... -->}\mbox{}\newline 
{</\textbf{p}>}\mbox{}\newline 
\textit{<!-- ... -->}\mbox{}\newline 
{<\textbf{bibl}\hspace*{6pt}{xml:id}="{chicago\textunderscore 15\textunderscore ed}">}\mbox{}\newline 
\hspace*{6pt}{<\textbf{title}\hspace*{6pt}{level}="{m}">}The Chicago Manual of Style{</\textbf{title}>},\mbox{}\newline 
{<\textbf{edition}>}15th edition{</\textbf{edition}>}.\mbox{}\newline 
{<\textbf{pubPlace}>}Chicago{</\textbf{pubPlace}>}:\mbox{}\newline 
{<\textbf{publisher}>}University of Chicago Press{</\textbf{publisher}>} \mbox{}\newline 
 ({<\textbf{date}>}2003{</\textbf{date}>}),\mbox{}\newline 
{<\textbf{biblScope}\hspace*{6pt}{unit}="{page}">}p.147{</\textbf{biblScope}>}.\mbox{}\newline 
\mbox{}\newline 
{</\textbf{bibl}>}\end{shaded}\egroup 


    \item[{Example}]
  \leavevmode\bgroup\exampleFont \begin{shaded}\noindent\mbox{}{<\textbf{elementRef}\hspace*{6pt}{key}="{p}"\hspace*{6pt}{source}="{tei:2.0.1}"/>}\end{shaded}\egroup 

Include in the schema an element named <p> available from the TEI P5 2.0.1 release.
\end{reflist}  
\begin{reflist}
\item[]\begin{specHead}{TEI.att.handFeatures}{att.handFeatures}\index{att.handFeatures (attribute class)|oddindex}\index{scribe=@scribe!att.handFeatures (attribute class)|oddindex}\index{scribeRef=@scribeRef!att.handFeatures (attribute class)|oddindex}\index{script=@script!att.handFeatures (attribute class)|oddindex}\index{scriptRef=@scriptRef!att.handFeatures (attribute class)|oddindex}\index{medium=@medium!att.handFeatures (attribute class)|oddindex}\index{scope=@scope!att.handFeatures (attribute class)|oddindex} provides attributes describing aspects of the hand in which a manuscript is written. [\xref{http://www.tei-c.org/release/doc/tei-p5-doc/en/html/PH.html\#PHDH}{11.3.2.1. Document Hands}]\end{specHead} 
    \item[{Module}]
  tei
    \item[{Members}]
  handNote handShift scriptNote typeNote
    \item[{Attributes}]
  Attributes\hfil\\[-10pt]\begin{sansreflist}
    \item[@scribe]
  gives a name or other identifier for the scribe believed to be responsible for this hand.
\begin{reflist}
    \item[{Status}]
  Optional
    \item[{Datatype}]
  teidata.name
\end{reflist}  
    \item[@scribeRef]
  points to a full description of the scribe concerned, typically supplied by a <person> element elsewhere in the description.
\begin{reflist}
    \item[{Status}]
  Optional
    \item[{Datatype}]
  1–∞ occurrences of teidata.pointer separated by whitespace
\end{reflist}  
    \item[@script]
  characterizes the particular script or writing style used by this hand, for example \textit{secretary}, \textit{copperplate}, \textit{Chancery}, \textit{Italian}, etc.
\begin{reflist}
    \item[{Status}]
  Optional
    \item[{Datatype}]
  1–∞ occurrences of teidata.name separated by whitespace
\end{reflist}  
    \item[@scriptRef]
  points to a full description of the script or writing style used by this hand, typically supplied by a <scriptNote> element elsewhere in the description.
\begin{reflist}
    \item[{Status}]
  Optional
    \item[{Datatype}]
  1–∞ occurrences of teidata.pointer separated by whitespace
\end{reflist}  
    \item[@medium]
  describes the tint or type of ink, e.g. \textit{brown}, or other writing medium, e.g. \textit{pencil}
\begin{reflist}
    \item[{Status}]
  Optional
    \item[{Datatype}]
  1–∞ occurrences of teidata.enumerated separated by whitespace
\end{reflist}  
    \item[@scope]
  specifies how widely this hand is used in the manuscript.
\begin{reflist}
    \item[{Status}]
  Optional
    \item[{Datatype}]
  teidata.enumerated
    \item[{Legal values are:}]
  \begin{description}

\item[{sole}]only this hand is used throughout the manuscript
\item[{major}]this hand is used through most of the manuscript
\item[{minor}]this hand is used occasionally in the manuscript
\end{description} 
\end{reflist}  
\end{sansreflist}  
    \item[{Note}]
  \par
Usually either {\itshape script} or {\itshape scriptRef}, and similarly, either {\itshape scribe} or {\itshape scribeRef}, will be supplied.
\end{reflist}  
\begin{reflist}
\item[]\begin{specHead}{TEI.att.internetMedia}{att.internetMedia}\index{att.internetMedia (attribute class)|oddindex}\index{mimeType=@mimeType!att.internetMedia (attribute class)|oddindex} provides attributes for specifying the type of a computer resource using a standard taxonomy.\end{specHead} 
    \item[{Module}]
  tei
    \item[{Members}]
  att.media[graphic] ptr ref
    \item[{Attributes}]
  Attributes\hfil\\[-10pt]\begin{sansreflist}
    \item[@mimeType]
  (MIME media type) specifies the applicable multimedia internet mail extension (MIME) media type
\begin{reflist}
    \item[{Status}]
  Optional
    \item[{Datatype}]
  1–∞ occurrences of teidata.word separated by whitespace
\end{reflist}  
\end{sansreflist}  
    \item[{Example}]
  In this example {\itshape mimeType} is used to indicate that the URL points to a TEI XML file encoded in UTF-8.\leavevmode\bgroup\exampleFont \begin{shaded}\noindent\mbox{}{<\textbf{ref}\hspace*{6pt}{mimeType}="{application/tei+xml; charset=UTF-8}"\mbox{}\newline 
\hspace*{6pt}{target}="{http://sourceforge.net/p/tei/code/HEAD/tree/trunk/P5/Source/guidelines-en.xml}"/>}\end{shaded}\egroup 


    \item[{Note}]
  \par
This attribute class provides an attribute for describing a computer resource, typically available over the internet, using a value taken from a standard taxonomy. At present only a single taxonomy is supported, the Multipurpose Internet Mail Extensions (MIME) Media Type system. This typology of media types is defined by the Internet Engineering Task Force in \xref{http://www.ietf.org/rfc/rfc2046.txt}{RFC 2046}. The \xref{http://www.iana.org/assignments/media-types/}{list of types} is maintained by the Internet Assigned Numbers Authority (IANA). The {\itshape mimeType} attribute must have a value taken from this list.
\end{reflist}  
\begin{reflist}
\item[]\begin{specHead}{TEI.att.interpLike}{att.interpLike}\index{att.interpLike (attribute class)|oddindex}\index{type=@type!att.interpLike (attribute class)|oddindex}\index{inst=@inst!att.interpLike (attribute class)|oddindex} provides attributes for elements which represent a formal analysis or interpretation. [\xref{http://www.tei-c.org/release/doc/tei-p5-doc/en/html/AI.html\#AIATTS}{17.2. Global Attributes for Simple Analyses}]\end{specHead} 
    \item[{Module}]
  tei
    \item[{Members}]
  interp interpGrp span spanGrp
    \item[{Attributes}]
  Attributes\hfil\\[-10pt]\begin{sansreflist}
    \item[@type]
  indicates what kind of phenomenon is being noted in the passage.
\begin{reflist}
    \item[{Status}]
  Recommended
    \item[{Datatype}]
  teidata.enumerated
    \item[{Sample values include:}]
  \begin{description}

\item[{image}]identifies an image in the passage.
\item[{character}]identifies a character associated with the passage.
\item[{theme}]identifies a theme in the passage.
\item[{allusion}]identifies an allusion to another text.
\end{description} 
\end{reflist}  
    \item[@inst]
  (instances) points to instances of the analysis or interpretation represented by the current element.
\begin{reflist}
    \item[{Status}]
  Optional
    \item[{Datatype}]
  1–∞ occurrences of teidata.pointer separated by whitespace
    \item[{Note}]
  \par
The current element should be an analytic one. The element pointed at should be a textual one.
\end{reflist}  
\end{sansreflist}  
\end{reflist}  
\begin{reflist}
\item[]\begin{specHead}{TEI.att.measurement}{att.measurement}\index{att.measurement (attribute class)|oddindex}\index{unit=@unit!att.measurement (attribute class)|oddindex}\index{quantity=@quantity!att.measurement (attribute class)|oddindex}\index{commodity=@commodity!att.measurement (attribute class)|oddindex} provides attributes to represent a regularized or normalized measurement.\end{specHead} 
    \item[{Module}]
  tei
    \item[{Members}]
  measure measureGrp
    \item[{Attributes}]
  Attributes\hfil\\[-10pt]\begin{sansreflist}
    \item[@unit]
  indicates the units used for the measurement, usually using the standard symbol for the desired units.
\begin{reflist}
    \item[{Status}]
  Optional
    \item[{Datatype}]
  teidata.enumerated
    \item[{Suggested values include:}]
  \begin{description}

\item[{m}](metre) SI base unit of length
\item[{kg}](kilogram) SI base unit of mass
\item[{s}](second) SI base unit of time
\item[{Hz}](hertz) SI unit of frequency
\item[{Pa}](pascal) SI unit of pressure or stress
\item[{Ω}](ohm) SI unit of electric resistance
\item[{L}](litre) 1 dm³
\item[{t}](tonne) 10³ kg
\item[{ha}](hectare) 1 hm²
\item[{Å}](ångström) 10⁻¹⁰ m
\item[{mL}](millilitre)
\item[{cm}](centimetre)
\item[{dB}](decibel) see remarks, below
\item[{kbit}](kilobit) 10³ or 1000 bits
\item[{Kibit}](kibibit) 2¹⁰ or 1024 bits
\item[{kB}](kilobyte) 10³ or 1000 bytes
\item[{KiB}](kibibyte) 2¹⁰ or 1024 bytes
\item[{MB}](megabyte) 10⁶ or 1 000 000 bytes
\item[{MiB}](mebibyte) 2²⁰ or 1 048 576 bytes
\end{description} 
    \item[{Note}]
  \par
If the measurement being represented is not expressed in a particular unit, but rather is a number of discrete items, the unit count should be used, or the {\itshape unit} attribute may be left unspecified.\par
Wherever appropriate, a recognized SI unit name should be used (see further \url{http://www.bipm.org/en/publications/si-brochure/}; \url{http://physics.nist.gov/cuu/Units/}). The list above is indicative rather than exhaustive.
\end{reflist}  
    \item[@quantity]
  specifies the number of the specified units that comprise the measurement
\begin{reflist}
    \item[{Status}]
  Optional
    \item[{Datatype}]
  teidata.numeric
\end{reflist}  
    \item[@commodity]
  indicates the substance that is being measured
\begin{reflist}
    \item[{Status}]
  Optional
    \item[{Datatype}]
  1–∞ occurrences of teidata.word separated by whitespace
    \item[{Note}]
  \par
In general, when the commodity is made of discrete entities, the plural form should be used, even when the measurement is of only one of them.
\end{reflist}  
\end{sansreflist}  
    \item[{Note}]
  \par
This attribute class provides a triplet of attributes that may be used either to regularize the values of the measurement being encoded, or to normalize them with respect to a standard measurement system. \par\bgroup\exampleFont \begin{shaded}\noindent\mbox{}{<\textbf{l}>}So weren't you gonna buy {<\textbf{measure}\hspace*{6pt}{commodity}="{ice cream}"\mbox{}\newline 
\hspace*{6pt}\hspace*{6pt}{quantity}="{0.5}"\hspace*{6pt}{unit}="{gal}">}half\mbox{}\newline 
\hspace*{6pt}\hspace*{6pt} a gallon{</\textbf{measure}>}, baby{</\textbf{l}>}\mbox{}\newline 
{<\textbf{l}>}So won't you go and buy {<\textbf{measure}\hspace*{6pt}{commodity}="{ice cream}"\mbox{}\newline 
\hspace*{6pt}\hspace*{6pt}{quantity}="{1.893}"\hspace*{6pt}{unit}="{L}">}half\mbox{}\newline 
\hspace*{6pt}\hspace*{6pt} a gallon{</\textbf{measure}>}, baby?{</\textbf{l}>}\end{shaded}\egroup\par 
    \item[{Note}]
  \par
The unit should normally be named using the standard abbreviation for an SI unit (see further \url{http://www.bipm.org/en/publications/si-brochure/}; \url{http://physics.nist.gov/cuu/Units/}). However, encoders may also specify measurements using informally defined units such as lines or characters.
\end{reflist}  
\begin{reflist}
\item[]\begin{specHead}{TEI.att.media}{att.media}\index{att.media (attribute class)|oddindex}\index{width=@width!att.media (attribute class)|oddindex}\index{height=@height!att.media (attribute class)|oddindex}\index{scale=@scale!att.media (attribute class)|oddindex} provides attributes for specifying display and related properties of external media.\end{specHead} 
    \item[{Module}]
  tei
    \item[{Members}]
  graphic
    \item[{Attributes}]
  Attributes att.internetMedia (\textit{@mimeType}) \hfil\\[-10pt]\begin{sansreflist}
    \item[@width]
  Where the media are displayed, indicates the display width
\begin{reflist}
    \item[{Status}]
  Optional
    \item[{Datatype}]
  teidata.outputMeasurement
\end{reflist}  
    \item[@height]
  Where the media are displayed, indicates the display height
\begin{reflist}
    \item[{Status}]
  Optional
    \item[{Datatype}]
  teidata.outputMeasurement
\end{reflist}  
    \item[@scale]
  Where the media are displayed, indicates a scale factor to be applied when generating the desired display size
\begin{reflist}
    \item[{Status}]
  Optional
    \item[{Datatype}]
  teidata.numeric
\end{reflist}  
\end{sansreflist}  
\end{reflist}  
\begin{reflist}
\item[]\begin{specHead}{TEI.att.milestoneUnit}{att.milestoneUnit}\index{att.milestoneUnit (attribute class)|oddindex}\index{unit=@unit!att.milestoneUnit (attribute class)|oddindex} provides an attribute to indicate the type of section which is changing at a specific milestone. [\xref{http://www.tei-c.org/release/doc/tei-p5-doc/en/html/CO.html\#CORS5}{3.10.3. Milestone Elements} \xref{http://www.tei-c.org/release/doc/tei-p5-doc/en/html/HD.html\#HD54M}{2.3.6.3. Milestone Method} \xref{http://www.tei-c.org/release/doc/tei-p5-doc/en/html/HD.html\#HD54}{2.3.6. The Reference System Declaration}]\end{specHead} 
    \item[{Module}]
  core
    \item[{Members}]
  milestone refState
    \item[{Attributes}]
  Attributes\hfil\\[-10pt]\begin{sansreflist}
    \item[@unit]
  provides a conventional name for the kind of section changing at this milestone.
\begin{reflist}
    \item[{Status}]
  Required
    \item[{Datatype}]
  teidata.enumerated
    \item[{Suggested values include:}]
  \begin{description}

\item[{page}]physical page breaks (synonymous with the <pb> element).
\item[{column}]column breaks.
\item[{line}]line breaks (synonymous with the <lb> element).
\item[{book}]any units termed book, liber, etc.
\item[{poem}]individual poems in a collection.
\item[{canto}]cantos or other major sections of a poem.
\item[{speaker}]changes of speaker or narrator.
\item[{stanza}]stanzas within a poem, book, or canto.
\item[{act}]acts within a play.
\item[{scene}]scenes within a play or act.
\item[{section}]sections of any kind.
\item[{absent}]passages not present in the reference edition.
\item[{unnumbered}]passages present in the text, but not to be included as part of the reference.
\end{description} 
    \item[]\exampleFont {<\textbf{milestone}\hspace*{6pt}{ed}="{La}"\mbox{}\newline 
\hspace*{6pt}{n}="{23}"\mbox{}\newline 
\hspace*{6pt}{unit}="{Dreissiger}"/>}\mbox{}\newline 
 ... {<\textbf{milestone}\hspace*{6pt}{ed}="{AV}"\mbox{}\newline 
\hspace*{6pt}{n}="{24}"\mbox{}\newline 
\hspace*{6pt}{unit}="{verse}"/>} ...
    \item[{Note}]
  \par
If the milestone marks the beginning of a piece of text not present in the reference edition, the special value \textit{absent} may be used as the value of {\itshape unit}. The normal interpretation is that the reference edition does not contain the text which follows, until the next <milestone> tag for the edition in question is encountered.\par
In addition to the values suggested, other terms may be appropriate (e.g. \textit{Stephanus} for the Stephanus numbers in Plato).
\end{reflist}  
\end{sansreflist}  
\end{reflist}  
\begin{reflist}
\item[]\begin{specHead}{TEI.att.msExcerpt}{att.msExcerpt}\index{att.msExcerpt (attribute class)|oddindex}\index{defective=@defective!att.msExcerpt (attribute class)|oddindex} (manuscript excerpt) provides attributes used to describe excerpts from a manuscript placed in a description thereof. [\xref{http://www.tei-c.org/release/doc/tei-p5-doc/en/html/MS.html\#msco}{10.6. Intellectual Content}]\end{specHead} 
    \item[{Module}]
  msdescription
    \item[{Members}]
  explicit incipit msContents msItem msItemStruct quote
    \item[{Attributes}]
  Attributes\hfil\\[-10pt]\begin{sansreflist}
    \item[@defective]
  indicates whether the passage being quoted is defective, i.e. incomplete through loss or damage.
\begin{reflist}
    \item[{Status}]
  Optional
    \item[{Datatype}]
  teidata.xTruthValue
    \item[{Default}]
  false \par \begin{tabular}{P{0.4969230769230769\textwidth}P{0.35307692307692307\textwidth}}
\xref{http://www.tei-c.org/Activities/Council/Working/tcw27.xml}{Deprecated}\tabcellsep The value will no longer be a default after 2017-09-05\end{tabular}
\end{reflist}  
\end{sansreflist}  
    \item[{Note}]
  \par
In the case of an incipit, indicates whether the incipit as given is defective, i.e. the first words of the text as preserved, as opposed to the first words of the work itself. In the case of an explicit, indicates whether the explicit as given is defective, i.e. the final words of the text as preserved, as opposed to what the closing words would have been had the text of the work been whole.
\end{reflist}  
\begin{reflist}
\item[]\begin{specHead}{TEI.att.naming}{att.naming}\index{att.naming (attribute class)|oddindex}\index{role=@role!att.naming (attribute class)|oddindex}\index{nymRef=@nymRef!att.naming (attribute class)|oddindex} provides attributes common to elements which refer to named persons, places, organizations etc. [\xref{http://www.tei-c.org/release/doc/tei-p5-doc/en/html/CO.html\#CONARS}{3.5.1. Referring Strings} \xref{http://www.tei-c.org/release/doc/tei-p5-doc/en/html/ND.html\#NDNYM}{13.3.5. Names and Nyms}]\end{specHead} 
    \item[{Module}]
  tei
    \item[{Members}]
  att.personal[addName forename genName name orgName persName placeName roleName surname] affiliation author birth bloc climate collection country death district editor education event geogFeat geogName institution nationality occupation offset origPlace population pubPlace region repository residence rs settlement socecStatus state terrain trait
    \item[{Attributes}]
  Attributes att.canonical (\textit{@key}, \textit{@ref}) \hfil\\[-10pt]\begin{sansreflist}
    \item[@role]
  may be used to specify further information about the entity referenced by this name in the form of a set of whitespace-separated values, for example the occupation of a person, or the status of a place.
\begin{reflist}
    \item[{Status}]
  Optional
    \item[{Datatype}]
  1–∞ occurrences of teidata.enumerated separated by whitespace
\end{reflist}  
    \item[@nymRef]
  (reference to the canonical name) provides a means of locating the canonical form (\textit{nym}) of the names associated with the object named by the element bearing it.
\begin{reflist}
    \item[{Status}]
  Optional
    \item[{Datatype}]
  1–∞ occurrences of teidata.pointer separated by whitespace
    \item[{Note}]
  \par
The value must point directly to one or more XML elements by means of one or more URIs, separated by whitespace. If more than one is supplied, the implication is that the name is associated with several distinct canonical names.
\end{reflist}  
\end{sansreflist}  
\end{reflist}  
\begin{reflist}
\item[]\begin{specHead}{TEI.att.notated}{att.notated}\index{att.notated (attribute class)|oddindex}\index{notation=@notation!att.notated (attribute class)|oddindex} provides an attribute to indicate any specialised notation used for element content.\end{specHead} 
    \item[{Module}]
  tei
    \item[{Members}]
  formula
    \item[{Attributes}]
  Attributes\hfil\\[-10pt]\begin{sansreflist}
    \item[@notation]
  names the notation used for the content of the element.
\begin{reflist}
    \item[{Status}]
  Optional
    \item[{Datatype}]
  teidata.enumerated
\end{reflist}  
\end{sansreflist}  
\end{reflist}  
\begin{reflist}
\item[]\begin{specHead}{TEI.att.patternReplacement}{att.patternReplacement}\index{att.patternReplacement (attribute class)|oddindex}\index{matchPattern=@matchPattern!att.patternReplacement (attribute class)|oddindex}\index{replacementPattern=@replacementPattern!att.patternReplacement (attribute class)|oddindex} provides attributes for regular-expression matching and replacement. [\xref{http://www.tei-c.org/release/doc/tei-p5-doc/en/html/SA.html\#SAPU}{16.2.3. Using Abbreviated Pointers} \xref{http://www.tei-c.org/release/doc/tei-p5-doc/en/html/HD.html\#HD54M}{2.3.6.3. Milestone Method} \xref{http://www.tei-c.org/release/doc/tei-p5-doc/en/html/HD.html\#HD54}{2.3.6. The Reference System Declaration} \xref{http://www.tei-c.org/release/doc/tei-p5-doc/en/html/HD.html\#HD54S}{2.3.6.2. Search-and-Replace Method}]\end{specHead} 
    \item[{Module}]
  header
    \item[{Members}]
  cRefPattern prefixDef
    \item[{Attributes}]
  Attributes\hfil\\[-10pt]\begin{sansreflist}
    \item[@matchPattern]
  specifies a regular expression against which the values of other attributes can be matched.
\begin{reflist}
    \item[{Status}]
  Required
    \item[{Datatype}]
  teidata.pattern
    \item[{Note}]
  \par
The syntax used should follow that defined by \xref{http://www.w3.org/TR/xpath-functions/\#regex-syntax}{W3C XPath syntax}. Note that parenthesized groups are used not only for establishing order of precedence and atoms for quantification, but also for creating subpatterns to be referenced by the {\itshape replacementPattern} attribute.
\end{reflist}  
    \item[@replacementPattern]
  specifies a ‘replacement pattern’, that is, the skeleton of a relative or absolute URI containing references to groups in the {\itshape matchPattern} which, once subpattern substitution has been performed, complete the URI.
\begin{reflist}
    \item[{Status}]
  Required
    \item[{Datatype}]
  teidata.replacement
    \item[{Note}]
  \par
The strings \textit{\$1}, \textit{\$2} etc. are references to the corresponding group in the regular expression specified by {\itshape matchPattern} (counting open parenthesis, left to right). Processors are expected to replace them with whatever matched the corresponding group in the regular expression.\par
If a digit preceded by a dollar sign is needed in the actual replacement pattern (as opposed to being used as a back reference), the dollar sign must be written as \texttt{\%24}.
\end{reflist}  
\end{sansreflist}  
\end{reflist}  
\begin{reflist}
\item[]\begin{specHead}{TEI.att.personal}{att.personal}\index{att.personal (attribute class)|oddindex}\index{full=@full!att.personal (attribute class)|oddindex}\index{sort=@sort!att.personal (attribute class)|oddindex} (attributes for components of names usually, but not necessarily, personal names) common attributes for those elements which form part of a name usually, but not necessarily, a personal name. [\xref{http://www.tei-c.org/release/doc/tei-p5-doc/en/html/ND.html\#NDPER}{13.2.1. Personal Names}]\end{specHead} 
    \item[{Module}]
  tei
    \item[{Members}]
  addName forename genName name orgName persName placeName roleName surname
    \item[{Attributes}]
  Attributes att.naming (\textit{@role}, \textit{@nymRef})  (att.canonical (\textit{@key}, \textit{@ref})) \hfil\\[-10pt]\begin{sansreflist}
    \item[@full]
  indicates whether the name component is given in full, as an abbreviation or simply as an initial.
\begin{reflist}
    \item[{Status}]
  Optional
    \item[{Datatype}]
  teidata.enumerated
    \item[{Legal values are:}]
  \begin{description}

\item[{yes}]the name component is spelled out in full.{[Default] }
\item[{abb}](abbreviated) the name component is given in an abbreviated form.
\item[{init}](initial letter) the name component is indicated only by one initial.
\end{description} 
\end{reflist}  
    \item[@sort]
  specifies the sort order of the name component in relation to others within the name.
\begin{reflist}
    \item[{Status}]
  Optional
    \item[{Datatype}]
  teidata.count
\end{reflist}  
\end{sansreflist}  
\end{reflist}  
\begin{reflist}
\item[]\begin{specHead}{TEI.att.placement}{att.placement}\index{att.placement (attribute class)|oddindex}\index{place=@place!att.placement (attribute class)|oddindex} provides attributes for describing where on the source page or object a textual element appears. [\xref{http://www.tei-c.org/release/doc/tei-p5-doc/en/html/CO.html\#COEDADD}{3.4.3. Additions, Deletions, and Omissions} \xref{http://www.tei-c.org/release/doc/tei-p5-doc/en/html/PH.html\#PHAD}{11.3.1.4. Additions and Deletions}]\end{specHead} 
    \item[{Module}]
  tei
    \item[{Members}]
  add addSpan figure fw label metamark notatedMusic note stage witDetail
    \item[{Attributes}]
  Attributes\hfil\\[-10pt]\begin{sansreflist}
    \item[@place]
  specifies where this item is placed.
\begin{reflist}
    \item[{Status}]
  Recommended
    \item[{Datatype}]
  1–∞ occurrences of teidata.enumerated separated by whitespace
    \item[{Suggested values include:}]
  \begin{description}

\item[{below}]below the line
\item[{bottom}]at the foot of the page
\item[{margin}]in the margin (left, right, or both)
\item[{top}]at the top of the page
\item[{opposite}]on the opposite, i.e. facing, page
\item[{overleaf}]on the other side of the leaf
\item[{above}]above the line
\item[{end}]at the end of e.g. chapter or volume.
\item[{inline}]within the body of the text.
\item[{inspace}]in a predefined space, for example left by an earlier scribe.
\end{description} 
    \item[]\exampleFont {<\textbf{add}\hspace*{6pt}{place}="{margin}">}[An addition written in the margin]{</\textbf{add}>}\mbox{}\newline 
{<\textbf{add}\hspace*{6pt}{place}="{bottom opposite}">}[An addition written at the\mbox{}\newline 
 foot of the current page and also on the facing page]{</\textbf{add}>}
    \item[]\exampleFont {<\textbf{note}\hspace*{6pt}{place}="{bottom}">}Ibid, p.7{</\textbf{note}>}
\end{reflist}  
\end{sansreflist}  
\end{reflist}  
\begin{reflist}
\item[]\begin{specHead}{TEI.att.pointing}{att.pointing}\index{att.pointing (attribute class)|oddindex}\index{targetLang=@targetLang!att.pointing (attribute class)|oddindex}\index{target=@target!att.pointing (attribute class)|oddindex}\index{evaluate=@evaluate!att.pointing (attribute class)|oddindex} provides a set of attributes used by all elements which point to other elements by means of one or more URI references. [\xref{http://www.tei-c.org/release/doc/tei-p5-doc/en/html/ST.html\#STGAla}{1.3.1.1.2. Language Indicators} \xref{http://www.tei-c.org/release/doc/tei-p5-doc/en/html/CO.html\#COXR}{3.6. Simple Links and Cross-References}]\end{specHead} 
    \item[{Module}]
  tei
    \item[{Members}]
  att.pointing.group[altGrp joinGrp linkGrp] calendar catRef citedRange gloss join licence link locus note ptr ref span substJoin term witDetail
    \item[{Attributes}]
  Attributes\hfil\\[-10pt]\begin{sansreflist}
    \item[@targetLang]
  specifies the language of the content to be found at the destination referenced by {\itshape target}, using a ‘language tag’ generated according to \xref{http://www.rfc-editor.org/rfc/bcp/bcp47.txt}{BCP 47}.
\begin{reflist}
    \item[{Status}]
  Optional
    \item[{Datatype}]
  teidata.language
    \item[{Schematron}]
   <sch:rule context="tei:*[not(self::tei:schemaSpec)][@targetLang]"> <sch:assert test="@target">@targetLang should only be used on <sch:name/> if @target is specified.</sch:assert> </sch:rule>
    \item[]\exampleFont {<\textbf{linkGrp}\hspace*{6pt}{xml:id}="{pol-swh\textunderscore aln\textunderscore 2.1-linkGrp}">}\mbox{}\newline 
\hspace*{6pt}{<\textbf{ptr}\hspace*{6pt}{target}="{pol/UDHR/text.xml\#pol\textunderscore txt\textunderscore 1-head}"\mbox{}\newline 
\hspace*{6pt}\hspace*{6pt}{targetLang}="{pl}"\mbox{}\newline 
\hspace*{6pt}\hspace*{6pt}{type}="{tuv}"\mbox{}\newline 
\hspace*{6pt}\hspace*{6pt}{xml:id}="{pol-swh\textunderscore aln\textunderscore 2.1.1-ptr}"/>}\mbox{}\newline 
\hspace*{6pt}{<\textbf{ptr}\hspace*{6pt}{target}="{swh/UDHR/text.xml\#swh\textunderscore txt\textunderscore 1-head}"\mbox{}\newline 
\hspace*{6pt}\hspace*{6pt}{targetLang}="{sw}"\mbox{}\newline 
\hspace*{6pt}\hspace*{6pt}{type}="{tuv}"\mbox{}\newline 
\hspace*{6pt}\hspace*{6pt}{xml:id}="{pol-swh\textunderscore aln\textunderscore 2.1.2-ptr}"/>}\mbox{}\newline 
{</\textbf{linkGrp}>}In the example above, the <linkGrp> combines pointers at parallel fragments of the \textit{Universal Declaration of Human Rights}: one of them is in Polish, the other in Swahili.
    \item[{Note}]
  \par
The value must conform to BCP 47. If the value is a private use code (i.e., starts with x- or contains -x-), a <language> element with a matching value for its {\itshape ident} attribute should be supplied in the TEI header to document this value. Such documentation may also optionally be supplied for non-private-use codes, though these must remain consistent with their  ( {\abbr IETF}) {\expan Internet Engineering Task Force} definitions.
\end{reflist}  
    \item[@target]
  specifies the destination of the reference by supplying one or more URI References
\begin{reflist}
    \item[{Status}]
  Optional
    \item[{Datatype}]
  1–∞ occurrences of teidata.pointer separated by whitespace
    \item[{Note}]
  \par
One or more syntactically valid URI references, separated by whitespace. Because whitespace is used to separate URIs, no whitespace is permitted inside a single URI. If a whitespace character is required in a URI, it should be escaped with the normal mechanism, e.g. \texttt{TEI\%20Consortium}.
\end{reflist}  
    \item[@evaluate]
  specifies the intended meaning when the target of a pointer is itself a pointer.
\begin{reflist}
    \item[{Status}]
  Optional
    \item[{Datatype}]
  teidata.enumerated
    \item[{Legal values are:}]
  \begin{description}

\item[{all}]if the element pointed to is itself a pointer, then the target of that pointer will be taken, and so on, until an element is found which is not a pointer.
\item[{one}]if the element pointed to is itself a pointer, then its target (whether a pointer or not) is taken as the target of this pointer.
\item[{none}]no further evaluation of targets is carried out beyond that needed to find the element specified in the pointer's target.
\end{description} 
    \item[{Note}]
  \par
If no value is given, the application program is responsible for deciding (possibly on the basis of user input) how far to trace a chain of pointers.
\end{reflist}  
\end{sansreflist}  
\end{reflist}  
\begin{reflist}
\item[]\begin{specHead}{TEI.att.pointing.group}{att.pointing.group}\index{att.pointing.group (attribute class)|oddindex}\index{domains=@domains!att.pointing.group (attribute class)|oddindex}\index{targFunc=@targFunc!att.pointing.group (attribute class)|oddindex} provides a set of attributes common to all elements which enclose groups of pointer elements. [\xref{http://www.tei-c.org/release/doc/tei-p5-doc/en/html/SA.html\#SA}{16. Linking, Segmentation, and Alignment}]\end{specHead} 
    \item[{Module}]
  tei
    \item[{Members}]
  altGrp joinGrp linkGrp
    \item[{Attributes}]
  Attributes att.pointing (\textit{@targetLang}, \textit{@target}, \textit{@evaluate}) att.typed (\textit{@type}, \textit{@subtype}) \hfil\\[-10pt]\begin{sansreflist}
    \item[@domains]
  optionally specifies the identifiers of the elements within which all elements indicated by the contents of this element lie.
\begin{reflist}
    \item[{Status}]
  Optional
    \item[{Datatype}]
  2–∞ occurrences of teidata.pointer separated by whitespace
    \item[{Note}]
  \par
If this attribute is supplied every element specified as a target must be contained within the element or elements named by it. An application may choose whether or not to report failures to satisfy this constraint as errors, but may not access an element of the right identifier but in the wrong context. If this attribute is not supplied, then target elements may appear anywhere within the target document.
\end{reflist}  
    \item[@targFunc]
  (target function) describes the function of each of the values of the {\itshape target} attribute of the enclosed <link>, <join>, or <alt> tags.
\begin{reflist}
    \item[{Status}]
  Optional
    \item[{Datatype}]
  2–∞ occurrences of teidata.word separated by whitespace
    \item[{Note}]
  \par
The number of separate values must match the number of values in the {\itshape target} attribute in the enclosed <link>, <join>, or <alt> tags (an intermediate <ptr> element may be needed to accomplish this). It should also match the number of values in the {\itshape domains} attribute, of the current element, if one has been specified.
\end{reflist}  
\end{sansreflist}  
\end{reflist}  
\begin{reflist}
\item[]\begin{specHead}{TEI.att.ranging}{att.ranging}\index{att.ranging (attribute class)|oddindex}\index{atLeast=@atLeast!att.ranging (attribute class)|oddindex}\index{atMost=@atMost!att.ranging (attribute class)|oddindex}\index{min=@min!att.ranging (attribute class)|oddindex}\index{max=@max!att.ranging (attribute class)|oddindex}\index{confidence=@confidence!att.ranging (attribute class)|oddindex} provides attributes for describing numerical ranges.\end{specHead} 
    \item[{Module}]
  tei
    \item[{Members}]
  att.dimensions[att.damaged[damage damageSpan] att.editLike[att.transcriptional[add addSpan del delSpan mod redo restore retrace subst substJoin undo] affiliation age am birth climate corr date death education event ex expan faith floruit gap geogFeat geogName langKnowledge langKnown location name nationality occupation offset org orgName origDate origPlace origin persName person place placeName population reg relation residence secl sex socecStatus state supplied surplus terrain time trait unclear] depth dim dimensions height space width] num
    \item[{Attributes}]
  Attributes\hfil\\[-10pt]\begin{sansreflist}
    \item[@atLeast]
  gives a minimum estimated value for the approximate measurement.
\begin{reflist}
    \item[{Status}]
  Optional
    \item[{Datatype}]
  teidata.numeric
\end{reflist}  
    \item[@atMost]
  gives a maximum estimated value for the approximate measurement.
\begin{reflist}
    \item[{Status}]
  Optional
    \item[{Datatype}]
  teidata.numeric
\end{reflist}  
    \item[@min]
  where the measurement summarizes more than one observation or a range, supplies the minimum value observed.
\begin{reflist}
    \item[{Status}]
  Optional
    \item[{Datatype}]
  teidata.numeric
\end{reflist}  
    \item[@max]
  where the measurement summarizes more than one observation or a range, supplies the maximum value observed.
\begin{reflist}
    \item[{Status}]
  Optional
    \item[{Datatype}]
  teidata.numeric
\end{reflist}  
    \item[@confidence]
  specifies the degree of statistical confidence (between zero and one) that a value falls within the range specified by {\itshape min} and {\itshape max}, or the proportion of observed values that fall within that range.
\begin{reflist}
    \item[{Status}]
  Optional
    \item[{Datatype}]
  teidata.probability
\end{reflist}  
\end{sansreflist}  
    \item[{Example}]
  \leavevmode\bgroup\exampleFont \begin{shaded}\noindent\mbox{}The MS. was lost in transmission by mail from {<\textbf{del}\hspace*{6pt}{rend}="{overstrike}">}\mbox{}\newline 
\hspace*{6pt}{<\textbf{gap}\hspace*{6pt}{atLeast}="{1}"\hspace*{6pt}{atMost}="{2}"\mbox{}\newline 
\hspace*{6pt}\hspace*{6pt}{extent}="{one or two letters}"\hspace*{6pt}{reason}="{illegible}"\hspace*{6pt}{unit}="{chars}"/>}\mbox{}\newline 
{</\textbf{del}>} Philadelphia to the Graphic office, New York.\mbox{}\newline 
\end{shaded}\egroup 


\end{reflist}  
\begin{reflist}
\item[]\begin{specHead}{TEI.att.rdgPart}{att.rdgPart}\index{att.rdgPart (attribute class)|oddindex}\index{wit=@wit!att.rdgPart (attribute class)|oddindex} provides attributes to mark the beginning or ending of a fragmentary manuscript or other witness. [\xref{http://www.tei-c.org/release/doc/tei-p5-doc/en/html/TC.html\#TCAPMI}{12.1.5. Fragmentary Witnesses}]\end{specHead} 
    \item[{Module}]
  textcrit
    \item[{Members}]
  lacunaEnd lacunaStart wit witEnd witStart
    \item[{Attributes}]
  Attributes\hfil\\[-10pt]\begin{sansreflist}
    \item[@wit]
  (witness or witnesses) contains a space-delimited list of one or more sigla indicating the witnesses to this reading beginning or ending at this point.
\begin{reflist}
    \item[{Status}]
  Optional
    \item[{Datatype}]
  1–∞ occurrences of teidata.pointer separated by whitespace
\end{reflist}  
\end{sansreflist}  
    \item[{Note}]
  \par
These elements may appear anywhere within the elements <lem> and <rdg>, and also within any of their constituent elements.
\end{reflist}  
\begin{reflist}
\item[]\begin{specHead}{TEI.att.resourced}{att.resourced}\index{att.resourced (attribute class)|oddindex}\index{url=@url!att.resourced (attribute class)|oddindex} provides attributes by which a resource (such as an externally held media file) may be located.\end{specHead} 
    \item[{Module}]
  tei
    \item[{Members}]
  graphic media schemaRef
    \item[{Attributes}]
  Attributes\hfil\\[-10pt]\begin{sansreflist}
    \item[@url]
  (uniform resource locator) specifies the URL from which the media concerned may be obtained.
\begin{reflist}
    \item[{Status}]
  Required
    \item[{Datatype}]
  teidata.pointer
\end{reflist}  
\end{sansreflist}  
\end{reflist}  
\begin{reflist}
\item[]\begin{specHead}{TEI.att.segLike}{att.segLike}\index{att.segLike (attribute class)|oddindex}\index{function=@function!att.segLike (attribute class)|oddindex} provides attributes for elements used for arbitrary segmentation. [\xref{http://www.tei-c.org/release/doc/tei-p5-doc/en/html/SA.html\#SASE}{16.3. Blocks, Segments, and Anchors} \xref{http://www.tei-c.org/release/doc/tei-p5-doc/en/html/AI.html\#AILC}{17.1. Linguistic Segment Categories}]\end{specHead} 
    \item[{Module}]
  tei
    \item[{Members}]
  c cl m pc phr s seg w
    \item[{Attributes}]
  Attributes att.datcat (\textit{@datcat}, \textit{@valueDatcat}) att.fragmentable (\textit{@part}) \hfil\\[-10pt]\begin{sansreflist}
    \item[@function]
  characterizes the function of the segment.
\begin{reflist}
    \item[{Status}]
  Optional
    \item[{Datatype}]
  teidata.enumerated
    \item[{Note}]
  \par
Attribute values will often vary depending on the type of element to which they are attached. For example, a <cl>, may take values such as coordinate, subject, adverbial etc. For a <phr>, such values as subject, predicate etc. may be more appropriate. Such constraints will typically be implemented by a project-defined customization.
\end{reflist}  
\end{sansreflist}  
\end{reflist}  
\begin{reflist}
\item[]\begin{specHead}{TEI.att.sortable}{att.sortable}\index{att.sortable (attribute class)|oddindex}\index{sortKey=@sortKey!att.sortable (attribute class)|oddindex} provides attributes for elements in lists or groups that are sortable, but whose sorting key cannot be derived mechanically from the element content. [\xref{http://www.tei-c.org/release/doc/tei-p5-doc/en/html/DI.html\#DIBO}{9.1. Dictionary Body and Overall Structure}]\end{specHead} 
    \item[{Module}]
  tei
    \item[{Members}]
  bibl biblFull biblStruct correspAction event idno item list listApp listBibl listChange listEvent listNym listOrg listPerson listPlace listRelation listWit msDesc nym org person personGrp place relation term witness
    \item[{Attributes}]
  Attributes\hfil\\[-10pt]\begin{sansreflist}
    \item[@sortKey]
  supplies the sort key for this element in an index, list or group which contains it.
\begin{reflist}
    \item[{Status}]
  Optional
    \item[{Datatype}]
  teidata.word
    \item[]\exampleFont David's other principal backer, Josiah\mbox{}\newline 
 ha-Kohen {<\textbf{index}\hspace*{6pt}{indexName}="{NAMES}">}\mbox{}\newline 
\hspace*{6pt}{<\textbf{term}\hspace*{6pt}{sortKey}="{Azarya\textunderscore Josiah\textunderscore Kohen}">}Josiah ha-Kohen b. Azarya{</\textbf{term}>}\mbox{}\newline 
{</\textbf{index}>} b. Azarya, son of one of the last gaons of Sura was David's own first\mbox{}\newline 
 cousin.
    \item[{Note}]
  \par
The sort key is used to determine the sequence and grouping of entries in an index. It provides a sequence of characters which, when sorted with the other values, will produced the desired order; specifics of sort key construction are application-dependent\par
Dictionary order often differs from the collation sequence of machine-readable character sets; in English-language dictionaries, an entry for \textit{4-H} will often appear alphabetized under ‘fourh’, and \textit{McCoy} may be alphabetized under ‘maccoy’, while \textit{A1}, \textit{A4}, and \textit{A5} may all appear in numeric order ‘alphabetized’ between ‘a-’ and ‘AA’. The sort key is required if the orthography of the dictionary entry does not suffice to determine its location.
\end{reflist}  
\end{sansreflist}  
\end{reflist}  
\begin{reflist}
\item[]\begin{specHead}{TEI.att.spanning}{att.spanning}\index{att.spanning (attribute class)|oddindex}\index{spanTo=@spanTo!att.spanning (attribute class)|oddindex} provides attributes for elements which delimit a span of text by pointing mechanisms rather than by enclosing it. [\xref{http://www.tei-c.org/release/doc/tei-p5-doc/en/html/PH.html\#PHAD}{11.3.1.4. Additions and Deletions} \xref{http://www.tei-c.org/release/doc/tei-p5-doc/en/html/ST.html\#STECAT}{1.3.1. Attribute Classes}]\end{specHead} 
    \item[{Module}]
  tei
    \item[{Members}]
  addSpan cb damageSpan delSpan gb index lb metamark milestone mod pb redo retrace undo
    \item[{Attributes}]
  Attributes\hfil\\[-10pt]\begin{sansreflist}
    \item[@spanTo]
  indicates the end of a span initiated by the element bearing this attribute.
\begin{reflist}
    \item[{Status}]
  Optional
    \item[{Datatype}]
  teidata.pointer
    \item[{Schematron}]
  The @spanTo attribute must point to an element following the current element <sch:rule context="tei:*[@spanTo]"> <sch:assert test="id(substring(@spanTo,2)) and following::*[@xml:id=substring(current()/@spanTo,2)]">The element indicated by @spanTo (<sch:value-of select="@spanTo"/>) must follow the current element <sch:name/> </sch:assert> </sch:rule>
\end{reflist}  
\end{sansreflist}  
    \item[{Note}]
  \par
The span is defined as running in document order from the start of the content of the pointing element to the end of the content of the element pointed to by the {\itshape spanTo} attribute (if any). If no value is supplied for the attribute, the assumption is that the span is coextensive with the pointing element. If no content is present, the assumption is that the starting point of the span is immediately following the element itself.
\end{reflist}  
\begin{reflist}
\item[]\begin{specHead}{TEI.att.styleDef}{att.styleDef}\index{att.styleDef (attribute class)|oddindex}\index{scheme=@scheme!att.styleDef (attribute class)|oddindex}\index{schemeVersion=@schemeVersion!att.styleDef (attribute class)|oddindex} provides attributes to specify the name of a formal definition language used to provide formatting or rendition information.\end{specHead} 
    \item[{Module}]
  tei
    \item[{Members}]
  rendition styleDefDecl
    \item[{Attributes}]
  Attributes\hfil\\[-10pt]\begin{sansreflist}
    \item[@scheme]
  identifies the language used to describe the rendition.
\begin{reflist}
    \item[{Status}]
  Optional
    \item[{Datatype}]
  teidata.enumerated
    \item[{Legal values are:}]
  \begin{description}

\item[{css}]Cascading Stylesheet Language
\item[{xslfo}]Extensible Stylesheet Language Formatting Objects
\item[{free}]Informal free text description
\item[{other}]A user-defined rendition description language
\end{description} 
    \item[{Note}]
  \par
If no value for the @scheme attribute is provided, then the default assumption should be that CSS is in use. 
\end{reflist}  
    \item[@schemeVersion]
  supplies a version number for the style language provided in {\itshape scheme}.
\begin{reflist}
    \item[{Status}]
  Optional
    \item[{Datatype}]
  teidata.versionNumber
    \item[{Schematron}]
   <sch:rule context="tei:*[@schemeVersion]"> <sch:assert test="@scheme and not(@scheme = 'free')"> @schemeVersion can only be used if @scheme is specified. </sch:assert> </sch:rule>
    \item[{Note}]
  \par
If {\itshape schemeVersion} is used, then {\itshape scheme} should also appear, with a value other than free.
\end{reflist}  
\end{sansreflist}  
\end{reflist}  
\begin{reflist}
\item[]\begin{specHead}{TEI.att.tableDecoration}{att.tableDecoration}\index{att.tableDecoration (attribute class)|oddindex}\index{role=@role!att.tableDecoration (attribute class)|oddindex}\index{rows=@rows!att.tableDecoration (attribute class)|oddindex}\index{cols=@cols!att.tableDecoration (attribute class)|oddindex} provides attributes used to decorate rows or cells of a table. [\xref{http://www.tei-c.org/release/doc/tei-p5-doc/en/html/FT.html\#FT}{14. Tables, Formulæ, Graphics and Notated Music}]\end{specHead} 
    \item[{Module}]
  tei
    \item[{Members}]
  cell row
    \item[{Attributes}]
  Attributes\hfil\\[-10pt]\begin{sansreflist}
    \item[@role]
  indicates the kind of information held in this cell or in each cell of this row.
\begin{reflist}
    \item[{Status}]
  Optional
    \item[{Datatype}]
  teidata.enumerated
    \item[{Suggested values include:}]
  \begin{description}

\item[{label}]labelling or descriptive information only.
\item[{data}]data values.{[Default] }
\end{description} 
    \item[{Note}]
  \par
When this attribute is specified on a row, its value is the default for all cells in this row. When specified on a cell, its value overrides any default specified by the {\itshape role} attribute of the parent <row> element.
\end{reflist}  
    \item[@rows]
  indicates the number of rows occupied by this cell or row.
\begin{reflist}
    \item[{Status}]
  Optional
    \item[{Datatype}]
  teidata.count
    \item[{Default}]
  1
    \item[{Note}]
  \par
A value greater than one indicates that this cell spans several rows. Where several cells span multiple rows, it may be more convenient to use nested tables.
\end{reflist}  
    \item[@cols]
  (columns) indicates the number of columns occupied by this cell or row.
\begin{reflist}
    \item[{Status}]
  Optional
    \item[{Datatype}]
  teidata.count
    \item[{Default}]
  1
    \item[{Note}]
  \par
A value greater than one indicates that this cell or row spans several columns. Where an initial cell spans an entire row, it may be better treated as a heading.
\end{reflist}  
\end{sansreflist}  
\end{reflist}  
\begin{reflist}
\item[]\begin{specHead}{TEI.att.textCritical}{att.textCritical}\index{att.textCritical (attribute class)|oddindex}\index{type=@type!att.textCritical (attribute class)|oddindex}\index{cause=@cause!att.textCritical (attribute class)|oddindex}\index{varSeq=@varSeq!att.textCritical (attribute class)|oddindex}\index{require=@require!att.textCritical (attribute class)|oddindex} defines a set of attributes common to all elements representing variant readings in text critical work. [\xref{http://www.tei-c.org/release/doc/tei-p5-doc/en/html/TC.html\#TCAPLL}{12.1. The Apparatus Entry, Readings, and Witnesses}]\end{specHead} 
    \item[{Module}]
  textcrit
    \item[{Members}]
  lem rdg rdgGrp
    \item[{Attributes}]
  Attributes att.written (\textit{@hand}) \hfil\\[-10pt]\begin{sansreflist}
    \item[@type]
  classifies the reading according to some useful typology.
\begin{reflist}
    \item[{Status}]
  Optional
    \item[{Datatype}]
  teidata.enumerated
    \item[{Sample values include:}]
  \begin{description}

\item[{substantive}]the reading offers a substantive variant.
\item[{orthographic}]the reading differs only orthographically, not in substance, from other readings.
\end{description} 
\end{reflist}  
    \item[@cause]
  classifies the cause for the variant reading, according to any appropriate typology of possible origins.
\begin{reflist}
    \item[{Status}]
  Optional
    \item[{Datatype}]
  teidata.enumerated
    \item[{Sample values include:}]
  \begin{description}

\item[{homeoteleuton}]
\item[{homeoarchy}]
\item[{paleographicConfusion}]
\item[{haplography}]
\item[{dittography}]
\item[{falseEmendation}]
\end{description} 
\end{reflist}  
    \item[@varSeq]
  (variant sequence) provides a number indicating the position of this reading in a sequence, when there is reason to presume a sequence to the variants. 
\begin{reflist}
    \item[{Status}]
  Optional
    \item[{Datatype}]
  teidata.count
    \item[{Note}]
  \par
Different variant sequences could be coded with distinct number trails: 1-2-3 for one sequence, 5-6-7 for another. More complex variant sequences, with (for example) multiple branchings from single readings, may be expressed through the <join> element.
\end{reflist}  
    \item[@require]
  points to other readings that are required when adopting the current reading or lemma.
\begin{reflist}
    \item[{Status}]
  Optional
    \item[{Datatype}]
  1–∞ occurrences of teidata.pointer separated by whitespace
\end{reflist}  
\end{sansreflist}  
    \item[{Note}]
  \par
This element class defines attributes inherited by <rdg>, <lem>, and <rdgGrp>.
\end{reflist}  
\begin{reflist}
\item[]\begin{specHead}{TEI.att.timed}{att.timed}\index{att.timed (attribute class)|oddindex}\index{start=@start!att.timed (attribute class)|oddindex}\index{end=@end!att.timed (attribute class)|oddindex} provides attributes common to those elements which have a duration in time, expressed either absolutely or by reference to an alignment map. [\xref{http://www.tei-c.org/release/doc/tei-p5-doc/en/html/TS.html\#TSBATI}{8.3.5. Temporal Information}]\end{specHead} 
    \item[{Module}]
  tei
    \item[{Members}]
  gap media
    \item[{Attributes}]
  Attributes\hfil\\[-10pt]\begin{sansreflist}
    \item[@start]
  indicates the location within a temporal alignment at which this element begins.
\begin{reflist}
    \item[{Status}]
  Optional
    \item[{Datatype}]
  teidata.pointer
    \item[{Note}]
  \par
If no value is supplied, the element is assumed to follow the immediately preceding element at the same hierarchic level.
\end{reflist}  
    \item[@end]
  indicates the location within a temporal alignment at which this element ends.
\begin{reflist}
    \item[{Status}]
  Optional
    \item[{Datatype}]
  teidata.pointer
    \item[{Note}]
  \par
If no value is supplied, the element is assumed to precede the immediately following element at the same hierarchic level.
\end{reflist}  
\end{sansreflist}  
\end{reflist}  
\begin{reflist}
\item[]\begin{specHead}{TEI.att.transcriptional}{att.transcriptional}\index{att.transcriptional (attribute class)|oddindex}\index{status=@status!att.transcriptional (attribute class)|oddindex}\index{cause=@cause!att.transcriptional (attribute class)|oddindex}\index{seq=@seq!att.transcriptional (attribute class)|oddindex} provides attributes specific to elements encoding authorial or scribal intervention in a text when transcribing manuscript or similar sources. [\xref{http://www.tei-c.org/release/doc/tei-p5-doc/en/html/PH.html\#PHAD}{11.3.1.4. Additions and Deletions}]\end{specHead} 
    \item[{Module}]
  tei
    \item[{Members}]
  add addSpan del delSpan mod redo restore retrace subst substJoin undo
    \item[{Attributes}]
  Attributes att.editLike (\textit{@evidence}, \textit{@instant})  (att.dimensions (\textit{@unit}, \textit{@quantity}, \textit{@extent}, \textit{@precision}, \textit{@scope}) (att.ranging (\textit{@atLeast}, \textit{@atMost}, \textit{@min}, \textit{@max}, \textit{@confidence})) ) att.written (\textit{@hand}) \hfil\\[-10pt]\begin{sansreflist}
    \item[@status]
  indicates the effect of the intervention, for example in the case of a deletion, strikeouts which include too much or too little text, or in the case of an addition, an insertion which duplicates some of the text already present.
\begin{reflist}
    \item[{Status}]
  Optional
    \item[{Datatype}]
  teidata.enumerated
    \item[{Sample values include:}]
  \begin{description}

\item[{duplicate}]all of the text indicated as an addition duplicates some text that is in the original, whether the duplication is word-for-word or less exact.
\item[{duplicate-partial}]part of the text indicated as an addition duplicates some text that is in the original
\item[{excessStart}]some text at the beginning of the deletion is marked as deleted even though it clearly should not be deleted.
\item[{excessEnd}]some text at the end of the deletion is marked as deleted even though it clearly should not be deleted.
\item[{shortStart}]some text at the beginning of the deletion is not marked as deleted even though it clearly should be.
\item[{shortEnd}]some text at the end of the deletion is not marked as deleted even though it clearly should be.
\item[{partial}]some text in the deletion is not marked as deleted even though it clearly should be.
\item[{unremarkable}]the deletion is not faulty.{[Default] }
\end{description} 
    \item[{Note}]
  \par
Status information on each deletion is needed rather rarely except in critical editions from authorial manuscripts; status information on additions is even less common.\par
Marking a deletion or addition as faulty is inescapably an interpretive act; the usual test applied in practice is the linguistic acceptability of the text with and without the letters or words in question.
\end{reflist}  
    \item[@cause]
  documents the presumed cause for the intervention.
\begin{reflist}
    \item[{Status}]
  Optional
    \item[{Datatype}]
  teidata.enumerated
\end{reflist}  
    \item[@seq]
  (sequence) assigns a sequence number related to the order in which the encoded features carrying this attribute are believed to have occurred.
\begin{reflist}
    \item[{Status}]
  Optional
    \item[{Datatype}]
  teidata.count
\end{reflist}  
\end{sansreflist}  
\end{reflist}  
\begin{reflist}
\item[]\begin{specHead}{TEI.att.translatable}{att.translatable}\index{att.translatable (attribute class)|oddindex}\index{versionDate=@versionDate!att.translatable (attribute class)|oddindex} provides attributes used to indicate the status of a translatable portion of an ODD document.\end{specHead} 
    \item[{Module}]
  tei
    \item[{Members}]
  desc gloss
    \item[{Attributes}]
  Attributes\hfil\\[-10pt]\begin{sansreflist}
    \item[@versionDate]
  specifies the date on which the source text was extracted and sent to the translator
\begin{reflist}
    \item[{Status}]
  Optional
    \item[{Datatype}]
  teidata.temporal.w3c
    \item[{Note}]
  \par
The {\itshape versionDate} attribute can be used to determine whether a translation might need to be revisited, by comparing the modification date on the containing file with the {\itshape versionDate} value on the translation. If the file has changed, changelogs can be checked to see whether the source text has been modified since the translation was made.
\end{reflist}  
\end{sansreflist}  
\end{reflist}  
\begin{reflist}
\item[]\begin{specHead}{TEI.att.typed}{att.typed}\index{att.typed (attribute class)|oddindex}\index{type=@type!att.typed (attribute class)|oddindex}\index{subtype=@subtype!att.typed (attribute class)|oddindex} provides attributes which can be used to classify or subclassify elements in any way. [\xref{http://www.tei-c.org/release/doc/tei-p5-doc/en/html/ST.html\#STECAT}{1.3.1. Attribute Classes} \xref{http://www.tei-c.org/release/doc/tei-p5-doc/en/html/AI.html\#AILCW}{17.1.1. Words and Above} \xref{http://www.tei-c.org/release/doc/tei-p5-doc/en/html/CO.html\#CONARS}{3.5.1. Referring Strings} \xref{http://www.tei-c.org/release/doc/tei-p5-doc/en/html/CO.html\#COXR}{3.6. Simple Links and Cross-References} \xref{http://www.tei-c.org/release/doc/tei-p5-doc/en/html/CO.html\#CONAAB}{3.5.5. Abbreviations and Their Expansions} \xref{http://www.tei-c.org/release/doc/tei-p5-doc/en/html/CO.html\#COVE}{3.12.1. Core Tags for Verse} \xref{http://www.tei-c.org/release/doc/tei-p5-doc/en/html/DR.html\#DRPAL}{7.2.5. Speech Contents} \xref{http://www.tei-c.org/release/doc/tei-p5-doc/en/html/DS.html\#DSDIV1}{4.1.1. Un-numbered Divisions} \xref{http://www.tei-c.org/release/doc/tei-p5-doc/en/html/DS.html\#DSDIV2}{4.1.2. Numbered Divisions} \xref{http://www.tei-c.org/release/doc/tei-p5-doc/en/html/DS.html\#DSHD}{4.2.1. Headings and Trailers} \xref{http://www.tei-c.org/release/doc/tei-p5-doc/en/html/DS.html\#DSVIRT}{4.4. Virtual Divisions} \xref{http://www.tei-c.org/release/doc/tei-p5-doc/en/html/ND.html\#NDPERSREL}{13.3.2.3. Personal Relationships} \xref{http://www.tei-c.org/release/doc/tei-p5-doc/en/html/PH.html\#PHCO}{11.3.1.1. Core Elements for Transcriptional Work} \xref{http://www.tei-c.org/release/doc/tei-p5-doc/en/html/SA.html\#SAPTL}{16.1.1. Pointers and Links} \xref{http://www.tei-c.org/release/doc/tei-p5-doc/en/html/SA.html\#SASE}{16.3. Blocks, Segments, and Anchors} \xref{http://www.tei-c.org/release/doc/tei-p5-doc/en/html/TC.html\#TCAPLK}{12.2. Linking the Apparatus to the Text} \xref{http://www.tei-c.org/release/doc/tei-p5-doc/en/html/TD.html\#TDTAGCONT}{22.5.2. RELAX NG Content Models} \xref{http://www.tei-c.org/release/doc/tei-p5-doc/en/html/TS.html\#TSBA}{8.3. Elements Unique to Spoken Texts} \xref{http://www.tei-c.org/release/doc/tei-p5-doc/en/html/USE.html\#MDMDAL}{23.3.1.4. Modification of Attribute and Attribute Value Lists}]\end{specHead} 
    \item[{Module}]
  tei
    \item[{Members}]
  att.pointing.group[altGrp joinGrp linkGrp] TEI ab accMat add addName addSpan alt altIdentifier am anchor application bibl biblStruct bloc c cb change charProp cit cl climate collection corr correspDesc country custEvent damage damageSpan date decoNote del delSpan desc dim district div event explicit figure filiation finalRubric floatingText forename g gb genName geogFeat geogName gloss group head incipit join label lb lg line link listApp listBibl listChange listEvent listNym listOrg listPerson listPlace listRelation location m mapping measureGrp media milestone mod msDesc msFrag msName msPart name nameLink notatedMusic note nym offset org orgName origDate origPlace pb pc persName phr place placeName population provenance ptr quote ref reg region relatedItem relation restore roleName rs rubric s schemaRef seal seg settlement space stamp state surface surfaceGrp surname table teiCorpus term terrain text time trailer trait w xenoData zone
    \item[{Attributes}]
  Attributes\hfil\\[-10pt]\begin{sansreflist}
    \item[@type]
  characterizes the element in some sense, using any convenient classification scheme or typology.
\begin{reflist}
    \item[{Status}]
  Optional
    \item[{Datatype}]
  teidata.enumerated
    \item[]\exampleFont {<\textbf{div}\hspace*{6pt}{type}="{verse}">}\mbox{}\newline 
\hspace*{6pt}{<\textbf{head}>}Night in Tarras{</\textbf{head}>}\mbox{}\newline 
\hspace*{6pt}{<\textbf{lg}\hspace*{6pt}{type}="{stanza}">}\mbox{}\newline 
\hspace*{6pt}\hspace*{6pt}{<\textbf{l}>}At evening tramping on the hot white road{</\textbf{l}>}\mbox{}\newline 
\hspace*{6pt}\hspace*{6pt}{<\textbf{l}>}…{</\textbf{l}>}\mbox{}\newline 
\hspace*{6pt}{</\textbf{lg}>}\mbox{}\newline 
\hspace*{6pt}{<\textbf{lg}\hspace*{6pt}{type}="{stanza}">}\mbox{}\newline 
\hspace*{6pt}\hspace*{6pt}{<\textbf{l}>}A wind sprang up from nowhere as the sky{</\textbf{l}>}\mbox{}\newline 
\hspace*{6pt}\hspace*{6pt}{<\textbf{l}>}…{</\textbf{l}>}\mbox{}\newline 
\hspace*{6pt}{</\textbf{lg}>}\mbox{}\newline 
{</\textbf{div}>}
    \item[{Note}]
  \par
The {\itshape type} attribute is present on a number of elements, not all of which are members of \textsf{att.typed}, usually because these elements restrict the possible values for the attribute in a specific way.
\end{reflist}  
    \item[@subtype]
  provides a sub-categorization of the element, if needed
\begin{reflist}
    \item[{Status}]
  Optional
    \item[{Datatype}]
  teidata.enumerated
    \item[{Note}]
  \par
The {\itshape subtype} attribute may be used to provide any sub-classification for the element additional to that provided by its {\itshape type} attribute.
\end{reflist}  
\end{sansreflist}  
    \item[{Schematron}]
   <sch:rule context="tei:*[@subtype]"> <sch:assert test="@type">The <sch:name/> element should not be categorized in detail with @subtype unless also categorized in general with @type</sch:assert> </sch:rule>
    \item[{Note}]
  \par
When appropriate, values from an established typology should be used. Alternatively a typology may be defined in the associated TEI header. If values are to be taken from a project-specific list, this should be defined using the \texttt{<valList>} element in the project-specific schema description, as described in \xref{http://www.tei-c.org/release/doc/tei-p5-doc/en/html/USE.html\#MDMDAL}{23.3.1.4. Modification of Attribute and Attribute Value Lists} .
\end{reflist}  
\begin{reflist}
\item[]\begin{specHead}{TEI.att.witnessed}{att.witnessed}\index{att.witnessed (attribute class)|oddindex}\index{wit=@wit!att.witnessed (attribute class)|oddindex} supplies the attribute used to identify the witnesses supporting a particular reading in a critical apparatus. [\xref{http://www.tei-c.org/release/doc/tei-p5-doc/en/html/TC.html\#TCAPLL}{12.1. The Apparatus Entry, Readings, and Witnesses}]\end{specHead} 
    \item[{Module}]
  textcrit
    \item[{Members}]
  lem rdg
    \item[{Attributes}]
  Attributes\hfil\\[-10pt]\begin{sansreflist}
    \item[@wit]
  (witness or witnesses) contains a space-delimited list of one or more pointers indicating the witnesses which attest to a given reading.
\begin{reflist}
    \item[{Status}]
  Optional
    \item[{Datatype}]
  1–∞ occurrences of teidata.pointer separated by whitespace
    \item[{Note}]
  \par
If the apparatus contains readings only for a single witness, this attribute may be consistently omitted.\par
This attribute may occur both within an apparatus gathering variant readings in the transcription of an individual witness and within an apparatus gathering readings from different witnesses.\par
Additional descriptions or alternative versions of the sigla referenced may be supplied as the content of a child <wit> element.
\end{reflist}  
\end{sansreflist}  
\end{reflist}  
\begin{reflist}
\item[]\begin{specHead}{TEI.att.written}{att.written}\index{att.written (attribute class)|oddindex}\index{hand=@hand!att.written (attribute class)|oddindex} provides an attribute to indicate the hand in which the textual content of an element was written in the source being transcribed. [\xref{http://www.tei-c.org/release/doc/tei-p5-doc/en/html/ST.html\#STECAT}{1.3.1. Attribute Classes}]\end{specHead} 
    \item[{Module}]
  tei
    \item[{Members}]
  att.damaged[damage damageSpan] att.textCritical[lem rdg rdgGrp] att.transcriptional[add addSpan del delSpan mod redo restore retrace subst substJoin undo] ab closer div fw head hi label line note opener p salute seg signed text zone
    \item[{Attributes}]
  Attributes\hfil\\[-10pt]\begin{sansreflist}
    \item[@hand]
  points to a <handNote> element describing the hand considered responsible for the textual content of the element concerned.
\begin{reflist}
    \item[{Status}]
  Optional
    \item[{Datatype}]
  teidata.pointer
\end{reflist}  
\end{sansreflist}  
\end{reflist}  
\section[{Macros}]{Macros}\index{macro.limitedContent (macro)|oddindex}
\begin{reflist}
\item[]\begin{specHead}{TEI.macro.limitedContent}{macro.limitedContent} (paragraph content) defines the content of prose elements that are not used for transcription of extant materials. [\xref{http://www.tei-c.org/release/doc/tei-p5-doc/en/html/ST.html\#STEC}{1.3. The TEI Class System}]\end{specHead} 
    \item[{Module}]
  tei
    \item[{Used by}]
  desc figDesc meeting rendition tagUsage witness
    \item[{Content model}]
  \mbox{}\hfill\\[-10pt]\begin{Verbatim}[fontsize=\small]
<content>
 <alternate maxOccurs="unbounded"
  minOccurs="0">
  <textNode/>
  <classRef key="model.limitedPhrase"/>
  <classRef key="model.inter"/>
 </alternate>
</content>
    
\end{Verbatim}

    \item[{Declaration}]
  \mbox{}\hfill\\[-10pt]\begin{Verbatim}[fontsize=\small]
macro.limitedContent = ( text | model.limitedPhrase | model.inter )*
\end{Verbatim}

\end{reflist}  \index{macro.paraContent (macro)|oddindex}
\begin{reflist}
\item[]\begin{specHead}{TEI.macro.paraContent}{macro.paraContent} (paragraph content) defines the content of paragraphs and similar elements. [\xref{http://www.tei-c.org/release/doc/tei-p5-doc/en/html/ST.html\#STEC}{1.3. The TEI Class System}]\end{specHead} 
    \item[{Module}]
  tei
    \item[{Used by}]
  ab add corr damage del docEdition emph hi imprimatur mod orig p ref reg restore retrace salute secl seg sic signed supplied surplus title titlePart unclear
    \item[{Content model}]
  \mbox{}\hfill\\[-10pt]\begin{Verbatim}[fontsize=\small]
<content>
 <alternate maxOccurs="unbounded"
  minOccurs="0">
  <textNode/>
  <classRef key="model.gLike"/>
  <classRef key="model.phrase"/>
  <classRef key="model.inter"/>
  <classRef key="model.global"/>
  <elementRef key="lg"/>
  <classRef key="model.lLike"/>
 </alternate>
</content>
    
\end{Verbatim}

    \item[{Declaration}]
  \mbox{}\hfill\\[-10pt]\begin{Verbatim}[fontsize=\small]
macro.paraContent =
   (
      text
    | model.gLike    | model.phrase    | model.inter    | model.global    | lg    | model.lLike   )*
\end{Verbatim}

\end{reflist}  \index{macro.phraseSeq (macro)|oddindex}
\begin{reflist}
\item[]\begin{specHead}{TEI.macro.phraseSeq}{macro.phraseSeq} (phrase sequence) defines a sequence of character data and phrase-level elements. [\xref{http://www.tei-c.org/release/doc/tei-p5-doc/en/html/ST.html\#STECST}{1.4.1. Standard Content Models}]\end{specHead} 
    \item[{Module}]
  tei
    \item[{Used by}]
  abbr addName addrLine affiliation author biblScope birth bloc catchwords citedRange cl colophon country death distinct distributor district docAuthor docDate edition editor education email expan explicit extent faith finalRubric floruit foreign forename fw genName geoDecl geogFeat geogName gloss headItem headLabel heraldry incipit label material measure mentioned name nameLink nationality num objectType offset orgName origPlace persName phr placeName pubPlace publisher region residence roleName rs rubric s secFol settlement sex soCalled socecStatus speaker stamp street surname term textLang watermark wit witDetail
    \item[{Content model}]
  \mbox{}\hfill\\[-10pt]\begin{Verbatim}[fontsize=\small]
<content>
 <alternate maxOccurs="unbounded"
  minOccurs="0">
  <textNode/>
  <classRef key="model.gLike"/>
  <classRef key="model.phrase"/>
  <classRef key="model.global"/>
 </alternate>
</content>
    
\end{Verbatim}

    \item[{Declaration}]
  \mbox{}\hfill\\[-10pt]\begin{Verbatim}[fontsize=\small]
macro.phraseSeq = ( text | model.gLike | model.phrase | model.global )*
\end{Verbatim}

\end{reflist}  \index{macro.phraseSeq.limited (macro)|oddindex}
\begin{reflist}
\item[]\begin{specHead}{TEI.macro.phraseSeq.limited}{macro.phraseSeq.limited} (limited phrase sequence) defines a sequence of character data and those phrase-level elements that are not typically used for transcribing extant documents. [\xref{http://www.tei-c.org/release/doc/tei-p5-doc/en/html/ST.html\#STECST}{1.4.1. Standard Content Models}]\end{specHead} 
    \item[{Module}]
  tei
    \item[{Used by}]
  age authority classCode funder langKnown language principal resp span sponsor
    \item[{Content model}]
  \mbox{}\hfill\\[-10pt]\begin{Verbatim}[fontsize=\small]
<content>
 <alternate maxOccurs="unbounded"
  minOccurs="0">
  <textNode/>
  <classRef key="model.limitedPhrase"/>
  <classRef key="model.global"/>
 </alternate>
</content>
    
\end{Verbatim}

    \item[{Declaration}]
  \mbox{}\hfill\\[-10pt]\begin{Verbatim}[fontsize=\small]
macro.phraseSeq.limited = ( text | model.limitedPhrase | model.global )*
\end{Verbatim}

\end{reflist}  \index{macro.specialPara (macro)|oddindex}
\begin{reflist}
\item[]\begin{specHead}{TEI.macro.specialPara}{macro.specialPara} ('special' paragraph content) defines the content model of elements such as notes or list items, which either contain a series of component-level elements or else have the same structure as a paragraph, containing a series of phrase-level and inter-level elements. [\xref{http://www.tei-c.org/release/doc/tei-p5-doc/en/html/ST.html\#STEC}{1.3. The TEI Class System}]\end{specHead} 
    \item[{Module}]
  tei
    \item[{Used by}]
  accMat acquisition additions cell change collation condition custEvent decoNote filiation foliation handNote item layout licence metamark musicNotation note occupation origin provenance q quote said scriptNote signatures source stage summary support surrogates typeNote
    \item[{Content model}]
  \mbox{}\hfill\\[-10pt]\begin{Verbatim}[fontsize=\small]
<content>
 <alternate maxOccurs="unbounded"
  minOccurs="0">
  <textNode/>
  <classRef key="model.gLike"/>
  <classRef key="model.phrase"/>
  <classRef key="model.inter"/>
  <classRef key="model.divPart"/>
  <classRef key="model.global"/>
 </alternate>
</content>
    
\end{Verbatim}

    \item[{Declaration}]
  \mbox{}\hfill\\[-10pt]\begin{Verbatim}[fontsize=\small]
macro.specialPara =
   (
      text
    | model.gLike    | model.phrase    | model.inter    | model.divPart    | model.global   )*
\end{Verbatim}

\end{reflist}  \index{macro.xtext (macro)|oddindex}
\begin{reflist}
\item[]\begin{specHead}{TEI.macro.xtext}{macro.xtext} (extended text) defines a sequence of character data and gaiji elements.\end{specHead} 
    \item[{Module}]
  tei
    \item[{Used by}]
  c collection depth dim ex height institution locus mapping msName repository value width
    \item[{Content model}]
  \mbox{}\hfill\\[-10pt]\begin{Verbatim}[fontsize=\small]
<content>
 <alternate maxOccurs="unbounded"
  minOccurs="0">
  <textNode/>
  <classRef key="model.gLike"/>
 </alternate>
</content>
    
\end{Verbatim}

    \item[{Declaration}]
  \fbox{\ttfamily macro.xtext = ( text | model.gLike )*} 
\end{reflist}  
\section[{Datatypes}]{Datatypes}
\begin{reflist}
\item[]\begin{specHead}{TEI.teidata.certainty}{teidata.certainty} defines the range of attribute values expressing a degree of certainty.\end{specHead} 
    \item[{Module}]
  tei
    \item[{Used by}]
  teidata.probCert
    \item[{Content model}]
  \mbox{}\hfill\\[-10pt]\begin{Verbatim}[fontsize=\small]
<content>
 <valList type="closed">
  <valItem ident="high"/>
  <valItem ident="medium"/>
  <valItem ident="low"/>
  <valItem ident="unknown"/>
 </valList>
</content>
    
\end{Verbatim}

    \item[{Declaration}]
  \fbox{\ttfamily teidata.certainty = "high" | "medium" | "low" | "unknown"} 
    \item[{Note}]
  \par
Certainty may be expressed by one of the predefined symbolic values high, medium, or low. The value unknown should be used in cases where the encoder does not wish to assert an opinion about the matter.
\end{reflist}  
\begin{reflist}
\item[]\begin{specHead}{TEI.teidata.count}{teidata.count} defines the range of attribute values used for a non-negative integer value used as a count.\end{specHead} 
    \item[{Module}]
  tei
    \item[{Used by}]
  Element: \begin{itemize}
\item age/@value
\item handDesc/@hands
\item layout/@columns
\item layout/@ruledLines
\item layout/@writtenLines
\item refState/@length
\item table/@rows
\item table/@cols
\item tagUsage/@occurs
\item tagUsage/@withId
\item zone/@rotate
\end{itemize} 
    \item[{Content model}]
  \mbox{}\hfill\\[-10pt]\begin{Verbatim}[fontsize=\small]
<content>
 <dataRef name="nonNegativeInteger"/>
</content>
    
\end{Verbatim}

    \item[{Declaration}]
  \fbox{\ttfamily teidata.count = xsd:nonNegativeInteger} 
    \item[{Note}]
  \par
Any positive integer value or zero is permitted
\end{reflist}  
\begin{reflist}
\item[]\begin{specHead}{TEI.teidata.duration.iso}{teidata.duration.iso} defines the range of attribute values available for representation of a duration in time using ISO 8601 standard formats\end{specHead} 
    \item[{Module}]
  tei
    \item[{Used by}]
  
    \item[{Content model}]
  \mbox{}\hfill\\[-10pt]\begin{Verbatim}[fontsize=\small]
<content>
 <dataRef name="token"
  restriction="[0-9.,DHMPRSTWYZ/:+\-]+"/>
</content>
    
\end{Verbatim}

    \item[{Declaration}]
  \mbox{}\hfill\\[-10pt]\begin{Verbatim}[fontsize=\small]
teidata.duration.iso = token { pattern = "[0-9.,DHMPRSTWYZ/:+\-]+" }
\end{Verbatim}

    \item[{Example}]
  \leavevmode\bgroup\exampleFont \begin{shaded}\noindent\mbox{}{<\textbf{time}\hspace*{6pt}{dur-iso}="{PT0,75H}">}three-quarters of an hour{</\textbf{time}>}\end{shaded}\egroup 


    \item[{Example}]
  \leavevmode\bgroup\exampleFont \begin{shaded}\noindent\mbox{}{<\textbf{date}\hspace*{6pt}{dur-iso}="{P1,5D}">}a day and a half{</\textbf{date}>}\end{shaded}\egroup 


    \item[{Example}]
  \leavevmode\bgroup\exampleFont \begin{shaded}\noindent\mbox{}{<\textbf{date}\hspace*{6pt}{dur-iso}="{P14D}">}a fortnight{</\textbf{date}>}\end{shaded}\egroup 


    \item[{Example}]
  \leavevmode\bgroup\exampleFont \begin{shaded}\noindent\mbox{}{<\textbf{time}\hspace*{6pt}{dur-iso}="{PT0.02S}">}20 ms{</\textbf{time}>}\end{shaded}\egroup 


    \item[{Note}]
  \par
A duration is expressed as a sequence of number-letter pairs, preceded by the letter P; the letter gives the unit and may be Y (year), M (month), D (day), H (hour), M (minute), or S (second), in that order. The numbers are all unsigned integers, except for the last, which may have a decimal component (using either \texttt{.} or \texttt{,} as the decimal point; the latter is preferred). If any number is \textit{0}, then that number-letter pair may be omitted. If any of the H (hour), M (minute), or S (second) number-letter pairs are present, then the separator \texttt{T} must precede the first ‘time’ number-letter pair.\par
For complete details, see ISO 8601 \textit{Data elements and interchange formats — Information interchange — Representation of dates and times}.
\end{reflist}  
\begin{reflist}
\item[]\begin{specHead}{TEI.teidata.duration.w3c}{teidata.duration.w3c} defines the range of attribute values available for representation of a duration in time using W3C datatypes.\end{specHead} 
    \item[{Module}]
  tei
    \item[{Used by}]
  
    \item[{Content model}]
  \fbox{\ttfamily <content>\newline
 <dataRef name="duration"/>\newline
</content>\newline
    } 
    \item[{Declaration}]
  \fbox{\ttfamily teidata.duration.w3c = xsd:duration} 
    \item[{Example}]
  \leavevmode\bgroup\exampleFont \begin{shaded}\noindent\mbox{}{<\textbf{time}\hspace*{6pt}{dur}="{PT45M}">}forty-five minutes{</\textbf{time}>}\end{shaded}\egroup 


    \item[{Example}]
  \leavevmode\bgroup\exampleFont \begin{shaded}\noindent\mbox{}{<\textbf{date}\hspace*{6pt}{dur}="{P1DT12H}">}a day and a half{</\textbf{date}>}\end{shaded}\egroup 


    \item[{Example}]
  \leavevmode\bgroup\exampleFont \begin{shaded}\noindent\mbox{}{<\textbf{date}\hspace*{6pt}{dur}="{P7D}">}a week{</\textbf{date}>}\end{shaded}\egroup 


    \item[{Example}]
  \leavevmode\bgroup\exampleFont \begin{shaded}\noindent\mbox{}{<\textbf{time}\hspace*{6pt}{dur}="{PT0.02S}">}20 ms{</\textbf{time}>}\end{shaded}\egroup 


    \item[{Note}]
  \par
A duration is expressed as a sequence of number-letter pairs, preceded by the letter P; the letter gives the unit and may be Y (year), M (month), D (day), H (hour), M (minute), or S (second), in that order. The numbers are all unsigned integers, except for the \texttt{S} number, which may have a decimal component (using \texttt{.} as the decimal point). If any number is \textit{0}, then that number-letter pair may be omitted. If any of the H (hour), M (minute), or S (second) number-letter pairs are present, then the separator \texttt{T} must precede the first ‘time’ number-letter pair.\par
For complete details, see the \xref{http://www.w3.org/TR/2004/REC-xmlschema-2-20041028/\#duration}{W3C specification}.
\end{reflist}  
\begin{reflist}
\item[]\begin{specHead}{TEI.teidata.enumerated}{teidata.enumerated} defines the range of attribute values expressed as a single XML name taken from a list of documented possibilities.\end{specHead} 
    \item[{Module}]
  tei
    \item[{Used by}]
  Element: \begin{itemize}
\item abbr/@type
\item alt/@mode
\item altGrp/@mode
\item app/@type
\item availability/@status
\item correction/@status
\item correction/@method
\item correspAction/@type
\item dimensions/@type
\item distinct/@type
\item divGen/@type
\item fw/@type
\item gap/@agent
\item geoDecl/@datum
\item hyphenation/@eol
\item idno/@type
\item join/@scope
\item list/@type
\item measure/@type
\item normalization/@method
\item num/@type
\item objectDesc/@form
\item pc/@force
\item pc/@unit
\item person/@role
\item person/@age
\item personGrp/@role
\item personGrp/@age
\item punctuation/@marks
\item punctuation/@placement
\item q/@type
\item quotation/@marks
\item relation/@name
\item rendition/@scope
\item space/@dim
\item stage/@type
\item supportDesc/@material
\item surface/@attachment
\item timeline/@unit
\item title/@type
\item title/@level
\item titlePage/@type
\item titlePart/@type
\item unclear/@agent
\item variantEncoding/@method
\item variantEncoding/@location
\item when/@unit
\item witDetail/@type
\end{itemize} 
    \item[{Content model}]
  \fbox{\ttfamily <content>\newline
 <dataRef key="teidata.word"/>\newline
</content>\newline
    } 
    \item[{Declaration}]
  \fbox{\ttfamily teidata.enumerated = teidata.word} 
    \item[{Note}]
  \par
Attributes using this datatype must contain a single ‘word’ which contains only letters, digits, punctuation characters, or symbols: thus it cannot include whitespace.\par
Typically, the list of documented possibilities will be provided (or exemplified) by a value list in the associated attribute specification, expressed with a \texttt{<valList>} element.
\end{reflist}  
\begin{reflist}
\item[]\begin{specHead}{TEI.teidata.interval}{teidata.interval} defines attribute values used to express an interval value.\end{specHead} 
    \item[{Module}]
  tei
    \item[{Used by}]
  Element: \begin{itemize}
\item timeline/@interval
\item when/@interval
\end{itemize} 
    \item[{Content model}]
  \mbox{}\hfill\\[-10pt]\begin{Verbatim}[fontsize=\small]
<content>
 <alternate>
  <dataRef name="float"/>
  <valList>
   <valItem ident="regular"/>
   <valItem ident="irregular"/>
   <valItem ident="unknown"/>
  </valList>
 </alternate>
</content>
    
\end{Verbatim}

    \item[{Declaration}]
  \mbox{}\hfill\\[-10pt]\begin{Verbatim}[fontsize=\small]
teidata.interval = xsd:float | ( "regular" | "irregular" | "unknown" )
\end{Verbatim}

    \item[{Note}]
  \par
Any value greater than zero or any one of the values regular, irregular, unknown.
\end{reflist}  
\begin{reflist}
\item[]\begin{specHead}{TEI.teidata.language}{teidata.language} defines the range of attribute values used to identify a particular combination of human language and writing system. [\xref{http://www.tei-c.org/release/doc/tei-p5-doc/en/html/CH.html\#CHSH}{6.1. Language Identification}]\end{specHead} 
    \item[{Module}]
  tei
    \item[{Used by}]
  Element: \begin{itemize}
\item langKnowledge/@tags
\item langKnown/@tag
\item language/@ident
\item textLang/@mainLang
\item textLang/@otherLangs
\end{itemize} 
    \item[{Content model}]
  \mbox{}\hfill\\[-10pt]\begin{Verbatim}[fontsize=\small]
<content>
 <alternate>
  <dataRef name="language"/>
  <valList>
   <valItem ident=""/>
  </valList>
 </alternate>
</content>
    
\end{Verbatim}

    \item[{Declaration}]
  \fbox{\ttfamily teidata.language = xsd:language | ( "" )} 
    \item[{Note}]
  \par
The values for this attribute are language ‘tags’ as defined in \xref{https://tools.ietf.org/html/bcp47}{BCP 47}. Currently BCP 47 comprises RFC 5646 and RFC 4647; over time, other IETF documents may succeed these as the best current practice.\par
A ‘language tag’, per BCP 47, is assembled from a sequence of components or \textit{subtags} separated by the hyphen character (\textit{-}, U+002D). The tag is made of the following subtags, in the following order. Every subtag except the first is optional. If present, each occurs only once, except the fourth and fifth components (variant and extension), which are repeatable. \begin{description}

\item[{language}]The IANA-registered code for the language. This is almost always the same as the ISO 639 2-letter language code if there is one. The list of available registered language subtags can be found at \url{http://www.iana.org/assignments/language-subtag-registry}. It is recommended that this code be written in lower case.
\item[{script}]The ISO 15924 code for the script. These codes consist of 4 letters, and it is recommended they be written with an initial capital, the other three letters in lower case. The canonical list of codes is maintained by the Unicode Consortium, and is available at \url{http://unicode.org/iso15924/iso15924-codes.html}. The IETF recommends this code be omitted unless it is necessary to make a distinction you need.
\item[{region}]Either an ISO 3166 country code or a UN M.49 region code that is registered with IANA (not all such codes are registered, e.g. UN codes for economic groupings or codes for countries for which there is already an ISO 3166 2-letter code are not registered). The former consist of 2 letters, and it is recommended they be written in upper case; the list of codes can be searched or browsed at \url{https://www.iso.org/obp/ui/\#search/code/}. The latter consist of 3 digits; the list of codes can be found at \url{http://unstats.un.org/unsd/methods/m49/m49.htm}.
\item[{variant}]An IANA-registered variation. These codes ‘are used to indicate additional, well-recognized variations that define a language or its dialects that are not covered by other available subtags’.
\item[{extension}]An extension has the format of a single letter followed by a hyphen followed by additional subtags. These exist to allow for future extension to BCP 47, but as of this writing no such extensions are in use.
\item[{private use}]An extension that uses the initial subtag of the single letter \textit{x} (i.e., starts with \texttt{x-}) has no meaning except as negotiated among the parties involved. These should be used with great care, since they interfere with the interoperability that use of RFC 4646 is intended to promote. In order for a document that makes use of these subtags to be TEI-conformant, a corresponding <language> element must be present in the TEI header.
\end{description} \par
There are two exceptions to the above format. First, there are language tags in the \xref{http://www.iana.org/assignments/language-subtag-registry}{IANA registry} that do not match the above syntax, but are present because they have been ‘grandfathered’ from previous specifications.\par
Second, an entire language tag can consist of only a private use subtag. These tags start with \texttt{x-}, and do not need to follow any further rules established by the IETF and endorsed by these Guidelines. Like all language tags that make use of private use subtags, the language in question must be documented in a corresponding <language> element in the TEI header.\par
Examples include \begin{description}

\item[{sn}]Shona
\item[{zh-TW}]Taiwanese
\item[{zh-Hant-HK}]Chinese written in traditional script as used in Hong Kong
\item[{en-SL}]English as spoken in Sierra Leone
\item[{pl}]Polish
\item[{es-MX}]Spanish as spoken in Mexico
\item[{es-419}]Spanish as spoken in Latin America
\end{description} \par
The W3C Internationalization Activity has published a useful introduction to BCP 47, \xref{http://www.w3.org/International/articles/language-tags/Overview.en.php}{Language tags in HTML and XML}.
\end{reflist}  
\begin{reflist}
\item[]\begin{specHead}{TEI.teidata.name}{teidata.name} defines the range of attribute values expressed as an XML Name.\end{specHead} 
    \item[{Module}]
  tei
    \item[{Used by}]
  Element: \begin{itemize}
\item application/@ident
\item index/@indexName
\item join/@result
\item joinGrp/@result
\item tagUsage/@gi
\end{itemize} 
    \item[{Content model}]
  \fbox{\ttfamily <content>\newline
 <dataRef name="Name"/>\newline
</content>\newline
    } 
    \item[{Declaration}]
  \fbox{\ttfamily teidata.name = xsd:Name} 
    \item[{Note}]
  \par
Attributes using this datatype must contain a single word which follows the rules defining a legal XML name (see \url{http://www.w3.org/TR/REC-xml/\#dt-name}): for example they cannot include whitespace or begin with digits.
\end{reflist}  
\begin{reflist}
\item[]\begin{specHead}{TEI.teidata.namespace}{teidata.namespace} defines the range of attribute values used to indicate XML namespaces as defined by the W3C \xref{http://www.w3.org/TR/1999/REC-xml-names-19990114/}{Namespaces in XML} Technical Recommendation.\end{specHead} 
    \item[{Module}]
  tei
    \item[{Used by}]
  Element: \begin{itemize}
\item namespace/@name
\end{itemize} 
    \item[{Content model}]
  \fbox{\ttfamily <content>\newline
 <dataRef name="anyURI"/>\newline
</content>\newline
    } 
    \item[{Declaration}]
  \fbox{\ttfamily teidata.namespace = xsd:anyURI} 
    \item[{Note}]
  \par
The range of syntactically valid values is defined by \xref{http://www.ietf.org/rfc/rfc3986.txt}{RFC 3986 \textit{Uniform Resource Identifier (URI): Generic Syntax}}
\end{reflist}  
\begin{reflist}
\item[]\begin{specHead}{TEI.teidata.numeric}{teidata.numeric} defines the range of attribute values used for numeric values.\end{specHead} 
    \item[{Module}]
  tei
    \item[{Used by}]
  Element: \begin{itemize}
\item num/@value
\end{itemize} 
    \item[{Content model}]
  \mbox{}\hfill\\[-10pt]\begin{Verbatim}[fontsize=\small]
<content>
 <alternate>
  <dataRef name="double"/>
  <dataRef name="token"
   restriction="(\-?[\d]+/\-?[\d]+)"/>
  <dataRef name="decimal"/>
 </alternate>
</content>
    
\end{Verbatim}

    \item[{Declaration}]
  \mbox{}\hfill\\[-10pt]\begin{Verbatim}[fontsize=\small]
teidata.numeric =
   xsd:double | token { pattern = "(\-?[\d]+/\-?[\d]+)" } | xsd:decimal
\end{Verbatim}

    \item[{Note}]
  \par
Any numeric value, represented as a decimal number, in floating point format, or as a ratio.\par
To represent a floating point number, expressed in scientific notation, ‘E notation’, a variant of ‘exponential notation’, may be used. In this format, the value is expressed as two numbers separated by the letter E. The first number, the significand (sometimes called the mantissa) is given in decimal format, while the second is an integer. The value is obtained by multiplying the mantissa by 10 the number of times indicated by the integer. Thus the value represented in decimal notation as 1000.0 might be represented in scientific notation as 10E3.\par
A value expressed as a ratio is represented by two integer values separated by a solidus (/) character. Thus, the value represented in decimal notation as 0.5 might be represented as a ratio by the string 1/2.
\end{reflist}  
\begin{reflist}
\item[]\begin{specHead}{TEI.teidata.outputMeasurement}{teidata.outputMeasurement} defines a range of values for use in specifying the size of an object that is intended for display.\end{specHead} 
    \item[{Module}]
  tei
    \item[{Used by}]
  
    \item[{Content model}]
  \mbox{}\hfill\\[-10pt]\begin{Verbatim}[fontsize=\small]
<content>
 <dataRef name="token"
  restriction="[\-+]?\d+(\.\d+)?(%|cm|mm|in|pt|pc|px|em|ex|gd|rem|vw|vh|vm)"/>
</content>
    
\end{Verbatim}

    \item[{Declaration}]
  \mbox{}\hfill\\[-10pt]\begin{Verbatim}[fontsize=\small]
teidata.outputMeasurement =
   token
   {
      pattern = "[\-+]?\d+(\.\d+)?(%|cm|mm|in|pt|pc|px|em|ex|gd|rem|vw|vh|vm)"
   }
\end{Verbatim}

    \item[{Example}]
  \leavevmode\bgroup\exampleFont \begin{shaded}\noindent\mbox{}{<\textbf{figure}>}\mbox{}\newline 
\hspace*{6pt}{<\textbf{head}>}The TEI Logo{</\textbf{head}>}\mbox{}\newline 
\hspace*{6pt}{<\textbf{figDesc}>}Stylized yellow angle brackets with the letters {<\textbf{mentioned}>}TEI{</\textbf{mentioned}>} in\mbox{}\newline 
\hspace*{6pt}\hspace*{6pt} between and {<\textbf{mentioned}>}text encoding initiative{</\textbf{mentioned}>} underneath, all on a white\mbox{}\newline 
\hspace*{6pt}\hspace*{6pt} background.{</\textbf{figDesc}>}\mbox{}\newline 
\hspace*{6pt}{<\textbf{graphic}\hspace*{6pt}{height}="{600px}"\mbox{}\newline 
\hspace*{6pt}\hspace*{6pt}{url}="{http://www.tei-c.org/logos/TEI-600.jpg}"\hspace*{6pt}{width}="{600px}"/>}\mbox{}\newline 
{</\textbf{figure}>}\end{shaded}\egroup 


    \item[{Note}]
  \par
These values map directly onto the values used by XSL-FO and CSS. For definitions of the units see those specifications; at the time of this writing the most complete list is in the \xref{http://www.w3.org/TR/2005/WD-css3-values-20050726/\#numbers0}{CSS3 working draft}.
\end{reflist}  
\begin{reflist}
\item[]\begin{specHead}{TEI.teidata.pattern}{teidata.pattern} defines attribute values which are expressed as a regular expression.\end{specHead} 
    \item[{Module}]
  tei
    \item[{Used by}]
  
    \item[{Content model}]
  \fbox{\ttfamily <content>\newline
 <dataRef name="token"/>\newline
</content>\newline
    } 
    \item[{Declaration}]
  \fbox{\ttfamily teidata.pattern = token} 
    \item[{Note}]
  \par

\begin{quote}
 A regular expression, often called a \textit{pattern}, is an expression that describes a set of strings. They are usually used to give a concise description of a set, without having to list all elements. For example, the set containing the three strings \textit{Handel}, \textit{Händel}, and \textit{Haendel} can be described by the pattern \texttt{H(ä|ae?)ndel} (or alternatively, it is said that the pattern \texttt{H(ä|ae?)ndel} \textit{matches} each of the three strings)\xref{http://en.wikipedia.org/wiki/Regular_expression\#Basic_concepts}{Wikipedia}\par

\end{quote}
\par
This TEI datatype is mapped to the XSD token datatype, and may therefore contain any string of characters. However, it is recommended that the value used conform to the particular flavour of regular expression syntax supported by XSD Schema. 
\end{reflist}  
\begin{reflist}
\item[]\begin{specHead}{TEI.teidata.point}{teidata.point} defines the data type used to express a point in cartesian space.\end{specHead} 
    \item[{Module}]
  tei
    \item[{Used by}]
  
    \item[{Content model}]
  \mbox{}\hfill\\[-10pt]\begin{Verbatim}[fontsize=\small]
<content>
 <dataRef name="token"
  restriction="(\-?[0-9]+\.?[0-9]*,\-?[0-9]+\.?[0-9]*)"/>
</content>
    
\end{Verbatim}

    \item[{Declaration}]
  \mbox{}\hfill\\[-10pt]\begin{Verbatim}[fontsize=\small]
teidata.point = token { pattern = "(\-?[0-9]+\.?[0-9]*,\-?[0-9]+\.?[0-9]*)" }
\end{Verbatim}

    \item[{Example}]
  \leavevmode\bgroup\exampleFont \begin{shaded}\noindent\mbox{}{<\textbf{facsimile}>}\mbox{}\newline 
\hspace*{6pt}{<\textbf{surface}\hspace*{6pt}{lrx}="{400}"\hspace*{6pt}{lry}="{280}"\hspace*{6pt}{ulx}="{0}"\hspace*{6pt}{uly}="{0}">}\mbox{}\newline 
\hspace*{6pt}\hspace*{6pt}{<\textbf{zone}\hspace*{6pt}{points}="{220,100 300,210 170,250 123,234}">}\mbox{}\newline 
\hspace*{6pt}\hspace*{6pt}\hspace*{6pt}{<\textbf{graphic}\hspace*{6pt}{url}="{handwriting.png }"/>}\mbox{}\newline 
\hspace*{6pt}\hspace*{6pt}{</\textbf{zone}>}\mbox{}\newline 
\hspace*{6pt}{</\textbf{surface}>}\mbox{}\newline 
{</\textbf{facsimile}>}\end{shaded}\egroup 


    \item[{Note}]
  \par
A point is defined by two numeric values, which may be expressed in any notation permitted.
\end{reflist}  
\begin{reflist}
\item[]\begin{specHead}{TEI.teidata.pointer}{teidata.pointer} defines the range of attribute values used to provide a single URI, absolute or relative, pointing to some other resource, either within the current document or elsewhere.\end{specHead} 
    \item[{Module}]
  tei
    \item[{Used by}]
  Element: \begin{itemize}
\item alt/@target
\item app/@from
\item app/@to
\item catRef/@scheme
\item change/@target
\item classCode/@scheme
\item event/@where
\item g/@ref
\item gap/@hand
\item handShift/@new
\item keywords/@scheme
\item locus/@scheme
\item locusGrp/@scheme
\item metamark/@target
\item msContents/@class
\item msItem/@class
\item msItemStruct/@class
\item note/@targetEnd
\item nym/@parts
\item occupation/@scheme
\item occupation/@code
\item redo/@target
\item relatedItem/@target
\item relation/@active
\item relation/@mutual
\item relation/@passive
\item socecStatus/@scheme
\item socecStatus/@code
\item space/@resp
\item span/@from
\item span/@to
\item tagUsage/@render
\item timeline/@origin
\item unclear/@hand
\item undo/@target
\item w/@lemmaRef
\item when/@since
\item witDetail/@wit
\end{itemize} 
    \item[{Content model}]
  \fbox{\ttfamily <content>\newline
 <dataRef name="anyURI"/>\newline
</content>\newline
    } 
    \item[{Declaration}]
  \fbox{\ttfamily teidata.pointer = xsd:anyURI} 
    \item[{Note}]
  \par
The range of syntactically valid values is defined by \xref{http://www.ietf.org/rfc/rfc3986.txt}{RFC 3986} \textit{Uniform Resource Identifier (URI): Generic Syntax}. Note that the values themselves are encoded using \xref{http://www.ietf.org/rfc/rfc3987.txt}{RFC 3987} \textit{Internationalized Resource Identifiers} (IRIs) mapping to URIs. For example, \texttt{https://secure.wikimedia.org/wikipedia/en/wiki/\%} is encoded as \texttt{https://secure.wikimedia.org/wikipedia/en/wiki/\%25} while \texttt{http://موقع.وزارة-الاتصالات.مصر/} is encoded as \texttt{http://xn--4gbrim.xn----rmckbbajlc6dj7bxne2c.xn--wgbh1c/}
\end{reflist}  
\begin{reflist}
\item[]\begin{specHead}{TEI.teidata.prefix}{teidata.prefix} defines a range of values that may function as a URI scheme name.\end{specHead} 
    \item[{Module}]
  tei
    \item[{Used by}]
  Element: \begin{itemize}
\item prefixDef/@ident
\end{itemize} 
    \item[{Content model}]
  \mbox{}\hfill\\[-10pt]\begin{Verbatim}[fontsize=\small]
<content>
 <dataRef name="token"
  restriction="[a-z][a-z0-9\+\.\-]*"/>
</content>
    
\end{Verbatim}

    \item[{Declaration}]
  \mbox{}\hfill\\[-10pt]\begin{Verbatim}[fontsize=\small]
teidata.prefix = token { pattern = "[a-z][a-z0-9\+\.\-]*" }
\end{Verbatim}

    \item[{Note}]
  \par
This datatype is used to constrain a string of characters to one that can be used as a URI scheme name according to \xref{http://www.ietf.org/rfc/rfc3986.txt}{RFC 3986}, \xref{https://tools.ietf.org/html/rfc3986\#section-3.1}{section 3.1}. Thus only the 26 lowercase letters a–z, the 10 digits 0–9, the plus sign, the period, and the hyphen are permitted, and the value must start with a letter.
\end{reflist}  
\begin{reflist}
\item[]\begin{specHead}{TEI.teidata.probCert}{teidata.probCert} defines a range of attribute values which can be expressed either as a numeric probability or as a coded certainty value.\end{specHead} 
    \item[{Module}]
  tei
    \item[{Used by}]
  
    \item[{Content model}]
  \mbox{}\hfill\\[-10pt]\begin{Verbatim}[fontsize=\small]
<content>
 <alternate>
  <dataRef key="teidata.probability"/>
  <dataRef key="teidata.certainty"/>
 </alternate>
</content>
    
\end{Verbatim}

    \item[{Declaration}]
  \mbox{}\hfill\\[-10pt]\begin{Verbatim}[fontsize=\small]
teidata.probCert = teidata.probability | teidata.certainty
\end{Verbatim}

\end{reflist}  
\begin{reflist}
\item[]\begin{specHead}{TEI.teidata.probability}{teidata.probability} defines the range of attribute values expressing a probability.\end{specHead} 
    \item[{Module}]
  tei
    \item[{Used by}]
  teidata.probCertElement: \begin{itemize}
\item alt/@weights
\end{itemize} 
    \item[{Content model}]
  \fbox{\ttfamily <content>\newline
 <dataRef name="double"/>\newline
</content>\newline
    } 
    \item[{Declaration}]
  \fbox{\ttfamily teidata.probability = xsd:double} 
    \item[{Note}]
  \par
Probability is expressed as a real number between 0 and 1; 0 representing \textit{certainly false} and 1 representing \textit{certainly true}.
\end{reflist}  
\begin{reflist}
\item[]\begin{specHead}{TEI.teidata.replacement}{teidata.replacement} defines attribute values which contain a replacement template.\end{specHead} 
    \item[{Module}]
  tei
    \item[{Used by}]
  
    \item[{Content model}]
  \fbox{\ttfamily <content>\newline
 <textNode/>\newline
</content>\newline
    } 
    \item[{Declaration}]
  \fbox{\ttfamily teidata.replacement = text} 
\end{reflist}  
\begin{reflist}
\item[]\begin{specHead}{TEI.teidata.sex}{teidata.sex} defines the range of attribute values used to identify human or animal sex.\end{specHead} 
    \item[{Module}]
  tei
    \item[{Used by}]
  Element: \begin{itemize}
\item person/@sex
\item personGrp/@sex
\item sex/@value
\end{itemize} 
    \item[{Content model}]
  \fbox{\ttfamily <content>\newline
 <dataRef key="teidata.word"/>\newline
</content>\newline
    } 
    \item[{Declaration}]
  \fbox{\ttfamily teidata.sex = teidata.word} 
    \item[{Note}]
  \par
Values for attributes using this datatype may be locally defined by a project, or may refer to an external standard, such as vCard's sex property \url{http://microformats.org/wiki/gender-formats} (in which M indicates male, F female, O other, N none or not applicable, U unknown), or the often used ISO 5218:2004 \textit{Representation of Human Sexes} \url{http://standards.iso.org/ittf/PubliclyAvailableStandards/c036266\textunderscore ISO\textunderscore IEC\textunderscore 5218\textunderscore 2004(E\textunderscore F).zip} (in which 0 indicates unknown; 1 male; 2 female; and 9 not applicable, although the ISO standard is widely considered inadequate); cf. CETH's \textit{Recommendations for Inclusive Data Collection of Trans People} \url{http://transhealth.ucsf.edu/trans?page=lib-data-collection}.
\end{reflist}  
\begin{reflist}
\item[]\begin{specHead}{TEI.teidata.temporal.iso}{teidata.temporal.iso} defines the range of attribute values expressing a temporal expression such as a date, a time, or a combination of them, that conform to the international standard \textit{Data elements and interchange formats – Information interchange – Representation of dates and times}.\end{specHead} 
    \item[{Module}]
  tei
    \item[{Used by}]
  
    \item[{Content model}]
  \mbox{}\hfill\\[-10pt]\begin{Verbatim}[fontsize=\small]
<content>
 <alternate>
  <dataRef name="date"/>
  <dataRef name="gYear"/>
  <dataRef name="gMonth"/>
  <dataRef name="gDay"/>
  <dataRef name="gYearMonth"/>
  <dataRef name="gMonthDay"/>
  <dataRef name="time"/>
  <dataRef name="dateTime"/>
  <dataRef name="token"
   restriction="[0-9.,DHMPRSTWYZ/:+\-]+"/>
 </alternate>
</content>
    
\end{Verbatim}

    \item[{Declaration}]
  \mbox{}\hfill\\[-10pt]\begin{Verbatim}[fontsize=\small]
teidata.temporal.iso =
   xsd:date
 | xsd:gYear
 | xsd:gMonth
 | xsd:gDay
 | xsd:gYearMonth
 | xsd:gMonthDay
 | xsd:time
 | xsd:dateTime
 | token { pattern = "[0-9.,DHMPRSTWYZ/:+\-]+" }
\end{Verbatim}

    \item[{Note}]
  \par
If it is likely that the value used is to be compared with another, then a time zone indicator should always be included, and only the dateTime representation should be used.\par
For all representations for which ISO 8601 describes both a \textit{basic} and an \textit{extended} format, these Guidelines recommend use of the extended format.\par
While ISO 8601 permits the use of both \texttt{00:00} and \texttt{24:00} to represent midnight, these Guidelines strongly recommend against the use of \texttt{24:00}.
\end{reflist}  
\begin{reflist}
\item[]\begin{specHead}{TEI.teidata.temporal.w3c}{teidata.temporal.w3c} defines the range of attribute values expressing a temporal expression such as a date, a time, or a combination of them, that conform to the W3C \textit{XML Schema Part 2: Datatypes Second Edition} specification.\end{specHead} 
    \item[{Module}]
  tei
    \item[{Used by}]
  Element: \begin{itemize}
\item docDate/@when
\item when/@absolute
\end{itemize} 
    \item[{Content model}]
  \mbox{}\hfill\\[-10pt]\begin{Verbatim}[fontsize=\small]
<content>
 <alternate>
  <dataRef name="date"/>
  <dataRef name="gYear"/>
  <dataRef name="gMonth"/>
  <dataRef name="gDay"/>
  <dataRef name="gYearMonth"/>
  <dataRef name="gMonthDay"/>
  <dataRef name="time"/>
  <dataRef name="dateTime"/>
 </alternate>
</content>
    
\end{Verbatim}

    \item[{Declaration}]
  \mbox{}\hfill\\[-10pt]\begin{Verbatim}[fontsize=\small]
teidata.temporal.w3c =
   xsd:date
 | xsd:gYear
 | xsd:gMonth
 | xsd:gDay
 | xsd:gYearMonth
 | xsd:gMonthDay
 | xsd:time
 | xsd:dateTime
\end{Verbatim}

    \item[{Note}]
  \par
If it is likely that the value used is to be compared with another, then a time zone indicator should always be included, and only the dateTime representation should be used.
\end{reflist}  
\begin{reflist}
\item[]\begin{specHead}{TEI.teidata.text}{teidata.text} defines the range of attribute values used to express some kind of identifying string as a single sequence of unicode characters possibly including whitespace.\end{specHead} 
    \item[{Module}]
  tei
    \item[{Used by}]
  Element: \begin{itemize}
\item distinct/@time
\item distinct/@space
\item distinct/@social
\item refState/@delim
\item rendition/@selector
\item w/@lemma
\end{itemize} 
    \item[{Content model}]
  \fbox{\ttfamily <content>\newline
 <dataRef name="string"/>\newline
</content>\newline
    } 
    \item[{Declaration}]
  \fbox{\ttfamily teidata.text = string} 
    \item[{Note}]
  \par
Attributes using this datatype must contain a single ‘token’ in which whitespace and other punctuation characters are permitted.
\end{reflist}  
\begin{reflist}
\item[]\begin{specHead}{TEI.teidata.truthValue}{teidata.truthValue} defines the range of attribute values used to express a truth value.\end{specHead} 
    \item[{Module}]
  tei
    \item[{Used by}]
  Element: \begin{itemize}
\item listChange/@ordered
\item note/@anchored
\item pc/@pre
\item surface/@flipping
\item tagsDecl/@partial
\end{itemize} 
    \item[{Content model}]
  \fbox{\ttfamily <content>\newline
 <dataRef name="boolean"/>\newline
</content>\newline
    } 
    \item[{Declaration}]
  \fbox{\ttfamily teidata.truthValue = xsd:boolean} 
    \item[{Note}]
  \par
The possible values of this datatype are 1 or true, or 0 or false.\par
This datatype applies only for cases where uncertainty is inappropriate; if the attribute concerned may have a value other than true or false, e.g. unknown, or inapplicable, it should have the extended version of this datatype: \textsf{data.xTruthValue}.
\end{reflist}  
\begin{reflist}
\item[]\begin{specHead}{TEI.teidata.version}{teidata.version} defines the range of attribute values which may be used to specify a TEI or Unicode version number.\end{specHead} 
    \item[{Module}]
  tei
    \item[{Used by}]
  Element: \begin{itemize}
\item TEI/@version
\item teiCorpus/@version
\item unicodeName/@version
\end{itemize} 
    \item[{Content model}]
  \mbox{}\hfill\\[-10pt]\begin{Verbatim}[fontsize=\small]
<content>
 <dataRef name="token"
  restriction="[\d]+(\.[\d]+){0,2}"/>
</content>
    
\end{Verbatim}

    \item[{Declaration}]
  \mbox{}\hfill\\[-10pt]\begin{Verbatim}[fontsize=\small]
teidata.version = token { pattern = "[\d]+(\.[\d]+){0,2}" }
\end{Verbatim}

    \item[{Note}]
  \par
The value of this attribute follows the pattern specified by the Unicode consortium for its version number (\url{http://unicode.org/versions/}). A version number contains digits and fullstop characters only. The first number supplied identifies the major version number. A second and third number, for minor and sub-minor version numbers, may also be supplied.
\end{reflist}  
\begin{reflist}
\item[]\begin{specHead}{TEI.teidata.versionNumber}{teidata.versionNumber} defines the range of attribute values used for version numbers.\end{specHead} 
    \item[{Module}]
  tei
    \item[{Used by}]
  Element: \begin{itemize}
\item application/@version
\end{itemize} 
    \item[{Content model}]
  \mbox{}\hfill\\[-10pt]\begin{Verbatim}[fontsize=\small]
<content>
 <dataRef name="token"
  restriction="[\d]+[a-z]*[\d]*(\.[\d]+[a-z]*[\d]*){0,3}"/>
</content>
    
\end{Verbatim}

    \item[{Declaration}]
  \mbox{}\hfill\\[-10pt]\begin{Verbatim}[fontsize=\small]
teidata.versionNumber =
   token { pattern = "[\d]+[a-z]*[\d]*(\.[\d]+[a-z]*[\d]*){0,3}" }
\end{Verbatim}

\end{reflist}  
\begin{reflist}
\item[]\begin{specHead}{TEI.teidata.word}{teidata.word} defines the range of attribute values expressed as a single word or token.\end{specHead} 
    \item[{Module}]
  tei
    \item[{Used by}]
  teidata.enumerated teidata.sexElement: \begin{itemize}
\item app/@loc
\item gap/@reason
\item langKnown/@level
\item locus/@from
\item locus/@to
\item m/@baseForm
\item media/@mimeType
\item metamark/@function
\item org/@role
\item personGrp/@size
\item secl/@reason
\item supplied/@reason
\item surplus/@reason
\item unclear/@reason
\end{itemize} 
    \item[{Content model}]
  \mbox{}\hfill\\[-10pt]\begin{Verbatim}[fontsize=\small]
<content>
 <dataRef name="token"
  restriction="(\p{L}|\p{N}|\p{P}|\p{S})+"/>
</content>
    
\end{Verbatim}

    \item[{Declaration}]
  \mbox{}\hfill\\[-10pt]\begin{Verbatim}[fontsize=\small]
teidata.word = token { pattern = "(\p{L}|\p{N}|\p{P}|\p{S})+" }
\end{Verbatim}

    \item[{Note}]
  \par
Attributes using this datatype must contain a single ‘word’ which contains only letters, digits, punctuation characters, or symbols: thus it cannot include whitespace.
\end{reflist}  
\begin{reflist}
\item[]\begin{specHead}{TEI.teidata.xTruthValue}{teidata.xTruthValue} (extended truth value) defines the range of attribute values used to express a truth value which may be unknown.\end{specHead} 
    \item[{Module}]
  tei
    \item[{Used by}]
  Element: \begin{itemize}
\item binding/@contemporary
\item said/@aloud
\item said/@direct
\item seal/@contemporary
\end{itemize} 
    \item[{Content model}]
  \mbox{}\hfill\\[-10pt]\begin{Verbatim}[fontsize=\small]
<content>
 <alternate>
  <dataRef name="boolean"/>
  <valList>
   <valItem ident="unknown"/>
   <valItem ident="inapplicable"/>
  </valList>
 </alternate>
</content>
    
\end{Verbatim}

    \item[{Declaration}]
  \mbox{}\hfill\\[-10pt]\begin{Verbatim}[fontsize=\small]
teidata.xTruthValue = xsd:boolean | ( "unknown" | "inapplicable" )
\end{Verbatim}

    \item[{Note}]
  \par
In cases where where uncertainty is inappropriate, use the datatype \textsf{data.TruthValue}.
\end{reflist}  
\begin{reflist}
\item[]\begin{specHead}{TEI.teidata.xmlName}{teidata.xmlName} defines attribute values which contain an XML name.\end{specHead} 
    \item[{Module}]
  tei
    \item[{Used by}]
  Element: \begin{itemize}
\item schemaRef/@key
\end{itemize} 
    \item[{Content model}]
  \fbox{\ttfamily <content>\newline
 <dataRef name="NCName"/>\newline
</content>\newline
    } 
    \item[{Declaration}]
  \fbox{\ttfamily teidata.xmlName = xsd:NCName} 
    \item[{Note}]
  \par
The rules defining an XML name form a part of the XML Specification.
\end{reflist}  
\end{document}
